\documentclass[11pt,a4paper]{article}
\usepackage[english,greek]{babel}
\usepackage[utf8]{inputenc}
\usepackage{nimbusserif}
\usepackage[T1]{fontenc}
\usepackage[left=2.00cm, right=2.00cm, top=3.00cm, bottom=2.00cm]{geometry}
\usepackage{amsmath}
\let\myBbbk\Bbbk
\let\Bbbk\relax
\usepackage[amsbb,subscriptcorrection,zswash,mtpcal,mtphrb,mtpfrak]{mtpro2}
\usepackage{graphicx,multicol,multirow,enumitem,tabularx,mathimatika,gensymb,venndiagram,hhline,longtable,tkz-euclide,fontawesome5,eurosym,tcolorbox}
\tcbuselibrary{skins,theorems,breakable}
\newlist{rlist}{enumerate}{3}
\setlist[rlist]{itemsep=0mm,label=\roman*.}
\newlist{alist}{enumerate}{3}
\setlist[alist]{itemsep=0mm,label=\alph*.}
\newlist{balist}{enumerate}{3}
\setlist[balist]{itemsep=0mm,label=\bf\alph*.}
\newlist{Alist}{enumerate}{3}
\setlist[Alist]{itemsep=0mm,label=\Alph*.}
\newlist{bAlist}{enumerate}{3}
\setlist[bAlist]{itemsep=0mm,label=\bf\Alph*.}
\renewcommand{\textstigma}{\textsigma\texttau}
\newlist{thema}{enumerate}{3}
\setlist[thema]{label=\bf\large{ΘΕΜΑ \textcolor{black}{\Alph*}},itemsep=0mm,leftmargin=0cm,itemindent=18mm}
\newlist{erwthma}{enumerate}{3}
\setlist[erwthma]{label=\bf{\large{\textcolor{black}{\Alph{themai}.\arabic*}}},itemsep=0mm,leftmargin=0.8cm}

\newcommand{\lysh}{\textcolor{black}{\textbf{\faCheck\ \ ΛΥΣΗ}}}
\renewcommand{\textstigma}{\textsigma\texttau}
%----------- ΟΡΙΣΜΟΣ------------------
\newcounter{orismos}[section]
\renewcommand{\theorismos}{\thesection.\arabic{orismos}}   
\newcommand{\Orismos}{\refstepcounter{orismos}{\textbf{\textcolor{black}{\kerkissans{Ορισμός\hspace{2mm}\theorismos}}\;:\;}{}}}

\newenvironment{orismos}[1]
{\begin{tcolorbox}[title=\Orismos {\textcolor{black}{\kerkissans{#1}}},breakable,bottomtitle=-1.5mm,
enhanced standard,titlerule=-.2pt,toprule=0pt, rightrule=0pt, bottomrule=0pt,
colback=white,left=2mm,top=1mm,bottom=0mm,
boxrule=0pt,
colframe=white,borderline west={1.5mm}{0pt}{black},leftrule=2mm,sharp corners,coltitle=black]}
{\end{tcolorbox}}

\newcommand{\kerkissans}[1]{{\fontfamily{maksf}\selectfont \textbf{#1}}}
\renewcommand{\textdexiakeraia}{}

\usepackage[
backend=biber,
style=alphabetic,
sorting=ynt
]{biblatex}

\begin{document}
\begin{enumerate}[label=\bf\fontfamily{maksf}\selectfont \arabic*.]
\item \kerkissans{Τετράγωνο αθροίσματος:}\\ $(a+\beta)^2=a^2+2a\beta+\beta^2$
\item \kerkissans{Τετράγωνο διαφοράς:}\\ $(a-\beta)^2=a^2-2a\beta+\beta^2$
\item \kerkissans{Κύβος αθροίσματος:}\\ $(a+\beta)^3=a^3+3a^2\beta+3a\beta^2+\beta^3$
\item \kerkissans{Κύβος διαφοράς:}\\ $(a-\beta)^3=a^3-3a^2\beta+3a\beta^2-\beta^3$
\item \kerkissans{Γινόμενο αθροίσματος επί διαφορά:}\\ $(a+\beta)(a-\beta)=a^2-\beta^2$
\item \kerkissans{Άθροισμα κύβων:}\\ $(a+\beta)(a^2-a\beta+\beta^2)=a^3+\beta^3$
\item \kerkissans{Διαφορά κύβων:}\\ $(a-\beta)(a^2+a\beta+\beta^2)=a^3-\beta^3$
\end{enumerate}
\begin{enumerate}[label=\bf\fontfamily{maksf}\selectfont \arabic*.]
\item \kerkissans{Κοινός παράγοντας:}\\ $a\beta+a\gamma=a(\beta+\gamma)$
\item \kerkissans{Ομαδοποίηση}\\ $ ax+ay+\beta x+\beta y=a(x+y)+\beta(x+y)=(x+y)(a+\beta) $
\item \kerkissans{Διαφορά τετραγώνων:}\\ $a^2-\beta^2=(a+\beta)(a-\beta)$
\item \kerkissans{Άθροισμα κύβων:}\\ $a^3+\beta^3=(a+\beta)(a^2-a\beta+\beta^2)$
\item \kerkissans{Διαφορά κύβων:}\\ $a^3-\beta^3=(a-\beta)(a^2+a\beta+\beta^2)$
\item \kerkissans{Ανάπτυγμα αθροίσματος:}\\
$a^2+2a\beta+\beta^2=(a+\beta)^2$
\item \kerkissans{Ανάπτυγμα διαφοράς:}\\
$a^2-2a\beta+\beta^2=(a-\beta)^2$
\item \kerkissans{Τριώνυμο}\\
$ax^2+\beta x+\gamma=\begin{cdcases}
a(x-x_1)(x-x_2) & ,\textrm{αν }\varDelta>0\\
a(x-x_0)^2 & ,\textrm{αν }\varDelta=0\\
\textrm{δεν παραγοντοποιείται} & ,\textrm{αν }\varDelta<0
\end{cdcases}$
\item \kerkissans{Τέλεια διαίρεση πολυωνύμων}\\
$P(x)=\delta(x)\cdot\pi(x)$ όπου $\delta(x)$ ο διαιρέτης και $\pi(x)$ το πηλίκο.
\end{enumerate}
\newpage
\begin{enumerate}[label=\arabic*(a), leftmargin=1cm, series=l_after]
\item A
\item B
\end{enumerate}
\begin{enumerate}
\item 
\item 
\item 
\end{enumerate}
\begin{enumerate}[label=\arabic*(b), resume*=l_after]
% or [label=\arabic*(b), l_after]
\item A
\item B
\end{enumerate}
\end{document}
