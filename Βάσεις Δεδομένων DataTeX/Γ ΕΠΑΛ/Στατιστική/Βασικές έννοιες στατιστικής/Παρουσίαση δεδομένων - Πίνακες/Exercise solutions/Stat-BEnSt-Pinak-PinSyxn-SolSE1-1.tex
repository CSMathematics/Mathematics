%# File Id : Stat-BEnSt-Pinak-PinSyxn-SolSE1-1
%@ FilesDB Id : G_Epal
\begin{alist}
\item Μέσα στο δείγμα παρατηρούνται 4 τιμές για τη μεταβλητή αγαπημένη ποδοσφαιρική ομάδα:
\[ x_1:\text{ΟΣΦΠ}\ ,\ x_2:\text{ΠΑΟ}\ ,\ x_3:\text{ΑΕΚ}\ ,\ x_4:\text{ΠΑΟΚ} \]
Κάνοντας διαλογή των παρατηρήσεων υπολογίζουμε αρχικά τις συχνότητες των τιμών όπως φαίνεται στον παρακάτω πίνακα:
\begin{center}
\begin{tabular}{|c|c|c|}
\hline $ x_i $ & Διαλογή & $ \nu_i $\\
\hline ΟΣΦΠ & \tally 12 & $12$ \\
\hline ΠΑΟ & \tally 10 & $10$ \\
\hline ΑΕΚ & \tally 8 & $8$ \\
\hline ΠΑΟΚ & \tally 10 & $10$ \\
\hline \textbf{Σύνολο} &  & $50$ \\
\hline
\end{tabular}
\end{center}
Για τις σχετικές συχνότητες $ f_i $ καθώς και τις σχετικές συχνότητες $ f_i\% $ θα έχουμε αντίστοιχα:
\begin{multicols}{2}
\begin{itemize}
\item $ f_1=\dfrac{\nu_1}{\nu}=\dfrac{12}{40}=0{,}3 $
\item $ f_2=\dfrac{\nu_2}{\nu}=\dfrac{10}{40}=0{,}25 $
\item $ f_3=\dfrac{\nu_3}{\nu}=\dfrac{8}{40}=0{,}2 $
\item $ f_4=\dfrac{\nu_4}{\nu}=\dfrac{10}{40}=0{,}25 $
\columnbreak
\item $ f_1\%=f_1\cdot 100=0{,}3\cdot 100=30 $
\item $ f_2\%=f_2\cdot 100=0{,}25\cdot 100=25 $
\item $ f_3\%=f_3\cdot 100=0{,}2\cdot 100=20 $
\item $ f_4\%=f_4\cdot 100=0{,}25\cdot 100=25 $
\end{itemize}
\end{multicols}
Παίρνουμε λοιπόν έτσι τον ακόλουθο πίνακα κατανομής συχνοτήτων, σχετικών συχνοτήτων και σχετικών συχνοττηων τοις 100
\begin{center}
\begin{tabular}{|c|c|c|c|}
\hline $ x_i $ & $ \nu_i $ & $ f_i $ & $ f_i\% $\\
\hline ΟΣΦΠ & $12$ & $ 0{,}3 $ & $30$\\
\hline ΠΑΟ & $10$ & $ 0{,}25 $ & $25$\\
\hline ΑΕΚ & $8$ & $ 0{,}2 $ & $20$\\
\hline ΠΑΟΚ & $10$ & $ 0{,}25 $ & $25$\\
\hline \textbf{Σύνολο} & $50$ & $1$ & $100$ \\
\hline
\end{tabular}
\end{center}
\item Το πλήθος των μαθητών που υποστηρίζουν τον ΠΑΟ είναι $ \nu_2=10 $ ενώ το ποσοστό των μαθητών που υποστηρίζει την ΑΕΚ είναι $ f_3\%=20 $.
\end{alist}
%# End of file Stat-BEnSt-Pinak-PinSyxn-SolSE1-1