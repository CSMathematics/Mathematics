%# File Id : Stat-MThD-MetDias-Diak-SolSE1-1
%@ FilesDB Id : G_Epal
\begin{Alist}
\item Υπολογίζουμε αρχικά τη μέση τιμή του δείγματος παρατηρήσεων:
\[ \bar{x}=\frac{1}{\nu}\sum_{i=1}^{\nu}{t_i}=\dfrac{1+4+5+5+7+8}{6}=\dfrac{30}{6}=5 \]
Καθώς η μέση τιμή είναι ακέραια τότε για τη διακύμανση θα έχουμε
\begin{align*}
s^2&=\frac{1}{\nu}\sum_{n=1}^{\nu}{(t_i-\bar{x})^2}=\dfrac{(1-5)^2+(4-5)^2+(5-5)^2+(5-5)^2+(7-5)^2+(7-5)^2}{6}=\\
&=\dfrac{16+1+0+0+4+9}{6}=\frac{30}{6}=5
\end{align*}
\item Η μέση τιμή του δείγματος θα ισούται με:
\[ \bar{x}=\frac{1}{\nu}\sum_{i=1}^{\nu}{t_i}=\dfrac{2+3+4+4+5+6+8+8+9}{9}=\dfrac{49}{9}=5{,}\bar{4} \]
Η μέση τιμή είναι δεν ακέραια οπότε για τον υπολογισμό της διακύμανσης θα έχουμε
\[
s^2&=\frac{1}{\nu}\left\{\sum_{n=1}^{\nu}{t_i^2}-\dfrac{\left(\sum_{i=1}^{\nu}{t_i}\right)^2}{\nu}\right\} \]
Θα είναι
\begin{align*}
\sum_{n=1}^{9}{t_i^2}&=2^2+3^2+4^2+4^2+5^2+6^2+8^2+8^2+9^2=\\&=4+9+16+16+25+36+64+64+81=315
\end{align*}
άρα \[ s^2=\dfrac{1}{9}\left(315-\frac{49^2}{9}\right)=\frac{1}{9}(315-266{,}77)=\frac{48{,}23}{9}=5{,}358 \]
\end{itemize}
\end{Alist}
%# End of file Stat-MThD-MetDias-Diak-SolSE1-1