%# Database File : Alg-FysAr-FDSt-SL-SectEx1
%@ FilesDB Id : a_gymnasiou
Να χαρακτηρίσετε τις παρακάτω προτάσεις ως σωστές (Σ) ή λανθασμένες (Λ).
\begin{rlist}
\item Σε έναν φυσικό αριθμό η θέση των δεκάδων χιλιάδων (ΔΧ) δηλώνει μεγαλύτερη αξία πό τη θέση των χιλιάδων (Χ).
\item Στο φυσικό αριθμό $ 39.817 $ το ψηφίο $ 9 $ είναι στη θέση των δεκάδων χιλιάδων.
\item Ανάμεσα σε δύο τριψήφιους αριθμούς, μεγαλύτερος είναι εκείνος που έχει περισσότερες εκατοντάδες (Ε).
\item Ανάμεσα σε δύο φυσικούς αριθμούς, μεγαλύτερος είναι εκείνος που έχει περισσότερες εκατοντάδες (Ε).
\item Όταν στρογγυλοποιούμε έναν αριθμό τότε η αξία του αριθμού αυτού μειώνεται.
\end{rlist}
%# End of file Alg-FysAr-FDSt-SL-SectEx1%# File Id : Alg-FysAr-FDSt-SL-SectEx1
%@ FilesDB Id : a_gymnasiou
Να χαρακτηρίσετε τις παρακάτω προτάσεις ως σωστές (Σ) ή λανθασμένες (Λ).
\begin{rlist}
\item Σε έναν φυσικό αριθμό η θέση των δεκάδων χιλιάδων (ΔΧ) δηλώνει μεγαλύτερη αξία πό τη θέση των χιλιάδων (Χ).
\item Στο φυσικό αριθμό $ 39.817 $ το ψηφίο $ 9 $ είναι στη θέση των δεκάδων χιλιάδων.
\item Ανάμεσα σε δύο τριψήφιους αριθμούς, μεγαλύτερος είναι εκείνος που έχει περισσότερες εκατοντάδες (Ε).
\item Ανάμεσα σε δύο φυσικούς αριθμούς, μεγαλύτερος είναι εκείνος που έχει περισσότερες εκατοντάδες (Ε).
\item Όταν στρογγυλοποιούμε έναν αριθμό τότε η αξία του αριθμού αυτού μειώνεται.
\end{rlist}
%# End of file Alg-FysAr-FDSt-SL-SectEx1