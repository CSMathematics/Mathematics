%# Database File : Alg-FysAr-FDSt-AxonAr-SectEx1
%@ FilesDB Id : a_gymnasiou
Να κατασκευάσετε έναν άξονα φυσικών αριθμών με αρχή το σημείο $ O $ όπου κάθε μονάδα απέχει από την επόμενη $ 1\,cm $.
\begin{center}
\begin{tikzpicture}
\tkzSetUpPoint[size=7,fill=white]
\clip (-.2,-.5) rectangle (6.8,0.5);
\begin{axis}[aks_on,belh ar,axis y line=none,xmin=0,xmax=7.3,x=.9cm,ymin=0,ymax=2]
\end{axis}
\tkzDrawPoint(0,0)
\tkzLabelPoint[above](0,0){$O$}
\end{tikzpicture}
\end{center}
Στη συνέχεια να πάρετε σημεία $ A,B,\varGamma $ πάνω στον άξονα ώστε $ OA=3\,cm $, $ AB=7\,cm $ και $ B\varGamma=2\,cm $. Σε ποιους αριθμούς αντιστοιχούν αυτά τα σημεία;
%# End of file Alg-FysAr-FDSt-AxonAr-SectEx1%# File Id : Alg-FysAr-FDSt-AxonAr-SectEx1
%@ FilesDB Id : a_gymnasiou
Να κατασκευάσετε έναν άξονα φυσικών αριθμών με αρχή το σημείο $ O $ όπου κάθε μονάδα απέχει από την επόμενη $ 1\,cm $.
\begin{center}
\begin{tikzpicture}
\tkzSetUpPoint[size=7,fill=white]
\clip (-.2,-.5) rectangle (6.8,0.5);
\begin{axis}[aks_on,belh ar,axis y line=none,xmin=0,xmax=7.3,x=.9cm,ymin=0,ymax=2]
\end{axis}
\tkzDrawPoint(0,0)
\tkzLabelPoint[above](0,0){$O$}
\end{tikzpicture}
\end{center}
Στη συνέχεια να πάρετε σημεία $ A,B,\varGamma $ πάνω στον άξονα ώστε $ OA=3\,cm $, $ AB=7\,cm $ και $ B\varGamma=2\,cm $. Σε ποιους αριθμούς αντιστοιχούν αυτά τα σημεία;
%# End of file Alg-FysAr-FDSt-AxonAr-SectEx1