%# Database File : Alg-FysAr-FDSt-DekTax-SectEx2
%@ FilesDB Id : a_gymnasiou
Σε έναν τετραψήφιο αριθμό το ψηφίο των δεκάδων είναι το ίδιο με το ψηφίο των χιλιάδων. Επίσης ο αριθμός αυτός έχει 7 εκατοντάδες ενώ το ψηφίο των μονάδων είναι κατά 2 μικρότερο από το ψηφίο των δεκάδων. Αν γράψουμε τα ψηφία του αριθμού αυτού με αντίστροφη σειρά τότε ο αριθμός που προκύπτει είναι τριψήφιος. Ποιος είναι ο αρχικός αριθμός;
%# End of file Alg-FysAr-FDSt-DekTax-SectEx2%# File Id : Alg-FysAr-FDSt-DekTax-SectEx2
%@ FilesDB Id : a_gymnasiou
Σε έναν τετραψήφιο αριθμό το ψηφίο των δεκάδων είναι το ίδιο με το ψηφίο των χιλιάδων. Επίσης ο αριθμός αυτός έχει 7 εκατοντάδες ενώ το ψηφίο των μονάδων είναι κατά 2 μικρότερο από το ψηφίο των δεκάδων. Αν γράψουμε τα ψηφία του αριθμού αυτού με αντίστροφη σειρά τότε ο αριθμός που προκύπτει είναι τριψήφιος. Ποιος είναι ο αρχικός αριθμός;
%# End of file Alg-FysAr-FDSt-DekTax-SectEx2