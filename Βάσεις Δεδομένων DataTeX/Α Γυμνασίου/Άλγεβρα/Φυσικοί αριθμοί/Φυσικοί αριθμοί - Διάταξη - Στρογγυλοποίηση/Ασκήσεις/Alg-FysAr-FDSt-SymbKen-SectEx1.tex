%# Database File : Alg-FysAr-FDSt-SymbKen-SectEx1
%@ FilesDB Id : a_gymnasiou
Να συμπληρώσετε τα κενά στις παρακάτω προτάσεις.
\begin{rlist}
\item Κάθε φυσικός αριθμός έχει έναν \ldots\ldots\ldots\ldots\, και έναν επόμενο αριθμό, εκτός από το 0 το οποίο έχει μόνο \ldots\ldots\ldots\ldots
\item Η διάταξη είναι η ιδιότητα των φυσικών αριθμών η οποία μας επιτρέπει να \ldots\ldots\ldots\ldots\, φυσικούς αριθμούς μεταξύ τους και τα τους τοποθετούμε σε σειρά.
\item Στη στρογγυλοποίηση ενός αριθμού αν το ψηφίο αμέσως μικρότερης αξίας από τη θέση στρογγυλοποίησης είναι ένα από τα \ldots\ldots\ldots\ldots,\, τότε το ψηφίο της θέσης στρογγυλοποίησης αυξάνεται κατά 1 ενώ όλα τα επόμενα μηδενίζονται.
\item Στη στρογγυλοποίηση ενός αριθμού αν το ψηφίο αμέσως μικρότερης αξίας από τη θέση στρογγυλοποίησης είναι ένα από τα \ldots\ldots\ldots\ldots,\, τότε το ψηφίο της θέσης στρογγυλοποίησης παραμένει ίδιο ενώ όλα τα επόμενα μηδενίζονται.
\item Για να συγκρίνουμε δύο φυσικούς αριθμούς μεταξύ τους, ξεκινάμε τον έλεγχο από το ψηφίο της \ldots\ldots\ldots\ldots\, αξίας κάθε αριθμού.
\end{rlist}
%# End of file Alg-FysAr-FDSt-SymbKen-SectEx1%# File Id : Alg-FysAr-FDSt-SymbKen-SectEx1
%@ FilesDB Id : a_gymnasiou
Να συμπληρώσετε τα κενά στις παρακάτω προτάσεις.
\begin{rlist}
\item Κάθε φυσικός αριθμός έχει έναν \ldots\ldots\ldots\ldots\, και έναν επόμενο αριθμό, εκτός από το 0 το οποίο έχει μόνο \ldots\ldots\ldots\ldots
\item Η διάταξη είναι η ιδιότητα των φυσικών αριθμών η οποία μας επιτρέπει να \ldots\ldots\ldots\ldots\, φυσικούς αριθμούς μεταξύ τους και τα τους τοποθετούμε σε σειρά.
\item Στη στρογγυλοποίηση ενός αριθμού αν το ψηφίο αμέσως μικρότερης αξίας από τη θέση στρογγυλοποίησης είναι ένα από τα \ldots\ldots\ldots\ldots,\, τότε το ψηφίο της θέσης στρογγυλοποίησης αυξάνεται κατά 1 ενώ όλα τα επόμενα μηδενίζονται.
\item Στη στρογγυλοποίηση ενός αριθμού αν το ψηφίο αμέσως μικρότερης αξίας από τη θέση στρογγυλοποίησης είναι ένα από τα \ldots\ldots\ldots\ldots,\, τότε το ψηφίο της θέσης στρογγυλοποίησης παραμένει ίδιο ενώ όλα τα επόμενα μηδενίζονται.
\item Για να συγκρίνουμε δύο φυσικούς αριθμούς μεταξύ τους, ξεκινάμε τον έλεγχο από το ψηφίο της \ldots\ldots\ldots\ldots\, αξίας κάθε αριθμού.
\end{rlist}
%# End of file Alg-FysAr-FDSt-SymbKen-SectEx1