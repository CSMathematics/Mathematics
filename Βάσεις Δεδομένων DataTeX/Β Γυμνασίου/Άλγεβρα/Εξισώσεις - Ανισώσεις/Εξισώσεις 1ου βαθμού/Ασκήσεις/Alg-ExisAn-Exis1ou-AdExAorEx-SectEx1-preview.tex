\documentclass[11pt,a4paper]{article}
\usepackage[english,greek]{babel}
\usepackage[utf8]{inputenc}
\usepackage{nimbusserif}
\usepackage[T1]{fontenc}
\usepackage[left=2.00cm, right=2.00cm, top=3.00cm, bottom=2.00cm]{geometry}
\usepackage{amsmath}
\let\myBbbk\Bbbk
\let\Bbbk\relax
\usepackage[amsbb,subscriptcorrection,zswash,mtpcal,mtphrb,mtpfrak]{mtpro2}
\usepackage{graphicx,multicol,multirow,enumitem,tabularx,mathimatika,gensymb,venndiagram,hhline,longtable,tkz-euclide,fontawesome5,eurosym,tcolorbox}
\usepackage[misc]{ifsym}
\tcbuselibrary{skins,theorems,breakable}
\newlist{rlist}{enumerate}{3}
\setlist[rlist]{itemsep=0mm,label=\roman*.}
\newlist{alist}{enumerate}{3}
\setlist[alist]{itemsep=0mm,label=\alph*.}
\newlist{balist}{enumerate}{3}
\setlist[balist]{itemsep=0mm,label=\bf\alph*.}
\newlist{Alist}{enumerate}{3}
\setlist[Alist]{itemsep=0mm,label=\Alph*.}
\newlist{bAlist}{enumerate}{3}
\setlist[bAlist]{itemsep=0mm,label=\bf\Alph*.}
\renewcommand{\textstigma}{\textsigma\texttau}
\newlist{thema}{enumerate}{3}
\setlist[thema]{label=\bf\large{ΘΕΜΑ \textcolor{black}{\Alph*}},itemsep=0mm,leftmargin=0cm,itemindent=18mm}
\newlist{erwthma}{enumerate}{3}
\setlist[erwthma]{label=\bf{\large{\textcolor{black}{\Alph{themai}.\arabic*}}},itemsep=0mm,leftmargin=0.8cm}
\newlist{bhma}{enumerate}{3}
\setlist[bhma]{label=\bf\textit{\arabic*\textsuperscript{o}\;Βήμα :},leftmargin=0cm,itemindent=1.8cm,ref=\bf{\arabic*\textsuperscript{o}\;Βήμα}}
\newcommand{\lysh}{\textcolor{black}{\textbf{\faCheck\ \ ΛΥΣΗ}}}
\renewcommand{\textstigma}{\textsigma\texttau}
%----------- ΟΡΙΣΜΟΣ------------------
\newcounter{orismos}[section]
\renewcommand{\theorismos}{\thesection.\arabic{orismos}}   
\newcommand{\Orismos}{\refstepcounter{orismos}{\textbf{\textcolor{black}{\kerkissans{Ορισμός\hspace{2mm}\theorismos}}\;:\;}{}}}

\newenvironment{orismos}[1]
{\begin{tcolorbox}[title=\Orismos {\textcolor{black}{\kerkissans{#1}}},breakable,bottomtitle=-1.5mm,
enhanced standard,titlerule=-.2pt,toprule=0pt, rightrule=0pt, bottomrule=0pt,
colback=white,left=2mm,top=1mm,bottom=0mm,
boxrule=0pt,
colframe=white,borderline west={1.5mm}{0pt}{black},leftrule=2mm,sharp corners,coltitle=black]}
{\end{tcolorbox}}

\newcommand{\kerkissans}[1]{{\fontfamily{maksf}\selectfont \textbf{#1}}}
\renewcommand{\textdexiakeraia}{}

\usepackage[
backend=biber,
style=alphabetic,
sorting=ynt
]{biblatex}

\DeclareMathSizes{10.95}{10.95}{7}{5}
\DeclareMathSizes{6}{6}{3.8}{2.7}
\DeclareMathSizes{8}{8}{5.1}{3.6}
\DeclareMathSizes{9}{9}{5.8}{4.1}
\DeclareMathSizes{10}{10}{6.4}{4.5}
\DeclareMathSizes{12}{12}{7.7}{5.5}
\DeclareMathSizes{14.4}{14.4}{9.2}{6.5}
\DeclareMathSizes{17.28}{17.28}{11}{7.9}
\DeclareMathSizes{20.74}{20.74}{13.3}{9.4}
\DeclareMathSizes{24.88}{24.88}{16}{11.3}

\makeatletter
\AtBeginDocument{
\check@mathfonts
\fontdimen16\textfont2=2.5pt
\fontdimen17\textfont2=2.5pt
\fontdimen14\textfont2=4.5pt
\fontdimen13\textfont2=4.5pt 
}
\makeatother


\newcount\tallycount
\newcommand{\dotally}{%
    \loop\ifnum\tallycount>4\relax\StrokeFive\  
         \advance\tallycount by-5%
    \repeat%
    \ifcase \tallycount\or\StrokeOne\or\StrokeTwo\or\StrokeThree\or\StrokeFour
    \else   ERROR
    \fi%
    \unskip}
\newcommand{\tally}{\afterassignment\dotally\tallycount }

\begin{document}

%# Database File : Alg-ExisAn-Exis1ou-AdExAorEx-SectEx1
%@ FilesDB Id : b_gymnasiou
Να λυθούν οι παρακάτω εξισώσεις
\begin{rlist}
\item $ 2x-1=4+2x $
\item $ 7-3x=-3x+7 $
\item $ 5x-3=x-3+4x $
\item $ 2x+1-x=x-3+4 $
\end{rlist}
%# End of file Alg-ExisAn-Exis1ou-AdExAorEx-SectEx1

 \end{document}\documentclass[11pt,a4paper]{article}
\usepackage[english,greek]{babel}
\usepackage[utf8]{inputenc}
\usepackage{nimbusserif}
\usepackage[T1]{fontenc}
\usepackage[left=2.00cm, right=2.00cm, top=3.00cm, bottom=2.00cm]{geometry}
\usepackage{amsmath}
\let\myBbbk\Bbbk
\let\Bbbk\relax
\usepackage[amsbb,subscriptcorrection,zswash,mtpcal,mtphrb,mtpfrak]{mtpro2}
\usepackage{graphicx,multicol,multirow,enumitem,tabularx,mathimatika,gensymb,venndiagram,hhline,longtable,tkz-euclide,fontawesome5,eurosym,tcolorbox}
\usepackage[misc]{ifsym}
\tcbuselibrary{skins,theorems,breakable}
\newlist{rlist}{enumerate}{3}
\setlist[rlist]{itemsep=0mm,label=\roman*.}
\newlist{alist}{enumerate}{3}
\setlist[alist]{itemsep=0mm,label=\alph*.}
\newlist{balist}{enumerate}{3}
\setlist[balist]{itemsep=0mm,label=\bf\alph*.}
\newlist{Alist}{enumerate}{3}
\setlist[Alist]{itemsep=0mm,label=\Alph*.}
\newlist{bAlist}{enumerate}{3}
\setlist[bAlist]{itemsep=0mm,label=\bf\Alph*.}
\renewcommand{\textstigma}{\textsigma\texttau}
\newlist{thema}{enumerate}{3}
\setlist[thema]{label=\bf\large{ΘΕΜΑ \textcolor{black}{\Alph*}},itemsep=0mm,leftmargin=0cm,itemindent=18mm}
\newlist{erwthma}{enumerate}{3}
\setlist[erwthma]{label=\bf{\large{\textcolor{black}{\Alph{themai}.\arabic*}}},itemsep=0mm,leftmargin=0.8cm}
\newlist{bhma}{enumerate}{3}
\setlist[bhma]{label=\bf\textit{\arabic*\textsuperscript{o}\;Βήμα :},leftmargin=0cm,itemindent=1.8cm,ref=\bf{\arabic*\textsuperscript{o}\;Βήμα}}
\newcommand{\lysh}{\textcolor{black}{\textbf{\faCheck\ \ ΛΥΣΗ}}}
\renewcommand{\textstigma}{\textsigma\texttau}
%----------- ΟΡΙΣΜΟΣ------------------
\newcounter{orismos}[section]
\renewcommand{\theorismos}{\thesection.\arabic{orismos}}   
\newcommand{\Orismos}{\refstepcounter{orismos}{\textbf{\textcolor{black}{\kerkissans{Ορισμός\hspace{2mm}\theorismos}}\;:\;}{}}}

\newenvironment{orismos}[1]
{\begin{tcolorbox}[title=\Orismos {\textcolor{black}{\kerkissans{#1}}},breakable,bottomtitle=-1.5mm,
enhanced standard,titlerule=-.2pt,toprule=0pt, rightrule=0pt, bottomrule=0pt,
colback=white,left=2mm,top=1mm,bottom=0mm,
boxrule=0pt,
colframe=white,borderline west={1.5mm}{0pt}{black},leftrule=2mm,sharp corners,coltitle=black]}
{\end{tcolorbox}}

\newcommand{\kerkissans}[1]{{\fontfamily{maksf}\selectfont \textbf{#1}}}
\renewcommand{\textdexiakeraia}{}

\usepackage[
backend=biber,
style=alphabetic,
sorting=ynt
]{biblatex}

\DeclareMathSizes{10.95}{10.95}{7}{5}
\DeclareMathSizes{6}{6}{3.8}{2.7}
\DeclareMathSizes{8}{8}{5.1}{3.6}
\DeclareMathSizes{9}{9}{5.8}{4.1}
\DeclareMathSizes{10}{10}{6.4}{4.5}
\DeclareMathSizes{12}{12}{7.7}{5.5}
\DeclareMathSizes{14.4}{14.4}{9.2}{6.5}
\DeclareMathSizes{17.28}{17.28}{11}{7.9}
\DeclareMathSizes{20.74}{20.74}{13.3}{9.4}
\DeclareMathSizes{24.88}{24.88}{16}{11.3}

\makeatletter
\AtBeginDocument{
\check@mathfonts
\fontdimen16\textfont2=2.5pt
\fontdimen17\textfont2=2.5pt
\fontdimen14\textfont2=4.5pt
\fontdimen13\textfont2=4.5pt 
}
\makeatother


\newcount\tallycount
\newcommand{\dotally}{%
    \loop\ifnum\tallycount>4\relax\StrokeFive\  
         \advance\tallycount by-5%
    \repeat%
    \ifcase \tallycount\or\StrokeOne\or\StrokeTwo\or\StrokeThree\or\StrokeFour
    \else   ERROR
    \fi%
    \unskip}
\newcommand{\tally}{\afterassignment\dotally\tallycount }

\begin{document}

%# File Id : Alg-ExisAn-Exis1ou-AdExAorEx-SectEx1
%@ FilesDB Id : b_gymnasiou
Να λυθούν οι παρακάτω εξισώσεις
\begin{rlist}
\item $ 2x-1=4+2x $
\item $ 7-3x=-3x+7 $
\item $ 5x-3=x-3+4x $
\item $ 2x+1-x=x-3+4 $
\end{rlist}
%# End of file Alg-ExisAn-Exis1ou-AdExAorEx-SectEx1

 \end{document}