\documentclass[twoside,11pt,a4paper]{book}
\usepackage[amsbb,mtpfrak,zswash,mtpcal]{mtpro2}
\usepackage[no-math,cm-default]{fontspec}
\usepackage{xgreek}
\defaultfontfeatures{Mapping=tex-text,Scale=MatchLowercase}
\setmainfont[Mapping=tex-text,Numbers=Lining,Scale=1.0,BoldFont={Times New Roman Bold}]{Times New Roman}
\defaultfontfeatures{Ligatures=TeX}
\font\kefalaio="Times New Roman Bold" at 36pt
\font\ArKef="Times New Roman Bold Italic" at 72pt
\font\OnKef="Times New Roman" at 20pt
\font\OnPar="Times New Roman Bold" at 18pt
\font\onoma="Rounded Mplus 1c Bold" at 10pt
\font\onomaaskpar="Rounded Mplus 1c Bold" at 16pt
\newfontfamily\scfont{GFS Artemisia}
\usepackage{fontawesome5}
%\newfontfamily{\FA}{fontawesome.otf}
\usepackage[inner=2.00cm, outer=1.50cm, top=3.00cm, bottom=2.00cm %,paperwidth=17cm,paperheight=24cm
]{geometry}
\usepackage{amsmath}
\usepackage[amsbb,mtpfrak,zswash,mtpcal]{mtpro2}
\usepackage{makeidx,eurosym}
\usepackage[usenames,dvipsnames,cmyk,table,x11names]{xcolor}
\usepackage{longtable,varwidth,tcolorbox}
\usepackage{float}
\usepackage{subfig}
\def\xrwma{cyan!70!black}
\def\xrwmath{black}
\usepackage{etoolbox}
\makeatletter
\newif\ifLT@nocaption
\preto\longtable{\LT@nocaptiontrue}
\appto\endlongtable{%
\ifLT@nocaption
\addtocounter{table}{\m@ne}%
\fi}
\preto\LT@caption{%
\noalign{\global\LT@nocaptionfalse}}
\makeatother
\makeindex
\usepackage{tikz,pgfplots}
\usepackage{tkz-euclide,tkz-fct}
\usetikzlibrary{fadings}
\usepackage{wrap-rl}
\usetkzobj{all}
\usepackage{calc}
\usepackage[colorlinks=false, pdfborder={0 0 0}]{hyperref}
\usepackage{cleveref}
\usepackage[framemethod=TikZ]{mdframed}
\definecolor{steelblue}{cmyk}{.7,.278,0,.294}
\definecolor{doc}{cmyk}{1,0.455,0,0.569}
\definecolor{olivedrab}{cmyk}{0.25,0,0.75,0.44}
\usepackage{capt-of}
\usepackage{titletoc}
\usepackage[explicit]{titlesec}
\usepackage{graphicx}
\usepackage{multicol}
\usepackage{multirow}
\usepackage{enumitem}
\usepackage{tabularx}
\tikzset{>=latex}
\makeatletter
\pretocmd{\@part}{\gdef\parttitle{#1}}{}{}
\pretocmd{\@spart}{\gdef\parttitle{#1}}{}{}
\makeatother
\usepackage[titletoc]{appendix}
\usepackage{fancyhdr}
\pagestyle{fancy}
\fancyheadoffset{0cm}
\renewcommand{\headrulewidth}{\iftopfloat{0pt}{.5pt}}
\renewcommand{\chaptermark}[1]{\markboth{#1}{}}
\renewcommand{\sectionmark}[1]{\markright{\it\thesection\ #1}}
\fancyhf{}
\fancyhead[LE]{\thepage\ $\cdot$\ \scfont\scshape\nouppercase{\leftmark}}
\fancyhead[RO]{\nouppercase{\rightmark} $\cdot$\ \thepage}
\fancypagestyle{plain}{%
\fancyhead{} %
\renewcommand{\headrulewidth}{0pt}}

%\newcounter{thewrhma}[chapter]
%\renewcommand{\thethewrhma}{\thechapter.\arabic{thewrhma}} 
%\newcommand{\Thewrhma}[1]{\refstepcounter{thewrhma}{\textbf{\textcolor{\xrwmath}{{\large Θεώρημα\hspace{2mm}\thethewrhma\;}:\;}\hspace{1mm}}} \MakeUppercase{\textbf{#1}}\\}{}

\newcounter{porisma}[chapter]
\renewcommand{\theporisma}{\thechapter.\arabic{porisma}}\newcommand{\Porisma}[1]{\refstepcounter{porisma}\textcolor{black}{\textbf{ΠΟΡΙΣΜΑ\hspace{2mm}\theporisma\hspace{1mm} \MakeUppercase{#1}}}\\}{}

\newcounter{protasi}[chapter]
\renewcommand{\theprotasi}{\thechapter.\arabic{protasi}}\newcommand{\Protasi}[1]{\refstepcounter{protasi}\textcolor{black}{\textbf{ΠΡΟΤΑΣΗ\hspace{2mm}\theprotasi\hspace{1mm} \MakeUppercase{#1}}}\\}{}


\usepackage{venndiagram,mathimatika,tkz-tab,gensymb}
\usepackage{tikzpagenodes}
\usetkzobj{all}
%--------- tkz-tab Πίνακες --------
\makeatletter
\xpatchcmd{\tkzTabLine}
{\node at (Z\thetkz@cnt@impair\thetkz@cnt@lg){$0$};} % search
{\node[fill=white,inner sep=.5mm] at (Z\thetkz@cnt@impair\thetkz@cnt@lg){$0$};} % replace
{}{}
\makeatother
%----------------------------------
%-------- ΣΤΥΛ ΚΕΦΑΛΑΙΟΥ ---------
\newcommand*\chapterlabel{}
\newcommand{\fancychapter}{%
\titleformat{\chapter}
{
\normalfont\Huge}
{\gdef\chapterlabel{\thechapter\ }}{0pt}
{\begin{tikzpicture}[remember picture,overlay]
\node[yshift=-7cm] at (current page.north west)
{\begin{tikzpicture}[remember picture, overlay]
%\node[inner sep=0pt] at ($(current page.north) +			(0cm,-1.38in)$) {\includegraphics[width=17cm]{Kefalaio}};
\node[anchor=west,xshift=.1\paperwidth,yshift=.14\paperheight,rectangle]
{{\color{white}\fontsize{30}{20}\textbf{\textcolor{black}{\contour{white}{ΚΕΦΑΛΑΙΟ}}}}};
\node[anchor=west,xshift=.09\paperwidth,yshift=.08\paperheight,rectangle] {\fontsize{24}{20} {\color{black}{\textcolor{black}{\sc##1}}}};
%\fill[fill=black] (12.2,2) rectangle (14.8,4.7);
\node[anchor=west,xshift=.74\paperwidth,yshift=0.1\paperheight,rectangle]
{\color{white}\fontsize{80}{20}\textbf{\textit{\textcolor{black}{\thechapter}}}};
\end{tikzpicture}
};
\end{tikzpicture}
}
\titlespacing*{\chapter}{0pt}{20pt}{10pt}
}
%------------------------------------------------


\usepackage[outline]{contour}
\newcommand{\regularchapter}{%
\titleformat{\chapter}[display]
{\normalfont\huge\bfseries}{\chaptertitlename\ \thechapter}{20pt}{\Huge##1}
\titlespacing*{\chapter}
{0pt}{-20pt}{40pt}
}

\apptocmd{\mainmatter}{\fancychapter}{}{}
\apptocmd{\backmatter}{\regularchapter}{}{}
\apptocmd{\frontmatter}{\regularchapter}{}{}

\titlespacing*{\section}
{0pt}{30pt}{0pt}
\usepackage{booktabs}
\usepackage{hhline}
\DeclareRobustCommand{\perthousand}{%
\ifmmode
\text{\textperthousand}%
\else
\textperthousand
\fi}


\contentsmargin{0cm}
\titlecontents{part}[-1pc]
{\addvspace{10pt}%
\bf\Large ΜΕΡΟΣ\quad }%
{}
{}
{\;\dotfill}%
%------------------------------------------
\titlecontents{chapter}[0pc]
{\addvspace{30pt}%
\begin{tikzpicture}[remember picture, overlay]%
\draw[fill=black,draw=black] (-.3,.5) rectangle (3.7,1.1); %
\pgftext[left,x=0cm,y=0.75cm]{\color{white}\sc\Large\bfseries Κεφάλαιο\ \thecontentslabel};%
\end{tikzpicture}\large\sc}%
{}
{}
{\hspace*{-2.3em}\hfill\normalsize Σελίδα \thecontentspage}%
\titlecontents{section}[2.4pc]
{\addvspace{1pt}}
{\contentslabel[\thecontentslabel]{2pc}}
{}
{\;\dotfill\;\small \thecontentspage}
[]
\titlecontents*{subsection}[4pc]
{\addvspace{-1pt}\small}
{}
{}
{\ --- \small\thecontentspage}
[ \textbullet\ ][]

\makeatletter
\renewcommand{\tableofcontents}{%
\chapter*{%
\vspace*{-20\p@}%
\begin{tikzpicture}[remember picture, overlay]%
\pgftext[right,x=12cm,y=0.2cm]{\Huge\sc\bfseries \contentsname};%
\draw[fill=black,draw=black] (9.5,-.75) rectangle (12.5,1);%
\clip (9.5,-.75) rectangle (15,1);
\pgftext[right,x=12cm,y=0.2cm]{\color{white}\Huge\bfseries \contentsname};%
\end{tikzpicture}}%
\@starttoc{toc}}
\makeatother

\usepackage[contents={},scale=1,opacity=1,color=black,angle=0]{background}

\newcommand\blfootnote[1]{%
\begingroup
\renewcommand\thefootnote{}\footnote{#1}%
\addtocounter{footnote}{-1}%
\endgroup
}
\usepackage{epstopdf}
\epstopdfsetup{update}
\usepackage{textcomp}

\titleformat{\section}
{\normalfont\Large\bf}%
{}{0em}%
{{\color{black}\titlerule[0pt]}\vskip-.2\baselineskip{\parbox[t]{\dimexpr\textwidth-2\fboxsep\relax}{\raggedright\strut\itshape{\LARGE{\thesection~#1}}\strut}}}[\vskip 0\baselineskip{\color{black}\titlerule[1pt]}]
\titlespacing*{\section}{0pt}{0pt}{30pt}

\newcommand{\methodologia}{\begin{center}
{\large \textbf{ΜΕΘΟΔΟΛΟΓΙΑ}}\\\vspace{-2mm}
\begin{tikzpicture}
\shade[left color=white, right color=black,] (-3cm,0) rectangle (0,.2mm);
\shade[left color=black, right color=white,] (0,0) rectangle (3cm,.2mm);   
\end{tikzpicture}
\end{center}}

\newcommand{\orismoi}{\begin{center}
\vspace{-3mm}{\large \textbf{\textcolor{\xrwma}{ΟΡΙΣΜΟΙ}}}\\\vspace{-2mm}
\begin{tikzpicture}
\shade[left color=white, right color=black!80!black,] (-3cm,0) rectangle (0,.2mm);
\shade[left color=black!80!black, right color=white,] (0,0) rectangle (3cm,.2mm);   
\end{tikzpicture}
\end{center}}
\newcommand{\thewrhmata}{\begin{center}
{\large \textbf{\textcolor{\xrwmath}{ΘΕΩΡΗΜΑΤΑ - ΠΟΡΙΣΜΑΤΑ - ΠΡΟΤΑΣΕΙΣ\\ΚΡΙΤΗΡΙΑ - ΙΔΙΟΤΗΤΕΣ}}}\\\vspace{-2mm}
\begin{tikzpicture}
\shade[left color=white, right color=\xrwmath,] (-5cm,0) rectangle (0,.2mm);
\shade[left color=\xrwmath, right color=white,] (0,0) rectangle (5cm,.2mm);   
\end{tikzpicture}
\end{center}}
\usepackage[labelfont={footnotesize,it,bf},font={footnotesize},skip=5pt]{caption}


\newcommand{\thewria}{\begin{center}
\vspace{-3mm}{\large \textbf{\textcolor{black}{ΘΕΩΡΙΑ}}}\\\vspace{-2mm}
\begin{tikzpicture}
\shade[left color=white, right color=black!80!black,] (-3cm,0) rectangle (0,.2mm);
\shade[left color=black!80!black, right color=white,] (0,0) rectangle (3cm,.2mm);   
\end{tikzpicture}
\end{center}}

%-------- ΠΙΝΑΚΕΣ ---------
\usepackage{booktabs}
%----------------------
%----- ΥΠΟΛΟΓΙΣΤΗΣ ----------
%\usepackage{calculator}
%----------------------------

%----- ΟΡΙΖΟΝΤΙΑ ΛΙΣΤΑ ------
\usepackage{xparse}
\newcounter{answers}
\renewcommand\theanswers{\arabic{answers}}
\ExplSyntaxOn
\NewDocumentCommand{\results}{m}
{
\seq_set_split:Nnn \l_results_a_seq {,}{#1}
\par\nobreak\noindent\setcounter{answers}{0}
\seq_map_inline:Nn \l_results_a_seq
{
\makebox[.18\linewidth][l]{\stepcounter{answers}\theanswers.~##1}\hfill
}
\par
}
\seq_new:N \l_results_a_seq
\ExplSyntaxOff
%----------------------------
%------ ΜΗΚΟΣ ΓΡΑΜΜΗΣ ΚΛΑΣΜΑΤΟΣ ---------
\DeclareRobustCommand{\frac}[3][0pt]{%
{\begingroup\hspace{#1}#2\hspace{#1}\endgroup\over\hspace{#1}#3\hspace{#1}}}
%----------------------------------------
\usepackage{microtype}
\usepackage{float}

\usepackage{caption}

\tikzstyle{pl}=[line width=0.3mm]
\tikzstyle{plm}=[line width=0.4mm]
%%------- ΣΤΥΛ ΠΑΡΑΔΕΙΓΜΑΤΟΣ -------
%\newcounter{paradeigma}[section]
%\renewcommand{\theparadeigma}{\bf\thechapter.\arabic{paradeigma}}   
%\newcommand{\Paradeigma}[1]{\refstepcounter{paradeigma}\textcolor{black}{\textbf{{\large Παράδειγμα\hspace{2mm}\theparadeigma\;:\;}\hspace{1mm}}} \MakeUppercase{\textbf{#1}}\\}{}
%%-----------------------------------
%
%%------- ΣΤΥΛ ΛΥΣΗΣ ------------------
%\newcommand{\lysh}{{\textbf{ΛΥΣΗ}}}
%%------------------------------------

%------ ΛΥΜΕΝΑ ΠΑΡΑΔΕΙΓΜΑΤΑ ΤΙΤΛΟΣ ---------
\newcommand{\Lymena}{\begin{center}
\begin{tikzpicture}
\path[left color=black!40!white,right color=black!40!white,middle color=black!10!white] (-9cm,-.6cm) rectangle (8.5cm,.6cm);
\node at (-.25cm,0) {\Large \textcolor{black}{\textbf{ΛΥΜΕΝΑ ΠΑΡΑΔΕΙΓΜΑΤΑ}}};  
\end{tikzpicture}
\end{center}}
%--------------------------------------

%--------- ΑΛΥΤΕΣ ΑΣΚΗΣΕΙΣ ΤΙΤΛΟΣ ----------
\newcommand{\Alyta}{\begin{center}
\begin{tikzpicture}
\path[left color=black!70!black,right color=black!80!black,middle color=black!80!white] (-7cm,-.6cm) rectangle (6.5cm,.6cm);
\node at (-.25cm,0) {\Large \textcolor{white}{\textbf{ΑΣΚΗΣΕΙΣ - ΠΡΟΒΛΗΜΑΤΑ}}};  
\end{tikzpicture}
\end{center}}
%--------------------------------------------
\usetikzlibrary{shadows,calc}
\usepackage{tcolorbox}
\tcbuselibrary{skins,theorems,breakable}
%---------- ΜΕΘΟΔΟΣ --------------
\newcounter{Methodos}[chapter]
\renewcommand{\theMethodos}{\thechapter.\arabic{Methodos}}
\newenvironment{Methodos}[2][\linewidth]
{\refstepcounter{Methodos}
\begin{tcolorbox}[breakable,
enhanced standard,
boxrule=0.7pt,titlerule=-.2pt,drop fuzzy shadow southeast=black!50,
width=\linewidth,
title style={color=white},
overlay unbroken and first={
\path[left color=cyan!80!black,right color=cyan!50!white,draw=black]
([yshift=-\pgflinewidth+0.2pt,xshift=.2pt]frame.north west) to ([yshift=-5pt,xshift=-.5pt]title.south west)[rounded corners=2pt] -- ([xshift=-#2-15pt,yshift=-5pt]title.south east) to[rounded corners=2pt] ([xshift=-#2,yshift=-\pgflinewidth+0.2pt]frame.north east) -- cycle;
},
fonttitle=\bfseries,
before=\par\medskip\noindent,
after=\par\medskip,
toptitle=3pt,
top=11pt,topsep at break=-5pt,
colback=white,title={\large Μέθοδος \theMethodos} : {\textcolor{black}{\MakeUppercase{\onoma#1}}}]}
{\end{tcolorbox}}
%------------------------------------------
%-------- ΠΑΡΑΤΗΡΗΣΕΙΣ -----------------
\newcounter{parathrhsh}[chapter]
\renewcommand{\theparathrhsh}{\arabic{parathrhsh}}  
\newcommand{\Parathrhsh}[1]{\refstepcounter{parathrhsh}{\faLightbulb\ \ \ \textbf{Παρατήρηση\hspace{2mm}\thechapter.\theparathrhsh}}}{}

%----------- ΠΑΡΑΤΗΡΗΣΗ------------------
\newenvironment{parat}[1]
{\begin{tcolorbox}[title=\Parathrhsh,
breakable,
enhanced standard,lifted shadow={1mm}{-2mm}{3mm}{0.3mm}%
{black!50!white},
colback=yellow!5!white,
boxrule=0.1pt,
colframe=yellow!70!black,
fonttitle=\bfseries,width=#1]}
{\end{tcolorbox}}
%-----------------------------------------
%-----------------------------------------
%---------------------------------------
%---------- ΛΙΣΤΕΣ ----------------------
\newlist{bhma}{enumerate}{3}
\setlist[bhma]{label=\bf\textit{\arabic*\textsuperscript{o}\;Βήμα :},leftmargin=3mm,itemindent=1.8cm,ref=\bf{\arabic*\textsuperscript{o}\;Βήμα}}
\newlist{rlist}{enumerate}{3}
\setlist[rlist]{itemsep=0mm,label=\roman*.}


%----ΣΤΥΛ ΑΣΚΗΣΗΣ ----------
\newcounter{askhsh}[chapter]
\renewcommand{\theaskhsh}{\bf{{\large{\thechapter}}.\arabic{askhsh}}}   
\newcommand{\Askhsh}{\refstepcounter{askhsh}\textcolor{\xrwma}{{\theaskhsh}\hspace{1mm}}}{}
%---------------------------
\newlist{alist}{enumerate}{3}
\setlist[alist]{itemsep=0mm,label=\alph*.}
\newlist{brlist}{enumerate}{3}
\setlist[brlist]{itemsep=0mm,label=\bf\roman*.}
\newlist{tropos}{enumerate}{3}
\setlist[tropos]{label=\bf\textit{\arabic*\textsuperscript{oς}\;Τρόπος :},leftmargin=0cm,itemindent=2.3cm,ref=\bf{\arabic*\textsuperscript{oς}\;Τρόπος}}
% Αν μπει το bhma μεσα σε tropo τότε
%\begin{bhma}[leftmargin=.7cm]
\newcommand{\tss}[1]{\textsuperscript{#1}}
\newcommand{\tssL}[1]{\MakeLowercase{\textsuperscript{#1}}}
\newcommand{\tssLb}[1]{\MakeLowercase{\textsuperscript{\textbf{#1}}}}
%------------------------------------------
\setlength{\parindent}{0pt}
\setlist[itemize]{itemsep=0mm}
\tkzSetUpPoint[size=7,fill=white]
\newcommand{\twocolkentro}[1]{
\twocolumn[
\begin{@twocolumnfalse}
#1
\end{@twocolumnfalse}]}
\newcommand{\bcc}[1]{
\begin{center}
{\color{\xrwma}{\hrulefill}\raisebox{-2.5mm}{\rule{.4pt}{5mm}}}\hspace{1em}\raisebox{-.65ex}{\begin{varwidth}{\dimexpr0.7\textwidth-2em\relax}\centering{\textbf{\textcolor{\xrwma}{#1}}}\end{varwidth}}\hspace*{1em}{\color{\xrwma}{\raisebox{-2.5mm}{\rule{.4pt}{5mm}}\hrulefill}}
\end{center}}
\usepackage{etoolbox}
\makeatletter
\renewrobustcmd{\anw@true}{\let\ifanw@\iffalse}
\renewrobustcmd{\anw@false}{\let\ifanw@\iffalse}\anw@false
\newrobustcmd{\noanw@true}{\let\ifnoanw@\iffalse}
\newrobustcmd{\noanw@false}{\let\ifnoanw@\iffalse}\noanw@false
\renewrobustcmd{\anw@print}{\ifanw@\ifnoanw@\else\numer@lsign\fi\fi}
\makeatother

\DeclareMathSizes{10.95}{10.95}{7}{5}
\DeclareMathSizes{6}{6}{3.8}{2.7}
\DeclareMathSizes{8}{8}{5.1}{3.6}
\DeclareMathSizes{9}{9}{5.8}{4.1}
\DeclareMathSizes{10}{10}{6.4}{4.5}
\DeclareMathSizes{12}{12}{7.7}{5.5}
\DeclareMathSizes{14.4}{14.4}{9.2}{6.5}
\DeclareMathSizes{17.28}{17.28}{11}{7.9}
\DeclareMathSizes{20.74}{20.74}{13.3}{9.4}
\DeclareMathSizes{24.88}{24.88}{16}{11.3}

\makeatletter
\AtBeginDocument{
\check@mathfonts
\fontdimen16\textfont2=2.5pt
\fontdimen17\textfont2=2.5pt
\fontdimen14\textfont2=4.5pt
\fontdimen13\textfont2=4.5pt 
}
\makeatother

%----------- ΟΡΙΣΜΟΣ------------------
\newcounter{orismos}[chapter]
\renewcommand{\theorismos}{\thechapter.\arabic{orismos}}   
\newcommand{\Orismos}{\refstepcounter{orismos}{\textbf{\textbf{\textcolor{\xrwma}{{\large Ορισμός\hspace{2mm}\theorismos}}}}}\hspace{1mm}}{}
%------------------------------------
\newenvironment{orismos}[1]
{\begin{tcolorbox}[title=\Orismos\ \ :\ \  {\textcolor{black}{\MakeUppercase{\onoma#1}}},breakable,
enhanced standard,titlerule=-.2pt,toprule=0pt, rightrule=0pt, bottomrule=0pt,
colback=white,left=2mm,top=1mm,bottom=0mm,
boxrule=0pt,
colframe=white,borderline west={1.5mm}{0pt}{\xrwma},leftrule=2mm,sharp corners,coltitle=\xrwma]}
{\end{tcolorbox}}
%-----------------------------------------

%----------- ΘΕΩΡΗΜΑ------------------
\newcounter{thewrhma}[chapter]
\renewcommand{\thethewrhma}{\thechapter.\arabic{thewrhma}} 
\newcommand{\Thewrhma}{\refstepcounter{thewrhma}{\textbf{\textcolor{olivedrab}{{\large Θεώρημα\hspace{2mm}\thethewrhma}}\hspace{1mm}}}}{}

\newenvironment{thewrhma}[1]
{\begin{tcolorbox}[title=\Thewrhma\ \ :\ \  {\textcolor{black}{\MakeUppercase{\onoma#1}}},breakable,
enhanced standard,titlerule=-.2pt,toprule=0pt, rightrule=0pt, bottomrule=0pt,
colback=white,left=2mm,top=1mm,bottom=0mm,
boxrule=0pt,
colframe=white,borderline west={1.5mm}{0pt}{olivedrab},leftrule=2mm,sharp corners,coltitle=black]}
{\end{tcolorbox}}
%-----------------------------------------

%------- ΣΤΥΛ ΠΑΡΑΔΕΙΓΜΑΤΟΣ -------
\newcounter{paradeigma}[chapter]
\renewcommand{\theparadeigma}{\bf\thechapter.\arabic{paradeigma}}   
\newcommand{\Paradeigma}[1]{\refstepcounter{paradeigma}\textcolor{magenta!80!black}{\textbf{{\large \faPlay\ \ Παράδειγμα\hspace{2mm}\theparadeigma\;:\;}\hspace{1mm}}} {\onoma{#1}}\\}{}
%-----------------------------------

%------- ΣΤΥΛ ΛΥΣΗΣ ------------------
\newcommand{\lysh}{\textcolor{cyan!80!black}{\onoma{\faCheck\ ΛΥΣΗ}}}
%------------------------------------

%-------- ΠΡΟΣΟΧΗ -----------------
\newcounter{prosoxi}[chapter]
\renewcommand{\theprosoxi}{\arabic{prosoxi}}  
\newcommand{\Prosoxi}[1]{\refstepcounter{prosoxi}{\faExclamationTriangle\ \ \ \textbf{Προσοχή\hspace{2mm}\thechapter.\theprosoxi}}}{}

%----------- ΠΡΟΣΟΧΗ------------------
\newenvironment{prosoxi}[1]
{\begin{tcolorbox}[title=\Prosoxi,
breakable,
enhanced standard,lifted shadow={1mm}{-2mm}{3mm}{0.3mm}%
{black!50!white},
colback=red!5!white,
boxrule=0.1pt,
colframe=red!80!black,
fonttitle=\bfseries,width=#1]}
{\end{tcolorbox}}
%-----------------------------------------
\usepackage{lipsum}



\begin{document}
\title{\MakeUppercase{Προετοιμασία για τη Γ΄ Λυκείου}}
\pagestyle{empty}
\frontmatter
\begin{titlepage}
\newgeometry{left=2.5cm,top=2.5cm} %defines the geometry for the titlepage
\pagecolor{white}
\begin{center}
\begin{tabular}{cc}
Αλέξανδρος Πολίτης & Σπύρος Φρόνιμος\\Μαθηματικός & Μαθηματικός
\end{tabular}
\end{center}
\noindent
\par
\noindent
\mbox{}\\\\
\begin{center}
\textbf{\fontsize{20}{40}\selectfont{ΕΠΑΝΑΛΗΨΗ ΣΤΑ ΜΑΘΗΜΑΤΙΚΑ}}\par\mbox{}\\\vspace{-3mm}
\textbf{\fontsize{20}{40}\selectfont{ΓΙΑ ΤΗ Γ΄ ΛΥΚΕΙΟΥ}}\par\mbox{}\\
\vspace{-4mm}
\rule{12cm}{0.1mm}\\
\vspace{3mm}
{\fontsize{15}{15}\MakeUppercase{βασική θεωρία και μεθοδολογία }}\\
\vspace{.7mm}
{\fontsize{15}{15}\MakeUppercase{από την Α΄ και Β΄ ΛΥΚΕΙΟΥ}}\\
\vspace{.7mm}
\rule{12cm}{0.1mm}\\
\end{center}
\vspace{3cm}
\begin{flushright}
\begin{itemize}
\item 100 Ορισμοί
\item 250 Θεωρήματα
\item 400 Μέθοδοι για λύση ασκήσεων
\item 200 Λυμένα παραδείγματα
\item 500 Άλυτες ασκήσεις και προβλήματα
\item 200 Επαναληπτικά θέματα
\item Απαντήσεις ασκήσεων
\end{itemize}
\end{flushright}

\vfill
\noindent
\color{black}
\begin{center}
{\large{ΕΚΔΟΣΕΙΣ \_\_\_\_\_}\\
\large{ΚΕΡΚΥΡΑ 2018}}
\vskip\baselineskip
\end{center}
\hbox{ % Horizontal box
\hspace*{0.2\textwidth} % Whitespace to the left of the title page
\rule{1pt}{\textheight} % Vertical line
\hspace*{0.05\textwidth} % Whitespace between the vertical line and title page text
\parbox[b]{0.75\textwidth}{ % Paragraph box which restricts text to less than the width of the page

{\textbf{Επανάληψη στα μαθηματικά}\\\textbf{για τη Γ΄ Λυκείου}\\\\\noindent 
\textbf{Αλέξανδρος Πολίτης - Μαθηματικός}\\
\textbf{Σπύρος Φρόνιμος - Μαθηματικός}\\
e-mail : spyrosfronimos@gmail.com\\[0.5\baselineskip]
Σελίδες : ...\\
ISBN : ...\\
Εκδόσεις : ...\\
\textcopyright Copyright 2018}\\[2\baselineskip] % Title
{Φιλολογική Επιμέλεια :\\\textbf{Μαρία Πρεντουλή}}
{- e-mail : predouli@yahoo.com}\\[0.5\baselineskip]
{Εξώφυλλο : \\\textbf{}}\\[1\baselineskip] % Tagline or further description
% Author name

\vspace{.4\textheight} % Whitespace between the title block and the publisher
{Πνευματικά Δικαιώματα : ...}\\[\baselineskip]}}
\vspace*{2\baselineskip}
\newpage
\mbox{}\\\\\\\\\\\\
\hspace*{0.75\textwidth}
\textit{{\large Στη γυναίκα μου.}}
\newpage
\mbox{}
\newpage
\mbox{}
{\LARGE \textbf{Πρόλογος}}\\\\\\\\


\newpage
\end{titlepage}
\restoregeometry % restores the geometry
\tableofcontents
\newcolumntype{K}{>{\centering\arraybackslash}m{7.75cm} }
\chapter*{Πίνακας Συμβόλων}
\addtocontents{toc}{{\sc{\large Πίνακας Συμβόλων}}\dotfill\thepage\\\\}
{\par\centering

\begin{tabularx}{\textwidth}{c>{\centering}m{3.1cm}K}
\hline\rule[-2ex]{0pt}{5.5ex}\textbf{Σύμβολο} & \textbf{Όνομα} & \textbf{Περιγραφή}\\
\hhline{===}\rule[-2ex]{0pt}{8.5ex}
$ \neq $ & Διάφορο & Εκφράζει ότι δύο στοιχεία είναι διαφορετικά μεταξύ τους.\\
\rule[-2ex]{0pt}{5.5ex}
$ > $ & Μεγαλύτερο & Δηλώνει ανισότητα ανάμεσα σε δύο στοιχεία. (Το 1\textsuperscript{ο} μεγαλύτερο του 2\textsuperscript{ου}).\\
\rule[-2ex]{0pt}{5.5ex}
$ < $ & Μικρότερο & Δηλώνει ανισότητα ανάμεσα σε δύο στοιχεία. (Το 1\textsuperscript{ο} μικρότερο του 2\textsuperscript{ου}).\\
\rule[-2ex]{0pt}{5.5ex}
$ \geq $ & Μικρότερο ίσο & Συνδυασμός των σχέσεων $=$ και $>$.\\
\rule[-2ex]{0pt}{5.5ex}
$ \leq $ & Μικρότερο ίσο & Συνδυασμός των σχέσεων $=$ και $<$.\\
\rule[-2ex]{0pt}{5.5ex}
$ \gtrless $ & Μεγαλύτερο μικρότερο ίσο & Συνδυασμός των σχέσεων $>$ και $<$.\\
\rule[-2ex]{0pt}{5.5ex}
$ \pm $ & Συν Πλην & Συνδυασμός των προσήμων $+$ και $-$.\\
\rule[-2ex]{0pt}{5.5ex}
$ \mp $ & Πλην Συν & Έχει την ίδια σημασία με το συμβολισμό $\pm$  και χρησιμοποιείται όταν θέλουμε να αλλάξουμε τη σειρά με την οποία θα εμφανιστούν τα πρόσημα $+\;,\;-$. \\
\rule[-2ex]{0pt}{7ex}
$ \Rightarrow $ & Συνεπαγωγή & Συνδέει δύο μαθηματικές προτάσεις, όταν η μια έχει σαν συμπέρασμα την άλλη. \\
\rule[-2ex]{0pt}{7ex}
$ \Leftarrow $ & Αντίστροφη συνεπαγωγή & Συνδέει δύο μαθηματικές προτάσεις με φορά αντίστροφη από το σύνδεσμο $ \Rightarrow $. \\
\rule[-2ex]{0pt}{7ex}
$ \Leftrightarrow $ & Διπλή συνεπαγωγή & Συνδέει δύο μαθηματικές προτάσεις με διπλή φορά. Δηλώνει ισοδυναμία μεταξύ τους.  \\
\rule[-2ex]{0pt}{7ex}
$ \% $ & Ποσοστό τοις εκατό & Μέρος μιας ποσότητας μοιρασμένης σε 100 ίσα κομμάτια. \\
\rule[-2ex]{0pt}{7ex}
$ \perthousand $ & Ποσοστό τοις χιλίοις & Μέρος μιας ποσότητας μοιρασμένης σε 1000 ίσα κομμάτια. \\
\rule[-2ex]{0pt}{5.5ex}
$ |\;\;| $ & Απόλυτη τιμή & Απόσταση ενός αριθμού από το 0. \\
\rule[-2ex]{0pt}{5.5ex}
$ \sqrt{\;\;} $ & Τετραγωνική ρίζα & Βλ. \textbf{Ορισμό ...} \\
\hline
\end{tabularx}
\par}
\newpage
\noindent
{\par\centering
\begin{tabularx}{\textwidth}{c|>{\centering}m{3.1cm}|>{\centering\arraybackslash}m{7.75cm}}
\hline\rule[-2ex]{0pt}{5.5ex}\textbf{Σύμβολο} & \textbf{Όνομα} & \textbf{Περιγραφή}\\
\hhline{===}\rule[-2ex]{0pt}{8.5ex}
$ \sqrt[\nu]{\;\;} $ & ν-οστή ρίζα & Βλ. \textbf{Ορισμό ...} \\
$ \in $ & Ανήκει & Σύμβολο το οποίο δηλώνει ότι ένα στοιχείο ανήκει σε ένα σύνολο. \\
\rule[-2ex]{0pt}{7.5ex}
$ \ni $ & Ανήκει & Έχει την ίδια χρησιμότητα με το σύμβολο $ \in $ και χρησιμοποιείται όταν το σύνολο γράφεται πριν το στοιχείο. \\
\rule[-2ex]{0pt}{7.5ex}
$ \notin $ & Δεν ανήκει & Έχει την αντίθετη σημασία από το σύμβολο $ \in $ και δηλώνει ότι ένα στοιχείο δεν ανήκει σε ένα σύνολο. \\
\rule[-2ex]{0pt}{5.5ex}
$ \subseteq $ & Υποσύνολο & Βλ. \textbf{Ορισμό ...} \\
\rule[-2ex]{0pt}{5.5ex}
$ \cup\;,\;\cap $ & Ένωση, Τομή & Βλ. \textbf{Ορισμό ...} \\
\rule[-2ex]{0pt}{5.5ex}
$ \varnothing $ & Κενό σύνολο & Βλ. \textbf{Ορισμό ...} \\
\rule[-2ex]{0pt}{5.5ex}
$ \infty $ & Άπειρο & \\
\rule[-2ex]{0pt}{5.5ex}
$ \bot $ & Κάθετο & \\
\hline
\end{tabularx}
\par}
\mainmatter

%\part{Άλγεβρα}
%\pagestyle{fancy}
%%\chapter{Πραγματικοί αριθμοί}
\section{Αριθμοί - Πράξεις - Ισότητες}
\Thewria
\orismoi
\Orismos{Δύναμη πραγματικου αριθμου}
Δύναμη ενός πραγματικού αριθμού $ a $ ονομάζεται το γινόμενο $ \nu $ ίσων παραγόντων του αριθμού αυτού. Συμβολίζεται με $ a^\nu $ όπου $ \nu\in\mathbb{N} $ είναι το πλήθος των παραγόντων. 
\[ \undercbrace{a\cdot a\cdot\ldots a}_{\nu\textrm{ παράγοντες }}=a^\nu \]
\begin{itemize}[itemsep=0mm]
\item Ο αριθμός $ a $ ονομάζεται \textbf{βάση} και ο αριθμός $ \nu $ \textbf{εκθέτης} της δύναμης.
\end{itemize}
\thewrhmata
\Thewrhma{Ιδιότητεσ των Πράξεων}
Στον παρακάτω πίνακα βλέπουμε τις βασικές ιδιότητες της πρόσθεσης και του πολλαπλασιασμού στο σύνολο των πραγματικών αριθμών.
\begin{center}
\begin{tabular}{ccc}
\hline \rule[-2ex]{0pt}{5.5ex} \textbf{Ιδιότητα} & \textbf{Πρόσθεση} & \textbf{Πολλαπλασιασμός} \\ 
\hhline{===} \rule[-2ex]{0pt}{5.5ex} \textbf{Αντιμεταθετική} & $ a+\beta=\beta+a $ & $ a\cdot\beta=\beta\cdot a $ \\
\rule[-2ex]{0pt}{5ex} \textbf{Προσεταιριστική} & $ a+\left( \beta+\gamma\right) =\left( a+\beta\right) +\gamma $ & $ a\cdot\left( \beta\cdot\gamma\right) =\left( a\cdot\beta\right)\cdot\gamma $\\
\rule[-2ex]{0pt}{5ex} \textbf{Ουδέτερο στοιχείο} & $ a+0=a $ & $ a\cdot1= a $\\
\rule[-2ex]{0pt}{5ex} \textbf{Αντίθετοι / Αντίστροφοι} & $ a+(-a)=0 $ & $ a\cdot\frac{1}{a}= 1 $\\
\rule[-2ex]{0pt}{5ex} \textbf{Επιμεριστική} & \multicolumn{2}{c}{$ a\cdot\left( \beta\pm\gamma\right)=a\cdot\beta\pm a\cdot\gamma  $}\\
\hline
\end{tabular}\captionof{table}{Ιδιότητες πράξεων}
\end{center}
Επιπλέον ισχύει ότι:
\begin{itemize}[itemsep=0mm]
\item Δύο αριθμοί που έχουν άθροισμα 0 λέγονται \textbf{αντίθετοι}.
\item Δύο αριθμοί που έχουν γινόμενο 1 λέγονται \textbf{αντίστροφοι}.
\item Το 0 δεν έχει αντίστροφο.
\end{itemize}
\Thewrhma{Ιδιότητεσ ισοτήτων}
Για κάθε ισότητα $ a=\beta $ με $ a,\beta $ πραγματικούς αριθμούς ισχύουν οι παρακάτω ιδιότητες.
\begin{rlist}
\item Εάν προσθέσουμε, αφαιρέσουμε, πολλαπλασιάζουμε ή διαιρέσουμε κάθε μέλος μιας ισότητας με τον ίδιο πραγματικό αριθμό, προκύπτει ξανά ισότητα.
\begin{multicols}{2}
\begin{enumerate}
\item $ a=\beta\Rightarrow a+\gamma=\beta+\gamma $
\item $ a=\beta\Rightarrow a-\gamma=\beta-\gamma $
\item $ a=\beta\Rightarrow a\cdot\gamma=\beta\cdot\gamma $
\item $ a=\beta\Rightarrow \dfrac{a}{\gamma}=\dfrac{\beta}{\gamma} $ όπου $ \gamma\neq 0 $
\end{enumerate}
\end{multicols}
\item Εάν υψώσουμε κάθε μέλος μιας ισότητας στον ίδιο \begin{enumerate}
\item \textbf{άρτιο} εκθέτη, προκύπτει ξανά ισότητα. Το αντίστροφο δεν ισχύει πάντα.
\item \textbf{περιττό} εκθέτη, προκύπτει ξανά ισότητα. Το αντίστροφο ισχύει.
\begin{gather*}
a=\beta\Rightarrow a^{2\nu}=\beta^{2\nu}\\
a=\beta\Leftrightarrow a^{2\nu+1}=\beta^{2\nu+1}
\end{gather*}
\end{enumerate}
\item Εάν δύο μη αρνητικοί πραγματικοί αριθμοί $ a,\beta\geq0 $ είναι ίσοι τότε και οι ν-οστές ρίζες τους, $ \nu\in\mathbb{N} $, θα είναι με ίσες και αντίστροφα.
\begin{gather*}
a=\beta\Leftrightarrow\sqrt[\nu]{a}=\!\sqrt[\nu]{\beta}
\end{gather*}
\end{rlist}
\Thewrhma{Πράξεισ μεταξύ ισοτήτων}
Προσθέτοντας κατά μέλη δύο ισότητες $ a=\beta $ και $ \gamma=\delta $ προκύπτει ισότητα. Η ιδιότητα αυτή ισχύει και για αφαίρεση, πολλαπλασιασμό και διαίρεση κατά μέλη.
\[ a=\beta\;\;\textrm{και}\;\;\gamma=\delta\Rightarrow
\ccases{
\textrm{\textbf{{1. Πρόσθεση κατά μέλη }}}& a+\gamma=\beta+\delta\\\textrm{\textbf{{2. Αφαίρεση κατά μέλη }}}& a-\gamma=\beta-\delta\\\textrm{\textbf{{3. Πολλαπλασιασμός κατά μέλη }}}& a\cdot\gamma=\beta\cdot\delta\\\textrm{\textbf{{4. Διαίρεση κατά μέλη }}}& \dfrac{a}{\gamma}=\dfrac{\beta}{\delta}\;\;,\;\;\gamma\cdot\delta\neq0} \]
Ο κανόνας αυτός επεκτείνεται και για πράξεις κατά μέλη σε περισσότερες από δύο ισότητες, στις πράξεις της πρόσθεσης και του πολλαπλασιασμού\\\\
\Thewrhma{Νόμοσ διαγραφησ προσθεσησ \& πολλαπλασιασμου}
Για οποιουσδήποτε πραγματικούς αριθμούς $ a,x,y\in\mathbb{R} $ ισχύουν οι παρακάτω σχέσεις.
\[ a+x=a+y\Rightarrow x=y\;\;\textrm{ και }\;\;a\cdot x=a\cdot y\Rightarrow x=y \]
Διαγράφουμε απ' τα μέλη μιας ισότητας τον ίδιο προσθετέο ή τον ίδιο \textbf{μη μηδενικό} παράγοντα.\\\\
\Thewrhma{Μηδενικό και μη μηδενικό γινόμενο}
Εάν το γινόμενο δύο πραγματικών αριθμών είναι μηδέν τότε τουλάχιστον ένα από τους δύο είναι μηδέν.
\[ a\cdot\beta=0\Leftrightarrow a=0\textrm{ \textbf{ή} }\beta=0 \]
Εάν το γινόμενο δύο πραγματικών αριθμών είναι μη μηδενικό τότε κανένας από τους δύο δεν ισούται με το μηδέν.
\[ a\cdot\beta\neq0\Leftrightarrow a\neq0\textrm{ \textbf{και} }\beta\neq0 \]
Γενικότερα για $ \nu $ σε πλήθος πραγματικών αριθμών θα ισχύει 
\begin{rlist}
\item $ a_1\cdot a_2\cdot\ldots\cdot a_\nu=0\Leftrightarrow a_1=0\textrm{ ή }a_2=0\textrm{ ή }\ldots\textrm{ ή }a_\nu=0 $.
\item $ a_1\cdot a_2\cdot\ldots\cdot a_\nu\neq0\Leftrightarrow a_1\neq0\textrm{ και }a_2\neq0\textrm{ και }\ldots\textrm{ και }a_\nu\neq0 $.
\end{rlist}
\Thewrhma{Ιδιότητεσ δυνάμεων}
Για κάθε δύναμη με βάση έναν πραγματικό αριθμό $ a\in\mathbb{R} $ ισχύει ότι
\[ a^1=a\;\;,\;\;a^0=1\;,\;\textrm{όπου }a\neq0\;\;,\;\;a^{-\nu}=\dfrac{1}{a^\nu}\;,\;\textrm{όπου }a\neq0 \]
Επίσης για οποιουσδήποτε πραγματικούς αριθμούς $ a,\beta\in\mathbb{R} $ και φυσικούς αριθμούς $ \nu,\mu\in\mathbb{N} $ ισχύουν οι παρακάτω ιδιότητες:
\begin{center}
\begin{tabular}{cc}
\hline \rule[-2ex]{0pt}{5.5ex} \textbf{Ιδιότητα} & \textbf{Συνθήκη} \\
\hhline{==}\rule[-2ex]{0pt}{5.5ex} Γινόμενο δυνάμεων με κοινή βάση & $ a^\nu\cdot a^\mu=a^{\nu+\mu} $ \\
\rule[-2ex]{0pt}{5.5ex} Πηλίκο δυνάμεων με κοινή βάση & $ a^\nu: a^\mu=a^{\nu-\mu} $\\
\rule[-2ex]{0pt}{5.5ex}  Γινόμενο δυνάμεων με κοινό εκθέτη & $ \left(a\cdot\beta\right)^\nu=a^\nu\cdot\beta^\nu $ \\
\rule[-2ex]{0pt}{5.5ex}  Πηλίκο δυνάμεων με κοινό εκθέτη & $ \left(\dfrac{a}{\beta}\right)^\nu=\dfrac{a^\nu}{\beta^\nu}\;\;,\;\;\beta\neq0 $ \\
\rule[-2ex]{0pt}{5.5ex}  Δύναμη υψωμένη σε δύναμη & $ \left( a^\nu\right)^\mu=a^{\nu\cdot\mu} $ \\
\rule[-2ex]{0pt}{5.5ex}  Κλάσμα με αρνητικό εκθέτη & $ \left( \dfrac{a}{\beta}\right)^{-\nu}=\left(\dfrac{\beta}{a}\right)^\nu\;\;,\;\;a,\beta\neq0 $ \vspace{2mm}\\
\hline
\end{tabular}\captionof{table}{Ιδιότητες δυνάμεων}
\end{center}
Οι ιδιότητες 1 και 3 ισχύουν και για γινόμενο περισσότερων των δύο παραγόντων. Θα έχουμε ότι:
\begin{gather*}
a^{\nu_1}\cdot a^{\nu_2}\cdot\ldots\cdot a^{\nu_\kappa}=a^{\nu_1+\nu_2+\ldots+\nu_\kappa}\\
\left( a_1\cdot a_2\cdot\ldots\cdot a_\kappa\right)^\nu=a_1^\nu\cdot a_2^\nu\cdot\ldots\cdot a_\kappa^\nu
\end{gather*}
όπου $ a,a_1,a_2,\ldots,a_\nu\in\mathbb{R} $ και $ \nu,\nu_1,\nu_2,\ldots,\nu_\kappa\in\mathbb{N} $.\\\\
\Lymena
\section{Ταυτότητες}
\Thewria
\orismoi
\Orismos{ΤΑΥΤΌΤΗΤΑ}
Ταυτότητα ονομάζεται κάθε ισότητα που περιέχει μεταβλητές και επαληθεύεται για κάθε τιμή των μεταβλητών. Παρακάτω βλέπουμε μερικές από τις βασικές ταυτότητες.
\begin{center}
\textbf{ΒΑΣΙΚΕΣ ΤΑΥΤΟΤΗΤΕΣ}
\end{center}
\begin{multicols}{2}
\begin{enumerate}[itemsep=0mm,label=\bf\arabic*.]
\item \parbox[t]{7cm}{\textbf{Άθροισμα στο τετράγωνο}\\$ (a+\beta)^2=a^2+2a\beta+\beta^2 $}
\item \parbox[t]{7cm}{\textbf{Διαφορά στο τετράγωνο}\\$ (a-\beta)^2=a^2-2a\beta+\beta^2 $}
\item \parbox[t]{7cm}{\textbf{Άθροισμα στον κύβο}\\$ (a+\beta)^3=a^3+3a^2\beta+3a\beta^2+\beta^3 $}
\item \parbox[t]{7cm}{\textbf{Διαφορά στον κύβο}\\$ (a-\beta)^3=a^3-3a^2\beta+3a\beta^2-\beta^3 $}
\item \parbox[t]{7cm}{\textbf{Γινόμενο αθροίσματος επί διαφορά}\\$ (a+\beta)(a-\beta)=a^2-\beta^2 $}
\item \parbox[t]{7cm}{\textbf{Άθροισμα κύβων}\\$ (a+\beta)\left(a^2-a\beta+\beta^2 \right)=a^3+\beta^3 $}
\item \parbox[t]{7cm}{\textbf{Διαφορά κύβων}\\$ (a-\beta)\left(a^2+a\beta+\beta^2 \right)=a^3-\beta^3 $}
\end{enumerate}
\end{multicols}\mbox{}\\
Εξίσου χρήσιμες και σημαντικές είναι και οι ακόλουθες ταυτότητες, τις οποίες συναντούμε συχνά και αξίζει να αναφερθούν.
\begin{enumerate}[itemsep=0mm,label=\bf\arabic*.,start=8]
\item \textbf{Άθροισμα τετραγώνων δύο όρων}\\
Η ακόλουθη ταυτότητα μας δίνει μια σχέση για το άθροισμα των τετραγώνων δύο πραγματικών αριθμών $ a,\beta\in\mathbb{R} $ με τη βοήθεια των βασικών ταυτοτήτων \textbf{1} και \textbf{2} :
\[ a^2+\beta^2=(a+\beta)^2-2a\beta=(a-\beta)^2+2a\beta \]
\item \textbf{Άθροισμα κύβων}\\
Μια επιπλέον σχέση η οποία μας δίνει το άθροισμα κύβων δύο οποιονδήποτε πραγματικών αριθμών $ a,\beta\in\mathbb{R} $ είναι η εξής :
\[ a^3+\beta^3=(a+\beta)^3-3a\beta(a+\beta) \]
\item \textbf{Διαφορά κύβων}\\
Αντίστοιχη σχέση για τη διαφορά κύβων δύο πραγματικών αριθμών $ a,\beta\in\mathbb{R} $ είναι :
\[ a^3-\beta^3=(a-\beta)^3+3a\beta(a-\beta) \]
\item \textbf{Τετράγωνο τριωνύμου}\\
Το ανάπτυγμα του τετραγώνου ενός αθροίσματος τριών πραγματικών αριθμών $ a,\beta,\gamma\in\mathbb{R} $ είναι :
\[ (a+\beta+\gamma)^2=a^2+\beta^2+\gamma^2+2a\beta+2\beta\gamma+2a\gamma \]
\item \textbf{Άθροισμα - Διαφορά ν-οστών δυνάμεων}\\
Οι ακόλουθες ταυτότητες αποτελούν μια γενίκευση για το άθροισμα ή τη διαφορά δύο $ \nu- $οστών δυνάμεων δύο πραγματικών αριθμών $ x,y\in\mathbb{R} $:
\begin{gather*}
x^\nu+ y^\nu=(x+ y)\left(x^{\nu-1}- x^{\nu-2}y+x^{\nu-3}y^2-\ldots- xy^{\nu-2}+y^{\nu-1}\right)\\
x^\nu-y^\nu=(x-y)\left(x^{\nu-1}+ x^{\nu-2}y+x^{\nu-3}y^2+\ldots+ xy^{\nu-2}+y^{\nu-1}\right)
\end{gather*}
\end{enumerate}
\Lymena
\section{Παραγοντοποίηση}
\Thewria
\orismoi
\Orismos{Παραγοντοποίηση αλγεβρικών παραστάσεων}
Παραγοντοποίηση ονομάζεται η διαδικασία με την οποία μια αλγεβρική παράσταση από άθροισμα μετατρέπεται σε γινόμενο.
\begin{center}
\textbf{ΒΑΣΙΚΟΙ ΚΑΝΟΝΕΣ ΠΑΡΑΓΟΝΤΟΠΟΙΗΣΗΣ}
\end{center}
\begin{enumerate}[itemsep=0mm,label=\bf\arabic*.]
\item \textbf{Κοινός Παράγοντας}\\
Σύμφωνα με την επιμεριστική ιδιότητα ισχύει:
\[ a\cdot\beta\pm a\cdot\gamma=a\cdot(\beta\pm\gamma) \]
\item \textbf{Ομαδοποίηση}\\
Εάν δεν υπάρχει σε όλους τους όρους μιας παράστασης κοινός παράγοντας οπότε μοιράζονται οι όροι σε ομάδες έτσι ώστε κάθε ομάδα να έχει δικό της κοινό παράγοντα.
\item \textbf{Διαφορά Τετραγώνων}\\
Κάθε παράσταση της μορφής $ a^2-\beta^2 $ γράφεται : \[ a^2-\beta^2=(a-\beta)(a+\beta) \]
\item \textbf{Διαφορά - Άθροισμα Κύβων}\\
Κάθε παράσταση της μορφής $ a^3-\beta^3 $ ή $ a^3+\beta^3 $ γράφεται : \begin{gather*}
a^3-\beta^3=(a-\beta)\left(a^2+a\beta+\beta^2 \right)\\
a^3+\beta^3=(a+\beta)\left(a^2-a\beta+\beta^2 \right)
\end{gather*}
\item \textbf{Ανάπτυγμα Τετραγώνου}\\
Κάθε παράσταση της μορφής $ a^2\pm2a\beta+\beta^2 $ γίνεται :
\begin{gather*}
a^2+2a\beta+\beta^2=(a+\beta)^2\\
a^2-2a\beta+\beta^2=(a-\beta)^2
\end{gather*}
\end{enumerate}
\Lymena
\begin{Methodos}[Κοινός παράγοντας]{7cm}
\begin{bhma}
\item Κοινό παράγοντα από τους όρους μιας παράστασης βγάζουμε Μ.Κ.Δ. τους.
\item Μετά τον κοινό παράγοντα ανοίγουμε παρένθεση και διαιρούμε κάθε όρο της παράστασης με τον κοινό παράγοντα.
\end{bhma}
\end{Methodos}
\Paradeigma{Κοινός παράγοντας - Απλή παράσταση}
\bmath{Να παραγοντοποιηθούν οι παραστάσεις:
\begin{multicols}{2}
\begin{rlist}
\item $ A=4x^2-8xy^3+6x^2y $
\item $ B=2x(a-3)+y(a-3) $
\end{rlist}
\end{multicols}}
\textbf{ΛΥΣΗ}
\begin{rlist}
\item Σύμφωνα με τη μέθοδο, κοινός παράγοντας των όρων της παράστασης θα είναι ο $ 2xy $ (βλ \textbf{Μέθοδο}). Επιπλέον οι όροι μέσα στην παρένθεση θα βρεθούν διαιρώντας τους αρχικούς όρους με τον κοινό παράγοντα.
\[ \frac{4x^2y^2}{2xy}=2xy\ \ ,\ \ \frac{-8xy^3}{2xy}=-4y^2\ \ ,\ \ \frac{6x^2y}{2xy}=3x \]
Έτσι η παράσταση γράφεται:
\begin{align*}
A&=4x^2y^2-8xy^3+6x^2y=\\
&=2xy\cdot\left(2xy-4y^2+3x \right) 
\end{align*}
\item Σ' αυτή την παράσταση παρατηρούμε ότι κοινός παράγοντας είναι μια ολόκληρη παράσταση, η $ a-3 $. Έτσι η παραγοντοποίηση, ομοίως με πριν, θα μας δώσει:
\begin{align*}
B&=2x(a-3)+y(a-3)=\\
&=(a-3)\cdot(2x+y)
\end{align*}
\end{rlist}
\begin{Methodos}[Ομαδοποίηση]{7cm}
\begin{bhma}
\item Χωρίζουμε την παράσταση σε ομάδες ανάλογα με το πλήθος των όρων της.
\item Παραγοντοποιούμε κάθε ομάδα υπολογίζοντας τον κοινό παράγοντα.
\item Συνεχίζουμε την παραγοντοποίηση .......
\end{bhma}
\end{Methodos}
\Paradeigma{Ομαδοποίηση}
\bmath{Να παραγοντοποιηθούν οι παραστάσεις
\begin{multicols}{2}
\begin{rlist}
\item $ A=x^3-4x^2+3x-12 $
\item $ B=x^3-x^2+x-1 $
\end{rlist}
\end{multicols}}
\textbf{ΛΥΣΗ}
\begin{rlist}
\wrapr{-15mm}{11}{5.1cm}{3mm}{\begin{parat}{5.1cm}
Η επιλογή των ομάδων δεν είναι αυστηρή, αρκεί σε κάθε ομάδα να υπάρχει κοινός παράγοντας. 
\end{parat}}{
\item Επιλέγουμε τους δύο πρώτους όρους ως πρώτη ομάδα και τους άλλους δύο ως δεύτερη.
\begin{align*}
A=x\tikzmark{a}^3-4\tikzmark{b}x^2+3\tikzmark{c}x-1\tikzmark{d}2\tikz[overlay,remember picture]
{\draw[latex-latex,square arrow] (a.south) to (b.south);} \tikz[overlay,remember picture]
{\draw[latex-latex,square arrow] (c.south) to (d.south);}&=x^2(x-4)+3(x-4)
\end{align*}
Μετά την παραγοντοποίηση κάθε ομάδας, οι τέσσερις όροι έγιναν δύο. Αυτοί στη συνέχεια έχουν κοινό παράγοντα το $ x-4 $ άρα:
\[ A=x^2(x-4)+3(x-4)=(x-4)\left( x^2+3\right)  \]}
\item Ομοίως και στην παράσταση αυτή επιλέγουμε τους δύο πρώτους όρους ως πρώτη ομάδα. Παρατηρούμε όμως ότι στη δεύτερη κοινός παράγοντας είναι το $ 1 $. Παραγοντοποιώντας όμως την πρώτη, η παράσταση μέσα στην παρένθεση ταυτίζεται με αυτήν της δεύτερης ομάδας. Έτσι
\begin{align*}
B&=x^3-x^2+x-1=\\&=x^2(x-1)+(x-1)=(x-1)\left( x^2+1\right)
\end{align*}
\end{rlist}
\begin{Methodos}[Διαφορά τετραγώνων]{8cm}
Σε κάθε παράσταση που έχει ή μπορεί να γραφτεί ως διαφορά τετραγώνων $ a^2-\beta^2 $:
\begin{bhma}
\item Γράφουμε κάθε όρο της ως δύναμη στο τετράγωνο. (Οι βάσεις των δυνάμεων μπορούν να θεωρηθούν ως $ a $ και $ \beta $.)
\item Σχηματίζουμε το γινόμενο $ (a-\beta)(a+\beta) $.
\end{bhma}
\end{Methodos}
\Paradeigma{Διαφορά τετραγώνων}
\bmath{Να παραγοντοποιηθεί η παράσταση $ 4x^2-25 $.}\\\\
\textbf{ΛΥΣΗ}\\
Αφού γραφτεί κάθε όρος της παράστασης ως δύναμη στη δευτέρα τότε θα δούμε ότι $ a=2x $ και $ \beta=5 $. Πράγματι:
\[ 4x^2-25=(2x)^2-5^2=(2x+5)(2x-5) \]
\section{Διάταξη}
\Thewria
\orismoi
\Orismos{Διάστημα - κεντρο - ακτινα διαστηματοσ}
Κλειστό διάστημα ονομάζεται το σύνολο των πραγματικών αριθμών που βρίσκονται μεταξύ δύο αριθμών $ a,\beta\in\mathbb{R} $. Συμβολίζεται με $ [a,\beta] $.
\[ [a,\beta]=\{x\in\mathbb{R}|a\leq x\leq \beta\} \]
\begin{itemize}[itemsep=0mm]
\item Οι $ a,\beta $ ονομάζονται \textbf{άκρα} του διαστήματος.
\item Κάθε διάστημα μπορεί να εκφραστεί σαν ανισότητα και αντίστροφα.
\item Αν από το κλειστό διάστημα παραλείψουμε τα άκρα $ a,\beta $ τό διάστημα που προκύπτει ονομάζεται \textbf{ανοιχτό διάστημα} $ (a,\beta) $.
\item Το σύνολο των πραγματικών αριθμών $ x $ για τους οποίους ισχύει $ x\geq a $ ορίζουν το διάστημα $ [a,+\infty) $. Ομοίως οι ανισότητες $ x>a,x\leq a $ και $ x<a $ ορίζουν τα διαστήματα $ (a,+\infty), (-\infty,a] $ και $ (-\infty,a) $ αντίστοιχα .
\end{itemize}
Στον παρακάτω πίνακα βλέπουμε όλους τους τύπους διαστημάτων, τη γραφική παράστασή τους καθώς και το πως παριστάνεται το καθένα σαν ανισότητα.
\begin{center}
\begin{longtable}{cc>{\centering\arraybackslash}m{4cm}c}
\hline \rule[-2ex]{0pt}{5.5ex} \textbf{Διάστημα} & \textbf{Ανισότητα} & \textbf{Σχήμα} & \textbf{Περιγραφή} \\ 
\hhline{====} \rule[-2ex]{0pt}{5.5ex} $ [a,\beta] $ & $ a\leq x\leq\beta $ & \begin{tikzpicture}
\tkzDefPoint(0,.57){A}
\diasthma{a}{ \beta }{.7}{2.3}{.3}{\xrwma!30}
\axonas{0}{3}
\akro{k}{.7}
\akro{k}{2.3}
\end{tikzpicture} & Κλειστό $ a,\beta $ \\ 
$ (a,\beta) $ & $ a< x<\beta $ & \begin{tikzpicture}
\tkzDefPoint(0,.57){A}
\diasthma{a}{ \beta }{.7}{2.3}{.3}{\xrwma!30}
\axonas{0}{3}
\akro{a}{.7}
\akro{a}{2.3}
\end{tikzpicture} & Ανοιχτό $ a,\beta $\\
$ [a,\beta) $ & $ a\leq x<\beta $ & \begin{tikzpicture}
\tkzDefPoint(0,.57){A}
\diasthma{a}{ \beta }{.7}{2.3}{.3}{\xrwma!30}
\axonas{0}{3}
\akro{k}{.7}
\akro{a}{2.3}
\end{tikzpicture} & Κλειστό $a$ ανοιχτό $\beta$\\
$ (a,\beta] $ & $ a< x\leq\beta $ & \begin{tikzpicture}
\tkzDefPoint(0,.57){A}
\diasthma{a}{ \beta }{.7}{2.3}{.3}{\xrwma!30}
\axonas{0}{3}
\akro{a}{.7}
\akro{k}{2.3}
\end{tikzpicture} & Ανοιχτό $a$ κλειστό $\beta$ \\
$ [a,+\infty) $ & $ x\geq a $ & \begin{tikzpicture}
\tkzDefPoint(0,.57){A}
\Xapeiro{a}{.7}{3}{.3}{\xrwma!30}
\axonas{0}{3}
\akro{k}{.7}
\end{tikzpicture} & Κλειστό $a$ συν άπειρο \\
$ (a,+\infty) $ & $ x>a $ & \begin{tikzpicture}
\tkzDefPoint(0,.57){A}
\Xapeiro{a}{.7}{3}{.3}{\xrwma!30}
\axonas{0}{3}
\akro{a}{.7}
\end{tikzpicture} & Ανοιχτό $a$ συν άπειρο \\
$ (-\infty,a] $ & $ x\leq a $ & \begin{tikzpicture}
\tkzDefPoint(0,.57){A}
\apeiroX{a}{2.3}{0}{.35}{\xrwma!30}
\axonas{0}{3}
\akro{k}{2.3}
\end{tikzpicture} & Μείον άπειρο $a$ κλειστό \\
$ (-\infty,a) $ & $ x<a $ & \begin{tikzpicture}
\tkzDefPoint(0,.57){A}
\apeiroX{a}{2.3}{0}{.35}{\xrwma!30}
\axonas{0}{3}
\akro{a}{2.3}
\end{tikzpicture} & Μείον άπειρο $a$ ανοιχτό \\
\hline 
\end{longtable}
\vspace{-2mm}
\captionof{table}{Διαστήματα αριθμών}
\end{center}
\thewrhmata
\Thewrhma{Ιδιότητεσ διάταξησ}\label{th:idan}
\vspace{-5mm}
\begin{enumerate}
\item Προσθέτοντας ή αφαιρώντας τον ίδιο αριθμό σε κάθε μέλος μιας ανισότητας η φορά της παραμένει ίδια.
\begin{gather*}
a>\beta\Leftrightarrow a+\gamma>\beta+\gamma\\
a>\beta\Leftrightarrow a-\gamma>\beta-\gamma
\end{gather*}
\item Εάν πολλαπλασιάσουμε ή διαιρέσουμε και τα δύο μέλη μιας ανισότητας με τον ίδιο
\begin{rlist}
\item \textbf{θετικό} αριθμό, τότε η φορά της παραμένει \textbf{ίδια}.
\item \textbf{αρνητικό} αριθμό, τότε η φορά της αλλάζει.
\end{rlist}
\begin{gather*}
\textrm{Αν }\gamma>0\textrm{ τότε }a>\beta\Leftrightarrow a\cdot\gamma>\beta\cdot\gamma\textrm{ και }\dfrac{a}{\gamma}>\dfrac{\beta}{\gamma}\\
\textrm{Αν }\gamma<0\textrm{ τότε }a>\beta\Leftrightarrow a\cdot\gamma<\beta\cdot\gamma\textrm{ και }\dfrac{a}{\gamma}<\dfrac{\beta}{\gamma}
\end{gather*}
\item Για να υψώσουμε κάθε μέλος μιας ανισότητας $ a>\beta $ με $ a,\beta\in\mathbb{R} $ σε έναν ακέραιο εκθέτη $ \nu\in\mathbb{Z} $ διακρίνουμε τις παρακάτω περιπτώσεις :
\begin{rlist}
\item Αν $ \nu>0 $ \textbf{άρτιος} εκθέτης και 
\begin{itemize}
\item $ a,\beta>0 $ τότε $ a>\beta\Leftrightarrow a^\nu>\beta^\nu \;\;${\footnotesize{ (Η φορά παραμένει ίδια.)}}
\item $ a,\beta<0 $ τότε $ a>\beta\Leftrightarrow a^\nu<\beta^\nu \;\;${\footnotesize{ (Η φορά αλλάζει.)}}
\item $ a,\beta<0 $ ετερόσημοι τότε δεν υψώνουμε.
\end{itemize}
\item Αν $ \nu>0 $ \textbf{περιττός} εκθέτης τότε $ a>\beta\Leftrightarrow a^\nu>\beta^\nu $
\end{rlist}
\item Εάν δύο θετικοί πραγματικοί αριθμοί $ a,\beta>0 $ είναι άνισοι τότε και οι ν-οστές ρίζες τους, $ \nu\in\mathbb{N} $, θα είναι με την ίδια φορά άνισες και αντίστροφα.
\begin{gather*}
a>\beta\Leftrightarrow\sqrt[\nu]{a}>\!\sqrt[\nu]{\beta}
\end{gather*}
\end{enumerate}
Τις περιπτώσεις όπου ο εκθέτης είναι αρνητικός θα τις δούμε αναλυτικά στο επόμενο θεώρημα. Ανάλογα συμπεράσματα ισχύουν και για τις ανισότητες $ a<\beta,a\geq\beta $ και $ a\leq\beta $.\\\\
\Thewrhma{Αντιστροφή μελών - Δύναμη με αρνητικό εκθέτη}
Εάν αντιστρέψουμε τα μέλη μιας ανισότητας, τότε προκύπτει ανισότητα με φορά
\begin{itemize}[itemsep=0mm]
\item αντίθετη της αρχικής αν τα μέλη της είναι ομόσημα
\item ίδια της αρχικής αν τα μέλη της είναι ετερόσημα.
\end{itemize}
\begin{gather*}
\textrm{Αν }a,\beta\textrm{ ομόσημοι τότε } a>\beta\Leftrightarrow \dfrac{1}{a}<\dfrac{1}{\beta}\\
\textrm{Αν }a,\beta\textrm{ ετερόσημοι τότε } a>\beta\Leftrightarrow \dfrac{1}{a}>\dfrac{1}{\beta}
\end{gather*}
Μπορούμε να γενικεύσουμε τον κανόνα αυτό για αρνητικό εκθέτη, συνεχίζοντας να διακρίνουμε τις περιπτώσεις του προηγούμενου θεωρήματος.
\begin{enumerate}[itemsep=0mm]
\item Αν $ \nu>0 $ \textbf{άρτιος} εκθέτης και 
\begin{enumerate}[itemsep=0mm,label=\roman*.]
\item $ a,\beta>0 $ τότε $ a>\beta\Leftrightarrow a^{-\nu}<\beta^{-\nu} \;\;${\footnotesize{ (Η φορά αλλάζει.)}}
\item $ a,\beta<0 $ τότε $ a>\beta\Leftrightarrow a^{-\nu}>\beta^{-\nu} \;\;${\footnotesize{ (Η φορά παραμένει ίδια.)}}
\item $ a,\beta $ ετερόσημοι τότε δεν υψώνουμε.
\end{enumerate}
\item Αν $ \nu>0 $ \textbf{περιττός} εκθέτης και
\begin{enumerate}[itemsep=0mm,label=\roman*.]
\item $ a,\beta $ ομόσημοι τότε $ a>\beta\Leftrightarrow a^{-\nu}<\beta^{-\nu} \;\;${\footnotesize{ (Η φορά αλλάζει.)}}
\item $ a,\beta $ ετερόσημοι τότε $ a>\beta\Leftrightarrow a^{-\nu}>\beta^{-\nu} \;\;${\footnotesize{ (Η φορά παραμένει ίδια.)}}
\end{enumerate}
\end{enumerate}
\Thewrhma{Πράξεισ κατά μέλη ανισοτήτων}
Μπορούμε να προσθέτουμε κατά μέλη δύο ανισότητες με ίδια φορά και να πολλαπλασιάσουμε κατά μέλη δύο ανισότητες ίδιας φοράς αρκεί όλοι οι όροι τους να είναι θετικοί.
\[ a>\beta\;\;\textrm{και}\;\;\gamma>\delta\Rightarrow\begin{cases}
\textrm{\textbf{{1. Πρόσθεση κατά μέλη }}}& a+\gamma>\beta+\delta\\\textrm{\textbf{{2. Πολλαπλασιασμός κατά μέλη }}}& a\cdot\gamma>\beta\cdot\delta\;\;,\;\;\textrm{με }a,\beta,\gamma,\delta>0
\end{cases} \]
\textbf{Δεν} μπορούμε να αφαιρέσουμε ή να διαιρέσουμε ανισότητες κατά μέλη.\\\\
\Thewrhma{Δύναμη με άρτιο εκθέτη}
Το τετράγωνο κάθε πραγματικού αριθμού $ a\in\mathbb{R} $ είναι μη αρνητικός αριθμός :
\[ a^2\geq0\ \ ,\ \ a^{2\kappa}\geq0\;\;,\;\;\kappa\in\mathbb{Z} \]
Η ιδιότητα ισχύει και για οποιοδήποτε άρτιο εκθέτη του αριθμού $ a $. Η ισότητα ισχύει όταν η βάση της δύναμης, είναι 0.\\\\
\Thewrhma{Άθροισμα δυνάμεων με άρτιο εκθέτη}
Το άθροισμα τετραγώνων οποιονδήποτε πραγματικών αριθμών $ a,\beta\in\mathbb{R} $ είναι μη αρνητικός αριθμός \[ a^2+\beta^2\geq0 \]
Η ιδιότητα αυτή γενικεύεται και για άθροισμα πολλών πραγματικών αριθμών υψωμένων σε οποιοδήποτε άρτιο εκθέτη.
\[ a_1^{2\kappa_1}+a_2^{2\kappa_2}+\ldots+a_\nu^{2\kappa_\nu}\geq0\;\;,\;\;\kappa_i\in\mathbb{Z}\;,\;i=1,2,\ldots,\nu \]
Η ισότητα ισχύει όταν οι βάσεις των δυνάμεων είναι μηδενικές.\\\\
\Lymena
\section{Απόλυτες τιμές}
\Thewria
\orismoi
\Orismos{απολυτη τιμη πραγματικου αριθμου}
Η απόλυτη τιμή ενός πραγματικού αριθμού $ a\in\mathbb{R} $ είναι η απόσταση του σημείου του αριθμού από την αρχή του άξονα των πραγματικών αριθμών. Συμβολίζεται με $ |a| $.
\begin{center}
\begin{tabular}{c >{\centering\arraybackslash}m{6cm}}
$ |a|=\begin{cases}
\begin{aligned}
a & \;,\;a\geq0\\
-a & \;,\;a<0
\end{aligned}
\end{cases} $  & \begin{tikzpicture}
\draw[-latex] (-1,0) -- coordinate (x axis mid) (4.4,0) node[right,fill=white] {{\footnotesize $ x $}};
\draw (0,.5mm) -- (0,-.5mm) node[anchor=north,fill=white] {{\footnotesize 0}};
\draw (3,.5mm) -- (3,-.5mm) node[anchor=north,fill=white] {{\footnotesize $ a $}};
\draw[line width=.7mm] (0,0) -- (3,0);
\tkzText(1.5,.34){$ \overcbrace{\rule{28mm}{0mm}}^{{\scriptsize |a|}} $}
\tkzDefPoint(3,0){A}
\tkzDrawPoint[size=7,fill=white](A)
\tkzDrawPoint[size=7,fill=white](0,0)
\tkzLabelPoint[above right](A){{\scriptsize $A(a)$}}
\tkzLabelPoint[above left](0,0){{\scriptsize $O(0)$}}
\end{tikzpicture}
\end{tabular} 
\end{center}
\begin{itemize}[itemsep=0mm]
\item Η απόλυτη τιμή ενός θετικού αριθμού $ a $ είναι ίση με τον ίδιο τον αριθμό ενώ η απόλυτη τιμή ενός αρνητικού αριθμού $ a $ είναι ίση με τον αντίθετο του αριθμού δηλαδή: $ -a $.
\item Η απόσταση δύο αριθμών μεταξύ τους ορίζεται ως η απόλυτη τιμή της διαφοράς τους.
\begin{center}
\begin{tabular}{c >{\centering\arraybackslash}m{6cm}}
$ |a-\beta|=d(a,\beta) $  & \begin{tikzpicture}
\draw[-latex] (-1,0) -- coordinate (x axis mid) (4.4,0) node[right,fill=white] {{\footnotesize $ x $}};
\draw (0,.5mm) -- (0,-.5mm) node[anchor=north,fill=white] {{\footnotesize 0}};
\draw (3,.5mm) -- (3,-.5mm) node[anchor=north,fill=white] {{\footnotesize $ a $}};
\draw[line width=.7mm] (0,0) -- (3,0);
\tkzText(1.5,.34){$ \overcbrace{\rule{28mm}{0mm}}^{{\scriptsize |a-\beta|=d(a,\beta)}} $}
\tkzDefPoint(3,0){A}
\tkzDrawPoint[size=7,fill=white](A)
\tkzDrawPoint[size=7,fill=white](0,0)
\tkzLabelPoint[above right](A){{\scriptsize $B(\beta)$}}
\tkzLabelPoint[above left](0,0){{\scriptsize $A(a)$}}
\end{tikzpicture}
\end{tabular} 
\end{center}
\end{itemize}
\thewrhmata
\Thewrhma{ιδιοτητεσ απολυτων τιμων}
Για οποιουσδήποτε πραγματικούς αριθμούς $ a,\beta\in\mathbb{R} $ ισχύουν οι παρακάτω ιδιότητες για τις απόλυτες τιμές τους:
\begin{center}
\begin{longtable}{cc}
\hline \rule[-2ex]{0pt}{5.5ex} \textbf{Ιδιότητα} & \textbf{Συνθήκη} \\
\hhline{==}\rule[-2ex]{0pt}{5.5ex} Απόλυτες τιμές αντίθετων & $ |a|=|-a|\geq0 $ \\
\rule[-2ex]{0pt}{5.5ex}  Απόλυτη τιμή μηδενός & $ |a|=0\Leftrightarrow a=0 $\\
\rule[-2ex]{0pt}{5.5ex}  Ανισότητα προσήμων & $ -|a|\leq a\leq|a| $ \\
\rule[-2ex]{0pt}{5.5ex}  Απόλυτη τιμή γινομένου & $ |a\cdot\beta|=|a|\cdot|\beta| $ \\
\rule[-2ex]{0pt}{5.5ex}  Απόλυτη τιμή πηλίκου & $ \left| \dfrac{a}{\beta}\right|=\dfrac{|a|}{|\beta|} $ \\
\rule[-2ex]{0pt}{5.5ex} Τετράγωνο απόλυτης τιμής & $ |a|^2=a^2 $ \\ 
\rule[-2ex]{0pt}{5.5ex} Τριγωνική ανισότητα & $ \left||a-\beta| \right|\leq|a\pm\beta|\leq|a|+|\beta|  $ \\
\hline
\end{longtable}\vspace{-4mm}\captionof{table}{Ιδιότητες απόλυτης τιμής}
\end{center}
\Lymena
\section{Ρίζες}
\Thewria
\orismoi
\Orismos{Τετραγωνική Ρίζα}
Τετραγωνική ρίζα ενός \textbf{μη αρνητικού} πραγματικού αριθμού $ x $ ονομάζεται ο \textbf{μη αρνητικός} αριθμός $ a $ ο οποίος αν υψωθεί στο τετράγωνο δίνει τον αριθμό $ x $. Συμβολίζεται με $ \sqrt{x} $.
\[ \sqrt{x}=a\;\;,\;\;\textrm{ όπου }x\geq0\textrm{ και }a\geq0 \]
\begin{itemize}[itemsep=0mm]
\item Ο αριθμός $ x $ ονομάζεται \textbf{υπόριζο}.
\item Δεν ορίζεται ρίζα αρνητικού αριθμού.
\end{itemize}
\Orismos{ριζα \MakeLowercase{ν}-ταξησ πραγματικου αριθμού}
Ρίζα $ \nu $-οστής τάξης ενός \textbf{μη αρνητικού} αριθμού $ x $ ονομάζεται ο \textbf{μη αρνητικός} αριθμός $ a $ που αν υψωθεί στη δύναμη $ \nu $ δίνει αποτέλεσμα $ x $ (υπόριζο). Συμβολίζεται με $ \sqrt[\nu]{x} $.
\[ \sqrt[\nu]{x}=a\;\;,\;\;\textrm{ όπου }x\geq0\textrm{ και }a\geq0 \]
\Orismos{Δυναμη με ρητό εκθετη}
Η δύναμη ενός \textbf{θετικού} αριθμού $ a $ με εκθέτη ένα ρητό αριθμό $ \frac{\mu}{\nu} $, όπου $ \mu\in\mathbb{Z} $ και $ \nu\in\mathbb{N}^* $, ορίζεται ως η ρίζα $ \nu $-τάξης του αριθμού $ a $ υψωμένο στη δύναμη $ \mu $.
\[ a^{\frac{\mu}{\nu}}=\!\sqrt[\nu]{a^\mu}\ ,\ \textrm{ όπου } a>0 \]

\thewrhmata
\Thewrhma{Ιδιότητεσ Ριζών}
Για κάθε $ x,y\in\mathbb{R} $ πραγματικούς αριθμούς και $ \nu,\mu,\rho\in\mathbb{Ν} $ φυσικούς αριθμούς ισχύουν οι παρακάτω ιδιότητες για την τετραγωνική και ν-οστή ρίζα τους.
\begin{center}
\begin{longtable}{cc}
\hline \rule[-2ex]{0pt}{5.5ex} \textbf{Ιδιότητα} & \textbf{Συνθήκη} \\
\hhline{==}\rule[-2ex]{0pt}{5.5ex}  Τετράγωνο ρίζας & $ \left(\!\sqrt{x}\;\right)^2=x\;\;,\;\; x\geq0  $ \\
\rule[-2ex]{0pt}{5.5ex}  Ν-οστή δύναμη ν-οστής ρίζας & $ \left(\!\sqrt[\nu]{x}\;\right)^\nu=x\;\;,\;\; x\geq0  $ \\
\rule[-2ex]{0pt}{5.5ex}  Ρίζα τετραγώνου & $ \sqrt{x^2}=|x|\;\;,\;\; x\in\mathbb{R} $\\
\rule[-2ex]{0pt}{5.5ex}  Ν-οστή ρίζα ν-οστής δύναμης & $ \sqrt[\nu]{x^\nu}=\begin{cases}
|x|&  x\in\mathbb{R}\textrm{ αν }\nu\textrm{ άρτιος}\\ \ x&  x\geq0\textrm{ και } \nu\in\mathbb{N}\end{cases} $\\
\hhline{~-} \multirow{3}{*}{Ρίζα γινομένου} & $ \sqrt{x\cdot y}=\!\sqrt{x}\cdot\!\sqrt{y}\;\;,\;\; x,y\geq0 $ \rule[-2ex]{0pt}{5.5ex}\\
\rule[-2ex]{0pt}{5.5ex} & $ \sqrt[\nu]{x\cdot y}=\!\sqrt[\nu]{x}\cdot\!\sqrt[\nu]{y}\;\;,\;\; x,y\geq0 $ \\
\hhline{~-} \multirow{3}{*}{Ρίζα πηλίκου} & $ \sqrt{\dfrac{x}{y}}\;=\dfrac{\sqrt{x}}{\sqrt{y}}\;\;,\;\; x\geq0\textrm{ και }y>0 $ \rule[-2ex]{0pt}{6.5ex}\\
\rule[-2ex]{0pt}{7.5ex} & $ \sqrt[\nu]{\dfrac{x}{y}}\;=\dfrac{\sqrt[\nu]{x}}{\sqrt[\nu]{y}}\;\;,\;\; x\geq0\textrm{ και }y>0 $ \\
\hhline{~-}\rule[-2ex]{0pt}{5.5ex}  Μ-οστή ρίζα ν-οστής ρίζας  & $ \sqrt[\mu]{\!\sqrt[\nu]{x}}=\!\sqrt[\nu\cdot\mu]{x}\;\;,\;\; x\geq0 $ \\
\rule[-2ex]{0pt}{5.5ex}  Απλοποίηση ρίζας & $ \sqrt[\nu]{x^\nu\cdot y}=x\!\sqrt[\nu]{y}\;\;,\;\; x,y\geq0  $ \\
\rule[-2ex]{0pt}{5.5ex} Απλοποίηση τάξης και δύναμης & $ \sqrt[\mu\cdot\rho]{x^{\nu\cdot\rho}}=\!\sqrt[\mu]{x^{\nu}}\;\;,\;\; x\geq0 $ \\
\hline
\end{longtable}\vspace{-4mm}\captionof{table}{Ιδιότητες ριζών}
\end{center}
\vspace{-7mm}
\begin{itemize}[itemsep=0mm]
\item Η ιδιότητα 5 ισχύει και για γινόμενο περισσότερων των δύο παραγόντων. \[ \sqrt[\nu]{x_1\cdot x_2\cdot\ldots\cdot x_\nu}=\!\sqrt[\nu]{x_1}\cdot\!\sqrt[\nu]{x_2}\cdot\ldots\cdot\!\sqrt[\nu]{x_\nu} \] όπου $ x_1,x_2,\ldots x_\nu\geq0 $ και $ \nu\in\mathbb{N} $.
\item Η ιδιότητα 7 ισχύει και για παραστάσεις που περιέχουν πολλές ρίζες διαφόρων τάξεων στις οποίες η μια ρίζα βρίσκεται μέσα στην άλλη. \[ \sqrt[\mu_1]{\!\sqrt[\mu_2]{\mbox{}^{\ddots}\sqrt[\mu_{\nu}]{x}}}\;\;\;=\sqrt[\mu_1\cdot\mu_2\cdot\ldots\cdot\mu_\nu]{x} \] με $ x\geq0 $ και $ \mu_1,\mu_2,\ldots,\mu_\nu\in\mathbb{N} $.
\end{itemize}
\Lymena
\begin{Methodos}[Παραγοντοποίηση]{4cm}

\end{Methodos}

%\PassOptionsToPackage{no-math,cm-default}{fontspec}
\documentclass[twoside,nofonts,internet]{askhseis}
\usepackage{amsmath}
\usepackage{xgreek}
\let\hbar\relax
\defaultfontfeatures{Mapping=tex-text,Scale=MatchLowercase}
\setmainfont[Mapping=tex-text,Numbers=Lining,Scale=1.0,BoldFont={Minion Pro Bold}]{Minion Pro}
\newfontfamily\scfont{GFS Artemisia}
\font\icon = "Webdings"
\usepackage[amsbb,subscriptcorrection,zswash,mtpcal,mtphrb]{mtpro2}
\xroma{red!70!black}
%------TIKZ - ΣΧΗΜΑΤΑ - ΓΡΑΦΙΚΕΣ ΠΑΡΑΣΤΑΣΕΙΣ ----
\usepackage{tikz}
\usepackage{tkz-euclide}
\usetkzobj{all}
\usepackage[framemethod=TikZ]{mdframed}
\usetikzlibrary{decorations.pathreplacing}
\usepackage{pgfplots}
\usetkzobj{all}
%-----------------------
\usepackage{calc}
\usepackage{hhline}
\usepackage[explicit]{titlesec}
\usepackage{graphicx}
\usepackage{multicol}
\usepackage{multirow}
\usepackage{enumitem}
\usepackage{tabularx}
\usepackage[decimalsymbol=comma]{siunitx}
\usetikzlibrary{backgrounds}
\usepackage{sectsty}
\sectionfont{\centering}
\usepackage{enumitem}
\setlist[enumerate]{label=\bf{\large \arabic*.}}
\usepackage{adjustbox}
\usepackage{mathimatika,gensymb,eurosym,wrap-rl}
\usepackage{systeme,regexpatch}
%-------- ΜΑΘΗΜΑΤΙΚΑ ΕΡΓΑΛΕΙΑ ---------
\usepackage{mathtools}
%----------------------
%-------- ΠΙΝΑΚΕΣ ---------
\usepackage{booktabs}
%----------------------
%----- ΥΠΟΛΟΓΙΣΤΗΣ ----------
\usepackage{calculator}
%----------------------------
%------ ΔΙΑΓΩΝΙΟ ΣΕ ΠΙΝΑΚΑ -------
\usepackage{array}
\newcommand\diag[5]{%
\multicolumn{1}{|m{#2}|}{\hskip-\tabcolsep
$\vcenter{\begin{tikzpicture}[baseline=0,anchor=south west,outer sep=0]
\path[use as bounding box] (0,0) rectangle (#2+2\tabcolsep,\baselineskip);
\node[minimum width={#2+2\tabcolsep-\pgflinewidth},
minimum  height=\baselineskip+#3-\pgflinewidth] (box) {};
\draw[line cap=round] (box.north west) -- (box.south east);
\node[anchor=south west,align=left,inner sep=#1] at (box.south west) {#4};
\node[anchor=north east,align=right,inner sep=#1] at (box.north east) {#5};
\end{tikzpicture}}\rule{0pt}{.71\baselineskip+#3-\pgflinewidth}$\hskip-\tabcolsep}}
%---------------------------------
%---- ΟΡΙΖΟΝΤΙΟ - ΚΑΤΑΚΟΡΥΦΟ - ΠΛΑΓΙΟ ΑΓΚΙΣΤΡΟ ------
\newcommand{\orag}[3]{\node at (#1)
{$ \overcbrace{\rule{#2mm}{0mm}}^{{\scriptsize #3}} $};}
\newcommand{\kag}[3]{\node at (#1)
{$ \undercbrace{\rule{#2mm}{0mm}}_{{\scriptsize #3}} $};}
\newcommand{\Pag}[4]{\node[rotate=#1] at (#2)
{$ \overcbrace{\rule{#3mm}{0mm}}^{{\rotatebox{-#1}{\scriptsize$#4$}}}$};}
%-----------------------------------------


%------------------------------------------
\newcommand{\tss}[1]{\textsuperscript{#1}}
\newcommand{\tssL}[1]{\MakeLowercase{\textsuperscript{#1}}}
%---------- ΛΙΣΤΕΣ ----------------------
\newlist{brlist}{enumerate}{3}
\setlist[brlist]{itemsep=0mm,label=\bf\roman*.}
\newlist{tropos}{enumerate}{3}
\setlist[tropos]{label=\bf\textit{\arabic*\textsuperscript{oς}\;Τρόπος :},leftmargin=0cm,itemindent=2.3cm,ref=\bf{\arabic*\textsuperscript{oς}\;Τρόπος}}
% Αν μπει το bhma μεσα σε tropo τότε
%\begin{bhma}[leftmargin=.7cm]
\tkzSetUpPoint[size=7,fill=white]
\tikzstyle{pl}=[line width=0.3mm]
\tikzstyle{plm}=[line width=0.4mm]

\begin{document}
\titlos{Μαθηματικά Β΄ Γυμνασίου}{Εξισώσεις - Ανισώσεις}{Εξισώσεις}
\thewria
\begin{enumerate}
\item 
\end{enumerate}
\twocolkentro{\askhseis}
\begin{enumerate}
\item Να λυθούν οι παρακάτω εξισώσεις
\begin{rlist}
\begin{multicols}{2}
\item $ 2x-1=3 $
\item $ 4-3x=1 $
\item $ 5x-4=x $
\item $ 2x-3=-x $
\end{multicols}
\end{rlist}
\item Να λυθούν οι παρακάτω εξισώσεις
\begin{rlist}[leftmargin=4mm]
\begin{multicols}{2}
\item $ 2x-3=x+7 $
\item $ 3x+7=x-5 $
\item $ 7+x=4x-8 $
\item $ 3+4x+5=2x+4 $
\end{multicols}
\end{rlist}
\item Να λυθούν οι παρακάτω εξισώσεις
\begin{rlist}
\item $ 2x-1=4+2x $
\item $ 7-3x=-3x+7 $
\item $ 5x-3=x-3+4x $
\item $ 2x+1-x=x-3+4 $
\end{rlist}
\item Να λυθούν οι παρακάτω εξισώσεις
\begin{rlist}
\item $ 2(x-1)=4 $
\item $ 1-3(1-x)=4 $
\item $ 3(2x-1)=2(1-x) $
\item $ 5(1-x)+7=6-(x+2) $
\end{rlist}
\item Να λυθούν οι παρακάτω εξισώσεις
\begin{rlist}
\item $ 2(x-1)=3(2-x)+7 $
\item $ 4(x-3)-1=3-(3x+2) $
\item $ 5-2(x+3)=7(x-2)+4 $
\item $ 3(2x-5)-(4-x)=3(x+2) $
\end{rlist}
\item Να λυθούν οι παρακάτω εξισώσεις
\begin{rlist}
\item $ 3(x-2)+4=3x-2 $
\item $ 4x-(5+x)=2(x-3)+x $
\item $ 2(4-x)+3(3+2x)=4x-1 $
\item $ 3(1-3x)-(2-x)=4(1-2x)+3 $
\end{rlist}
\item Να λυθούν οι παρακάτω εξισώσεις
\begin{multicols}{2}
\begin{rlist}[leftmargin=4mm]
\item $ \dfrac{x-1}{2}=\dfrac{2x-1}{3} $
\item $ \dfrac{3x-1}{5}=\dfrac{4-x}{2} $
\item $ \dfrac{2x-3}{3}=\dfrac{7}{5} $
\item $ \dfrac{2x-4}{2}=5x $
\end{rlist}
\end{multicols}
\item Να λυθούν οι παρακάτω εξισώσεις
\begin{rlist}
\item $ \dfrac{x-5}{2}+\dfrac{2x-4}{3}=2 $
\item $ \dfrac{3x-8}{4}-\dfrac{1}{2}=\dfrac{7x+8}{10}-\dfrac{x}{2} $
\item $ \dfrac{x+1}{3}=\dfrac{2x-9}{4}+\dfrac{1}{12} $
\item $ \dfrac{1}{4}(x+3)-\dfrac{1}{5}(2x-1)=2+\dfrac{1}{10}x $
\end{rlist}
\item Να λυθούν οι παρακάτω εξισώσεις
\begin{rlist}
\item $ \dfrac{2x-3}{2}-\dfrac{3x+1}{4}=\dfrac{x-3}{4}-1 $
\item $ \dfrac{x-1}{4}+\dfrac{2-x}{3}=1-x $
\item $ \dfrac{2(3-x)}{5}+x=\dfrac{4(x-3)}{7}+\dfrac{x}{35} $
\end{rlist}
\item Να λυθούν οι παρακάτω εξισώσεις
\begin{multicols}{2}
\begin{rlist}
\item $ \dfrac{5+\frac{x-2}{3}}{3}=3 $
\item $ \dfrac{\frac{x-1}{2}+\frac{1}{5}}{4}=\frac{1}{10} $
\end{rlist}
\end{multicols}
\item Δίνεται η παραμετρική εξίσωση \[ (3\lambda-1)x-\lambda x+5=5\lambda x-12 \]
όπου $ \lambda $ είναι γνωστός αριθμός και $ x $ ο άγνωστος. Να βρεθεί η τιμή που πρέπει να έχει το $ \lambda $ ώστε η εξίσωση να έχει λύση το $ x=1 $.
\item Να βρεθεί η τιμή του $ \mu $ ώστε η εξίσωση \[ \dfrac{\mu-1}{2}x+\dfrac{1}{3}=\dfrac{x+1}{3} \] να είναι ταυτότητα. (Να είναι δηλαδή της μορφής $ 0x=0 $).
\item Δίνεται η εξίσωση \[ (\lambda+2)x-(x-1)\lambda=x+\lambda\lambda+1 \]
\begin{enumerate}[label=\roman*.,itemsep=0mm]
\item Αν $ \lambda=3 $ να αποδειχθεί ότι η εξίσωση έχει λύση $ x=1 $.
\item Να λυθεί η εξίσωση για $ \lambda=1 $.
\end{enumerate}
\item Να βρεθεί ο αριθμός $ x $ έτσι ώστε το τρίγωνο $AB\varGamma$ του διπλανού σχήματος να είναι ισοσκελές με 
\begin{enumerate}[itemsep=0mm,label=\roman*.]
\item βάση την πλευρά $B\varGamma$.
\item βάση την πλευρά $AB$.
\end{enumerate}
Να αποδειχθεί επίσης ότι δεν υπάρχει τιμή του $x$ ώστε το τρίγωνο να είναι ισοσκελές με βάση την πλευρά $B\varGamma$.
\end{enumerate}
\end{document}


%
%\part{Γεωμετρία}
%\backmatter
%\pagestyle{empty}
%\listoffigures
%\listoftables

\part{Διαγώνισμα}
\pagestyle{fancy}
\PassOptionsToPackage{no-math,cm-default}{fontspec}
\documentclass[twoside,nofonts,internet]{askhseis}
\usepackage{amsmath}
\usepackage{xgreek}
\let\hbar\relax
\defaultfontfeatures{Mapping=tex-text,Scale=MatchLowercase}
\setmainfont[Mapping=tex-text,Numbers=Lining,Scale=1.0,BoldFont={Minion Pro Bold}]{Minion Pro}
\newfontfamily\scfont{GFS Artemisia}
\font\icon = "Webdings"
\usepackage[amsbb,subscriptcorrection,zswash,mtpcal,mtphrb]{mtpro2}
\xroma{red!70!black}
%------TIKZ - ΣΧΗΜΑΤΑ - ΓΡΑΦΙΚΕΣ ΠΑΡΑΣΤΑΣΕΙΣ ----
\usepackage{tikz}
\usepackage{tkz-euclide}
\usetkzobj{all}
\usepackage[framemethod=TikZ]{mdframed}
\usetikzlibrary{decorations.pathreplacing}
\usepackage{pgfplots}
\usetkzobj{all}
%-----------------------
\usepackage{calc}
\usepackage{hhline}
\usepackage[explicit]{titlesec}
\usepackage{graphicx}
\usepackage{multicol}
\usepackage{multirow}
\usepackage{enumitem}
\usepackage{tabularx}
\usepackage[decimalsymbol=comma]{siunitx}
\usetikzlibrary{backgrounds}
\usepackage{sectsty}
\sectionfont{\centering}
\usepackage{enumitem}
\setlist[enumerate]{label=\bf{\large \arabic*.}}
\usepackage{adjustbox}
\usepackage{mathimatika,gensymb,eurosym,wrap-rl}
\usepackage{systeme,regexpatch}
%-------- ΜΑΘΗΜΑΤΙΚΑ ΕΡΓΑΛΕΙΑ ---------
\usepackage{mathtools}
%----------------------
%-------- ΠΙΝΑΚΕΣ ---------
\usepackage{booktabs}
%----------------------
%----- ΥΠΟΛΟΓΙΣΤΗΣ ----------
\usepackage{calculator}
%----------------------------
%------ ΔΙΑΓΩΝΙΟ ΣΕ ΠΙΝΑΚΑ -------
\usepackage{array}
\newcommand\diag[5]{%
\multicolumn{1}{|m{#2}|}{\hskip-\tabcolsep
$\vcenter{\begin{tikzpicture}[baseline=0,anchor=south west,outer sep=0]
\path[use as bounding box] (0,0) rectangle (#2+2\tabcolsep,\baselineskip);
\node[minimum width={#2+2\tabcolsep-\pgflinewidth},
minimum  height=\baselineskip+#3-\pgflinewidth] (box) {};
\draw[line cap=round] (box.north west) -- (box.south east);
\node[anchor=south west,align=left,inner sep=#1] at (box.south west) {#4};
\node[anchor=north east,align=right,inner sep=#1] at (box.north east) {#5};
\end{tikzpicture}}\rule{0pt}{.71\baselineskip+#3-\pgflinewidth}$\hskip-\tabcolsep}}
%---------------------------------
%---- ΟΡΙΖΟΝΤΙΟ - ΚΑΤΑΚΟΡΥΦΟ - ΠΛΑΓΙΟ ΑΓΚΙΣΤΡΟ ------
\newcommand{\orag}[3]{\node at (#1)
{$ \overcbrace{\rule{#2mm}{0mm}}^{{\scriptsize #3}} $};}
\newcommand{\kag}[3]{\node at (#1)
{$ \undercbrace{\rule{#2mm}{0mm}}_{{\scriptsize #3}} $};}
\newcommand{\Pag}[4]{\node[rotate=#1] at (#2)
{$ \overcbrace{\rule{#3mm}{0mm}}^{{\rotatebox{-#1}{\scriptsize$#4$}}}$};}
%-----------------------------------------


%------------------------------------------
\newcommand{\tss}[1]{\textsuperscript{#1}}
\newcommand{\tssL}[1]{\MakeLowercase{\textsuperscript{#1}}}
%---------- ΛΙΣΤΕΣ ----------------------
\newlist{brlist}{enumerate}{3}
\setlist[brlist]{itemsep=0mm,label=\bf\roman*.}
\newlist{tropos}{enumerate}{3}
\setlist[tropos]{label=\bf\textit{\arabic*\textsuperscript{oς}\;Τρόπος :},leftmargin=0cm,itemindent=2.3cm,ref=\bf{\arabic*\textsuperscript{oς}\;Τρόπος}}
% Αν μπει το bhma μεσα σε tropo τότε
%\begin{bhma}[leftmargin=.7cm]
\tkzSetUpPoint[size=7,fill=white]
\tikzstyle{pl}=[line width=0.3mm]
\tikzstyle{plm}=[line width=0.4mm]

\begin{document}
\titlos{Μαθηματικά Β΄ Γυμνασίου}{Εξισώσεις - Ανισώσεις}{Εξισώσεις}
\thewria
\begin{enumerate}
\item 
\end{enumerate}
\twocolkentro{\askhseis}
\begin{enumerate}
\item Να λυθούν οι παρακάτω εξισώσεις
\begin{rlist}
\begin{multicols}{2}
\item $ 2x-1=3 $
\item $ 4-3x=1 $
\item $ 5x-4=x $
\item $ 2x-3=-x $
\end{multicols}
\end{rlist}
\item Να λυθούν οι παρακάτω εξισώσεις
\begin{rlist}[leftmargin=4mm]
\begin{multicols}{2}
\item $ 2x-3=x+7 $
\item $ 3x+7=x-5 $
\item $ 7+x=4x-8 $
\item $ 3+4x+5=2x+4 $
\end{multicols}
\end{rlist}
\item Να λυθούν οι παρακάτω εξισώσεις
\begin{rlist}
\item $ 2x-1=4+2x $
\item $ 7-3x=-3x+7 $
\item $ 5x-3=x-3+4x $
\item $ 2x+1-x=x-3+4 $
\end{rlist}
\item Να λυθούν οι παρακάτω εξισώσεις
\begin{rlist}
\item $ 2(x-1)=4 $
\item $ 1-3(1-x)=4 $
\item $ 3(2x-1)=2(1-x) $
\item $ 5(1-x)+7=6-(x+2) $
\end{rlist}
\item Να λυθούν οι παρακάτω εξισώσεις
\begin{rlist}
\item $ 2(x-1)=3(2-x)+7 $
\item $ 4(x-3)-1=3-(3x+2) $
\item $ 5-2(x+3)=7(x-2)+4 $
\item $ 3(2x-5)-(4-x)=3(x+2) $
\end{rlist}
\item Να λυθούν οι παρακάτω εξισώσεις
\begin{rlist}
\item $ 3(x-2)+4=3x-2 $
\item $ 4x-(5+x)=2(x-3)+x $
\item $ 2(4-x)+3(3+2x)=4x-1 $
\item $ 3(1-3x)-(2-x)=4(1-2x)+3 $
\end{rlist}
\item Να λυθούν οι παρακάτω εξισώσεις
\begin{multicols}{2}
\begin{rlist}[leftmargin=4mm]
\item $ \dfrac{x-1}{2}=\dfrac{2x-1}{3} $
\item $ \dfrac{3x-1}{5}=\dfrac{4-x}{2} $
\item $ \dfrac{2x-3}{3}=\dfrac{7}{5} $
\item $ \dfrac{2x-4}{2}=5x $
\end{rlist}
\end{multicols}
\item Να λυθούν οι παρακάτω εξισώσεις
\begin{rlist}
\item $ \dfrac{x-5}{2}+\dfrac{2x-4}{3}=2 $
\item $ \dfrac{3x-8}{4}-\dfrac{1}{2}=\dfrac{7x+8}{10}-\dfrac{x}{2} $
\item $ \dfrac{x+1}{3}=\dfrac{2x-9}{4}+\dfrac{1}{12} $
\item $ \dfrac{1}{4}(x+3)-\dfrac{1}{5}(2x-1)=2+\dfrac{1}{10}x $
\end{rlist}
\item Να λυθούν οι παρακάτω εξισώσεις
\begin{rlist}
\item $ \dfrac{2x-3}{2}-\dfrac{3x+1}{4}=\dfrac{x-3}{4}-1 $
\item $ \dfrac{x-1}{4}+\dfrac{2-x}{3}=1-x $
\item $ \dfrac{2(3-x)}{5}+x=\dfrac{4(x-3)}{7}+\dfrac{x}{35} $
\end{rlist}
\item Να λυθούν οι παρακάτω εξισώσεις
\begin{multicols}{2}
\begin{rlist}
\item $ \dfrac{5+\frac{x-2}{3}}{3}=3 $
\item $ \dfrac{\frac{x-1}{2}+\frac{1}{5}}{4}=\frac{1}{10} $
\end{rlist}
\end{multicols}
\item Δίνεται η παραμετρική εξίσωση \[ (3\lambda-1)x-\lambda x+5=5\lambda x-12 \]
όπου $ \lambda $ είναι γνωστός αριθμός και $ x $ ο άγνωστος. Να βρεθεί η τιμή που πρέπει να έχει το $ \lambda $ ώστε η εξίσωση να έχει λύση το $ x=1 $.
\item Να βρεθεί η τιμή του $ \mu $ ώστε η εξίσωση \[ \dfrac{\mu-1}{2}x+\dfrac{1}{3}=\dfrac{x+1}{3} \] να είναι ταυτότητα. (Να είναι δηλαδή της μορφής $ 0x=0 $).
\item Δίνεται η εξίσωση \[ (\lambda+2)x-(x-1)\lambda=x+\lambda\lambda+1 \]
\begin{enumerate}[label=\roman*.,itemsep=0mm]
\item Αν $ \lambda=3 $ να αποδειχθεί ότι η εξίσωση έχει λύση $ x=1 $.
\item Να λυθεί η εξίσωση για $ \lambda=1 $.
\end{enumerate}
\item Να βρεθεί ο αριθμός $ x $ έτσι ώστε το τρίγωνο $AB\varGamma$ του διπλανού σχήματος να είναι ισοσκελές με 
\begin{enumerate}[itemsep=0mm,label=\roman*.]
\item βάση την πλευρά $B\varGamma$.
\item βάση την πλευρά $AB$.
\end{enumerate}
Να αποδειχθεί επίσης ότι δεν υπάρχει τιμή του $x$ ώστε το τρίγωνο να είναι ισοσκελές με βάση την πλευρά $B\varGamma$.
\end{enumerate}
\end{document}


\end{document}







