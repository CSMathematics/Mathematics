\documentclass[11pt,a4paper,twoside]{book}
\usepackage[english,greek]{babel}
\usepackage[utf8]{inputenc}
%------------------ Γραμματοσειρές -------------------------------------
\usepackage{nimbusserif}% Βασικό font : 
\usepackage[T1]{fontenc}
\newcommand{\titlefont}[1]{{\fontfamily{maksf}\selectfont #1}}
%-----------------------------------------------------------------------

%------------------- Διάφορα ---------------------------------------------
\usepackage{gensymb,fontawesome5,eurosym,titletoc,sidenotes,multicol,adjmulticol}
\usepackage[framemethod=tikz]{mdframed}
\usepackage[explicit]{titlesec}
%--------------------------------------------------------------------------

%------------------- Γεωμετρία - Στύλ ----------------------------------
\usepackage[inner=1.5cm, top=3cm, bottom=2cm]{geometry}
\geometry{textwidth=12cm,marginparsep=5mm,marginparwidth=5.5cm}
\newcommand{\full}[1]{\begin{mdframed}[outermargin=\dimexpr-\marginparwidth-\marginparsep\relax,innerleftmargin=0mm,innerrightmargin=0mm,hidealllines=true]
#1
\end{mdframed}}
\newcommand{\fulltwoc}[1]{\begin{adjmulticols}{2}{0cm}{\dimexpr-\marginparwidth-\marginparsep\relax}
#1
\end{adjmulticols}}
%-----------------------------------------------------------------------

%---------------- Μαθηματικά -----------------------------------------
\usepackage{amsmath}
\let\myBbbk\Bbbk
\let\Bbbk\relax
\usepackage[amsbb,subscriptcorrection,zswash,mtpcal,mtphrb,mtpfrak]{mtpro2}
\usepackage{mathimatika,venndiagram,tkz-euclide,diffcoeff,mathtools}

\DeclareMathSizes{10.95}{10.95}{7}{5}
\DeclareMathSizes{6}{6}{3.8}{2.7}
\DeclareMathSizes{8}{8}{5.1}{3.6}
\DeclareMathSizes{9}{9}{5.8}{4.1}
\DeclareMathSizes{10}{10}{6.4}{4.5}
\DeclareMathSizes{12}{12}{7.7}{5.5}
\DeclareMathSizes{14.4}{14.4}{9.2}{6.5}
\DeclareMathSizes{17.28}{17.28}{11}{7.9}
\DeclareMathSizes{20.74}{20.74}{13.3}{9.4}
\DeclareMathSizes{24.88}{24.88}{16}{11.3}

\makeatletter
\AtBeginDocument{
\check@mathfonts
\fontdimen16\textfont2=2.5pt
\fontdimen17\textfont2=2.5pt
\fontdimen14\textfont2=4.5pt
\fontdimen13\textfont2=4.5pt}
\makeatother
%----------------------------------------------------------------------

%----------------- Γραφικά και σχήματα --------------------------------
\usepackage{graphicx,tikz,pgfplots,tkz-euclide}
\usetikzlibrary{shadows,calc,fadings}
\tikzset{>=latex}
\tikzstyle{l3}=[line width=0.3mm]
\tikzstyle{l4}=[line width=0.4mm]
\tikzset{labelbox/.style args={#1}{align=center,draw=#1,fill=#1!30!white,rectangle,rounded corners=1,font=\tiny\linespread{0.8}\selectfont}}
\tkzSetUpPoint[size=2.8,fill=white]
\pgfplotsset{compat=newest}
%----------------------------------------------------------------------

%------------------- Χρώματα -------------------------------------
\usepackage[usenames,dvipsnames,table,x11names]{xcolor}
\definecolor{maincolor}{RGB}{35,150,210}
\definecolor{secondarycolor}{RGB}{210,20,210}
\definecolor{thirdcolor}{RGB}{110,195,7}
%-----------------------------------------------------------------

%------------------ Κεφαλίδα και υποσέλιδο ----------------------------
\usepackage{fancyhdr,lipsum}
\pagestyle{fancy}
\fancyheadoffset[LE]{\marginparwidth+\marginparsep}
\fancyheadoffset[RO]{\marginparwidth+\marginparsep}
\renewcommand{\headrulewidth}{\iftopfloat{0pt}{.5pt}}
\renewcommand{\chaptermark}[1]{\markboth{#1}{}}
\renewcommand{\sectionmark}[1]{\markright{\it\thesection\ #1}}
\fancyhf{}
\fancyhead[LE]{\thepage\ $\cdot$\ \nouppercase{\leftmark}}
\fancyhead[RO]{\nouppercase{\rightmark} $\cdot$\ \thepage}
\fancypagestyle{plain}{%
\fancyhead{} %
\renewcommand{\headrulewidth}{0pt}}
%------------------------------------------------------------------------

%------------------- Ορισμοί - Θεωρήματα - Διάφορα πλαίσια -----------------
\usepackage[most]{tcolorbox}
\tcbuselibrary{skins,theorems,breakable}
% ΟΡΙΣΜΟΣ
\newcounter{orismos}[chapter]
\renewcommand{\theorismos}{\thechapter.\arabic{orismos}}   
\newcommand{\orism}{\refstepcounter{orismos}{\bf\titlefont{\textcolor{maincolor}{\large{Ορισμός}\hspace{2mm}\theorismos}}}\hspace{1mm}}{}

\newenvironment{Orismos}[1]
{\begin{tcolorbox}[title=\orism:\ \  {\bf{\large\titlefont{#1}}},
breakable,
enhanced standard,
titlerule=-.2pt,
toprule=0pt, 
rightrule=0pt, 
bottomrule=0pt,
colback=white,
opacityfill=0,
left=2mm,
top=1mm,
bottom=0mm,
boxrule=0pt,
colframe=white,
borderline west={1.5mm}{0pt}{maincolor},
leftrule=2mm,
sharp corners,
coltitle=black]}
{\end{tcolorbox}}

% ΘΕΩΡΗΜΑ
\newcounter{thewrhma}[chapter]
\renewcommand{\thethewrhma}{\thechapter.\arabic{thewrhma}} 
\newcommand{\thewr}{\refstepcounter{thewrhma}{\bf\titlefont{\textcolor{secondarycolor}{\large Θεώρημα\hspace{2mm}\thethewrhma}}}\hspace{1mm}}{}

\newenvironment{Thewrhma}[1]
{\begin{tcolorbox}[title=\thewr\ \ :\ \  {\textcolor{black}{\bf{\large\titlefont{#1}}}},
breakable,
enhanced standard,
titlerule=-.2pt,
toprule=0pt, 
rightrule=0pt, 
bottomrule=0pt,
colback=white,
left=2mm,
top=1mm,
bottom=0mm,
boxrule=0pt,
colframe=white,
borderline west={1.5mm}{0pt}{secondarycolor},
leftrule=2mm,
sharp corners,
coltitle=secondarycolor]}
{\end{tcolorbox}}

% ΠΑΡΑΤΗΡΗΣΗ
\newenvironment{Parathrhsh}[1]
{\begin{tcolorbox}[title=\textbf{\faInfoCircle\ \ \ \titlefont{{\large Παρατήρηση}}},
breakable,
enhanced standard,
lifted shadow={1mm}{-2mm}{3mm}{0.3mm}{black!50!white},
colback=yellow!5!white,
boxrule=0.1pt,
colframe=yellow!80!black,
hyphenationfix=true,
fonttitle=\bfseries,
toptitle=1mm,
bottomtitle=1mm,
width=#1]}
{\end{tcolorbox}}

% ΣΗΜΕΙΩΣΗ
\newenvironment{Shmeiwsh}[1]
{\begin{tcolorbox}[title=\textbf{\faBook\ \ \ \titlefont{{\large Σημείωση}}},
breakable,
enhanced standard,
lifted shadow={1mm}{-2mm}{3mm}{0.3mm}{black!50!white},
colback=thirdcolor!5!white,
boxrule=0.1pt,
colframe=thirdcolor!80!black,
hyphenationfix=true,
fonttitle=\bfseries,
toptitle=1mm,
bottomtitle=1mm,
width=#1]}
{\end{tcolorbox}}

% ΠΡΟΣΟΧΗ
\newenvironment{Prosoxi}[1]
{\begin{tcolorbox}[title=\textbf{\faExclamationTriangle\ \ \ \titlefont{{\large Προσοχή}}},
breakable,
enhanced standard,
lifted shadow={1mm}{-2mm}{3mm}{0.3mm}{black!50!white},
colback=red!5!white,
boxrule=0.1pt,
colframe=red!80!black,
fonttitle=\bfseries,
toptitle=1mm,
bottomtitle=1mm,
width=#1]}
{\end{tcolorbox}}

% ΣΤΥΛ ΠΑΡΑΔΕΙΓΜΑΤΟΣ
\newcounter{paradeigma}[chapter]
\renewcommand{\theparadeigma}{\bf\titlefont{\thechapter.\arabic{paradeigma}}} 

\newenvironment{Paradeigma}[1]{\refstepcounter{paradeigma}\textcolor{magenta!70!black}{\textbf{\large \faPlay\ \ \titlefont{Παράδειγμα}\hspace{2mm}\theparadeigma\;:\;}\hspace{1mm}} {\titlefont{#1}}\\\bfseries\boldmath}{\mbox{}\newline\lysh\mbox{}\newline}

% ΣΤΥΛ ΛΥΣΗΣ 
\newcommand{\lysh}{\textcolor{cyan!80!black}{\titlefont{\faCheck\ ΛΥΣΗ}}}

% ΛΥΜΕΝΑ ΠΑΡΑΔΕΙΓΜΑΤΑ ΤΙΤΛΟΣ 
\newcommand{\Lymena}{\begin{center}
\begin{tikzpicture}
\path[left color=cyan!70!black,right color=cyan!80!black,middle color=cyan!80!white] (-7cm,-.6cm) rectangle (6.5cm,.6cm);
\node at (-.25cm,0) {\Large \textcolor{white}{\textbf{ΛΥΜΕΝΑ ΠΑΡΑΔΕΙΓΜΑΤΑ}}};  
\end{tikzpicture}
\end{center}}

% ΑΛΥΤΕΣ ΑΣΚΗΣΕΙΣ ΤΙΤΛΟΣ 
\newcommand{\Alyta}{
\begin{tikzpicture}[overlay,remember picture]
\path[left color=maincolor,right color=maincolor!80!black,middle color=maincolor!70!white] (-.5cm,.7cm) --(-.5cm,-.5cm) -- (7.9cm,-.5cm) -- (8.7cm,.63cm) -- (18.3cm,.63cm) -- (18.3cm,.7cm)-- cycle;
\path[bottom color=white, middle color=maincolor,top color=maincolor] (-0.5cm,-0.5cm)--(-0.5cm,-3cm)--(-0.45cm,-3cm)--(-0.45cm,-0.5cm)--cycle;
\path[bottom color=white,top color=maincolor!80!black] (18.3cm,0.7cm)--(18.3cm,-3cm)--(18.25cm,-3cm)--(18.25cm,0.7cm)--cycle;
\node at (3.8cm,.1) {\LARGE \textcolor{white}{\textbf{\faPenSquare\ \ \rule{.5mm}{5mm}\ \ \titlefont{ΑΣΚΗΣΕΙΣ - ΠΡΟΒΛΗΜΑΤΑ}}}};
\end{tikzpicture}\mbox{}\\}

\newcommand{\alyta}{
\begin{tikzpicture}[overlay,remember picture]
\path[left color=maincolor,right color=maincolor,middle color=maincolor!50!white] (-.5cm,0cm) --(-.5cm,.05cm) -- (18.3cm,.05cm) -- (18.3cm,.0cm) -- cycle;
\path[top color=white,bottom color=maincolor] (-0.5cm,0cm)--(-0.5cm,2cm)--(-0.45cm,2cm)--(-0.45cm,0cm)--cycle;
\path[top color=white,bottom color=maincolor] (18.3cm,0cm)--(18.3cm,2cm)--(18.25cm,2cm)--(18.25cm,0cm)--cycle;
\end{tikzpicture}\mbox{}\\}

\newcounter{porisma}[chapter]
\renewcommand{\theporisma}{\thechapter.\arabic{porisma}}\newcommand{\Porisma}[1]{\refstepcounter{porisma}\textcolor{black}{\textbf{ΠΟΡΙΣΜΑ\hspace{2mm}\theporisma\hspace{1mm} \MakeUppercase{#1}}}\\}{}

\newcounter{protasi}[chapter]
\renewcommand{\theprotasi}{\thechapter.\arabic{protasi}}\newcommand{\Protasi}[1]{\refstepcounter{protasi}\textcolor{black}{\textbf{ΠΡΟΤΑΣΗ\hspace{2mm}\theprotasi\hspace{1mm} \MakeUppercase{#1}}}\\}{}
%---------------------------------------------------------------------------

%------------------ Λίστες ------------------------------------
\usepackage{enumitem}
\newlist{rlist}{enumerate}{3}
\setlist[rlist]{itemsep=0mm,label=\roman*.}
\newlist{alist}{enumerate}{3}
\setlist[alist]{itemsep=0mm,label=\alph*.}
\newlist{balist}{enumerate}{3}
\setlist[balist]{itemsep=0mm,label=\bf\alph*.}
\newlist{Alist}{enumerate}{3}
\setlist[Alist]{itemsep=0mm,label=\Alph*.}
\newlist{bAlist}{enumerate}{3}
\setlist[bAlist]{itemsep=0mm,label=\bf\Alph*.}
\newlist{askhseis}{enumerate}{3}
\setlist[askhseis]{label={\Large\thesection}.\arabic*.}
\renewcommand{\textstigma}{\textsigma\texttau}
\renewcommand{\textdexiakeraia}{}
% Στυλ άσκησης
\newcounter{askhsh}[chapter]
\renewcommand{\theaskhsh}{\bf{\textit{{\Large{\thechapter}}.\arabic{askhsh}}}}   
\newcommand{\Askhsh}{\refstepcounter{askhsh}\textcolor{maincolor}{{\theaskhsh}\hspace{2mm}}}{}
\setlist[itemize]{itemsep=0mm}
%---------------------------------------------------------------

%----------------- Βιβλιογραφία --------------------------------
\usepackage[backend=biber,style=alphabetic,sorting=ynt]{biblatex}
%-----------------------------------------------------------------

%----------------- Πίνακες ------------------------------------
\usepackage{tabularx,longtable,tabularray,capt-of,caption}
\DeclareTblrTemplate{caption}{nocaptemplate}{}
\DeclareTblrTemplate{capcont}{nocaptemplate}{}
\DeclareTblrTemplate{contfoot}{nocaptemplate}{}
\NewTblrTheme{mytabletheme}{
\SetTblrTemplate{caption}{nocaptemplate}{}
\SetTblrTemplate{capcont}{nocaptemplate}{}
\SetTblrTemplate{contfoot}{nocaptemplate}{}}

\NewTblrEnviron{mytblr}
\SetTblrStyle{firsthead}{font=\bfseries}
\SetTblrStyle{firstfoot}{fg=red2}
\SetTblrOuter[mytblr]{theme=mytabletheme}
\SetTblrInner[mytblr]{
rowspec={t{7mm}},
columns = {c},
width = 0.85\linewidth,
row{odd} = {bg=maincolor!20!white,fg=black,ht=8mm},
row{even} = {bg=gray!10!white,fg=black,ht=8mm},
hlines={white},
vlines={white},
row{1} = {bg=maincolor, fg=white, font=\bfseries\fontfamily{maksf}},
rowhead = 1,
hline{2} = {.7mm}, % midrule  
}

\captionsetup[figure]{format=hang,labelsep=period,name={\titlefont{\textbf{Σχήμα}}}}
\renewcommand\thefigure{{\bf\titlefont{\thechapter.\arabic{figure}}}}
%-------------------------------------------------------------------

%-------------- Περιεχόμνα ------------------------------------------
\contentsmargin{0cm}
\titlecontents{part}[-1pc]
{\addvspace{10pt}%
\bf\Large ΜΕΡΟΣ\quad }%
{}
{}
{\;\dotfill\;\normalsize\ Σελίδα}%
%------------------------------------------
\titlecontents{chapter}[0pc]
{\addvspace{30pt}%
\begin{tikzpicture}[remember picture, overlay]%
\draw[fill=black,draw=black] (-.3,.5) rectangle (3.7,1.1); %
\pgftext[left,x=0cm,y=0.75cm]{\color{white}\sc\Large\bfseries Κεφάλαιο\ \thecontentslabel};%
\end{tikzpicture}\footnotesize}%
{}
{}
{\hspace*{-2.3em}\hfill\normalsize Σελίδα \thecontentspage}%
\titlecontents{section}[2.4pc]
{\addvspace{1pt}}
{\contentslabel[\thecontentslabel]{2pc}}
{}
{\;\dotfill\;\small \thecontentspage}
[]
\titlecontents*{subsection}[4pc]
{\addvspace{-1pt}\small}
{}
{}
{\ --- \small\thecontentspage}
[ \textbullet\ ][]

\makeatletter
\renewcommand{\tableofcontents}{%
\chapter*{%
\vspace*{-20\p@}%
\begin{tikzpicture}[remember picture, overlay]%
\pgftext[right,x=15cm,y=0.2cm]{\Huge\sc\bfseries \contentsname};%
\draw[fill=black,draw=black] (12.5,-.75) rectangle (15.5,1);%
\clip (12.5,-.75) rectangle (18,1);
\pgftext[right,x=15cm,y=0.2cm]{\color{white}\Huge\bfseries \contentsname};%
\end{tikzpicture}}%
\@starttoc{toc}}
\makeatother
%---------------------------------------------------------------

\titleformat{\section}{\Large}{\titlefont{\textbf{\thesection}}}{10pt}{\Large\titlefont{\textbf{#1}}}

\setlength{\columnsep}{5mm}
\titleformat{\paragraph}
{\large}%
{}{0em}%
{\textcolor{maincolor}{\faSquare\ \ \titlefont{\bmath{#1}}}}
\setlength{\parindent}{0pt}
\titlespacing{\paragraph}{0mm}{2mm}{2mm}

\newcommand{\eng}[1]{\selectlanguage{english}#1\selectlanguage{greek}}
\newcommand{\tss}[1]{\textsuperscript{#1}}
\newcommand{\tssL}[1]{\MakeLowercase{\textsuperscript{#1}}}


\begin{document}
\pagestyle{empty}
\frontmatter
\newgeometry{inner=2.00cm, top=3.00cm, bottom=2.00cm,outer=1.50cm,textwidth=17.5cm}
\tableofcontents
\restoregeometry % restores the geometry
\mainmatter
\pagestyle{fancy}
\chapter{Εισαγωγή}
\section{Βασικές έννοιες}
...\\
Σε μια διαφορική εξίσωση ο σκοπός είναι η εύρεση μιας άγνωστης συνάρτησης $y$, μίας ή περισσοτέρων μεταβλητών.
\begin{Orismos}{Διαφορική εξίσωση}
Έστω μια παραγωγίσιμη συνάρτηση $y$. Διαφορική ονομάζεται κάθε εξίσωση που περιέχει την άγνωστη συνάρτηση $y$ και τις παραγώγους αυτής.
\end{Orismos}
\section{Ταξινόμηση διαφορικών εξισώσεων}
Οι διαφορικές εξισώσεις χωρίζονται αρχικά σε δύο μεγάλες κατηγορίες. Οι μεν \textbf{συνήθεις} διαφορικές εξισώσεις (ΣΔΕ) περιέχουν άγνωστη συνάρτηση $y$ μιας ανεξάρτητης μεταβλητής $x$ καθώς και παραγώγους αυτής. Η \textbf{γενική} ή \textbf{πεπλεγμένη} μορφή της είναι
\begin{equation}\label{eq:intro:1}
F\left(x,y,y',\ldots,y^{(\nu)}\right)=0
\end{equation}
όπου $y^{(\nu)}=\diff[\nu]{y}{x}$ η συνήθης παράγωγος $\nu-$οστής τάξης. Αν η δομή της εξίσωσης είναι τέτοια ώστε να επιτρέπει να γραφτεί η παράγωγος μέγιστης τάξης συναρτήσει των υπολοίπων παραγώγων και της συνάρτησης $y$ τότε έχουμε τη λεγόμενη \textbf{λυμένη} ή \textbf{άμεση} μορφή:
\[ y^{(\nu)}=f\left(x,y,y',\ldots,y^{(\nu-1)}\right) \]
Οι δε \textbf{μερικές} διαφορικές εξισώσεις (ΜΔΕ) περιέχουν άγνωστη συνάρτηση $u$ πολλών μεταβλητών καθώς και μερικές παραγώγους αυτής. Για παράδειγμα η διαφορική εξίσωση
\[ x^2y''-\sin{x}y'+xy=e^x \]
είναι μια συνήθης διαφορική εξίσωση ενώ η
\[ u_{xx}-cu_y+u_{yy}=0 \]
αποτελεί μερική διαφορική εξίσωση, όπου $u_{xx}=\diffp[2]{u}{x}, u_y=\diffp{u}{y}$ και $u_{yy}=\diffp[2]{u}{y}$ οι μερικές παράγωγοι της συνάρτησης $u(x,y)$. Στο βιβλίο αυτό θα μας απασχολήσουν κατά κύριο λόγο οι συνήθεις διαφορικές εξισώσεις και οι μέθοδοι επίλυσής τους.\\\\
\begin{Paradeigma}{Ταξινόμηση διαφορικών εξισώσεων}
Χαρακτηρίστε τις ακόλουθες διαφορικές εξισώσεις ως συνήθεις ή μερικές.
\begin{multicols}{2}
\begin{alist}
\item $y'+2y=x$
\item $yy'=e^x$
\item $u_x+u_y=0$
\item $y'=3y-x^2$
\item $u_{xx}=c^2u_t$
\item $(x+1)\d x+\cos{y}\d y=0$
\end{alist}
\end{multicols}\vspace{-5mm}
\end{Paradeigma}
Σύμφωνα με τις βασικές έννοιες που δώσαμε προηγουμένως, οι εξισώσεις α.,β.,δ. και στ. είναι ΣΔΕ, με την στ. να περιέχει την παράγωγο της συνάρτησης $y$ στη διαφορική μορφή της, ενώ οι γ. και ε. είναι ΜΔΕ. Για κάθε είδος εξίσωσης θα μας απασχολήσουν επίσης έννοιες όπως η \textbf{τάξη} και ο \textbf{βαθμός} μιας διαφορικής εξίσωσης. Στο εξής οι έννοιες και οι ορισμού που θα δώσουμε θα αφορούν αποκλειστικά τις συνήθεις διαφορικές εξισώσεις.
\begin{Orismos}{Τάξη και βαθμός Δ.Ε.}
\begin{alist}
\item Τάξη μιας διαφορικής εξίσωσης ονομάζεται η μεγαλύτερη τάξη παραγώγου που περιέχεται στην εξίσωση. 
\item Βαθμός μιας διαφορικής εξίσωσης ονομάζεται ο εκθέτης της παραγώγου μεγαλύτερης τάξης.
\end{alist}
\end{Orismos}
\begin{marginfigure}[-3cm]
\begin{Parathrhsh}{5cm}
Για μια διαφορική εξίσωση ορίζεται βαθμός εφόσον οι όροι της μπορούν να γραφτούν σε πολυωνυμική μορφή ως προς την άγνωστη συνάρτηση $y$.
\end{Parathrhsh}
\end{marginfigure}
Πριν δούμε παραδείγματα πάνω στις έννοιες αυτές, θα εμβαθύνουμε περισσότερο στην ταξινόμηση των διαφορικών εξισώσεων ως προς τη δομή τους. Μια διαφορική εξίσωση λέγεται \textbf{γραμμική} αν μπορεί να γραφτεί στη μορφή
\[ a_{\nu}(x)y^{(\nu)}+a_{\nu-1}(x)y^{(\nu-1)}+\ldots+a_1(x)y'+a_0(x)y=\beta(x) \]
όπου $a_i(x),\ i=0,1,\ldots,\nu$ και $\beta(x)$ συνεχείς συναρτήσεις σε ένα διάστημα $[a,\beta]$ του $\mathbb{R}$. Σε κάθε άλλη περίπτωση η εξίσωση λέγεται \textbf{μη γραμμική}. Οι μη γραμμικές διαφορικές εξισώσεις με τη σειρά τους ταξινομούνται περαιτέρω σε επιμέρους κατηγορίες ως προς τη σχέση της συνάρτησης $y$ και των παραγώγους της. Μία μη γραμμική ΣΔΕ ονομάζεται
\begin{itemize}
\item \textbf{ημιγραμμική} εάν είναι μη γραμμική ως προς τη συνάρτηση $y$ και γραμμική ως προς τις παραγώγους της.
\item \textbf{σχεδόν γραμμική} εάν είναι μη γραμμική ως προς τις $y,y',\ldots,y^{(\nu-1)}$ και γραμμική ως προς την παράγωγο $y^{(\nu)}$ μεγαλύτερης τάξης.
\item \textbf{πλήρως μη γραμμική} εάν είναι μη γραμμική τουλάχιστον ως προς την παράγωγο $y^{(\nu)}$ μεγαλύτερης τάξης.
\end{itemize}
\begin{Paradeigma}{Ταξινόμηση τάξη και βαθμός ΣΔΕ}
Ταξινομήστε τις παρακάτω διαφορικές εξισώσεις σε γραμμικές ή μη γραμμικές, καθώς και το είδος αυτών και βρείτε την τάξη και το βαθμό της καθεμίας, όπου αυτός ορίζεται.
\begin{multicols}{2}
\begin{alist}
\item $y''-x^2y'+xy=0$
\item $y'''+2y''-3y'+y=x $
\item $yy'=\sin{x}$
\item $\left(y'\right)^2+2y=e^x$
\item $y^{(4)}=y^2$
\item $y'''=x^2y''-y$
\item $y'=\sin{y}$
\item $\left(y'''\right)^2-3yy''+xy=0$
\item $y'''+\left(y''\right)^3-xy^2y'=\ln{x}$
\item $x^3\d y+y\d x=0$
\end{alist}
\end{multicols}
\end{Paradeigma}
\begin{Orismos}{Λύση διαφορικής εξίσωσης}
Λύση ή ολοκλήρωμα μιας συνήθους διαφορικής εξίσωσης της μορφής \eqref{eq:intro:1} ονομάζεται κάθε συνάρτηση $y\in C^{\nu}(\varDelta)$ που επαληθεύει την εξίσωση για κάθε $x\in\varDelta$.
\end{Orismos}
Η γραφική παράσταση μιας λύσης ονομάζεται \textbf{ολοκληρωτική καμπύλη}. Όταν η λύση εκφράζεται με την βοήθεια $\nu$ σε πλήθος παραμέτρων $c_i,i=1,2,\ldots,\nu$ στη μορφή \[y=y(x,c_1,\ldots,c_{\nu})\] τότε έχουμε τη λεγόμενη \textbf{γενική λύση} ή \textbf{γενικό ολοκλήρωμα} της εξίσωσης. Συγκεκριμένα στη μορφή αυτή, η λύση είναι μια σχέση λυμένη ως προς τη συνάρτηση $y$ η οποία γράφεται ως συνάρτηση της ανεξάρτητης μεταβλητής $x$ και των παραμέτρων. Σε περιπτώσεις που η σχέση αυτή δεν είναι δυνατόν να λυθεί ως προς $y$, έχουμε την \textbf{πεπλεγμένη γενική λύση} ή \textbf{πεπλεγμένο γενικό ολοκλήρωμα} στη μορφή \[G(x,y,c_1,\ldots,c_{\nu})=0.\]
Για παράδειγμα, η συνάρτηση $y(x)=c_1\cos{x}+c_2\sin{x}$ αποτελεί γενική λύση της διαφορικής εξίσωσης $y''=-y$, δοσμένη στην απλή (λυμένη) μορφή. Από την άλλη μεριά, η σχέση $y^3+y=\cos{x}$ είναι πεπλεγμένη λύση της διαφορικής εξίσωσης $y'\left(3y^2+1\right)=-\sin{x}$ το οποίο επαληθεύεται εύκολα παραγωγίζοντας την πρώτη σχέση. Αν επιλέξουμε συγκεκριμένες τιμές για τις παραμέτρους $c_i$ τότε από τη γενική λύση παίρνουμε μια \textbf{ειδική} ή \textbf{μερική λύση} της διαφορικής εξίσωσης. Υπάρχουν λύσεις διαφορικών εξισώσεων που δεν προκύπτουν από τη γενική λύση για καμία τιμή των παραμέτρων $c_i$. Αυτές ονομάζονται \textbf{ιδιάζουσες λύσεις} της εξίσωσης. Συγκεντρώνοντας όλες τις λύσεις μιας διαφορικής εξίσωσης, παίρνουμε την \textbf{πλήρη λύση} της.\\\\
\begin{Paradeigma}{Λύσεις διαφορικής εξίσωσης}
Χρησιμοποιώντας τις βασικές γνώσεις πάνω στις παραγώγους γνωστών συναρτήσεων, να προσδιορίσετε τις λύσεις των παρακάτω εξισώσεων...
\end{Paradeigma}
\section{Προβλήματα αρχικών και συνοριακών τιμών}
\begin{Orismos}{Πρόβλημα αρχικών τιμών}

\end{Orismos}
\begin{Thewrhma}{Σύγκριση αριθμών}
Ένας αριθμός $ a $ λέγεται μεγαλύτερος από έναν αριθμό $ \beta $ όταν η διαφορά $ a-\beta $ είναι θετικός αριθμός.
\end{Thewrhma}
\begin{marginfigure}[-1cm]
\begin{Prosoxi}{5cm}
Δεν ορίζεται ρίζα αρνητικού αριθμού.
\end{Prosoxi}
\end{marginfigure}
\lipsum\lipsum\\\\
\full{\Alyta}\vspace{-5mm}
\fulltwoc{\begin{adjustwidth}{3mm}{3mm}
\Askhsh Ταξινομήστε τις ακόλουθες διαφορικές εξισώσεις σε συνήθεις και μερικές.
\begin{alist}
\item $y''+2y'+y=x^2$
\item $u_x+u_y=x$
\item $\left(y''\right)^3-2y'=e^x$
\item $\diffp[2]{u}{x}+\diffp{u}{x,y}=u$
\item $\Delta^2y=0$
\item $\sin{x}\d y+e^y\d x=0$
\item $r'-(1-\theta)r=\cos{\theta}$
\item $uu_y=1-u_x$
\end{alist}
\Askhsh Χαρακτηρίστε καθεμία από τις ακόλουθες συνήθεις διαφορικές εξισώσεις ως γραμμική ή μη γραμμική. Στην περίπτωση μη γραμμικής να αναφέρετε αν είναι ημιγραμμική, σχεδόν γραμμική ή πλήρως μη γραμμική.
\begin{alist}
\item $x^3y''+xy'-2y=\cos{x}$
\item $x^2y''+3xy'+y^2=1$
\item $yy'=x^3$
\item $y'''+\sin{y}=e^y$
\item $\sqrt{y''}+y=0$
\item $x^2 \d y-y^3\d x=0$
\item $(1+x^2)y''-2xy'+y=e^x$
\end{alist}
\end{adjustwidth}}
\full{\alyta}
\chapter{Διαφορικές εξισώσεις 1\tss{ης} τάξης}
\section{Εξισώσεις χωριζομένων μεταβλητών}
Το όνομά τους υποδεικνύει και τη δομή τους. Είναι τέτοια ώστε να μπορέσουμε να διαχωρίσουμε στα δύο μέλη της εξίσωσης την ανεξάρτητη μεταβλητή $x$ από την εξαρτημένη $y$. Οι διαφορικές εξισώσεις χωριζομένων μεταβλητών είναι από τις πιο απλές, στην επίλυσή τους, που θα συναντήσουμε. Αν η εξίσωση είναι γραμμένη στην επιλύσιμη μορφή της, η εύρεση της λύσης γίνεται με άμεση ολοκλήρωση.
\begin{Orismos}{Εξίσωση χωριζομένων μεταβλητών}
Μια διαφορική εξίσωση 1ης τάξης, της μορφής
\[ y'=h(x,y) \]
θα λέγεται εξίσωση χωριζομένων μεταβλητών εάν η συνάρτηση $h$ μπορεί να γραφτεί ως γινόμενο δύο συναρτήσεων $f(x)$, μεταβλητής $x$ και $g(y)$, μεταβλητής $y$. Θα έχει δηλαδή τη μορφή
\begin{equation}\label{eq:sepvar:1}
y'=f(x)g(y)
\end{equation}
\end{Orismos}
Βλέπουμε ότι πρόκειται για μια \textbf{μη γραμμική} διαφορική εξίσωση 1ης τάξης. Η διαδικασία επίλυσής της έχει τα εξής βήματα. 
\paragraph{Γενική λύση}
Η παράγωγος $y'$ γράφεται στη διαφορική της μορφή δηλαδή $\diff{y}{x}$. Στη συνέχεια, αν είναι απαραίτητο, παραγοντοποιούμε το δεύτερο μέλος της εξίσωσης ώστε να προκύψουν οι παράγοντες $f(x)$ και $g(y)$. Η εξίσωση τότε μπορεί να πάρει τη μορφή
\[ \diff{y}{x}=f(x)g(y)\Rightarrow \diff{y}{x}\frac{1}{g(y)}=f(x)\d x \]
Ολοκληρώνουμε τα δύο μέλη της ως προς $x$ και καταλήγουμε στη σχέση
\begin{equation}\label{eq:sepvar:2}
\int\frac{1}{g(y)}\diff{y}{x}\d x=\int f(x)\d x+c
\end{equation}
η οποία, χρησιμοποιώντας την αντικατάσταση $y=y(x)\Rightarrow\d y=\d y(x)\Rightarrow \d y=\diff{y}{x}\d x$, γράφεται ισοδύναμα
\begin{equation}\label{eq:sepvar:3}
\int\frac{1}{g(y)}\d y=\int f(x)\d x+c
\end{equation}
από την οποία οδηγούμαστε στη γενική λύση της εξίσωσης, είτε σε λυμένη είτε σε πεπλεγμένη μορφή. Καθώς οι σχέσεις \eqref{eq:sepvar:2} και \eqref{eq:sepvar:3} είναι ισοδύναμες, μπορούμε από δω και πέρα χάριν συντομίας, να χρησιμοποιούμε κατευθείαν τον τύπο \eqref{eq:sepvar:3} μόλις επιτύχουμε το διαχωρισμό των μεταβλητών.\\\\
\begin{Paradeigma}{Εξίσωση χωριζομένων μεταβλητών}
Βρείτε τη γενική λύση της εξίσωσης.
\[ y'=-xy^2 \]
όπου $x\in\mathbb{R}^*$.
\end{Paradeigma}
\begin{marginfigure}[-3cm]
\begin{tikzpicture}
\begin{axis}[width=6.5cm,height=7cm,
xmin=-5,xmax=5,
ymin=-8,ymax=8,
xtick={-5,-3,...,5},
ytick={-8,-6,...,8},
xlabel={\footnotesize $ x $},
ylabel={\footnotesize $ y $},
belh ar,aks_on,
restrict y to domain=-20:20,
grid=both,
grid style={line width=.1pt, draw=gray!10},
major grid style={line width=.2pt,draw=gray!50},
minor tick num=4]
\begin{scope}
\clip (axis cs:-5,-8) rectangle (axis cs:5,8);
\addplot[grafikh parastash,domain=-7:7,maincolor]{1/x^2};
\addplot[grafikh parastash,domain=-7:7,secondarycolor]{-1/x^2};
\end{scope}
\node at (axis cs:3,3) {$y(x)=\frac{2+c}{x^2}$};
\node at (axis cs:-.5,-0.5) {\footnotesize$O$};
\end{axis}
\end{tikzpicture}\captionof{figure}{Ολοκληρωτικές καμπύλες της εξίσωσης $y'=-xy^2$}
\end{marginfigure}
Η εξίσωση είναι μη γραμμική διαφορική 1ης τάξης. Σύμφωνα με την παραπάνω μέθοδο έχουμε
\begin{gather*}
y'=-xy^2\Rightarrow
\diff{y}{x}=-xy^2\Rightarrow\\
-\frac{1}{y^2}\diff{y}{x}=x\Rightarrow
-\int{\frac{1}{y^2}\d y}=\int{x\d x}\Rightarrow\\
\frac{1}{y}=\frac{x^2}{2}+c_1\Rightarrow y(x)=\frac{2+c}{x^2}
\end{gather*}
\begin{Paradeigma}{Πεπλεγμένη λύση}
Βρείτε τη γενική λύση της ακόλουθης διαφορικής εξίσωσης χωριζομένων μεταβλητών
\[ \left(3y^2-1\right)y'=1\ ,\ x\in\mathbb{R} \]
\end{Paradeigma}
Παρατηρούμε ότι στην εξίσωση είναι ήδη διαχωρισμένες οι μεταβλητές
\begin{gather*}
\left(3y^2-1\right)y'=1\Leftrightarrow\\
3y^2y'-y'=2x\Leftrightarrow\\
\int{\left(3y^2y'-y'\right)\d y}=\int{1\d x}+c\Leftrightarrow\\
y^3-y=x+c
\end{gather*}
με $c\in\mathbb{R}$ μια αυθαίρετη σταθερά. Η λύση της εξίσωσης είναι δοσμένη σε πεπλεγμένη μορφή. Στο διπλανό σχήμα βλέπουμε την ολοκληρωτική καμπύλη της ειδικής λύσης της, για $c=0$.
\begin{marginfigure}[-5cm]
\begin{tikzpicture}
\begin{axis}[width=6.5cm,height=5.9cm,
xmin=-6,xmax=6,
ymin=-3,ymax=3,
xtick={-6,-4,...,6},
ytick={-3,-2,...,3},
xlabel={\footnotesize $ x $},
ylabel={\footnotesize $ y $},
belh ar,aks_on,
restrict y to domain=-20:20,
grid=both,
grid style={line width=.1pt, draw=gray!10},
major grid style={line width=.2pt,draw=gray!50},
minor tick num=4]
\begin{scope}
\clip (axis cs:-7,-7) rectangle (axis cs:7,7);
\addplot +[grafikh parastash,domain=-7:7,maincolor,no markers,
      raw gnuplot,
      thick,
      empty line = jump] gnuplot {
      set contour base;
      set cntrparam levels discrete 0.003;
      unset surface;
      set view map;
      set isosamples 500;
      set samples 500;
      splot y^3-y-x;
    };
\end{scope}
\node at (axis cs:-3,1.5) {$y^3-y=x$};
\node at (axis cs:-.5,-0.5) {\footnotesize$O$};
\end{axis}
\end{tikzpicture}\captionof{figure}{Πεπλεγμένη λύση της εξίσωσης για $c=0$}
\end{marginfigure}
\paragraph{Πρόβλημα αρχικών τιμών}
Ένα πρόβλημα αρχικών τιμών με διαφορική εξίσωση χωριζομένων μεταβλητών, μπορεί να αντιμετωπιστεί με δύο τρόπους.
\begin{itemize}
\item Είτε εύρεση της γενικής λύσης της εξίσωσης και κατόπιν χρήση της αρχικής συνθήκης $y(x_0)=y_0$ για την εύρεση των σταθερών
\item είτε ολοκλήρωση των μελών της εξίσωσης χρησιμοποιώντας ορισμένο ολοκλήρωμα.
\end{itemize}
Ας δούμε την κατασκευή του τύπου για τη δεύτερη μέθοδο. Το πρόβλημα αποτελείται από την εξίσωση \eqref{eq:sepvar:1} με αρχική $y(x_0)=y_0$. Όταν η εξίσωση γραφτεί στη μορφή
\[ \frac{1}{g(y(x))}\diff{y}{x}=f(x)\d x \]
ολοκληρώνουμε κάθε μέλος της από $x_0$ έως $x$ και παίρνουμε
\[ \int_{x_0}^{x}{\frac{1}{g(y(t))}\diff{y}{t}\d t}=\int_{x_0}^{x}{f(t)\d t} \]
Χρησιμοποιούμε την αντικατάσταση $s=y(t)$ η οποία μας δίνει
\begin{itemize}
\item $s=y(t)\Rightarrow \d s=\d y(t)=\diff{y}{t}\d t$ και
\item για $t=x_0\Rightarrow s=y(x_0)=y_0$ καθώς και $t=x\Rightarrow s=y(x)=y$.
\end{itemize}
Έτσι η τελευταία ισότητα μας δίνει το ζητούμενο τύπο 
\begin{equation}\label{eq:sepvar:4}
\int_{y_0}^{y}{\frac{1}{g(s)}\d s}=\int_{x_0}^{x}{f(t)\d t}
\end{equation}
Στο επόμενο παράδειγμα θα αντιμετωπίσουμε ένα πρόβλημα αρχικών τιμών με την πρώτη μέθοδο.\\\\
\begin{Paradeigma}{Εξίσωση χωριζομένων μεταβλητών}
Διαπιστώστε ότι η διαφορική εξίσωση 
\[y'=2xy-4x+y-2\]
είναι εξίσωση χωριζομένων μεταβλητών και βρείτε τη λύση του προβλήματος αρχικών τιμών στο οποίο $y(1)=0$.
\end{Paradeigma}
Παραγοντοποιούμε το δεύτερο μέλος της εξίσωσης και έχουμε
\begin{align*}
y'&=2xy-4x+y-2=\\&=2x(y-2)+(y-2)=\\&=(y-2)(2x-1)
\end{align*}
επομένως πρόκειται για μια εξίσωση χωριζομένων μεταβλητών. Αυτή τώρα μας δίνει
\begin{gather*}
y'=(y-2)(2x-1)\Leftrightarrow\\
\diff{y}{x}=(y-2)(2x-1)\Leftrightarrow
\frac{1}{y-2}\diff{y}{x}=(2x-1)\d x\Leftrightarrow\\
\int{\frac{\d y}{y-2}}=\int{(2x-1)\d x}+c_1\Leftrightarrow\\
\ln{|y-2|}=x^2-x+c_1\Leftrightarrow\\
|y-2|=ce^{x^2-x}
\end{gather*}
\begin{marginfigure}[-3cm]
\begin{tikzpicture}
\begin{axis}[width=6.5cm,height=7cm,
xmin=-1,xmax=2,
ymin=-3,ymax=1,
xtick={-1,-0.5,...,2},
ytick={-3,-2.5,...,1},
xlabel={\footnotesize $ x $},
ylabel={\footnotesize $ y $},
belh ar,aks_on,
grid=both,
grid style={line width=.1pt, draw=gray!10},
major grid style={line width=.2pt,draw=gray!50},
minor tick num=4]
\begin{scope}
\clip (axis cs:-1,-3) rectangle (axis cs:2,1);
\addplot[grafikh parastash,domain=-1:2,maincolor]{2-2*exp(x^2-x)};
\end{scope}
\node at (axis cs:1.2,0.7) {\footnotesize $y(x)=2-2e^{x^2-x}$};
\node[fill=white,inner sep=0.2mm,opacity=0.7,text opacity=1] at (axis cs:-.15,-0.15) {\footnotesize$O$};
\node[labelbox={maincolor}](A) at (0.5,-0.75){Αρχική\\συνθήκη};
\draw[-latex] (A.30)--(axis cs:1,0);
\coordinate (iv) at (axis cs:1,0);
\end{axis}
\fill[maincolor] (iv) circle(0.07);
\end{tikzpicture}\captionof{figure}{Ολοκληρωτική καμπύλη της λύσης της $y'=2xy-4x+y-2$.}
\end{marginfigure}
όπου $c_1,c=e^{c_1}>0$ αυθαίρετες σταθερές. Η λύση αυτή είναι γραμμένη σε πεπλεγμένη μορφή. Επιπλέον γνωρίζουμε ότι η συνάρτηση $y$ και κατά συνέπεια η $y-2$ ανήκει στο χώρο $C(\mathbb{R})$ των συνεχώς παραγωγίσιμων συναρτήσεων, δεν μηδενίζεται διότι ούτε και η $ce^{x^2-x}$ μηδενίζεται άρα η τελευταία σχέση γράφεται
\[ y-2=\pm ce^{x^2-x} \]
Σύμφωνα τώρα με την αρχική συνθήκη $y(1)=0$ παίρνουμε $y(1)-2=-2<0$ άρα $y-2<0$ για κάθε $x\in\mathbb{R}$ καθώς και
\[|y(1)-2|=ce^{0}\Rightarrow |-2|=c\Rightarrow c=2\]
και παίρνουμε έτσι τη λύση του Π.Α.Τ.: $2-y=2e^{x^2-x}\Rightarrow y(x)=2-2e^{x^2-x}$.\\\\
\begin{Paradeigma}{Εξίσωση χωριζομένων μεταβλητών - Π.Α.Τ.}
Λύστε το ακόλουθο πρόβλημα αρχικών τιμών
\[\ccases{e^{x}y'=\dfrac{1-x}{2y}\\y(1)=\frac{1}{\sqrt{e}}}\]
με $y\neq 0$ και $x\in[0,+\infty)$.
\end{Paradeigma}
\begin{marginfigure}[0mm]
\begin{tikzpicture}
\begin{axis}[width=6.5cm,height=5.5cm,
xmin=-4,xmax=16,
ymin=-0.25,ymax=0.75,
xtick={-4,0,...,16},
ytick={-0.25,0,...,1},
xlabel={\footnotesize $ x $},
ylabel={\footnotesize $ y $},
belh ar,aks_on,
grid=both,
grid style={line width=.1pt, draw=gray!10},
major grid style={line width=.2pt,draw=gray!50},
minor tick num=4]
\begin{scope}
\clip (axis cs:-1,-1) rectangle (axis cs:15,1);
\addplot[grafikh parastash,domain=0:15,secondarycolor]{sqrt(x*exp(-x))};
\end{scope}
\node at (axis cs:3,3) {$y(x)=-\frac{2}{x}$};
\node at (axis cs:-1,-0.05) {\footnotesize$O$};
\node[xshift=1.7cm,yshift=-5mm,labelbox={secondarycolor}](A) at (1,0.61){Αρχική\\συνθήκη};
\draw[-latex] (A.west)--(axis cs:1,0.61);
\coordinate (iv) at (axis cs:1,0.61);
\end{axis}
\fill[secondarycolor] (iv) circle (0.07);
\end{tikzpicture}\captionof{figure}{Λύση του προβλήματος $e^{x}y'=\frac{1-x}{2y}$}
\end{marginfigure}
Διαχωρίζουμε αρχικά τις μεταβλητές στην εξίσωση
\[ e^{x}y'=\dfrac{1-x}{2y}\Rightarrow 2yy'=e^{-x}(1-x) \]
εφαρμόζουμε στη τελευταία, τον τύπο \eqref{eq:sepvar:4} οπότε παίρνουμε
\begin{gather*}
\int_{y(1)}^{y}2ss'\d s=\int_{1}^{x}e^{-t}(1-t)\d t\Rightarrow\\
\left.s^2\right|_{\frac{1}{\sqrt{e}}}^{y}=\left.te^{-t}\right|_{1}^{x}\Rightarrow y^2-\dfrac{1}{e}=xe^{-x}-\dfrac{1}{e}\Rightarrow\\
y^2=xe^{-x}
\end{gather*}
Αφού $y\in C((0,+\infty)),\ y\neq 0$ και $y(1)>0$ τότε $y(x)>0,\ x\in(0,+\infty)$ άρα η λύση του προβλήματος είναι $y(x)=\sqrt{xe^{-x}}$.
\section{Πλήρεις διαφορικές εξισώσεις}
\begin{Orismos}{Πλήρης διαφορική εξίσωση 1ης τάξης}
Μια διαφορική εξίσωση της μορφής
\begin{equation}\label{eq:exact:1}
M(x,y)\d x+N(x,y)\d y=0
\end{equation}
ονομάζεται \textbf{πλήρης} διαφορική εξίσωση 1ης τάξης...
\end{Orismos}
\begin{marginfigure}[0mm]
\begin{Shmeiwsh}{5.4cm}
Αν $F(x,y)$ είναι μια συνεχής συνάρτηση δύο μεταβλητών σε κάποιο χωρίο $D\subseteq\mathbb{R}^2$ τότε η παράσταση
\[ \d F=\diffp{F}{x}\d x+\diffp{F}{y}\d y \]
ονομάζεται πλήρες διαφορικό της $F$.
\end{Shmeiwsh}
\end{marginfigure}
... Στην περίπτωση ικανοποίησης της συνθήκης πληρότητας, η παράσταση $M\d x+N\d y$ αποτελεί ένα \textbf{πλήρες διαφορικό}. Μας είναι γνωστό από την πραγματική ανάλυση ότι το διαφορικό μιας συνάρτησης $f(x,y)$ δίνεται από τη σχέση
\begin{equation}\label{eq:exact:4}
\d f(x,y)=\diffp{f}{x}\d x+\diffp{f}{y}\d y
\end{equation}
Θέλουμε έτσι η παράσταση αυτή να ταυτίζεται με το πλήρες διαφορικό που αναφέραμε, επομένως πρέπει
\[ \diffp{f}{x}=M(x,y)\ \text{ και }\ \diffp{f}{y}=N(x,y) \]
και καθώς το διαφορικό αυτό είναι ταυτοτικά μηδέν, τότε η ζητούμενη συνάρτηση $f(x,y)$ είναι σταθερή δηλαδή
\begin{equation}\label{eq:exact:3}
f(x,y)=c
\end{equation}
Έτσι συμπεραίνουμε ότι η λύση μιας πλήρους διαφορικής εξίσωσης δίνεται σε πεπλεγμένη μορφή από την παραπάνω σχέση. Η εξίσωση \eqref{eq:exact:1} γράφεται στη μορφή
$\d f(x,y)=0$. (εισαγωγή στη συνθήκη πληρότητας...)
\begin{Thewrhma}{Συνθήκη πληρότητας}
Η διαφορική εξίσωση $M\d x+N\d y=0$ είναι πλήρης αν και μόνο αν ισχύει η σχέση
\begin{equation}\label{eq:exact:2}
\diffp{M}{y}=\diffp{N}{x}
\end{equation}
\end{Thewrhma}
\paragraph{Βασική μέθοδος επίλυσης}
Για την εύρεση της λύσης επιλέγουμε μια από τις σχέσεις \eqref{eq:exact:4}, για παράδειγμα την $f_x=M$ και ολοκληρώνουμε κάθε μέλος ως προς $x$. Έτσι παίρνουμε
\begin{equation}\label{eq:exact:5}
f(x,y)=\int{M(x,y)\d x}+g(y)
\end{equation}
όπου $g(y)$ συνεχής στο διάστημα $\varDelta$. Έπειτα παραγωγίζουμε την παραπάνω σχέση ως προς $y$ και προσδιορίζουμε τη συνάρτηση $g(y)$ εξισώνοντας την παράσταση που θα προκύψει με την $N(x,y)$:
\begin{gather}\label{eq:exact:8}
\diffp{f}{y}=\diffp{}{y}\left[\int{M\d x}+g(y)\right]\Rightarrow N(x,y)=\int{\diffp{M}{y}\d x}+g'(y)
\end{gather}
Από την τελευταία ισότητα προσδιορίζουμε την $g(y)$ και με αντικατάσταση στη σχέση \eqref{eq:exact:5} οδηγούμαστε στη λύση της εξίσωσης. Η διαδικασία ακολουθεί τα ίδια βήματα κι στην περίπτωση όπου επιλέξουμε να ξεκινήσουμε από τη δεύτερη σχέση της \eqref{eq:exact:5}. την οποία θα ολοκληρώσουμε ως προς $y$. Θα έχουμε 
\begin{equation}\label{eq:exact:6}
f_y=N\Leftrightarrow f(x,y)=\int{N(x,y)\d y}+h(x)
\end{equation}
Παραγωγίζουμε την \eqref{eq:exact:6} ως προς $x$ και προσδιορίζουμε την συνάρτηση $h(x)$
\begin{gather}\label{eq:exact:9}
\diffp{f}{x}=\diffp{}{x}\left[\int{N\d y}+h(x)\right]\Rightarrow M(x,y)=\int{\diffp{N}{x}\d y}+h'(x)
\end{gather}
οποία και αντικαθιστούμε στην \eqref{eq:exact:6} και προκύπτει η πεπλεγμένη λύση. Έχουμε επίσης τη δυνατότητα να εργαστούμε περαιτέρω με τις ισότητες \eqref{eq:exact:8} και \eqref{eq:exact:9} ώστε αυτές να μας δώσουν τύπους για άμεσο προσδιορισμό των συναρτήσεων $g(y)$ και $h(x)$. Παίρνουμε έτσι αντίστοιχα:
\begin{gather}\label{eq:exact:10}
N(x,y)=\int{\diffp{N}{x}\d x}+g'(y)\Leftrightarrow g(y)=\int{\left(N-\int{N_x\d x}\right)\d y}\\
M(x,y)=\int{\diffp{M}{y}\d y}+h'(x)\Leftrightarrow h(x)=\int{\left(M-\int{M_y}\d y\right)\d x}
\end{gather}
\begin{Paradeigma}{Πλήρης διαφορική εξίσωση}
\[ ye^x\d x+(e^x+\sin{y})\d y=0 \]
\end{Paradeigma}
Η εξίσωση έχει τη μορφή \eqref{eq:exact:1} και βλέπουμε ότι $M(x,y)=ye^x$ και $N(x,y)=e^x+\sin{y}$. Στη συνέχεια έχουμε 
\[ \diffp{M}{y}=\diffp{(ye^x)}{y}=e^x=\diffp{(e^x+\sin{y})}{x}=\diffp{N}{x} \]
οπότε η εξίσωση είναι πλήρης αφού ικανοποιείται η συνθήκη πληρότητας. Υπάρχει λοιπόν σταθερή συνάρτηση της μορφής $f(x,y)=c$ για την οποία $f_x=M$ και $f_y=N$. Ολοκληρώνουμε την πρώτη σχέση ως προς $x$ και έχουμε
\begin{gather}
f_x=ye^x\Rightarrow f(x,y)=\int{ye^x\d x}+g(y)\Rightarrow \nonumber\\
f(x,y)=ye^x+g(y)\label{example:exact:1}
\end{gather}
Παραγωγίζουμε στη συνέχεια την τελευταία ως προς $y$ και είναι
\begin{gather*}
f_y=e^x+g'(y)\Rightarrow e^x+\sin{y}=e^x+g'(y)\Rightarrow\\
g'(y)=\sin{y}\Rightarrow g(y)=-\cos{y}+c_1
\end{gather*}
Αν αντικαταστήσουμε έτσι την $g(y)$ στην \eqref{example:exact:1} καταλήγουμε στην πεπλεγμένη λύση της εξίσωσης
\[ f(x,y)=ye^x-\cos{y}+c_1\Rightarrow ye^x-\cos{y}=c \]
\begin{Paradeigma}{}
Να λυθεί η διαφορική εξίσωση
\[ y' = -\frac{y^2 + e^x}{2xy} \]
με $y\neq 0$ και $x\in\mathbb{R}$.
\end{Paradeigma}
\begin{marginfigure}[0mm]
\begin{tikzpicture}
\begin{axis}[width=6.5cm,height=7cm,
xmin=-3,xmax=2,
ymin=-4,ymax=4,
xtick={-3,-2,...,2},
ytick={-4,-3,...,4},
xlabel={\footnotesize $ x $},
ylabel={\footnotesize $ y $},
belh ar,aks_on,
grid=both,
grid style={line width=.1pt, draw=gray!10},
major grid style={line width=.2pt,draw=gray!50},
minor tick num=4,
legend style={draw=none, fill opacity=0.7,text opacity=0.7,font=\scriptsize},
legend pos=north west,
legend image post style={scale=0.5}]
\begin{scope}
\foreach [evaluate=\c as \percent using 100-30*\c] \c in {3,2,1.5,0.5,0.15}
{\edef\temp{\noexpand%
\addplot +[grafikh parastash,domain=-3:3,maincolor!\percent!secondarycolor,no markers,
      raw gnuplot,
      thick,
      empty line = jump] gnuplot {
      set contour base;
      set cntrparam levels discrete 0.003;
      unset surface;
      set view map;
      set isosamples 500;
      set samples 200;
      splot x*y^2+exp(x)-\c;
    };}\temp
}
\end{scope}
\node at (axis cs:3,3) {$y(x)=-\frac{2}{x}$};
\node at (axis cs:-.25,-0.5) {\footnotesize$O$};
\legend{$c=3$,$c=2$,$c=1.5$,$c=0.5$,$c=0.15$}
\end{axis}
\end{tikzpicture}\captionof{figure}{Μερικές λύσεις της εξίσωσης $y' = -\frac{y^2 + e^x}{2xy}$}
\end{marginfigure}
Η εξίσωση γράφεται ισοδύναμα στη μορφή
\begin{gather*}
y' = -\frac{y^2 + e^x}{2xy}\Leftrightarrow\diff{y}{x}=-\frac{y^2 + e^x}{2xy}\Leftrightarrow\\
\left(y^2+e^x\right)\d x+2xy\d y=0
\end{gather*}
από την οποία βλέπουμε ότι $M(x,y)=y^2+e^x$ και $N(x,y)=2xy$. Ελέγχουμε στη συνέχεια αν η εξίσωση είναι πλήρης. Πράγματι
\[ \diffp{M}{y}=\diffp{\left(y^2+e^x\right)}{y}=2y=\diffp{(2xy)}{x}=\diffp{N}{x} \]
επομένως υπάρχει συνάρτηση $f(x,y)=c$ η οποία ικανοποιεί την εξίσωση. Γι αυτήν ισχύει $\diffp{f}{x}=M$ και $\diffp{f}{y}=N$. Με ολοκλήρωση της δεύτερης σχέσης ως προς $y$ έχουμε
\[ f_y=2xy\Leftrightarrow f(x,y)=xy^2+h(x) \]
Παραγωγίζουμε ως προς $x$ και παίρνουμε
\[ f_x=y^2+h'(x)\Leftrightarrow y^2+e^x=y^2+h'(x)\Leftrightarrow h(x)=e^x \]
Συνεπώς η λύση δίνεται από τον τύπο
\[ f(x,y)=c\Leftrightarrow xy^2+e^x=c \]
όπου $c$ αυθαίρετη σταθερά. Αξίζει να αναφέρουμε ότι εξίσωση του παραδείγματος είναι μια διαφορική εξίσωση \eng{Bernoulli} με την οποία θα ασχοληθούμε στην παράγραφο \ref{sec:Bernoulli} όπου και θα την επιλύσουμε ξανά, χρησιμοποιώντας την αντίστοιχη μέθοδο.
\paragraph{Ολοκληρωτικός παράγοντας}
Όταν η συνθήκη \eqref{eq:exact:2} δεν ικανοποιείται, η εξίσωση \eqref{eq:exact:1} δεν είναι πλήρης. Υπάρχει ωστόσο η δυνατότητα να αναχθεί σε πλήρη με τη βοήθεια ενός \textbf{ολοκληρωτικού παράγοντα}. Ο πολλαπλασιασμός της εξίσωσης με τον παράγοντα αυτό έχει σκοπό να σχηματίζει ένα πλήρες διαφορικό στο πρώτο μέλος της. Όταν αυτό επιτευχθεί παίρνουμε μια πλήρη διαφορική εξίσωση την οποία επιλύουμε όπως είδαμε προηγουμένως. Αναζητούμε λοιπόν μία συνεχή συνάρτηση $\mu(x,y)$ η οποία καθιστά την εξίσωση
\begin{equation}
\mu(x,y) M(x,y)\d x+\mu(x,y)N(x,y)\d y=0
\end{equation}
πλήρη και κατά συνέπεια θα ισχύει η συνθήκη πληρότητας
\[ \diffp{(\mu M)}{y}=\diffp{(\mu N)}{x} \]
Από τη σχέση αυτή παίρνουμε
\begin{gather}
\diffp{(\mu M)}{y}=\diffp{(\mu N)}{x}\Leftrightarrow\nonumber\\
\diffp{\mu}{y}M+\mu\diffp{M}{y}=\diffp{\mu}{x}N+\mu\diffp{N}{x}\Leftrightarrow\nonumber\\
\diffp{\mu}{y}M-\diffp{\mu}{x}N=\mu\diffp{N}{x}-\mu\diffp{M}{y}\Leftrightarrow\nonumber\\
\frac{\mu_x N-\mu_y M}{\mu}=M_y-N_x\label{eq:exact:7}
\end{gather}
Η τελευταία σχέση αποτελεί μια μερική διαφορική εξίσωση ως προς τη ζητούμενη συνάρτηση $\mu$ και η επίλυσή της έχει αυξημένο βαθμό δυσκολίας σε σχέση με την αρχική μας εξίσωση. Για το λόγο αυτό θα εξετάσουμε μόνο κάποιες ειδικές περιπτώσεις για τη συνάρτηση $\mu(x,y)$ ώστε η \eqref{eq:exact:7} να μας δώσει κάποια γενική λύση.
\begin{bAlist}
\item \bmath{$\mu=\mu(x)$}\\
Αν η ζητούμενη συνάρτηση $\mu$ εξαρτάται μόνο από τη μεταβλητή $x$ δηλαδή $\mu=\mu(x)$ τότε και τα δύο μέλη της \eqref{eq:exact:7} αποτελούν συναρτήσεις ως προς $x$ ενώ παίρνουμε
\[ \mu_x=\mu'\ \textrm{και}\ \mu_y=0 \]
οπότε η \eqref{eq:exact:7} απλοποιείται, ανάγεται σε συνήθη διαφορική εξίσωση και γίνεται
\begin{gather*}
\frac{\mu'}{\mu}=\frac{M_y-N_x}{N}\Leftrightarrow
\ln{\mu}=\int{\frac{M_y-N_x}{N}\d x}\Leftrightarrow\\
\mu(x)=e^{\dintt{P(x)\d x}}
\end{gather*}
όπου $P(x)=\frac{M_y-N_x}{N}$. Καταλήγουμε έτσι στο εξής συμπέρασμα: Εάν η παράσταση $\frac{M_y-N_x}{N}$ είναι συνάρτηση μόνο του $x$ τότε διαφορική εξίσωση \eqref{eq:exact:1} έχει ολοκληρωτικό παράγοντα τη συνάρτηση
\begin{equation}
\mu(x)=e^{\dintt{P(x)\d x}}
\end{equation}
\end{bAlist}
\begin{Paradeigma}{Ολοκληρωτικός παράγοντας $\mu=\mu(x)$}
Ελέγξτε αν η ακόλουθη διαφορική εξίσωση 
\[\left(x^2+3xy\right)\d x+\left(\frac{y}{x}+x^2\right)\d y=0\]
είναι πλήρης. Αν όχι, βρείτε τη γενική λύση της με τη βοήθεια ενός ολοκληρωτικού παράγοντα.
\end{Paradeigma}
Έχουμε $M(x,y)=x^2+3xy$ και $N(x,y)=\frac{y}{x}+x^2$. Ελέγχουμε την ισχύ της συνθήκης
\[ \diffp{M}{y}=3x\neq -\frac{y}{x^2}+2x=\diffp{N}{x} \]
και συμπεραίνουμε πως δεν είναι πλήρης. Για την εύρεση κατάλληλου ολοκληρωτικού παράγοντα, εξετάζουμε αν η παράσταση $\frac{M_y-N_x}{N}$ μας δίνει συνάρτηση μόνο ως προς $x$. Πράγματι έχουμε
\[ \frac{M_y-N_x}{N}=\frac{3x+\frac{y}{x^2}-2x}{\frac{y}{x}+x^2}=\frac{\frac{y}{x^2}+x}{x\left(\frac{y}{x^2}+x\right)}=\frac{1}{x} \]
άρα ο ολοκληρωτικός παράγοντας έχει τη μορφή \eqref{eq:exact:7} και είναι
\[ \mu(x)=e^{\dintt{\frac{1}{x}\d x}}=e^{\ln{x}}=x \]
Πολλαπλασιάζουμε στη συνέχεια την αρχική εξίσωση με $\mu(x)=x$
\[ \left(x^3+3x^2y\right)\d x+\left(y+x^3\right)\d y=0 \]
και ελέγχουμε εκ νέου τη συνθήκη πληρότητας:
\[ \diffp{M}{y}=\diffp{\left(x^3+3x^2y\right)}{y}=3x^2=\diffp{\left(y+x^3\right)}{x}=\diffp{N}{x} \]
\begin{marginfigure}[0mm]
\begin{tikzpicture}
\begin{axis}[width=6.8cm,height=8cm,
xmin=-4,xmax=4,
ymin=-5,ymax=5,
xtick={-4,-3,...,4},
ytick={-5,-4,...,5},
xlabel={\footnotesize $ x $},
ylabel={\footnotesize $ y $},
belh ar,aks_on,
restrict y to domain=-20:20,
grid=both,
grid style={line width=.1pt, draw=gray!10},
major grid style={line width=.2pt,draw=gray!50},
minor tick num=4,
legend style={draw=none, fill opacity=0.7,text opacity=0.7,font=\scriptsize},
legend pos=north east,
legend image post style={scale=0.5}]
\begin{scope}
\foreach [evaluate=\c as \percent using 80-20*\c] \c in {-1,0,1,2,3,4}
{\edef\temp{\noexpand%
\addplot +[grafikh parastash,domain=-3:3,maincolor!\percent!secondarycolor,no markers,
      raw gnuplot,
      thick,
      empty line = jump] gnuplot {
      set contour base;
      set cntrparam levels discrete 0.003;
      unset surface;
      set view map;
      set isosamples 500;
      set samples 200;
      splot x^4/4+x^3*y+y^2/2-\c;
    };}\temp
}
\end{scope}
\node at (axis cs:-.5,-0.5) {\footnotesize$O$};
\legend{$c=-1$,$c=0$,$c=1$,$c=2$,$c=3$,$c-4$}
\end{axis}
\end{tikzpicture}\captionof{figure}{Ολοκληρωτικές καμπύλες της $ \left(x^3+3x^2y\right)\d x+\left(y+x^3\right)\d y=0$}
\end{marginfigure}
Η νέα εξίσωση είναι πλήρης. Αναζητούμε έτσι μια συνάρτηση $f(x,y)=c$ που επαληθεύει την εξίσωση και ισχύει γι αυτήν $f_x=M$ και $f_y=N$. Ολοκληρώνουμε την πρώτη σχέση ως προς $x$ και τη δεύτερη ως προς $y$ και παίρνουμε
\begin{align*}
&f(x,y)=\int{\left(x^3+3x^2y\right)\d x}+g(y)=\frac{x^4}{4}+x^3y+g(y)\ \text{και}\\
&f(x,y)=\int{\left(y+x^3\right)\d y}+h(x)=\frac{y^2}{2}+x^3y+h(x)
\end{align*}
Προκειμένου να ταυτίζονται οι δύο τελευταίες σχέσεις παίρνουμε
\[ g(y)=\frac{y^2}{2}\ \text{ και }\ h(x)=\frac{x^4}{4} \]
επομένως η γενική λύση της εξίσωσης θα δίνεται από την ακόλουθη πεπλεγμένη συνάρτηση
\[ f(x,y)=c\Rightarrow \frac{x^4}{4}+x^3y+\frac{y^2}{2}=c \]
\begin{bAlist}[resume]
\item \bmath{$\mu=\mu(y)$}\\
Ομοίως με την προηγούμενη περίπτωση, αν ο ολοκληρωτικός παράγοντας εξαρτάται μόνο από τη μεταβλητή $y$ τότε η \eqref{eq:exact:7} γίνεται Σ.Δ.Ε. ως προς $y$ και είναι
\begin{gather*}
-\frac{\mu'}{\mu}=\frac{M_y-N_x}{M}\Leftrightarrow
\ln{\mu}=\int{\frac{N_x-M_y}{M}\d y}\Leftrightarrow\\
\mu(y)=e^{\dintt{Q(y)\d y}}
\end{gather*}
όπου $Q(y)=\frac{N_x-My}{M}$. Το αντίστοιχο συμπέρασμα για το ζητούμενο ολοκληρωτικό παράγοντα είναι: Εάν η παράσταση $\frac{N_x-M_y}{M}$ είναι συνάρτηση μόνο του $y$ τότε η διαφορική εξίσωση \eqref{eq:exact:1} έχει ολοκληρωτικό παράγοντα τη συνάρτηση
\begin{equation}
\mu(y)=e^{\dintt{Q(y)\d y}}
\end{equation}
\end{bAlist}
\begin{Paradeigma}{Ολοκληρωτικός παράγοντας $\mu=\mu(y)$}
\[ \left(xe^{-y}+cos{x}\right)\d x+\sin{x}\d y=0 \]
\end{Paradeigma}
Στην εξίσωση έχουμε $M(x,y)=xe^{-y}+\cos{x}$ και $N(x,y)=\sin{x}$. Παρατηρούμε ότι
\[ \diffp{M}{y}=-xe^{-y}\neq \cos{x}=\diffp{N}{x} \]
άρα η εξίσωση δεν είναι πλήρης. Η παράσταση $\frac{N_x-M_y}{M}$ θα μας δώσει
\[ \frac{N_x-M_y}{M}=\frac{\cos{x}+xe^{-y}}{xe^{-y}+\cos{x}}=1 \]
επομένως ο ολοκληρωτικός παράγοντας θα είναι $e^{\int{\d y}}=e^y$. Πολλαπλασιάζοντας η εξίσωση γίνεται
\[\left(x+e^y\cos{x}\right)\d x+e^y\sin{x}\d y=0\]
στην οποία πλέον έχουμε $M(x,y)=x+e^y\cos{x}$ και $N(x,y)=e^y\sin{x}$. Εύκολα βλέπουμε πως είναι πλήρης καθώς ικανοποιείται η \eqref{eq:exact:2} και έτσι η γενική λύση της θα δίνεται από τη σχέση $f(x,y)=c$ με $f_x=M$ και $f_y=N$. Από την πρώτη σχέση παίρνουμε
\[f_x=M\Leftrightarrow f(x,y)=\int{(x+e^y\cos{x})\d x}+g(y)=\frac{x^2}{2}+e^y\sin{x}+g(y)\]
Με άμεσο προσδιορισμό της συνάρτησης $g(y)$ από τον τύπο \eqref{eq:exact:10} παίρνουμε
\begin{align*}
g(y)&=\int{\left(N-\int{N_x\d x}\right)\d y}=\\&=\int{\left(\sin{x}-\int{\cos{x}\d x}\right)\d y}=\int{0\d y}=c_1
\end{align*}
Η γενική λύση θα δίνεται από τον τύπο
\[ f(x,y)=c\Leftrightarrow\frac{x^2}{2}+e^y\sin{x}=c \]
\begin{bAlist}[resume]
\item \bmath{$\mu=\mu(x+y)$}
\item \bmath{$\mu=\mu(xy)$}
\item \bmath{$\mu=\mu\left(\frac{y}{x}\right)$}
\end{bAlist}
\section{Ομογενείς διαφορικές εξισώσεις}
\lipsum[1]
\section{Γραμμικές διαφορικές εξισώσεις}
Όπως αναφέραμε στην εισαγωγή, οι γραμμικές Σ.Δ.Ε έχουν πολυωνυμική μορφή, με συντελεστές συνεχείς συναρτήσεις $a_i(x),\ x\in\varDelta\subseteq\mathbb{R}$. Η γενική μορφή μιας 1ης τάξης γραμμικής διαφορικής εξίσωσης θα είναι
\[ a_1(x)y'+a_0(x)y=\beta(x) \]
όπου $a_1,a_0$ και $\beta$ συνεχείς συναρτήσεις σε ένα διάστημα $\varDelta$. Για κάθε $x\in\varDelta$ για το οποίο $a_1(x)\neq 0$, η προηγούμενη εξίσωση παίρνει την λυμένη μορφή της όπως βλέπουμε στον επόμενο ορισμό.
\begin{Orismos}{Γραμμική εξίσωση 1ης τάξης}
Μια διαφορική εξίσωση ονομάζεται γραμμική 1ης τάξης, αν έχει την
μορφή
\begin{equation}\label{eq:linode1:1}
y'+p(x)y=q(x)\ ,\ x\in\varDelta
\end{equation}
όπου $p(x)$ και $q(x)$ συνεχείς συναρτήσεις στο διάστημα $\varDelta\subseteq\mathbb{R}$.
\end{Orismos}
\begin{marginfigure}[0mm]
\begin{Parathrhsh}{5.2cm}
Μια ομογενής γραμμική διαφορική εξίσωση 1ης τάξης έχει τετριμμένη λύση την $y(x)=0,x\in\varDelta$.
\end{Parathrhsh}
\end{marginfigure}
Θέσαμε $p(x)=\frac{a_0(x)}{a_1(x)}$ και $q(x)=\frac{\beta(x)}{a_1(x)}$. Στην περίπτωση όπου $q(x)\equiv0$ έχουμε την αντίστοιχη \textbf{ομογενή} γραμμική διαφορική εξίσωσης της \eqref{eq:linode1:1}. 
\paragraph{Βασική μέθοδος επίλυσης}
Ας δούμε λίγο την κατασκευή της λύσης της ομογενούς εξίσωσης. Θα σχηματίσουμε παράγωγο γινομένου στο 1ο μέλος της, χρησιμοποιώντας τον παράγοντα $e^{\int{p(x)}\d x}$ με $a\in\varDelta$. Πολλαπλασιάζοντας έχουμε
\begin{gather*}
e^{\dintt{p(x)}\d x}y'+e^{\dintt{p(x)}\d x}p(x)y=0\Leftrightarrow\\
\left(e^{\dintt{p(x)}\d x}y\right)'=0\Leftrightarrow e^{\dintt{p(x)}\d x}y=c
\end{gather*}
όπου $c$ μια σταθερά. Έτσι, σύμφωνα με την τελευταία σχέση η γενική λύση της \eqref{eq:linode1:1} θα δίνεται από τον τύπο:
\begin{equation}\label{eq:linode1:2}
y(x)=ce^{-\dintt{p(x)}\d x}
\end{equation}
Χρησιμοποιούμε την ίδια τεχνική ώστε να κατασκευάσουμε και τη λύση της μη ομογενούς διαφορικής εξίσωσης \eqref{eq:linode1:1}. Ο πολλαπλασιασμός με τον ίδιο ολοκληρωτικό παράγοντα θα μας δώσει
\begin{gather}
e^{\dintt{p(x)}\d x}y'+e^{\dintt{p(x)}\d x}p(x)y=q(x)e^{\dintt{p(x)}\d x}\Leftrightarrow\nonumber\\
\left(e^{\dintt{p(x)}\d x}y\right)'=q(x)e^{\dintt{p(x)}\d x}\Leftrightarrow\nonumber\\e^{\dintt{p(x)}\d x}y=\int{q(x)e^{\dintt{p(x)}\d x}\d x}+c\Leftrightarrow\nonumber\\
y(x)=e^{-\dintt{p(x)}\d x}\left[c+\int{q(x)e^{\dintt{p(x)}\d x}\d x}\right]\label{eq:linode1:3}
\end{gather}
\begin{Paradeigma}{Ομογενής γραμμική εξίσωση 1ης τάξης}
Να βρεθεί η γενική λύση της γραμμικής διαφορικής εξίσωσης
\[xy'-y = 0\]
και στη συνέχεια, με αρχική συνθήκη της $y(1)=-2$ να βρεθεί η ειδική λύση.
\end{Paradeigma} 
\begin{marginfigure}[-0.5cm]
\begin{tikzpicture}
\begin{axis}[width=6.5cm,height=6.5cm,
xmin=-7,xmax=7,
ymin=-7,ymax=7,
xtick={-6,-4,...,6},
ytick={-6,-4,...,6},
xlabel={\footnotesize $ x $},
ylabel={\footnotesize $ y $},
belh ar,aks_on,
restrict y to domain=-20:20,
grid=both,
grid style={line width=.1pt, draw=gray!10},
major grid style={line width=.2pt,draw=gray!50},
minor tick num=4,
legend style={draw=none, fill opacity=0.7,text opacity=0.7,font=\scriptsize},
legend pos=north west,
legend image post style={scale=0.5}]
\begin{scope}
\clip (axis cs:-7,-7) rectangle (axis cs:7,7);
\addplot[grafikh parastash,domain=-7:7,maincolor]{x};
\addplot[grafikh parastash,domain=-7:7,thirdcolor]{2*x};
\addplot[grafikh parastash,domain=-7:7,secondarycolor]{x/2};
\addplot[grafikh parastash,domain=-7:7,maincolor]{-x};
\addplot[grafikh parastash,domain=-7:7,thirdcolor]{-2*x};
\addplot[grafikh parastash,domain=-7:7,secondarycolor]{-x/2};
\end{scope}
\node at (axis cs:-.5,-0.5) {\footnotesize$O$};
\legend{${c=1,-1}$,${c=2,-2}$,${c=\frac{1}{2},-\frac{1}{2}}$};
\end{axis}
\end{tikzpicture}\captionof{figure}{Λύσεις της εξίσωσης \eqref{example:linode1:1}}\label{plot:linode1:1}
\end{marginfigure}
Η αρχική εξίσωση γράφεται στη μορφή
\begin{equation}\label{example:linode1:1}
y'-\frac{1}{x}y = 0
\end{equation}
για κάθε $x\neq 0$. Έχοντας λοιπόν $p(x)=-\frac{1}{x}$ και $q(x)=0$, η γενική λύση \eqref{eq:linode1:2} της εξίσωσης θα δίνεται από τον τύπο
\[ y(x)=ce^{-\dintt{-\frac{1}{x}\d x}}=ce^{\ln{x}}=cx \]
όπου $c$ μια αυθαίρετη σταθερά. Στο \ref{plot:linode1:1} δείχνουμε τα γραφήματα κάποιων λύσεων της  \eqref{eq:linode1:1} που αντιστοιχούν σε διάφορες τιμές της σταθεράς $c$. Σύμφωνα τώρα με την αρχική συνθήκη έχουμε
\[ y(1)=-2\Rightarrow c=-2 \]
έτσι η λύση του Π.Α.Τ. θα είναι η $y(x)=-2x$.\\\\
\begin{Paradeigma}{Μη ομογενής γραμμική εξίσωση 1ης τάξης}
Βρείτε τη γενική λύση της ακόλουθης γραμμικής διαφορικής εξίσωσης
\begin{equation}\label{example:linode1:2}
x^2y'+xy=1,\ x>0
\end{equation}
\end{Paradeigma}
\begin{marginfigure}[0cm]
\begin{tikzpicture}
\begin{axis}[width=6.5cm,height=7cm,
xmin=-1,xmax=7,
ymin=-7,ymax=9,
xtick={-1,0,...,7},
ytick={-7,-5,...,9},
xlabel={\footnotesize $ x $},
ylabel={\footnotesize $ y $},
belh ar,aks_on,
grid=both,
grid style={line width=.1pt, draw=gray!10},
major grid style={line width=.2pt,draw=gray!50},
minor tick num=4,
legend style={draw=none, fill opacity=0.7,text opacity=0.7,font=\scriptsize},
legend pos=north east,
legend image post style={scale=0.5},
legend columns=2
]
\begin{scope}
\clip (axis cs:0,-7) rectangle (axis cs:7,8);
\addplot[grafikh parastash,domain=-0.01:7,secondarycolor!70!maincolor]{(-1+ln(x))/x};
\addplot[grafikh parastash,domain=-0.01:7,secondarycolor]{(ln(x))/x};
\addplot[grafikh parastash,domain=-0.01:7,maincolor!70!secondarycolor]{(1+ln(x))/x};
\addplot[grafikh parastash,domain=-0.01:7,maincolor]{(2+ln(x))/x};
\addplot[grafikh parastash,domain=-0.01:7,maincolor!80!black]{(3+ln(x))/x};
\end{scope}
\node at (axis cs:3,-3) {$y(x)=\dfrac{c+\ln{x}}{x}$};
\node at (axis cs:-.5,-0.5) {\footnotesize$O$};\\
\legend{$c=-1$,$c=0$,$c=1$,$c=2$,$c=3$}
\end{axis}
\end{tikzpicture}\captionof{figure}{Ολοκληρωτικές καμπύλες της εξίσωσης \eqref{example:linode1:2}}
\end{marginfigure}
Η εξίσωση, για κάθε $x>0$ γράφεται στη μορφή
\[ x^2y'+xy=1\Leftrightarrow y'+\frac{1}{x}y=\frac{1}{x^2} \]
οπότε και έχουμε $p(x)=\frac{1}{x}$ και $q(x)=\frac{1}{x^2}$. Η γενική λύση αυτής θα δίνεται από τον παρακάτω τύπο
\begin{align*}
y(x)&=e^{-\dintt{\frac{1}{x}\d x}}\left[c+\int\frac{1}{x^2}e^{\dintt{\frac{1}{x}\d x}}\d x\right]=\\&=e^{-\ln{x}}\left(c+\int\frac{1}{x^2}e^{\ln{x}}\d x\right)=\\&=\frac{1}{x}\left(c+\int\frac{1}{x}\d x\right)=\\&=\frac{c+\ln{x}}{x}
\end{align*}
όπου $c$ μια αυθαίρετη σταθερά. Στο διπλανό σχήμα βλέπουμε τις ολοκληρωτικές καμπύλες μερικών ειδικών λύσεων της εξίσωσης.
\paragraph{Ομογενής γραμμική - Μέθοδος χωριζομένων μεταβλητών}
Εύκολα παρατηρούμε ότι μια ομογενής γραμμική διαφορική εξίσωση 1ης τάξης, μετατρέπεται σε εξίσωση χωριζομένων μεταβλητών. Εάν $y$ είναι μια μη τετριμμένη λύση της εξίσωσης $y'+p(x)y=0$ τότε έχουμε τα εξής:
\begin{align*}
&y'+p(x)y=0\Rightarrow\\\Rightarrow &y'=-p(x)y\Rightarrow\\\Rightarrow &\diff{y}{x}=-p(x)y\Rightarrow \frac{\d y}{y}=-p(x)\d x\Rightarrow
\end{align*}
Έτσι σύμφωνα με όσα αναπτύξαμε στην παράγραφο των διαφορικών εξισώσεων χωριζομένων μεταβλητών, από την τελευταία σχέση παίρνουμε
\begin{align*}
\ln{|y|}=-\int{p(x)\d x}+c&\Rightarrow |y|=ce^{-\dintt{p(x)\d x}}\Rightarrow\\
&\xRightarrow[y\neq 0]{y\in C(\varDelta)}y=\pm ce^{-\dintt{p(x)\d x}}
\end{align*}
για κάθε $x\in\varDelta$. Ας δούμε τώρα στο επόμενο παράδειγμα τον εναλλακτικό αυτό τρόπο επίλυσης.\\\\
\begin{Paradeigma}{Ομογενής εξίσωση - Επίλυση με ολοκλήρωση}
Ζητείται η γενική λύση της γραμμικής διαφορικής εξίσωσης
\[ y'+2xy=0 \]
όπου $y>0$ και $x\in\mathbb{R}$.
\end{Paradeigma}
\begin{marginfigure}[-10mm]
\begin{tikzpicture}
\begin{axis}[width=6.5cm,height=6.5cm,
xmin=-3,xmax=3,
ymin=-1,ymax=2.5,
xtick={-3,-2,...,3},
ytick={-1,-0.5,...,2.5},
xlabel={\footnotesize $ x $},
ylabel={\footnotesize $ y $},
belh ar,aks_on,
grid=both,
grid style={line width=.1pt, draw=gray!10},
major grid style={line width=.2pt,draw=gray!50},
minor tick num=4,
legend style={draw=none, fill opacity=0.7,text opacity=0.7,font=\scriptsize},
legend pos=north west,
legend image post style={scale=0.5}]
\begin{scope}
\clip (axis cs:-3,-3) rectangle (axis cs:3,3);
\addplot[grafikh parastash,domain=-2.5:2.5,maincolor]{2*exp(-x^2)};
\addplot[grafikh parastash,domain=-2.5:2.5,secondarycolor]{exp(-x^2)};
\addplot[grafikh parastash,domain=-2.5:2.5,thirdcolor]{exp(-x^2)/2};
\end{scope}
\node[fill=white,opacity=0.7,text opacity=1] at (axis cs:1.7,-0.7) {$y(x)=ce^{-x^2}$};
\node at (axis cs:-.5,-0.25) {\footnotesize$O$};
\legend{${c=2}$,${c=1}$,${c=\frac{1}{2}}$};
\end{axis}
\end{tikzpicture}\captionof{figure}{Μερικές λύσεις της εξίσωσης $y'+2xy=0$.}
\end{marginfigure}
Καθώς έχουμε $y>0$ τότε αναζητούμε μη τετριμμένες λύσης της εξίσωσης. Με διαχωρισμό των μεταβλητών έχουμε:
\begin{gather*}
y'+2xy=0\Leftrightarrow
y'=-2xy\Leftrightarrow\\
\diff{y}{x}=-2xy\Leftrightarrow \int{\dfrac{\d y}{y}}=\int{-2x\d x}+c_1\Leftrightarrow\\
\ln{y}=-x^2+c_1\Leftrightarrow y=e^{-x^2+c_1}\Leftrightarrow\\
y=ce^{-x^2}
\end{gather*}
όπου $c_1,c=e^{c_1}$ αυθαίρετες σταθερές.
\paragraph{Πρόβλημα αρχικών τιμών}
Στην περίπτωση ενός προβλήματος αρχικών τιμών, με αρχική συνθήκη $y(a)=y_0$ εύκολα βλέπουμε ότι $c=y_0$, ο ολοκληρωτικός παράγοντας θα έχει τη μορφή $e^{\int_{a}^{x}{p(t)\d t}}$ και άρα οι τύποι \eqref{eq:linode1:2} και \eqref{eq:linode1:3} για τη λύση του γράφονται
\begin{equation}\label{eq:linode1:4}
y(x)=y_0e^{-\dintt_{\!\!\!a}^{x}{p(t)}\d t}\ ,\ y(x)=e^{-\dintt_{\!\!\!a}^{x}{p(t)}\d t}\left[y_0+\int_{a}^{x}{q(t)e^{\dintt_{\!\!\!a}^{t}{p(s)}\d s}\d t}\right]
\end{equation}
\begin{Paradeigma}{Μη ομογενής γραμμική εξίσωση 1ης τάξης - Π.Α.Τ.}
Βρείτε τη λύση του παρακάτω προβλήματος αρχικών τιμών.
\[ \ccases{y'+y=x^2\\y(0)=1} \]
\vspace{-5mm}
\end{Paradeigma}
\begin{marginfigure}[-0.5cm]
\begin{tikzpicture}
\begin{axis}[width=6cm,height=7.5cm,
xmin=-3,xmax=3,
ymin=-3,ymax=5,
xtick={-3,-2,...,3},
ytick={-3,-2,...,5},
xlabel={\footnotesize $ x $},
ylabel={\footnotesize $ y $},
belh ar,aks_on,
grid=both,
grid style={line width=.1pt, draw=gray!10},
major grid style={line width=.2pt,draw=gray!50},
minor tick num=4]
\begin{scope}
\clip (axis cs:-3,-3) rectangle (axis cs:4,5);
\addplot[grafikh parastash,domain=-3:3,secondarycolor]{-exp(-x)+x^2-2*x+2};
\end{scope}
\node[fill=white,inner sep=.2mm,opacity=0.7,text opacity=1] at (axis cs:0,-1.5) {\footnotesize$y(x)=e^{-x}+x^2-2x+2$};
\node at (axis cs:-.5,-0.5) {\footnotesize$O$};
\node[xshift=0.7cm,yshift=1cm,labelbox={secondarycolor}](A) at (axis cs:0,1) { Αρχική\\συνθήκη};
\draw[-latex] (A.south)--(0,1);
\fill[secondarycolor] (axis cs:0,1) circle(0.1);
\end{axis}
\end{tikzpicture}\captionof{figure}{Λύση του προβλήματος αρχικών τιμών.}
\end{marginfigure}
Η εξίσωση έχει ήδη τη μορφή \eqref{eq:linode1:1} με $p(x)=1$ και $q(x)=x^2$. Η λύση της εξίσωσης, σύμφωνα με τον τύπο \eqref{eq:linode1:4}β είναι:
\begin{align*}
y(x)&=e^{-\dintt_{\!\!\!0}^{x}{\d t}}\left[y(0)+\int_{0}^{x}t^2e^{\dintt_{\!\!\!0}^{t}{\d s}}\d t\right]=\\&=e^{-x}\left(1+\int_{0}^{x}t^2e^t\d t\right)=\\&=e^{-x}\left[1+\left(x^2-2x+2\right)e^x-2\right]\textbf=\\&=-e^{-x}+x^2-2x+2
\end{align*}
\section{Εξισώσεις \eng{Bernoulli - Ricatti}}\label{sec:Bernoulli}
Στην ενότητα αυτή θα μελετήσουμε δύο ειδικές, μη γραμμικές διαφορικές εξισώσεις 1ης τάξης, τις εξισώσεις \eng{Bernoulli} και \eng{Ricatti}. Οι ειδικές αυτές μορφές σχετίζονται άμεσα με τις γραμμικές διαφορικές εξισώσεις 1ης τάξης καθώς η μεν \eng{Ricatti} μετασχηματίζεται σε \eng{Bernoulli}, η δε \eng{Bernoulli} με τη σειρά της μετατρέπεται σε γραμμική 1ης τάξης. 
\begin{Orismos}{Διαφορική εξίσωση \eng{Bernoulli}}
Κάθε διαφορική εξίσωση 1ης τάξης της μορφής
\begin{equation}\label{eq:Bernoulli}
y'+p(x)y=q(x)y^{\sigma}(x)
\end{equation}
με $x\in\varDelta\subseteq\mathbb{R}$ και $\sigma\in\mathbb{R}$, ονομάζεται εξίσωση \eng{Bernoulli}.
\end{Orismos}
\begin{marginfigure}[0mm]
\begin{Parathrhsh}{5.2cm}
Μια διαφορική εξίσωση \eng{Bernoulli} έχει τετριμμένη λύση την $y(x)=0,\ x\in\varDelta$.
\end{Parathrhsh}
\end{marginfigure}
Στην ειδική περίπτωση όπου $\sigma=1$ η \eqref{eq:Bernoulli} είναι μια ομογενής γραμμική διαφορική εξίσωση, ενώ για $\sigma=0$ παίρνουμε την αντίστοιχη μη γραμμική, τις οποίες μελετήσαμε στην προηγούμενη παράγραφο. Η αναγωγή της σε γραμμική 1ης τάξης επιτυγχάνεται με τον ακόλουθο μετασχηματισμό
\begin{equation}\label{eq:bernoulli_to_1st}
z=y^{1-\sigma}
\end{equation}
Παραγωγίζοντας παίρνουμε
\[ z'=(1-\sigma)y^{-\sigma}y'\Rightarrow y^{-\sigma}y'=\frac{z'}{1-\sigma} \]
Πολλαπλασιάζουμε στη συνέχεια την \eqref{eq:Bernoulli} με $y^{-\sigma}$ και αντικαθιστώντας τις σχέσεις αυτές προκύπτει
\begin{gather*}
y^{-\sigma}y'+p(x)y^{1-\sigma}=q(x)\Leftrightarrow
\frac{z'}{1-\sigma}+p(x)z=q(x)\Leftrightarrow\\
z'+(1-\sigma)p(x)z=(1-\sigma)q(x)
\end{gather*}
και καταλήξαμε στη ζητούμενη γραμμική. Επιλύουμε τη γραμμική και χρησιμοποιώντας ξανά τον μετασχηματισμό \eqref{eq:bernoulli_to_1st} οδηγούμαστε στην λύση $y(x)$ της αρχικής εξίσωσης.\\\\
\begin{Paradeigma}{Εξίσωση \eng{Bernoulli}}
Να λύσετε την εξίσωση
\[ y'-\frac{1}{x}y=y^3\ \text{ , }\ x<0 \]
\end{Paradeigma}
Η αρχική διαφορική εξίσωση είναι \eng{Bernoulli} με $\sigma=3,p(x)=-\frac{1}{x}$ και $q(x)=1$. Χρησιμοποιούμε το μετασχηματισμό 
\[ z=y^{1-3}=y^{-2} \]
από τον οποίο παραγωγίζοντας παίρνουμε
\[z'=-2y^{-3}y'\Rightarrow y^{-3}y'=\frac{z'}{2}\]
Πολλαπλασιάζουμε στη συνέχεια την αρχική εξίσωση με $y^{-3}$ και αντικαθιστούμε τα παραπάνω οπότε παίρνουμε
\begin{gather*}
y^{-3}y'-\frac{1}{x}y^{-2}=1\Rightarrow \frac{z'}{2}-\frac{1}{x}z=1\Rightarrow\\
z'-\frac{2}{x}z=2
\end{gather*}
\begin{marginfigure}[0mm]
\begin{tikzpicture}
\begin{axis}[width=7cm,height=7cm,
xmin=-7,xmax=1,
ymin=-1,ymax=7,
xtick={-7,-6,...,1},
ytick={-1,0,...,7},
xlabel={\footnotesize $ x $},
ylabel={\footnotesize $ y $},
belh ar,aks_on,
restrict y to domain=-20:20,
grid=both,
grid style={line width=.1pt, draw=gray!10},
major grid style={line width=.2pt,draw=gray!50},
minor tick num=4,
legend style={draw=none, fill opacity=0.7,text opacity=0.7,font=\scriptsize},
legend pos=north west,
legend image post style={scale=0.5}]
\begin{scope}
%\clip (axis cs:-7,-7) rectangle (axis cs:7,7);
\addplot[grafikh parastash,domain=-7:1,maincolor]{1/sqrt(x^2-2*x)};
\addplot[grafikh parastash,domain=-4:1,maincolor!50!secondarycolor]{1/sqrt(-0.5*x^2-2*x)};
\addplot[grafikh parastash,domain=-2:1,secondarycolor]{1/sqrt(-x^2-2*x)};
\end{scope}
\node at (axis cs:-.5,-0.5) {\footnotesize$O$};
\legend{$c=1$,$c=-0.5$,$c=-1$}
\end{axis}
\end{tikzpicture}\captionof{figure}{Λύσεις της εξίσωσης \eng{Bernoulli} $y'-\frac{1}{x}y=y^3$.}
\end{marginfigure}
Η τελευταία είναι γραμμική 1ης τάξης και σύμφωνα με τη σχέση \eqref{eq:linode1:3} η γενική της λύση θα δίνεται από τον τύπο
\begin{align*}
z(x)&=e^{\dintt{\frac{2}{x}}\d x}\left[c+\int{2e^{-\dintt{\frac{2}{x}}\d x}\d x}\right]=\\
&=e^{2\ln{x}}\left[c+\int{2e^{-2\ln{x}}\d x}\right]=\\
&=x^2\left(c+\int{\frac{2}{x^2}\d x}\right)=\\
&=x^2\left(c-\frac{2}{x}\right)=cx^2-2x
\end{align*}
Τέλος, χρησιμοποιώντας ξανά τον αρχικό μετασχηματισμό θα έχουμε
\begin{gather*}
z=cx^2-2x\Rightarrow y^{-2}=cx^2-2x\Rightarrow\\
y(x)=\frac{1}{\sqrt{cx^2-2x}}
\end{gather*}
όπου $c$ αυθαίρετη σταθερά.\\\\
\begin{Paradeigma}{}
\[y'=-\frac{y^2+e^x}{2xy}\]
\end{Paradeigma}
\begin{Orismos}{Διαφορική εξίσωση \eng{Ricatti}}
\begin{equation}\label{eq:Ricatti}
y'=a(x)y^2+\beta(x)y+\gamma(x)
\end{equation}
\end{Orismos}
Παρατηρούμε ότι αν $\gamma(x)\equiv 0$, η \eqref{eq:Ricatti} αποτελεί διαφορική εξίσωση \eng{Bernoulli} με $\sigma=2$ ενώ για $\beta(x)\equiv 0$ έχουμε μια γραμμική διαφορική εξίσωση 1ης τάξης. Η βασική μέθοδος επίλυσης μιας διαφορικής εξίσωσης \eng{Ricatti} θέλει να γνωρίζουμε μία μερική λύση της εξίσωσης, έστω $y_1(x)$. Με τη χρήση αυτής και το μετασχηματισμό
\[ y=y_1+w \]
η \eqref{eq:Ricatti} μετατρέπεται σε διαφορική εξίσωση \eng{Bernoulli}. Παραγωγίζοντας έχουμε $y'=y_1'+w'$ και με αντικατάσταση παίρνουμε
\begin{gather}
y_1'+w'=a(y_1+w)^2+\beta(y_1+w)+\gamma\Leftrightarrow\nonumber\\
y_1'+w'=ay_1^2+2ay_1w+aw^2+\beta y_1+\beta w+\gamma\Leftrightarrow\nonumber\\
\left(ay_1^2+\beta y+\gamma\right)+w'=aw^2+2ay_1w+\beta w+\left(ay_1^2+\beta y+\gamma\right)\Leftrightarrow\nonumber\\
w'-\left(2ay_1+\beta\right)w=aw^2\label{eq:Ricatti:1}
\end{gather}
Η \eng{Bernoulli} στην οποία καταλήξαμε έχει $\sigma=2$. Σύμφωνα με όσα μελετήσαμε προηγουμένως, ο μετασχηματισμός
\[ z=w^{1-2}=\frac{1}{w} \]
θα μετατρέψει την \eqref{eq:Ricatti:1} σε γραμμική. Έχουμε $z'=-\frac{w'}{w^2}$ άρα
\begin{gather*}
w'-\left(2ay_1+\beta\right)w=aw^2\Leftrightarrow
\frac{w'}{w^2}-\left(2ay_1+\beta\right)\frac{1}{w}=a\Leftrightarrow\\
z'+\left(2ay_1+\beta\right)z=-a
\end{gather*}
Η τελευταία έχει γενική λύση την
\begin{equation}\label{eq:Ricatti:solution}
z(x)=e^{-\dintt{p(x)\d x}}\left(c-\int{a(x)e^{\dintt{p(x)\d x}}\d x}\right)
\end{equation}
όπου $p(x)=2a(x)y_1(x)+\beta(x)$. Αντικαθιστούμε ξανά στην τελευταία σχέση $z=\frac{1}{w}=\frac{1}{y-y_1}$ και οδηγούμαστε στην γενική λύση της \eqref{eq:Ricatti}. Παρατηρούμε ότι εάν συνδυάσουμε τους δύο μετασχηματισμούς μπορούμε να μεταβούμε κατευθείαν από την εξίσωση \eng{Ricatti} στην τελική γραμμική διαφορική εξίσωση και στη γενική λύση της, μέθοδο την οποία θα ακολουθήσουμε, με τον άμεσο μετασχηματισμό 
\[y=y_1+\frac{1}{z}\]
παρακάμπτοντας την ενδιάμεση \eqref{eq:Bernoulli}.\\\\
\begin{Paradeigma}{}
\[y'=xy^2-y+e^x\]
\end{Paradeigma}
Εάν όπως είδαμε θέσουμε $y=y_1+\frac{1}{z}$...
\lipsum[1-2]
\section{Περιοδικές εξισώσεις}
\lipsum[1-2]
\section{Ιδιάζουσες λύσεις}
\lipsum[1-2]
\section{Μέθοδος ολοκλήρωσης με παραγώγιση}
\lipsum[1-2]
\subsection{Εξίσωση \eng{D' Alambert}}
\lipsum[1-2]
\subsection{Εξίσωση \eng{Lagrange}}
\subsection{Εξίσωση \eng{Clairaut}}
\subsection{Νόμοι \eng{Kepler}}
\section{Αντικατάσταση}
\chapter{Διαφορικές εξισώσεις 2ης τάξης}
\section{Γραμμικές εξισώσεις με σταθερούς συντελεστές}
\section{Εξίσωση \eng{Euler}}
\section{Υποβιβασμός τάξης}
\section{Ολοκληρωτική καμπύλη}
\section{Ακριβείς διαφορικές εξισώσεις}
\section{Ομογενείς εξισώσεις}
\section{Θεωρήματα διαχωρισμού και σύγκρισης \eng{Sturm}}
\section{Μη ομογενείς εξισώσεις}
\section{Μέθοδος \eng{Lagrange}}
\section{Δυναμοσειρές}
\chapter{Γραμμικές διαφορικές εξισώσεις}
\section{Ομογενείς εξισώσεις}
\section{Γραμμική ανεξαρτησία - Ορίζουσα \eng{Wronski}}
\section{Βασικά σύνολα λύσεων}
\section{Υποβιβασμός τάξης}
\section{Μη ομογενείς εξισώσεις - Μερικές λύσεις}
\section{Μέθοδος μεταβολής σταθερών}
\section{Εξισώσεις με σταθερούς συντελεστές}
\section{Εξισώσεις με μεταβλητούς συντελεστές}
\section{Ομογενείς διαφορικές εξισώσεις και συζυγείς}
\section{Μέθοδος απροσδιόριστων συντελεστών}
\section{Μετασχηματισμός $Y'=gY$}
\section{Δυναμοσειρές}
\subsection{\eng{Taylor}}
\subsection{\eng{Mc Laurin}}
\subsection{\eng{Frobenius}}
\subsection{\eng{Fuchs}}
\section{Ειδικές συναρτήσεις}
\section{Μέθοδος μεταβολής σταθερών}
\section{Μέθοδος διαφορικών τελεστών}
\section{Μέθοδος προσδιορισμού συντελεστών}
\section{Προβλήματα αρχικών και συνοριακών τιμών}
\section{\eng{Sturm - Liouville}}
\chapter{Συστήματα διαφορικών εξισώσεων}
\section{Ομογενή γραμμικά συστήματα}
\section{Πίνακες λύσεων - Τύπος \eng{Jacobi}}
\section{Στοιχεία γραμμικής άλγεβρας - Ανάλυση πινάκων}
\section{Βασικοί πίνακες - Σύνολα λύσεων}
\section{Υποβιβασμός τάξης}
\section{Μη ομογενή γραμμικά συστήματα - Μερικές λύσεις}
\section{Ομογενή γραμμικά συστήματα με σταθερούς συντελεστές}
\section{Μέθοδος απαλοιφής}
\section{Ευστάθεια συστημάτων}
\section{Μέθοδος πινάκων}
\section{Πρώτα ολοκληρώματα}
\section{Γεωμετρικές ερμηνείες συστημάτων διαφορικών εξισώσεων}
\section{Διαφορικοί τελεστές}
\section{Μέθοδος εκθετικής αντικατάστασης}
\section{Μέθοδος κανονικών συντεταγμένων}
\section{Μέθοδος τελεστή εξέλιξης}
\end{document}
