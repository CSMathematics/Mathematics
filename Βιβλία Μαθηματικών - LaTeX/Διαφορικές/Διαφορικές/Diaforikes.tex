\documentclass[a4paper,twoside,11pt]{book}
\usepackage[amsbb,subscriptcorrection,zswash,mtpccal,mtphrb,mtpscr]{mtpro2}
\usepackage[no-math,cm-default]{fontspec}
\usepackage{xunicode}
\usepackage{xgreek}
\defaultfontfeatures{Mapping=tex-text,Scale=MatchLowercase}
\setmainfont[Mapping=tex-text,Numbers=Lining,Scale=1.0,BoldFont={Nimbus Roman Bold}]{Nimbus Roman}
\defaultfontfeatures{Ligatures=TeX}
\usepackage{fontawesome5}
\font\kefalaio="Nimbus Roman Bold" at 24pt
\font\ArKef="Nimbus Roman Bold" at 72pt
\font\OnKef="Nimbus Roman Bold" at 20pt
\font\OnKeftoc="Nimbus Roman" at 14pt
\font\OnKefh="Alegreya SC" at 10pt
\font\OnPar="Nimbus Roman Bold" at 14pt
\font\OnAsk="Alegreya SC Bold" at 12pt
\font\OnAskm="URWGothic-Demi" at 12pt
\newfontfamily\tnr{Nimbus Roman}
\def\chpcolor{red!80!black}
\def\chpcolortxt{red!80!black}
\setcounter{secnumdepth}{2}
\usepackage[inner=2.00cm, outer=1.50cm, top=3.00cm, bottom=2.00cm,xetex]{geometry}
\usepackage{amsmath,diffcoeff}
\usepackage{makeidx}
\usepackage{longtable,mathimatika}
\usepackage{etoolbox}
\makeatletter
\newif\ifLT@nocaption
\preto\longtable{\LT@nocaptiontrue}
\appto\endlongtable{%
\ifLT@nocaption
\addtocounter{table}{\m@ne}%
\fi}
\preto\LT@caption{%
\noalign{\global\LT@nocaptionfalse}}
\makeatother
\makeindex

%------ ΕΙΚΟΝΑ ΓΥΡΩ ΑΠΟ ΚΕΙΜΕΝΟ ------------
\usepackage{wrapfig}
\newenvironment{WrapText1}[3][r]
{\wrapfigure[#2]{#1}{#3}}
{\endwrapfigure}

\newenvironment{WrapText2}[3][l]
{\wrapfigure[#2]{#1}{#3}}
{\endwrapfigure}

\newcommand{\wrapr}[6]{
\begin{minipage}{\linewidth}\mbox{}\\
\vspace{#1}
\begin{WrapText1}{#2}{#3}
\vspace{#4}#5\end{WrapText1}#6
\end{minipage}}

\newcommand{\wrapl}[6]{
\begin{minipage}{\linewidth}\mbox{}\\
\vspace{#1}
\begin{WrapText2}{#2}{#3}
\vspace{#4}#5\end{WrapText2}#6
\end{minipage}}
%-------------------------------------------
\usepackage{tikz,pgfplots}
\usepackage{tkz-euclide}
\usepackage{calc}
\usepackage[framemethod=TikZ]{mdframed}
\usetikzlibrary{backgrounds}
\renewcommand{\thepart}{\arabic{part}}
\definecolor{steelblue}{cmyk}{.7,.278,0,.294}
\definecolor{askcolor}{cmyk}{0,1,1,0.2}
\definecolor{olivedrab}{cmyk}{0.75,0,0.75,0.44}
\usepackage{capt-of}
\usepackage{titletoc}
\usepackage[explicit]{titlesec}
\usepackage{graphicx}
\usepackage{multicol}
\usepackage{multirow}
\usepackage{enumitem}
\usepackage{cancel}
\usepackage{tabularx}
%\usepackage[decimalsymbol=comma]{siunitx}
\tikzset{>=latex}
\makeatletter
\pretocmd{\@part}{\gdef\parttitle{#1}}{}{}
\pretocmd{\@spart}{\gdef\parttitle{#1}}{}{}
\makeatother
\usepackage[titletoc]{appendix}
\usepackage{fancyhdr}
\pagestyle{fancy}
\fancyheadoffset{0cm}
\renewcommand{\headrulewidth}{\iftopfloat{0pt}{.5pt}}
\renewcommand{\chaptermark}[1]{\markboth{#1}{}}
\renewcommand{\sectionmark}[1]{\markright{\it\thesection\ #1}}
\fancyhf{}
\fancyhead[LE]{\thepage\ $\cdot$\ \OnKefh\nouppercase{\leftmark}}
\fancyhead[RO]{\nouppercase{\rightmark} $\cdot$\ \thepage}
\fancypagestyle{plain}{%
\fancyhead{} %
\renewcommand{\headrulewidth}{0pt}}

\newcounter{thewrhma}[chapter]
\renewcommand{\thethewrhma}{\thechapter.\arabic{thewrhma}}   

\newcommand{\Thewrhma}[1]{\refstepcounter{thewrhma}\textcolor{black}{\textbf{ΘΕΩΡΗΜΑ\hspace{2mm}\thethewrhma\hspace{1mm} \MakeUppercase{#1}}}\\}{}

\newcounter{porisma}[chapter]
\renewcommand{\theporisma}{\thechapter.\arabic{porisma}}\newcommand{\Porisma}[1]{\refstepcounter{porisma}\textcolor{black}{\textbf{ΠΟΡΙΣΜΑ\hspace{2mm}\theporisma\hspace{1mm} \MakeUppercase{#1}}}\\}{}

\newcounter{protasi}[chapter]
\renewcommand{\theprotasi}{\thechapter.\arabic{protasi}}\newcommand{\Protasi}[1]{\refstepcounter{protasi}\textcolor{black}{\textbf{ΠΡΟΤΑΣΗ\hspace{2mm}\theprotasi\hspace{1mm} \MakeUppercase{#1}}}\\}{}

\newcounter{orismos}[chapter]
\renewcommand{\theorismos}{\thechapter.\arabic{orismos}}   
\newcommand{\Orismos}[1]{\refstepcounter{orismos}\textcolor{red!80!black}{\textbf{ΟΡΙΣΜΟΣ\hspace{2mm}\theorismos}}\hspace{1mm} \MakeUppercase{\textbf{#1}}\\}{}

\usepackage{venndiagram}
\usepackage[outline]{contour}
\newcommand{\regularchapter}{%
\titleformat{\chapter}[display]
{\normalfont\huge\bfseries}{\chaptertitlename\ \thechapter}{20pt}{\Huge##1}
\titlespacing*{\chapter}
{0pt}{10pt}{10pt}
}

\newcommand{\dgr}{\\\leavevmode\\}
\newcommand{\epask}{\\\\\\}
\usepackage{booktabs}
\usepackage{hhline}
\DeclareRobustCommand{\perthousand}{%
\ifmmode
\text{\textperthousand}%
\else
\textperthousand
\fi}
\newcounter{askhsh}[section]
\renewcommand{\theaskhsh}{\arabic{askhsh}}
\newcommand{\Askhsh}[1]{\refstepcounter{askhsh}{\begin{tikzpicture}[remember picture,overlay]
\path[left color=red!80!black,right color=white](0,0.45)  rectangle (7.5,0.455);
\pgftext[left,y=.7cm]{\textbf{\textcolor{red!80!black}{\OnAsk{Ασκηση}\quad }\OnAskm{#1.\theaskhsh}}\\};
\end{tikzpicture}}}{}
\allowdisplaybreaks

%---------- ΜΕΘΟΔΟΣ --------------
\usetikzlibrary{shadows,calc}
\usepackage{tcolorbox}
\tcbuselibrary{skins,theorems,breakable}

\newenvironment{Askhshs}[1][\linewidth]
{\refstepcounter{askhsh}
\begin{tcolorbox}[breakable,
enhanced standard,
boxrule=0pt,titlerule=-0pt,fuzzy shadow={1.5mm}{-1.5mm}{0mm}{.35mm}{black!70!white},
width=\linewidth,
title style={color=white},
overlay unbroken and first={
\path[left color=red!80!black,right color=red!70!white,draw=none]
([yshift=-\pgflinewidth]frame.north west) to ([yshift=-5pt]title.south west)[rounded corners=0pt] -- ([yshift=-5pt]title.south east) to[rounded corners=2pt] ([yshift=-\pgflinewidth]frame.north east)[rounded corners=2pt] -- cycle;
},
fonttitle=\bfseries,
before=\par\medskip\noindent,
after=\par\medskip,
toptitle=4pt,
top=10pt,topsep at break=-5pt,
colback=white!20,title={\large {\raisebox{-.7mm}{\OnAskm{Άσκηση #1.\theaskhsh}}\ \  |\ \ \raisebox{-.4mm}{\textcolor{white}{\large \faIcon{pen}}}}}]}
{\end{tcolorbox}}
%------------------------------------------
\newcommand{\lysh}{{\begin{tikzpicture}[remember picture,overlay]
\draw[red!80!black] (2,.95)--(0,.95)--(0,.45);
\pgftext[left,x=.15cm,y=.7cm]{\textcolor{red!80!black}{\textbf{\OnAskm{ΛΥΣΗ}}}\ \ \ \textcolor{red!80!black}{{\large \faIcon{check}}}\\};
\end{tikzpicture}}}{}

%-------- ΣΤΥΛ ΚΕΦΑΛΑΙΟΥ ---------
\newcommand*\chapterlabel{}
\newcommand{\fancychapter}{%
\titleformat{\chapter}[display]
{\normalfont\color{red!70!black}}
{\gdef\chapterlabel{\thechapter\ }}{0pt}
{\filright
\begin{tikzpicture}[baseline={([yshift=-.6ex]current bounding box.center)}]
\node [fill=red!70!black,rectangle,text=white,minimum height=2.5cm,minimum width=1.55cm]{\ArKef\thechapter};
\pgftext[left,x=1.5cm,y=-.5cm] {\kefalaio\chaptertitlename};
\pgftext[left,x=1.5cm,y=.7cm] {\OnKef##1};
\end{tikzpicture}}
\titlespacing*{\chapter}{0pt}{30pt}{20pt}
}
%------------------------------------------------
\titleformat{\section}
{\normalfont\Large\bf}%
{}{0em}%
{\raggedright\strut\textcolor{red!80!black}{{ \textit{\LARGE\thesection}}\ \ #1}\strut\\\vspace{-3mm}\begin{tikzpicture}[remember picture,overlay]
\path[left color=red!70!black,right color=white](-.1,0)  rectangle (17.5,0.01);
\end{tikzpicture}}[]
\titlespacing*{\section}{0pt}{10pt}{10pt}

\AtBeginDocument{\renewcommand*{\d}{\mathop{\kern0pt\mathrm{d}}\!{}}}

\apptocmd{\mainmatter}{\fancychapter}{}{}
\apptocmd{\backmatter}{\regularchapter}{}{}
\apptocmd{\frontmatter}{\regularchapter}{}{}


\contentsmargin{0cm}
\titlecontents{part}[-1pc]
{\addvspace{10pt}%
\bf\Large ΜΕΡΟΣ\quad }%
{}
{}
{\;\dotfill\;\normalsize\ Σελίδα}%
%------------------------------------------
\titlecontents{chapter}[0pc]
{\addvspace{30pt}%
\begin{tikzpicture}[remember picture, overlay]%
\draw[fill=black,draw=black] (-.3,.5) rectangle (3.7,1.1); %
\pgftext[left,x=0cm,y=0.75cm]{\color{white}\sc\Large\bfseries Κεφάλαιο\ \thecontentslabel};%
\end{tikzpicture}\footnotesize\OnKeftoc}%
{}
{}
{\hspace*{-2.3em}\hfill\normalsize Σελίδα \thecontentspage}%
\titlecontents{section}[2.4pc]
{\addvspace{1pt}}
{\contentslabel[\thecontentslabel]{2pc}}
{}
{\;\dotfill\;\small \thecontentspage}
[]
\titlecontents*{subsection}[4pc]
{\addvspace{-1pt}\small}
{}
{}
{\ --- \small\thecontentspage}
[ \textbullet\ ][]

\makeatletter
\renewcommand{\tableofcontents}{%
\chapter*{%
\vspace*{-20\p@}%
\begin{tikzpicture}[remember picture, overlay]%
\pgftext[right,x=15cm,y=0.2cm]{\Huge\sc\bfseries \contentsname};%
\draw[fill=black,draw=black] (12.5,-.75) rectangle (15.5,1);%
\clip (12.5,-.75) rectangle (18,1);
\pgftext[right,x=15cm,y=0.2cm]{\color{white}\Huge\bfseries \contentsname};%
\end{tikzpicture}}%
\@starttoc{toc}}
\makeatother

\usepackage[contents={},scale=1,opacity=1,color=black,angle=0]{background}

\newcommand\blfootnote[1]{%
\begingroup
\renewcommand\thefootnote{}\footnote{#1}%
\addtocounter{footnote}{-1}%
\endgroup
}
\usepackage{textcomp}


\usepackage[labelfont={footnotesize,it,bf},font={footnotesize}]{caption}

%----------------------------------------
%-------- ΜΑΘΗΜΑΤΙΚΑ ΕΡΓΑΛΕΙΑ ---------
\usepackage{mathtools}
%----------------------
%-------- ΠΙΝΑΚΕΣ ---------
\usepackage{booktabs}
%----------------------
%----- ΥΠΟΛΟΓΙΣΤΗΣ ----------
%\usepackage{calculator}
%----------------------------
\mathtoolsset{showonlyrefs}
%----- ΟΡΙΖΟΝΤΙΑ ΛΙΣΤΑ ------
\usepackage{xparse}
\newcounter{answers}
\renewcommand\theanswers{\arabic{answers}}
\ExplSyntaxOn
\NewDocumentCommand{\results}{m}
{
\seq_set_split:Nnn \l_results_a_seq {,}{#1}
\par\nobreak\noindent\setcounter{answers}{0}
\seq_map_inline:Nn \l_results_a_seq
{
\makebox[.18\linewidth][l]{\stepcounter{answers}\theanswers.~##1}\hfill
}
\par
}
\seq_new:N \l_results_a_seq
\ExplSyntaxOff
%----------------------------
%------ ΜΗΚΟΣ ΓΡΑΜΜΗΣ ΚΛΑΣΜΑΤΟΣ ---------
\DeclareRobustCommand{\frac}[3][0pt]{%
{\begingroup\hspace{#1}#2\hspace{#1}\endgroup\over\hspace{#1}#3\hspace{#1}}}
%----------------------------------------
\usepackage{microtype}
\usepackage{float}

\usepackage{caption}
%----------- ΓΡΑΦΙΚΕΣ ΠΑΡΑΣΤΑΣΕΙΣ ---------
\pgfkeys{/pgfplots/aks_on/.style={axis lines=center,
xlabel style={at={(current axis.right of origin)},xshift=1.5ex, anchor=center},
ylabel style={at={(current axis.above origin)},yshift=1.5ex, anchor=center}}}
\pgfkeys{/pgfplots/grafikh parastash/.style={cyan,line width=.4mm,samples=200}}
\pgfkeys{/pgfplots/belh ar/.style={axis line style={-latex}}}
%-----------------------------------------
%----- ΧΡΗΣΙΜΟΙ ΟΡΙΣΜΟΙ ---------------


\tikzstyle{pl}=[line width=0.2mm]
\tikzstyle{plm}=[line width=0.4mm]
\tkzSetUpPoint[size=7,fill=white]

\makeatletter
\def\closedcycley{%
-| (perpendicular cs: 
horizontal line through={(current plot begin)}, 
vertical line through={(\pgfplots@ZERO@x,\pgfplots@ZERO@y)})
-- cycle
}%
\makeatother
\newcommand{\tss}[1]{\textsuperscript{#1}}
\newcommand{\tssL}[1]{\textsuperscript{\MakeLowercase{#1}}}


\makeatletter
\patchcmd{\@caption}{\csname fnum@#1\endcsname:
\ignorespaces#3}{\Centering
\csname fnum@#1\endcsname\ifblank{#3}{}{: \ignorespaces#3}}{}{}
\makeatother 
\usepackage[labelfont={footnotesize,it,bf},font={footnotesize}]{caption}
\usepackage[parfill]{parskip}


\newlist{rlist}{enumerate}{3}
\setlist[rlist]{itemsep=0mm,label=\roman*.}
\newlist{brlist}{enumerate}{3}
\setlist[brlist]{itemsep=0mm,label=\bf\roman*.}
\setlist[itemize]{itemsep=0mm}



\makeatletter
\renewrobustcmd{\anw@true}{\let\ifanw@\iffalse}
\renewrobustcmd{\anw@false}{\let\ifanw@\iffalse}\anw@false
\newrobustcmd{\noanw@true}{\let\ifnoanw@\iffalse}
\newrobustcmd{\noanw@false}{\let\ifnoanw@\iffalse}\noanw@false
\renewrobustcmd{\anw@print}{\ifanw@\ifnoanw@\else\numer@lsign\fi\fi}
\makeatother

\newlist{alist}{enumerate}{3}
\setlist[alist]{itemsep=0mm,label=\alph*.}

\DeclareMathSizes{10.95}{10.95}{7}{5}
\DeclareMathSizes{6}{6}{3.8}{2.7}
\DeclareMathSizes{8}{8}{5.1}{3.6}
\DeclareMathSizes{9}{9}{5.8}{4.1}
\DeclareMathSizes{10}{10}{6.4}{4.5}
\DeclareMathSizes{12}{12}{7.7}{5.5}
\DeclareMathSizes{14.4}{14.4}{9.2}{6.5}
\DeclareMathSizes{17.28}{17.28}{11}{7.9}
\DeclareMathSizes{20.74}{20.74}{13.3}{9.4}
\DeclareMathSizes{24.88}{24.88}{16}{11.3}

\makeatletter
\AtBeginDocument{
\check@mathfonts
\fontdimen16\textfont2=2.5pt
\fontdimen17\textfont2=2.5pt
\fontdimen14\textfont2=4.5pt
\fontdimen13\textfont2=4.5pt 
}
\makeatother



\begin{document}
\begin{titlepage}
\newgeometry{left=2.5cm,top=2.5cm} %defines the geometry for the titlepage
\pagecolor{white}
\begin{center}
{\large Ευστράτιος Χατζαράκης\\Μαθηματικός}
\end{center}
\noindent
\par
\noindent
\mbox{}\\\\
\begin{center}
\textbf{\fontsize{20}{40}\selectfont{ΕΙΣΑΓΩΓΗ ΣΤΙΣ}}\par\mbox{}\\\vspace{-3mm}
\textbf{\fontsize{20}{40}\selectfont{ΔΙΑΦΟΡΙΚΕΣ ΕΞΙΣΩΣΕΙΣ}}\par\mbox{}\\
\vspace{-4mm}
\rule{12cm}{0.1mm}\\
\vspace{3mm}
{\fontsize{15}{15}\MakeUppercase{Λύσεις των ασκήσεων}}\\
\vspace{.7mm}
{\fontsize{15}{15}\MakeUppercase{του φυλλαδίου}}\\
\end{center}
\vspace{3cm}
\begin{flushright}
\begin{itemize}
\item Διαφορικές εξισώσεις 1ης τάξης
\item Γραμμικές διαφορικές εξισώσεις
\item Δυναμοσειρές λύσεις διαφορικών εξισώσεων 2ης τάξης
\end{itemize}
\end{flushright}

\vfill
\noindent
\color{black}
\begin{center}
{\large{ΦΡΟΝΤΙΣΤΗΡΙΟ {\LARGE  $ \pi $}}\\
\large{ΙΩΑΝΝΙΝΑ 2022}}
\vskip\baselineskip
\end{center}
\hbox{ % Horizontal box
\hspace*{0.2\textwidth} % Whitespace to the left of the title page
\rule{1pt}{\textheight} % Vertical line
\hspace*{0.05\textwidth} % Whitespace between the vertical line and title page text
\parbox[b]{0.75\textwidth}{ % Paragraph box which restricts text to less than the width of the page

{\textbf{Εισαγωγή στις}\\\textbf{διαφορικές εξισώσεις}\\\\\noindent \textbf{Ευστράτιος Χατζαράκης - Μαθηματικός}\\e-mail : \\[0.5\baselineskip]
}\\[2\baselineskip] % Title

\vspace{.4\textheight} % Whitespace between the title block and the publisher
{Πνευματικά Δικαιώματα : ...}\\[\baselineskip]}}
\vspace*{2\baselineskip}
\newpage
\mbox{}\\\\\\\\\\\\
\hspace*{0.75\textwidth}
\textit{{\large Αφιέρωση}}
\newpage
\mbox{}
\newpage
\newpage
\end{titlepage}
\pagestyle{empty}
\frontmatter
\tableofcontents
\mainmatter
\pagestyle{fancy}
\chapter[Ασκησεις Φυλλαδιου]{Ασκήσεις Φυλλαδίου}
\section{A - Διαφορικές εξισώσεις πρώτης τάξης}
\begin{Askhshs}[A]
\textbf{Με τη βοήθεια του μετασχηματισμού {\boldmath{$z=\tan y$}}, να αποδειχθεί ότι η λύση του προβλήματος αρχικών τιμών
{\boldmath{\[ \frac{1}{{\cos }^{2}y}\frac{dy}{dx}+x\tan y+x{\tan }^{3}y=0\quad,\quad y\left( 0 \right)=\frac{\pi }{4} \]}}
έχει την ιδιότητα 
{\boldmath{\[\underset{x\to \infty }{\mathop{\lim }}\,y\left( x \right)=0 \]}}}
\end{Askhshs}\mbox{}\\
\lysh
\noindent
Η εξίσωση έχει προφανή λύση την $ y=0 $ η οποία όμως δεν πληροί την αρχική συνθήκη του προβλήματος $ y(0)=\frac{\pi}{4} $. Εκτελώντας το μετασχηματισμό $z=\tan y$ θα έχουμε \[ \dfrac{dz}{dx}={{\left( \tan y \right)}^{\prime }}=\dfrac{1}{{{\cos }^{2}}y}\dfrac{dy}{dx} \]
Αντικαθιστώντας τις σχέσεις αυτές στην αρχική εξίσωση θα πάρει τη μορφή:
\begin{equation}\label{a1:1}
z'+xz+xz^{3}=0
\end{equation} Η εξίσωση αυτή είναι μια διαφορική εξίσωση \textbf{Bernoulli} με $ r=3 $. Επίσης σύμφωνα με το μετασχηματισμό αυτό η αρχική συνθήκη θα έχει ως εξής.
\[ \textrm{Για }x=0\ :\ y(0)=\frac{\pi}{4}\Rightarrow z(0)=\tan{\frac{\pi}{4}}=1  \]
Θα χρησιμοποιήσουμε το μετασχηματισμό $u={{z}^{1-r}}$ με $r=3$ δηλαδή $u=\dfrac{1}{{{z}^{2}}}$ ο οποίος μας δίνει ${u}'=-\dfrac{2{z}'}{{{z}^{2}}}$.
Ο μετασχηματισμός αυτός θα μετατρέψει την εξίσωση \eqref{a1:1} σε μια \textbf{γραμμική εξίσωση 1ης τάξης} :
\begin{equation}\label{a1:2}
{u}'-2xu-2x=0
\end{equation}
\noindent
Η γενική λύση της τελευταίας εξίσωσης θα δίνεται από τον τύπο 
\begin{align*}
u\left( x \right)&={{e}^{\int{2xdx}}}\left[ c+\int{2x{{e}^{-\int{2xdx}}}}dx \right]=\\
&={{e}^{{{x}^{2}}}}\left[ c-{{e}^{-{{x}^{2}}}} \right]=c{{e}^{{{x}^{2}}}}-1
\end{align*}
όπου $ c $ είναι μια αυθαίρετη σταθερά. Αντικαθιστώντας στον τύπο της γενικής λύσης τις σχέσεις $u=\dfrac{1}{{{z}^{2}}}$ και $ z=\tan{y} $ θα έχουμε 
\[ z=\dfrac{1}{\sqrt{c{{e}^{{{x}^{2}}}}-1}}=\tan y\Rightarrow y=\arctan  \dfrac{1}{\sqrt{c{{e}^{{{x}^{2}}}}-1}} \]
Σύμφωνα με την αρχική συνθήκη $y\left( 0 \right)=\dfrac{\pi }{4}$ η προηγούμενη σχέση θα μας δώσει την τιμή της σταθεράς $ c $:
\[\arctan \dfrac{1}{\sqrt{c-1}}=\dfrac{\pi }{4}\Rightarrow \dfrac{1}{\sqrt{c-1}}=1\Rightarrow c=2\]
Επομένως η λύση του προβλήματος αρχικών τιμών θα είναι
\[ y\left( x \right)=\arctan \dfrac{1}{\sqrt{2{{e}^{{{x}^{2}}}}-1}} \]
Επιπλέον παρατηρούμε ότι όταν $x\to \infty \Rightarrow \dfrac{1}{\sqrt{2{{e}^{{{x}^{2}}}}-1}}\to 0$ άρα θα ισχύει $\underset{x\to \infty }{\mathop{\lim }}\,y\left( x \right)=0$.\epask
\begin{Askhshs}[A]
\bmath{Να επιλυθεί το πρόβλημα αρχικών τιμών
\[ q(x)y'=q'(x)y-y^2\ ,\ y(0)=1 \]
όπου $ q $ είναι μια θετική συνάρτηση με συνεχή παράγωγο στο $ \mathbb{R} $ και $ q(0)=1 $.}
\end{Askhshs}\mbox{}\\
\lysh
Διαιρώντας και τα δύο μέλη της αρχικής εξίσωσης $ q(x)y'=q'(x)y-y^2 $ με τη θετική συνάρτηση $ q(x)>0 $ αυτή θα γραφτεί ως εξής:
\begin{equation}\label{eq:a2}
y'=\frac{q'(x)}{q(x)}y-\frac{y^2}{q(x)}
\end{equation}
η οποία είναι μια εξίσωση \textbf{Bernoulli} με $ r=2 $.
Παρατηρούμε ότι η $ y=0 $ είναι λύση της εξίσωσης που όμως δεν ικανοποιεί την αρχική συνθήκη $ y(0)=1 $ άρα την απορρίπτουμε. Με την αντικατάσταση $ z=\frac{1}{y} $ η οποία δίνει $ z'=-\frac{y'}{y^2} $ η \eqref{eq:a2} μετασχηματίζεται στην γραμμική εξίσωση πρώτης τάξης :
\begin{equation}
z'-\frac{q'(x)}{q(x)}z=\frac{1}{q(x)}
\end{equation}
Η γενική λύση της παραπάνω εξίσωσης θα δίνεται από τον τύπο:
\begin{align*}
z(x)&=e^{-\dintt{\frac{q'(x)\d x}{q(x)}}}\left[ c+\int{\frac{e^{-\dintt{\frac{q'(x)dx}{q(x)}\ }}}{q(x)}}dx\right]\eq{q>0}\\
&=e^{-\log{q(x)}}\left(c+\int\frac{e^{\log{q(x)}}}{q(x)}{\d x}\right)=\frac{1}{q(x)}\left(c+\int{\d x}\right)=\frac{x+c}{q(x)}
\end{align*}
Επιπλέον, μετά το μετασχηματισμό, η αρχική συνθήκη θα γίνει : $ y(0)=1\Rightarrow \frac{1}{z(0)}=1\Rightarrow z(0)=1 $ και σύμφωνα μ' αυτήν θα έχουμε
\[ z(0)=1\Rightarrow \frac{0+c}{q(0)}=1\Rightarrow c=1  \]
Η τιμή αυτή της αυθαίρετης σταθεράς $ c $ μας δίνει τη λύση του προβλήματος αρχικών τιμών η οποία θα είναι :  \[ z=\frac{x+1}{q(x)}\Rightarrow y=\dfrac{q(x)}{x+1} \]
\begin{Askhshs}[A]
\bmath{Να επιλυθεί η διαφορική εξίσωση
\[ (y-x)e^{y/x}\frac{dy}{dx}+y\left( 1+e^{y/x}\right) =0 \]
Ισχύει ότι $ \displaystyle\int{\frac{z-1}{ze^{-1/z}+z^2}dz}=\log{\left|1+ze^{1/z} \right|+c } $.}
\end{Askhshs}\mbox{}\\
\vspace{5mm}
\noindent
\lysh
\textbf{1\tss{ος} Τρόπος}\\
\noindent
Η αρχική διαφορική εξίσωση μπορεί ισοδύναμα να γραφτεί στη μορφή \[ \frac{dy}{dx}=\frac{y\left( 1+e^{y/x}\right) }{(y-x)e^{x/y}} \] η οποία είναι μια ομογενής εξίσωση με βαθμό ομογένειας $ 1 $. Χρησιμοποιούμε το μετασχηματισμό $ y=xz $ και ύστερα από παραγώγιση θα πάρουμε $ y'=xz'+z $. Έτσι αντικαθιστώντας τις συναρτήσεις αυτές στην εξίσωση, αυτή θα γίνει:
\begin{align*}
xz'+z=-\dfrac{xz\left( 1+e^{1/z}\right) }{\left( xz-x\right) e^{1/z}}&\Rightarrow 
xz'+z=-\dfrac{xz\left( 1+e^{1/z}\right) }{x(z-1)e^{1/z}}\Rightarrow\\
&\Rightarrow xz'=-\dfrac{z+z^2e^{1/z}}{(z-1)e^{1/z}}\Rightarrow z'=\dfrac{1}{x}\cdot\dfrac{z e^{-1/z}+z^2}{z-1}
\end{align*}
η οποία είναι μια διαφορική εξίσωση χωριζομένων μεταβλητών και άρα μπορεί να πάρει τη μορφή:
\[ \frac{z-1}{ze^{-1/z}+z^2}dz=-\frac{1}{x}dx \]
Ολοκληρώνοντας και τα δύο μέλη της παραπάνω εξίσωσης προκύπτει ότι:
\begin{gather*} \int\frac{z-1}{ze^{-1/z}+z^2}dz=-\int\frac{1}{x}dx+c'\Rightarrow\\
\log{\left|1+ze^{1/z} \right|}=-\log{|x|}+c'\Rightarrow\\
\log{\left|1+ze^{1/z} \right|}+\log{|x|}=c'\Rightarrow\\
\left|x\left( 1+ze^{1/z}\right) \right|=e^{c'}\Rightarrow
x\left( 1+ze^{1/z}\right)=\pm e^{c'}
\end{gather*}
όπου $ c' $ είναι μια αυθαίρετη σταθερά. Θέτοντας $ \pm e^{c'}=c $ και κάνοντας αναδρομική αντικατάσταση παίρνουμε όλες τις λύσεις της αρχικής εξίσωσης οι οποίες θα δίνονται από τον τύπο:
\[ x\left( 1+\frac{y}{x}e^{x/y}\right)=c\Rightarrow x+ye^{x/y}=c \]
\textbf{2\tss{ος} Τρόπος}\\
Η αρχική διαφορική εξίσωση γράφεται και στη μορφή
\begin{equation}\label{eq:a3}
\undercbrace{y\left(1+e^{x/y} \right)}_{M}dx+\undercbrace{(y-x)e^{x/y}}_{N}dy=0 
\end{equation}
Εξετάζουμε αν πρόκειται για μια εξίσωση αμέσως ολοκληρώσιμη. Θα έχουμε :
\begin{gather*} \dfrac{\partial M}{\partial y}=\dfrac{\partial}{\partial y}\left[y\left(1+e^{x/y} \right) \right]=1+e^{x/y}-\frac{x}{y}e^{x/y}\ \textrm{ και }\ 
\dfrac{\partial N}{\partial x}=\dfrac{\partial}{\partial x}\left[(y-x)e^{x/y} \right]=-\frac{x}{y}e^{x/y} 
\end{gather*}
Διαπιστώνουμε ότι δεν πρόκειται για μια εξίσωση αμέσως ολοκληρώσιμη αφού $ \frac{\partial M}{\partial y}\neq\frac{\partial N}{\partial x} $
άρα θα αναζητήσουμε έναν ολοκληρωτικό παράγοντα. Έχουμε λοιπόν:
\begin{align*}
\frac{\frac{\partial N}{\partial x}-\frac{\partial M}{\partial y}}{M}&=\dfrac{\frac{\partial}{\partial x}\left[(y-x)e^{x/y} \right]-\frac{\partial}{\partial y}\left[y\left(1+e^{x/y} \right) \right]}{y\left(1+e^{x/y} \right)} =\\
&=\dfrac{-\frac{x}{y}e^{x/y}-1-e^{x/y}+\frac{x}{y}e^{x/y}}{y\left(1+e^{x/y} \right)}=\dfrac{-\left(1+e^{x/y} \right)}{y\left(1+e^{x/y} \right)}=-\frac{1}{y}
\end{align*}
Η τελευταία είναι μια παράσταση μόνο του $ y $ οπότε η συνάστηση $ \rho(y)=e^{-\int\frac{1}{y}dy}=e^{-\log{|y|}dy}=\frac{1}{y} $ είναι ο ζητούμενος ολοκληρωτικός παράγοντας. Πολλαπλασιάζοντας μ' αυτόν την εξίσωση \eqref{eq:a3} θα προκύψει :
\begin{gather}\label{eq:a32}
\frac{1}{y}y\left(1+e^{x/y} \right)dx+\frac{1}{y}(y-x)e^{x/y}dy=0\Rightarrow\undercbrace{\left(1+e^{x/y} \right)}_{M}dx+\undercbrace{\left( 1-\frac{x}{y}\right) e^{x/y}}_{N}dy=0
\end{gather}
Εύκολα διαπιστώνουμε ότι η εξίσωση \eqref{eq:a32} είναι μια αμέσως ολοκληρώσιμη εξίσωση αφού $ \frac{\partial M}{\partial y}=\frac{\partial N}{\partial x} $. Αυτό σημαίνει ότι θα υπάρχει μια συνάρτηση $ f(x,y) $ τέτοια ώστε η \eqref{eq:a32} να γίνεται $ df(x,y)=Mdx+Ndy=0 $. Οι λύσεις θα δίνονται από τον τύπο $ f(x,y)=c $. Θα έχουμε :
\begin{equation}\label{eq:a31}
\frac{\partial f}{\partial x}=1+e^{x/y}\ \textrm{και }\ \frac{\partial f}{\partial y}=\left( 1-\frac{x}{y}\right) e^{x/y} 
\end{equation}
Από την πρώτη σχέση προκύπτει :
\[ f(x,y)=\int{\left( 1+e^{x/y}\right) dx}+g(y)=x+ye^{x/y}+g(y) \] για κάποια συνάρτηση $ g(y) $. Παραγωγίζοντας την τελευταία σχέση ως προς $ y $ θα έχουμε
\begin{equation}\label{eq:a33}
\frac{\partial f}{\partial y}=e^{x/y}-\frac{x}{y}e^{x/y}+g'(y) 
\end{equation}
Από τις σχέσεις \eqref{eq:a31} και \eqref{eq:a33} έχουμε $ \left( 1-\frac{x}{y}\right) e^{x/y}=e^{x/y}-\frac{x}{y}e^{x/y}+g'(y) $ άρα $ g'(y)=0 $. Αυτή μας δίνει $ g(y)=c' $ και επιλέγοντας $ c'=0 $ δηλαδή $ g(y)=0 $ προκύπτει ότι η συνάρτηση $ f(x,y) $ θα δίνεται απο τη σχέση
\[ f(x,y)=x+ye^{x/y} \]
Όλες οι λύσεις λοιπόν της αρχικής εξίσωσης θα δίνονται από τον τύπο \[ f(x,y)=c\Rightarrow x+ye^{x/y}=c \]
\begin{Askhshs}[A]
\bmath{Να επιλυθεί η εξίσωση
\[ \dfrac{dy}{dx}=-\dfrac{y(x+y+1)}{x(x+3y+2)} \]}
\end{Askhshs}\mbox{}\\
\lysh
Παρατηρούμε ότι μια λύση της εξίσωσης είναι η $ y=0 $. Γράφουμε τώρα την εξίσωση στη μορφή
\[ \undercbrace{y(x+y+1)}_{M}dx+\undercbrace{x(x+3y+2)}_{N}dy=0 \] η οποία είναι ισοδύναμη με την αρχική εξίσωση. Παρατηρούμε ότι
\begin{gather*} 
\dfrac{\partial M}{\partial y}=\dfrac{\partial}{\partial y}\left[y(x+y+1)\right]=x+2y+1\ \textrm{και}\ 
\dfrac{\partial N}{\partial x}=\dfrac{\partial}{\partial y}\left[x(x+3y+2)\right]=2x+3y+2
\end{gather*}
Συμπεραίνουμε ότι δεν είναι αμέσως ολοκληρώσιμη άρα θα εξετάσουμε την ύπαρξη ενός ολοκληρωτικού παράγοντα. Θα έχουμε
\begin{align*} \dfrac{\frac{\partial N}{\partial x}-\frac{\partial M}{\partial y}}{M}&=\dfrac{\frac{\partial}{\partial y}\left[x(x+3y+2)\right]-\frac{\partial}{\partial y}\left[y(x+y+1)\right]}{y(x+y+1)}=\\
&=\dfrac{2x+3y+2-x-2y-1}{y(x+y+1)}=\dfrac{x+y+1}{y(x+y+1)}=\dfrac{1}{y}
\end{align*}
Η τελευταία παράσταση αποτελεί μαι συνάρτηση με μόνη μεταβλητή το $ y $ οπότε ένας ολοκληρωτικός παράγοντας είναι η συνάρτηση $ \rho(y)=e^{\dintt{\frac{1}{y}dy}}=e^{\log|y|}=y $.  
Πολλαπλασιάζοντας την αρχική εξίσωση με $ y $ προκύπτει :
\begin{equation}\label{eq:a4}
\undercbrace{y^2(x+y+1)}_{M}dx+\undercbrace{xy(x+3y+2)}_{N}dy=0
\end{equation} 
η οποία είναι αμέσως ολοκληρώσιμη αφού $ \frac{\partial M}{\partial y}=\frac{\partial N}{\partial x}=2xy+3y^2+2y $. Επομένως $ \exists f(x,y) $ συνάρτηση τέτοια ώστε η εξίσωση \eqref{eq:a4} να γίνει $ df(x,y)=0 $. Οι λύσεις της θα δίνονται από τον τύπο $ f(x,y)=c $.
Σύμφωνα μ' αυτά θα ισχύει
\begin{equation}\label{eq:a41}
\frac{\partial f}{\partial x}=y^2(x+y+1)\ \textrm{ και }\ \frac{\partial f}{\partial y}=xy(x+3y+2)
\end{equation}
Ολοκληρώνοντας την πρώτη σχέση της \eqref{eq:a41} ως προς $ x $ αποκτάμε τη σχέση
\begin{align*} 
f(x,y)&=\int{y^2(x+y+1)\ dx}+g(y)\\&=y^2\left(\frac{x^2}{2}+xy+x \right)+g(y)=\frac{y^2x^2}{2}+xy^3+xy^2+g(y) 
\end{align*}
για κάποια συνάρτηση $ g(y) $. Παραγωγίζοντας την παραπάνω σχέση ως προς $ y $ έχουμε :
\[ \frac{\partial f}{\partial y}=2yx^2+3xy^2+2xy+g'(y) \]
Θα πρέπει όμως να ισχύει $ \frac{\partial f}{\partial y}=N $, σχέση η οποία μας δίνει
\[ 2yx^2+3xy^2+2xy+g'(y)=xy(x+3y+2)\Rightarrow g'(y)=0 \]
Επιλέγοντας $ g(y)=0 $ θα έχουμε τον τύπο 
\[ f(x,y)=c\Rightarrow\frac{y^2x^2}{2}+xy^3+xy^2=c \]
όπου $ c $ είναι μια αυθαίρετη σταθερά. Όλες οι λύσεις της αρχικής εξίσωσης θα δίνονται από τις σχέσεις
$ y=0\ \textrm{ και }\ \frac{y^2x^2}{2}+xy^3+xy^2=c $ οι οποίες συμπτύσσονται στον γενικό τύπο
\[ \frac{y^2x^2}{2}+xy^3+xy^2=c \]
\begin{Askhshs}[A]
\bmath{Να επιλυθεί η διαφορική εξίσωση
\[ \left( x^2+xy^2\right) y'-3xy+2y^3=0 \]
αφού βρεθεί ένας ολοκληρωτικός παράγοντας της μορφής $ \rho(x,y)=x^n\varphi(y) $.}
\end{Askhshs}\mbox{}\\
\lysh
Η αρχική διαφορική εξίσωση έχει προφανή λύση την $ y=0 $. Επίσης γράφεται ισοδύναμα
\begin{equation}\label{a50}
\left( 2y^3-3xy\right) dx+\left( x^2+xy^2\right) dy=0 
\end{equation} 
Πολλαπλασιάζοντας και τα δύο μέλη με τον ολοκληρωτικό παράγοντα $ \rho(x,y) $ παίρνουμε
\begin{gather*}
x^n\varphi(y)\left( 2y^3-3xy\right) dx+x^n\varphi(y)\left( x^2+xy^2\right) dy=0\Rightarrow\\
\undercbrace{\varphi(y)\left( 2y^3x^n-3x^{n+1}y\right) }_{M}dx+\undercbrace{\varphi(y)\left( x^{n+2}+x^{n+1}y^2\right) }_{N}dy=0
\end{gather*} 
Η παραπάνω θα είναι μια εξίσωση αμέσως ολοκληρώσιμη αν και μόνο αν ισχύει $ \frac{\partial M}{\partial y}=\frac{\partial N}{\partial x} $. Θα έχουμε λοιπόν
\begin{align*}
\frac{\partial M}{\partial y}=\frac{\partial N}{\partial x}&\Rightarrow \frac{\partial }{\partial y}\left[\varphi(y)\left( 2y^3x^n-3x^{n+1}y\right)  \right] =\frac{\partial }{\partial x}\left[\varphi(y)\left( x^{n+2}+x^{n+1}y^2\right)  \right]\Rightarrow\\
&\phantom{\Rightarrow}\varphi'(y)\left(2x^ny^3-3x^{n+1}y \right)+\varphi(y)\left(6x^ny^2-3x^{n+1} \right)=\\&\phantom{f(x)}=\varphi(y)\left[(n+2)x^{n+1}+(n+1)y^2x^n \right]\Rightarrow\\&\phantom{\Rightarrow}\varphi'(y)2x^ny^3-\varphi'(y)3x^{n+1}y+\varphi(y)\cdot 6x^ny^2-\varphi(y)\cdot 3x^{n+1}=\\&\phantom{f(x)}=\varphi(y)\cdot(n+2)x^{n+1}+\varphi(y)\cdot(n+1)y^2x^n\Rightarrow\\&\phantom{\Rightarrow}x^{n+1}\left[-3y\varphi'(y)-3\varphi(y) \right]+x^n\left[ 2y^3\varphi'(y)+6y^2\varphi(y)\right]=\\&\phantom{f(x)}=(n+2)\varphi(y)x^{n+1}+(n+1)y^2\varphi(y)x^n  
\end{align*}
Εξισώνοντας τους ομοβάθμιους όρους των παραπάνω πολυωνύμων παίρνουμε τις εξισώσεις :\begin{equation}\label{a5}\begin{gathered}
-3y\varphi'(y)-3\varphi(y)=(n+2)\varphi(y)\Rightarrow\\ -3y\varphi'(y)=3\varphi(y)+(n+2)\varphi(y)\Rightarrow\\-3y\varphi'(y)=(n+5)\varphi(y)\Rightarrow \frac{\varphi'(y)}{\varphi(y)}=-\dfrac{n+5}{3y}\ \textrm{ και }
\end{gathered}
\end{equation}
\begin{equation}\label{a51}
\begin{gathered}
2y^3\varphi'(y)+6y^2\varphi(y)=(n+1)y^2\varphi(y)\Rightarrow\\
2y^3\varphi'(y)=-6y^2\varphi(y)+(n+1)y^2\varphi(y)\Rightarrow\\
2y^3\varphi'(y)=\left(-6y^2+(n+1)y^2\right) \varphi(y)\Rightarrow\frac{\varphi'(y)}{\varphi(y)}=-\dfrac{n-5}{2y}
\end{gathered}
\end{equation}
Επομένως από τις σχέσεις \eqref{a5} και \eqref{a51} έχουμε
\[ -\dfrac{n+5}{3y}=\dfrac{n-5}{2y}\Rightarrow 2n+10=-3n+15\Rightarrow n=1 \]
Με αντικατάσταση στη σχέση \eqref{a5} παίρνουμε την εξίσωση 
\begin{equation}\label{a52}
\varphi'(y)+\frac{2}{y}\varphi(y)=0
\end{equation}
η οποία είναι μαι γραμμική φιαφορική εξίσωση πρώτης τάξης. Η γενική λύσης της \eqref{a52} θα δίνεται από τον τύπο
\[ \varphi(y)=c'e^{-\dintt{\frac{2}{y}dy}}=c'e^{-\log\left(y^2 \right) }=\frac{c'}{y^2} \]
όπου $ c' $ είναι μια αυθαίρετη σταθερά. Επιλέγουμε χωρίς βλάβη της γενικότητας $ \varphi(y)=\frac{c'}{y^2} $ η οποία μας δίνει τον ζητούμενο ολοκληρωτικό παράγοντα $ \rho(x,y)=x^n\varphi(y)=\frac{x}{y^2} $.
Πολλαπλασιάζοντας τώρα την αρχική εξίσωση \eqref{a50} με τον ολοκληρωτικό παράγοντα που μόλις υπολογίσαμε παίρνουμε την αμέσως ολοκληρώσιμη εξίσωση
\begin{gather}\label{a53}
\dfrac{x}{y^2}\left(2y^3-3xy\right) dx+\dfrac{x}{y^2}\left( x^2+xy^2\right) dy=0\Rightarrow\\
\undercbrace{\left(2xy-\frac{3x^2}{y}\right)}_{M}dx+\undercbrace{\left(\frac{x^3}{y^2}+x^2\right)}_{N} dy=0
\end{gather} 
Εύκολα επαληθεύουμε ότι ισχύει $ \frac{\partial M}{\partial y}=\frac{\partial N}{\partial x} $ άρα θα υπάρχει μια συνάρτηση $ f(x,y) $ ώστε να ισχύει $ df(x,y)=Mdx+Ndy=0 $. Τότε οι λύσεις της εξίσωσης θα δίνονται από τον τύπο $ f(x,y)=c $. Έχουμε λοιπόν ότι
\begin{equation}\label{a54}
\frac{\partial f}{\partial x}=2xy-\frac{3x^2}{y}\ \textrm{ και }\ \frac{\partial f}{\partial y}=\frac{x^3}{y^2}+x^2
\end{equation}
Προκύπτει ότι
\[ f(x,y)=\int{\left(2xy-\frac{3x^2}{y} \right)dx+g(y) }=x^2y-\frac{x^3}{y}+g(y) \]
για κάποια συνάρτηση $ g(y) $. Παραγωγίζοντας την παραπάνω σχέση ως προς $ y $ έχουμε :
\[ \frac{\partial f}{\partial y}=x^2-\frac{x^3}{y^2}+g'(y) \]
Εξισώνοντας τη δεύτερη σχέση της \eqref{a54} με την προηγούμενη παίρνουμε ότι 
\[ x^2-\frac{x^3}{y^2}+g'(y)=\frac{x^3}{y^2}+x^2\Rightarrow g'(y)=0\Rightarrow g(y)=c' \]
όπου $ c' $ είναι μια αυθαίρετη σταθερά. Επιλέγοντας δίχως βλάβη της γενικότητας όπου $ g(y)=0 $ αποκτάμε τη συνάρτηση $ f(x,y)=x^2y-\frac{x^3}{y} $ οπότε όλες οι λύσεις της εξίσωσης δίνονται από τους τύπους
\[ y=0\ \textrm{ και }\ x^2y-\frac{x^3}{y}=c \]\
\begin{Askhshs}[A]
\bmath{Να επιλυθούν τα προβλήματα αρχικών τιμών.
\begin{alignat*}{4}
\textrm{i.} &\ \  \dfrac{dy}{dx}-\dfrac{1}{x}y\log{y}=-\dfrac{y}{2\log{y}} & ,\ y(-1)=e^2\\\textrm{ii.} &\ \  \dfrac{dy}{dx}=-\dfrac{(1+y)^2}{x-x^2+xy} & ,\ y(1)=1
\end{alignat*}}
\end{Askhshs}\mbox{}\\
\lysh
\begin{rlist}
\item Η αρχική διαφορική εξίσωση γράφεται στη μορφή
$ \frac{y'}{y}-\frac{1}{x}\log{y}=-\frac{1}{2\log{y}} $ και θέτοντας σ' συτήν όπου $ z=\log{y}\Rightarrow z'=\frac{y'}{y} $ αυτή μετατρέπεται ως εξής:
\begin{equation}\label{a6}
z'-\frac{1}{x}z=-\frac{z^{-1}}{2}
\end{equation}
η οποία είναι μια διαφορική εξίσωση Bernoulli με $ r=-1 $. Εκτελούμε λοιπόν το μετασχηματισμό $ u=z^{1-(-1)}=z^2 $ που δίνει $ u'=2zz' $, με τον οποίο μετατρέπουμε την \eqref{a6} σε μια γραμμική διαφορική εξίσωση 1ης τάξης :
\begin{equation}
\frac{1}{2}u'-\frac{1}{x}u=-\frac{1}{2}\Rightarrow u'-\frac{2}{x}u=-1
\end{equation}
Η γενική λύση αυτής θα είναι :
\begin{align*}
u(x)&=e^{-\dintt{\left(-\frac{2}{x}\right) dx}}\left[c+\int{(-1)\cdot e^{\dintt{\left( -\frac{2}{x}\right) dx}}dx} \right]=\\
&=e^{\log{x^2}}\left[c-\int{e^{\log{\frac{1}{x^2}}}dx}\right]=x^2\left( c-\int{\dfrac{dx}{x^2}}\right)=x^2\left(c+\frac{1}{x} \right)=cx^2+x
\end{align*}
όπου $ c $ είναι μια αυθαίρετη σταθερά. Με αναδρομική αντικατάσταση όλες οι λύσεις $ y $ θα δίνονται από τον τύπο
\begin{gather}
u(x)=cx^2+x\Rightarrow z^2(x)=cx^2+x\Rightarrow z(x)=\pm\sqrt{cx^2+x}\Rightarrow\\
\log{y}=\pm\sqrt{cx^2+x}\Rightarrow y(x)=e^{\pm\sqrt{cx^2+x}}
\end{gather}
Σύμφωνα τώρα με την αρχική συνθήκη $ y(-1)=e^2 $ θα προκύψει :
\[ y(-1)=e^2\Rightarrow e^2=e^{\sqrt{c(-1)^2-1}}\Rightarrow c-1=4\Rightarrow c=5 \]
Η τιμή αυτή μας δίνει τη λύση του προβλήματος αρχικών τιμών η οποία θα είναι 
\[ y(x)=e^{\sqrt{5x^2+x}} \]
\item Η αρχική διαφορική εξίσωση έχει λύση την $ y=-1 $  η οποία όμως δεν ικανοποιεί την αρχική συνθήκη $ y(1)=1 $. Θέτοντας τώρα $ z=1+y\Rightarrow z'=y' $ η εξίσωση παίρνει τη μορφή 
\begin{equation}\label{a61}
z'=-\dfrac{z^2}{z-x^2}
\end{equation}  και πρόκειται για μια ομογενή εξίσωση με βαθμό ομογένιας $ 2 $. Η τελευταία έχει λύση την $ z=0\Rightarrow y=-1 $ την οποία όμως έχουμε απορρίψει προηγουμένως. Με το μετασχηματισμό $ z=xu $ ο οποίος δίνει $ z'=xu'+u $ η \eqref{a61} παίρνει τη μορφή
\begin{align}\label{a2}
z'=-\dfrac{z^2}{z-x^2}&\Rightarrow xu'+u=-\frac{x^2u^2}{x^2u-x^2}\Rightarrow\\&\Rightarrow xu'+u=\dfrac{u^2}{1-u}\Rightarrow\\
&\Rightarrow xu'=\dfrac{2u^2-u}{1-u}\Rightarrow \dfrac{1-u}{2u^2-u}du=\dfrac{dx}{x}
\end{align}
Φτάσαμε σε μια διαφοριική εξίσωση χωριζομένων μεταβλητών οπότε με άμεση ολοκλήρωση και στα δύο μέλη της \eqref{a2} θα έχουμε
\begin{align*} &\int\dfrac{1-u}{2u^2-u}du=\int\dfrac{dx}{x}+c'\Rightarrow -\int\dfrac{1}{u}du+\int\dfrac{1}{2u-1}du=\int\dfrac{dx}{x}+c'\Rightarrow\\
&-\log|u|+\frac{1}{2}\log|2u-1|=\log|x|+c'\Rightarrow \log{\left| \dfrac{2u-1}{x^2u^2}\right|}=2c'\Rightarrow\\
&\dfrac{2u-1}{x^2u^2}=\pm e^{2c'}\ \textrm{ και θέτοντας }\pm e^{2c'}=c\ \textrm{ παίρνουμε }\dfrac{2u-1}{x^2u^2}=c
\end{align*}
Οι λύσεις της διαφορικής εξίσωσης θα δίνονται από τον τύπο
\[ \dfrac{2u-1}{x^2u^2}=c\xRightarrow{u=\frac{z}{x}}\dfrac{2\frac{z}{x}-1}{x^2\left(
\frac{z}{x}\right)^2}=c\xRightarrow{z=y+1}\dfrac{2\frac{y+1}{x}-1}{(y+1)^2}=c \]
Από την παραπάνω σχέση οι λύσεις της εξίσωσης θα δίνονται από τον τύπο :
\begin{gather}
2\frac{y+1}{x}-1=c(y+1)^2\Rightarrow cx(y+1)^2-2(y+1)+x=0\Rightarrow\\\label{a62}
y=\dfrac{1-cx\pm\sqrt{1-cx^2}}{cx}
\end{gather}
Από την αρχική συνθήκη του προβλήματος $ y(1)=1 $ υπολογίζουμε την τιμή της σταθεράς $ c $ :
\[ c1(1+1)^2-2(1+1)+1=0\Rightarrow 4c-3=0\Rightarrow c=\frac{3}{4} \]
η οποία συμφωνεί με την αρχική συνθήκη μόνο τη λύση για τη θετική ρίζα $ (+) $ της \eqref{a62} οπότε και αποκτάμε τον τύπο που μας δίνει τη λύση του προβλήματος αρχικών τιμών :
\[ y=\dfrac{4-3x+\sqrt{4-3x^2}}{3x} \]
\end{rlist}
\begin{Askhshs}[A]
\bmath{Να επιλυθεί η διαφορική εξίσωση \[ (x+2y-3)y'+x-y+3=0 \]
με τη βοήθεια ενός μετασχηματισμού της μορφής $ t=x+a $, $ z=y+\beta $ (όπου $ a $ και $ \beta $ είναι σταθερέςπου πρέπει να προσδιοριστούν).}
\end{Askhshs}\mbox{}\\
\lysh
Η διαφορική εξίσωση μπορεί να γραφτεί στην ακόλουθη μορφή \[ \frac{dy}{dx}=\dfrac{-x+y-3}{x+2y-3} \]
Θέτουμε $ t=x+a\Rightarrow x=t-a $ και $ z=y+\beta\Rightarrow y=z-\beta $. Επιπλέον θα ισχύει ότι  \[\frac{dy}{dx}=\frac{dy}{dz}\cdot\frac{dz}{dt}\cdot\frac{dt}{dx}=1\cdot\frac{dz}{dt}\cdot1=\frac{dz}{dt} \]
Σύμφωνα με τα παραπάνω η εξίσωση γίνεται
\begin{gather}
\frac{dz}{dt}=\dfrac{-(t-a)+(z-\beta)-3}{(t-a)+2(z-\beta)-3}\Rightarrow \frac{dz}{dt}=\dfrac{-t+a+z-\beta-3}{t-a+2z-2\beta-3}\Rightarrow\\\label{a7}
\frac{dz}{dt}=\dfrac{-t+z+(a-\beta-3)}{t+2z+(-a-2\beta-3)}
\end{gather}
Μπορούμε να επιλέξουμε κατάλληλες τιμές για τα $ a $ και $ \beta $ ώστε η \eqref{a7} να γίνει ομογενής διαφορική εξίσωση με βαθμό ομογένειας $ 1 $. Λύνουμε λοιπόν το σύστημα
\[ \systeme[a\beta]{a-\beta-3=0,-a-2\beta-3=0}\xRightarrow{(+)}-3\beta-6=0\Rightarrow \beta=-2\Rightarrow a=1 \]
Οι τιμές αυτές των σταθερών $ a $ και $ \beta $ μας δίνουν τους μετασχηματισμούς $ x=t-1 $ και $ y=z+2 $ και καταλήγουμε στην ομογενή εξίσωση :
\begin{equation}\label{a71}
\frac{dz}{dt}=\dfrac{-t+z}{t+2z}
\end{equation}
Θέτουμε λοιπόν σ' αυτήν $ z=tu\Rightarrow z'=tu'+u $ και παίρνουμε :
\begin{align}
tu'+u=\dfrac{-t+tu}{t+2tu}&\Rightarrow tu'=\dfrac{t(-1+u)}{t(1+2u)}-u\Rightarrow\\
&\Rightarrow tu'=-\dfrac{2u^2+1}{1+2u}\Rightarrow \dfrac{1+2u}{2u^2+1}du=-\frac{dt}{t}
\end{align}
Φτάσαμε σε μια εξίσωση χωριζομένων μεταβλητών. Οι λύσεις της θα δίνονται από τον τύπο \[ \int\dfrac{1+2u}{2u^2+1}du=-\int\frac{dt}{t}+c' \]
όπου $ c' $ είναι μια αυθαίρετη σταθερά. Έτσι έχουμε :
\begin{gather}
\int\dfrac{1+2u}{2u^2+1}du=-\int\frac{dt}{t}+c' \Rightarrow \int\dfrac{1}{2u^2+1}du+\int\dfrac{2u}{2u^2+1}du=-\int\frac{dt}{t}+c'\Rightarrow\\
\frac{\sqrt{2}}{2}\arctan{\left( \sqrt{2}u\right) }+\frac{1}{2}\log\left|2u^2+1 \right|=-\log|t|+c'\Rightarrow \\
\sqrt{2}\arctan{\left( \sqrt{2}u\right) }+\log\left|t^2\left( 2u^2+1\right)  \right|=2c'
\end{gather}
Αντικαθιστώντας $ u=\frac{z}{t},\ z=y-2 $ και $ t=x+1 $ και θέτοντας $ c=2c' $ οι λύσεις της εξίσωσης θα δίνονται από τον τύπο
\[ \sqrt{2}\arctan{\left( \sqrt{2}\frac{y-2}{x+1}\right) }+\log\left[(x+1)^2+(y-2)^2  \right]=c \]
όπου $ c $ είναι μια αυθαίρετη σταθερά.\\\\\\
\begin{Askhshs}[A]
\bmath{Να επιλυθεί το πρόβλημα αρχικών τιμών
\[ y'+x+y+1=(x+y)^2e^{2x}\ ,\ y(0)=1 \]}
\end{Askhshs}\mbox{}\\
\lysh
Η διαφορική εξίσωση αυτή έχει λύση την $ y=-x $ η οποία όμως δεν πληροί την αρχική συνθήκη του προβλήματος αφού $ y(0)=1\Rightarrow 0=1 $. Για να βρούμε τις υπόλοιπες λύσεις χρησιμοποιούμε το μετασχηματισμό $ z=x+y $ από τον οποίο παίρνουμε $ z'=y'+1 $. Έτσι η εξίσωση θα πάρει τη μορφή
\begin{equation}\label{a8}
z'+z=z^2e^{2x}
\end{equation}
η οποία είναι μια διαφορική εξίσωση Bernoulli με $ r=2 $. Η \eqref{a8} έχει λύση την $ z=0 $ η οποία ισοδυναμεί με την $ y=-x $ την οποία έχουμε απορρίψει διότι δεν πληροί την αρχική συνθήκη. Η τελευταία εξίσωση όμως γράφεται στη μορφή 
\[ z^{-2}\cdot z'+z\cdot z^{-2}=z^{-2}\cdot z^2e^{2x}\Rightarrow z^{-2}\cdot z'+ z^{-1}=e^{2x} \]
Σ' αυτήν θέτουμε $ u=z^{1-2}=z^{-1}\Rightarrow u'=-z^{-2}z' $ και παίρνουμε τη γραμμική εξίσωση 1ης τάξης :

\begin{equation}
u'-u=-e^{2x}
\end{equation}
Η γενική λυση αυτής θα δίνεται από τον τύπο 
\begin{align}
u(x)&=e^{-\int{(-1)dx}}\left[ c+\int{-e^{2x}\cdot e^{\int{(-1)dx}}dx}\right]=\\
&=e^{x}\left(c+\int{-e^{2x}\cdot e^{-x}dx} \right)= e^{x}\left(c-\int{e^{x}dx} \right)=e^{x}\left(c-e^{x}dx \right)=ce^x-e^{2x}
\end{align}
όπου $ c $ είναι μια αυθαίρετη σταθερά. Αντικαθιστώντας ξανά $ u=z^{-1} $ και $ z=x+y $ στην προηγούμενη σχέση παίρνουμε τους τύπους που μας δίνουν όλες τις λύσεις της εξίσωσης:
\[ y=-x\ \textrm{ και }\ y=-x+\frac{1}{ce^x-e^{2x}} \]
Όπως αναφέραμε και προηγουμένως η λύση $ y=-x $ δεν πληροί την αρχική συνθήκη του προβλήματος, ενώ από τον τύπο $ y=-x+\frac{1}{ce^x-e^{2x}} $ παίρνουμε :
\[ y(0)=1\Rightarrow \frac{1}{ce^0-e^{2\cdot0}}=1\Rightarrow c=2 \]
Έτσι η λύση του προβλήματος αρχικών τιμών θα είναι η 
\[ y=-x+\frac{1}{2e^x-e^{2x}} \]\mbox{}\\
\begin{Askhshs}[A]
\bmath{Με τη βοήθεια του μετασχηματισμού $ z=x+y $ να επιλυθεί η διαφορική εξίσωση
\[ y'=(x+y)\left( x^4+2x^3y+x^2y^2-1\right)-1  \]}
\end{Askhshs}\mbox{}\\
\lysh
Η εξίσωση αυτή έχει λύση την $ y=-x $. Για τις υπόλοιπες λύσεις γράφεται ισοδύναμα 
\begin{gather}
y'=(x+y)\left( x^4+2x^3y+x^2y^2-1\right)-1\Rightarrow\\
y'=(x+y)\left[x^2\left(x^2+2xy+y^2\right) -1\right]-1\Rightarrow\\\label{a9}
y'+1=(x+y)\left[x^2\left(x+y\right)^2-1\right]
\end{gather}
Ο μετασχηματισμός $ z=x+y\Rightarrow z'=y'+1 $ φέρνει την \eqref{a9} στη μορφή
\begin{equation}\label{a91}
z'=xz(x^2z^2-1)\Rightarrow z'+xz=z^3x^3
\end{equation}
Η τελευταία είναι μια διαφορική εξίσωση Bernoulli με $ r=3 $. Μια λύση αυτής είναι η $ z=0 $ η οποία αντιστοιχεί στην $ y=-x $ που είδαμε προηγουμένως. Επιπλέον θέτοντας $ u=z^{1-3}=z^{-2} $ παίρνουμε $ u'=-2z^{-3}z' $ οπότε η \eqref{a91} μετατρέπεται σε μια γραμμική διαφορική εξίσωση 1ης τάξης :
\begin{equation}
z'+xz=z^3x^3\Rightarrow z^{-3}z'+xz^{-2}=x^3\Rightarrow u'-2xu=-2x^3
\end{equation}
Η γενική λύση αυτή θα δίνεται από τον τύπο
\begin{align*}
u(x)&=e^{-\int{-2x}dx}\left[c+\int{-2x^3\cdot e^{\int{-2x}dx}dx}\right]=\\
&=e^{x^2}\left[c+\int{x^2\cdot (-2x)\cdot e^{-x^2}dx}\right]=e^{x^2}\left[c+\int{x^2\cdot d\left(e^{-x^2}\right) }\right]=\\
&=e^{x^2}\left(c+x^2\cdot e^{-x^2}-\int{2x\cdot e^{-x^2}dx }\right)=e^{x^2}\left(c+x^2\cdot e^{-x^2}-e^{-x^2}\right)
\end{align*}
όπου $ c $ είναι μια αυθαίρετη σταθερά. Αντικαθιστώντας αναδρομικά στην τελευταία σχέση τους μετασχηματισμούς που χρησιμοποιήσαμε έχουμε :
\[ u(x)=e^{x^2}\left(c+x^2\cdot e^{-x^2}-e^{-x^2}\right)\xRightarrow{u=z^{-2}\ ,\ z=y+x}y=-x\pm\dfrac{1}{\sqrt{ce^{x^2}+x^2+1}} \]
Όλες οι λύσεις της αρχικής διαφορικής εξίσωσης θα δίνονται από τους παρακάτω τύπους :
\[ y=-x\ \textrm{ και }\ y=-x\pm\dfrac{1}{\sqrt{ce^{x^2}+x^2+1}} \]\mbox{}\\\\\\
\begin{Askhshs}[A]
\bmath{Να επιλυθεί η διαφορική εξίσωση
\[ \left( y-xy+y^3\cos{y}\right)y'+xy^3+y^2=0  \]}
\end{Askhshs}\mbox{}\\
\lysh
Η εξίσωση αυτή έχει προφανή λύση την $ y=0 $. Για να βρούμε τις υπόλοιπες λύσεις την γράφουμε ισοδύναμα στη μορφή :
\begin{equation}\label{a10}
\undercbrace{\left( xy^3+y^2\right)}_{M}dx+\undercbrace{\left(y-xy+y^3\cos{y}\right)}_{N}dy=0
\end{equation} 
Οι συναρτήσεις $ M(x,y)=xy^3+y^2 $ και $ N(x,y)=y-xy+y^3\cos{y} $ είναι συνεχείς και έχουν συνεχείς μερικές παραγώγους. Εξετάζουμε στη συνέχεια αν η \eqref{a10} αποτελεί μια αμέσως ολοκληρώσιμη εξίσωση. Παρατηρούμε ότι 
\[ \frac{\partial M}{\partial y}=3xy^2+2y\neq -y=\frac{\partial N}{\partial x} \] κάτι που σημαίνει ότι η εξίσωση δεν είναι αμέσως ολοκληρώσιμη άρα θα εξετάσουμε την ύπαρξη ενός ολοκληρωτικού παράγοντα. Έχουμε :
\[ \dfrac{\frac{\partial N}{\partial x}-\frac{\partial M}{\partial y}}{M}=\dfrac{-y-3xy^2-2y}{xy^3+y^2}=\dfrac{-3\left(xy^2+y \right) }{y\left( xy^2+y\right) }=-\frac{3}{y} \]
η οποία είναι μια συνάρτηση με μοναδική μεταβλητή το $ y $ οπότε ένας ολοκληρωτικός παράγοντας θα είναι ο $ \rho(y)=e^{\dintt{-\frac{3}{y}dy}}=\frac{1}{y^3} $. Πολλαπλασιάζοντας μ' αυτόν την \eqref{a10} θα προκύψει η αμέσως ολοκληρώσιμη εξίσωση 
\begin{equation}\label{a101}
\left( x+\frac{1}{y}\right)dx+\left(\frac{1}{y^2}-\frac{x}{y^2}+\cos{y}\right)dy=0
\end{equation}
Θα υπάρχει λοιπόν μια συνάρτηση $ f(x,y) $ ώστε η παραπάνω εξίσωση να γραφτεί στη μορφή $ df(x,y)=Mdx+Ndy=0 $. Οι λύσεις της θα δίνονται από τον τύπο $ f(x,y)=c $. Θα έχουμε
\[ \frac{\partial f}{\partial x}=x+\frac{1}{y}\ \textrm{ και }\ \frac{\partial f}{\partial y}=\frac{1}{y^2}-\frac{x}{y^2}+\cos{y} \]
Έτσι προκύπτει
\[ f(x,y)=\int{\left( x+\frac{1}{y}\right) dx}+g(y)=\frac{x^2}{2}+\frac{x}{y}+g(y) \]
για κάποια συνάρτηση $ g(y) $. Τότε
\begin{gather*}
\frac{\partial f}{\partial y}=-\frac{x}{y^2}+g'(y)\eq{\eqref{a101}}\frac{1}{y^2}-\frac{x}{y^2}+\cos{y}\Rightarrow\\
g'(y)=\frac{1}{y^2}+\cos{y}\Rightarrow g(y)=-\frac{1}{y}+\sin{y}+c'
\end{gather*} και επιλέγουμε δίχως βλάβη της γενικότητας $ g(y)=-\frac{1}{y}+\sin{y} $.Συνεπώς η συνάρτηση $ f $ θα είναι $ f(x,y)=\frac{x^2}{2}+\frac{x}{y}-\frac{1}{y}+\sin{y} $ άρα οι λύσεις της εξίσωσης θα δίνονται από τον τύπο 
\[ \frac{x^2}{2}+\frac{x}{y}-\frac{1}{y}+\sin{y}=c \]
όπου $ c $ είναι μια αυθαίρετη σταθερά.\epask
\begin{Askhshs}[A]
\bmath{Να επιλυθεί η διαφορική εξίσωση
\[ \left(2x^2+x^3y+y\right)dx+\left(x+4xy^4+8y^3\right)dy=0\quad,\ x>0,\ y>0   \]
αφού βρεθεί ένας ολοκληρωτικός παράγοντας της μορφής $ \rho(x,y)=\varphi(xy) $ (όπου $ \varphi $ είναι μια συνάρτηση που πρέπει να προσδιοριστεί).}
\end{Askhshs}\mbox{}\\
\lysh
Η συνάρτηση $ \rho(x,y)=\varphi(xy) $ είναι ένας ολοκληρωτικός παράγοντας αν και μόνο αν ισχύει $ \frac{\partial M}{\partial y}=\frac{\partial N}{\partial x} $. Πολλαπλασιάζοντας την αρχική εξίσωση με τη συνάρτηση $ \rho(x,y) $ προκύπτει
\begin{equation}\label{a11}
\undercbrace{\varphi(xy)\left(2x^2+x^3y+y\right)}_{M}dx+\undercbrace{\varphi(xy)\left(x+4xy^4+8y^3\right)}_{N}dy=0
\end{equation}
Υπολογίζοντας τις παταγώγους $ \frac{\partial M}{\partial y},\ \frac{\partial N}{\partial x} $ θα έχουμε :
\begin{align}
&\frac{\partial M}{\partial y}=x\varphi'(xy)\left(2x^2+x^3y+y\right)+\varphi(xy)\left(x^3+1\right)\ \textrm{ και }\\
&\frac{\partial N}{\partial x}=y\varphi'(xy)\left(x+4xy^4+8y^3\right)+\varphi(xy)\left(1+4y^4\right)
\end{align}
οπότε απαιτώντας να ισχύει $ \frac{\partial M}{\partial y}=\frac{\partial N}{\partial x} $ παίρνουμε :
\begin{align*}
\frac{\partial M}{\partial y}=\frac{\partial N}{\partial x}
&\Rightarrow x\varphi'(xy)\left(2x^2+x^3y+y\right)+\varphi(xy)\left(x^3+1\right)=\\
&\phantom{\Rightarrow}=y\varphi'(xy)\left(x+4xy^4+8y^3 \right)+\varphi(xy)\left(1+4y^4\right)\Rightarrow\\
&\Rightarrow \varphi'(xy)\left(2x^3+x^4y+xy\right)-\varphi'(xy)\left(xy+4xy^5+8y^4\right)=\\
&\phantom{\Rightarrow}=\varphi(xy)\left(1+4y^4\right)-\varphi(xy)\left(x^3+1\right)\Rightarrow\\
&\Rightarrow \varphi'(xy)\left(2x^3+x^4y-4xy^5-8y^4\right)=\varphi(xy)\left(4y^4-x^3\right)\Rightarrow\\
&\Rightarrow\varphi'(xy)\left(x^3(2+xy)-4y^4(xy+2)\right)=\varphi(xy)\left(4y^4-x^3\right)\Rightarrow\\
& \Rightarrow\varphi'(xy)(xy+2)\left(x^3-4y^4\right)-\varphi(xy)\left(4y^4-x^3\right)=0
\end{align*}
Διαιρώντας και τα δύο μέλη της τελευταίας σχέσης με την παράσταση $ x^3-4y^4 $ παίρνουμε την ομογενή γραμμική εξίσωση 1ης τάξης :
\begin{equation}
\varphi'(xy)(xy+2)+\varphi(xy)=0\Rightarrow \varphi'(xy)+\frac{1}{xy+2}\cdot\varphi(xy)=0
\end{equation}
Θέτοντας $ xy=z $ γενική λύση αυτής θα δίνεται από τον τύπο
\begin{gather*}
\varphi(z)=ce^{-\dintt{\frac{1}{z+2}dz}}=ce^{-\log{z+2}}=\frac{c}{z+2}
\end{gather*}
Επομένως ο ζητούμενος ολοκληρωτιοκός παράγοντας τα είναι ο $ \rho(x,y)=\varphi(xy)=\frac{c}{xy+2} $. Μπορούμε χωρίς βάβη της γενικότητας να επιλέξουμε $ c=1 $ και να έχουμε $ \rho(x,y)=\frac{1}{xy+2} $. Έτσι η εξίσωση \eqref{a11} θα πάρει τη μορφή 
\begin{gather}
\frac{1}{xy+2}\left(2x^2+x^3y+y\right)dx+\frac{1}{xy+2}\left(x+4xy^4+8y^3\right)dy=0\Rightarrow\nonumber\\ \undercbrace{\left(x^2+\frac{y}{xy+2}\right)}_{M}dx+\undercbrace{\left(\frac{x}{xy+2}+4y^3\right)}_{N}dy=0\label{a111}
\end{gather}
Εύκολα διαπιστώνουμε ότι η \eqref{a111} είναι μια αμέσως ολοκληρώσιμη εξίσωση αφού
\[ \dfrac{\partial M}{\partial y}=\dfrac{\partial N}{\partial x}=\frac{2}{(xy+2)^2} \]
Θα υπάρχει λοιπόν μια συνάρτηση $ f(x,y) $ ώστε η \eqref{a111} να ισχύει $ df(x,y)=Mdx+Ndy=0 $. Οι λύσεις της εξίσωσης θα δίνονται από τον τύπο $ f(x,y)=c $.
Σύμφωνα με τα παραπάνω θα ισχύει 
\[ \frac{\partial f}{\partial x}=M=x^2+\frac{y}{xy+2}\ \textrm{ και }\ \frac{\partial f}{\partial y}=N=\frac{x}{xy+2}+4y^3 \]
Ολοκληρώνοντας την πρώτη σχέση ως προς $ x $ προκύπτει
\begin{gather*}
f(x,y)=\int{\left(x^2+\frac{y}{xy+2}\right)dx}+g(y)=\frac{x^3}{3}+\log(xy+2)+g(y)
\end{gather*}
Παραγωγίζουμε την τελευταία σχέση ως προς τη μεταβλητή $ y $ και εξισώνουμε την παράσταση που θα προκύψει με τη συνάρτηση $ N(x,y) $. Έχουμε λοιπόν :
\begin{gather*}
\frac{\partial f}{\partial y}=\frac{x}{xy+2}+g'(y)=\frac{x}{xy+2}+4y^3\Rightarrow\\
g'(y)=4y^3\Rightarrow g(y)=y^4+c'
\end{gather*}
όπου $ c' $ είναι μια αυθαίρετη σταθερά. Επιλέγοντας $ c'=0 $ αποκτάμε τη ζητούμενη συνάρτηση 
\[ f(x,y)=\frac{x^3}{3}+\log(xy+2)+y^4 \]
οπότε όλες οι λύσεις της εξίσωσης θα δίνονται από τον τύπο 
\[ \frac{x^3}{3}+\log(xy+2)+y^4=c \]
όπου $ c $ είναι μια αυθαίρετη σταθερά.\epask
\begin{Askhshs}[A]
\bmath{Ας είναι $ a,\beta,\gamma,a_1,\beta_1,\gamma_1 $ σταθερές με $ a\beta_1-a_1\beta\neq0 $ και ας θεωρήσουμε τη λύση $ (x_0,y_0) $ του συστήματος 
\[ \ccases{ax_0+\beta y_0+\gamma=0\\a_1x_0+\beta_1 y_0+\gamma_1=0} \]
Με τη βοήθεια της αντικατάστασης $ X=x-x_0,\ Y=y-y_0 $ να επιλυθεί η διαφορική εξίσωση 
\[ \frac{dy}{dx}=\frac{ax+\beta y+\gamma}{a_1x+\beta_1y+\gamma_1} \]
\textit{Εφαρμογή} : Να επιλυθεί η διαφορική εξίσωση
\[ \frac{dy}{dx}=\frac{-x+y-3}{x+2y-3} \]}
\end{Askhshs}\mbox{}\\
\lysh
Για την επίλυση της εξίσωσης $ \frac{dy}{dx}=\frac{ax+\beta y+\gamma}{a_1x+\beta_1y+\gamma_1} $ διακρίνουμε τις εξής περιπτώσεις.
\begin{rlist}
\item Εαν $ (\gamma,\gamma_1)=(0,0) $ τότε πρόκειται για μια ομογενή διαφορική εξίσωση με βαθμό ομογένειας 1 δηλαδή την 
\begin{equation}\label{a12}
\frac{dy}{dx}=
\frac{ax+\beta y}{a_1x+\beta_1 y}
\end{equation} 
\item Στην περίπτωση όπου $ (\gamma,\gamma_1)\neq(0,0) $ τότε αναπτύσουμε την εξής μέθοδο για την επίλυσή της. Έχουμε :
\[ \frac{dy}{dx}=\frac{dy}{dY}\cdot\frac{dY}{dX}\cdot\frac{dX}{dx}=1\cdot\frac{dY}{dX}\cdot1=\frac{dY}{dX} \]
Σύμφωνα μ' αυτό και με τη βοήθεια της αντικατάστασης $ X=x-x_0,\ Y=y-y_0 $ η αρχική εξίσωση θα πάρει τη μορφή :
\[ \frac{dY}{dX}=\frac{a\left(X+x_0\right) +\beta \left(Y+y_0\right)+\gamma}{a_1\left(X+x_0\right)+\beta_1\left(Y+y_0\right)+\gamma_1}=
\frac{aX+\beta Y+(ax_0+\beta y_0+\gamma)}{a_1X+\beta_1 Y+(a_1x_0+\beta_1 y_0+\gamma_1)} \]
Από την υπόθεση έχουμε γνωστό ότι $ ax_0+\beta y_0+\gamma=0 $ και $ a_1x_0+\beta_1 y_0+\gamma_1=0 $ και αυτό μας μετατρέπει την εξίσωση στην ακόλουθη ομογενή διαφορική εξίσωση 
\begin{equation}\label{a121}
\frac{dY}{dX}=
\frac{aX+\beta Y}{a_1X+\beta_1 Y}
\end{equation}
με βαθμό ομογένειας 1. Παρατηρούμε ότι η \eqref{a121} είναι της ίδιας μορφής με την \eqref{a12} στην περίπτωση i.
\end{rlist}
\textit{\textbf{Εφαρμογή : }}\\
Η διαφορική εξίσωση $ \frac{dy}{dx}=\frac{-x+y-3}{x+2y-3} $ είναι της ίδιας μορφής με την αρχική από την οποία παίρνουμε τους συντελεστές : $ a=-1, \beta=1, \gamma=-3 $ και $ a_1=1, \beta_1=2, \gamma_1=-3 $. Με τους συντελεστές αυτούς αποκτάμε το σύστημα
\[ \systeme{-x_0+y_0-3=0,x_0+2y_0-3=0} \]
της οποίας η λύση θα είναι η $ (x_0,y_0)=(-1,2) $. Ακολυθώντας την ίδια διαδικασία επίλυσης με προηγουμένως χρησιμοποιώντας τις αντικαταστάσεις $ x=X-1 $ και $ y=Y+2 $ θα καταλήξουμε στην ομογενή εξίσωση 
\begin{equation}\label{a122}
\frac{dY}{dX}=
\frac{-X+Y}{X+2Y}
\end{equation}
Έτσι λοιπόν θέτουμε $ Y=XZ $ και παίρνουμε $ Y'=XZ'+Z $. Αντικαθιστώντας τις σχέσεις αυτές στην \eqref{a122} προκύπτει :
\begin{align*}
XZ'+Z=\frac{-X+XZ}{X+2XZ}&\Rightarrow XZ'+Z=\frac{-1+Z}{1+2Z}\Rightarrow\\
&\Rightarrow XZ'=\frac{-1-2Z^2}{1+2Z}\Rightarrow
\Rightarrow\frac{dZ}{dX}\cdot\frac{1+2Z}{-1-2Z^2}=\frac{1}{X}\Rightarrow \frac{1+2Z}{1+2Z^2}dZ=-\frac{dX}{X}
\end{align*}
Οι λύσεις της εξίσωσης θα δίνονται από τον τύπο :
\begin{gather*}
\int\frac{1+2Z}{1+2Z^2}dZ=\int-\frac{dX}{X}+c'
\Rightarrow
\int\frac{1}{1+2Z^2}dZ+\int\frac{2Z}{1+2Z^2}dZ=-\int\frac{dX}{X}+c'\Rightarrow\\
\frac{\sqrt{2}}{2}\arctan{\left(\sqrt{2}Z \right)}+\frac{1}{2}\log{\left| 1+Z^2\right|}=-\log{|X|+c'}\Rightarrow\\
\sqrt{2}\arctan{\left(\sqrt{2}Z \right)}+\log{\left| \left( 1+Z^2\right) X^2\right|}=2c'
\end{gather*}
όπου $ c' $ είναι μια αυθαίρετη σταθερά. Θέτοντας στην τελευταία σχέση $ Z=Y/X $ και $ X=x+1,Y=y-2 $ τότε παίρνουμε τον τύπο από τον οποίο θα δίνονται όλες οι λύσεις της εξίσωσης :
\[ \sqrt{2}\arctan{\left(\sqrt{2}\frac{y-2}{x-1} \right)}+\log{\left[(x+1)^2+(y-2)^2\right]}=c \] όπου $ c $ είναι μια αυθαίρετη σταθερά με $ c=2c' $.\epask
\begin{Askhshs}[A]
\bmath{Με τη βοήθεια ενός μετασχηματισμού της μορφής $ X=x-a $, $ Y=y-\beta $ (όπου $ a $ και $ \beta $ κατάλληλοι αριθμοί που θα πρέπει να προσδιοριστούν), να επιλυθεί η διαφορική εξίσωση :
\[ \dfrac{dy}{dx}=\dfrac{x+y+1}{x+2}-e^{\frac{x+y+1}{x+2}} \]}
\end{Askhshs}\mbox{}\\
\lysh
Θέτοντας $ X=x-a\Rightarrow x=X+a $ και $ Y=y-\beta\Rightarrow y=Y+\beta $ θα ισχύει
\[ \dfrac{dy}{dx}=\dfrac{dy}{dY}\cdot\dfrac{dY}{dX}\cdot\dfrac{dX}{dx}=1\cdot\dfrac{dY}{dX}\cdot1=\dfrac{dY}{dX} \]
Σύμφωνα με τις παραπάνω σχέσεις λοιπόν η εξίσωση θα πάρει τη μορφή
\begin{equation}\label{a13}
\dfrac{dY}{dX}=\dfrac{X+Y+(a+\beta+1)}{X+(a+2)}-e^{\frac{X+Y+(a+\beta+1)}{X+(a+2)}}
\end{equation} 
Με κατάλληλες τιμές για τους αριθμούς $ a $ και $ \beta $ μπορούμε να μετατρέψουμε την \eqref{a13} σε μια ομογενή διαφορική εξίσωση με βαθμό ομογένειας $ 1 $. Γι αυτό θα πρέπει να ισχύει :
\[ \systeme[a\beta]{a+\beta+1=0,a+2=0} \]
Το σύστημα μας δίνει τη λύση $ (a,\beta)=(-2,1) $. Έτσι οι μετασχηματισμοί $ x=X-2 $ και $ y=Y+1 $ μας δίνουν την ομογενή εξίσωση :
\begin{equation}\label{a131}
\dfrac{dY}{dX}=\dfrac{X+Y}{X}-e^{\frac{X+Y}{X}}
\end{equation}
Στην \eqref{a131} θέτοντας $ Y=XZ\Rightarrow Y'=XZ'+Z $ οδηγούμεστε σε μια διαφορική εξίσωση \textbf{χωριζομένων μεταβλητών} :
\begin{align*}
\dfrac{dY}{dX}=\dfrac{X+Y}{X}-e^{\frac{X+Y}{X}}&\Rightarrow XZ'+Z=\dfrac{X+XZ}{X}-e^{\frac{X+XZ}{X}}\Rightarrow\\
&\Rightarrow XZ'+Z=1+Z-e^{1+Z}\Rightarrow XZ'=1-e^{1+Z}
\end{align*}
Η τελευταία έχει λύση την $ Z=1 $ η οποία μας δίνει τη λύση της αρχικής εξίσωσης : $ y=-x-1 $. Για τις υπόλοιπες λύσεις θα έχουμε :
\begin{gather*}
\dfrac{dZ}{1-e^{1+Z}}=\dfrac{dX}{X}\Rightarrow \int\dfrac{dZ}{1-e^{1+Z}}=\int\dfrac{dX}{X}+c'\Rightarrow\\ \int\dfrac{1+e^{1+Z}-e^{1+Z}}{1-e^{1+Z}}dZ=\int\dfrac{dX}{X}+c'\Rightarrow\\
Z-\log\left| 1-e^{1+Z}\right|=\log|X|+c'\Rightarrow\\ \log\left|\left( 1-e^{1+Z}\right)X\right|=Z+c'\Rightarrow\\
\left( 1-e^{1+Z}\right)X=\pm e^Z\cdot e^{c'}
\end{gather*}
όπου $ c' $ είναι μια αυθαίρετη σταθερά. Θέτοντας $ \pm e^{c'}=c'' $ και αντικαθιστώντας ξανά $ Y=XZ $ και $ X=x+2 $, $ Y=y-1 $ αποκτάμε τον τύπο από τον οποίο δίνεται η γενική λύση της αρχικής διαφορικής εξίσωσης :
\begin{gather*}
\left( 1-e^{1+\frac{y-1}{x+2}}\right)(x+2)=c'' e^{\frac{y-1}{x+2}}\Rightarrow
x+2-(x+2)e^{1+\frac{y-1}{x+2}}=c'' e^{\frac{y-1}{x+2}}\Rightarrow \\
x+2-(x+2)\cdot e\cdot e^{\frac{y-1}{x+2}}=c'' e^{\frac{y-1}{x+2}}\Rightarrow\\
x+2=\left[ (x+2)\cdot e +c''\right]  e^{\frac{y-1}{x+2}}\Rightarrow e^{\frac{y-1}{x+2}}=\dfrac{x+2}{x\cdot e+2e+c''}\Rightarrow
\frac{y-1}{x+2}=\log{\left(\dfrac{x+2}{x\cdot e+2e+c''} \right) }
\end{gather*}
όπου $ c'' $ είναι μια αυθαίρετη σταθερά. Θέτοντας $ c''+2e=c $ έχουμε τους τύπους οι οποίοι μας δίνουν όλες τις λύσεις της αρχικής διαφορικής εξίσωσης
\[ y=-x-1\ \textrm{ και }\ y=(x+2)\log{\left(\dfrac{x+2}{ex+c} \right)}+1 \]
\begin{Askhshs}[A]
\bmath{Με την αντικατάσταση $ x=e^t $ να επιλυθεί η διαφορική εξίσωση 
\[ xy\dfrac{d^2y}{dx^2}+x\left( \frac{dy}{dx}\right)^2-y\frac{dy}{dx}=0\ ,\ x>0 \]}
\end{Askhshs}\mbox{}\\
\lysh
Παρατηρούμε ότι η διαφορική εξίσωση αυτή ικανοποιείται από τη σχέση $ \frac{dy}{dx}=0 $ η οποία μας δίνει τις λύσεις της μορφής $ y=c' $ όπου $ c' $ είναι μια αυθαίρετη σταθερά. Επίσης έχει και τις λύσεις $ y=\pm x $. Για τις υπόλοιπες λύσεις θέτουμε $ x=e^t\Rightarrow t=\log(x) $ και προκύπτει 
\begin{align*}
\frac{dy}{dx}=\frac{dy}{dt}\cdot\frac{dt}{dx}&=\frac{1}{x}\frac{dy}{dt}\ \textrm{ και}\\
\frac{d^2y}{dx^2}=\frac{d}{dx}\left( \frac{dy}{dx}\right)&=\frac{d}{dx}\left( \frac{1}{x}\frac{dy}{dt}\right)=-\frac{1}{x^2}\frac{dy}{dt}+\frac{1}{x}\frac{d}{dx}\left(\frac{dy}{dt}\right) \\
&=-\frac{1}{x^2}\frac{dy}{dt}+\frac{1}{x}\frac{d}{dt}\left(\frac{dy}{dx}\right)=-\frac{1}{x^2}\frac{dy}{dt}+\frac{1}{x}\frac{d}{dt}\left(\frac{dy}{dx}\right)\\
&=\frac{1}{x^2}\frac{d^2y}{dt^2}-\frac{1}{x^2}\frac{dy}{dt}=\frac{1}{x^2}\left( \frac{d^2y}{dt^2}-\frac{dy}{dt}\right) 
\end{align*}
Οι μετασχηματισμοί αυτοί μετατρέπουν την αρχική εξίσωση στη μορφή :
\begin{gather*}
e^ty\frac{1}{e^{2t}}\left( \frac{d^2y}{dt^2}-\frac{dy}{dt}\right)+e^t\left( \frac{1}{e^t}\frac{dy}{dt}\right)^2-y\frac{1}{e^t}\frac{dy}{dt}=0\Rightarrow
y\frac{d^2y}{dt^2}-2y\frac{dy}{dt}+\left( \frac{dy}{dt}\right)^2=0
\end{gather*}
η οποία είναι μια διαφορική εξίσωση 2\tss{ης} τάξης μη περιέχουσα την ανεξάρτητη μεταβλητή $ t $. Έτσι θέτουμε $ \frac{dy}{dt}=z $ και $ \frac{d^2y}{dt^2}=z\frac{dy}{dz} $. Τότε η εξίσωση θα γίνει :
\begin{equation}\label{a14}
yz\frac{dy}{dz}-2yz+z^2=0
\end{equation}
Η \eqref{a14} έχει λύση την $ z=0 $ η οποία αντιστοιχεί στην $ y=c' $ που συναντήσαμε προηγουμένως. Για τις υπόλοιπες μη μηδενικές λύσεις θα έχουμε
\[ yz\frac{dy}{dz}-2yz+z^2=0\Rightarrow \frac{dy}{dz}=\frac{2yz-z^2}{yz} \]
Η τελευταία εξίσωση είναι μια ομογενής εξίσωση με βαθμό ομογένειας 2. Ο μετασχηματισμός $ z=uy\Rightarrow z'=u'y+u $ θα την ανάγει σε μια εξίσωση χωριζομένων μεταβλητών :
\begin{align*}
u'y+u=\frac{2yuy-(uy)^2}{yuy}&\Rightarrow u'y+u=\frac{2y^2u-u^2y^2}{y^2u}\Rightarrow\\
&\Rightarrow u'y=\frac{2u-u^2}{u}-u\Rightarrow u'y=\frac{2u-2u^2}{u}\Rightarrow\\
&\Rightarrow u'y=2-2u\Rightarrow \dfrac{u}{2-2u}=\frac{1}{y}
\end{align*}
Όλες οι λύσεις της θα δίνονται από τον τύπο
\begin{gather*}
\int\dfrac{du}{2-2u}=\int\frac{dy}{y}+a\Rightarrow -\dfrac{1}{2}\log|1-u|=\log|y|+a\Rightarrow\\
\log|y|+\dfrac{1}{2}\log|1-u|=-a\Rightarrow \log\left|y^2(1-u) \right|=-2a \Rightarrow\\
y^2(1-u)=\beta
\end{gather*}
έχοντας θέσει $ \beta=\pm e^{-2a} $ όπου $ a,\beta $ είναι αυθαίρετες σταθερές. Αντικαθιστώντας ξανά $ u=\frac{z}{y} $ και $ z=y' $ παίρνουμε την 
\begin{align*} y^2(1-u)=\beta&\Rightarrow y^2\left( 1-\frac{z}{y}\right) =\beta\Rightarrow\\ &\Rightarrow z=\frac{y^2-\beta}{y}\Rightarrow\\ &\Rightarrow y'=\frac{y^2-\beta}{y}\Rightarrow
\frac{y}{y^2-\beta}y'=1
\end{align*}
Η εξίσωση αυτή είναι χωριζομένων μεταβλητών οπότε οι λύσεις της θα δίνονται από τον τύπο
\begin{gather*}
\int\frac{y}{y^2-\beta}dy=\int{dt}+\gamma\Rightarrow \frac{1}{2}\log\left| y^2-\beta\right|=t+\gamma\Rightarrow\\
\log\left| y^2-\beta\right|=2t+2\gamma\Rightarrow y^2-\beta=\pm e^{2t+2\gamma}\xRightarrow{t=\log{(x)}}\\
y=\pm\sqrt{c_1x^2+c_2}\ ,\ x>0
\end{gather*}
όπου έχουμε θέσει $ c_1=\pm e^{2\gamma} $ και $ c_2=\beta $ με $ c_1\neq0,c_2\neq0,\gamma $ είναι αυθαίρετες σταθερές. Οι λύσεις της αρχικής διαφορικής εξίσωσης θα δίνονται από τους τύπους
\[ y=a\ ,\ y=\pm x\ \textrm{ και }\ y=\pm\sqrt{c_1x^2+c_2}\ ,\ x>0 \] όπου συμπτύσσονταν αυτές έχουμε \[ y=a\ \textrm{ και }\ y=\pm\sqrt{c_1x^2+c_2}\ ,\ x>0 \]
\begin{Askhshs}[A]
\textbf{Να επιλυθεί η διαφορική εξίσωση}
{\boldmath\[ \left(2y^3-3xy\right)dx+\left(x^2+xy^2 \right)dy=0 \]}
\textbf{με τη βοήθεια ολοκληρωτικού παράγοντα της μορφής} {\boldmath$ \rho(x,y)=\frac{1}{y}\varPhi\left( \frac{1}{y}\right) $}\textbf{, όπου {\boldmath$ \varPhi $} είναι κατάλληλη συνάρτηση (που θα πρέπει να βρεθεί).}
\end{Askhshs}\mbox{}\\
\lysh
Μια λύση της εξίσωσης είναι η $ y=0 $. Η συνάρτηση $ \rho(x,y)=\frac{1}{y}\varPhi\left( \frac{x}{y}\right) $ αποτελεί ολοκληρωτικό παράγοντα αν και μόνο αν πολλαπλασιάζοντας και τα δύο μέλη της εξίσωσης μ' αυτήν η εξίσωση που θα προκύψει είναι αμέσως ολοκληρώσιμη. Θα έχουμε λοιπόν :
\begin{gather}
\frac{1}{y}\varPhi\left( \frac{x}{y}\right)\left(2y^3-3xy\right)dx+\frac{1}{y}\varPhi\left( \frac{x}{y}\right)\left(x^2+xy^2 \right)dy=0\nonumber\\\label{a15}
\undercbrace{\varPhi\left( \frac{x}{y}\right)\left(2y^2-3x\right)}_{M}dx+\undercbrace{\varPhi\left( \frac{x}{y}\right)\left(\frac{x^2}{y}+xy \right)}_{N}dy=0
\end{gather}
Απαιτούμε λοιπόν να ισχύει η ισότητα $ \frac{\partial M}{\partial y}=\frac{\partial N}{\partial x} $. Έχουμε λοιπόν :
\begin{align*}
\frac{\partial M}{\partial y}&=\frac{\partial}{\partial y}\left[\varPhi\left( \frac{x}{y}\right)\left(2y^2-3x\right)\right]=\\
&=\varPhi'\left( \frac{x}{y}\right)\cdot\left( -\frac{x}{y^2}\right) \left(2y^2-3x\right)+\varPhi\left( \frac{x}{y}\right)\cdot 4y=\varPhi'\left( \frac{x}{y}\right)\left(\frac{3x^2}{y^2}-2x\right)+\varPhi\left( \frac{x}{y}\right)\cdot 4y\quad\textrm{και}\\
\frac{\partial N}{\partial x}&=\frac{\partial}{\partial x}\left[\varPhi\left( \frac{x}{y}\right)\left(\frac{x^2}{y}+xy \right)\right]=\\
&=\varPhi'\left( \frac{x}{y}\right)\cdot \frac{1}{y} \left(\frac{x^2}{y}+xy\right)+\varPhi\left( \frac{x}{y}\right)\cdot\left(\frac{2x}{y}+y \right) =\varPhi'\left( \frac{x}{y}\right) \left(\frac{x^2}{y^2}+x\right)+\varPhi\left( \frac{x}{y}\right)\cdot\left(\frac{2x}{y}+y \right)
\end{align*}
Σύμφωνα με τα παραπάνω λοιπόν θα ισχύει : 
\begin{gather*}
\varPhi'\left( \frac{x}{y}\right)\left(\frac{3x^2}{y^2}-2x\right)+\varPhi\left( \frac{x}{y}\right)\cdot 4y=\varPhi'\left( \frac{x}{y}\right) \left(\frac{x^2}{y^2}+x\right)+\varPhi\left( \frac{x}{y}\right)\cdot\left(\frac{2x}{y}+y \right)\Rightarrow\\
\varPhi'\left( \frac{x}{y}\right)\left(\frac{2x^2}{y^2}-3x\right)+\varPhi\left( \frac{x}{y}\right)\cdot \left(3y-\frac{2x}{y} \right)=0\Rightarrow\\
\varPhi'\left( \frac{x}{y}\right)\left[\frac{x}{y} \left(\frac{2x}{y}-3x\right)\right]-\varPhi\left( \frac{x}{y}\right)\cdot\left(\frac{2x}{y}-3x\right)=0\Rightarrow
\varPhi'\left( \frac{x}{y}\right)\cdot\frac{x}{y} -\varPhi\left( \frac{x}{y}\right)=0
\end{gather*}
Θέτοντας στην τελευταία εξίσωση $ z=\frac{x}{x} $ αποκτάμε τη γραμμική διαφορική εξίσωση 1\tss{ου} βαθμού :
\[ \varPhi'\left(z\right)\cdot z -\varPhi\left(z\right)=0\Rightarrow \varPhi'\left(z\right) -\frac{1}{z}\varPhi\left(z\right)=0\]
Η γενική λύση αυτής θα δίνεται από τον παρακάτω τύπο :
\[ \varPhi(z)=ce^{-\dintt{-\frac{1}{z}}dz}=ce^{\log{z}}=cz \]
όπου $ c $ είναι μια αυθαίρετη σταθερά. Χωρίς βλάβη της γενικοτητας επιλέγουμε $ c=1 $ και παίρνουμε τον ολοκληρωτικό παράγοντα $ \rho(x,y)=\frac{x}{y^2} $. Ο παράγοντας αυτός φέρνει την εξίσωση \eqref{a15} στη μορφή
\begin{gather*}
\frac{x}{y^2}\left(2y^3-3xy\right)dx+\frac{x}{y^2}\left(x^2+xy^2 \right)dy=0\Rightarrow
\undercbrace{\left(2xy-\frac{3x^2}{y}\right)}_{M}dx+\undercbrace{\left(\frac{x^3}{y^2}+x^2 \right)}_{N}dy=0
\end{gather*}
Εύκολα διαπιστώνουμε ότι η παραπάνω εξίσωση είναι αμέσως ολοκληρώσιμη αφού $ \frac{\partial M}{\partial y}=\frac{\partial N}{\partial x}=\frac{2xy^2+3x^2}{y^2} $. Θα υπάρχει λοιπόν συνάρτηση $ f(x,y) $ ώστε η τελευταία εξίσωση να γραφτεί στη μορφή $ df(x,y)=Mdx+Ndy=0 $. Όλες οι μη μηδενικές λύσεις της θα δίνονται από τον τύπο $ f(x,y)=c $. Θα ισχύει λοιπόν 
\[ \frac{\partial f}{\partial x}=M=2xy-\frac{3x^2}{y}\ \textrm{ και }\ \frac{\partial f}{\partial y}=N=\frac{x^3}{y^2}+x^2 \]
Ολοκληρώνοντας μια από τις δύο σχέσεις παίρνουμε τη συνάρτηση $ f(x,y) $. Από τη δεύτερη σχέση με ολοκλήρωση ως προς $ y $ παίρνουμε :
\[ f(x,y)=\int{\left( \frac{x^3}{y^2}+x^2\right) }dy+g(x)=-\frac{x^3}{y}+x^2y+g(x) \]
για μια συνάρτηση $ g(x) $. Παραγωγίζουμε την τελευταία παράσταση ως προς $ x $ και εξισώνουμε τη συνάρτηση που θα προκύψει με τη συνάρτηση $ M $.
\begin{gather*}
\frac{\partial f}{\partial x}=-\frac{3x^2}{y}+2xy+g'(x)\ \textrm{ άρα}\\
-\frac{3x^2}{y}+2xy+g'(x)=2xy-\frac{3x^2}{y}\Rightarrow g'(x)=0
\end{gather*}
Προκύπτει λοιπόν $ g(x)=c $. Επιλέγοντας δίχως βλάβη της γενικότητας $ g(x)=0 $ παίρνουμε τον τύπο που μας δίνει τις μη μηδενικές λύσεις της εξίσωσης :
\[ f(x,y)=x^2y-\frac{x^3}{y}=c\Rightarrow y=\dfrac{c\pm\sqrt{4x^5+c^2}}{2x^2} \] όπου $ c $ είναι μια αυθαίρετη σταθερά. Όλες οι λύσεις της αρχικής διαφορικής εξίσωσης θα είναι :
\[ y=0\ \textrm{ και }\  y=\dfrac{c\pm\sqrt{4x^5+c^2}}{2x^2} \]
\begin{Askhshs}[A]
\textbf{\boldmath{Ας είναι $ p $ και $ q $ συνεχείς πραγματικές συναρτήσεις στο διάστημα $ [0,+\infty) $ τέτοιες ώστε 
\[ |p(x)|\geq|q(x)|\ \ ,\ \ \forall x\geq 0 \]
και ας θεωρήσουμε τις πρώτης τάξης ομογενείς γραμμικές διαφορικές εξισώσεις 
\begin{equation}
y'+py=0\tag{P}
\end{equation}
\begin{equation}
z'+qz=0\tag{Q}
\end{equation}
Να εξεταστεί αν είναι αληθής ή ψευδής η πρόταση :\\
Αν όλες οι λύσεις της $ (Q) $ τείνουν προς το $ 0 $ για $ x\rightarrow\infty $, τότε όλες της $ (P) $ τείνουν προς το $ 0 $ για $ x\rightarrow\infty $.}}
\end{Askhshs}\mbox{}\\
\lysh
Όλες οι λύσεις της εξίσωσης $ (Q) : z'+qz=0 $ δίνονται από τον τύπο
\[ z(x)=z(0)e^{-\int_{0}^{x}{q(t)dt}}\ \ ,\ \ x\geq0 \] Από την υπόθεση της άσκησης θα έχουμε το όριο της συνάρτησης $ z(x) $ για $ x\rightarrow\infty $ να είναι :
\[ \lim_{ x\rightarrow\infty }{z(x)}=0\Leftrightarrow \lim_{ x\rightarrow\infty }{z(0)e^{-\int_{0}^{x}{q(t)dt}}}=0\Leftrightarrow\lim_{ x\rightarrow\infty } \int_{0}^{x}{q(t)dt}=+\infty\Leftrightarrow\int_{0}^{+\infty}{\!\!\!q(x)dx}=+\infty \]
και αυτό γιατί η συνάρτηση $ q(x) $ είναι πραγματική συνάρτηση. Ομοίως οι λύσεις της εξίσωσης $ (P) : y'+py=0 $ θα δίνονται από τον τύπο
\[ y(x)=y(0)e^{-\int_{0}^{x}{p(t)dt}}\ \ ,\ \ x\geq0 \]
Για το όριο αυτής όταν $ x\rightarrow\infty $ θα έχουμε :
\[ \lim_{ x\rightarrow\infty }{y(x)}=0\Leftrightarrow \lim_{ x\rightarrow\infty }{y(0)e^{-\int_{0}^{x}{p(t)dt}}}=0\Leftrightarrow\lim_{ x\rightarrow\infty } \int_{0}^{x}{p(t)dt}=+\infty\Leftrightarrow\int_{0}^{+\infty}{\!\!\!p(x)dx}=+\infty \]
Οπότε αρκεί να αποδείξουμε αν είναι αληθής ή ψευδής η ακόλουθη πρόταση :
\[ \int_{0}^{+\infty}{\!\!\!q(x)dx}=+\infty\Leftrightarrow\int_{0}^{+\infty}{\!\!\!p(x)dx}=+\infty \]
Αν γνωρίζαμε από την υπόθεση της άσκησης ότι ισχύει $ p(x)\geq q(x)\ \ ,\ \ \forall x\geq0 $ τότε η πρόταση θα ήταν αληθής. Αν επίσης η υπόθεση μας έδινε $ p(x)\geq q(x)\ ,\ \forall x\geq x_0 $ για κάποιο $ x_0\geq0 $ τότε πάλι η πρόταση θα ήταν αληθής. Όμως από την υπόθεση έχουμε γνωστό ότι $ |p(x)|\geq|q(x)|\ \ ,\ \ \forall x\geq0 $ και άρα υποπτευόμαστε ότι η πρόταση είναι ψευδής. Αρκεί λοιπόν να βρούμε ένα αντιπαράδειγμα. Οι συναρτήσεις $ p(x)=-1 $ και $ q(x)=1 $ ικανοποιούν τη σχέση $ |p(x)|\geq|q(x)| $. Έτσι θα έχουμε 
\[ \int_{0}^{+\infty}{\!\!\!1dx}=+\infty\ \textrm{ και }\int_{0}^{+\infty}{\!\!\!(-1)dx}=-\infty \]
Συνεπώς θα ισχύει 
\[ \int_{0}^{+\infty}{\!\!\!q(x)dx}=+\infty\ \cancel{\Rightarrow}\int_{0}^{+\infty}{\!\!\!p(x)dx}=+\infty \]
πράγμα που αποδεικνύει ότι η πρόταση είναι ψευδής.\\\\
\begin{Askhshs}[A]
\bmath{Να επιλυθεί η διαφορική εξίσωση
\[ \left(\frac{1}{x}+\frac{1}{y} \right)dx+\left(\frac{x}{y^2}+\frac{3}{y} \right)dy=0 \] αφού βρεθεί ένας ολοκληρωτικός παράγοντας της μορφής $ \rho(x,y)=x\varPhi(y) $ όπου $ \varPhi $ είναι μια κατάλληλη συνάρτηση (που θα πρέπει να προσδιοριστεί). Στη συνέχεια να επιλυθεί η παραπάνω διαφορική εξίσωση και με έναν άλλο τρόπο.}
\end{Askhshs}\mbox{}\\
\lysh
\textbf{1\tss{ος} Τρόπος}\\
Η συνάρτηση $ \rho(x,y)=x\varPhi(y) $ θα είναι ολοκληρωτικός παράγοντας της εξίσωσης αν και μόνο αν η εξίσωση που θα προκύψει, πολλαπλασιάζοντας παντού μ' αυτόν, είναι αμέσως ολοκληρώσιμη. Θα παρκύψει
\begin{gather}
x\varPhi(y)\left(\frac{1}{x}+\frac{1}{y} \right)dx+x\varPhi(y)\left(\frac{x}{y^2}+\frac{3}{y} \right)dy=0\Rightarrow\nonumber\\\label{a17}
\undercbrace{\varPhi(y)\left(1+\frac{x}{y} \right)}_{M}dx+\undercbrace{\varPhi(y)\left(\frac{x^2}{y^2}+\frac{3x}{y} \right)}_{N}dy=0
\end{gather}
Απαιτούμε λοιπόν για την εξίσωση \eqref{a17} προκειμένου να είναι αμέσως ολοκληρώσιμη, να ισχύει η σχέση $ \frac{\partial M}{\partial y}=\frac{\partial M}{\partial y} $ και θα έχουμε αναλυτικά :
\begin{align}
&\frac{\partial M}{\partial y}=\frac{\partial}{\partial y}\left[ \varPhi(y)\left(1+\frac{x}{y} \right)\right]=\varPhi'(y)\left(1+\frac{x}{y} \right)-\varPhi(y)\cdot\frac{x}{y^2}\ \ \textrm{και}\label{a171}\\
&\frac{\partial N}{\partial x}=\frac{\partial}{\partial x}\left[ \varPhi(y)\left(\frac{x^2}{y^2}+\frac{3x}{y} \right)\right]=\varPhi(y)\left(\frac{2x}{y^2}+\frac{3}{y} \right)\label{a172}
\end{align}
Εξισώνοντας τις συναρτήσεις \eqref{a171} και \eqref{a172} καταλήγουμε σε μια γραμμική διαφορική εξίσωση πρώτης τάξης από την οποία θα προκύψει η συνάρτηση $ \varPhi $. Έχουμε λοιπόν
\begin{gather}
\varPhi'(y)\left(1+\frac{x}{y} \right)-\varPhi(y)\cdot\frac{x}{y^2}=\varPhi(y)\left(\frac{2x}{y^2}+\frac{3}{y} \right)\Rightarrow\nonumber\\
\varPhi'(y)\left(1+\frac{x}{y} \right)-\varPhi(y)\left( \frac{3x}{y^2}+\frac{3}{y}\right)=0\Rightarrow\varPhi'(y)\left(1+\frac{x}{y} \right)-\frac{3}{y}\varPhi(y)\left( \frac{x}{y}+1\right)=0\Rightarrow\nonumber\\\label{a173}
\varPhi'(y)-\frac{3}{y}\varPhi(y)=0
\end{gather}
Η γενική λύση της εξίσωσης \eqref{a173} θα δίνεται από τον τύπο
\[ \varPhi(y)=ce^{-\dintt\left(-\frac{3}{y} \right)dy }=ce^{3\log{|y|}}=cy^3 \] όπου $ c\neq0 $ είναι μια αυθαίρετη μη μηδενική σταθερά. Επιλέγοντας χωρίς βλάβη της γενικότητας $ c=1 $ προκύπτει ο ζητούμενος ολοκληρωτικός παράγοντας $ \rho(x,y)=xy^3 $. Θα προκύψει έτσι η εξίσωση
\begin{gather}
xy^3\left(\frac{1}{x}+\frac{1}{y} \right)dx+xy^3\left(\frac{x}{y^2}+\frac{3}{y} \right)dy=0\Rightarrow\nonumber\\\label{a174}
\undercbrace{\left(y^3+xy^2 \right)}_{M}dx+\undercbrace{\left(x^2y+3xy^2\right)}_{N}dy=0
\end{gather}
Διαπιστώνουμε ότι η διαφορική εξίσωση \eqref{a174} είναι αμέσως ολοκληρώσιμη αφού ισχύει $ \frac{\partial M}{\partial y}=\frac{\partial M}{\partial y}=3y^2+2xy $. Θα υπάρχει λοιπόν μια συνάρτηση $ f(x,y) $ τέτοια ώστε η εξίσωση να πάρει τη μορφή $ df(x,y)=Mdx+Ndy=0 $. Όλες οι λύσεις της θα δίνονται από τον τύπο $ f(x,y)=c $. Αναλυτικά για τη συνάρτηση $ f $ θα ισχύει
\[ \frac{\partial f}{\partial x}=M=y^3+xy^2\ \textrm{ και }\ \frac{\partial f}{\partial y}=N=x^2y+3xy^2 \]
Ολοκληρώνοντας μια από τις δύο σχέσεις, ως προς την αντίστοιχη μεταβλητή, παιρνουμε τη συνάρτηση $ f $. Έτσι από τη δεύτερη σχέση  με ολοκλήρωση ως προς $ y $ προκύπτει :
\[ f(x,y)=\int{\left(x^2y+3xy^2 \right)dy}+g(x)=\frac{x^2y^2}{2}+xy^3+g(x) \] για κάποια συνάρτηση $ g(x) $. Παραγωγίζουμε την τελευταία συνάρτηση ως προς $ x $ και εξισώνουμε το αποτέλεσμα με τη συνάρτηση $ M $. Θα έχουμε λοιπόν
\begin{gather*}
\frac{\partial f}{\partial x}=xy^2+y^3+g'(x)\\
\textrm{άρα }\ xy^2+y^3+g'(x)=xy^2+y^3\Rightarrow g'(x)=0
\end{gather*}
Από την τελευταία σχέση προκύπτει $ g(x)=c $ όπου $ c $ είναι μια αυθαίρετη σταθερά. Επιλέγοντας $ c=0\Rightarrow g(x)=0 $ παίρνουμε τη συνάρτηση $ f(x,y)=\frac{x^2y^2}{2}+xy^3 $. Όλες οι λύσεις της αρχικής διαφορικής εξίσωσης θα δίνονται από τον τύπο $ f(x,y)=c $ δηλαδή
\[ \frac{x^2y^2}{2}+xy^3=c \] όπου $ c $ είναι μαι αυθαίρετη σταθερά.\\\\
\textbf{2\tss{ος} Τρόπος}\\
Η αρχική διαφορική εξίσωση μετασχηματίζεται μετά από πράξεις σε μια ομογενή διαφορική εξίσωση με βαθμό ομογένειας $ 2 $ ως εξής :
\begin{align}
\left(\frac{1}{x}+\frac{1}{y} \right)dx+\left(\frac{x}{y^2}+\frac{3}{y} \right)&dy=0\Rightarrow\frac{x+y}{xy}dx+\frac{x+3y}{y^2}dy=0\Rightarrow\nonumber\\\label{a175}
&\frac{dy}{dx}=-\dfrac{\frac{x+y}{xy}}{\frac{x+3y}{y^2}}\Rightarrow\frac{dy}{dx}=-\dfrac{y^2\left(x+y\right)}{xy\left( x+3y\right)}\Rightarrow y'=-\dfrac{xy+y^2}{x^2+3xy}
\end{align}
Η συνάρτηση $ y=\frac{x}{3} $ η οποία μηδενίζει τον παρονομαστή της τελευταίας διαφορικής εξίσωσης \textbf{δεν} είναι λύση της αρχικής διαφορικής εξίσωσης. Αυτό σημαίνει ότι είναι ισοδύναμες. Με το μετασχηματισμό $ y=zx\Rightarrow y'=z'x+z $ η \eqref{a175} γίνεται
\begin{align*}
z'x+z=-\dfrac{x^2z+x^2z^2}{x^2+3x^2z}&\Rightarrow z'x+z=-\dfrac{z+z^2}{1+3z}\Rightarrow\\ &\Rightarrow z'x=-\dfrac{2z+4z^2}{1+3z}\Rightarrow \dfrac{1+3z}{2z+4z^2}dz=-\frac{dx}{x}
\end{align*}
Οι λύσεις της θα δίνονται από τον τύπο
\begin{gather*}
\int\dfrac{1+3z}{2z+4z^2}dz=-\int\frac{dx}{x}+a
\end{gather*}
όπου $ a $ είναι μια αυθαίρετη σταθερά. Έτσι θα έχουμε
\begin{gather*}
\frac{1}{2}\int\dfrac{1}{z}dz+\frac{1}{4}\int\frac{2}{2z+1}dz=-\int\frac{dx}{x}+a\Rightarrow \frac{1}{2}\log{|z|}+\frac{1}{4}\log{|2z+1|}=-\log{|x|}+a\Rightarrow\\
2\log{|z|}+\log{|2z+1|}=-4\log{|x|}+4a\Rightarrow \log{\left|z^2(2z+1)x^4 \right| }=4a\Rightarrow\\
z^2(2z+1)x^4 =c
\end{gather*}
έχοντας θέσει όπου $ \pm e^{4a}=c\neq0 $ με $ a,c $ αυθαίρετες σταθερές. Καταλήγουμε, κάνοντας αναδρομική αντικατάσταση $ z=y/x $, στον τύπο από τον οποίο δίνονται όλες οι λύσεις της αρχικής διαφορικής εξίσωσης
\[ \frac{y^2}{x^2}\left( 2\frac{y}{x}+1\right) x^4 =c\Rightarrow \frac{x^2y^2}{2}+xy^3=c \]
\begin{Askhshs}[A]
\bmath{Να αποδειχθεί ότι η διαφορική εξίσωση
\[ \frac{dy}{dx}=\left(x^2+y+1 \right)\left(x^2+y-\frac{3}{2} \right)+1-2x \]
δέχεται λύσεις της μορφής $ y=\lambda-x^2 $ (όπου $ \lambda $ είναι σταθερά). Στη συνέχεια να επιλυθεί η διαφορική εξίσωση αυτή.}
\end{Askhshs}\mbox{}\\
\lysh
Αρκεί να δείξουμε ότι υπάρχουν τιμές της παραμέτρου $ \lambda $ ώστε η συνάρτηση $ y=\lambda-x^2 $ να επαληθεύει την αρχική διαφορική εξίσωση. Αντικαθιστώντας $ y=\lambda-x^2 $ και $ y'=-2x $ παίρνουμε
\begin{gather*}
-2x=\left(x^2+\lambda-x^2+1 \right)\left(x^2+\lambda-x^2-\frac{3}{2} \right)+1-2x\Rightarrow\\
\left(\lambda+1 \right)\left(\lambda-\frac{3}{2} \right)+1=0\Rightarrow 2\lambda^2-\lambda-1=0
\end{gather*}
Η πολυωνυμική εξίσωση 2\tss{ου} βαθμού που προέκυψε έχει λύσεις τις $ \lambda_1=1 $ και $ \lambda_2=-\frac{1}{2} $. Για κάθεμιά από τις τιμές αυτές της παραμέτρου παίρνουμε αντίστοιχα τις μερικές λύσεις
\[ y=1-x^2\ \textrm{ και }\ y=-\frac{1}{2}-x^2 \]
Εκτελώντας πράξεις στην αρχική διαφορική εξίσωση τη φέρνουμε στη μορφή :
\[ y'+\left(\frac{1}{2}-x^2 \right)y-y^2+\left( \frac{x^2}{2}+2x+\frac{1}{2}-x^4\right)=0 \]
η οποία είναι μια διαφορική εξίσωση Ricatti. Γι αυτήν γνωρίζουμε ήδη μια μερική λύση την $ y_1=1-x^2 $. Θέτοντας λοιπόν $ y=y_1+\frac{1}{z}=1-x^1+\frac{1}{z} $ και $ y'=-2x-\frac{z'}{z^2} $ στην αρχική εξλισωση παίρνουμε 
\begin{gather*}
-2x-\frac{z'}{z^2}=\left(2+ \frac{1}{z}\right)\left(\frac{1}{z}-\frac{1}{2} \right)+1-2x\Rightarrow\\
-\frac{z'}{z^2}=-1+\frac{2}{z}-\frac{1}{2z}+\frac{1}{z^2}+1\Rightarrow
-\frac{z'}{z^2}=\frac{3}{2z}+\frac{1}{z^2}\Rightarrow\\ z'+\frac{3}{2}z=-1
\end{gather*}
η οποία είναι μια γραμμική διαφορική εξίσωση 1\tss{ης} τάξης. Η γενική λύση της θα δίνεται από τον τύπο
\begin{align}
z(x)&=e^{-\dintt{\frac{3}{2}dx}}\left[c+\int{(-1)\cdot e^{\dintt{\frac{3}{2}dx}}dx} \right]=\\&=e^{-\frac{3x}{2}}\left(c-\int{e^{-\frac{3x}{2}}dx} \right)=e^{-\frac{3x}{2}}\left(c-\frac{2}{3}e^{-\frac{3x}{2}}\right)=ce^{-\frac{3x}{2}}-\frac{2}{3}
\end{align}
όπου $ c $ είναι μια αυθαίρετη σταθερά. Συνεπώς οι λύσεις της αρχικής διαφορικής εξίσωσης θα δίνονται από τους τύπους
\[ y=1-x^2\ \textrm{ και }\ y=1-x^2+\frac{3}{3ce^{-\frac{3x}{2}}-2} \]
Ομοίως θα μπορούσαμε να είχαμε επιλύσει την αρχική διαφορική εξίσωση χρησιμοποιώντας τη μερική λύση $ y=-\frac{1}{2}+x^2 $ ακολουθώντας την ίδια διαδικασία.\epask
\begin{Askhshs}[A]
\bmath{Να επιλυθεί η διαφορική εξίσωση
\[ \left[y+x\left(x^2+y^2 \right)^2 \right]dx+\left[y\left(x^2+y^2 \right)^2-x \right]dy=0 \]
αφού πρώτα βρεθεί ένας ολοκληρωτικός παράγοντας της μορφής $ \rho(x,y)=\varPhi\left(x^2+y^2 \right) $(όπου $ \varPhi $ είναι μια συνάρτηση που θα πρέπει να προσδιοριστεί).}
\end{Askhshs}\mbox{}\\
\lysh
Η συνάρτηση $ \rho(x,y)=\varPhi\left(x^2+y^2 \right) $ θα είναι ένας ολοκληρωτικός παράγοντας της αρχικής εξίσωσης αν και μόνο αν πολλαπλασιάζοντας και τα δύο μέλη της μ' αυτήν προκύψει αμέσως ολοκληρώσιμη εξίσωση. Θα έχουμε αναλυτικά
\begin{equation}\label{a19}
\undercbrace{\varPhi\left(x^2+y^2 \right)\left[y+x\left(x^2+y^2 \right)^2 \right]}_{M}dx+\undercbrace{\varPhi\left(x^2+y^2 \right)\left[y\left(x^2+y^2 \right)^2-x \right]}_{N}dy=0
\end{equation}
Για να είναι η \eqref{a19} αμέσως ολοκληρώσιμη απαιτούμε να ισχύει $ \frac{\partial M}{\partial y}=\frac{\partial N}{\partial x} $. Οι μερικές παράγωγοι έχουν ως εξής
\begin{alignat}{2}
\frac{\partial M}{\partial y}&=\frac{\partial}{\partial y}\left[\varPhi\left(x^2+y^2 \right)\left[y+x\left(x^2+y^2 \right)^2 \right] \right]=\notag\\\label{a191}&=2y\varPhi'\left(x^2+y^2 \right)\left[y+x\left(x^2+y^2 \right)^2 \right]+\varPhi\left(x^2+y^2 \right)\left[1+4xy\left(x^2+y^2 \right) \right]\ \textrm{ και}\\
\frac{\partial N}{\partial x}&=\frac{\partial}{\partial x}\left[\varPhi\left(x^2+y^2 \right)\left[y\left(x^2+y^2 \right)^2-x \right] \right]=\notag\\\label{a192}
&=2x\varPhi'\left(x^2+y^2 \right)\left[y\left(x^2+y^2 \right)^2-x \right]+\varPhi\left(x^2+y^2 \right)\left[4xy\left(x^2+y^2 \right)-1 \right]
\end{alignat}
Εξισώνοντας τις σχέσεις \eqref{a191} και \eqref{a192} θα καταλλήξουμε σε μια διαφορική εξίσωση από την οποία θα προσδιορίσουμε τη συνάρτηση $ \varPhi $.
\begin{alignat*}
2y\varPhi'\left(x^2+y^2 \right)&\left[y+x\left(x^2+y^2 \right)^2 \right]+\varPhi\left(x^2+y^2 \right)\left[1+4xy\left(x^2+y^2 \right) \right]=\\&2x\varPhi'\left(x^2+y^2 \right)\left[y\left(x^2+y^2 \right)^2-x \right]+\varPhi\left(x^2+y^2 \right)\left[4xy\left(x^2+y^2 \right)-1\right]\Rightarrow\\
\varPhi'\left(x^2+y^2 \right)&\left[2y^2+2xy\left(x^2+y^2 \right)^2 \right]+\varPhi\left(x^2+y^2 \right)\left[1+4xy\left(x^2+y^2 \right) \right]=\\&\varPhi'\left(x^2+y^2 \right)\left[2xy\left(x^2+y^2 \right)^2-2x^2 \right]+\varPhi\left(x^2+y^2 \right)\left[4xy\left(x^2+y^2 \right)-1\right]\Rightarrow\\
&\varPhi'\left(x^2+y^2 \right)\left(2x^2+2y^2 \right)+2\varPhi\left(x^2+y^2 \right)=0\Rightarrow\\
&\varPhi'\left(x^2+y^2 \right)+\frac{1}{x^2+y^2}\varPhi\left(x^2+y^2 \right)=0
\end{alignat*}
Θέτοντας στην τελευταά εξίσωση όπου $ z=x^2+y^2 $ παίρνουμε τη ζητούμενη εξίσωση
\[ \varPhi'\left(z \right)+\frac{1}{z}\varPhi\left(z\right)=0 \]
Για τη γενική λύση αυτής έχουμε τον τύπο
\[ \varPhi(z)=c'e^{-\dintt{\frac{1}{z}}dz}=c'e^{-\log|z|}=\frac{c'}{z} \]
όπου $ c'\neq0 $ είναι μια αυθαίρετη σταθερά. Χωρίς βλάβη της γενικότητας επιλέγουμε $ c'=1 $ και αποκτάμε τη συνάρτηση $ \varPhi(z)=\frac{1}{z} $ η οποία μας δίνει τον ολοκληρωτικό παράγοντα $ \rho(x,y)=\frac{1}{x^2+y^2} $. Συνεπώς η εξίσωση
\begin{gather}
\frac{1}{x^2+y^2}\left[y+x\left(x^2+y^2 \right)^2 \right]dx+\frac{1}{x^2+y^2}\left[y\left(x^2+y^2 \right)^2-x \right]dy=0\Rightarrow\nonumber\\\label{a193}
\undercbrace{\left[\frac{y}{x^2+y^2}+x\left(x^2+y^2 \right) \right]}_{M}dx+\undercbrace{\left[y\left(x^2+y^2 \right)-\frac{x}{x^2+y^2} \right]}_{N}dy=0
\end{gather}
είναι μια διαφορική εξίσωση αμέσως ολοκληρώσιμη αφού ισχύει $ \frac{\partial M}{\partial y}=\frac{\partial N}{\partial x}=2xy+\frac{x^2-y^2}{\left( x^2+y^2\right)^2} $. Άρα θα υπάρχει μια συνάρτηση $ f(x,y) $ τέτοια ώστε η εξίσωση \eqref{a193} να γραφτεί στη μορφή $ df(x,y)=0 $. Γι αυτήν θα ισχύουν οι σχέσεις
\[ \frac{\partial f}{\partial x}=M=\frac{y}{x^2+y^2}+x\left(x^2+y^2 \right)\ \textrm{ και }\ \frac{\partial f}{\partial y}=N=y\left(x^2+y^2 \right)-\frac{x}{x^2+y^2} \]
Ολοκληρώνοντας την πρώτη σχέση ως προς $ x $ παίρνουμε τη συνάρτηση $ f(x,y) $ η οποία θα είναι
\[ f(x,y)=\int{\left[\frac{y}{x^2+y^2}+x\left(x^2+y^2 \right) \right] dx}+g(y)=\arctan{\left(\frac{x}{y}\right)}+\frac{x^4}{4}+\frac{x^2y^2}{2}+g(y) \]
για κάποια συνάρτηση $ g(y) $. Παραγωγίζουμε τη συνάρτηση που προέκυψε και εξισώνουμε το αποτέλεσμα με τη συνάρτηση $ N $ οπότε
\begin{gather*}
\frac{\partial f}{\partial y}=-\frac{x}{x^2+y^2}+x^2y+g'(y)\ \ \textrm{άρα }\\
-\frac{x}{x^2+y^2}+x^2y+g'(y)=y\left(x^2+y^2 \right)-\frac{x}{x^2+y^2}\Rightarrow\\ -\frac{x}{x^2+y^2}+x^2y+g'(y)=x^2y+y^3-\frac{x}{x^2+y^2}\Rightarrow g'(y)=y^3
\end{gather*}
για κάποια συνάρτηση $ g(y) $. Η τελευταία σχέση μας δίνει $ g(y)=\frac{y^4}{4}+c $ και μπορούμε να επιλέξουμε τη συνάρτηση $ g(y)=\frac{y^4}{4} $. Αποκτάμε έτσι τη συνάρτηση $ f $ η οποία θα είναι
\[ f(x,y)=\arctan{\left(\frac{x}{y}\right)}+\frac{x^4}{4}+\frac{x^2y^2}{2}+\frac{y^4}{4} \]
Ο τύπος από τον οποίο θα δίνονται όλες οι λύσεις της αρχικής διαφορικής εξίσωσης θα είναι $ f(x,y)=c $ δηλαδή
\[ \arctan{\left(\frac{x}{y}\right)}+\frac{x^4}{4}+\frac{x^2y^2}{2}+\frac{y^4}{4}=c \]
όπου $ c $ είναι μια αυθαίρετη σταθερά.\epask
\begin{Askhshs}[A]
\bmath{Να επιλυθεί η διαφορική εξίσωση
\[ \left(2y^3-3xy\right)dx+\left(x^2+xy^2\right)dy=0 \]
με τη βοήθεια ενός ολοκληρωτικού παράγοντα της μορφής $ \rho(x,y)=\varPhi\left(\frac{x}{y^2}\right) $ (όπου $ \varPhi $ είναι μια συνάρτηση που θα πρέπει να προσδιοριστεί).}
\end{Askhshs}\mbox{}\\
\lysh
Μια προφανής λύση της εξίσωσης είναι η $ y=0 $. Η συνάρτηση $ \rho(x,y)=\varPhi\left(\frac{x}{y^2}\right) $ είναι ένας ολοκληρωτικός παράγοντας της αρχικής εξίσωσης αν και μόνο αν πολλαπλασιάζοντας και τα δύο μέλη της μ' αυτήν προκύψει αμέσως ολοκληρώσιμη εξίσωση. Θα έχουμε αναλυτικά
\begin{equation}\label{a202}
\undercbrace{\varPhi\left(\frac{x}{y^2}\right)\left(2y^3-3xy\right)}_{M}dx+\undercbrace{\varPhi\left(\frac{x}{y^2}\right)\left(x^2+xy^2\right)}_{N}dy=0
\end{equation}
Θα πρέπει λοιπόν να ισχύει $ \frac{\partial M}{\partial y}=\frac{\partial N}{\partial x} $ οπότε θα έχουμε αναλυτικά
\begin{align}
\frac{\partial M}{\partial y}&=\frac{\partial }{\partial y}\left[\varPhi\left(\frac{x}{y^2}\right)\left(2y^3-3xy\right)\right]=-\frac{2x}{y^3}\cdot\varPhi'\left(\frac{x}{y^2}\right)\left(2y^3-3xy\right)+\varPhi\left(\frac{x}{y^2}\right)\left(6y^2-3x\right)=\nonumber\\\label{a20}
&=\varPhi'\left(\frac{x}{y^2}\right)\left(\frac{6x^2}{y^2}-4x\right)+\varPhi\left(\frac{x}{y^2}\right)\left(6y^2-3x\right)\ \textrm{ και}\\
\frac{\partial N}{\partial x}&=\frac{\partial }{\partial x}\left[\varPhi\left(\frac{x}{y^2}\right)\left(x^2+xy^2\right)\right]=\frac{1}{y^2}\cdot\varPhi'\left(\frac{x}{y^2}\right)\left(x^2+xy^2\right)+\varPhi\left(\frac{x}{y^2}\right)\left(2x+y^2\right)=\nonumber\\\label{a201}
&=\varPhi'\left(\frac{x}{y^2}\right)\left(\frac{x^2}{y^2}+x\right)+\varPhi\left(\frac{x}{y^2}\right)\left(2x+y^2\right)
\end{align}
Απαιτώντας οι συναρτήσεις \eqref{a20} και \eqref{a201} να είναι ίσες θα καταλήξουμε σε μια διαφορική εξίσωση από την οποία θα προσδιορίσουμε τη συνάρτηση $ \varPhi $. Έχουμε λοιπόν
\begin{gather*}
\varPhi'\left(\frac{x}{y^2}\right)\left(\frac{6x^2}{y^2}-4x\right)+\varPhi\left(\frac{x}{y^2}\right)\left(6y^2-3x\right)=\varPhi'\left(\frac{x}{y^2}\right)\left(\frac{x^2}{y^2}+x\right)+\varPhi\left(\frac{x}{y^2}\right)\left(2x+y^2\right)\Rightarrow\\
\varPhi'\left(\frac{x}{y^2}\right)\left(\frac{5x^2}{y^2}-5x\right)+\varPhi\left(\frac{x}{y^2}\right)\left(5y^2-5x\right)=0\Rightarrow\\
\frac{x}{y^2}\varPhi'\left(\frac{x}{y^2}\right)\left(x-y^2\right)-\varPhi\left(\frac{x}{y^2}\right)\left(x-y^2\right)=0\Rightarrow \varPhi'\left(\frac{x}{y^2}\right)-\frac{y^2}{x}\varPhi\left(\frac{x}{y^2}\right)=0
\end{gather*}
Θέτοντας όπου $ \frac{x}{y^2}=z $ η τελευταία εξίσωση μετατρέπεται σε μια ομογενή γραμμική διαφορική εξίσωση 1\tss{ης} τάξης
\[ \varPhi'\left(z\right)-\frac{1}{z}\varPhi\left(z\right)=0 \]
Η γενική λύση αυτής της εξίσωσης θα δίνεται από τον τύπο
\[ \varPhi(z)=ce^{-\dintt{-\frac{1}{z}dz}}=ce^{\log{|z|}}=cz \]
όπου $ c\neq0 $ μια αυθαίρετη σταθερά. Επιλέγοντας δίχως βλάβη της γενικότητας $ c=1 $ παίρνουμε τη συνάρτηση $ \varPhi(z)=z $ η οποία μας δίνει το ζητούμενο ολοκληρωτικό παράγοντα $ \rho(x,y)=\frac{x}{y^2} $. Έτσι η εξίσωση \eqref{a202} θα πάρει τη μορφή
\begin{gather}\label{a203}
\frac{x}{y^2}\left(2y^3-3xy\right)dx+\frac{x}{y^2}\left(x^2+xy^2\right)dy=0\Rightarrow
\undercbrace{\left(2xy-\frac{3x^2}{y}\right)}_{M}dx+\undercbrace{\left(\frac{x^3}{y^2}+x^2\right)}_{N}dy=0
\end{gather}
Η τελευταία εξίσωση είναι αμέσως ολοκληρώσιμη αφού ισχύει $ \frac{\partial M}{\partial y}=\frac{\partial N}{\partial x}=2x+\frac{3x^2}{y^2} $. Θα υπάρχει λοιπόν μια συνάρτηση $ f(x,y) $ ώστε η εξίσωση \eqref{a203} να γραφτεί στη μορφή $ df(x,y)=0 $. Όλες οι λύσεις της αρχικής διαφορικής εξίσωσης θα δίνονται από τον τύπο $ f(x,y)=c $. Για τη συνάρτηση $ f $ θα ισχύει
\[ \frac{\partial f}{\partial x}=M=2xy-\frac{3x^2}{y}\ \textrm{ και }\ \frac{\partial f}{\partial y}=N=\frac{x^3}{y^2}+x^2 \]
Ολοκληρώνουμε την πρώτη σχέση ως προς $ x $ και παίρνουμε τη συνάρτηση $ f $ η οποία θα είναι
\[ f(x,y)=\int{\left(2xy-\frac{3x^2}{y}\right) dx}+g(y)=x^2y-\frac{x^3}{y}+g(y) \]
Παραγωγίζουμε τη συνάρτηση που προέκυψε και εξισώνουμε το αποτέλεσμα με τη συνάρτηση $ N $ οπότε θα έχουμε
\begin{gather*}
\frac{\partial f}{\partial x}=x^2+\frac{x^3}{y^2}+g'(y)
\ \textrm{ άρα }\ x^2+\frac{x^3}{y^2}+g'(y)=\frac{x^3}{y^2}+x^2\Rightarrow g'(y)=0
\end{gather*}
για κάποια συνάρτηση $ g(y) $. Παίρνουμε λοιπόν τη συνάρτηση $ g(y)=c $ και επιλέγουμε δίχως βλάβη της γενικότητας $ g(y)=0 $ οπότε και αποκτάμε τη συνάρτηση $ f $ η οποία θα είναι
\[ f(x,y)=x^2y-\frac{x^3}{y} \]
Επομένως όλες οι λύσεις της αρχικής διαφορικής εξίσωσης θα δίνονται από τους τύπους $ f(x,y)=c $, όπου $ c $ είναι μια αυθαίρετη σταθερά και $ y=0 $ δηλαδή
\[ y=0\ \textrm{ και }\ x^2y-\frac{x^3}{y}=c \]
\begin{Askhshs}[A]
\bmath{Έστω η πρώτης τάξης γραμμική διαφορική εξίσωση
\begin{equation}\label{a21e}
y'+py=q\tag{E}
\end{equation}
όπου $ p $ και $ q $ είναι συνεχείς πραγματικές συναρτήσεις στο διάστημα $ [0,+\infty) $. Ας υποθέσουμε ότι υπάρχουν $ x_0\geq0 $  και μια σταθερά $ \mu>0 $, έτσι ώστε $ p(x)\geq\mu $ για όλα τα $ x\geq x_0 $ και ότι $ \displaystyle{\lim_{x\rightarrow\infty}q(x)=0} $. Να αποδειχθεί ότι όλες οι λύσεις της \eqref{a21e} τείνουν προς το $ 0 $ για $ x\rightarrow\infty $.\\
\textit{Εφαρμογή }: Να αποδειχθεί ότι κάθε λύση της γραμμικής διαφορικής εξίσωσης
\[ y'+y\log{\left(\frac{5}{2}+\sin{x}\right) }=\frac{\cos{x}}{(x+1)^2}\ \ ,\ \ x\geq0 \]
τείνει προς το $ 0 $ για $ x\rightarrow\infty $.}
\end{Askhshs}\mbox{}\\
\lysh
Η γενική λύση της διαφορικής εξίσωσης \eqref{a21e} δίνεται από τον τύπο
\begin{equation}\label{a21}
y(x)=e^{-\int_{0}^{x}{p(t)dt}}\left[y(0)+\int_{0}^{x}{q(s)e^{\int_{0}^{s}{p(t)dt}}ds} \right]\ \ ,\ \ x\geq0 
\end{equation}
Θα πρέπει να δείξουμε ότι όταν $ x\rightarrow\infty $ ισχύει $ \displaystyle{\lim_{x\rightarrow\infty}q(x)=0} $ δηλαδή ισοδύναμα ότι
\begin{gather}
\lim_{x\rightarrow\infty}\left\lbrace e^{-\int_{0}^{x}{p(t)dt}}\left[y(0)+\int_{0}^{x}{q(s)e^{\int_{0}^{s}{p(t)dt}}ds} \right] \right\rbrace=0\Rightarrow\nonumber\\\label{a211}
y(0)\lim_{x\rightarrow\infty} e^{-\int_{0}^{x}{p(t)dt}}+\lim_{x\rightarrow\infty}\left(e^{-\int_{0}^{x}{p(t)dt}}\cdot \int_{0}^{x}{q(s)e^{\int_{0}^{s}{p(t)dt}}ds}\right)=0
\end{gather}
Γνωρίζουμε από την υπόθεση της άσκησης ότι υπάρχει σταθερά $ \mu $ ώστε να ισχύει η σχέση $ p(x)\geq\mu $. Ολοκληρώνοντας τη σχέση αυτή παίρνουμε
\begin{equation}\label{a213}
\int_{0}^{x}{p(t)dt}\geq\int_{0}^{x}{\mu dt}\Rightarrow \int_{0}^{x}{p(t)dt}\geq\mu x\Rightarrow e^{-\int_{0}^{x}{p(t)dt}}\leq e^{-\mu x}
\end{equation}
Από την τελευταία σχέση έχουμε ότι $ \displaystyle{\lim_{x\rightarrow\infty}e^{-\mu x}=0} $ άρα θα ισχύει
\begin{equation}\label{a212}
\lim_{x\rightarrow}{e^{-\int_{0}^{x}{p(t)dt}}}=0
\end{equation}
Επομένως η σχέση \eqref{a211} μέσω της \eqref{a212} θα γίνει
\begin{align}
\lim_{x\rightarrow\infty}{y(x)}&=y(0)\cdot0+\lim_{x\rightarrow\infty}\left(e^{-\int_{0}^{x}{p(t)dt}}\cdot \int_{0}^{x}{q(s)e^{\int_{0}^{s}{p(t)dt}}ds}\right)=\\
&=\lim_{x\rightarrow\infty}\left(e^{-\int_{0}^{x}{p(t)dt}}\cdot \int_{0}^{x}{q(s)e^{\int_{0}^{s}{p(t)dt}}ds}\right)=\\
&=\lim_{x\rightarrow\infty}\int_{0}^{x}{q(s)e^{-\int_{s}^{0}{p(t)dt}-\int_{0}^{x}{p(t)dt}}ds}=\lim_{x\rightarrow\infty}\int_{0}^{x}{q(s)e^{-\int_{s}^{x}{p(t)dt}}ds}
\end{align}
Συνδυάζοντας το ολοκλήρωμα με τη σχέση \eqref{a213} θα έχουμε 
\[ \left|\int_{0}^{x}q(s)e^{-\int_{s}^{x}{p(t)dt}}ds \right|\leq\int_{0}^{x}|q(s)|e^{-\int_{s}^{x}{p(t)dt}}ds\leq\int_{0}^{x}|q(s)|e^{-\mu(x-s)}ds=e^{-\mu x}\int_{0}^{x}|q(s)|e^{\mu s}ds \]
Η συνάρτηση $ f(s)=|q(s)| $ είναι μια συνεχής μη αρνητική συνάρτηση στο $ [0,\infty) $ και από την υπόθεση γνωρίζουμε επίσης ότι $ \displaystyle{\lim_{s\rightarrow\infty}q(s)=0} $. Θα έχουμε λοιπόν
\[ \lim_{x\rightarrow\infty}{e^{-\mu x}\int_{0}^{x}|q(s)|e^{\mu s}ds}=\lim_{x\rightarrow\infty}{\left(\frac{\int_{0}^{x}f(s)e^{\mu s}ds}{e^{\mu x}}\right)} \]
Για το όριο $ \displaystyle\lim_{x\rightarrow\infty}\int_{0}^{x}f(s)e^{\mu s}ds $ διακρίνουμε τις εξής περιπτώσεις :
\begin{rlist}
\item Να είναι πεπερασμένο οπότε θα έχουμε
\[ \lim_{x\rightarrow\infty}{\left(\frac{\int_{0}^{x}f(s)e^{\mu s}ds}{e^{\mu x}}\right)}=\frac{c}{\infty}=0 \]
\item Να είναι μη πεπερασμένο οπότε εφαρμόζοντας τον κανόνα του De L' Hospital θα ισχύει
\[ \lim_{x\rightarrow\infty}{\left(\frac{\int_{0}^{x}f(s)e^{\mu s}ds}{e^{\mu x}}\right)}=\lim_{x\rightarrow\infty}{\left(\frac{f(x)e^{\mu x}}{\mu e^{\mu x}}\right)}=\lim_{x\rightarrow\infty}\frac{f(x)}{\mu}=0 \]
\end{rlist}
Άρα οι λύσεις της εξίσωσης \eqref{a21e} τείνουν προς το $ 0 $ για $ x\rightarrow\infty $.\\\\
\textit{\textbf{Εφαρμογή}} :\\
Η δοσμένη εξίσωση είναι της ίδιας μορφής με την εξίσωση \eqref{a21e} με $ p(x)=\log\left(\frac{5}{2}+\sin{x} \right) $ και $ q(x)=\frac{\cos{x}}{(x+1)^2} $ οι οποίες είναι συνεχείς συναρτήσεις στο $ [0,\infty) $. Από τη γνωστή ιδιότητα $ 1-\frac{1}{x}\leq\log{x}\leq x-1 $ θα πάρουμε με αντικατάσταση
\[ 1-\frac{1}{\frac{5}{2}+\sin{x}}\leq\log{\left( \frac{5}{2}+\sin{x}\right) }\leq \frac{5}{2}+\sin{x}-1 \]
Έτσι θα έχουμε
\begin{equation}\label{a214}
p(x)=\log\left(\frac{5}{2}+\sin{x} \right)\geq 1-\frac{2}{5+2\sin{x}}
\end{equation}
Γνωρίζουμε ότι ισχύει $ \sin{x}\geq-1\Rightarrow 2\sin{x}\geq-2\Rightarrow 5+2\sin{x}\geq3\Rightarrow \frac{2}{5+2\sin{x}}\leq \frac{2}{3} $. Σύμφωνα μ' αυτό η \eqref{a214} θα μας δώσει
\[ p(x)=\log\left(\frac{5}{2}+\sin{x} \right)\geq 1-\frac{2}{5+2\sin{x}}\geq\frac{1}{3} \]
Αποδείξαμε λοιπόν ότι η συνάρτηση $ p(x) $ είναι κάτω φραγμένη οπότε πληροί τη συνθήκη $ p(x)\geq\mu $ με $ \mu=\frac{1}{3} $. Επίσης για τη συνάρτηση $ q(x) $ θα ισχύει $ {\displaystyle{\lim_{x\rightarrow\infty}}}{\frac{\cos{x}}{(x+1)^2}=0} $ ως μηδενική επί φραγμένη. Έτσι η γενική λύση της αρχικής διαφορικής εξίσωσης θα δίνεται από τον τύπο
\[ y(x)=e^{-\dintt_{\!\!0}^{x}{\log\left(\frac{5}{2}+\sin{t}\right) dt}}\left[y(0)+\int_{0}^{x}{\frac{\cos{s}}{(s+1)^2}e^{\dintt_{\!\!0}^{s}{\log\left(\frac{5}{2}+\sin{t}\right)dt}}ds} \right]\ \ ,\ \ x\geq0 \]
Συνεχίζοντας την επίλυση εκτελώντας την ίδια διαδικασία όπως προηγουμένως αποδεικνύουμε ότι για $ x\rightarrow\infty $ οι λύσεις $ y(x) $ θα τείνουν στο $ 0 $.\epask
\begin{Askhshs}[A]
\bmath{Να επιλυθεί το πρόβλημα αρχικών τιμών
\[ 2\left(y'\right)^2=(y-1)y''\ \ ,\ \ y(1)=2\ ,\ y'(1)=-1 \]}
\end{Askhshs}\mbox{}\\
\lysh
Η διαφορική εξίσωση αυτή είναι μια διαφορική εξίσωση δεύτερης τάξης μη περιέχουσα την ανεξάρτητη μεταβλητή. Έχει λύση τη συνάρτηση $ y=c $, όπου $ c $ είναι μια αυθαίρετη σταθερά, που όμως δεν πληροί την αρχική συνθήκη $ y'(1)=-1 $. Θέτοντας $ y'=z\Rightarrow y''=z\cdot\frac{dz}{dy} $ πετυχαίνουμε υποβιβασμό της τάξης της εξίσωσης οπότε θα έχουμε
\[  2z^2=(y-1)z\cdot\frac{dz}{dy}\Rightarrow \frac{z}{2z^2}dz=\frac{1}{y-1}dy\Rightarrow \frac{1}{2z}dz=\frac{1}{y-1}dy \]
Καταλήξαμε λοιπόν σε μια εξίσωση χωριζομένων μεταβλητών της οποίας οι λύσεις θα δίνονται από τον τύπο
\begin{gather*}
\int{\frac{1}{2z}dz}=\int{\frac{1}{y-1}dy}+c'\Rightarrow \frac{1}{2}\log|z|=\log|y-1|+c'\Rightarrow\\
\log|z|=\log|y-1|^2+2c'\Rightarrow \log{\left| \frac{z}{(y-1)^2}\right| }=2c'\Rightarrow
\frac{z}{(y-1)^2}=\pm e^{2c'}
\end{gather*}
Θέτουμε όπου $ \pm e^{2c'}=c_1 $ και με την αντικατάσταση $ z=y' $ οδηγούμαστε στην εξίσωση
\[ y'=c_1(y-1)^2\Rightarrow \frac{dy}{dx}=c_1(y-1)^2\Rightarrow \frac{dy}{(y-1)^2}=c_1dx \]
η οποία είναι μια εξίσωση χωριζομένων μεταβλητών. Οι λύσεις αυτής θα δίνονται από τον τύπο :
\begin{gather*}
\int\frac{dy}{(y-1)^2}=\int c_1dx+c_2\Rightarrow -\frac{1}{y-1}=c_1x+c_2\Rightarrow y=1-\frac{1}{c_1x+c_2}
\end{gather*}
όπου $ c_1\neq0\ ,\ c_2 $ είναι αυθαίρετες σταθερές. Συνεπώς όλες οι λύσεις της αρχικής διαφορικής εξίσωσης θα δίνονται από τους τύπους
\[ y=c\ \textrm{ και }\ y=1-\frac{1}{c_1x+c_2} \]
Για τη λύση $ y $ η οποία πληροί τις αρχικές συνθήκες θα έχουμε $ y'=\frac{c_1}{(c_1x+c_2)^2} $ οπότε
\[ \ccases{y(1)=2\Rightarrow 2=1-\frac{1}{c_1+c_2}\Rightarrow c_1+c_2=-1\\
y'(1)=-1\Rightarrow -1=\frac{c_1}{(-c_1+c_2)^2}\Rightarrow (c_2-c_1)^2+c_1=0}\Rightarrow c_1=-1\ \textrm{και}\ c_2=0 \]
Η λύση λοιπόν η οποία πληροί τις αρχικές συνθήκες του προβλήματος αρχικών τιμών θα είναι η
\[ y=1+\frac{1}{x} \]
\begin{Askhshs}[A]
\bmath{Με την αντικατάσταση $ \frac{y'}{y}=z $, να επιλυθεί το πρόβλημα αρχικών τιμών
\[ x^2yy''-\left(xy'-y \right)^2=0\ \ ,\ \ y(1)=1,\ y'(1)=0 \]}
\end{Askhshs}\mbox{}\\
\lysh
Η διαφορική εξίσωση έχει λύση την $ y=0 $ η οποία όμως δεν πληροί τις αρχικές συνθήκες $ y(1)=1 $ και $ y'(1)=0 $ του προβλήματος. Για τις μη μηδενικές λύσεις θα έχουμε :
\begin{gather}
x^2yy''-\left(xy'-y \right)^2=0\Rightarrow yy''-\left(y'-\frac{y}{x} \right)^2=0\Rightarrow\nonumber\\\label{a23}
\frac{y''}{y}-\left(\frac{y'}{y}-\frac{1}{x} \right)^2=0
\end{gather}
Θέτουμε στην \eqref{a23} όπου $ \frac{y'}{y}=z\Rightarrow z'=\frac{y''y-y'^2}{y^2}=\frac{y''}{y}-\left( \frac{y'}{y}\right)^2 $ οπότε και παίρνει τη μορφή
\[ z'+z^2-\left(z-\frac{1}{x} \right)^2=0\Rightarrow z'+\frac{2}{x}z=\frac{1}{x^2} \]
Η τελευταία εξίσωση είναι γραμμική διαφορική εξίσωση 1\tss{ης} τάξης και οι λύσεις της θα δίνονται από τον τύπο
\[ z(x)=e^{-\dintt{\frac{2}{x}dx}}\left[c_1+\int{\frac{1}{x^2}\cdot e^{\dintt{\frac{2}{x}dx}}dx}\right]=\frac{1}{x^2}\left(c_1+x \right)=\frac{c_1+x}{x^2} \]
Αντικαθιστώντας ξανά όπου $ z=\frac{y'}{y} $ παίρνουμε μια εξίσωση χωριζομένων μεταβλητών.
\begin{equation}\label{a231}
\frac{y'}{y}=\frac{c_1+x}{x^2}\Rightarrow \frac{1}{y}dy=\frac{c_1+x}{x^2}dx
\end{equation}
Οι λύσεις αυτής θα δίνονται από τον τύπο
\begin{gather}
\int\frac{1}{y}dy=\int\frac{c_1+x}{x^2}dx+c_2\Rightarrow \log|y|=\log|x|-\frac{c_1}{x}+c_2\Rightarrow\\
\log\left| \frac{y}{x}\right| =-\frac{c_1}{x}+c_2\Rightarrow \left| \frac{y}{x}\right|=e^{-\frac{c_1}{x}}\cdot e^{c_2}\Rightarrow y=\frac{cx}{e^{c_1/x}}
\end{gather}
όπου έχουμε θέσει $ c=\pm e^{c_2} $. Εξετάζοντας τώρα τις αρχικές συνθήκες του προβλήματος και χρησιμοποιώντας τη σχέση \eqref{a231}, θα υπολογίσουμε τις σταθερές $ c $ και $ c_1 $. Θα έχουμε
\[ \ccases{y(1)=1\Rightarrow 1=\frac{c}{e^{c_1}}\\
y'(1)=0\Rightarrow \frac{y'(1)}{y(1)}=\frac{c_1+1}{1^2}\Rightarrow c_1+1=0}\Rightarrow c_1=-1\ \textrm{και}\ c=\frac{1}{e}\]
Επομένως η λύση του προβλήματος αρχικών τιμών θα είναι η \[ y=\frac{xe^{1/x}}{e}=xe^{\frac{1-x}{x}} \]
\begin{Askhshs}[A]
\bmath{Δίνεται η εξίσωση 
\begin{equation}\tag{E}
\left( axy+\beta y^2\right)dx+\left(axy+\beta x^2\right)dy=0  
\end{equation}
όπου $ a $ και $ \beta\neq0 $ είναι πραγματικές σταθερές. Να βρεθεί ένας ολοκληρωτικός παράγοντας της $ (E) $ της μορφής $ \rho(x,y)=\varPhi(x+y) $, όπου $ \varPhi $ είναι κατάλληλη συνάρτηση. Στη συνέχεια, με χρήση αυτού του ολοκληρωτικού παράγοντα ή με άλλο τρόπο να επιλυθεί η $ (E) $.}
\end{Askhshs}\mbox{}\\
\lysh
\textbf{1\tss{ος} Τρόπος}\\
Μια προφανής λύση της αρχικής εξίσωσης είναι η $ y=0 $. Η συνάρτηση $ \rho(x,y)=\varPhi(x+y) $ θα αποτελεί ολοκληρωτικό παράγοντα της αρχικής εξίσωσης αν και μόνο αν η εξίσωση που θα προκύψει αν πολλαπλασιάσουμε και τα δύο μέλη της $ (E) $ με τη συνάρτηση $ \rho $ είναι αμέσως ολοκληρώσιμη. Η νέα εξίσωση θα έχει τη μορφή
\begin{equation}\label{a242}
\undercbrace{\varPhi(x+y)\left( axy+\beta y^2\right)}_{M}dx+\undercbrace{\varPhi(x+y)\left(axy+\beta x^2\right)}_{N}dy=0
\end{equation}
και απαιτούμε να ισχύει η συνθήκη $ \frac{\partial M}{dy}=\frac{\partial N}{dx} $. Αναλυτικά οι δύο μερικές παράγωγοι έχουν ως εξής :
\begin{align}\label{a24}
&\frac{\partial M}{dy}=\frac{\partial }{dy}\left[\varPhi(x+y)\left( axy+\beta y^2\right)\right]=\varPhi'(x+y)\left( axy+\beta y^2\right)+\varPhi(x+y)\left( ax+2\beta y\right)\ \textrm{και}\\\label{a241}
&\frac{\partial N}{dx}=\frac{\partial}{dy}\left[\varPhi(x+y)\left(axy+\beta x^2\right) \right]=\varPhi'(x+y)\left(axy+\beta x^2\right)+\varPhi(x+y)\left(ay+2\beta x\right) 
\end{align}
Εξισώνουμε τις παραστάσεις \eqref{a24} και \eqref{a241} οπότε προκύπτει
\begin{gather*}
\varPhi'(x+y)\left( axy+\beta y^2\right)+\varPhi(x+y)\left( ax+2\beta y\right)=\varPhi'(x+y)\left(axy+\beta x^2\right)+\varPhi(x+y)\left(ay+2\beta x\right)\Rightarrow\\
\varPhi'(x+y)\left(\beta y^2-\beta x^2\right)+\varPhi(x+y)\left( ax+2\beta y-ay-2\beta x\right)=0\Rightarrow\\
\varPhi'(x+y)\beta\left(y-x\right)(y+x)+\varPhi(x+y)\left[
(y-x)(2\beta-a) \right]=0\Rightarrow\\
\varPhi'(x+y)\beta(y+x)+\varPhi(x+y)
(2\beta-a)=0\Rightarrow
\varPhi'(x+y)+\frac{2\beta-a}{\beta(y+x)}\varPhi(x+y)
=0
\end{gather*}
Στην τελευταία εξίσωση θέτουμε όπου $ x+y=z $ και καταλήγουμε στην 
\[ \varPhi'(z)+\frac{2\beta-a}{\beta z}\varPhi(z)
=0 \]
η οποία είναι μια ομογενής γραμμική διαφορική εξίσωση 1\tss{ης} τάξης της οποίας οι λύσεις θα δίνονται από τον παρακάτω τύπο
\[ \varPhi(z)=c_1e^{-\dintt{\frac{2\beta-a}{\beta z}dz}}=c_1e^{\frac{a-2\beta}{\beta }\log|z|}=c_1z^{\frac{a-2\beta}{\beta}} \]
όπου $ c_1\neq0 $ είναι μια αυθαίρετη σταθερά. Δίχως βλάβη της γενικότητας επιλέγουμε $ c_1=1 $ και παίρνουμε τν ολοκληρωτικό παράγοντα $ \rho(x,y)=(x+y)^{\frac{a-2\beta}{\beta}} $. Έτσι η εξίσωση \eqref{a242} θα γίνει
\[ \undercbrace{(x+y)^{\frac{a-2\beta}{\beta}}\left( axy+\beta y^2\right)}_{M}dx+\undercbrace{(x+y)^{\frac{a-2\beta}{\beta}}\left(axy+\beta x^2\right)}_{N}dy=0 \]
Γνωρίζουμε ήδη ότι η εξίσωση που προκύπτει είναι αμέσως ολοκληρώσιμη οπότε θα υπάρχει συνάρτηση $ f(x,y) $ τέτοια ώστε η εξίσωση να γραφτεί στη μορφή $ df(x,y)=0 $. Τότε θα ισχύουν οι σχέσεις
\[ \frac{\partial f}{dx}=M=(x+y)^{\frac{a-2\beta}{\beta}}\left( axy+\beta y^2\right)\ \textrm{ και }\ \frac{\partial f}{dy}=N=(x+y)^{\frac{a-2\beta}{\beta}}\left(axy+\beta x^2\right) \]
Ολοκληρώνοντας μια από τις δύο σχέσεις ως προς την αντίστοιχη μεταβλητή θα οδηγηθούμε στον τύπο της συνάρτησης $ f $. Θα έχουμε λοιπόν
\begin{align*}
f(x,y)&=\int{(x+y)^{\frac{a-2\beta}{\beta}}\left( axy+\beta y^2\right)dx}+g(y)=y\int{(x+y)^{\frac{a-2\beta}{\beta}}\left( ax+\beta y\right)dx}+g(y)=\\
&=y\int{ax(x+y)^{\frac{a-2\beta}{\beta}} dx}+y\int{\beta y(x+y)^{\frac{a-2\beta}{\beta}}dx}+g(y)=\\
&=ay\int{(x+y)^{\frac{a-2\beta}{\beta}}(x+y-y) dx}+\beta y^2\int{(x+y)^{\frac{a-2\beta}{\beta}}dx}+g(y)=\\
&=ay\int{(x+y)^{\frac{a-2\beta}{\beta}}(x+y) dx}-ay^2\int{(x+y)^{\frac{a-2\beta}{\beta}} dx}+\beta y^2\int{(x+y)^{\frac{a-2\beta}{\beta}}dx}+g(y)=\\
&=ay\int{(x+y)^{\frac{a-\beta}{\beta}} d(x+y)}-(\beta-a)y^2\int{(x+y)^{\frac{a-2\beta}{\beta}} d(x+y)}+g(y)=\\
&=\beta y(x+y)^{a/\beta}-\beta y^2(x+y)^{\frac{a+\beta}{\beta}}+g(y)
\end{align*}
όπου $ g(y) $ είναι μια συνάρτηση του $ y $. Παραγωγίζουμε την τελευταία σχέση ως προς $ y $ και εξισώνουμε την παράσταση που θα προκύψει με τη συνάρτηση $ N $. Έχουμε λοιπόν
\[ \frac{\partial f}{dy}=\beta(x+y)^{a/\beta}+(a-2\beta)y(x+y)^{\frac{a-\beta}{\beta}}-(\beta-a)y^2 (x+y)^{\frac{a-2\beta}{\beta}}+g'(y) \]
οπότε θα έχουμε
\begin{align*}
\beta(x+y)^{a/\beta}+(a-2\beta)y(x+y)^{\frac{a-\beta}{\beta}}&-(\beta-a)y^2 (x+y)^{\frac{a-2\beta}{\beta}}+g'(y)=\\
&=(x+y)^{\frac{a-2\beta}{\beta}}\left(axy+\beta x^2\right)\Rightarrow g'(y)=0
\end{align*}
Επιλέγουμε δίχως βλάβη της γενικότητας $ g(y)=0 $ καταλήγουμε στη συνάρτηση $ f(x,y)=\beta y(x+y)^{a/\beta}-\beta y^2(x+y)^{\frac{a+\beta}{\beta}} $. Αυτό μας δίνει τελικά τον τύπο από τον οποίο δίνονται όλες οι λύσεις της αρχικής διαφορικής εξίσωσης ο οποίος θα είναι
\[ \beta y(x+y)^{a/\beta}-\beta y^2(x+y)^{\frac{a+\beta}{\beta}}=c\]
όπου $ c $ είναι μια αυθαίρετη σταθερά. Η προφανής λύση $ y=0 $ επαληθεύει τον προηγούμενο τύπο.\\\\
\textbf{2\tss{ος} Τρόπος}\\
Μια προφανής λύση της εξίσωσης είναι η $ y=0 $, Για τις μη μηδενικές λύσεις θα φέρουμε την αρχική εξίσωση στη μορφή
\begin{equation}\label{a243}
\left( axy+\beta y^2\right)dx=-\left(axy+\beta x^2\right)dy\Rightarrow y'=-\frac{axy+\beta y^2}{axy+\beta x^2}
\end{equation}
Η ομογενής διαφορική εξίσωση \eqref{a243} με βαθμό ομογένειας $2$, στην οποία καταλήξαμε, είναι ισοδύναμη με την αρχική διαφορική εξίσωση διότι η συνάρτηση $ y=-\frac{\beta}{a}x $ δεν αποτελεί λύση της αρχικής. Θέτουμε στην πρώτη όπου $ y=xz\Rightarrow y'=xz'+z $ και έτσι παίρνει τη μορφή
\begin{gather*}
xz'+z=-\frac{ax^2z+\beta x^2z^2}{ax^2z+\beta x^2}\Rightarrow xz'=-\frac{az+\beta z^2}{az+\beta}-z\Rightarrow
xz'=-\frac{az+\beta z^2-az^2-\beta z}{az+\beta}\Rightarrow\\xz'=-\frac{(a+\beta)\left( z^2+z\right) }{az+\beta}\Rightarrow -\frac{az+\beta}{(a+\beta)\left( z^2+z\right)}dz=-\frac{dx}{x}
\end{gather*}
η οποία είναι μια διαφορική εξίσωση χωριζομένων μεταβλητών. Εύκολα παρατηρούμε ότι η $ z=-\frac{\beta}{a} $ η οποία αντιστοιχεί στη λύση $ y=-\frac{\beta}{a}x $ δεν αποτελεί λύση της, ενώ η προφανής $ z=0 $ αντιστοιχεί στην $ y=0 $ που συναντήσαμε προηγουμένως. Οι λύσεις αυτής θα δίνονται από τον τύπο
\begin{gather*}
\int\frac{az+\beta}{(a+\beta)\left( z^2+z\right)}dz=-\int\frac{dx}{x}+c_1\Rightarrow\\ \frac{\beta}{a+\beta}\log{|z|}+\frac{a-\beta}{a+\beta}\log{|z+1|}=-\log{|x|}+c_1\Rightarrow\\
\beta\log{|z|}+(a-\beta)\log{|z+1|}+(a+\beta)\log{|x|}=(a+\beta)c_1\Rightarrow\\
\left|z^\beta(z+1)^{a-\beta}\cdot x^{(a+\beta)}\right|=e^{c_1(a+\beta)}\Rightarrow\\ z^\beta(z+1)^{a-\beta}\cdot x^{(a+\beta)}=\pm e^{c_1(a+\beta)}
\end{gather*}
Θέτοντας όπου $ c=\pm e^{c_1(a+\beta)} $ αποκτάμε τον τύπο από τον οποίο δίνονται όλες οι λύσεις της αρχικής διαφορικής εξίσωσης
\[ z^\beta(z+1)^{a-\beta}\cdot x^{(a+\beta)}=\pm e^{c_1(a+\beta)} \]
όπου $ c $ είναι μια αυθαίρετη σταθερά. Η προφανής λύση $ y=0 $ ανήκει στο παραπάνω σύνολο λύσεων.\epask
\begin{Askhshs}[A]
\bmath{Να επιλυθεί η διαφορική εξίσωση
\[ \dfrac{dy}{dx}=-\frac{y\left(2x^2+2xy^2+1\right) }{x+3y^2} \]}
\end{Askhshs}\mbox{}\\
\lysh
Η αρχική διαφορική εξίσωση γράφεται ισοδύναμα στην παρακάτω μορφή
\begin{equation}\label{a25}
\undercbrace{\left(2yx^2+2xy^3+y\right)}_{M}dx+\undercbrace{\left(x+3y^2\right)}_{N}dy=0
\end{equation}
Οι συναρτήσεις $ M(x,y)=2yx^2+2xy^3+y $ και $ N(x,y)=x+3y^2 $ είναι κλάσεως $ C^1 $ που σημαίνει ότι είναι συνεχείς και έχουν συνεχείς μερικές παραγώγους 1\tss{ης} τάξης. Παρατηρούμε όμως ότι
\[ \frac{\partial M}{\partial y}=2x^2+6xy^2+1\qquad\textrm{και}\qquad\frac{\partial N}{\partial x}=1 \]
Συνεπώς αφού ισχύει $ \frac{\partial M}{\partial y}\neq\frac{\partial N}{\partial x} $ η εξίσωση \eqref{a25} δεν είναι αμέσως ολοκληρώσιμη. Θα αναζητήσουμε έναν ολοκληρωτικό παράγοντα. Παρατηρούμε ότι η παράσταση 
\[ \dfrac{\left(\frac{\partial M}{\partial y}-\frac{\partial N}{\partial x}\right) }{N}=\frac{2x^2+6xy^2+1-1}{x+3y^2}=\frac{2x\left(x+3y^2\right)}{x+3y^2}=2x \] αποτελεί συνάρτηση μόνο του $ x $. Αυτό σημαίνει ότι ένας ολοκληρωτικός παράγοντας της εξίσωσης θα είναι ο $ \rho(x)=e^{\int{2x}dx}=e^{x^2} $. Πολλαπλασιάζοντας και τα δύο μέλη της \eqref{a25} με τον παράγοντα αυτό αποκτάμε την εξίσωση 
\begin{equation}\label{a251}
\undercbrace{e^{x^2}\left(2yx^2+2xy^3+y\right)}_{M}dx+\undercbrace{e^{x^2}\left(x+3y^2\right)}_{N}dy=0
\end{equation}
η οποία είναι μια αμέσως ολοκληρώσιμη εξίσωση. Έτσι θα υπάρχει μια συνάρτηση $ f(x,y) $ τέτοια ώστε η \eqref{a251} να γραφτεί στη μορφή $ df(x,y)=0 $. Θα έχουμε λοιπόν
\[ \frac{\partial f}{\partial x}=M=e^{x^2}\left(2yx^2+2xy^3+y\right)\qquad\textrm{και}\qquad\frac{\partial f}{\partial y}=N=e^{x^2}\left(x+3y^2\right) \]
Ολοκληρώνοντας τη δεύτερη σχέση ως προς $ y $ θα έχουμε
\[ f(x,y)=\int{e^{x^2}\left(x+3y^2\right)}dy+g(x)=xye^{x^2}+y^3e^{x^2}+g(x) \] για κάποια συνάρτηση $ g(x) $. Παραγωγίζουμε την τελευταία σχέση ως προς $ x $ και εξισώνουμε τη συνάρτηση που θα προκύψει με την $ M(x,y) $.
\begin{gather*}
\frac{\partial f}{\partial x}=ye^{x^2}+2x^2ye^{x^2}+2xy^3e^{x^2}+g'(x)\Rightarrow\\
ye^{x^2}+2x^2ye^{x^2}+2xy^3e^{x^2}+g'(x)=e^{x^2}\left(2yx^2+2xy^3+y\right)\Rightarrow g'(x)=0
\end{gather*}
Επιλέγοντας δίχως βλάβη της γενικότητας όπου $ g(x)=0 $ καταλήγουμε στη συνάρτηση $ f(x,y) $ η οποία θα δίνεται από τον τύπο
\[ f(x,y)=xye^{x^2}+y^3e^{x^2} \]
Έτσι όλες οι λύσεις της αρχικής διαφορικής εξίσωσης θα δίνονται από τη σχέση $ f(x,y)=c $ δηλαδή
\[ xye^{x^2}+y^3e^{x^2} \]
όπου $ c $ είναι μια αυθαίρετη σταθερά. Η προφανής λύση $ y=0 $ έχει συμπεριληφθεί στην παραπάνω σχέση.\epask
\begin{Askhshs}[A]
\bmath{Ας είναι $ b,c $ θετικές σταθερές. Να αποδειχθεί ότι κάθε λύση $ y $  της λογιστικής εξίσωσης $ y'=y(b-cy) $, με $ y(0)>0 $, παραμένει θετική για $ x>0 $ και τείνει προς τη λύση $ b/c $ για $ x\rightarrow\infty $.}
\end{Askhshs}\mbox{}\\
\lysh
Η αρχική διαφορική εξίσωση γράφεται ισοδύναμα στη μορφή
\[ y'=yb-cy^2\Rightarrow y'-by=-cy^2 \]
η οποία είναι μια διαφορική εξίσωση Bernoulli με $ r=2 $. Αυτή έχει προφανή λύση την $ y=0 $ η οποία όμως δεν ικανοποιεί την αρχική συνθήκη του προβλήματος $ y(0)>0 $. Για τις μη μηδενικές λύσεις θα θέσουμε $ z=y^{-1}\Rightarrow z'=-\frac{y'}{y^2} $ οπότε έχουμε
\begin{equation}\label{a26}
y'-by=-cy^2\Rightarrow \frac{y'}{y^2}-\frac{b}{y}=-c\Rightarrow z'+bz=c
\end{equation}
η οποία είναι μια γραμμική διαφορική εξίσωση 1\tss{ης} τάξης. Η γενική λύση της \eqref{a26} θα δίνεται από τον τύπο
\begin{align*}
z(x)=e^{-\int_{0}^{x}{b}dt}\left[z(0)+\int_{0}^{x}{c\cdot e^{\int_{0}^{s}{b}dt}ds}\right]&=e^{-bx}\left[z(0)+\int{ce^{bs}ds}\right]=\\
&=e^{-bx}\left[z(0)+\frac{c}{b}\left( e^{bx}-1\right) \right]=z(0)e^{-bx}+\frac{c}{b}\left( 1-e^{-bx}\right)
\end{align*}
Με αντικατάσταση όπου $ z=\frac{1}{y} $ παίρνουμε τον τύπο από τον οποίο δίνονται όλες οι λύσεις της αρχικής διαφορικής εξίσωσης :
\[ y(x)=\frac{1}{\frac{e^{-bx}}{y(0)}+\frac{c}{b}\left( 1-e^{-bx}\right)}\;\; ,\;\;x\geq0 \]
Για το πρόσημο της συνάρτησης $ y(x) $ παρατηρούμε ότι $ \frac{e^{-bx}}{y(0)}>0 $ και $ \frac{c}{b}\left( 1-e^{-bx}\right)\geq0\ ,\ \forall x\geq0 $.
Συνεπώς προκύπτει $ y(x)>0\ ,\ \forall x>0 $. Επίσης για το όριο της συνάρτησης $ y $ καθώς $ x\rightarrow\infty $ θα έχουμε :
\[ \lim_{x\rightarrow\infty}{y(x)}=\lim_{x\rightarrow\infty}{\frac{1}{\frac{e^{-bx}}{y(0)}+\frac{c}{b}\left( 1-e^{-bx}\right)}}=\frac{1}{\frac{e^{-b\cdot0}}{y(0)}+\frac{c}{b}\left( 1-e^{-b\cdot0}\right)}=\frac{1}{\frac{c}{b}}=\frac{b}{c} \]\epask
\begin{Askhshs}[A]
\bmath{Έστω η πρώτης τάξης γραμμική διαφορική εξίσωση
\begin{equation}\label{a27}
y'=ay+b\tag{E}
\end{equation}
όπου $ a,b $ είναι συνεχείς συναρτήσεις στο διάστημα $ [0,\infty) $. Να αποδειχθεί ότι
\begin{rlist}
\item Αν $ a(x)\leq m $ για όλα τα $ x\geq0 $ όπου $ m $ είναι μια αρνητική σταθερά, και η συνάρτηση $ b $ είναι φραγμένη στο $ [0,\infty) $, τότε κάθε λύση της \eqref{a27} είναι φραγμένη στο διάστημα $ [0,\infty) $.
\item Αν $ a(x)\geq k $ για όλα τα $ x\geq0 $ όπου $ k $ είναι μια θετική σταθερά, και η συνάρτηση $ b $ είναι φραγμένη στο $ [0,\infty) $, τότε υπάρχει ακριβώς μια λύση της (Ε) που είναι φραγμένη στο διάστημα $ [0,\infty) $ και η λύση αυτή δίνεται από τον τύπο 
\[ y(x)=-\int_{x}^{\infty}{b(s)e^{\int_{s}^{x}{a(t)dt}}ds}\;,\;x\geq0 \]
\end{rlist}}
\end{Askhshs}\mbox{}\\
\lysh
\begin{rlist}
\item Οι λύσεις της εξίσωσης \eqref{a27} δίνονται από τον τύπο
\[ y(x)=e^{\int_{0}^{x}{a(t)dt}}\left[y(0)+\int_{0}^{x}{b(s)\cdot e^{\int_{0}^{s}{-a(t)dt}}ds}\right]\ ,\ x>0  \]
Από την υπόθεση γνωρίζουμε ότι ισχύει $ a(x)\leq m\Rightarrow -a(x)\geq -m $. Έτσι θα έχουμε
\[ \int_{0}^{x}{-a(t)dt}\geq\int_{0}^{x}{-mdt}\Rightarrow -\int_{0}^{x}{a(t)dt}\geq -mx\Rightarrow \int_{0}^{x}{a(t)dt}\leq mx\Rightarrow e^{\int_{0}^{x}{a(t)dt}}\leq e^{mx} \]
Για τη συνάρτηση $ b $ επίσης γνωρίζουμε ότι είναι φραγμένη οπότε θα υπάρχει σταθερός αριθμός $ M $ ώστε να ισχύει $ |b(x)|\leq M $. Εφαρμόζοντας τα παραπάνω στη γενική λύση της εξίσωσης θα έχουμε
\begin{gather*}
\left| e^{\int_{0}^{x}{a(t)dt}}\left[y(0)+\int_{0}^{x}{b(s)\cdot e^{\int_{0}^{s}{-a(t)dt}}ds}\right]\right|=
\left| y(0)e^{\int_{0}^{x}{a(t)dt}}+\int_{0}^{x}{b(s)\cdot e^{\int_{0}^{s}{a(t)dt}-\int_{0}^{x}{a(t)dt}}ds}\right|=\\
\left| y(0)e^{\int_{0}^{x}{a(t)dt}}+\int_{0}^{x}{b(s)\cdot e^{\int_{s}^{x}{a(t)dt}}ds}\right|\leq \left| y(0)\right|e^{mx} +M\int_{0}^{x}{e^{m(x-s)}ds}=\\\left| y(0)\right|e^{mx} +Me^{mx}\int_{0}^{x}{e^{-ms}ds}=\left| y(0)\right|e^{mx} -\frac{M}{m}e^{mx}\left(e^{-mx}-1\right)=\left| y(0)\right|e^{mx}-\frac{M}{m}+\frac{M}{m}e^{mx} 
\end{gather*}
\end{rlist}
Η τελευταία συνάρτηση συγκλίνει όταν $ x\rightarrow\infty $ καθώς ισχύει \[ \lim_{x\rightarrow\infty}{|y(x)|}\leq \lim_{x\rightarrow\infty}{\left( \left| y(0)\right|e^{mx}-\frac{M}{m}+\frac{M}{m}e^{mx}\right) }=-\frac{M}{m}\]
πράγμα που σημαίνει πως κάθε λύση της εξίσωσης \eqref{a27} είναι φραγμένη στο διάστημα $ (0,+\infty) $.\epask
\section{Β - Γραμμικές διαφορικές εξισώσεις}
\begin{Askhshs}[B]
\bmath{Να επιλυθεί η διαφορική εξίσωση
\[ x^2\frac{d^2y}{dx^2}-x\frac{dy}{dx}+y=x^2\log{x}\ ,\ x>0 \]}
\end{Askhshs}\mbox{}\\
\lysh
Η αρχική διαφορική εξίσωση αποτελεί μια διαφορική εξίσωση Euler. Συνεπώς με το μετασχηματισμό $ t=\log{x} $ θα μετατραπεί σε μια διαφορική εξίσωση 2\tss{ης} τάξης με σταθερούς συντελεστές. Με τη βοήθεια του μετασχηματισμού προκύπτει ότι
\begin{align}
x\frac{dy}{dx}&=x\frac{dy}{dt}\cdot\frac{dt}{dx}=x\frac{dy}{dt}\cdot\frac{1}{x}=\frac{dy}{dt}\quad\textrm{και}\label{b1}\\
x^2\frac{d^2y}{dx^2}&=x^2\frac{d}{dx}\left( \frac{dy}{dx}\right)=x^2\frac{d}{dx}\left( \frac{1}{x}\frac{dy}{dt}\right)=x^2\left[ -\frac{1}{x^2}\frac{dy}{dt}+\frac{1}{x}\frac{d}{dx}\left( \frac{dy}{dt}\right) \right]=\nonumber\\\label{b11}
&=x^2\left[ -\frac{1}{x^2}\frac{dy}{dt}+\frac{1}{x}\frac{d}{dt}\left( \frac{dy}{dx}\right) \right]=
x^2\left[ -\frac{1}{x^2}\frac{dy}{dt}+\frac{1}{x^2} \frac{d^2y}{dt^2} \right]=-\frac{dy}{dt}+\frac{d^2y}{dt^2}
\end{align}
Με τις σχέσεις \eqref{b1} και \eqref{b11} η αρχική διαφορική εξίσωση θα πάρει τη μορφή :
\begin{equation}
-\frac{dy}{dt}+\frac{d^2y}{dt^2}-\frac{dy}{dt}+y=e^{2t}t\Rightarrow \frac{d^2y}{dt^2}-2\frac{dy}{dt}+y=e^{2t}t\ ,\ t\in\mathbb{R}
\end{equation}
Η τελευταία είναι μια μη ομογενής γραμμική εξίσωση 2ης τάξης με σταθρούς συντελεστές $ a_2=1, a_1=-2 $ και $ a_0=1 $. Το χαρακτηριστικό πολυώνυμο της αντίστοιχης ομογενούς εξίσωσης θα είναι το \[ P(\lambda)=\lambda^2-2\lambda+1 \]
το οποίο έχει ρίζα την $ \lambda=1 $ με πολλαπλότητα $ 2 $. Συνεπώς οι συναρτήσεις 
\[ y_1(t)=e^t\quad\textrm{και}\quad y_2(t)=te^t\ ,\ t\in\mathbb{R} \]
οι οποίες είναι γραμμικά ανεξάρτητες θα αποτελούν ένα βασικό σύνολο λύσεων της ομογενούς εξίσωσης. Όλες οι λύσεις αυτής θα δίνονται από τον τύπο
\[ \tilde{y}(t)=c_1y_1(t)+c_2y_2(t)=c_1\cdot e^t+c_2\cdot te^t \]
όπου $ c_1,c_2 $ είναι αυθαίρετες σταθερές. Μένει να βρεθεί μια μερική λύση $ y_\mu(t) $ της μη ομογενούς εξίσωσης. Έτσι θέτουμε όπου $ y=ze^{2t} $ και θα έχουμε
\begin{gather}
y'=z'e^{2t}+2ze^{2t}=\left( z'+2z\right) e^{2t}\\
y''=\left( z''+2z'\right) e^{2t}+2\left( z'+2z\right) e^{2t}=\left(z''+4z'+4z\right)e^{2t} 
\end{gather}
Αντικαθιστώντας τις σχέσεις αυτές στη μή ομογενή εξίσωση θα προκύψει :
\begin{gather*}
\left(z''+4z'+4z\right)e^{2t}-\left(2z'+4z\right)e^{2t}+ze^{2t}=te^{2t}\Rightarrow\\
z''+2z'+z=t\ ,\ t\in\mathbb{R}
\end{gather*}
Μια μερική λύση της τελευταίας εξίσωσης είναι η $ z_\mu(t)=d_1t+d_0 $. Έτσι $ \forall t\in\mathbb{R} $ θα έχουμε $ z'_\mu(t)=d_1 $ και $ z''_\mu(t)=0 $. Με αντικατάσταση των σχέσεων αυτών η εξίσωση θα γίνει :
\[ 2d_1+d_1t+d_0=t\Rightarrow \ccases{d_1=1\\2d_1+d_0=0\Rightarrow d_0=-2} \]
Οι συντελεστές αυτοί μας δίνουν τη μερική λύση $ z_\mu(t)=t-2 $. Συνεπώς η μερική λύση της αρχικής μη ομογενούς θα έχει τη μορφή : 
\[ y_\mu(t)=z_\mu(t)e^{2t}=(t-2)e^{2t}\ ,\ t\in\mathbb{R} \]
Οπότε όλες οι λύσεις της αρχικής διαφορικής εξίσωσης θα δίνονται από τον τύπο
\[ y(t)=c_1\cdot e^t+c_2\cdot te^t+(t-2)e^{2t}\ ,\ t\in\mathbb{R} \]
όπου $ c_1,c_2 $ είναι αυθαίρετες σταθερές. Αντικαθιστώντας τέλος όπου $ t=\log{x} $ προκύπτει ο τύπος \[ y(x)=c_1x+c_2\cdot x\log{x}+(\log{x}-2)x^2\ ,\ x>0 \]\epask
\begin{Askhshs}[B]
\textbf{Να επιλυθεί η διαφορική εξίσωση}
{\boldmath\[ x^2\frac{d^2y}{dx^2}-2x\frac{dy}{dx}+2y=x^3e^x\ ,\ x>0 \]}
\end{Askhshs}\mbox{}\\
\lysh
Ξεκινάμε με την ομογενή εξίσωση η οποία παρατηρούμε ότι είναι μια διαφορική εξίσωση Euler.
\[ x^2\frac{d^2y}{dx^2}-2x\frac{dy}{dx}+2y=0\ ,\ x>0 \]
Σ' αυτήν θέτουμε $ t=\log{x}\ ,\ x>0 $. Τότε $ \forall x>0 $ θα έχουμε 
\begin{align}
x\frac{dy}{dx}&=x\frac{dy}{dt}\cdot\frac{dt}{dx}=x\frac{dy}{dt}\cdot\frac{1}{x}=\frac{dy}{dt}\quad\textrm{και}\label{b2}\\
x^2\frac{d^2y}{dx^2}&=x^2\frac{d}{dx}\left( \frac{dy}{dx}\right)=x^2\frac{d}{dx}\left( \frac{1}{x}\frac{dy}{dt}\right)=x^2\left[ -\frac{1}{x^2}\frac{dy}{dt}+\frac{1}{x}\frac{d}{dx}\left( \frac{dy}{dt}\right) \right]=\nonumber\\\label{b21}
&=x^2\left[ -\frac{1}{x^2}\frac{dy}{dt}+\frac{1}{x}\frac{d}{dt}\left( \frac{dy}{dx}\right) \right]=
x^2\left[ -\frac{1}{x^2}\frac{dy}{dt}+\frac{1}{x^2} \frac{d^2y}{dt^2} \right]=-\frac{dy}{dt}+\frac{d^2y}{dt^2}
\end{align}
Αντικαθιστούμε αυτές τις δύο σχέσεις στην ομογενή εξίσωση οπότε αυτή παίρνει την παρακάτω μορφή :
\[ -\frac{dy}{dt}+\frac{d^2y}{dt^2}-2\frac{dy}{dt}+2y=0\Rightarrow \frac{d^2y}{dt^2}-3\frac{dy}{dt}+2y=0\ ,\ t\in\mathbb{R} \]
Η αρχική εξίσωση μετατράπηκε σε μια ομογενή γραμμική διαφορική εξίσωση 2\tss{ου} βαθμού με σταθερούς συντελεστές $ a_2=1, a_1=-3 $ και $ a_0=2 $. Το χαρακτηριστικό πολυώνυμο αυτής της εξίσωσης είναι :
\[ P(\lambda)=\lambda^2-3\lambda+2 \]
Οι ρίζες αυτού είναι οι $ \lambda=1 $ και $ \lambda=2 $ οι οποίες μας δίνουν αντίστοιχα τις συναρτήσεις :
\[ y_1(t)=e^t\ \ \textrm{ και }\ \ y_2(t)=e^{2t}\ \ ,\ \ t\in\mathbb{R} \]
Οι παραπάνω συναρτήσεις ορίζουν το βασικό σύνολο λύσεων $ \left\lbrace e^t,e^{2t} \right\rbrace $ από το οποίο θα δίνονται όλες οι λύσεις της ομογενούς εξίσωσης με τη βοήθεια του τύπου : $ \tilde{y}(t)=c_1e^t+c_2e^{2t}\ ,\ t\in\mathbb{R} $ όπου $ c_1,c_2 $ είναι αυθαίρετες σταθερές. Ισοδύναμα αυτή γράφεται 
\[ \tilde{y}(x)=c_1x+c_2x^2\ ,\ x>0 \]
Θα πρέπει επιπλέον να βρεθεί μια μερική λύση $ y_\mu(x) $ της μη ομογενούς εξίσωσης. Για κάθε $ x>0 $ θα έχουμε :
\begin{gather*}
W(y_1,y_2)(x)=
\begin{vmatrix}
x & x^2\\ 1 & 2x
\end{vmatrix}=2x^2-x^2=x^2\neq0 \ \textrm{ και}\\
W_1(y_1,y_2)(x)=
\begin{vmatrix}
0 & x^2\\ 1 & 2x
\end{vmatrix}=-x^2\ \ ,\ \ W_2(y_1,y_2)(x)=
\begin{vmatrix}
x & 0\\ 1 & 1
\end{vmatrix}=x
\end{gather*}
Η μερική λύση της εξίσωσης θα δίνεται από τον τύπο
\begin{align*}
y_\mu(x)&=y_1(x)\int\frac{W_1(y_1,y_2)(x)}{W(y_1,y_2)(x)}\cdot\frac{b(x)}{a_2(x)}dx+y_2(x)\int\frac{W_2(y_1,y_2)(x)}{W(y_1,y_2)(x)}\cdot\frac{b(x)}{a_2(x)}dx=\\
&=x\int\frac{-x^2}{x^2}\cdot\frac{x^3e^x}{x^2}dx+x^2\int\frac{x}{x^2}\cdot\frac{x^3e^x}{x^2}dx=\\
&=-x\left(xe^x-e^x \right)+x^2e^x=-x^2e^x+xe^x+x^2e^x=xe^x
\end{align*}
Τελικά όλες οι λύσεις της αρχικής διαφορικής εξίσωσης δίνονται από τον τύπο
\[ y(x)=c_1x+c_2x^2+xe^x\ ,\ x>0 \]
όπου $ c_1,c_2 $ είναι αυθαίρετες σταθερές.\epask
\begin{Askhshs}[B]
\bmath{Να επιλυθεί το πρόβλημα αρχικών τιμών
\[ y''-2y'+y=\frac{1}{x}\cdot e^x\ \ ,\ \ x>0\ \ ,\ y(1)=0, y'(1)=1 \]}
\end{Askhshs}\mbox{}\\
\lysh
Το χαρακτηριστικό πολυώνυμο της αντίστοιχης ομογενούς εξίσωσης $ y''-2y'+y=0 $ είναι το $ P(\lambda)=\lambda^2-2\lambda+1=(\lambda-1)^2 $ το οποίο έχει διπλή ρίζα την $ \lambda=1 $. Συνεπώς ένα βασικό σύνολο λύσεων της ομογενούς θα είναι έχει στοιχεία τις συναρτήσεις  \[ y_1(x)=e^x \ \ \textrm{και}\ \  y_2(x)=xe^x\ \ ,\ \ χ>0 \] 
Έτσι οι λύσεις της ομογενούς θα δίνονται από τον τύπο:
\[ \tilde{y}(x)=c_1e^x+c_2xe^x\ ,\ x>0 \] με $ c_1,c_2 $ είναι αυθαίρετες σταθερές. Για την εύρεση μιας μερικής λύσης της μη ομογενούς εξίσωσης θα έχουμε:
\begin{gather*}
W(y_1,y_2)(x)=
\begin{vmatrix}
e^x & xe^x\\ e^x & e^x(x+1)
\end{vmatrix}=e^{2x}\begin{vmatrix}
1 & x\\ 1 & x+1
\end{vmatrix}=e^{2x}\neq0 \ \textrm{ και}\\
W_1(y_1,y_2)(x)=
\begin{vmatrix}
0 & xe^x\\ 1 & e^x(x+1)
\end{vmatrix}=-xe^x\ \ ,\ \ W_2(y_1,y_2)(x)=
\begin{vmatrix}
e^x & 0\\ e^x & 1
\end{vmatrix}=e^x
\end{gather*}
Συνεπώς για κάθε $ x>0 $ η μερική λύση θα δίνεται από τον τύπο :
\[ y_\mu(x)=y_1(x)\int\frac{W_1(y_1,y_2)(x)}{W(y_1,y_2)(x)}\cdot\frac{b(x)}{a_2(x)}dx+y_2(x)\int\frac{W_2(y_1,y_2)(x)}{W(y_1,y_2)(x)}\cdot\frac{b(x)}{a_2(x)}dx \] όπου $ b(x)=\frac{1}{x}e^x $. Άρα θα είναι :
\begin{align*}
y_\mu(x)&=e^x\int\frac{-xe^x}{e^{2x}}\cdot\frac{1}{x}e^xdx+xe^x\int\frac{e^x}{e^{2x}}\cdot\frac{1}{x}e^xdx=\\&=-xe^x+xe^x\log{x}=xe^x(\log{x}+1)\ ,\ x>0
\end{align*}
Προκύπτει λοιπόν ο τύπος από τον οποίο δίνονται όλες οι λύσεις της αρχικής διαφορικής εξίσωσης :
\[ y(x)=c_1e^x+c_2xe^x+xe^x(\log{x}+1)\ ,\ x>0 \] όπου $ c_1,c_2 $ είναι αυθαίρετες σταθερές. Για τη λύση τώρα που πληροί τις αρχικές συνθήκες του προβλήματος έχουμε
\[ y'(x)=c_1e^x+c_2e^x+c_2xe^x+e^x(\log{x}+1)+xe^x(\log{x}+1)+e^x\ ,\ x>0 \] Επομένως θα προκύψει το σύστημα
\[ \ccases{c_1e+c_2e+e(-1)=0\\c_1e+c_2e+c_2e-e-e+e=1}\Rightarrow\ccases{c_1e+c_2e=e\\c_1e+2ec_2=1+e}\Rightarrow c_1=1-\frac{1}{e}\ \textrm{και}\ c_2=\frac{1}{e} \]
Συνεπώς η λύση του προβλήματος αρχικών τιμών θα είναι 
\[ y(x)=\left(1-\frac{1}{e}\right) e^x+\frac{1}{e}xe^x+xe^x(\log{x}+1)\ ,\ x>0 \]\epask
\begin{Askhshs}[B]
\textbf{Να επιλυθεί η διαφορική εξίσωση}
{\boldmath\[ (1-x)y''+xy'-y=0\ ,\ x>1 \]}
\textbf{με το δεδομένο ότι έχει κοινή λύση με τη διαφορική εξίσωση}
{\boldmath\[ 2x(2x-1)y''-\left(4x^2+1\right)y'+(2x+1)y=0\ ,\ x>1  \]}
\end{Askhshs}\mbox{}\\
\lysh
Έστω $ y_1 $ η κοινή λύση των δύο διαφορικών εξισώσεων. Η λύση αυτή επαληθεύει την πρώτη εξίσωση οπότε αντικαθιστώντας θα έχουμε :
\begin{align}
(1-x)y_1''+xy_1'-y_1=0\Rightarrow (1-x)y_1''=y_1-xy_1'\Rightarrow y_1''=\frac{y_1-xy_1'}{1-x}\label{b4}
\end{align}
Αντικαθιστούμε τη σχέση \eqref{b4} και τη μερική λύση $ y_1 $ στη δεύτερη διαφορική εξίσωση και παίρνουμε :
\begin{gather*}
2x(2x-1)\frac{y_1-xy_1'}{1-x}-\left(4x^2+1\right)y_1'+(2x+1)y_1=0\Rightarrow\\ 2x(2x-1)\left( y_1-xy_1'\right) -\left(4x^2+1\right)(1-x)y_1'+(2x+1)(1-x)y_1=0\Rightarrow\\
(4x^2-2x)\left( y_1-xy_1'\right) -\left(4x^2-4x^3-x+1\right)y_1'+(-2x^2+x+1)y_1=0\Rightarrow \\
\left(-2x^2+x-1\right)(y_1'-y_1)=0\Rightarrow y_1'-y_1=0 \Rightarrow y_1'=y_1\Rightarrow y_1=ce^x
\end{gather*}
Χρησιμοποιούμε τη λύση αυτή για να κάνουμε υποβιβασμό της τάξης της πρώτης διαφορικής εξίσωσης θέτοντας $ z=\frac{y}{y_1}=\frac{y}{e^x}\Rightarrow y'=(z'+z)e^x\Rightarrow y''=\left(z''+2z'+z\right)e^x  $. Έτσι η εξίσωση θα πάρει τη μορφή :
\[ (1-x)\left(z''+2z'+z\right)e^x+x(z'+z)e^x-ze^x=0 \]
Εκτελώντας τις κατάλληλες πράξεις θα έχουμε :
\begin{gather*}
(1-x)\left(z''+2z'+z\right)e^x+x(z'+z)e^x-ze^x=0\Rightarrow\\
z''+2z'+z-xz''-2xz'-xz+xz'+xz-z=0\Rightarrow\\
(1-x)z''+(2-x)z'=0
\end{gather*}
Η τελευταία μετατρέπεται σε μια γραμμική διαφορική εξίσωση 1\tss{ης} τάξης θέτοντας όπου $ z'=u\Rightarrow z''=u' $. Έτσι θα έχουμε :
\[ (1-x)u'+(2-x)u=0\Rightarrow u'+\frac{2-x}{1-x}u=0 \]
της οποίας η γενική λύση θα δίνεται από τον τύπο
\[ u(x)=ce^{-\dintt{\frac{2-x}{1-x}dx}}=ce^{\log{(1-x)}-x}=\frac{c(1-x)}{e^x} \]
Αντικαθιστώντας ξανά χρησιμοποιώντας έναν έναν τους προηγούμενους μετασχηματισμούς καταλήγουμε στη γενική λύση της αρχικής διαφορικής εξίσωσης :
\begin{align*}
u(x)=\frac{c(1-x)}{e^x}&\xRightarrow{z'=u}z'(x)=\frac{c(1-x)}{e^x}=c\frac{e^x-xe^x}{e^{2x}}\Rightarrow\\&\Rightarrow z(x)=c\frac{x}{e^x}+c'\Rightarrow\\&\Rightarrow \frac{y}{e^x}=c\frac{x}{e^x}+c'\Rightarrow y(x)=cx+c'e^x
\end{align*}
όπου $ c,c' $ είναι αυθαίρετες σταθερές.\epask
\begin{Askhshs}[B]
\textbf{Να επιλυθεί η εξίσωση}
{\boldmath\[ y''-3y'+2y=\frac{1}{1+e^{-x}}\ ,\ x\in\mathbb{R} \]}
\end{Askhshs}\mbox{}\\
\lysh
Η αρχική εξίσωση αποτελεί μια γραμμική μη ομογενή διαφορική εξίσωση 2\tss{ης} τάξης με σταθερούς συντελεστές. Η αντίστοιχη ομογενής εξίσωση είναι
\[ y''-3y'+2y=0 \]
η οποία έχει χαρακτηριστικό πολυώνυμο το $ p(\lambda)=\lambda^2-3\lambda+2 $. Το πολυώνυμο αυτό έχει ρίζες τις $ \lambda_1=1 $ και $ \lambda_2=2 $ άρα ένα βασικό σύνολο λύσεων της ομογενούς θα είναι το $ \left\lbrace y_1(x)=e^x\ ,\ y_2(x)=e^{2x} \right\rbrace\ ,\ x\in\mathbb{R} $. Έτσι οι λύσεις της θα δίνονται από τον τύπο:
\[ y(x)=c_1e^x+c_2e^{2x} \]
όπου $ c_1,c_2 $ είναι αυθαίρετες σταθερές. Θα αναζητήσουμε στη συνέχεια μια μερική λύση της μη ομογενούς εξίσωσης. Έχουμε λοιπόν:
\begin{gather*}
W(y_1,y_2)(x)=
\begin{vmatrix}
e^x & e^{2x}\\ e^x & 2e^{2x}
\end{vmatrix}=2e^{3x}-e^{3x}=e^{3x}\neq0 \ \textrm{ και}\\
W_1(y_1,y_2)(x)=
\begin{vmatrix}
0 & e^{2x}\\ 1 & 2e^{2x}
\end{vmatrix}=-e^{2x}\ \ ,\ \ W_2(y_1,y_2)(x)=
\begin{vmatrix}
e^x & 0\\ e^x & 1
\end{vmatrix}=e^x
\end{gather*}
Η μερική λύση για κάθε $ x\in\mathbb{R} $ θα δίνεται από τον τύπο :
\[ y_\mu(x)=y_1(x)\int\frac{W_1(y_1,y_2)(x)}{W(y_1,y_2)(x)}\cdot\frac{b(x)}{a_2(x)}\d x+y_2(x)\int\frac{W_2(y_1,y_2)(x)}{W(y_1,y_2)(x)}\cdot\frac{b(x)}{a_2(x)}\d x \] με $ b(x)=\frac{1}{x}e^x $. Άρα θα έχουμε :\begin{align*}
y_\mu(x)&=e^x\int\frac{-e^{2x}}{e^{3x}}\cdot\frac{1}{1+e^{-x}}dx+e^{2x}\int\frac{e^x}{e^{3x}}\cdot\frac{1}{1+e^{-x}}dx=\\&=-e^{x}\int\dfrac{1}{e^x\left(1+e^{-x}\right)}dx+e^{2x}\int\dfrac{1}{e^{2x}\left(1+e^{-x}\right)}dx=-e^{x}\int\dfrac{e^{-x}}{1+e^{-x}}dx+e^{2x}\int\dfrac{e^{-2x}}{1+e^{-x}}dx=\\&=e^x\log\left(1+e^{-x}\right)-e^x+e^{2x}\log\left(1+e^{-x}\right)=\left(1+e^x\right)e^x\log\left(1+e^{-x}\right)-e^x\ ,\ x\in\mathbb{R}
\end{align*}
Τελικά όλες οι λύσεις της αρχικής διαφορικής εξίσωσης θα δίνονται από τον τύπο :
\[ y(x)=c_1e^x+c_2e^{2x}+\left(1+e^x\right)e^x\log\left(1+e^{-x}\right)-e^x\ ,\ x\in\mathbb{R} \]
όπου $ c_1,c_2 $ είναι αυθαίρετες σταθερές.\epask
\begin{Askhshs}[B]
\textbf{Να επιλυθεί η διαφορική εξίσωση}
{\boldmath\[ \dfrac{d^2y}{dx^2}-\frac{2x}{1+x^2}\frac{dy}{dx}+\frac{2}{1+x^2}y=1+x^2 \]}
\end{Askhshs}\mbox{}\\
\lysh
Για κάθε $ x\neq0 $ η διαφορική εξίσωση μπορεί να πάρει τη μορφή
\begin{gather*}
\left(1+x^2\right)\dfrac{d^2y}{dx^2}-2x\frac{dy}{dx}+2y=\left(1+x^2\right)^2\Rightarrow\\ \left(1+x^2\right)y''-2xy'+2y=\left(1+x^2\right)^2\ ,\ x\in\mathbb{R}-\{0\}
\end{gather*}
Η αντίστοιχη ομογενής εξίσωση θα είναι $ \left(1+x^2\right)y''-2xy'+2y=0 $ της οποίας παρατηρούμε ότι μια λύση είναι η $ y_1(x)=x\ ,\ x\in\mathbb{R}-\{0\} $. Χρησιμοποιώντας το μετασχηματισμό $ y=zx\ ,\ u=z' $ θα μετατρέψουμε τη μη ομογενή εξίσωση σε γραμμική διαφορική εξίσωση 1\tss{ης} τάξης. Θα έχουμε $ y=zx\Rightarrow y'=z'x+z\Rightarrow y''=z''x+2z' $ οπότε με αντικατάσταση προκύπτει :
\begin{gather*}
\left(1+x^2\right) \left(z''x+2z'\right)-2x\left( z'x+z\right) +2zx=\left(1+x^2\right)^2\Rightarrow\\
x\left(1+x^2\right)z''+2z'=\left(1+x^2\right)^2\Rightarrow u'+\frac{2}{x\left(1+x^2\right)}u=\frac{1+x^2}{x}
\end{gather*}
Η λύση της τελευταίας εξίσωσης δίνεται από τον τύπο :
\begin{align*}
u(x)&=e^{-\int{\frac{2}{x\left(1+x^2\right)}dx}}\left[c+\int{\frac{1+x^2}{x}e^{\int{\frac{2}{x\left(1+x^2\right)}dx}}} \right]=\frac{x^2+1}{x^2}\left(c+\int{\frac{1+x^2}{x}\cdot\frac{x^2}{x^2+1}dx}\right)=\\
&=\frac{x^2+1}{x^2}\left(c+\int x\ dx\right)=\frac{x^2+1}{x^2}\left(c+\frac{x^2}{2}\right)=\dfrac{c+cx^2}{x^2}+\frac{x^2+1}{2}
\end{align*}
Έτσι με διαδοχικές αντικαταστάσεις θα έχουμε :
\begin{gather*}
z'(x)=\dfrac{c+cx^2}{x^2}+\frac{x^2+1}{2}\Rightarrow z(x)=cx-\frac{c}{x}+\frac{x^3}{6}+\frac{x}{2}+c'\Rightarrow\\
y(x)=cx^2-c+\frac{x^4}{6}+\frac{x^2}{2}+c'x\ ,\ x\in\mathbb{R}-\{0\}
\end{gather*}
όπου $ c,c' $ είναι αυθαίρετες σταθερές.\epask
\begin{Askhshs}[B]
\textbf{Να επιλυθεί η διαφορική εξίσωση}
{\boldmath\[ x^3y'''+x^2y''-2xy'+2y=x^3\sin{x}\ ,\ x>0 \]}
\end{Askhshs}\mbox{}\\
\lysh
Η αντίστοιχη ομογενής εξίσωση της αρχικής είναι η διαφορική εξίσωση Euler :
\[ x^3y'''+x^2y''-2xy'+2y=0\ ,\ x>0 \]
Χρησιμοποιούμε λοιπόν το μετασχηματισμό $ t=\log{x} $ ο οποίος μας δίνει :
\begin{align}
x\frac{dy}{dx}&=x\frac{dy}{dt}\cdot\frac{dt}{dx}=x\frac{dy}{dt}\cdot\frac{1}{x}=\frac{dy}{dt}\label{b7}\\
x^2\frac{d^2y}{dx^2}&=x^2\frac{d}{dx}\left( \frac{dy}{dx}\right)=x^2\frac{d}{dx}\left( \frac{1}{x}\frac{dy}{dt}\right)=x^2\left[ -\frac{1}{x^2}\frac{dy}{dt}+\frac{1}{x}\frac{d}{dx}\left( \frac{dy}{dt}\right) \right]=\nonumber\\\label{b71}
&=x^2\left[ -\frac{1}{x^2}\frac{dy}{dt}+\frac{1}{x}\frac{d}{dt}\left( \frac{dy}{dx}\right) \right]=
x^2\left[ -\frac{1}{x^2}\frac{dy}{dt}+\frac{1}{x^2} \frac{d^2y}{dt^2} \right]=-\frac{dy}{dt}+\frac{d^2y}{dt^2}\ ,\ \textrm{ και }\\
x^3\frac{d^3y}{dx^3}&=x^3\frac{d}{dx}\left(\frac{d^2}{dx^2}\right)=x^3\frac{d}{dx}\left(-\frac{1}{x^2}\frac{dy}{dt}+\frac{1}{x^2} \frac{d^2y}{dt^2}\right)=\nonumber\\
&=x^3\left[\frac{2}{x^3}\frac{dy}{dt}-\frac{1}{x^2}\frac{d}{dx}\left(\frac{dy}{dt}\right)-\frac{2}{x^3}\frac{d^2y}{dt^2}+\frac{1}{x^2}\frac{d}{dt}\left(\frac{d^2y}{dx^2}\right) \right]=\nonumber\\
&=x^3\left[\frac{2}{x^3}\frac{dy}{dt}-\frac{1}{x^2}\frac{d}{dt}\left(\frac{dy}{dx}\right)-\frac{2}{x^3}\frac{d^2y}{dt^2}+\frac{1}{x^2}\frac{d^2}{dt^2}\left(\frac{dy}{dx}\right) \right]=\nonumber\\
&=x^3\left[\frac{2}{x^3}\frac{dy}{dt}-\frac{1}{x^3}\frac{d^2y}{dx^2}-\frac{2}{x^3}\frac{d^2y}{dt^2}+\frac{1}{x^3}\frac{d^3y}{dt^3}\right]=\frac{d^3y}{dt^3}-3\frac{d^2y}{dt^2}+2\frac{dy}{dt}\label{b72}
\end{align}
Αντικαθιστώντας τις σχέσεις \eqref{b7}, \eqref{b71}, \eqref{b72} στη διαφορική εξίσωση Euler αυτή θα πάρει τη μορφή :
\begin{gather}
\frac{d^3y}{dt^3}-3\frac{d^2y}{dt^2}+2\frac{dy}{dt}-\frac{dy}{dt}+\frac{d^2y}{dt^2}-2\frac{dy}{dt}+2y=0\Rightarrow\nonumber\\
\frac{d^3y}{dt^3}-2\frac{d^2y}{dt^2}-\frac{dy}{dt}+2y=0\label{a7:1}
\end{gather}
Καταλλήξαμε έτσι σε μια ομογενή γραμμική διαφορική εξίσωση 2\tss{ης} τάξης με σταθερούς συντελεστές. Το χαρακτηριστικό πολυώνυμο αυτής είναι το $ P(\lambda)=\lambda^3-2\lambda^2-\lambda+2=0 $ με ρίζες τις $ \lambda_1=1,\lambda_2=-1 $ και $ \lambda_3=2 $. Συνεπώς οι συναρτήσεις  \[ y_1(t)=e^t\ ,\ t\in\mathbb{R}\ \ ,\ \  y_2(t)=e^{-t}\ ,\ t\in\mathbb{R} \ \ \textrm{και}\ \  y_3(t)=e^{2t}\ ,\ t\in\mathbb{R} \] 
αποτελούν ένα βασικό σύνολο λύσεων της ομογενούς εξίσωσης \eqref{a7:1}. Έτσι μετά από αντικατάσταση οι συναρτήσεις $ y_1(x)=x\ ,\ x>0 $, $ y_2(x)=\frac{1}{x}\ ,\ x>0 $ και $ y_3(x)=x^2\ ,\ x>0 $ αποτελούν βασικό σύνολο λύσεων της αρχικής ομογενούς εξίσωσης. Θα έχουμε λοιπόν τον τύπο από τον οποίο προκύπτουν όλες οι λύσεις της:
\[ \tilde{y}(x)=c_1x+\frac{c_2}{x}+c_3x^2\ ,\ x>0 \]
όπου $ c_1,c_2,c_3 $ είναι αυθαίρετες σταθερές. Για τη μερική λύση της μη ομογενούς εξίσωσης θα χρησιμοποιήσουμε τη μέθοδο μεταβολής των σταθερών. Σχηματίζουμε το σύστημα :
\[ \ccases{\nu_1'(x)y_1(x)+\nu_2'(x)y_2(x)+\nu_3'(x)y_3(x)=0\\
\nu_1'(x)y_1'(x)+\nu_2'(x)y_2'(x)+\nu_3'(x)y_3'(x)=0\\
\nu_1'(x)y_1''(x)+\nu_2'(x)y_2''(x)+\nu_3'(x)y_3''(x)=\frac{b(x)}{a_3(x)}} \Rightarrow \ccases{\nu_1'(x)x+\nu_2'(x)\frac{1}{x}+\nu_3'(x)x^2=0\\
\nu_1'(x)-\nu_2'(x)\frac{1}{x^2}+\nu_3'(x)2x=0\\
\nu_2'(x)\frac{2}{x^3}+2\nu_3'(x)=\sin{x}} \]
Λύνοντας το παραπάνω σύστημα με τη μέθοδο των οριζουσών θα έχουμε :
\begin{gather*}
D=\begin{vmatrix}
x & \frac{1}{x} & x^2\\
1 & -\frac{1}{x^2} & 2x\\
0 & \frac{2}{x^3} & 2\\
\end{vmatrix}=\frac{6}{x}\neq0\quad,\quad
D_{\nu_1'(x)}=\begin{vmatrix}
0 & \frac{1}{x} & x^2\\
0 & -\frac{1}{x^2} & 2x\\
\sin{x} & \frac{2}{x^3} & 2\\
\end{vmatrix}=3\sin{x}\\
D_{v_2'(x)}=\begin{vmatrix}
x & 0 & x^2\\
1 & 0 & 2x\\
0 & \sin{x} & 2\\
\end{vmatrix}=-x^2\sin{x}\quad,\quad
D_{\nu_3'(x)}=\begin{vmatrix}
x & \frac{1}{x} & 0\\
1 & -\frac{1}{x^2} & 0\\
0 & \frac{2}{x^3} & \sin{x}\\
\end{vmatrix}=-\frac{2}{x}\sin{x}
\end{gather*}
Έτσι θα έχουμε $ \nu_1'(x)=-\frac{x\sin{x}}{2} $, $ \nu_2'(x)=\frac{x^3\sin{x}}{6} $ και $ \nu_3'(x)=\frac{\sin{x}}{3} $ άρα οι αρχικές αυτών θα είναι $ \nu_1(x)=\frac{1}{2}(x\cos{x}-\sin{x}) $, $ \nu_2(x)=-\frac{1}{6}x^3\cos{x}+\frac{1}{2}x^2\sin{x}+x\cos{x}-\sin{x} $ και $ \nu_3(x)=-\frac{1}{3}\cos{x} $ με $ x>0 $. Μια μερική λύση λοιπόν θα δίνεται από τον τύπο 
\begin{align*}
y_{\mu}(x)&=\nu_1(x)y_1(x)+\nu_2(x)y_2(x)+\nu_3(x)y_3(x)=\\&=\frac{x}{2}(x\cos{x}-\sin{x})-\frac{1}{6}x^2\cos{x}+\frac{1}{2}x\sin{x}+\cos{x}-\frac{\sin{x}}{x}-\frac{x^2}{3}\cos{x}=\\
&=\cos{x}-\frac{\sin{x}}{x}\ ,\ x>0
\end{align*}
Επομένως η γενική λύση της αρχικής διαφορικής εξίσωσης θα είναι η 
\[ y(x)=c_1x+\frac{c_2}{x}+c_3x^2+\cos{x}-\frac{\sin{x}}{x}\ ,\ x>0 \]
όπου $ c_1,c_2,c_3 $ είναι αυθαίρετες σταθερές.\epask
\begin{Askhshs}[B]
\textbf{Να επιλυθεί η διαφορική εξίσωση}
{\boldmath\[ y''+xy'+y=0\ ,\ x\in\mathbb{R} \]}
\textbf{αφού διαπιστωθεί ότι η {\boldmath$ y_1(x)=e^{-x^2/2}\ ,\ x\in\mathbb{R} $} είναι μια λύση της.}
\end{Askhshs}\mbox{}\\
\lysh
Για να είναι η συνάρτηση $ y_1(x)=e^{-x^2/2}\ ,\ x\in\mathbb{R} $ λύση της διαφορικής εξίσωσης θα πρέπει να την επαληθεύει δηλαδή να ισχύει $ y_1''+xy_1'+y_1=0\ ,\ \forall x\in\mathbb{R} $. Έτσι έχουμε:
\[ y_1(x)=e^{-x^2/2}\Rightarrow y_1'(x)=-xe^{-x^2/2}\Rightarrow y_1''(x)=-e^{-x^2/2}+x^2e^{-x^2/2}\ ,\ x\in\mathbb{R} \]
Αντικαθιστώντας τις παραγώγους αυτές στην αρχική διαφορική εξίσωση θα πάρουμε
\[ y''+xy'+y=0\Rightarrow -e^{-x^2/2}+x^2e^{-x^2/2}-x^2e^{-x^2/2}+e^{-x^2/2}=0\Rightarrow 0=0  \]
οπότε την επαληθεύει. Για την επίλυση της διαφορικής εξίσωσης έχουμε $ \forall x\in\mathbb{R} $ αφού $ y_1(x)=e^{-x^2/2}\neq 0 $:
\begin{align*}
y_2(x)&=y_1(x)\int_{0}^{x}{\frac{1}{y_1^2(t)}\cdot e^{-\dintt_{\!\!0}^{t}{\frac{a_1(s)}{a_2(s)}\ \mathrm{d}s}}\ \mathrm{d}t}=e^{-x^2/2}\int_{0}^{x}{\frac{1}{e^{-t^2}}\cdot e^{-\int_{0}^{t}{s\ \mathrm{d}s}}\ \mathrm{d}t}
=\\
&=e^{-x^2/2}\int_{0}^{x}{\frac{1}{e^{-t^2}}\cdot e^{-\frac{t^2}{2}}\ \mathrm{d}t}=
e^{-x^2/2}\int_{0}^{x}{e^{t^2/2}\ \mathrm{d}t}
\end{align*}
Επομένως οι συναρτήσεις $ y_1(x)=e^{-x^2/2} $ και $ y_2(x)=e^{-x^2/2}\int_{0}^{x}{e^{t^2/2}\ \mathrm{d}t} $ με $ x\in\mathbb{R} $ αποτελούν ένα βασικό σύνολο λύσεων της διαφορικής εξίσωσης. Έτσι όλες οι λύσεις της θα δίνονται από τον τύπο:
\[ y(x)=c_1e^{-x^2/2}+c_2e^{-x^2/2}\int_{0}^{x}{e^{t^2/2}\ \mathrm{d}t}\ ,\ x\in\mathbb{R} \]
όπου $ c_1,c_2 $ είναι αυθαίρετες σταθερές.\epask
\begin{Askhshs}[B]
\textbf{Να επιλυθεί το πρόβλημα αρχικών τιμών}
{\boldmath\[ (x+1)^2y''+(x+1)y'+y=(x+1)\log^2{(x+1)}\ ,\ x>-1,\ y(0)=0,\ y'(0)=1 \]}
\end{Askhshs}\mbox{}\\
\lysh
Η αντίστοιχη ομογενής εξίσωση θα είναι η $ (x+1)^2y''+(x+1)y'+y=0 $ η οποία θέτοντας $ t=x+1 $ μετατρέπεται σε μια διαφορική εξίσωση Euler. Θα ισχύει $ \frac{dy}{dx}=\frac{dy}{dt} $ και $ \frac{d^2y}{dx^2}=\frac{d^2y}{dt^2} $ και έτσι παίρνουμε \[ t^2y''+ty'+y=0\ ,\ t>0 \]
Η τελευταία με τη σειρά της μετατρέπεται σε μια γραμμική διαφορική εξίσωση 2\tss{ης} τάξης με σταθερούς συντελεστές θέτοντας $ u=\log{t} $. Προκύπτει ότι
\begin{align}
t\frac{dy}{dt}&=t\frac{dy}{du}\cdot\frac{du}{dt}=t\frac{dy}{du}\cdot\frac{1}{t}=\frac{dy}{du}\quad\textrm{και}\label{b1}\\
t^2\frac{d^2y}{dt^2}&=t^2\frac{d}{dt}\left( \frac{dy}{dt}\right)=t^2\frac{d}{dt}\left( \frac{1}{t}\frac{dy}{du}\right)=t^2\left[ -\frac{1}{t^2}\frac{dy}{du}+\frac{1}{t}\frac{d}{dt}\left( \frac{dy}{du}\right) \right]=\nonumber\\\label{b11}
&=t^2\left[ -\frac{1}{t^2}\frac{dy}{du}+\frac{1}{t}\frac{d}{du}\left( \frac{dy}{dt}\right) \right]=
t^2\left[ -\frac{1}{t^2}\frac{dy}{du}+\frac{1}{t^2} \frac{d^2y}{du^2} \right]=-\frac{dy}{du}+\frac{d^2y}{du^2}
\end{align}
οπότε και καταλλήγουμε στην εξίσωση
\[ -\frac{dy}{du}+\frac{d^2y}{du^2}+\frac{dy}{du}+y=0\Rightarrow \frac{d^2y}{du^2}+y=0\ ,\ u\in\mathbb{R} \]
Το χαρακτηριστικό πολυώνυμο της εξίσωσης αυτής είναι το $ P(\lambda)=\lambda^2+1 $ με ρίζες τις $ \lambda_1=i $ και $ \lambda_2=-i $. Αυτές μας δίνουν το βασικό σύνολο λύσεων $ \{y_1(u)=\cos{u},\ y_2(u)=\sin{u} \} $ και έτσι η γενική λύση της ομογενούς θα είναι η 
\[ \tilde{y}(u)=c_1\cos{u}+c_2\sin{u}\ ,\ u\in\mathbb{R} \]
όπου $ c_1,c_2 $ είναι αυθαίρετες σταθερές. Η μή γραμμική εξίσωση απ' την άλλη, μετά από το μετασχηματισμό θα πάρει τη μορφή $ \frac{d^2y}{du^2}+y=u^2e^u $ και για την εύρεση μια μερικής λύσης αυτής θα χρησιμοποιήσουμε τη μέθοδο των αγνώστων σταθερών. Θέτοντας $ y(u)=z(u)e^u $ παίρνουμε $ y'(u)=\left( z'+z\right)e^u $ και $ y''(u)=\left(z''+2z'+z\right)e^u $ και έτσι η εξίσωση γίνεται \[ z''+2z'+2z=u^2 \]
Αναζητούμε μια λύση της μορφής $ z(u)=au^2+\beta u+\gamma $ η οποία μας δίνει $ z'(u)=2au+\beta $ και $ z''(u)=2a $. Έτσι θα έχουμε
\[ 2a+4au+2\beta +2au^2+2\beta u+2\gamma=u^2 \]
Εξισώνοντας τους συντελεστές των δύο πολυωνύμων στην παραπάνω σχέση παίρνουμε το ακόλουθο $ 3\times3 $ σύστημα
\[ \systeme[a\beta\gamma]{2a=1,4a+2\beta=0,2a+2\beta+2\gamma=0} \]
το οποίο μας δίνει τη λύση $ (a,\beta,\gamma)=\left(\frac{1}{2},-1,\frac{1}{2}\right) $. Καταλήγουμε λοιπόν στη μερική λύση $ z(u)=\frac{u^2}{2}-u+\frac{1}{2}\Rightarrow y(u)=\left(\frac{u^2}{2}-u+\frac{1}{2}\right)e^u\ ,\ u\in\mathbb{R} $. Όλες οι λύσεις της αρχικής διαφορικής εξίσωσης θα δίνονται από τον τύπο:
\[ y(u)=c_1\cos{u}+c_2\sin{u}+\left(\frac{u^2}{2}-u+\frac{1}{2}\right)e^u\ ,\ u\in\mathbb{R} \] ο οποίος ισοδύναμα γράφεται
\[ y(x)=c_1\cos{(\log{(x+1)})}+c_2\sin{(\log{(x+1)})}+\left[\frac{\log^2{(x+1)}}{2}-\log{(x+1)}+\frac{1}{2}\right](x+1)\ ,\ x>-1 \]
Για τη λύση που πληροί το πρόβλημα αρχικών τιμών θα έχουμε:
\begin{align*}
y'(x)=-c_1\frac{\sin{(\log{(x+1)})}}{x+1}&+c_2\frac{\cos{(\log{(x+1)})}}{x+1}+\left[\frac{\log{(x+1)}}{x+1}-\frac{1}{x+1}\right](x+1)+\\&+\frac{\log^2{(x+1)}}{2}-\log{(x+1)}+\frac{1}{2}=\\
&=\frac{c_2\cos{(\log{(x+1)})}-c_1\sin{(\log{(x+1)})}}{x+1}+\frac{\log^2{(x+1)}}{2}-\frac{1}{2}\ ,\ x>-1
\end{align*}
Έτσι για $ x=0 $ από τις αρχικές συνθήκες θα ισχύει ότι:
\begin{gather*}
y(0)=0\Rightarrow c_1\cos{(\log{(1)})}+c_2\sin{(\log{(1)})}+\frac{\log^2{(1)}}{2}-\log{(1)}+\frac{1}{2}=0\Rightarrow c_1=-\frac{1}{2}\\
y'(0)=1\Rightarrow c_2\cos{(\log{(1)})}+\frac{1}{2}\sin{(\log{(1)})}+\frac{\log^2{(1)}}{2}-\frac{1}{2}=1\Rightarrow c_2=\frac{3}{2}
\end{gather*}
Επομένως η λύση θα είναι η 
\[ y(x)=-\frac{1}{2}\cos{(\log{(x+1)})}+\frac{3}{2}\sin{(\log{(x+1)})}+\left[\frac{\log^2{(x+1)}}{2}-\log{(x+1)}+\frac{1}{2}\right](x+1)\ ,\ x>-1 \]\epask
\begin{Askhshs}[B]
\textbf{Με τη βοήθεια του μετασχηματισμού {\boldmath$ z=xy $}, να επιλυθεί η γραμμική διαφορική εξίσωση}
{\boldmath\[
x(x+1)^2y''+(3x+2)(x+1)y'+y=\log{(x+1)}\ ,\ x>0 \]}\end{Askhshs}\mbox{}\\
\lysh
Για κάθε $ x>0 $ θέτουμε $ z=xy\Rightarrow y=\frac{z}{x} $ το οποίο μας δίνει με παραγώγιση τις $ y'=\frac{z'x-z}{x^2} $ και $ y''=\frac{z''x^2-2z'x+2z}{x^3} $. Αν αντικαταστήσουμε τις παραγώγους αυτές στη διαφορική εξίσωση αυτή θα γίνει:
\begin{gather*}
x(x+1)^2\frac{z''x^2-2z'x+2z}{x^3}+(3x+2)(x+1)
\frac{z'x-z}{x^2}+\frac{z}{x}=\log{(x+1)}\Rightarrow\\
(x+1)^2\left(z''x^2-2z'x+2z\right)+(3x+2)(x+1)
\left(z'x-z\right)+xz=x^2\log{(x+1)}\Rightarrow\\
(x+1)^2z''+(x+1)z'-z=\log{(x+1)}
\end{gather*}
Θέτοντας στην τελευταία όπου $ t=x+1 $ αυτή παίρνει τη μορφή $ t^2z''+tz'-z=\log{t}\ ,\ t>1 $ η οποία έχει αντίστοιχη ομογενή την 
\[ t^2z''+tz'-z=0\ ,\ t>1 \]
που είναι μια διαφορική εξίσωση Euler. Σ' αυτήν χρησιμοποιούμε το μετασχηματισμό $ u=\log{t} $ και παίρνουμε
\[ t\frac{dz}{dt}=\frac{dz}{du}\quad\textrm{και}\quad
t^2\frac{d^2z}{dt^2}=-\frac{dz}{du}+\frac{d^2z}{du^2} \]
Έτσι η εξίσωση Euler μετατρέπεται στη γραμμική ομογενή διαφορική εξίσωση 2\tss{ης} τάξης 
\[ \frac{d^2z}{du^2}-z=0 \]
Το χαρακτηριστικό πολυώνυμο αυτής είναι το $ P(\lambda)=\lambda^2-1 $ με ρίζες τις $ \lambda_1=1 $ και $ \lambda_2=-1 $. Αυτές μας δίνουν το βασικό σύνολο λύσεων $ \left\lbrace z_1(u)=e^u\ ,\ z_2(u)=e^{-u} \right\rbrace\ ,\ u\in\mathbb{R} $ που ισοδύναμα μετά από όλες τις αντικαταστάσεις γράφεται $ \left\lbrace y_1(x)=\frac{x+1}{x}\ ,\ y_2(x)=\frac{1}{x(x+1)}\right\rbrace \ ,\ x>0$. Η μη ομογενής εξίσωση ύστερα από το μετασχηματισμό θα γίνει
\[ \frac{d^2z}{du^2}-z=u \]
Αναζητούμε λοιπόν μια λύση $ z_\mu $ της μορφής $ z_\mu(u)=au+\beta $ για την οποία θα έχουμε $ z_\mu'(u)=a $ και $ z_\mu''(u)=0 $. Έτσι με αντικατάσταση παίρνουμε:
\[ -au-\beta=u\Rightarrow a=-1\ \textrm{ και }\ \beta=0 \]
άρα η μερική λύση θα είναι $ z_\mu(u)=-u\Rightarrow y(x)=-\frac{1}{x}\log{(x+1)}\ ,\ x>0 $. Οι λύσεις της αρχικής διαφορικής εξίσωσης θα δίνονται από τον τύπο
\[ y(x)=c_1\frac{x+1}{x}+c_2\frac{1}{x(x+1)}-\frac{1}{x}\log{(x+1)}\ ,\ x>0 \]
όπου $ c_1,c_2 $ είναι αυθαίρετες σταθερές.\epask
\begin{Askhshs}[B]
\textbf{Με τη βοήθεια ενός μετασχηματισμού της μορφής {\boldmath$ y=x^az $ (όπου $ a $ είναι κατάλληλος πραγματικός αριθμός), να επιλυθεί η γραμμική διαφορική εξίσωση
\[ x^2y''+xy'+\left(x^2-\frac{1}{4}\right)y=x^{3/2}\sin{x}\ ,\ x>0  \]}}
\end{Askhshs}\mbox{}\\
\lysh
Θέτουμε όπου $ y=x^az $ με $ a\in\mathbb{R} $. Τότε θα έχουμε $ y'=ax^{a-1}z+x^az'\Rightarrow y''=a(a-1)x^{a-2}z+2ax^{a-1}z'+x^az'' $ και η διαφορική εξίσωση μετασχηματίζεται ως εξής:
\begin{gather*}
x^2\left[a(a-1)x^{a-2}z+2ax^{a-1}z'+x^az''\right]+x\left(ax^{a-1}z+x^az' \right) +\left(x^2-\frac{1}{4}\right)x^az=x^{3/2}\sin{x}\Rightarrow\\
a(a-1)x^az+2ax^{a+1}z'+x^{a+2}z''+ax^az+x^{a+1}z'+\left(x^{a+2}-\frac{1}{4}x^a\right)z=x^{3/2}\sin{x}\Rightarrow\\
x^{a+2}z''+(2a+1)x^{a+1}z'+\left( x^2+a^2-\frac{1}{4}\right)x^az=x^{3/2}\sin{x}
\end{gather*}
Για $ a=-\frac{1}{2} $ η τελευταία εξίσωση θα γραφτεί
\[ x^{3/2}z''+x^{3/2}z=x^{3/2}\sin{x}\xRightarrow{x>0}
z''+z=\sin{x}\ ,\ x>0 \] και έτσι ο μετασχηματισμός που χρησιμοποιήσαμε ήταν ο $ y=x^{-1/2}z\ ,\ x>0 $. Το χαρακτηριστικό πολυώνυμο της αντίστοιχης ομογενούς εξίσωσης $ z''+z=0 $ θα είναι το $ P(\lambda)=\lambda^2+1 $ το οποίο έχει ρίζες τις $ \lambda_1=i $ και $ \lambda_2=-i $. Ένα βασικό σύνολο λύσεων της ομογενούς εξίσωσης θα είναι το $ \left\lbrace z_1(x)=\cos{x}\ ,\ z_2(x)=\sin{x} \right\rbrace\ ,\ x>0  $ και έτσι όλες οι λύσεις της θα δίνονται από τον τύπο:
\[ \tilde{y}(x)=c_1\cos{x}+c_2\sin{x}\ ,\ x>0 \] όπου $ c_1,c_2 $ είναι αυθαίρετες σταθερές. Για την εύρεση μιας μερικής λύσης της μη ομογενούς εξισωσης θεωρούμε τη διαφορική εξίσωση
\[ z''+z=e^{ix}\ ,\ x>0 \]
Σ' αυτήν θέτουμε $ z=ue^{ix}\ ,\ x>0 $ και έτσι θα έχουμε $ z'=u'e^{ix}+iue^{ix} $ και $ z''=\left( u''+2iu'-u\right) e^{ix} $. Η τελευταία εξίσωση θα γίνει \[ \left( u''+2iu'-u\right) e^{ix}+ue^{ix}=e^{ix}\Rightarrow u''+2iu'=1 \]
Για τη μερική λύση $ u_\mu $ αυτής θα έχουμε $ u'_\mu=c $ και έτσι $ 2ic=1\Rightarrow c=-\frac{i}{2} $. Συνεπώς 
\[ u_\mu(x)=-\frac{1}{2}ix\Rightarrow z_\mu(x)=-\frac{1}{2}ixe^{ix}=-\frac{1}{2}ix(\cos{x}+i\sin{x})=-\frac{1}{2}ix\cos{x}+\frac{1}{2}x\sin{x}\ ,\ x>0 \] και κατά συνέπεια μια μερική λύση θα είναι η $ \nu(x)=\mathrm{Im}{z_\mu(x)}=-\frac{1}{2}x\cos{x}\ ,\ x>0 $. Άρα όλες οι λύσεις της αρχικής διαφορικής εξίσωσης θα δίνονται από τον τύπο
\[ z(x)=c_1\cos{x}+c_2\sin{x}-\frac{1}{2}x\cos{x}\ ,\ x>0 \] ο οποίος ισοδύναμα γράφεται
\[ y(x)=c_1\frac{\sqrt{x}\cos{x}}{x}+c_2\frac{\sqrt{x}\sin{x}}{x}-\frac{1}{2}\sqrt{x}\cos{x}\ ,\ x>0 \]
όπου $ c_1,c_2 $ είναι αυθαίρετες σταθερές.\epask
\begin{Askhshs}[B]
\textbf{Να επιλυθεί η διαφορική εξίσωση}
{\boldmath\[ x^2y''-2xy'+2y=x^3\log{x}\ ,\ x>0 \]}
\end{Askhshs}\mbox{}\\
\lysh
Η αντίστοιχη ομογενής εξίσωση της αρχικής είναι η $ x^2y''-2xy'+2y=0 $ η οποία είναι μια διαφορική εξίσωση Euler. Θέτοντας $ t=\log{x} $ προκύπτουν οι σχέσεις 
\[ x\frac{dy}{dx}=\frac{dy}{dt}\quad\textrm{και}\quad
x^2\frac{d^2y}{dx^2}=-\frac{dy}{dt}+\frac{d^2y}{dt^2} \] τις οποίες αν αντικαταστήσουμε στην ομογενή εξίσωση αυτή θα γραφτεί στη μορφή
\[ -\frac{dy}{dt}+\frac{d^2y}{dt^2}-2\frac{dy}{dt}+2y=0\Rightarrow \frac{d^2y}{dt^2}-3\frac{dy}{dt}+2y=0  \]
Το χαρακτηριστικό πολυώνυμο της είναι το $ P(\lambda)=\lambda^2-3\lambda+2 $ το οποίο έχει ρίζες τις $ \lambda_1=1 $ και $ \lambda_2=2 $. Έτσι παίρνουμε το βασικό σύνολο λύσεων της ομογενούς εξίσωσης $ \left\lbrace y_1(t)=e^t\ ,\ y_2(t)=e^{2t}\right\rbrace\ ,\ t\in\mathbb{R}  $ και κατά συνέπεια τον τύπο από τον οποίο δίνονται όλες οι λύσεις της ομογενούς εξίσωσης:
\[ \tilde{y}(t)=c_1e^t+c_2e^{2t}\ ,\ t\in\mathbb{R} \]
Η μη ομογενής εξίσωση θα είναι $ \frac{d^2y}{dt^2}-3\frac{dy}{dt}+2y=te^{3t}\ ,\ t\in\mathbb{R} $ και για την εύρεση μιας μερικής λύσης της θέτουμε $ y(x)=z(x)e^{3t} $ και παίρνουμε $ y'=(z'+3z)e^{3t} $ και $ y''=(z''+6z'+9)e^{3t} $. Αντικαθιστούμε τις σχέσεις αυτές στη μη ομογενή και αυτή θα γίνει
\begin{equation}\label{b12}
(z''+6z'+9)e^{3t}-3(z'+3z)e^{3t}+2ze^{3t}=te^{3t}\Rightarrow z''+3z'+2z=t\ ,\ t\in\mathbb{R}
\end{equation}
Μια μερική λύση της \ref{b12} θα είναι της μορφής $ z_\mu(t)=at+\beta $. Θα έχουμε $ z'_\mu=a\Rightarrow z''_\mu=0 $ και έτσι προκύπτει:
\[ 3a+2at+2\beta=t \]
Από την τελευταία εξίσωση εξισώνοντας τα πολυώνυμα παίρνουμε το σύστημα
\[ \systeme[a\beta]{2a=1,3a+2\beta=0} \] το οποίο μας δίνει τη λύση $ (a,\beta)=\left(\frac{1}{2},-\frac{3}{4} \right) $ και έτσι η ζητούμενη μερική λύση θα είναι η $ z_\mu(t)=\frac{1}{2}t-\frac{3}{4}\Rightarrow y_\mu(t)=\left( \frac{1}{2}t-\frac{3}{4}\right)e^{3t}\ ,\ t\in\mathbb{R} $. Όλες οι λύσεις της αρχικής διαφορικής εξίσωσης θα δίνονται από τον τύπο
\[ y(t)=c_1e^t+c_2e^{2t}+\left( \frac{1}{2}t-\frac{3}{4}\right)e^{3t}\ ,\ t\in\mathbb{R} \] ο οποίος ισοδύναμα, ύστερα από αντικατάσταση, γράφεται
\[ y(x)=c_1x+c_2x^2+\left( \frac{1}{2}\log{x}-\frac{3}{4}\right)x^3\ ,\ x>0 \] όπου $ c_1,c_2 $ είναι αυθαίρετες σταθερές.\epask
\begin{Askhshs}[B]
\textbf{Με τη βοήθεια ενός μετασχηματισμού της μορφής {\boldmath$ y=u^m $ (όπου $ m\neq0 $ είναι ένας ακέραιος), να επιλυθεί η διαφορική εξίσωση
\[ yy''-2(y')^2+2yy'-y^2+xy^3=0 \]}}
\end{Askhshs}\mbox{}\\
\lysh
Μια προφανής λύση της εξίσωσης είναι η $ y=0 $. Για τις υπόλοιπες ο μετασχηματισμός $ y=u^m $ μας δίνει τις σχέσεις $ y'=mu^{m-1}u' $, $ y''=m(m-1)u^{m-2}(u')^2+mu^{m-1}u'' $, $ (y')^2=m^2u^{2m-2}(u')^2 $, $ y^2=u^{2m} $ και $ y^3=u^{3m} $. Αντικαθιστώντας τις σχέσεις αυτές στη διαφορική εξίσωση θα προκύψει:
\begin{gather*}
u^m\left[m(m-1)u^{m-2}(u')^2+mu^{m-1}u''\right] -2m^2u^{2m-2}(u')^2+2u^mmu^{m-1}u'-u^{2m}+xu^{3m}=0\Rightarrow\\
m(m-1)u^{2m-2}(u')^2+mu^{2m-1}u''-2m^2u^{2m-2}(u')^2+
2mu^{2m-1}u'-u^{2m}+xu^{3m}=0\Rightarrow\\
mu^{2m-1}u''+\left(-m^2-m\right)u^{2m-2}(u')^2+2mu^{2m-1}u'-u^{2m}+xu^{3m}=0
\end{gather*}
Επιλέγοντας $ m=-1 $ η τελευταία εξίσωση θα γίνει
\[ -u^{-3}u''-2u^{-3}u'-u^{-2}+xu^{-3}=0\Rightarrow u''+2u'+u=x \]
Η αντίστοιχη ομογενής εξίσωση θα είναι η $ u''+2u'+u=0 $ η οποία έχει χαρακτηριστικό πολυώνυμο το $ P(\lambda)=\lambda^2+2\lambda+1 $. Η ρίζα του πολυωνύμου είναι η $ \lambda=-1 $ πολλαπλότητας $ 2 $ άρα ένα βασικό σύνολο λύσεων θα είναι το $ \left\lbrace u_1(x)=e^{-x}\ ,\ u_2(x)=xe^{-x} \right\rbrace  $. Όλες οι λύσεις της ομογενούς εξίσωσης θα δίνονται από τον τύπο:
\[ u(x)=c_1e^{-x}+c_2xe^{-x} \] όπου $ c_1,c_2 $ είναι αυθαίρετες σταθερές. Για να βούμε μια μερική λύση της μη ομογενούς εξίσωσης αναζητούμε μια λύση της μορφής $ u_\mu=ax+\beta $ η οποία μας δίνει $ u_\mu'=a $ και $ u_\mu''=0 $. Έτσι θα έχουμε
\[ 2a+ax+\beta=x \]
Εξισώνοντας τα πολυώνυμα από την τελευταία σχέση, παίρνουμε το σύστημα
\[ \systeme[a\beta]{a=1,2a+\beta=0} \]
το οποίο έχει λύση την $ (a,\beta)=\left(1,-2\right) $ και κατά συνέπεια η μερική λύση θα είναι $ u_\mu(x)=x-2 $. Όλες οι λύσεις λοιπόν της διαφορικής εξίσωσης θα δίνονται από τον τύπο 
\[ u(x)=c_1e^{-x}+c_2xe^{-x}+x-2 \] όπου ύστερα από την αντικατάσταση $ y=u^{-1} $ θα γίνει
\[ y(x)=\frac{1}{c_1e^{-x}+c_2xe^{-x}+x-2} \] με $ c_1,c_2 $ αυθαίρετες σταθερές.\epask
\begin{Askhshs}[B]
\textbf{Να αποδειχθεί ότι οι συναρτήσεις {\boldmath$ y_1(x)=x^{-1/2}\cos{x}\ ,\ x>0 $ και $ y_2(x)=x^{-1/2}\sin{x}\ ,\ x>0 $ αποτελούν ένα βασικό σύνολο λύσεων της ομογενούς γραμμικής εξίσωσης
\[ x^2y''+xy'+\left(x^2-\frac{1}{4}\right)y=0\ ,\ x>0 \] Στη συνέχεια να επιλυθεί η μη ομογενής γραμμική διαφορική εξίσωση\[ x^2y''+xy'+\left(x^2-\frac{1}{4}\right)y=x^{3/2}\ ,\ x>0 \]}}
\end{Askhshs}\mbox{}\\
\lysh
Για να αποτελούν οι συναρτήσεις $ y_1(x)=x^{-1/2}\cos{x} $ και $ y_2(x)=x^{-1/2}\sin{x}\ ,\ x>0 $ βασικό σύνολο λύσεων της ομογενούς εξίσωσης θα πρέπει να την επαληθεύουν και συγχρόνως να ισχύει $ W(y_1,y_2)(x)\neq0\ ,\ \forall x>0 $. Για τη συνάρτηση $ y_1 $ θα ισχύει
\[ y_1'(x)=-\frac{x^{-3/2}}{2}\cos{x}-x^{-1/2}\sin{x}\Rightarrow y''_1(x)=\frac{3x^{-5/2}}{4}\cos{x}+x^{-3/2}\sin{x}+x^{-1/2}\cos{x} \] και έτσι αντικαθιστώντας τις σχέσεις αυτές στη διαφορική εξίσωση θα έχουμε 
\begin{gather*}
x^2\left(\frac{3\cos{x}}{4x^{5/2}}+\frac{\sin{x}}{x^{3/2}}+\frac{\cos{x}}{x^{1/2}}\right)-x\left(\frac{\cos{x}}{2x^{3/2}}+\frac{\sin{x}}{x^{1/2}}\right)+\left(x^2-\frac{1}{4}\right)\frac{\cos{x}}{x^{1/2}}=0\Rightarrow\\
\frac{3\cos{x}}{4x^{1/2}}+x^{1/2}\sin{x}+x^{3/2}\cos{x}-\frac{\cos{x}}{2x^{1/2}}-x^{1/2}\sin{x}-x^{3/2}\cos{x}-\frac{\cos{x}}{4x^{1/2}}=0\Rightarrow 0=0
\end{gather*} Άρα ισχύει. Ομοίως υπολογίζουμε και τις παραγώγους μέχρι και 2\tss{ης} τάξης της συνάρτησης $ y_2(x)=x^{-1/2}\sin{x} $ και εύκολα εξετάζουμε ότι και αυτή επαληθεύει την ομογενή εξίσωση. Μένει να δείξουμε ότι οι συναρτήσεις είναι γραμμικά ανεξάρτητες. Έχουμε λοιπόν
\begin{align*}
W(y_1,y_2)(x)&=
\begin{vmatrix}
x^{-1/2}\cos{x} & x^{-1/2}\sin{x} \\
-\frac{x^{-3/2}}{2}\cos{x}-x^{-1/2}\sin{x} & -\frac{x^{-3/2}}{2}\sin{x}+x^{-1/2}\cos{x}\\
\end{vmatrix}=\\
&=-\frac{x^{-2}}{2}\cos{x}\sin{x}+x^{-1}\cos^2{x}+\frac{x^{-2}}{2}\cos{x}\sin{x}+x^{-1}\sin^2{x}=\\
&=x^{-1}\left(\cos^2{x}+\sin^2{x}\right)=\frac{1}{x}\neq0\ ,\ \forall x>0
\end{align*}
Άρα οι συναρτήσεις $ y_1(x)=x^{-1/2}\cos{x}\ ,\ x>0 $ και $ y_2(x)=x^{-1/2}\sin{x}\ ,\ x>0 $ αποτελούν βασικό σύνολο λύσεων της ομογενούς διαφορικής εξίσωσης. Για την επίλυση της μη ομογενούς εξίσωσης \[ x^2y''+xy'+\left(x^2-\frac{1}{4}\right)y=x^{3/2} \]
αρκεί να βρούμε μια μερική λύση $ y_\mu $. Έχουμε λοιπόν 
\begin{gather*}
W_1(y_1,y_2)(x)=
\begin{vmatrix}
0 & x^{-1/2}\sin{x} \\
1 & -\frac{x^{-3/2}}{2}\sin{x}+x^{-1/2}\cos{x}\\
\end{vmatrix}=-x^{-1/2}\sin{x}\ ,\textrm{ και }\\
W_2(y_1,y_2)(x)=
\begin{vmatrix}
x^{-1/2}\cos{x} & 0 \\
-\frac{x^{-3/2}}{2}\cos{x}-x^{-1/2}\sin{x} & 1\\
\end{vmatrix}=x^{-1/2}\cos{x}
\end{gather*}
Συνεπώς $ \forall x>0 $ είναι:
\begin{align*}
y_\mu(x)&=y_1(x)\int{\frac{W_1(y_1,y_2)(x)}{W(y_1,y_2)(x)}\frac{b(x)}{a_2(x)}\ \mathrm{d}x}+y_2(x)\int{\frac{W_2(y_1,y_2)(x)}{W(y_1,y_2)(x)}\frac{b(x)}{a_2(x)}\ \mathrm{d}x}=\\
&=x^{-1/2}\cos{x}\int{\frac{-x^{-1/2}\sin{x}}{x^{-1}}\frac{x^{3/2}}{x^2}\ \mathrm{d}x}+x^{-1/2}\sin{x}\int{\frac{x^{-1/2}\cos{x}}{x^{-1}}\frac{x^{3/2}}{x^2}\ \mathrm{d}x}=\\
&=x^{-1/2}\cos{x}\int{-\sin{x}\ \mathrm{d}x}+x^{-1/2}\sin{x}\int{\cos{x}\ \mathrm{d}x}=\\
&=x^{-1/2}\cos^2{x}+x^{-1/2}\sin^2{x}=x^{-1/2}
\end{align*}
Άρα μια μερική λύση της εξίσωσης θα είναι η $ y_\mu(x)=x^{-1/2}\ ,\ x>0 $ οπότε όλες οι λύσεις της θα δίνονται από τον τύπο:
\[ y(x)=c_1x^{-1/2}\cos{x}+c_2\sin{x}+x^{-1/2}\ ,\ x>0 \]
με $ c_1,c_2 $ αυθαίρετες σταθερές.\epask
\begin{Askhshs}[B]
\textbf{Με τη βοήθεια του μετασχηματισμού {\boldmath$ y=\frac{z}{x}\sin{x} $, να επιλυθεί το πρόβλημα αρχικών τιμών
\[ xy''+2y'+xy=0\ ,\ x\in\left(0,\frac{\pi}{2}\right)\ ,\ y\left(\frac{\pi}{4}\right)=0\ ,\ y'\left(\frac{\pi}{4}\right)=1 \]}}
\end{Askhshs}\mbox{}\\
\lysh
Θέτοντας όπου $ y=\frac{z}{x}\sin{x} $ παίρνουμε $ y'=\left( -\frac{1}{x^2}\sin{x}+\frac{1}{x}\cos{x}\right)z+\frac{1}{x}\sin{x}z' $ και $ y''=\left( \frac{2}{x^3}\sin{x}-\frac{2}{x^2}\cos{x}-\right. $ $\left. \frac{1}{x}\sin{x}\right)z+\left( -\frac{2}{x^2}\sin{x}+\frac{2}{x}\cos{x}\right)z'+\frac{1}{x}\sin{x}z'' $. Έτσι αντικαθιστώντας στη διαφορική εξίσωση αυτή θα γίνει:
\begin{multline*}
\sin{x}z''+\left( -\frac{2}{x}\sin{x}+2\cos{x}\right)z'+\frac{1}{x}\sin{x}z'+\left( \frac{2}{x^3}\sin{x}-\frac{2}{x^2}\cos{x}-\frac{1}{x}\sin{x}\right)z+\\\left( -\frac{2}{x^2}\sin{x}+\frac{2}{x}\cos{x}\right)z+\frac{2}{x}\sin{x}z'+\sin{x}=0\Rightarrow
\end{multline*}
\[ \sin{x}z''+2\cos{x}z'=0\ ,\ x\in\left( 0,\frac{\pi}{2}\right)  \]
Στην τελευταία εξίσωση λείπει η άγνωστη συνάρτηση $ z $ επομένως θέτοντας $ z'=\nu $ πετυχαίνουμε υποβιβασμό της τάξης και παίρνουμε τη γραμμική διαφορική εξίσωση 1\tss{ης} τάξης
\[ \sin{x}\nu'+2\cos{x}\nu=0\ ,\ x\in\left( 0,\frac{\pi}{2}\right) \]
η οποία έχει λύση την
\[ \nu(x)=c_1e^{-\dintt{\frac{2\cos{x}}{\sin{x}}\ \mathrm{d}x}}=c_1e^{-2\log{(\sin{x})}}=\frac{c_1}{\sin^2{x}}\ ,\ x\in\left( 0,\frac{\pi}{2}\right) \]
Συνεπώς θα έχουμε 
\begin{gather*}
z'=\frac{c_1}{\sin^2{x}}\Rightarrow dz=\frac{c_1}{\sin^2{x}}dx\Rightarrow
z(x)=\int{\frac{c_1}{\sin^2{x}}\ \mathrm{d}x}=-c_1\cot{x}+c_2\ ,\ x\in\left( 0,\frac{\pi}{2}\right)
\end{gather*}
όπου $ c_1,c_2 $ είναι αυθαίρετες σταθερές. Τελικά όλες οι λύσεις της αρχικής διαφορικής εξίσωσης θα δίνονται από τον τύπο:
\[ y(x)=\frac{1}{x}\left( -c_1\cot{x}+c_2\right)=-c_1\frac{\cos{x}}{x}+c_2\frac{\sin{x}}{x}\ ,\ x\in\left( 0,\frac{\pi}{2}\right) \]
όπου $ c_1,c_2 $ είναι αυθαίρετες σταθερές. Για τη λύση που πληροί το πρόβλημα αρχικών τιμών θα έχουμε:
\[ y'(x)=c_1\frac{x\sin{x}+\cos{x}}{x^2}+c_2\frac{x\cos{x}-\sin{x}}{x^2} \]
Έτσι αποκτάμε το σύστημα
\[ \ccases{y\left(\frac{\pi}{4}\right)=0\\y'\left(\frac{\pi}{4}\right)=1}\!\!\!\Rightarrow\ccases{\frac{2\sqrt{2}}{\pi}(c_1-c_2)=0\\
c_1\frac{2\sqrt{2}\pi+8\sqrt{2}}{\pi^2}+c_2\frac{2\sqrt{2}\pi-8\sqrt{2}}{\pi^2}=1}\!\!\!\Rightarrow c_1=c_2=\frac{\pi}{4\sqrt{2}} \]
Επομένως η λύση του προβλήματος θα είναι η 
\[ y(x)=\frac{\pi}{4\sqrt{2}}\left(\frac{\sin{x}-x\cos{x}}{x}\right)\ ,\ x\in\left( 0,\frac{\pi}{2}\right) \]\epask
\begin{Askhshs}[B]
\textbf{Με την αντικατάσταση {\boldmath$ z=e^{-y/x} $, να επιλυθεί το πρόβλημα αρχικών τιμών
\[ x^3y''+2x^2=\left( xy'-y\right)^2 ,\ y(1)=0\ ,\ y'(1)=2 \]}}
\end{Askhshs}\mbox{}\\
\lysh
Θέτουμε $ z=e^{-y/x}\Rightarrow y=-x\log{z} $ και ύστερα από παραγώγιση παίρνουμε:
\[ y'=-\log{z}-\frac{x}{z}z'\ \textrm{ και }\ y''=-\frac{2}{z}z'+\frac{x}{z'}\left( z'\right) ^2-\frac{x}{z}z'' \]
Η διαφορική εξίσωση μετά από αντικατάσταση θα γίνει:
\begin{gather*}
x^3\left( -\frac{2}{z}z'+\frac{x}{z'}(z')^2-\frac{x}{z}z''\right) +2x^2=\left( -x\log{z}-\frac{x^2}{z}z'+x\log{z}\right)^2\Rightarrow\\
-\frac{2x^3}{z}z'+\frac{x^4}{z'}\left( z'\right) ^2-\frac{x^4}{z}z''+2x^2=\frac{x^4}{z^2}\left( z'\right) ^2\Rightarrow\\
-2x^3z'-x^4z''+2x^2z=0\Rightarrow x^2 z''+2xz'-2z=0
\end{gather*}
η οποία είναι μια διαφορική εξίσωση Euler. Θέτουμε λοιπόν σ' αυτήν όπου $ t=\log{x} $ και θα έχουμε:
\[ x\frac{dz}{dx}=\frac{dz}{dt}\quad\textrm{και}\quad
x^2\frac{d^2z}{dx^2}=-\frac{dz}{dt}+\frac{d^2z}{dt^2} \]
Οι παραπάνω σχέσεις θα μετατρέψουν την εξίσωση στην
\begin{equation*}\label{b16}
\frac{d^2z}{dt^2}+\frac{dz}{dt}-2z=0
\end{equation*}
η οποία είναι μια γραμμική διαφορική εξίσωση με σταθερούς συντελεστές. Αυτή έχει χαρακτηριστικό πολυώνυμο το $ P(\lambda)=\lambda^2+\lambda-2=0 $ το οποίο έχει ρίζες τις $ \lambda=1 $ και $ \lambda=-2 $. Έτσι οι συναρτήσεις $ z_1(t)=e^t $ και $ z_2=e^{-2t} $ αποτελούν βασικό σύνολο λύσεων της \ref{b16} άρα οι λύσεις θα δίνονται από τον τύπο
\[ z(t)=c_1e^t+c_2e^{-2t} \]
όπου $ c_1,c_2 $ είναι αυθαίρετες σταθερές. Ισοδύναμα ο προηγούμενος τύπος γράφεται $ z(x)=c_1x+c_2\frac{1}{x^2} $ και τελικά όλες οι λύσεις της αρχικής διαφορικής εξίσωσης θα δίνονται από τον τύπο:
\[ y(x)=-x\log\left(c_1x+c_2\frac{1}{x^2}\right) \]
όπου $ c_1,c_2 $ είναι αυθαίρετες σταθερές. Για τη λύση που πληροί το πρόβλημα αρχικών τιμών θα έχουμε:
\[ y'(x)=-\log\left(c_1x+c_2\frac{1}{x^2}\right)-\frac{x}{c_1x+c_2\frac{1}{x^2}}\left( c_1-c_2\frac{2}{x^3}\right) \]
Έτσι αποκτάμε το σύστημα
\[ \ccases{y(1)=0\\y'(1)=2}\!\!\!\Rightarrow\ccases{-\log{(c_1+c_2)}=0\\
-\log\left(c_1+c_2\right)-\frac{1}{c_1+c_2}\left( c_1-2c_2\right)=2}\!\!\!\Rightarrow c_1=0\ ,\ c_2=1 \]
Συνεπώς η λύση που ζητάμε θα είναι η $ y(x)=-x\log{\left( \frac{1}{x^2}\right) } $.\epask
\begin{Askhshs}[B]
\textbf{Να επιλυθεί η ομογενής γραμμική διαφορική εξίσωση {\boldmath\[ y''+(2x+1)y'+\left( x^2+x+\frac{1}{4}\right)y=0 \]
αφού πρώτα βρεθεί μια λύση $ y_1 $ της μορφής $ y_1(x)=e^{ax^2+\beta x}\ ,\ x\in\mathbb{R} $ (όπου $ a,\beta $ είναι πραγματικές σταθερές που πρέπει να προσδιοριστούν).}}
\end{Askhshs}\mbox{}\\\mbox{}\\
\lysh
Η συνάρτηση $ y_1(x)=e^{ax^2+\beta x}\ ,\ x\in\mathbb{R} $ είναι μια λύση της εξίσωσης αν και μόνο αν την επαληθεύει οπότε $ \forall x\in\mathbb{R} $ θα ισχύει
\begin{gather*}
y_1'(x)=(2ax+\beta)e^{ax^2+\beta x}\ \textrm{ και }\ 
y_1''(x)=\left( 4a^2x^2+4a\beta x+\beta^2+2a\right)e^{ax^2+\beta x}
\end{gather*}
Αν αντικαταστήσουμε τις παραγώγους αυτές και τη συνάρτηση στη διαφορική εξίσωση αυτή θα γίνει
\begin{gather*}
y_1''+(2x+1)y_1'+\left( x^2+x+\frac{1}{4}\right)y_1=0\Rightarrow\\
4a^2x^2+4a\beta x+\beta^2+2a+(2x+1)(2ax+\beta)+x^2+x+\frac{1}{4}=0\Rightarrow\\
\left( 4a^2+4a+1\right)x^2+(4a\beta+2\beta+2a+1)x+\beta^2+2a+\beta+\frac{1}{4}=0
\end{gather*}
Το πολυώνυμο αυτό είναι το μηδενικό πολυώνυμο αν και μόνο αν
\[ \ccases{4a^2+4a+1=0\\4a\beta+2\beta+2a+1=0\\\beta^2+2a+\beta+\frac{1}{4}=0} \]
Το σύστημα αυτό μας δίνει τις λύσεις $ (a,\beta)=\left(-\frac{1}{2},\frac{1}{2} \right)  $ και $ (a,\beta)=\left(-\frac{1}{2},-\frac{3}{2} \right) $ άρα μια λύση είναι η $ y_1(x)=e^{-\frac{1}{2}x^2+\frac{1}{2}x} $ ενώ μια δεύτερη λύση είναι η $ y_2(x)=e^{-\frac{1}{2}x^2-\frac{3}{2}x} $. Αν αποδείξουμε ότι οι δύο λύσεις αυτές είναι γραμμικά ανεξάρτητες τότε όλες οι λύσεις της εξίσωσης θα δίνονται από τον τύπο $ y(x)=c_1y_1(x)+c_2y_2(x) $. Πράγματι $ \forall x\in\mathbb{R} $ θα έχουμε:
\begin{align*}
W(y_1,y_2)(x)&=\begin{vmatrix}
e^{-\frac{1}{2}x^2+\frac{1}{2}x} & e^{-\frac{1}{2}x^2-\frac{3}{2}x}\\
\left( -x+\frac{1}{2}\right) e^{-\frac{1}{2}x^2+\frac{1}{2}x} & \left( -x-\frac{3}{2}\right) e^{-\frac{1}{2}x^2-\frac{3}{2}x}
\end{vmatrix}=\\
&=-e^{-x^2-x}\begin{vmatrix}
1 & 1\\
x-\frac{1}{2} & x+\frac{3}{2}
\end{vmatrix}=-2e^{-x^2-x}\neq 0
\end{align*}
Συνεπώς οι λύσεις αυτές αποτελούν βασικό σύνολο λύσεων της εξίσωσης και έτσι όλες οι λύσεις της θα δίνονται από τον τύπο
\[ y(x)=c_1e^{-\frac{1}{2}x^2+\frac{1}{2}x}+c_2e^{-\frac{1}{2}x^2-\frac{3}{2}x}\ ,\ x\in\mathbb{R} \]
όπου $ c_1,c_2 $ είναι αυθαίρετες σταθερές.\epask
\begin{Askhshs}[B]
\textbf{Με τη βοήθεια της αντικατάστασης {\boldmath$ y=ze^{-x^2/2} $, να επιλυθεί το πρόβλημα αρχικών τιμών
\[ y''+(2x-1)y'+\left( x^2-x+1\right)y=0\ \ ,\ y(0)=0\ ,\ y'(0)=5 \]}}
\end{Askhshs}\mbox{}\\
\lysh
Θέτουμε $ y=ze^{-x^2/2} $ και παραγωγίζοντας δύο φορές παίρνουμε:
\begin{gather*}
y'=z'e^{-x^2/2}+z\left(-2x\right)e^{-x^2/2}=\left( z'-xz\right)e^{-x^2/2}\ \textrm{ και }\\
y''=\left( z''-z-xz'\right)e^{-x^2/2}+\left( -xz'+x^2z\right)e^{-x^2/2}=\left[  z''-2xz'+\left( x^2+1\right)z\right]e^{-x^2/2}
\end{gather*}
Αντικαθιστούμε τις σχέσεις αυτές και η διαφορική εξίσωση γίνεται
\begin{gather}
y''+(2x-1)y'+\left( x^2-x+1\right)y=0\Rightarrow\nonumber\\
z''-2xz'+\left( x^2+1\right)z+(2x+1)\left(z'-xz\right)+\left( x^2-x+1\right)z=0\Rightarrow\nonumber\\
z''-2xz'+x^2z+z+2xz'-2x^2z+z'-xz+x^2z-xz+z=0\Rightarrow\nonumber\\
z''-z'=0\label{b18}
\end{gather}
Η τελευταία είναι μια γραμμική διαφορική εξίσωση 2\tss{ης} τάξης με σταθερούς συντελεστές η οποία έχει χαρακτηριστικό πολυώνυμο το $ P(\lambda)=\lambda^2-\lambda $. Το πολυώνυμο έχει ρίζες τις $ \lambda_1=0 $ και $ \lambda_2=1 $ οπότε οι συναρτήσεις
\[ z_1(x)=1\ \textrm{ και }\ z_2=e^x \]
αποτελούν βασικό σύνολο λύσεων της \eqref{b18} και έτσι όλες οι λύσεις της δίνονται από τον τύπο $ z(x)=c_1+c_2e^x $ που γράφεται ισοδύναμα
\[ y(x)=c_1e^{-x^2/2}+c_2e^{\left( 2x-x^2\right)/2} \]
όπου $ c_1,c_2 $ είναι αυθαίρετες σταθερές. Για τη λύση η οποία πληροί το πρόβλημα αρχικών τιμών θα έχουμε
\[ y'(x)=-2xc_1e^{-x^2/2}+(1-x)c_2e^{\left( 2x-x^2\right)/2} \] και έτσι από τις αρχικές συνθήκες για $ x=0 $ παίρνουμε το σύστημα:
\[ \syssubstitute{.,{a}{c_1}{b}{c_2}}
\systeme{a+b=0,b=5} \]
Έτσι έχουμε $ c_1=-5 $ και $ c_2=5 $ τα οποία μας δίνουν τη λύση
\[ y(x)=5e^{-x^2/2}\left( e^x-1\right)  \]\epask
\begin{Askhshs}[B]
\textbf{Με τη βοήθεια μιας αντικατάστασης της μορφής {\boldmath$ t=x^a $ (όπου $ a $ είναι κατάλληλος αριθμός που θα πρέπει να βρεθεί), να επιλυθεί η γραμμική διαφορική εξίσωση
\[ x\frac{d^2y}{dx^2}-\frac{dy}{dx}+4x^3y=4x^5e^{x^2}\sin{x^2}\ ,\ x>0 \]}}
\end{Askhshs}\mbox{}\\
\lysh
Θέτουμε $ t=x^a\ ,\ x>0 $ όπου $ a $ κατάλληλος αριθμός και $ \forall x\in\mathbb{R} $ έχουμε:
\begin{gather*}
\frac{dy}{dx}=\frac{dy}{dt}\cdot\frac{dt}{dx}=\frac{dy}{dt}ax^{a-1}\ \textrm{ και}\\
\frac{d^2y}{dx^2}=\frac{d}{dx}\left( \frac{dy}{dx}\right)=\frac{d}{dx}\left(\frac{dy}{dt}ax^{a-1} \right)=a(a-1)x^{a-2}\frac{dy}{dt}+a^2x^{2a-2}\frac{d^2y}{dt^2}
\end{gather*}
Έτσι η εξίσωση γίνεται
\begin{gather*} t^{1/a}\left[a(a-1)t^{\frac{a-2}{a}}\frac{dy}{dt}+a^2t^{\frac{2a-2}{a}}\frac{d^2y}{dt^2}\right]-\frac{dy}{dt}at^{\frac{a-1}{a}}+4t^{3/a}y=4t^{5/a}e^{t^{2/a}}\sin{t^{2/a}}\Rightarrow\\
a(a-1)t^{\frac{a-1}{a}}\frac{dy}{dt}+a^2t^{\frac{2a-1}{a}}\frac{d^2y}{dt^2}-\frac{dy}{dt}at^{\frac{a-1}{a}}+4t^{3/a}y=4t^{5/a}e^{t^{2/a}}\sin{t^{2/a}}\Rightarrow\\
a^2t^{\frac{2a-1}{a}}\frac{d^2y}{dt^2}+\left( a^2-2a\right)t^{\frac{a-1}{a}}\frac{dy}{dt}+4t^{3/a}y=4t^{5/a}e^{t^{2/a}}\sin{t^{2/a}}
\end{gather*}
Επιλέγοντας $ a=2 $ η τελευταία εξίσωση γίνεται:
\[ \frac{d^2y}{dt^2}+y=te^{t}\sin{t} \]
Ο μετασχηματισμός που χρησιμοποιήθηκε ήταν ο $ t=x^2 $ ενώ η τελευταία εξίσωση είναι μια γραμμική διαφορική εξίσωση 2\tss{ης} τάξης με σταθερούς συντελεστές. Το χαρακτηριστικό πολυώνυμο της αντίστοιχης ομογενούς εξίσωσης είναι το $ P(\lambda)=\lambda^2+1 $ με ρίζες τις $ \lambda=\pm i $. Έτσι οι συναρτήσεις $ y_1(t)=\cos{t}\ ,\ t>0 $ και $ y_2(t)=\sin{t}\ ,\ t>0 $ αποτελούν βασικό σύνολο λύσεων της εξίσωσης. Όλες οι λύσεις της ομογενούς εξίσωσης θα δίνονται από τον τύπο:
\[ y(t)=c_1\cos{t}+c_2\sin{t}\ ,\ t>0 \]
με $ c_1,c_2 $ αυθαίρετες σταθερές. Για να βρούμε μια μερική λύση της παραπάνω εξίσωσης θα θεωρήσουμε τη διαφορική εξίσωση
\begin{equation}\label{b19}
\frac{d^2y}{dt^2}+y=te^{(1+i)t}\ ,\ t>0
\end{equation}
Σ' αυτήν θέτουμε $ y=ze^{(1+i)t} $ και με παραγώγιση παίρνουμε
\[ y'=\left[ z'+(1+i)z\right]e^{(1+i)t}\Rightarrow y''=\left[ z''+2(1+i)z'+(1+i)^2z\right]e^{(1+i)t}\ ,\ t>0 \]
Έτσι η εξίσωση θα πάρει τη μορφή
\[ z''+2(1+i)z'+(1+2i)z=t\ ,\ t>0 \]
Θα αναζητήσουμε μια μερική λύση της μορφής $ z_\mu(t)=at+\beta $ η οποία μας δίνει $ z'_\mu=a $ και $ z''_\mu=0 $. Παίρνουμε λοιπόν 
\[ 2(1+i)a+(1+2i)(at+\beta)=t \Rightarrow \ccases{(1+2i)a=1\\
2(1+i)a+(1+2i)\beta=0}\]
Το σύστημα που προέκυψε έχει λύση την $ (a,\beta)=\left( \frac{1-2i}{5},\frac{-2+14i}{25}\right) $ οπότε η ζητούμενη μερική λύση της \eqref{b19} θα είναι η $ z_\mu(t)=\frac{1-2i}{5}t+\frac{-2+14i}{25} $ η οποία γράφεται ισοδύναμα στη μορφή $ y_\mu(t)=\left( \frac{1-2i}{5}t+\frac{-2+14i}{25}\right)e^{(1+i)t} $ ή τελικά ύστερα από πράξεις και αντικατάσταση
\begin{align*}
&y_\mu(x)=\left( \frac{1-2i}{5}x^2+\frac{-2+14i}{25}\right)e^{x^2}\left( \cos{x^2}+i\sin{x^2}\right)=\\
&=e^{x^2}\left( \frac{1-2i}{5}\cdot x^2\cdot \cos{x^2}+i\frac{1-2i}{5}\cdot x^2\sin{x^2}+\frac{-2+14i}{25}\cdot\cos{x^2}+i\frac{-2+14i}{25}\sin{x^2}\right)=\\
&=e^{x^2}\left\lbrace \frac{x^2\left( \cos{x^2}+2\sin{x^2}\right) }{5}-\frac{2\cos{x^2}+14\sin{x^2}}{25}+i\left[\frac{x^2\left(\sin{x^2}-2\cos{x^2}\right) }{5}+\frac{14\cos{x^2}-2\sin{x^2}}{25} \right]\right\rbrace  
\end{align*}
Η μερική λύση της αρχικής μη ομογενούς εξίσωσης σύμφωνα με τα παραπάνω θα είναι: $ \nu_\mu(x)=Im(y_\mu(x))=e^{x^2}\left[ \frac{x^2\left(\sin{x^2}-2\cos{x^2}\right) }{5}+\frac{14\cos{x^2}-2\sin{x^2}}{25}\right] $ και έτσι καταλήγουμε στον τύπο από τον οποίο δίνονται όλες οι λύσεις της αρχικής διαφορικής εξίσωσης
\[ y(x)=c_1\cos{x^2}+c_2\sin{x^2}+e^{x^2}\left[ \frac{x^2\left(\sin{x^2}-2\cos{x^2}\right) }{5}+\frac{14\cos{x^2}-2\sin{x^2}}{25}\right]\ ,\ x>0 \]
όπου $ c_1,c_2 $ είναι αυθαίρετες σταθερές.\epask
\begin{Askhshs}[B]
\textbf{Με την αντικατάσταση {\boldmath{$ u=y\sqrt{x}\ ,\ x>0 $} να επιλυθεί η ομογενής γραμμική διαφορική εξίσωση
\[ x^2y''+xy'+\left( x^2-\frac{1}{4}\right)y=0\ ,\ x>0 \]}}
\end{Askhshs}\mbox{}\\
\lysh
Θέτουμε στην εξίσωση όπου $ u=y\sqrt{x}\Rightarrow y=x^{-1/2}u $ και ύστερα από διαδοχικές παραγωγίσεις θα έχουμε $ y'=-\frac{1}{2}x^{-3/2}u+x^{-1/2}u' $ και $ y''=x^{-1/2}u''-x^{-3/2}u'+\frac{3}{4}x^{-5/2}u $. Αντικαθιστούμε στην αρχική διαφορική εξίσωση τα παραπάνω και αυτή θα γραφτεί:
\begin{gather}
x^2\left( x^{-1/2}u''-x^{-3/2}u'+\frac{3}{4}x^{-5/2}u\right) +x\left(-\frac{1}{2}x^{-3/2}u+x^{-1/2}u'\right)  +\left( x^2-\frac{1}{4}\right)x^{-1/2}u=0\Rightarrow\nonumber\\
x^{3/2}u''-x^{1/2}u'+\frac{3}{4}x^{-1/2}u-\frac{1}{2}x^{-1/2}u+x^{1/2}u'+x^{3/2}u-\frac{1}{4}x^{-1/2}u=0\Rightarrow\nonumber\\
x^{3/2}u''+x^{3/2}u=0\xRightarrow{x>0}u''+u=0\label{b20}\ ,\ x>0
\end{gather}
Καταλλήξαμε σε μια ομογενή γραμμική διαφορική εξίσωση 2\tss{ης} τάξης με σταθερούς συντελεστές η οποία έχει χαρακτηριστικό πολυώνυμο $ P(\lambda)=\lambda^2+1 $ το οποίο έχει ρίζες τις $ \lambda=\pm i $. Έτσι οι συναρτήσεις $ u_1(x)=\cos{x}\ ,\ x>0 $ και $ u_2(x)=\sin{x}\ ,\ x>0 $ αποτελούν ένα βασικό σύνολο λύσεων της \eqref{b20}. Οι λύσεις της θα δίνονται από τον τύπο
\[ u(x)=c_1\cos{x}+c_2\sin{x}\ ,\ x>0 \]
όπου $ c_1,c_2 $ είναι αυθαίρετες σταθερές. Έτσι οι λύσεις της αρχικής διαφορικής εξίσωσης θα είναι 
\[ y(x)=c_1\frac{\sqrt{x}\cos{x}}{x}+c_2\frac{\sqrt{x}\sin{x}}{x}\ ,\ x>0 \]
\epask
\begin{Askhshs}[B]
\textbf{Με τη χρήση του μετασχηματισμού {\boldmath$ t=1+x^2 $ να λυθεί η ομογενής διαφορική εξίσωση
\[ x\left(1+x^2 \right)^2y''-\left( 1-3x^2\right) \left( 1+x^2\right)y'-8x^3y=4x^3\left( 1+x^2\right)\ ,\ x>0 \]}}
\end{Askhshs}\mbox{}\\
\lysh
Με τη βοήθεια του μετασχηματισμού $ t=1+x^2 $ έχουμε $ \forall x>0 $:
\begin{align}
y'&=\frac{dy}{dx}=\frac{dy}{dt}\cdot\frac{dt}{dx}=\frac{dy}{dt}\cdot 2x\label{b211}\\
y''&=\frac{d^2y}{dx^2}=\frac{d}{dx}\left( \frac{dy}{dx}\right)= \frac{d}{dx}\left( 2x\frac{dy}{dt}\right)=2\frac{dy}{dt}+2x\frac{d}{dx}\left( \frac{dy}{dt}\right)=\nonumber\\
&=2\frac{dy}{dt}+2x\frac{d}{dt}\left( \frac{dy}{dx}\right)=2\frac{dy}{dt}+4x^2\frac{d^2y}{dt^2}\label{b212}
\end{align}
Αντικαθιστώντας τις σχέσεις \eqref{b211} και \eqref{b212} στην αρχική εξίσωση, αυτή θα πάρει τη μορφή:
\begin{gather*}
xt^2\left( 2\frac{dy}{dt}+4x^2\frac{d^2y}{dt^2}\right) -\left( 1-3x^2\right) t\frac{dy}{dt}\cdot 2x-8x^3y=4x^3t\Rightarrow\\
4x^3t^2\frac{d^2y}{dt^2}+\left( 2xt-2x+6x^3 \right)t\frac{dy}{dt}-8x^3y=4x^3t\Rightarrow
\end{gather*}
\begin{gather*}
4x^3t^2\frac{d^2y}{dt^2}+\left[ 2x\left( 1+x^2\right) -2x+6x^3 \right]t\frac{dy}{dt}-8x^3y=4x^3t\Rightarrow\\
4x^3t^2\frac{d^2y}{dt^2}+8x^3t\frac{dy}{dt}-8x^3y=4x^3t\xRightarrow{x>0}t^2\frac{d^2y}{dt^2}+2t\frac{dy}{dt}-2y=t
\end{gather*}
Η τελευταία διαφορική εξίσωση έχει αντίστοιχη ομογενή την $ t^2\frac{d^2y}{dt^2}+2t\frac{dy}{dt}-2y=0 $ η οποία αποτελεί μια διαφορική εξίσωση Euler. Αυτή θα μετατραπεί σε γραμμική εξίσωση 2\tss{ης} τάξης θέτοντας όπου $ z=\log{t} $. Ο μετασχηματισμός μας δίνει τις παρακάτω σχέσεις
\[ t\frac{dy}{dt}=\frac{dy}{dz}\quad\textrm{και}\quad
t^2\frac{d^2y}{dt^2}=-\frac{dy}{dz}+\frac{d^2y}{dz^2} \] τις οποίες αν αντικαταστήσουμε στην ομογενή εξίσωση αυτή θα γίνει:
\[ -\frac{dy}{dz}+\frac{d^2y}{dz^2}+2\frac{dy}{dz}-2y=0\Rightarrow \frac{d^2y}{dz^2}+\frac{dy}{dz}-2y=0\ ,\ z\in\mathbb{R} \]
Το χαρακτηριστικό πολυώνυμο αυτής είναι το $ P(\lambda)=\lambda^2+\lambda-2 $ το οποίο έχει ρίζες τις $ \lambda_1=1 $ και $ \lambda=-2 $. Έτσι οι συναρτήσεις $ y(z)=e^z\ ,\ z\in\mathbb{R} $ και $ e^{-2z}\ ,\ z\in\mathbb{R} $ αποτελούν ένα βασικό σύνολο λύσεων της εξίσωσης και άρα ο τύπος
\[ y(z)=c_1e^z+c_2e^{-2z}\ ,\ z\in\mathbb{R} \]
μας δίνει όλες τις λύσεις της, με $ c_1,c_2 $ να είναι αυθαίρετες σταθερές. Με τη βοήθεια του τελευταίου μετασχηματισμού η μη ομογενής εξίσωση θα πάρει τη μορφή
\begin{equation}\label{b21:1}
\frac{d^2y}{dz^2}+\frac{dy}{dz}-2y=e^z\ ,\ z\in\mathbb{R}
\end{equation}
οπότε αναζητούμε μια μερική λύση της μορφής $ y_\mu(z)=ue^z $. Παραγωγίζοντας έχουμε $ y'=(u'+u)e^z $ και $ y''=\left( u''+2u'+u\right)e^z $.  Έτσι η \eqref{b21:1} θα γίνει:
\begin{equation}\label{b21:2}
u''+2u'+u+u'+u-2u=1\Rightarrow u''+3u'=1
\end{equation}
Για την εύρεση μιας μερικής λύσης της \eqref{b21:1} θα έχουμε $ \forall z\in\mathbb{R}: u'_\mu(z)=a $ και έτσι θα προκύψει:
\[ 3a=1\Rightarrow a=\frac{1}{3} \]
Η μερική λύση της \eqref{b21:2} $ u_\mu(z)=\frac{1}{3}z $ μας δίνει τη μερική λύση της $ \eqref{b21:1} $ η οποία είναι η $ y_\mu(z)=\frac{1}{3}ze^z $. Αποκτάμε έτσι τον τύπο από τον οποίο δίνονται όλες οι λύσεις της διαφορικής εξίσωσης ο οποίος θα είναι:
\[ y(z)=c_1e^z+c_2e^{-2z}+\frac{1}{3}ze^z\ ,\ z\in\mathbb{R} \]
Ισοδύναμα ο τύπος αυτός ύστερα από αντικαταστάσεις θα γραφτεί στη μορφή:
\[ y(x)=c_1\left( 1+x^2\right) +\frac{c_2}{\left( 1+x^2\right)^2}+\frac{\left( 1+x^2\right)}{3}\log{\left( 1+x^2\right) }\ ,\ x>0 \]
όπου $ c_1,c_2 $ είναι αυθαίρετες σταθερές.
\epask
\begin{Askhshs}[B]
\bmath{Δίνεται η γραμμική διαφορική εξίσωση
\begin{equation}\label{b22:1}\tag{$ E $}
y''+p(x)y'+q(x)y=1+x
\end{equation}
καθώς και η αντίστοιχη ομογενής
\begin{equation}\label{b22:2}\tag{$ E_0 $}
y''+p(x)y'+q(x)y=0
\end{equation} όπου $ p $ και $ q $ είναι πραγματικές συναρτήσεις στο $ \mathbb{R} $. Ας υποθέσουμε ότι μια λύση της \eqref{b22:2} είναι η $ y_1(x)=\left( 1+x\right)^2 $ και ότι η ορίζουσα Wronski δύο οποιονδήποτε λύσεων της \eqref{b22:2}  είναι σταθερά. Να επιλυθεί η γραμμική διαφορική είσωση \eqref{b22:1}.}
\end{Askhshs}\mbox{}\\
\lysh
Έστω $ y_2(x)\ ,\ x\in\mathbb{R}-\{-1\} $ μια άλλη μερική λύση της \eqref{b22:2}. Τότε σύμφωνα με την υπόθεση, $ \forall x\in\mathbb{R}-\{-1\} $ θα ισχύει:
\begin{align}
W(y_1,y_2)(x)&=\begin{vmatrix}
y_1(x) & y_2(x)\\
y_1'(x) & y_2'(x)\\
\end{vmatrix}=c\Rightarrow \begin{vmatrix}
(1+x)^2 & y_2(x)\\
2(1+x) & y_2'(x)\\
\end{vmatrix}=c\Rightarrow\nonumber\\
&\Rightarrow (1+x)^2y_2'(x)-2(1+x)y_2(x)=c\ ,\ x\in\mathbb{R}-\{-1\}\label{b22:3}
\end{align}
όπου $ c $ είναι μια σταθερά. H \eqref{b22:3} αποτελεί μια γραμμική διαφορική εξίσωση 1\tss{ης} τάξης με άγνωστη συνάρτηση την $ y_2 $. Οι λύσεις της θα δίνονται από τον παρακάτω τύπο:
\begin{align*}
y_2(x)&=e^{\dintt{\frac{2(1+x)}{(1+x)^2}\d x}}\left[c'+\int{\frac{c}{(1+x)^2}e^{\dintt{-\frac{2(1+x)}{(1+x)^2}\d x}}\d x} \right]= (1+x)^2\left[ c'+\int{\frac{c}{(1+x)^2}\cdot\frac{1}{(1+x)^2}\d x}\right]=\\
&= (1+x)^2\left[ c'+\int{\frac{c}{(1+x)^4}\d x}\right]=(1+x)^2\left[ c'-\frac{c}{3(1+x)^3}\right]=c'(1+x)^2-\frac{c}{3(1+x)}\ ,\ x\neq-1
\end{align*}
όπου $ c' $ είναι μια αυθαίρετη σταθερά και $ c $ είναι η σταθερή ορίζουσα Wronski. Επιλέγοντας $ c'=0 $ παίρνουμε $ y_2(x)=-\frac{c}{3(1+x)}\ ,\ x\neq-1 $ ενώ για $ c=-3 $ έχουμε τη λύση $ y_2(x)=\frac{1}{1+x}\ ,\ x\neq-1 $ η οποία με την $ y_1(x)=\left(1+x\right)^2 $ είναι γραμμικά ανεξάρτητες και έτσι ορίζουν ένα βασικό σύνολο λύσεων της ομογενούς εξίσωσης. Οι λύσεις της θα δίνονται από τον τύπο:
\[ \tilde{y}(x)=c_1(1+x)^2-\frac{c_2}{1+x}\ ,\ x\neq-1 \]
Θα αναζητήσουμε μια μερική λύση της εξίσωσης \eqref{b22:1}. Η μέθοδος των αγνώστων σταθερών δεν μπορεί να εφαρμοστεί εδω διότι παρόλο που το δεύτερο μέλος της εξίσωσης είναι πολυώνυμο, δε γνωρίζουμε αν οι συναρτήσεις $ p,q $ είναι σταθερές. Θα εφαρμόσουμε τη μέθοδο της μεταβολής των σταθερών και έτσι θα έχουμε $ \forall x\in\mathbb{R}-\{-1\} $:
\begin{gather*}
W_1(y_1,y_2)=\begin{vmatrix}
0 & \frac{1}{1+x}\\
1 & -\frac{1}{(1+x)^2}\\
\end{vmatrix}=-\frac{1}{1+x}\ ,\  W_2(y_1,y_2)=\begin{vmatrix}
(1+x)^2 & 0\\
2(1+x) & 1\\
\end{vmatrix}=(1+x)^2\ \textrm{ και }\ 
W(y_1,y_2)(x)=-3
\end{gather*}
Έτσι η μερική λύση θα δίνεται από τον τύπο:
\begin{align*}
y_\mu(x)&=y_1(x)\int\frac{W_1(y_1,y_2)(x)}{W(y_1,y_2)(x)}\cdot\frac{b(x)}{a_2(x)}\d x+y_2(x)\int\frac{W_2(y_1,y_2)(x)}{W(y_1,y_2)(x)}\cdot\frac{b(x)}{a_2(x)}\d x\\
&=(1+x)^2\int{\frac{1}{3(1+x)}}\cdot(1+x)\d x-\frac{1}{1+x}\int\frac{(1+x)^2}{3}\cdot(1+x)\d x=\\
&=(1+x)^2\frac{x}{3}-\frac{1}{12}(1+x)^3=\frac{1}{12}(1+x)^2(3x-1)\ ,\ x\in\mathbb{R}-\{-1\}
\end{align*}
Καταλήγουμε λοιπόν στον τύπο από τον οποίο δίνονται όλες οι λύσεις της εξίσωσης ο οποίος θα είναι:
\[ y(x)=c_1(1+x)^2-\frac{c_2}{1+x}+\frac{1}{12}(1+x)^2(3x-1)\ ,\ x\in\mathbb{R}-\{-1\} \]
με $ c_1,c_2 $ να είναι αυθαίρετες σταθερές.\epask
\begin{Askhshs}[B]
\bmath{Έστω η τρίτης τάξης ομογενής γραμμική διαφορική εξίσωση
\begin{equation}\label{b23:1}\tag{$ E_0 $}
a_3y'''+a_2y''+a_1y'+a_0y=0
\end{equation}
όπου $ a_0,a_1,a_2 $ και $ a_3 $ είναι συνεχείς συναρτήσεις σε ένα διάστημα $ I $ της πραγματικής ευθείας, με $ a_3(x)\neq0 $ για όλα τα $ x\in I $. Ας είναι $ y_1 $ και $ y_2 $ δύο λύσεις της \eqref{b23:1} τέτοιες ώστε
\[ y_1(x)\neq0\ \textrm{ και }\ \left( \frac{y_2}{y_1}\right)'(x)\neq0\ ,\ \forall x\in I \] Να επιλυθεί η \eqref{b23:1} με αναγωγή αυτής σε μια πρώτης τάξης ομογενή γραμμική διαφορική εξίσωση.}
\end{Askhshs}\mbox{}\\
\lysh
Χρησιμοποιώντας τη λύση $ y_1(x) $ θα κάνουμε υποβιβασμό της τάξης της \eqref{b23:1} θέτοντας $ y=y_1u $ μιας και ισχύει $ y_1(x)\neq0\ ,\ \forall x\in I $. Με παραγώγιση θα έχουμε:
\begin{gather*}
y'=y_1'u+y_1u'\ ,\ y''=y_1''u+2y_1'u'+y_1u''\ \textrm{ και }\\
y'''=y_1'''u+3y_1''u+3y_1'u''+y_1u'''
\end{gather*}
Έτσι αντικαθιστώντας τις σχέσεις αυτές στην εξίσωση, αυτή θα γίνει:
\begin{gather}
a_3\left( y_1'''u+3y_1''u+3y_1'u''+y_1u'''\right) +a_2\left( y_1''u+2y_1'u'+y_1u''\right) +a_1\left( y_1'u+y_1u'\right) +a_0y_1u=0\Rightarrow\nonumber\\
a_3y_1'''u+3a_3y_1''u+3a_3y_1'u''+a_3y_1u''' +a_2y_1''u+2a_2y_1'u'+a_2y_1u'' + a_1y_1'u+a_1y_1u' +a_0y_1u=0\Rightarrow\nonumber\\
a_3y_1u'''+\left( 3a_3y_1'+a_2y_1\right)u''+\left( 3a_3y_1''+2a_2y_1'+a_1y_1\right)u'+\left(a_3y_1'''+a_2y_1''+a_1y_1'+a_0y_1\right)u=0\Rightarrow\nonumber\\
a_3y_1u'''+\left( 3a_3y_1'+a_2y_1\right)u''+\left( 3a_3y_1''+2a_2y_1'+a_1y_1\right)u'=0\label{b23:2}
\end{gather}
Θέτοντας στη \eqref{b23:2} όπου $ u'=\nu $ αυτή μετατρέπεται σε ομογενή γραμμική διαφορική εξίσωση 2\tss{ης} τάξης:
\begin{equation}\label{b23:3}
a_3y_1\nu''+\left( 3a_3y_1'+a_2y_1\right)\nu'+\left( 3a_3y_1''+2a_2y_1'+a_1y_1\right)\nu=0\ ,\ x\in I 
\end{equation}
με $ a_3(x)\neq0\ ,\ \forall x\in I $ και τις συναρτήσεις $ a_3y_1,3a_3y_1'+a_2y_1 $ και $ 3a_3y_1''+2a_2y_1'+a_1y_1 $ συνεχείς στο $ I $. Θέτουμε τώρα $ y_3=\frac{y_2}{y_1}\Rightarrow y_2=y_3y_1\ ,\ x\in I $ και έτσι θα προκύψει
\begin{gather*}
y'=y_1'y_3+y_1y_3'\ ,\ y''=y_1''y_3+2y_1'y_3'+y_1y_3''\ \textrm{ και }\\
y'''=y_1'''y_3+3y_1''y_3+3y_1'y_3''+y_1y_3'''
\end{gather*}
Η $ y_2 $ όμως είναι λύση της \eqref{b23:1} και έτσι εκτελώντας τον παραπάνω μετασχηματισμό σ' αυτήν παίρνουμε:
\begin{gather}
a_3\left( y_1'''y_3+3y_1''y_3+3y_1'y_3''+y_1y_3'''\right) +a_2\left( y_1''y_3+2y_1'y_3'+y_1y_3''\right) +a_1\left( y_1'y_3+y_1y_3'\right) +a_0y_1y_3=0\Rightarrow\nonumber\\
a_3y_1'''y_3+3a_3y_1''y_3+3a_3y_1'y_3''+a_3y_1y_3''' +a_2y_1''y_3+2a_2y_1'y_3'+a_2y_1y_3'' + a_1y_1'y_3+a_1y_1y_3' +a_0y_1y_3=0\Rightarrow\nonumber\\
a_3y_1y_3'''+\left( 3a_3y_1'+a_2y_1\right)y_3''+\left( 3a_3y_1''+2a_2y_1'+a_1y_1\right)y_3'+\left(a_3y_1'''+a_2y_1''+a_1y_1'+a_0y_1\right)y_3=0\Rightarrow\nonumber\\
a_3y_1y_3'''+\left( 3a_3y_1'+a_2y_1\right)y_3''+\left( 3a_3y_1''+2a_2y_1'+a_1y_1\right)y_3'=0\label{b23:4}
\end{gather}
Παρατηρούμε τώρα ότι η συνάρτηση $ y_3'(x)=\left(\frac{y_2}{y_1}\right)'(x)\ ,\ x\in I $ αποτελέι λύση της \eqref{b23:3} αφού είναι:
\begin{gather*}
a_3y_1(y_3')''+\left( 3a_3y_1'+a_2y_1\right)(y_3')'+\left( 3a_3y_1''+2a_2y_1'+a_1y_1\right)y_3'=0\Rightarrow\\
a_3y_1y_3'''+\left( 3a_3y_1'+a_2y_1\right)y_3''+\left( 3a_3y_1''+2a_2y_1'+a_1y_1\right)y_3'=0
\end{gather*}
κάτι που ισχύει λόγων της \eqref{b23:4}. Από την υπόθεση επίσης γνωρίζουμε ότι $ y_3'(x)=\left(\frac{y_2}{y_1}\right)'(x)\neq0\ ,\ x\in I $ και έτσι κάνουμε υποβιβασμό τάξης στην εξίσωση \eqref{b23:3} χρησιμοποιώντας το μετασχηματισμό $ \nu=y_3'z $. Για κάθε $ x\in I $ θα έχουμε:
\[ \nu'=y_3''z+y_3'z'\ ,\ \nu''=y_3'''z+2y_3''z'+y_3'z'' \]
και έτσι η εξίσωση θα γίνει:
\begin{gather}
a_3y_1\left( y_3'''z+2y_3''z'+y_3'z''\right)+\left( 3a_3y_1'+a_2y_1\right)\left( y_3''z+y_3'z'\right)+\left( 3a_3y_1''+2a_2y_1'+a_1y_1\right)y_3'z=0\Rightarrow\nonumber
\end{gather}
\vspace{-8mm}
\begin{multline*}
a_3y_1y_3'''z+2a_3y_1y_3''z'+a_3y_1y_3'z''+3a_3y_1'y_3''z+3a_3y_1'y_3'z'+a_2y_1y_3''z+a_2y_1y_3'z'+\\+ 3a_3y_1''y_3'z+2a_2y_1'y_3'z+a_1y_1y_3'z=0\Rightarrow
\end{multline*}
\vspace{-11mm}
\begin{multline*}
a_3y_1y_3'z''+\left( 2a_3y_1y_3''+3a_3y_1'y_3'+a_2y_1y_3'\right)z'+\\\left[ a_3y_1y_3'''+\left( 3a_3y_1'+a_2y_1\right)y_3''+\left( 3a_3y_1''+2a_2y_1'+a_1y_1\right)y_3'\right]z=0\Rightarrow
\end{multline*}
\begin{equation*}
a_3y_1y_3'z''+\left( 2a_3y_1y_3''+3a_3y_1'y_3'+a_2y_1y_3'\right)z'=0
\end{equation*}
Θέτοντας στην τελευταία εξίσωση $ z'=w $ καταλήγουμε στην πρώτης τάξης ομογενή διαφορική εξίσωση:
\[ a_3y_1y_3'w'+\left( 2a_3y_1y_3''+3a_3y_1'y_3'+a_2y_1y_3'\right)w=0 \]
της οποίας οι λύσεις θα δίνονται από τον τύπο:
\begin{align*}
w(x)&=c_1e^{-\dints{\frac{2a_3y_1y_3''+3a_3y_1'y_3'+a_2y_1y_3'}{a_3y_1y_3'}\d x}}=\\&=c_1e^{-\dints{\left[ \frac{2y_3''}{y_3'}+\frac{3y_1'}{y_1}+\frac{a_2}{a_3}\right] \d x}}=c_1e^{-\left[2\log{(y_3')}+3\log{y_1}+\frac{a_2}{a_3}x \right] }=\frac{c_1}{\left(y_3'\right)^2\cdot y_1^3\cdot e^{\frac{a_2}{a_3}x} }\ ,\ x\in I
\end{align*}
Αντικαθιστώντας διαδοχικά, Θα καταλήξουμε στη λύση $ y(x) $ της αρχικής διαφορικής εξίσωσης. Θα έχουμε λοιπόν:
\begin{gather*}
z(x)=\xl\int{\frac{c_1}{\left(y_3'\right)^2\cdot y_1^3\cdot e^{\frac{a_2}{a_3}x} }\d x}+c_2\Rightarrow \nu(x)=y_3'\xl\int{\frac{c_1}{\left(y_3'\right)^2\cdot y_1^3\cdot e^{\frac{a_2}{a_3}x} }\d x}+c_2y_3'\Rightarrow\\
u(x)=\xl\int{\left( y_3'\xl\int{\frac{c_1}{\left(y_3'\right)^2\cdot y_1^3\cdot e^{\frac{a_2}{a_3}x} }\d x}+c_2y_3'\right) \d x}+c_3\ ,\ x\in I
\end{gather*}
Έτσι οι λύσεις θα δίνονται από τον τύπο:
\[ y(x)=c_1y_1(x)\XL\int{\PARENS{\left( \frac{y_2}{y_1}\right)'(x)\XL\int{\frac{\vphantom{\left[\left( \frac{y_2}{y_1}\right)'(x)\right]^2\cdot y_1^3(x)\cdot e^{\frac{a_2}{a_3}x}}1}{\left[\left( \frac{y_2}{y_1}\right)'(x)\right]^2\cdot y_1^3(x)\cdot e^{\frac{a_2}{a_3}x} }\d x}} \d x}+c_2y_2(x)+c_3y_1(x)\ ,\ x\in I \] 
όπου $ c_1,c_2,c_3 $ είναι αυθαίρετες σταθερές.\epask
\begin{Askhshs}[B]
\bmath{Ας είναι $ \{y_1,y_2\} $ ένα βασικό σύνολο πραγματικών λύσεων μιας ομογενούς γραμμικής διαφορικής εξίσωσης δεύτερης τάξης με διάστημα ορισμού το $ (-\infty,+\infty) $. Να αποδειχθεί ότι μεταξύ δύο διαδοχικών ριζών της $ y_1 $ υπάρχει μια ακριβώς ρίζα της $ y_2 $.}
\end{Askhshs}\mbox{}\\
\lysh
Έστω $ x_1,x_2\in(-\infty,+\infty) $ δύο διαδοχικές ρίζες της $ y_1 $. Τότε θα έχουμε $ y_1(x_1)=y_1(x_2)=0 $. Θεωρούμε ότι $ x_1<x_2 $ οπότε από την υπόθεση θα έχουμε $ y_1(x)\neq0,\ \forall x\in (x_1,x_2) $. Θα ισχυριστούμε ότι η λύση $ y_2 $ δε μηδενίζεται πουθενά στο διάστημα $ (x_1,x_2) $. Έστω λοιπόν $ y_2(x)\neq0,\ \forall x\in(x_1,x_2) $. Επιπλέον αν $ y_2(x_1)=0 $ τότε θα έχουμε:
\[ W(y_1,y_2)(x_1)=\begin{vmatrix}
0 & 0\\y_1'(x_1) & y_2'(x_1)
\end{vmatrix}=0 \]
πράγμα άτοπο από την υπόθεση. Άρα $ y_2(x_1)\neq0 $ και ομοίως προκύπτει και $ y_2(x_2)\neq0 $. Έτσι ο αρχικός ισχυρισμός θα γίνει $ y_2(x)\neq0,\ \forall x\in [x_1,x_2] $. Ορίζουμε τη συνάρτηση $ \frac{y_1}{y_2}(x) $ η οποία είναι:
\begin{multicols}{2}
\begin{rlist}
\item συνεχής στο διάστημα $ [x_1,x_2] $
\item παραγωγίσιμη στο διάστημα $ (x_1,x_2) $ και
\item $ \frac{y_1}{y_2}(x_1)=\frac{y_1}{y_2}(x_2)=0 $
\end{rlist}
\end{multicols}
Έτσι σύμφωνα με το θεώρημα του Rolle θα υπάρχει $ \xi_1\in(x_1,x_2) $ τέτοιο ώστε να ισχύει:
\[ \left( \frac{y_1}{y_2}\right)'(\xi)=0\Rightarrow \frac{y_1'(\xi)y_2(\xi)-y_1(\xi)y_2'(\xi)}{y_2^2(\xi)}=0\Rightarrow\frac{-W(y_1,y_2)(\xi)}{y_2^2(\xi)}=0\Rightarrow W(y_1,y_2)(\xi)=0 \]
Άτοπο, αφού από την υπόθεση γνωρίζουμε ότι οι συναρτήσεις $ y_1,y_2 $ είναι γραμμικά ανεξάρτητες και κατά συνέπεια θα ισχύει $ W(y_1,y_2)(x)\neq0,\ \forall x\in[x_1,x_2] $. Έτσι καταλλήγουμε στο συμπέρασμα ότι η $ y_2 $ έχει τουλάχιστον μια λύση στο διάστημα $ (x_1,x_2) $. Μένει να δείξουμε ότι η ρίζα αυτή είναι και μοναδική. Θεωρούμε ότι έχει δύο ρίζες έστω $ x_3,x_4\in(x_1,x_2) $ με $ x_3<x_4 $. Εφαρμόζοντας το θεώρημα Rolle στο διάστημα $ [x_3,x_4] $ για τη συνάρτηση $ \frac{y_2}{y_1}(x) $ έχουμε ότι η συνάρτηση είναι:
\begin{multicols}{2}
\begin{rlist}
\item συνεχής στο διάστημα $ [x_3,x_4] $
\item παραγωγίσιμη στο διάστημα $ (x_3,x_4) $ και
\item $ \frac{y_2}{y_1}(x_3)=\frac{y_2}{y_1}(x_4)=0 $
\end{rlist}
\end{multicols}
Έτσι θα υπάρχει τουλάχιστον ένα $ \xi'\in(x_3,x_4)\subseteq(x_1,x_2) $ τέτοιο ώστε να ισχύει:
\[ \left( \frac{y_2}{y_1}\right)'(\xi')=0\Rightarrow \frac{y_2'(\xi')y_1(\xi')-y_2(\xi')y_1'(\xi')}{y_1^2(\xi')}=0\Rightarrow\frac{W(y_1,y_2)(\xi')}{y_2^2(\xi')}=0\Rightarrow W(y_1,y_2)(\xi')=0 \]
Άτοπο, μιας και ισχύει $ W(y_1,y_2)(x)\neq0,\ \forall x\in[x_1,x_2] $. Συμπεραίνουμε λοιπόν ότι η ρίζα αυτή είναι και μοναδική.\epask
\begin{Askhshs}[B]
 \bmath{Έστω η διαφορική εξίσωση $ y''+ay=0 $, όπου $ a $ είναι μια συνεχής συνάρτηση στο διάστημα $ (x_1,x_2) $ με $ -\infty\leq x_1<x_2\leq+\infty $. Να αποδειχθεί ότι οι ρίζες μιας μη μηδενικής λύσης της παραπάνω εξίσωσης είναι μεμονωμένες (δηλαδή κάθε σημείο του συνόλου των ριζών δεν είναι σημείο συσσώρευσης). [\textit{Υπόδειξη:} Να αποδειχθεί πρώτα, ότι για κάθε ρίζα $ x^* $ μιας μη μηδενικής λύσης $ y $ ισχύει $ y'(x^*)\neq0 $.]}
\end{Askhshs}\mbox{}\\
\lysh
Έστω $ y(x) $ μια μη-μηδενική λύση της $ (E_0): y''(x)+a(x)y(x)=0 $. Υποθέτουμε ότι οι ρίζες της $ y(x) $ έχουν σημείο συσσώρευσης το $ x_0\in\mathbb{R} $. Τότε θα υπάρχει ακολουθία ριζών $ (x_n)_{n\in\mathbb{N}} $ της $ y(x) $ με $ x_n<x_{n+1}\ ,\forall n\in\mathbb{N} $ και $ \lim\limits_{n\to\infty}{x_n}=x_0 $. Η $ y(x) $ ως συνεχής λύση της $ E_0 $ θα ικανοποιεί τη σχέση
\[ \lim_{n\to\infty}{y(x_n)}=y(x_0) \]
Όμως οι $ (x_n)_{n\in\mathbb{N}} $ είναι ρίζες της $ y(x) $ που σημαίνει ότι 
\[ y(x_n)=0\ ,\ \forall n\in\mathbb{N} \]
Από τη μοναδικότητα του ορίου μιας ακολουθίας έχουμε τελικά ότι $ y(x_0)=0 $ (1). Εφαρμόζοντας το θεώρημα Rolle για τη συνάρτηση $ y(x) $ στα διαστήματα $ [x_n,x_{n+1}] $ έχουμε ότι είναι συνεχής σ' αυτά, παραγωγίσιμη στα αντίστοιχα ανοιχτά διαστήματα $ (x_n,x_{n+1}) $ και $ y(x_n)=y(x_{n+1})=0 $. Επομένως $ \exists\xi_{n}\in(x_n,x_{n+1}) $ τέτοιο ώστε
\[ y'(\xi_{n})=0\ ,\ \forall n\in\mathbb{N}\Rightarrow \lim_{n\to\infty}{y'(\xi_n)}=0\ ,\ \forall n\in\mathbb{N} \]
Εφόσον $ x_n<\xi_{n}<x_{n+1}\ ,\ \forall n\in\mathbb{N} $ και $ \lim\limits_{n\to\infty}{x_n}=\lim_{n\to\infty}{x_{n+1}}=x_0 $ τότε από το κριτήριο παρεμβολής λαμβάνουμε ότι 
\[ \lim_{n\to\infty}{\xi_n}=x_0 \]
Τέλος επειδή η $ y(x) $ είναι λύση της $ (E_0) $ συνεχής και παραγωγίσιμη με συνεχή 1η παράγωγο τότε θα ισχύει
\begin{equation}
\lim_{n\to\infty}{y'(\xi_n)}=y'(x_0)\xRightarrow{y'(\xi_n)=0,\forall n\in\mathbb{N}} y'(x_0)=0\label{(2)}
\end{equation}
Από τις (1),(2) και λόγω του ότι η $ y(x) $ είναι λύση της $ (E_0) $ τότε σύμφωνα με το θεώρημα ύπαρξης και μοναδικότητας λύσεων του Π.Α.Τ για γραμμικές Σ.Δ.Ε. προκύπτει ότι
\[ y(x)\equiv0\ \ ,\ \ \text{άτοπο} \]
Επομένως οι ρίζες της $ y(x) $ είναι μεμονωμένες. ο.ε.δ.
\epask
\begin{Askhshs}[B]
 \bmath{Έστω η ομογενής γραμμική διαφορική εξίσωση
\begin{equation}\label{b26:1}\tag{$ E_0 $}
y''+py'+qy=0
\end{equation} όπου $ p $ και $ q $ είναι συνεχείς συναρτήσεις σε ένα διάστημα $ (a,\beta) $ με $ -\infty\leq a<\beta\leq +\infty $. Ας είναι $ x_0\in(a,\beta) $ και $ y_1,y_2 $ δύο λύσεις της \eqref{b26:1}. Να αποδειχθεί ότι αν $ y_1(x_0)=0 $ και $ W(y_1,y_2)(x_0)=0 $, τότε είτε $ y_1=0 $ είτε $ y_2=\frac{y_2'(x_0)}{y_1'(x_0)}\cdot y_1 $. }
\end{Askhshs}\mbox{}\\
\lysh
Από την υπόθεση γνωρίζουμε ότι ισχύει $ W(y_1,y_2)(x_0)=0 $ και έτσι θα έχουμε:
\[ W(y_1,y_2)(x_0)=0\Rightarrow \begin{vmatrix}
0 & y_2(x_0)\\ y_1'(x_0) & y_2'(x_0)
\end{vmatrix}=0\Rightarrow y_1'(x_0)\cdot y_2(x_0)=0 \]
Διακρίνουμε τις εξής περιπτώσεις:
\begin{rlist}
\item\label{i1} Αν $ y_1'(x_0)=0 $ τότε σύμφωνα με το θεώρημα ύπαρξης και μονοσήμαντου των λύσεων των προβλημάτων αρχικών τιμών για γραμμικές διαφορικές εξισώσεις και αφού από την υπόθεση η εξίσωσή μας είναι ομογενής και ισχύει $ y_1(x_0) $ τότε παίρνουμε ότι $ y_1=0,\ \forall x\in(a,\beta) $.
\item Αν $ y_2(x_0)=0 $ και $ y_1'(x_0)\neq0 $ τότε ορίζουμε τη συνάρτηση $ u=\frac{y_2'(x_0)}{y_1'(x_0)}\cdot y_1\ ,\ x_0,x\in(a,\beta) $ και εύκολα παρατηρούμε ότι η $ u $ είναι λύση της \eqref{b26:1}. Επιπλέον έχουμε ότι:
\[ u(x_0)=y_2(x_0)=0\ \textrm{ και }\ u'(x_0)=\frac{y_2'(x_0)}{y_1'(x_0)}\cdot y_1'(x_0)=y_2'(x_0) \]
Ξανά λοιπόν από το θεώρημα ύπαρξης και μονοσήμαντου των λύσεων των προβλημάτων αρχικών τιμών θα έχουμε ότι οι συναρτήσεις $ u,y_2 $ ταυτίζονται μιας και πληρούν τις ίδιες αρχικές συνθήκες. Άρα 
\[ y_2=\frac{y_2'(x_0)}{y_1'(x_0)}\cdot y_1\ ,\ x\in(a,\beta) \]
\item Αν $ y_2(x_0)=0 $ και $ y_1'(x_0)=0 $ τότε επαγόμαστε στην περίπτωση \ref{i1}.
\end{rlist}\mbox{}
\epask
\begin{Askhshs}[B]
 \bmath{Έστω $ I $ ένα διάστημα της πραγματικής ευθείας και έστω $ x_0 $ ένα σημείο του $ I $. Ας είναι $ y_1,y_2,\ldots,y_m $ λύσεις μια $ n- $τάξης ονογενούς γραμμικής διαφορικής εξίσωσης με διάστημα ορισμού το $ I $. Να αποδειχθεί ότι οι λύσεις $ y_k,\ (k=1,2,\ldots,m) $ είναι γραμμικά ανεξάρτητες αν και μόνο αν τα διανύσματα
\[ \PARENS{\begin{gathered}
y_k(x_0)\\y_k'(x_0)\\\vdots\\y_k^{(m-1)}(x_0)
\end{gathered}}\ ,\ (k=1,2,\ldots,m) \]
είναι γραμμικά ανεξάρτητα.}
\end{Askhshs}\mbox{}\\
\lysh
Αρκεί να δείξουμε ότι ισχύει η ακόλουθη ισοδυναμία: Οι συναρτήσεις $ y_1,y_2,\ldots,y_m $ είναι γραμμικά εξαρτημένες αν και μόνο αν τα διανύσματα \[ \left( y_k^{(i)}(x_0)\right)=\PARENS{\begin{gathered}
y_k(x_0)\\y_k'(x_0)\\\vdots\\y_k^{(m-1)}(x_0)
\end{gathered}}\ ,\ (k=1,2,\ldots,m),\ (i=0,1,2,\ldots,m-1) \] είναι γραμμικά εξαρτημένα.\\
\bmath{Ορθό $ \Rightarrow $}\\
Θεωρούμε ότι οι συναρτήσεις $ y_1,y_2,\ldots,y_m $ είναι γραμμικά εξαρτημένες και άρα θα υπάρχουν $ m $ σε πλήθος σταθερές $ c_1,c_2,\ldots,c_m $, όχι όλες μηδενικές, έτσι ώστε να ισχύει:
\[ c_1y_1+c_2y_2+\ldots+c_my_m=0 \]
Παραγωγίζοντας την παραπάνω σχέση $ m-1 $ φορές αποκτάμε το παρακάτω σύστημα γραμμικών διαφορικών εξισώσεων:
\[ \ccases{c_1y_1+c_2y_2+\ldots+c_my_m=0\\
c_1y_1'+c_2y_2'+\ldots+c_my_m'=0\\
\ \ \ \vdots\quad\quad\quad\vdots\qquad\qquad\quad\ \ \vdots\\
c_1y_1^{(m-1)}+c_2y_2^{(m-1)}+\ldots+c_my_m^{(m-1)}=0} \]
στο οποίο θέτοντας όπου $ x=x_0 $, ισοδύναμα γράφεται στη μορφή:
\begin{gather*}
c_1\PARENS{\begin{gathered}
y_1(x_0)\\y_1'(x_0)\\\vdots\\y_1^{(m-1)}(x_0)
\end{gathered}}+c_2\PARENS{\begin{gathered}
y_2(x_0)\\y_2'(x_0)\\\vdots\\y_2^{(m-1)}(x_0)
\end{gathered}}+\ldots+c_m\PARENS{\begin{gathered}
y_m(x_0)\\y_m'(x_0)\\\vdots\\y_m^{(m-1)}(x_0)
\end{gathered}}=\vec{0}\Rightarrow\\
c_1\left( y_1^{(i)}(x_0)\right)+c_2\left( y_2^{(i)}(x_0)\right)+\ldots+c_m\left( y_m^{(i)}(x_0)\right)=\vec{0}\quad,\ i=0,1,\ldots,m-1
\end{gather*}
Έτσι, αφού οι σταθερές $ c_1,c_2,\ldots,c_m $ δεν είναι όλες ίσες με το $ 0 $ τότε προκύπτει ότι τα διανύσματα $ \left( y_k^{(i)}(x_0)\right),\ k=1,2,\ldots,m $ είναι γραμμικά ανεξάρτητα.\\\\
\bmath{Αντίστροφο $ \Leftarrow $}\\
Έστω ότι τα διανύσματα $ \left( y_k^{(i)}(x_0)\right),\ k=1,2,\ldots,m $ είναι γραμμικά ανεξάρτητα. Τότε θα υπάρχουν $ m $ σε πλήθος σταθερές $ c_1,c_2,\ldots,c_m $, όχι όλες μηδενικές έτσι ώστε να ισχύει:
\[ c_1\left( y_1^{(i)}(x_0)\right)+c_2\left( y_2^{(i)}(x_0)\right)+\ldots+c_m\left( y_m^{(i)}(x_0)\right)=\vec{0}\quad,\ i=0,1,\ldots,m-1 \]
Τότε η συνάρτηση $ y=c_1y_1+c_2y_2+\ldots+c_my_m $ θα είναι μια λύση της ομογενούς εξίσωσης. Η λύση όμως αυτή πληροί της αρχικές συνθήκες $ y(x_0)=y'(x_0)=\ldots=y^{(m-1)}(x_0)=0 $. Άρα σύμφωνα με το θεώρημα της ύπαρξης και μονοσήμαντου των λύσεων των προβλημάτων αρχικών τιμών για γραμμικές διαφορικές εξισώσεις, η λύση αυτή θα είναι η $ y=0 $. Συνεπώς θα ισχύει:
\[ c_1y_1+c_2y_2+\ldots+c_my_m=0 \]
και αφού οι σταθερές $ c_1,c_2\ldots,c_m $ δεν είναι όλες μηδενικές, τότε οι συναρτήσεις $ y_1,y_2,\ldots,y_m $ θα είναι γραμμικά εξαρτημένες.\epask
\begin{Askhshs}[B]
 \bmath{Ας είναι $ a,\omega $ και $ c $ σταθερές με $ 0\leq a<\omega $ και $ c>0 $. Να επιλυθεί η γραμμική διαφορική εξίσωση
\[ y''+2ay'+\omega^2y=c\sin{\omega x} \]}
\end{Askhshs}\mbox{}\\
\lysh
Ξεκινάμε διακρίνοντας τις εξής περιπτώσεις για τη σταθερά $ a $:
\begin{rlist}
\item Έστω $ a>0 $. Τότε η αντίστοιχη ομογενής εξίσωση της αρχικής θα είναι η 
\[ y''+2ay'+\omega^2y=0 \]
Χαρακτηριστικό πολυώνυμο αυτής είναι το $ P(\lambda)=\lambda^2+2a\lambda+\omega^2 $ με ρίζες τις $ \lambda_1=-a+i\sqrt{\omega^2-a^2} $ και $ \lambda_2=-a-i\sqrt{\omega^2-a^2} $. Έτσι οι συναρτήσεις $ y_1=e^{-ax}\cos{\delta x},\ x\in\mathbb{R} $ και $ y_2=e^{-ax}\sin{\delta x},\ x\in\mathbb{R} $ αποτελούν ένα βασικό σύνολο λύσεων της ομογενούς εξίσωσης, οπου $ \delta=\sqrt{\omega^2-a^2} $ και έτσι οι λύσεις της θα είναι
\[ \tilde{y}(x)=c_1e^{-ax}\cos{\delta x}+c_2e^{-ax}\sin{\delta x}\ ,\ x\in\mathbb{R} \]
όπου $ c_1,c_2 $ είναι αυθαίρετες σταθερές. Για την εύρεση μιας μερικής λύσης της μη ομογενούς διαφορικής εξίσωσης θεωρούμε την εξίσωση
\begin{equation}\label{b28:1}
\nu''+2a\nu+\omega^2\nu=ce^{i\omega x}
\end{equation}
στην οποία θέτουμε $ \nu=ze^{i\omega x} $. Παραγωγίζοντας το μετασχηματισμό αυτό θα έχουμε $ \nu'=\left( z'+i\omega z\right) e^{i\omega x} $ και $ \nu''=\left( z''+2i\omega z'-\omega ^2z\right)e^{i\omega x} $ και έτσι η παραπάνω εξίσωση θα πάρει τη μορφή:
\[ z''+2i\omega z'-\omega ^2z+2a\left( z'+i\omega z\right)+\omega^2z=c\Rightarrow z''+2(a+i\omega) z'+2ai\omega z=c \]
Μια μερική λύση της τελευταίας εξίσωσης θα είναι της μορφής $ z_\mu=\beta $ η οποία θα μας δώσει $ \beta=-i\frac{c}{2a\omega} $. Η ζητούμενη μερική λύση της αρχικής εξίσωσης θα είναι το φανταστικό μέρος της μερικής λύσης της \eqref{b28:1} δηλαδή της $ \nu_\mu=-i\frac{c}{2a\omega}e^{i\omega x} $:
\[ y_\mu(x)=\mathrm{Im}\left( \nu_{\mu}(x)\right)=-\frac{c}{2a\omega}\cos(\omega x)\ ,\ x\in\mathbb{R} \]
Όλες οι λύσεις λοιπόν της αρχικής μη ομογενούς γραμμικής διαφορικής εξίσωσης θα δίνονται από τον τύπο
\[ y(x)=c_1e^{-ax}\cos{\delta x}+c_2e^{-ax}\sin{\delta x}-\frac{c}{2a\omega}\cos(\omega x)\ ,\ x\in\mathbb{R} \]
όπου $ \delta=\sqrt{\omega^2-a^2} $ και $ c_1,c_2 $ είναι αυθαίρετες σταθερές.
\item Αν τώρα $ a=0 $ η αρχική διαφορική εξίσωση θα γίνει \[ y''+\omega^2y=c\sin{\omega x} \]
Η αντίστοιχη ομογενής εξίσωση θα είναι $ y''+\omega^2y=0 $ με χαρακτηριστικό πολυώνυμο το $ P(\lambda)=\lambda^2+\omega^2 $. Οι ρίζες αυτού είναι οι $ \lambda_1=i\omega $ και $ \lambda_2=-i\omega $ και έτσι οι συναρτήσεις $ y_1=\cos{\omega x},\ x\in\mathbb{R} $ και $ y_2=\sin{\omega x},\ x\in\mathbb{R} $ αποτελούν ένα βασικό σύνολο λύσεων της ομογενούς εξίσωσης. Όλες οι λύσεις θα δίνονται από τον τύπο:
\[ \tilde{y}(x)=c_1\cos{\omega x}+c_2\sin{\omega x}\ ,\ x\in\mathbb{R} \]
όπου $ c_1,c_2 $ είναι αυθαίρετες σταθερές. Για την εύρεση μιας μερικής λύσης θεωρούμε τη διαφορική εξίσωση
\begin{equation}\label{b28:2}
\nu''+\omega^2\nu=ce^{i\omega x}\ ,\ x\in\mathbb{R}
\end{equation}
και θέτοντας όπου $ \nu=ze^{i\omega x} $ παίρνουμε $ \nu'=\left( z'+i\omega z\right) e^{i\omega x} $ και $ \nu''=\left( z''+2i\omega z'-\omega ^2z\right)e^{i\omega x} $. Αντικαθιστώντας στην παραπάνω εξίσωση τις συναρτήσεις αυτές, αυτή θα γίνει:
\begin{equation}\label{b28:3}
z''+2i\omega z'-\omega ^2z+\omega^2z=c\Rightarrow z''+2i\omega z'=c\ ,\ x\in\mathbb{R}
\end{equation}
Για τη μερική λύση $ z_\mu $ θα ισχύει $ z'=a\Rightarrow z''=0 $ και έτσι παίρνουμε
\[ 2i\omega a=c\Rightarrow a=-i\frac{c}{2\omega}\ ,\ x\in\mathbb{R} \]
Άρα η μερική λύση της \eqref{b28:3} θα είναι η $ z_\mu=-\frac{c}{2\omega}ix,\ x\in\mathbb{R} $ οπότε
\[ \nu_\mu(x)=-\frac{c}{2\omega}ixe^{i\omega x}=\frac{c}{2\omega}x\sin{\omega x}-i\frac{c}{2\omega}x\cos{\omega x},\ x\in\mathbb{R} \]
Η μερική λύση της αρχικής μη ομογενούς διαφορικής εξίσωσης θα είναι το φανταστικό μέρος της $ \nu_\mu $ και έτσι θα έχουμε:
\[ y_\mu(x)=\mathrm{Im}\left( \nu_{\mu}(x)\right) =-\frac{c}{2\omega}x\cos{\omega x},\ x\in\mathbb{R} \]
Όλες οι λύσεις της θα δίνονται από τον παρακάτω τύπο:
\[ y(x)=c_1\cos{\omega x}+c_2\sin{\omega x}-\frac{c}{2\omega}x\cos{\omega x},\ x\in\mathbb{R} \]
όπου $ \delta=\sqrt{\omega^2-a^2} $ και $ c_1,c_2 $ είναι αυθαίρετες σταθερές.
\end{rlist}
Συνοψίζοντας, ο γενικός τύπος για τις λύσεις της αρχικής μη ομογενούς γραμμικής διαφορικής εξίσωσης, για τις διάφορες τιμές της σταθεράς $ a $ θα είναι:
\[ y(x)=\ccases{c_1e^{-ax}\cos{\delta x}+c_2e^{-ax}\sin{\delta x}-\frac{c}{2a\omega}\cos\omega x & \textrm{με }0<a<\omega\ \textrm{και }c>0\\
c_1\cos{\omega x}+c_2\sin{\omega x}-\frac{c}{2\omega}x\cos{\omega x} & \textrm{με }a=0\ \textrm{και } \omega,c>0} \]
όπου $ x\in\mathbb{R}, \delta=\sqrt{\omega^2-a^2} $ και $ c_1,c_2 $ αυθαίρετες σταθερές.\epask
\begin{Askhshs}[B]
 \bmath{Έστω η ομογενής γραμμική διαφορική εξίσωση
\begin{equation}\label{b29:1}\tag{*}
\sum_{k=0}^{5}{y^{(k)}}=0
\end{equation}
\begin{rlist}
\item Να βρεθεί ένα βασικό σύνολο λύσεων της \eqref{b29:1}.
\item Να αποδειχθεί ότι το βασικό σύνολο όλων των πραγματικών λύσεων της \eqref{b29:1}, οι οποίες τέινουν προς το $ 0 $ για $ x\to\infty $, είναι ένας γραμμικός χώρος επί του $ \mathbb{R} $ και στη συνέχεια να βρεθεί μια βάση αυτού.
\end{rlist}}
\end{Askhshs}\mbox{}\\
\lysh
\begin{rlist}
\item Η αρχική διαφορική εξίσωση \eqref{b29:1} ισοδύναμα μπορεί να γραφτεί στη μορφή:
\[ y^{(5)}+y^{(4)}+y'''+y''+y'+y=0 \]
Το χαρακτηριστικό πολυώνυμό της θα είναι το $ P(\lambda)=\lambda^5+\lambda^4+\lambda^3+\lambda^2+\lambda+1 $, το οποίο αποτελεί άθροισμα όρων γεωμετρικής προόδου με πρώτο όρο τον $ a_1=1 $ και λόγο $ \lambda $. Επομένως μπορεί να γραφτεί:
\[ P(\lambda)=\lambda^5+\lambda^4+\lambda^3+\lambda^2+\lambda+1=\frac{\lambda^6-1}{\lambda-1}=(\lambda+1)\left( \lambda^2+\lambda+1\right)\left( \lambda^2-\lambda+1\right) \]
Οι ρίζες του πολυωνύμου θα είναι οι εξής:
\[ \lambda_1=-1\ ,\ \lambda_2=\frac{-1+i\sqrt{3}}{2}\ ,\ \lambda_3=\frac{-1-i\sqrt{3}}{2}\ ,\ \lambda_4=\frac{1+i\sqrt{3}}{2}\ ,\ \lambda_5=\frac{1-i\sqrt{3}}{2} \] κατά συνέπεια οι συναρτήσεις $ y_1(x)=e^{-x}\ ,\ y_2(x)=e^{-\frac{x}{2}}\cos{\left( \frac{\sqrt{3}}{2}x\right)}\ ,\ y_3(x)=e^{-\frac{x}{2}}\sin{\left( \frac{\sqrt{3}}{2}x\right)}\ ,\ y_4(x)=e^{\frac{x}{2}}\cos{\left( \frac{\sqrt{3}}{2}x\right)} $ και $ y_5(x)=e^{\frac{x}{2}}\sin{\left( \frac{\sqrt{3}}{2}x\right)} $ με $ x\in\mathbb{R} $ αποτελούν ένα βασικό σύνολο \textbf{πραγματικών} λύσεων της ομογενούς εξίσωσης. Συνεπώς όλες οι λύσεις της θα δίνονται από τον τύπο:
\[ y(x)=c_1e^{-x}+c_2e^{-\frac{x}{2}}\cos{\left( \frac{\sqrt{3}}{2}x\right)}+c_3e^{-\frac{x}{2}}\sin{\left( \frac{\sqrt{3}}{2}x\right)}+c_4e^{\frac{x}{2}}\cos{\left( \frac{\sqrt{3}}{2}x\right)}+c_5e^{\frac{x}{2}}\sin{\left( \frac{\sqrt{3}}{2}x\right)} \]
όπου $ x\in\mathbb{R} $ και $ c_i,\ i=1,2,\ldots,5 $ είναι αυθαίρετες σταθερές.
\item Έστω $ \mathcal{L} $ το σύνολο των πραγματικών λύσεων της \eqref{b29:1} και $ \mathcal{L}_0=\left\lbrace y\in\mathcal{L}: {\displaystyle{\lim_{x\to\infty}{y(x)}=0}} \right\rbrace $ ένα υποσύνολο του $ \mathcal{L} $. Παρατηρούμε ότι $ \mathcal{L}_0\neq\varnothing $ διότι ισχύει $ y(x)=0\in\mathcal{L} $ και $ {\displaystyle{\lim_{x\to\infty}{y(x)}=\lim_{x\to\infty}{0}=0}} $ άρα $ 0\in\mathcal{L}_0 $. Για να δείξουμε τώρα ότι το σύνολο $ \mathcal{L}_0 $ αποτελεί γραμμικό χώρο αρκεί να δείξουμε ότι οποιοσδήποτε γραμμικός συνδυασμός δύο στοιχείων του $ \mathcal{L}_0 $ αποτελεί επίσης στοιχείο του $ \mathcal{L}_0 $. Έστω λοιπόν $ \lambda,\mu $ δύο οποιοιδήποτε πραγματικοί αριθμοί και $ y_1,y_2\in\mathcal{L}_0 $ δύο λύσεις της \eqref{b29:1}. Τότε η συνάρτηση $ y=\lambda y_1+\mu y_2 $ αποτελεί επίσης λύσης της \eqref{b29:1} ως γραμμικός συνδυασμός δύο λύσεων της και επίσης ισχύει:
\[ \lim_{x\to\infty}{y}=\lim_{x\to\infty}{\left( \lambda y_1+\mu y_2\right)}=\lambda\cdot0+\mu\cdot0=0 \]
Άρα προκύπτει ότι $ y\in\mathcal{L}_0 $ οπότε το σύνολο $ \mathcal{L}_0 $ αποτελεί γραμμικό χώρο. Στη συνέχεια, για την εύρεση μιας βάσης του χώρου $ \mathcal{L}_0 $ παρατηρούμε ότι ισχύει:
\[ \lim_{x\to\infty}{y_1(x)}=\lim_{x\to\infty}{y_2(x)}=\lim_{x\to\infty}{y_3(x)}=0 \] καθώς και \begin{equation}\label{b29:2}
\lim_{x\to\infty}{y_4(x)}\neq 0\quad,\quad\lim_{x\to\infty}{y_5(x)}\neq 0
\end{equation}
Παίρνουμε λοιπόν ότι οι συναρτήσεις $ y_1,y_2,y_3$ ανήκουν στο χώρο $\mathcal{L}_0 $. Αρκεί να δείξουμε ότι αυτές είναι γραμμικά ανεξάρτητες και παράγουν το χώρο $ \mathcal{L}_0 $ δηλαδή κάθε στοιχείο του γράφεται ως γραμμικός συνδυασμός των $ y_1,y_2,y_3$.
\begin{alist}
\item Θεωρούμε τις σταθερές $ a_1,a_2,a_3 $ τέτοιες ώστε να ισχύει
\[ a_1y_1+a_2y_2+a_3y_3=0 \] τότε όμως θα είναι  $ a_1y_1+a_2y_2+a_3y_3+0y_4+0y_5=0 $ και επειδή το $ \left\lbrace y_1,y_2,y_3,y_4,y_5\right\rbrace $ είναι ένα βασικό σύνολο λύσεων της \eqref{b29:1} τότε προκύπτει $ a_1=a_2=a_3=0 $. Άρα οι $ y_1,y_2,y_3 $ θα είναι γραμμικά ανεξάρτητες.
\item Έστω τώρα $ y\in\mathcal{L}_0 $ μια λύση της \eqref{b29:1}. Τότε $ y\in\mathcal{L} $ και άρα θα υπάρχουν σταθερές $ \beta_1,\beta_2,\beta_3,\beta_4,\beta_5 $ έτσι ώστε η $ y $ να γραφτεί σαν γραμμικός συνδυασμός των $ y_i,(i=1,\ldots,5) $ δηλαδή
\[ y=\beta_1y_1+\beta_2y_2+\beta_3y_3+\beta_4y_4+\beta_5y_5 \]
Η $ y $ όμως ως στοιχείο και του $ \mathcal{L}_0 $ θα έχει την ιδιότητα $ {\displaystyle\lim_{x\to\infty}{y}=0} $ οπότε λόγω της \eqref{b29:2} παίρνουμε $ \beta_4=\beta_5=0 $. Έτσι οι συναρτήσεις $ y_1,y_2,y_3 $ παράγουν το χώρο $ \mathcal{L}_0 $.
\end{alist}
\end{rlist}\mbox{}\epask
\begin{Askhshs}[B]
 \bmath{Έστω η ομογενής γραμμική διαφορική εξίσωση
\begin{equation}\label{b30:1}\tag{$ E_0 $}
a_ny^{(n)}+a_{n-1}y^{(n-1)}+\ldots+a_1y'+a_0y=0,
\end{equation}
όπου $ a_i,\ i=0,1,\ldots,n $ είναι συνεχείς συναρτήσεις σε ένα διάστημα $ [x_0,\infty) $ και $ a_n(x)\neq0 $ για κάθε $ x\geq x_0 $. Να βρεθεί μια αναγκαία συνθήκη ώστε κάθε λύση της \eqref{b30:1} καθώς και οι παράγωγοι μέχρι $ n-1 $ τάξης αυτής, να είναι φραγμένες στο διάστημα $ [x_0,\infty) $.\\
\textit{Εφαρμογή Ι:} Να αποδειχθεί ότι η ομογενής γραμμική διαφορική εξίσωση
\[ y''-xy'+y\cos{x}=0 \] έχει μια τουλάχιστον λύση τέτοια ώστε αυτή ή η παράγωγός της να μην είναι φραγμένη στο διάστημα $ [0,\infty) $.\\
\textit{Εφαρμογή ΙΙ:} Ας θεωρήσουμε την Euler ομογενή γραμμική διαφορική εξίσωση
\[ a_nx^ny^{(n)}+a_{n-1}x^{n-1}y^{(n-1)}+\ldots+a_1xy'+a_0y=0\ ,\ x\geq1 \] όπου $ a_i,\ (i=0,1,\ldots,n) $ είναι πραγματικοί αριθμοί με $ a_n\cdot a_{n-1}<0 $. Να αποδειχθεί ότι η διαφορική εξίσωση αυτή έχει μια τουλάχιστον λύση $ y $, τέτοια ώστε η λύση $ y $ ή μια τουλάχιστον από τις παραγώγους $ y^{(k)},\ (k=1,\ldots,n-1) $ αυτής να είναι μη φραγμένη συνάρτηση στο διάστημα $ [1,\infty) $.}
\end{Askhshs}\mbox{}\\
\lysh
Ας θεωρήσουμε ότι κάθε λύση της \eqref{b30:1} καθώς και οι παράγωγοι μέχρι $ n-1 $ τάξης αυτής είναι φραγμένες στο διάστημα $ [x_0,\infty) $ και ας υποθέσουμε ότι το $ \left\lbrace y_1,y_2,\ldots,y_n \right\rbrace  $ είναι ένα βασικό σύνολο λύσεών της. Σύμφωνα με τον τύπο του Liouville ισχύει:
\[ W(y_1,\ldots,y_n)(x)=W(y_1,\ldots,y_n)(x_0)e^{-\dintt_{\!\!x_0}^{x}{\frac{a_{n-1}}{a_n}\d t}}\ ,\ x\in[x_0,\infty) \]
Λαμβάνοντας υπόψιν ότι $ W(y_1,\ldots,y_n)(x)\neq0 $ και ότι κάθε λύση που γράφεται ως γραμμικός συνδυασμός των $ y_1,y_2,\ldots,y_n $ θα είναι και αυτή φραγμένη, καταλήγουμε στο ότι και η συνάρτηση $ W(y_1,\ldots,y_n)(x) $ θα είναι επίσης φραγμένη. Παίρνουμε λοιπόν ότι \[ \frac{W(y_1,\ldots,y_n)(x)}{ W(y_1,\ldots,y_n)(x_0)}=e^{-\dintt_{\!\!x_0}^{x}{\frac{a_{n-1}}{a_n}\d t}} \]
και έτσι, κάθε λύση $ y $ της \eqref{b30:1} καθώς και οι παράγωγοι μέχρι $ n-1 $ τάξης αυτής είναι φραγμένες στο διάστημα $ [x_0,\infty) $ αν και μόνο αν η συνάρτηση $ e^{-\dintt_{\!\!x_0}^{x}{\frac{a_{n-1}}{a_n}\d t}},\ x\in[x_0,\infty) $ είναι φραγμένη στο $ [x_0,\infty) $ δηλαδή αν και μόνο αν η συνάρτηση
\[ e^{-\mathrm{Re}\left[\dintt_{\!\!x_0}^{x}{\frac{a_{n-1}}{a_n}\d t} \right] }
\left\lbrace \cos{\left[ -\mathrm{Im}\left(\int_{x_0}^{x}{\frac{a_{n-1}}{a_n}\d t}\right)\right]}+i\sin{\left[-\mathrm{Im}\left(\int_{x_0}^{x}{\frac{a_{n-1}}{a_n}\d t}\right)\right]}\right\rbrace\ ,\ x\in[x_0,\infty) \] είναι φραγμένη στο $ [x_0,\infty) $. Γνωρίζουμε ότι οι συναρτήσεις $ \cos{x} $ και $ \sin{x} $ είναι φραγμένες και άρα μας αρκεί η συνάρτηση $ e^{-\mathrm{Re}\left[\dintt_{\!\!x_0}^{x}{\frac{a_{n-1}}{a_n}\d t} \right] },\ x\in[x_0,\infty) $ να είναι φραγμένη. Αυτή είναι κάτω φραγμένη από το $ 0 $ άρα η θα πρέπει αυτή, να είναι και άνω φραγμένη ή ισοδύναμα η $ \mathrm{Re}\left[\dints_{\!\!x_0}^{x}{\frac{a_{n-1}}{a_n}\d t} \right] $ να είναι κάτω φραγμένη, δηλαδή να υπάρχει αριθμός $ m\in\mathbb{R} $ τέτοιος ώστε να ισχύει:
\[ \mathrm{Re}\left[\int_{x_0}^{x}{\frac{a_{n-1}}{a_n}\d t} \right]\geq m\ ,\ \forall x\in[x_0,\infty) \]
\textit{Εφαρμογή Ι:}\\
Η διαφορική εξίσωση $ y''-xy'+y\cos{x}=0 $ έχει τη μορφή της \eqref{b30:1} με τάξη $ n=2 $ και συντελεστές $ a_2=1,a_1=-x,a_0=\cos{x} $. Έτσι, σύμφωνα με το προηγούμενο ερώτημα παρατηρούμε ότι η συνάρτηση
\[ \int_{0}^{x}{\frac{-t}{1}\d t}=-\frac{x^2}{2}\ ,\ x\in[0,\infty) \]
δεν είναι κάτω φραγμένη άρα η δοσμένη εξίσωση έχει τουλάχιστον μια λύση ώστε αυτή ή η παράγωγός της να μην είναι φραγμένη στο $ [0,\infty) $.\\\\
\textit{Εφαρμογή ΙΙ:} Για τη διαφορική εξίσωση Euler αρκεί να δείξουμε, όπως και στο προηγούμενο ερώτημα ότι η συνάρτηση $ \mathrm{Re}\left[\dints_{\!\!x_0}^{x}{\frac{a_{n-1}t^{n-1}}{a_nt^n}\d t} \right] $ δεν είναι κάτω φραγμένη στο διάστημα $ [1,\infty) $. Πράγματι θα έχουμε:
\[ \mathrm{Re}\left[\int_{x_0}^{x}{\frac{a_{n-1}t^{n-1}}{a_nt^n}\d t} \right]=\frac{a_{n-1}}{a_n}\int_{1}^{x}{\frac{t^{n-1}}{t^n}\d t}=\frac{a_{n-1}}{a_n}\int_{1}^{x}{\frac{1}{t}\d t}=\frac{a_{n-1}}{a_n}\cdot\log{x} \]
Η συνάρτηση αυτή παρατηρούμε ότι δεν είναι κάτω φραγμένη καθώς για $ x\to\infty $ έχουμε 
\[ \lim_{x\to\infty}{\frac{a_{n-1}}{a_n}\cdot\log{x}}\eq{a_n\cdot a_{n-1}<0}-\infty \]
επομένως η εξίσωση θα έχει μια τουλάχιστον λύση τέτοια ώστε αυτή ή κάποια από τις παραγώγους της μέχρι $ n-1 $ τάξης να είναι μη φραγμένη στο διάστημα $ [1,\infty) $.\epask
\begin{Askhshs}[B]
 \bmath{Ας είναι $ \beta $ και $ \gamma $ θετικοί αριθμοί με $ \beta^2\neq4\gamma $. Να αποδειχθεί ότι όλες οι λύσεις της γραμμικής διαφορικής εξίσωσης
\[ y''+\beta y'+\gamma y=(x+1)^{-2}\sin{x}\ ,\ x\geq0  \]
τείνουν προς το μηδέν για $ x\to\infty $.}
\end{Askhshs}\mbox{}\\
\lysh
Η αντίστοιχη ομογενής εξίσωση της αρχικής είναι η $ y''+\beta y'+\gamma y=0 $ και έχει χαρακτηριστικό πολυώνυμο το $ P(\lambda)=\lambda^2+\beta\lambda+\gamma $. Έχουμε τώρα τις εξής περιπτώσεις:
\begin{rlist}
\item Αν $ \beta^2>4\gamma $ τότε $ \beta^2-4\gamma>0 $ και έτσι οι ρίζες του πολυωνύμου θα είναι οι $ \lambda_1=\frac{-\beta+\sqrt{\beta^2-4\gamma}}{2} $ και $ \lambda_2=\frac{-\beta-\sqrt{\beta^2-4\gamma}}{2} $. Ένα βασικό σύνολο λύσεων της ομογενούς εξίσωσης θα είναι το $ \left\lbrace y_1(x)=e^{\lambda_1x},y_2(x)=e^{\lambda_2x} \right\rbrace  $. Είναι φανερό ότι $ \lambda_2<0 $. Για τη λύση $ \lambda_1 $ τώρα θα έχουμε:
\[ -4\gamma<0\Rightarrow \beta^2-4\gamma<\beta^2\Rightarrow\sqrt{\beta^2-4\gamma}<\beta\Rightarrow -\beta+\sqrt{\beta^2-4\gamma}<0\Rightarrow\lambda_1<0 \]
Έτσι για κάθε βασική λύση της ομογενούς εξίσωσης θα ισχύει:
\[ \lim_{x\to\infty}{y_1(x)}=\lim_{x\to\infty}{e^{\lambda_1x}}=0\ \ \textrm{και}\ \ \lim_{x\to\infty}{y_2(x)}=\lim_{x\to\infty}{e^{\lambda_2x}}=0 \]
\item Αν $ \beta^2<4\gamma $ τότε $ \beta^2-4\gamma<0 $ οπότε οι ρίζες του πολυωνύμου θα είναι οι $ \lambda_1=\frac{-\beta+i\sqrt{\beta^2-4\gamma}}{2} $ και $ \lambda_2=\frac{-\beta-i\sqrt{\beta^2-4\gamma}}{2} $. Το βασικό σύνολο λύσεων στην περίπτωση αυτή θα είναι το \[ \left\lbrace y_1(x)=e^{\phantom{\lambda}\!\!\!\!\sigma x}\cos{\tau x},y_2(x)=e^{\sigma x}\sin{\tau x} \right\rbrace\ \ ,\ \ x\in\mathbb{R} \]
όπου $ \sigma=-\frac{\beta}{2} $ και $ \tau=\frac{\sqrt{\beta^2-4\gamma}}{2} $. Γνωρίζουμε ότι οι συναρτήσεις $ \cos{\tau x} $ και $ \sin{\tau x} $ είναι φραγμένες ενώ η συνάρτηση τείνει προς το μηδέν καθώς $ x\to \infty $. Έτσι ως γινόμενα μηδενικών επί φραγμένων συναρτήσεων θα δώσουν
\[ \lim_{x\to\infty}{y_1(x)}= \lim_{x\to\infty}{y_2(x)}=0 \]
\end{rlist}
Τέλος θα αναζητήσουμε μια μερική λύση της μη ομογενούς διαφορικής εξίσωσης. Αυτή θα δίνεται από τον τύπο:
\begin{align*}
y_\mu(x)&=\int_{0}^{x}{\frac{y_1(t)y_2(x)-y_2(t)y_1(x)}{y_1(t)y_2'(t)-y_1'(t)y_2(t)}\cdot\frac{b(t)}{a_2(t)}\d t}=\int_{0}^{x}{\frac{e^{\lambda_1 t}\cdot e^{\lambda_2 x}-e^{\lambda_2 t}\cdot e^{\lambda_1 x}}{\left( \lambda_2-\lambda_1\right) e^{\left( \lambda_1+\lambda_2\right)t }}\cdot\frac{\sin{t}}{(t+1)^2}\d t}=\\
&=\frac{e^{\lambda_2 x}}{\left( \lambda_2-\lambda_1\right) }\int_{0}^{x}{e^{-\lambda_2 t}\cdot\frac{\sin{t}}{(t+1)^2}\d t}-\frac{e^{\lambda_1 x}}{\left( \lambda_2-\lambda_1\right) }\int_{0}^{x}{e^{-\lambda_1 t}\cdot\frac{\sin{t}}{(t+1)^2}\d t}
\end{align*}
Αρκεί να δείξουμε ότι $ {\displaystyle{\lim_{x\to\infty}}}{\frac{e^{\lambda_1 x}}{\left( \lambda_2-\lambda_1\right) }\dints_{\!\!0}^{x}{e^{-\lambda_1 t}\cdot\frac{\sin{t}}{(t+1)^2}\d t}}=0 $. Έστω λοιπόν $ L(x)=\frac{1}{\left( \lambda_2-\lambda_1\right) }\dints_{\!\!0}^{x}{e^{-\lambda_1 t}\cdot\frac{\sin{t}}{(t+1)^2}\d t} $ οπότε παίρνουμε τις εξής περιπτώσεις:
\begin{enumerate}
\item Αν $ {\displaystyle{\lim_{x\to\infty}}}{L(x)}=a<\infty $ όπου $ a\in\mathbb{R} $ τότε θα έχουμε:
\[ \lim_{x\to\infty}{e^{\lambda_1x}L(x)}=0\cdot a=0 \]
\item Αν τώρα $ {\displaystyle{\lim_{x\to\infty}}}{L(x)}=\infty $ τότε θα ισχύει:
\[ \lim_{x\to\infty}{e^{\lambda_1x}L(x)}=\lim_{x\to\infty}{\frac{L(x)}{e^{-\lambda_1x}}}\eq{\frac{\infty}{\infty}}
\lim_{x\to\infty}{
\frac{\frac{1}{\left( \lambda_2-\lambda_1\right) }e^{-\lambda_1 x}\cdot\frac{\sin{x}}{(x+1)^2}}{-\lambda_1e^{-\lambda_1x}}}=
-\frac{1}{\lambda_1\left( \lambda_2-\lambda_1\right) }\lim_{x\to\infty}{
\frac{\sin{x}}{(x+1)^2}\d t}=0 \]
ως γινόμενο μηδενικής συνάρτησης επί φραγμένη.
\end{enumerate}
Σε κάθε περίπτωση λοιπόν για τη μερική λύση θα ισχύει $ {\displaystyle{\lim_{x\to\infty}{y_\mu(x)}}}=0 $. Καταλήγουμε λοιπόν στο συμπέρασμα ότι όλες οι λύσεις της αρχικής διαφορικής εξίσωσης τείνουν προς το μηδέν καθώς $ x\to\infty $.\epask
\begin{Askhshs}[B]
 \bmath{Με τη βοήθεια ενός μετασχηματισμού της μορφής $ t=x^a $ (όπου $ a $ είναι κατάλληλη πραγματική σταθερά), να επιλυθεί η γραμμική διαφορική εξίσωση
\[ xy''-y'+x^3y=0\ \ ,\ \ x>0 \]}
\end{Askhshs}\mbox{}\\
\lysh
Χρησιμοποιώντας το μετασχηματισμό $ t=x^a $ προκύπτει, ύστερα από παραγώγιση 
\[ \frac{dy}{dx}=\frac{dy}{dt}\cdot\frac{dt}{dx}=ax^{a-1}\frac{dy}{dt}\ \ \textrm{και}\ \ \frac{d^2y}{dx^2}=a(a-1)x^{a-2}\frac{dy}{dt}+a^2x^{2a-2}\frac{d^2y}{dt^2} \]
Αντικαθιστούμε τις συναρτήσεις αυτές στη διαφορική εξίσωση οπότε αυτή θα γίνει:
\begin{gather*}
a(a-1)x^{a-1}\frac{dy}{dt}+a^2x^{2a-1}\frac{d^2y}{dt^2}-ax^{a-1}\frac{dy}{dt}+x^3y=0\Rightarrow\\ a^2x^{2a-1}\frac{d^2y}{dt^2}+a(a-2)x^{a-1}\frac{dy}{dt}+x^3y=0
\end{gather*}
Επιλέγοντας $ a=2 $ ο μετασχηματισμός θα γίνει $ t=x^2 $ ενώ η τελευταία εξίσωση θα πάρει τη μορφή:
\[ 4x^{3}\frac{d^2y}{dt^2}+x^3y=0\Rightarrow 4\frac{d^2y}{dt^2}+y=0\ ,\ t>0 \]
Η εξίσωση που προέκυψε είναι μια ομογενής γραμμική διαφορική εξίσωση 2\tss{ης} τάξης με σταθερούς συντελεστές. Το χαρακτηριστικό της πολυώνυμο είναι το $ P(\lambda)=4\lambda^2+1 $ με μιγαδικές ρίζες $ \lambda_1=\frac{i}{2} $ και $ \lambda_2=-\frac{i}{2} $. Έτσι οι συναρτήσεις 
\[ y_1(t)=\cos\left( \frac{t}{2}\right)\ ,\ t\in\mathbb{R}^*\ \ \textrm{και}\ \ y_2(t)=\sin\left( \frac{t}{2}\right)\ ,\ t>0 \]
αποτελούν ένα βασικό σύνολο λύσεων της ομογενούς εξίσωσης. Έτσι όλες οι λύσεις της αρχικής θα δίνονται από τον παρακάτω τύπο:
\[ y(t)=c_1\cos\left( \frac{t}{2}\right)+c_2\sin\left( \frac{t}{2}\right)\ ,\ t>0 \]
ή ισοδύναμα ύστερα από αντικατάσταση
\[ y(x)=c_1\cos\left( \frac{x^2}{2}\right)+c_2\sin\left( \frac{x^2}{2}\right)\ ,\ x>0 \]
όπου $ c_1,c_2 $ είναι αυθαίρετες σταθερές.\epask
\begin{Askhshs}[B]
 \bmath{Με τη βοήθεια της αντικατάστασης $ y=ze^{-\frac{x^2}{4}} $, να επιλυθεί η ομογενής γραμμική διαφορική εξισωση
\begin{equation}\label{b33:2}\tag{$ E_0 $}
y''+\left( \rho+\frac{1}{2}-\frac{1}{4}x^2\right)y=0
\end{equation} όπου $ \rho $ είναι μια πραγματική σταθερά.}
\end{Askhshs}\mbox{}\\
\lysh
Παραγωγίζουμε τη σχέση $ y=ze^{-\frac{x^2}{4}} $ και παίρνουμε:
\[ y'=\left(z'-\frac{x}{2}z\right)e^{-\frac{x^2}{4}}\Rightarrow y''=\left(z''-xz'+\frac{x^2-2}{4}z\right)e^{-\frac{x^2}{4}}
\]
Αν αντικαταστήσουμε τις συναρτήσεις αυτές στην αρχική διαφορική εξίσωση τότε αυτή θα μετασχηματιστεί στην εξίσωση:
\begin{gather}
\left(z''-xz'+\frac{x^2-2}{4}z\right)e^{-\frac{x^2}{4}}+\left( \rho+\frac{1}{2}-\frac{1}{4}x^2\right)ze^{-\frac{x^2}{4}}=0\Rightarrow\nonumber\\
z''-xz'+\frac{x^2-2}{4}z+ \left( \rho+\frac{1}{2}-\frac{1}{4}x^2\right)z=0\nonumber\\
z''-xz'+\rho z=0\label{b33:1}\tag{$ E_0' $}
\end{gather}
Παρατηρούμε ότι η συνάρτηση $ z_1=\rho x $ είναι μια μερική λύση της παραπάνω εξίσωσης. Θέτοντας τώρα όπου $ z=z_1u $ και παραγωγίζοντας παίρνουμε τις συναρτήσεις $ z'=z_1'u+z_1u' $ και $ z''=z_1''u+2z_1'u'+z_1u'' $. Με αντικατάσταση, η \eqref{b33:1} θα γίνει:
\begin{gather*}
z_1''u+2z_1'u'+z_1u''-x\left( z_1'u+z_1u'\right)+\rho z_1u=0\Rightarrow\\
z_1u''+\left( 2z_1'-xz_1\right)u'+\left(z_1''-xz_1'+\rho z_1 \right)u=0\xRightarrow{\eqref{b33:1}}\\
z_1u''+\left( 2z_1'-xz_1\right)u'=0
\end{gather*}
Στην τελευταία εξίσωση θέτουμε $ w=u' $ και $ w'=u'' $ ώστε να πετύχουμε υποβιβασμό της τάξης της και έτσι προκύπτει η 1\tss{ης} τάξης γραμμική διαφορική εξίσωση:
\[ z_1w'+\left( 2z_1'-xz_1\right)w=0\xRightarrow{z_1=\rho x}\rho xw'+\left( 2\rho-x^2\rho\right)w=0 \]
Αν $ \rho\neq0 $ τότε η προηγούμενη εξίσωση θα γίνει:
\[ w'+\frac{2-x^2}{x}w=0 \]
Όλες οι λύσεις της θα δίνονται από τον ακόλουθο τύπο:
\begin{align*}
w(x)=ce^{\dintt{\frac{x^2-2}{x}\d x}}=ce^{\dintt{\left(x-\frac{2}{x} \right) \d x}}=ce^{\frac{x^2}{2}-2\log{x}}=\frac{ce^{\frac{x^2}{2}}}{x^2}\ \ ,\ \ x>0
\end{align*}
όπου $ c $ είναι μια αυθαίρετη σταθερά. Επομένως ένα βασικό σύνολο λύσεων της \eqref{b33:1} θα είναι το 
\[ \left\lbrace z_1=\rho x,z_2=\rho x\int{\frac{ce^{\frac{x^2}{2}}}{x^2}\d x}\right\rbrace\ \ ,\ \ x>0  \]
και έτσι όλες οι λύσεις της θα δίνονται από τον τύπο:
\[ z(x)=c_1z_1+c_2z_2=c_1\rho x+c_2\rho x\int{\frac{ce^{\frac{x^2}{2}}}{x^2}\d x}\ \ ,\ \ x>0 \]
Άρα όλες οι λύσεις της αρχικής διαφορικής εξίσωσης \eqref{b33:2} θα δίνονται από τον τύπο:
\[ y(x)=c_1\rho xe^{-\frac{x^2}{4}}+c_2\rho xe^{-\frac{x^2}{4}}\int{\frac{ce^{\frac{x^2}{2}}}{x^2}\d x}\ \ ,\ \ x>0 \]\epask
\begin{Askhshs}[B]
 \bmath{Με τη βοήθεια ενός μετασχηματισμού της μορφής $ y=x^a z $ (όπου $ a $ είναι ένας πραγματικός αριθμός που θα πρέπει να βρεθεί), να επιλυθεί η γραμμική διαφορική εξίσωση
\[ x^2y''+xy'+\left(x^2-\frac{1}{4} \right)y=x^{3/2}e^{-x}\ \ ,\ \ x>0  \]}
\end{Askhshs}\mbox{}\\
\lysh
Θέτουμε όπου $ y=x^a z $ και παραγωγίζοντας παράλληλα το μετασχηματισμό αυτό παίρνουμε τις παραγώγους $ y'=ax^{a-1}z+x^az' $ και $ y''=a(a-1)x^{a-2}z+2ax^{a-1}z'+x^az'' $. Αντικαθιστούμε τις σχέσεις αυτές στην εξίσωση και θα έχουμε:
\begin{gather*}
x^2\left[ a(a-1)x^{a-2}z+2ax^{a-1}z'+x^az'' \right]+x\left( ax^{a-1}z+x^az'\right) +\left(x^2-\frac{1}{4} \right)x^a z=x^{3/2}e^{-x}\Rightarrow\\
a(a-1)x^{a}z+2ax^{a+1}z'+x^{a+2}z''+ax^{a}z+x^{a+1}z'+\left(x^{a+2}-\frac{x^a}{4} \right)z=x^{3/2}e^{-x}\Rightarrow\\
x^{a+2}z''+(2a+1)z'+\left( a^2+x^2-\frac{1}{4}\right)x^az=x^{3/2}e^{-x}\ \ ,\ \ x>0
\end{gather*}
Επιλέγοντας $ a=-\frac{1}{2} $ ο μετασχηματισμός που χρησιμοποιήσαμε είναι ο $ y=x^{-1/2}z $ ενώ η τελευταία εξίσωση θα γίνει:
\begin{equation}\label{b34:1}
x^{3/2}z''+x^{3/2}z=x^{3/2}e^{-x}\xRightarrow{x>0} z''+z=e^{-x}\ \ ,\ \ x>0
\end{equation}
η οποία είναι μια γραμμική διαφορική εξίσωση 2\tss{ης} τάξης με σταθερούς συντελεστές. Το χαρακτηριστικό πολυώνυμο της αντίστοιχης μη ομογενούς εξίσωσης είναι το \[ P(\lambda)=\lambda^2+1 \]
με ρίζες τις $ \lambda_1=i $ και $ \lambda_2=-i $. Έτσι οι συναρτήσεις $ y_1=\cos{x}\ ,\ x>0 $ και $ y_2=\sin{x}\ ,\ x>0 $ αποτελούν ένα βασικό σύνολο λύσεων της ομογενούς εξίσωσης. Συνεπώς όλες οι λύσεις της θα δίνονται από τον τύπο:
\[ \tilde{z}(x)=c_1\cos{x}+c_2\sin{x}\ \ ,\ \ x>0 \]
όπου $ c_1,c_2 $ είναι αυθαίρετες σταθερές. Θα αναζητήσουμε στη συνέχεια μια μερική λύση της μη ομογενούς εξίσωσης \eqref{b34:1}. Θέτοντας λοιπόν όπου $ z=ue^{-x} $ προκύπτει ύστερα από παραγώγιση $ z'=(u'-u)e^{-x} $ και $ z''=(u''-2u'+u)e^{-x} $. Με αντικατάσταση η \eqref{b34:1} θα γραφτεί:
\begin{equation}\label{b34:2}
(u''-2u'+u)e^{-x}+ue^{-x}=e^{-x}\Rightarrow u''-2u'+u+2u=1
\end{equation}
Η μερική λύση της τελευταίας εξίσωσης θα είναι της μορφής $ u_\mu=a $ από την οποία παίρνουμε $ u_\mu'=u_\mu''=0 $. Άρα θα έχουμε:
\[ 2a=1\Rightarrow a=\frac{1}{2} \]
Η ζητούμενη μερική λύση της \eqref{b34:2} θα είναι $ u=\frac{1}{2}\ ,\ x>0 $ άρα η μερική λύση της \eqref{b34:1} θα είναι η $ z_\mu=\frac{1}{2}e^{-x}\ ,\ x>0 $. Όλες οι λύσεις της θα δίνονται από τον τύπο:
\[ z(x)=c_1\cos{x}+c_2\sin{x}+\frac{1}{2}e^{-x}\ \ ,\ \ x>0 \]
ενώ όλες οι λύσεις της αρχικής διαφορικής εξίσωσης θα δίνονται από τον τύπο:
\[ y(x)=c_1\frac{\sqrt{x}\cos{x}}{x}+c_2\frac{\sqrt{x}\sin{x}}{x}+\frac{\sqrt{x}}{2x}e^{-x}\ \ ,\ \ x>0 \]
όπου $ c_1,c_2 $ είναι αυθαίρετες σταθερές.\epask
\begin{Askhshs}[B]
 \bmath{Με τη βοήθεια του μετασχηματισμού $ y=ue^{-\frac{x^2+x}{2}} $, να επιλυθεί η ομογενής γραμμική διαφορική εξίσωση
\[ y''+(2x+1)y'+\left( x^2+x+\frac{1}{4}\right)y=0 \]}
\end{Askhshs}\mbox{}\\
\lysh
Χρησιμοποιούμε το μετασχηματισμό $ y=ue^{-\frac{x^2+x}{2}} $ και ύστερα από παραγώγιση παίρνουμε:
\begin{align}
y'&=u'e^{-\frac{x^2+x}{2}}+u\cdot\left(-\frac{x^2+x}{2}\right)'e^{-\frac{x^2+x}{2}}=\nonumber\\
&=u'e^{-\frac{x^2+x}{2}}+u\cdot\left(-\frac{2x+1}{2}\right)'e^{-\frac{x^2+x}{2}}=\left[ u'-\left(x+\frac{1}{2} \right)u \right] e^{-\frac{x^2+x}{2}}\ \ \textrm{και}\label{b35:1}\\
y''&=\left[ u''-u-\left(x+\frac{1}{2} \right)u'- \right] e^{-\frac{x^2+x}{2}}-\frac{2x+1}{2}\left[ u'-\left(x+\frac{1}{2} \right)u \right] e^{-\frac{x^2+x}{2}}=\nonumber\\
&=\left[ u''-u-\left(x+\frac{1}{2} \right)u'- \right] e^{-\frac{x^2+x}{2}}-\left[ \frac{2x+1}{2}u'-\left(x+\frac{1}{2} \right)^2u \right] e^{-\frac{x^2+x}{2}}=\nonumber\\
&=\left[u''-(2x+1)u'+\left(x^2+x-\frac{3}{4}\right)u  \right] e^{-\frac{x^2+x}{2}}\label{b35:2}
\end{align}
Στη συνέχεια αντικαθιστούμε τις σχέσεις \eqref{b35:1} και \eqref{b35:2} στην αρχική διαφορική εξίσωση και αυτή θα μετασχηματιστεί ως εξής:
\begin{gather}
u''-(2x+1)u'+\left(x^2+x-\frac{3}{4}\right)u+(2x+1)\left[ u'-\left(x+\frac{1}{2} \right)u \right]+\left( x^2+x+\frac{1}{4}\right)u=0\Rightarrow\nonumber\\
u''-(2x+1)u'+\left(x^2+x-\frac{3}{4}\right)u+ (2x+1)u'-\left(2x^2+2x+\frac{1}{2} \right)u+\left( x^2+x+\frac{1}{4}\right)u=0\Rightarrow\nonumber\\
u''+\left(x^2+x-\frac{3}{4}-2x^2-2x-\frac{1}{2}+x^2+x+\frac{1}{4} \right)u=0\Rightarrow\nonumber\\
u''-u=0\label{b35:3}
\end{gather}
η οποία είναι μια ομογενής γραμμική διαφορική εξίσωσης 2\tss{ης} τάξης με σταθερούς συντελεστές. Το χαρακτηριστικό της πολυώνυμο είναι το 
\[ P(\lambda)=\lambda^2-1 \]
το οποίο έχει ρίζες τις $ \lambda_{1,2}=\pm1 $. Έτσι οι συναρτήσεις $ u_1(x)=e^x $ και $ u_2(x)=e^{-x} $, αποτελούν ένα βασικό σύνολο λύσεων της \eqref{b35:3}. Όλες οι λύσεις της θα δίνονται από τη σχέση:
\[ u(x)=c_1e^x+c_2e^{-x} \]
Κατά συνέπεια όλες οι λύσεις της αρχικής ομογενούς διαφορικής εξίσωσης θα δίνονται από τον τύπο:
\[ y(x)=c_1e^{-\frac{x^2-x}{2}}+c_2e^{-\frac{x^2+3x}{2}} \]
όπου $ c_1,c_2 $ είναι αυθαίρετες σταθερές.\epask
\begin{Askhshs}[B]
 \bmath{Έστω η ομογενής γραμμική διαφορική εξίσωση
\begin{equation}\label{b36:1}\tag{*}
\sum_{k=1}^{7}{y^{(k)}}=0
\end{equation}
\begin{brlist}
\item Να βρεθεί ένα βασικό σύνολο πραγματικών λύσεων της \eqref{b36:1}.
\item Να αποδειχθεί ότι το σύνολο όλων των πραγματικών λύσεων της \eqref{b36:1}, οι οποίες τείνουν προς το μηδέν για $ x\to\infty $, είναι ένας γραμμικός χώρος επί του $ \mathbb{R} $ και στη συνέχεια, να βρεθεί μια βάση αυτόυ.
\end{brlist}}
\end{Askhshs}\mbox{}\\
\lysh
\begin{rlist}
\item Η αρχική διαφορική εξίσωση ισοδύναμα γράφεται ως εξής:
\[ y^{(7)}+y^{(6)}+y^{(5)}+y^{(4)}+y'''+y''+y'+y=0\ \ ,\ \ x\in\mathbb{R} \]
Το χαρακτηριστικό της πολυώνυμο θα είναι το
\[ P(\lambda)=\lambda^7+\lambda^6+\lambda^5+\lambda^4+\lambda^3+\lambda^2+\lambda+1=0=\frac{\lambda^8-1}{\lambda-1} \]
ως άθροισμα $ 8 $ όρων γεωμετρικής προόδου με πρώτο όρο τον $ a_1=1 $ και λόγο $ \lambda $. Έτσι θα έχουμε
\[ P(\lambda)=\frac{\lambda^8-1}{\lambda-1}=\frac{\left( \lambda^4-1\right) \left(\lambda^4+1 \right) }{\lambda-1}=\frac{\left( \lambda^2-1\right)\left( \lambda^2+1\right)\left(\lambda^4+1 \right) }{\lambda-1}=(\lambda+1)\left( \lambda^2+1\right)\left(\lambda^4+1 \right) \]
Οι ρίζες του χαρακτηριστικού πολυωνύμου θα είναι οι
\[ \lambda_1=-1\ ,\ \lambda_{2,3}=\pm i\ ,\ \lambda_{4,5}=\frac{\sqrt{2}}{2}\pm\frac{\sqrt{2}}{2}\ ,\ \lambda_{6,7}=-\frac{\sqrt{2}}{2}\pm\frac{\sqrt{2}}{2} \]
από τις οποίες προκύπτουν οι γραμμικά ανεξάρτητες \textbf{πραγματικές} συναρτήσεις
\begin{gather*}
y_1(x)=e^{-x}\ ,\ y_2(x)=\cos{x}\ ,\ y_3(x)=\sin{x}\ ,\  y_4(x)=e^{\frac{\sqrt{2}}{2}x}\cos{\left(\frac{\sqrt{2}}{2}x\right) }\\
y_5(x)=e^{\frac{\sqrt{2}}{2}x}\sin{\left(\frac{\sqrt{2}}{2}x\right) }\ ,\ y_6(x)=e^{-\frac{\sqrt{2}}{2}x}\cos{\left(\frac{\sqrt{2}}{2}x\right) }\ \ \textrm{και}\ \ 
y_7(x)=e^{-\frac{\sqrt{2}}{2}x}\sin{\left(\frac{\sqrt{2}}{2}x\right) }
\end{gather*}
που αποτελούν ένα βασικό σύνολο πραγματικών λύσεων της αρχικής ομογενούς εξίσωσης. Συνεπώς όλες οι λύσεις της \eqref{b36:1} θα δίνονται από τον τύπο
\begin{multline*}
y(x)=c_1e^{-x}+c_2\cos{x}+c_3\sin{x}+c_4e^{\frac{\sqrt{2}}{2}x}\cos{\left(\frac{\sqrt{2}}{2}x\right) }+c_5e^{\frac{\sqrt{2}}{2}x}\sin{\left(\frac{\sqrt{2}}{2}x\right) }+\\+c_6e^{-\frac{\sqrt{2}}{2}x}\cos{\left(\frac{\sqrt{2}}{2}x\right) }+c_7e^{-\frac{\sqrt{2}}{2}x}\sin{\left(\frac{\sqrt{2}}{2}x\right) }\ \ ,\ \ x\in\mathbb{R}
\end{multline*}
όπου $ c_i,\ i=1,2,\ldots,7 $ είναι αυθαίρετες σταθερές.
\item Έστω $ \mathcal{L} $ το σύνολο όλων των πραγματικών λύσεων της \eqref{b36:1} και $ \mathcal{L}_0=\left\lbrace y\in\mathcal{L}: {\displaystyle{\lim_{x\to\infty}{y(x)}=0}} \right\rbrace  $ ένα υποσύνολο του $ \mathcal{L} $. Παρατηρούμε ότι $ \mathcal{L}_0\neq\varnothing $ διότι ισχύει $ y(x)=0\in\mathcal{L} $ και $ {\displaystyle{\lim_{x\to\infty}{y(x)}=\lim_{x\to\infty}{0}=0}} $ άρα $ 0\in\mathcal{L}_0 $. Για να δείξουμε τώρα ότι το σύνολο $ \mathcal{L}_0 $ αποτελεί γραμμικό χώρο αρκεί να δείξουμε ότι οποιοσδήποτε γραμμικός συνδυασμός δύο στοιχείων του $ \mathcal{L}_0 $ αποτελεί επίσης στοιχείο του $ \mathcal{L}_0 $. Έστω λοιπόν $ \lambda,\mu $ δύο οποιοιδήποτε πραγματικοί αριθμοί και $ y_\nu,y_\kappa\in\mathcal{L}_0 $ δύο λύσεις της \eqref{b36:1}. Τότε η συνάρτηση $ y=\lambda y_\nu+\mu y_\kappa $ αποτελεί επίσης λύσης της \eqref{b36:1} ως γραμμικός συνδυασμός δύο λύσεων της και επίσης ισχύει:
\[ \lim_{x\to\infty}{y}=\lim_{x\to\infty}{\left( \lambda y_\nu+\mu y_\kappa\right)}=\lambda\cdot0+\mu\cdot0=0 \]
Άρα προκύπτει ότι $ y\in\mathcal{L}_0 $ οπότε το σύνολο $ \mathcal{L}_0 $ αποτελεί γραμμικό χώρο. Στη συνέχεια, για την εύρεση μιας βάσης του χώρου $ \mathcal{L}_0 $ παρατηρούμε ότι ισχύει:
\[ \lim_{x\to\infty}{y_1(x)}=\lim_{x\to\infty}{y_6(x)}=\lim_{x\to\infty}{y_7(x)}=0 \] καθώς και \begin{equation}\label{b36:2}
\lim_{x\to\infty}{y_2(x)}\neq 0\quad,\quad\lim_{x\to\infty}{y_3(x)}\neq 0\quad,\quad\lim_{x\to\infty}{y_4(x)}\neq 0\quad,\quad\lim_{x\to\infty}{y_5(x)}\neq 0
\end{equation}
Παίρνουμε λοιπόν ότι οι συναρτήσεις $ y_1,y_6,y_7 $ ανήκουν στο χώρο $\mathcal{L}_0 $. Αρκεί να δείξουμε ότι αυτές είναι γραμμικά ανεξάρτητες και παράγουν το χώρο $ \mathcal{L}_0 $ δηλαδή κάθε στοιχείο του γράφεται ως γραμμικός συνδυασμός των $ y_1,y_6,y_7 $.
\begin{alist}
\item Θεωρούμε τις σταθερές $ a_1,a_6,a_7 $ τέτοιες ώστε να ισχύει
\[ a_1y_1+a_6y_6+a_7y_7=0. \] Τότε όμως θα είναι  $ a_1y_1+0y_2+0y_3+0y_4+0y_5+a_6y_6+a_7y_7=0 $ και αφού το $ \left\lbrace y_1,y_2,\ldots,y_7\right\rbrace $ είναι ένα βασικό σύνολο λύσεων της \eqref{b36:2} τότε προκύπτει $ a_1=a_6=a_7=0 $. Άρα οι $ y_1,y_6,y_7 $ θα είναι γραμμικά ανεξάρτητες.
\item Έστω τώρα $ y\in\mathcal{L}_0 $ μια λύση της \eqref{b36:1}. Τότε $ y\in\mathcal{L} $ και άρα θα υπάρχουν σταθερές $ \beta_1,\beta_2,\beta_3,\beta_4,\beta_5,\beta_6,\beta_7 $ έτσι ώστε η $ y $ να γραφτεί σαν γραμμικός συνδυασμός των $ y_i,(i=1,\ldots,7) $ δηλαδή
\[ y=\beta_1y_1+\beta_2y_2+\beta_3y_3+\beta_4y_4+\beta_5y_5+\beta_6y_6+\beta_7y_7 \]
Η $ y $ όμως ως στοιχείο και του $ \mathcal{L}_0 $ θα έχει την ιδιότητα $ {\displaystyle\lim_{x\to\infty}{y}=0} $ οπότε λόγω της \eqref{b36:2} παίρνουμε $ \beta_2=\beta_3=\beta_4=\beta_5=0 $. Έτσι οι συναρτήσεις $ y_1,y_6,y_7 $ παράγουν το χώρο $ \mathcal{L}_0 $.
\end{alist}
\end{rlist}\mbox{}\epask
\begin{Askhshs}[B]
 \bmath{Ας είναι $ a,\omega $ και $ c $ σταθερές με $ 0\leq a<\omega $ και $ c>0 $. Για τυχούσα λύση $ y $ της γραμμικής διαφορικής εξίσωσης
\[ y''+2ay'+\omega^2y=c\cos{\omega x}\ \ ,\ \ x\in\mathbb{R} \]
να βρεθεί το $ {\displaystyle{\limsup_{x\to\infty}{|y(x)|}}} $.}
\end{Askhshs}\mbox{}\\
\lysh
Η αντίστοιχη ομογενής διαφορική εξίσωση της αρχικής θα είναι η 
\[ y''+2ay'+\omega^2y=0\ \ ,\ \ x\in\mathbb{R} \]
με χαρακτηριστικό πολυώνυμο το \[ P(\lambda)=\lambda^2+2a\lambda+\omega^2 \]
Η διακρίνουσα του τριωνύμου αυτού είναι η $ \varDelta=4a^2-4\omega^2=4\left(a^2-\omega^2 \right)<0 $, αφού ισχύει η σχέση $ 0\leq a<\omega\Rightarrow a^2<\omega^2 $. Οι ρίζες λοιπόν του τριωνύμου θα είναι οι $ \lambda_{1,2}=-a\pm i\sqrt{\omega^2-a^2} $ και κατά συνέπεια οι συναρτήσεις $ y_1(x)=e^{-ax}\cos{\left(\sqrt{ \omega^2-a^2}x\right) }\ ,\ x\in\mathbb{R} $ και $ y_2(x)=e^{-ax}\sin{\left(\sqrt{ \omega^2-a^2}x\right) }\ ,\ x\in\mathbb{R} $ αποτελούν ένα βασικό σύνολο λύσεων της ομογενούς εξίσωσης. Όλες οι λύσεις της θα δίνονται από τον τύπο:
\[ \tilde{y}(x)=c_1e^{-ax}\cos{\left(\sqrt{ \omega^2-a^2}x\right) }+c_2e^{-ax}\sin{\left(\sqrt{ \omega^2-a^2}x\right) }\ ,\ x\in\mathbb{R} \]
όπου $ c_1,c_2 $ είναι αυθαίρετες σταθερές. Για την εύρεση μιας μερικής λύσης της αρχικής μη ομογενούς εξίσωσης, θεωρούμε τη διαφορική εξίσωση
\begin{equation}\label{b37:1}
u''+2au'+\omega^2u=ce^{i\omega x}\ \ ,\ \ x\in\mathbb{R}
\end{equation}
Σ' αυτήν θέτουμε $ u=ze^{i\omega x} $ και παραγωγίζοντας θα έχουμε $ \forall x\in\mathbb{R} $:
\[ u'=(z'+i\omega z)e^{i\omega x}\ \ \textrm{και}\ \ u''=\left(z''+2i\omega z'-\omega^2z\right)e^{i\omega x}  \]
Έτσι η εξίσωση \eqref{b37:1} θα γίνει:
\begin{gather}
z''+2i\omega z'-\omega^2z+2a(z'+i\omega z)+\omega^2z=c\Rightarrow\nonumber\\
z''+\left(2a+2i\omega\right)z'+2ai\omega z=c\label{b37:2}
\end{gather}
με $ a,\omega $ και $ c $ σταθερές με $ 0\leq a<\omega $ και $ c>0 $. Διακρίνουμε τις εξής περιπτώσεις:
\begin{rlist}
\item Αν $ a>0 $ τότε η ζητούμενη μερική λύση της \eqref{b37:2} θα είναι της μορφής $ z_\mu=\beta $ η οποία μας δίνει $ z_\mu'=z_\mu''=0 $. Έτσι παίρνουμε:
\[ 2ai\omega \beta=c\Rightarrow \beta=-i\frac{c}{2a\omega} \]
άρα η μερική λύση της θα είναι $ z_\mu=-i\frac{c}{2a\omega} $. Κατά συνέπεια η μερική λύση της \eqref{b37:1} θα είναι η 
\[ u_\mu=-i\frac{c}{2a\omega}e^{i\omega x}=-i\frac{c}{2a\omega}\left(\cos{\omega x}+i\sin{\omega x} \right)  \]
Η μερική λύση της αρχικής διαφορικής εξίσωσης θα είναι το πραγματικό μέρος της $ u_\mu $ άρα θα έχουμε:
\[ y_\mu(x)=\mathrm{Re}(x)\left(u_\mu(x)\right)=\frac{c}{2a\omega}\sin{\omega x} \]
Έτσι όλες οι λύσεις της αρχικής διαφορικής εξίσωσης θα δίνονται από τον τύπο:
\[ y(x)=c_1e^{-ax}\cos{\left(\sqrt{ \omega^2-a^2}x\right) }+c_2e^{-ax}\sin{\left(\sqrt{ \omega^2-a^2}x\right) }+\frac{c}{2a\omega}\sin{\omega x}\ ,\ x\in\mathbb{R} \]
όπου $ c_1,c_2 $ είναι αυθαίρετες σταθερές.
\item Αν $ a=0 $ τότε η εξίσωση \eqref{b37:2} θα γραφτεί ως εξής:
\[ z''+2i\omega z'=c \]
Για τη μερική λύση αυτής θα ισχύει η σχέση $ z_\mu'=\beta $ από την οποία προκύπτει $ z_\mu''=0 $. Άρα με αντικατάσταση θα έχουμε:
\[ 2i\omega\beta=c\Rightarrow \beta=-i\frac{c}{2\omega} \]
Έτσι παίρνουμε τη μερική λύση $ z_\mu=-i\frac{c}{2\omega}x $ η οποία μας δίνει \[ u_\mu=-i\frac{c}{2\omega}xe^{i\omega x}=-i\frac{c}{2\omega}x(\cos{\omega x}+i\sin{\omega x}) \]
Η μερική λύση της αρχικής διαφορικής εξίσωσης θα είναι το πραγματικό μέρος της $ u_\mu $ άρα θα έχουμε:
\[ y_\mu(x)=\mathrm{Re}(x)\left(u_\mu(x)\right)=\frac{c}{2a\omega}x\sin{\omega x} \]
Έτσι όλες οι λύσεις της αρχικής διαφορικής εξίσωσης θα δίνονται από τον τύπο:
\[ y(x)=c_1\cos{\omega x }+c_2\sin{\omega x }+\frac{c}{2a\omega}x\sin{\omega x}\ ,\ x\in\mathbb{R} \]
όπου $ c_1,c_2 $ είναι αυθαίρετες σταθερές.
\end{rlist}
Συνοψίζοντας τους δύο τύπους από τις περιπτώσεις i. και ii. για τη σταθερά $ a $ παίρνουμε το γενικό τύπο από τον οποίο δίνονται όλες οι λύσεις της αρχικής διαφορικής εξίσωσης ο οποίος θα είναι:
\[ y(x)=\ccases{c_1e^{-ax}\cos{\left(\sqrt{ \omega^2-a^2}x\right) }+c_2e^{-ax}\sin{\left(\sqrt{ \omega^2-a^2}x\right) }+\frac{c}{2a\omega}\sin{\omega x} & \textrm{με }0<a<\omega\ \textrm{και }c>0\\
c_1\cos{\omega x }+c_2\sin{\omega x }+\frac{c}{2a\omega}x\sin{\omega x} & \textrm{με }a=0\ \textrm{και } \omega,c>0} \]
όπου $ x\in\mathbb{R} $ και $ c_1,c_2 $ αυθαίρετες σταθερές. Συνεχίζουμε υπολογίζοντας το $ {\displaystyle{\limsup_{x\to\infty}{|y(x)|}}} $ για κάθε μια από τις συναρτήσεις του πολλαπλού τύπου για τη λύση $ y(x) $. Έτσι:
\begin{rlist}
\item Αν $ a>0 $ τότε $ {\displaystyle{\limsup_{x\to\infty}{|\tilde{y}(x)|}}}=0 $ και έτσι \[ \limsup_{x\to\infty}{|y(x)|}=\limsup_{x\to\infty}{|y_\mu(x)|}=\limsup_{x\to\infty}{\left| \frac{c}{2a\omega}\sin{\omega x}\right| }=\frac{c}{2a\omega} \]
\item Αν $ a=0 $ τότε όλες οι λύσεις της ομογενούς εξίσωσης είναι φραγμένες και άρα:
\[ \limsup_{x\to\infty}{|y(x)|}=\limsup_{x\to\infty}{\left| c_1\cos{\omega x }+c_2\sin{\omega x }+\frac{c}{2a\omega}x\sin{\omega x}\right| }=+\infty \]
\end{rlist}
Έτσι συνοψίζοντας τις δύο περιπτώσεις θα έχουμε:
\[ \limsup_{x\to\infty}{|y(x)|}=\ccases{\frac{c}{2a\omega} & \textrm{, αν }\ a>0\\+\infty & \textrm{, αν }\ a=0} \]\epask
\begin{Askhshs}[B]
 \bmath{Να επιλυθεί η ομογενής γραμμική διαφορική εξίσωση
\[ x^2y''+xy'+\left(x^2-\frac{1}{4} \right)y=0\ \ ,\ \ 0<x<\pi  \]
αφού βρεθεί πρώτα μια λύση $y_1$ αυτής, της μορφής $ y_1(x)=x^a\sin{x},x\in(0,\pi) $ (όπου $ a $ είναι ένας πραγματικός αριθμός που θα πρέπει να προσδιοριστεί).}
\end{Askhshs}\mbox{}\\\\
\lysh
Αρχικά υπολογίζουμε τις παραγώγους έως και 2\tss{ης} τάξης της συνάρτησης $ y_1(x)=x^a\sin{x} $, οι οποίες θα έχουν ως εξής:
\begin{align*}
y_1'(x)&=ax^{a-1}\sin{x}+x^a\cos{x}\ \ \textrm{και}\\
y_1''(x)&=a(a-1)x^{a-2}\sin{x}+ax^{a-1}\cos{x}+ax^{a-1}\cos{x}-x^a\sin{x}=\\
&=a(a-1)x^{a-2}\sin{x}+2ax^{a-1}\cos{x}-x^a\sin{x}
\end{align*}
Αντικαθιστούμε τις παραγώγους αυτές στην αρχική διαφορική εξίσωση και παίρνουμε:
\begin{gather*}
x^2\left(a(a-1)x^{a-2}\sin{x}+2ax^{a-1}\cos{x}-x^a\sin{x} \right) +x\left(ax^{a-1}\sin{x}+x^a\cos{x} \right) +\left(x^2-\frac{1}{4} \right)x^a\sin{x}=0\Rightarrow\\
a(a-1)x^{a}\sin{x}+2ax^{a+1}\cos{x}-x^{a+2}\sin{x}  +ax^{a}\sin{x}+x^{a+1}\cos{x}+x^{a+2}\sin{x}-\frac{1}{4} x^{a}\sin{x}=0\Rightarrow\\
\left( a^2-\frac{1}{4}\right) x^a\sin{x}+(2a+1)x^{a+1}\cos{x}=0
\end{gather*}
Έτσι η συνάρτηση $ y_1(x)=x^a\sin{x}\ ,\ x\in(0,\pi) $ είναι μια λύση της εξίσωσης αν και μόνο αν $ \forall x\in(0,\pi) $ έχουμε
\[ a^2-\frac{1}{4}=0\ \ \textrm{και}\ \ 2a+1=0\Rightarrow a=-\frac{1}{2} \]
Κατά συνέπεια μια λύση της αρχικής διαφορικής εξίσωσης θα είναι η $ y_1(x)=x^{-1/2}\sin{x}\ ,\ x\in(0,\pi) $. Σύμφωνα τώρα με το θεώρημα ............. θα έχουμε $ \forall x\in(0,\pi) $:
\begin{align*}
y_2(x)&=y_1(x)\int_{\pi/2}^{x}{\frac{1}{y_1^2(t)}\cdot e^{-\dintt_{\!\!\pi/2}^{t}{\frac{a_1(s)}{a_2(s)}\d s}}\d t}=
x^{-1/2}\sin{x}\int_{\pi/2}^{x}{\frac{1}{t^{-1}\sin^2{t}}\cdot e^{-\dintt_{\!\!\pi/2}^{t}{\frac{s}{s^2}\d s}}\d t}=\\
&=x^{-1/2}\sin{x}\int_{\pi/2}^{x}{\frac{1}{t^{-1}\sin^2{t}}\cdot e^{-\left[ \log{s}\right]_{\pi/2}^{t} }\d t}=
x^{-1/2}\sin{x}\int_{\pi/2}^{x}{\frac{1}{t^{-1}\sin^2{t}}\cdot e^{-\log{t}+\log{\frac{\pi}{2}}}\d t}=\\
&=x^{-1/2}\sin{x}\int_{\pi/2}^{x}{\frac{1}{t^{-1}\sin^2{t}}\cdot \frac{1}{t}\cdot\frac{\pi}{2}\d t}=
x^{-1/2}\sin{x}\cdot \frac{\pi}{2}\int_{\pi/2}^{x}{\frac{1}{\sin^2{t}}\d t}=\\
&=x^{-1/2}\sin{x}\cdot \frac{\pi}{2}\cdot\left[-\cot{t} \right]_{\pi/2}^{x}=-\frac{\pi}{2}x^{-1/2}\cos{x} 
\end{align*}
Συνεπώς, μια δεύτερη λύση της διαφορικής μας εξίσωσης θα είναι η $ y_2(x)=-\frac{\pi}{2}x^{-1/2}\cos{x},\ x\in (0,\pi) $. Επιπλέον, οι λύσεις $ y_1,y_2 $ είναι γραμμικά ανεξάρτητες δηλαδή αποτελούν βασικό σύνολο λύσεων της εξίσωσης και άρα όλες οι λύσεις της θα δίνονται από τον τύπο:
\begin{align*}
y(x)&=c_1y_1+c_2y_2=c_1 x^{-1/2}\sin{x}+c_2\left(-\frac{\pi}{2}x^{-1/2}\cos{x} \right)=\\
&=x^{-1/2}\left(c_1\sin{x}+c'_2\cos{x}\right)\ ,\ x\in(0,\pi)
\end{align*}
όπου $ c_1 $ και $ c'_2=-\frac{\pi}{2}c_2 $ είναι αυθαίρετες σταθερές.\epask
\begin{Askhshs}[B]
\bmath{Ας θεωρήσουμε τη γραμμική διαφορική εξίσωση
\begin{equation}\label{b39_1}\tag{E}
y^{(n)}+a_{n-1}y^{(n-1)}+\ldots+a_1y'+a_0y=b
\end{equation}
όπου $ a_i, i=0,1,\ldots,n-1 $ είναι σταθερές και $ b $ είναι μια συνεχής συνάρτηση στο $ \mathbb{R} $. Ας είναι $ y_0 $ η λύση της αντίστοιχης ομογενούς γραμμικής διαφορικής εξίσωσης που πληροί τις αρχικές συνθήκες
\[ y_0(0)=y'_0(0)=\ldots=y_0^{(n-2)}(0)=0\ ,\ y_0^{(n-1)}(0)=1 \]
Να αποδειχθεί ότι
\[ y_\mu(x)=\int_{0}^{x}{y_0(x-t)\cdot b(t)\d t}\ ,\ x\in\mathbb{R} \] είναι η λύση της \eqref{b39_1} που πληροί τις αρχικές συνθήκες
\[ y_\mu(0)=y'_\mu(0)=\ldots=y_\mu^{(n-1)}(0)=0 \]}
\end{Askhshs}\mbox{}\\\\
\lysh
Γνωρίζουμε από τον ολοκληρωτικό λογισμό ότι ισχύει η παρακάτω σχέση:
\[ \frac{d}{dx}\left[\int_{a(x)}^{\beta(x)}{F(x,t)\d t} \right]=\int_{a(x)}^{\beta(x)}{\frac{d}{dx}F(x,t)\d t}+F(x,\beta(x))\cdot\frac{d}{dx}\beta(x)-F(x,a(x))\cdot\frac{d}{dx}a(x) \]
Εφαρμόζουμε τη σχέση αυτή ώστε να υπολογίσουμε διαδοχικά τις παραγώγους, μέχρι τάξης $ n $, της συνάρτησης $ y_\mu $. Για κάθε $ x\in\mathbb{R} $ λοιπόν θα έχουμε:
\begin{align*}
y'_\mu(x)&=\int_{0}^{x}{y'_0(x-t)\cdot b(t)\d t}+y_0(0)\cdot b(x)\cdot 1-y_0(x)\cdot b(0)\cdot 0=\int_{0}^{x}{y'_0(x-t)\cdot b(t)\d t}\\
y''_\mu(x)&=\int_{0}^{x}{y''_0(x-t)\cdot b(t)\d t}+y'_0(0)\cdot b(x)\cdot 1-y'_0(x)\cdot b(0)\cdot 0=\int_{0}^{x}{y''_0(x-t)\cdot b(t)\d t}\\
\vdots & \qquad\qquad\qquad\vdots\hspace{3cm}\vdots\hspace{5.5cm}\vdots\\
y^{(n)}_\mu(x)&=\int_{0}^{x}{y''_0(x-t)\cdot b(t)\d t}+y'_0(0)\cdot b(x)\cdot 1-y'_0(x)\cdot b(0)\cdot 0=\int_{0}^{x}{y^{(n)}_0(x-t)\cdot b(t)\d t}+b(x)
\end{align*}
Έτσι, αντικαθιστούμε τις παραγώγους αυτές στην αρχική διαφορική εξίσωση και θα έχουμε:
\begin{gather*}
y^{(n)}+a_{n-1}y^{(n-1)}+\ldots+a_1y'+a_0y=b\Rightarrow
\end{gather*}
\vspace{-1cm}
\begin{multline*}
\int_{0}^{x}{y^{(n)}_0(x-t)\cdot b(t)\d t}+b(x)+a_{n-1}\int_{0}^{x}{y^{(n-1)}_0(x-t)\cdot b(t)\d t}+\ldots+\\+a_1\int_{0}^{x}{y'_0(x-t)\cdot b(t)\d t}+a_0\int_{0}^{x}{y_0(x-t)\cdot b(t)\d t}=b(x)\Rightarrow
\end{multline*}
\vspace{-8mm}
\begin{gather*}
b(x)+\int_{0}^{x}{\left[y_0^{(n)}(x-t)+a_{n-1}y_0^{(n-1)}(x-t)+\ldots+a_1y_0'(x-t)+a_0y_0(x-t) \right]}=b(x)\Rightarrow\\
b(x)+0=b(x)
\end{gather*}
αφού η $ y_0 $ είναι λύση της αντίστοιχης ομογενούς εξίσωσης. Συνεπώς η συνάρτηση $ y_mu $ αποτελεί λύση της \eqref{b39_1} ενώ εύκολα βλέπουμε ότι πληροί τις αρχικές συνθήκες
\[ y_\mu(0)=y'_\mu(0)=\ldots=y_\mu^{(n-1)}(0)=0 \]\epask
\begin{Askhshs}[B]
\bmath{Ας είναι $ a,b,c $ και $ \kappa $ θετικοί αριθμοί τέτοιοι ώστε να ισχύει $ b\kappa\neq a\kappa^2+c $. Να αποδειχθεί ότι όλες οι λύσεις της γραμμικής διαφορικής εξίσωσης
\[ ay''+by'+cy=e^{-\kappa x} \]
τείνουν προς το μηδέν για $ x\to\infty $.}
\end{Askhshs}\mbox{}\\\\
\lysh
Το χαρακτηριστικό πολυώνυμο της αντίστοιχης ομογενούς διαφορικής εξίσωσης είναι το $ P(\lambda)=a\lambda^2+b\lambda+c $ με διακρίνουσα $ \varDelta=b^2-4ac $. Εξετάζουμε τις εξής περιπτώσεις:
\begin{enumerate}
\item Αν $ \varDelta>0 $ τότε το $ P(\lambda) $ έχει ρίζες τις $ \lambda_{1,2}=\frac{-b\pm\sqrt{\varDelta}}{2a} $.
\item Αν $ \varDelta=0 $ τότε το $ P(\lambda) $ έχει μοναδική ρίζα πολλαπλότητας $ 2 $ την $ \lambda_0=\frac{-b}{2a} $.
\item Αν $ \varDelta<0 $ τότε το $ P(\lambda) $ έχει ρίζες τις συζυγείς $ \lambda_{1,2}=\frac{-b\pm i\sqrt{-\varDelta}}{2a} $.
\end{enumerate}
Σε καθεμία από τις παραπάνω περιπτώσεις, τα πραγματικά μέρη των ριζών του πολυωνύμου είναι αρνητικά δηλαδή ισχύει
\[ \mathrm{Re}\lambda_1<0\ \ ,\ \ \mathrm{Re}\lambda_2<0\ \ \textrm{και}\ \ \mathrm{Re}\lambda_0<0 \]
οπότε κατά συνέπεια όλες οι λύσεις της αντίστοιχης ομογενούς εξίσωσης τείνουν προς το μηδέν καθώς $ x\to\infty $. Μένει να βρούμε μια μερική λύση της αρχικής εξίσωσης και να δείξουμε ότι και αυτή τείνει στο μηδέν καθώς $ x\to \infty $. Θέτουμε λοιπόν $ y=ze^{-\kappa x} $ και για κάθε $ x\in \mathbb{R} $ θα έχουμε:
\begin{gather}
y'=z'e^{-\kappa x}-\kappa ze^{-\kappa x}=\left(z'-\kappa z \right)e^{-\kappa x}\\
y''=\left(z''-\kappa z' \right)e^{-\kappa x}-\left(\kappa z'-\kappa^2 z \right)e^{-\kappa x}=\left(z''-2\kappa z'+\kappa^2 z\right)e^{-\kappa x}
\end{gather}
Έτσι με αντικατάσταση στην αρχική εξίσωση παίρνουμε:
\begin{gather}
a\left(z''-2\kappa z'+\kappa^2 z\right)+b\left(z'-\kappa z \right)+cz=1\Rightarrow\nonumber\\
az''-2a\kappa z'+a\kappa^2 z+bz'-b\kappa z+cz=1\Rightarrow\nonumber\\
az''-(2a\kappa+b)z'+\left( a\kappa^2-b\kappa+c\right)z=1\ ,\ x\in\mathbb{R}\label{b40_1}
\end{gather}
Από την υπόθεση γνωρίζουμε ότι $ a\kappa^2-bk+c\neq0 $ οπότε για τη μερική λύση $ z_\mu $ της \eqref{b40_1} θα έχουμε για κάθε $ x\in\mathbb{R} $:
\[ z_\mu(x)=d_0\Rightarrow z'_\mu(x)=0\Rightarrow z''_\mu(x)=0 \]
και άρα αυτή γίνεται:
\[ \left( a\kappa^2-b\kappa+c\right)d_0=1\Rightarrow d_0=\frac{1}{a\kappa^2-b\kappa+c} \]
Οπότε η μερική λύση της θα είναι η $ z_\mu(x)=\frac{1}{a\kappa^2-b\kappa+c},\ x\in\mathbb{R} $ από την οποία παίρνουμε τη μερική λύση $ y_\mu(x) $:
\[ y_\mu(x)=\frac{1}{a\kappa^2-b\kappa+c}\cdot e^{-\kappa x}\ ,\ x\in\mathbb{R} \]
της αρχικής εξίσωσης, με $ a,b,c $ και $ \kappa $ θετικούς αριθμούς. Για τη λύση αυτή, καθώς $ x\to\infty $ θα έχουμε:
\[ \lim_{x\to\infty}{y_\mu(x)}=\lim_{x\rightarrow\infty}{\frac{1}{a\kappa^2-b\kappa+c}\cdot e^{-\kappa x}}=0 \]
Επομένως όλες οι λύσεις της αρχικής διαφορικής εξίσωσης θα τείνουν προς το μηδέν καθώς $ x\to\infty $.\epask
\begin{Askhshs}[B]
\bmath{Με την αντικατάσταση $ z=\left( 1+x^2\right)y  $ να επιλυθεί η γραμμική διαφορική εξίσωση
\[ x^2\left( 1+x^2\right)y''+x\left( 3x^2+1\right)y'+\left( 1+x^2\right) y=x^2\log{x}\ \ ,\ \ x>0 \]}
\end{Askhshs}\mbox{}\\\\
\lysh
Χρησιμοποιώντας το μετασχηματισμό $ z=\left( 1+x^2\right)y\Rightarrow y=\frac{z}{1+x^2} $, για κάθε $ x>0 $ παίρνουμε τις παραγώγους:
\begin{align}
y'&=\frac{1}{1+x^2}z'-\frac{2x}{\left( 1+x^2\right)^2}z\\
y''&=\frac{1}{1+x^2}z''-\frac{2x}{\left( 1+x^2\right)^2}z'-\frac{2x}{\left( 1+x^2\right)^2}z'+\frac{6x^2-2}{\left( 1+x^2\right)^3}z=\\
&=\frac{1}{1+x^2}z''-\frac{4x}{\left( 1+x^2\right)^2}z'+\frac{6x^2-2}{\left( 1+x^2\right)^3}z
\end{align}
Έτσι η αρχική διαφορική εξίσωση μετασχηματίζεται ως εξής:
\begin{multline*}
x^2\left( 1+x^2\right)\left[ \frac{1}{1+x^2}z''-\frac{4x}{\left( 1+x^2\right)^2}z'+\frac{6x^2-2}{\left( 1+x^2\right)^3}z\right] +\\+x\left( 3x^2+1\right)\left[ \frac{1}{1+x^2}z'-\frac{2x}{\left( 1+x^2\right)^2}z\right] +z=x^2\log{x}\Rightarrow
\end{multline*}
\vspace{-5mm}
\begin{gather*}
x^2z''-\frac{4x^3}{1+x^2}z'+\frac{6x^4-2x^2}{\left( 1+x^2\right)^2}z+ \frac{3x^3+x}{1+x^2}z'-\frac{6x^4-2x^2}{\left( 1+x^2\right)^2}z+z=x^2\log{x}\Rightarrow\\
x^2z''-\frac{x^3+x}{1+x^2}z'+z=x^2\log{x}\Rightarrow\\
x^2z''-xz'+z=x^2\log{x}\ ,\ x>0
\end{gather*}
Η τελευταία διαφορική εξίσωση αποτελεί μια διαφορική εξίσωση Euler. Με το μετασχηματισμό $ t=\log{x} $ μετατρέπεται σε μια διαφορική εξίσωση 2\tss{ης} τάξης με σταθερούς συντελεστές. Προκύπτει λοιπόν για κάθε $ x>0 $ ότι
\begin{align}
x\frac{dz}{dx}&=x\frac{dz}{dt}\cdot\frac{dt}{dx}=x\frac{dz}{dt}\cdot\frac{1}{x}=\frac{dz}{dt}\quad\textrm{και}\label{b41:1}\\
x^2\frac{d^2z}{dx^2}&=x^2\frac{d}{dx}\left( \frac{dz}{dx}\right)=x^2\frac{d}{dx}\left( \frac{1}{x}\frac{dz}{dt}\right)=x^2\left[ -\frac{1}{x^2}\frac{dz}{dt}+\frac{1}{x}\frac{d}{dx}\left( \frac{dz}{dt}\right) \right]=\nonumber\\\label{b41:2}
&=x^2\left[ -\frac{1}{x^2}\frac{dz}{dt}+\frac{1}{x}\frac{d}{dt}\left( \frac{dz}{dx}\right) \right]=
x^2\left[ -\frac{1}{x^2}\frac{dz}{dt}+\frac{1}{x^2} \frac{d^2z}{dt^2} \right]=-\frac{dz}{dt}+\frac{d^2z}{dt^2}
\end{align}
Με τις σχέσεις \eqref{b41:1} και \eqref{b41:2} η εξίσωση Euler θα πάρει τη μορφή:
\begin{equation}\label{b41:3}
-\frac{dz}{dt}+\frac{d^2z}{dt^2}-\frac{dz}{dt}+z=e^{2t}t\Rightarrow \frac{d^2z}{dt^2}-2\frac{dz}{dt}+z=e^{2t}t\ ,\ t\in\mathbb{R}
\end{equation}
Η τελευταία είναι μια μη ομογενής γραμμική εξίσωση 2ης τάξης με σταθερούς συντελεστές $ a_2=1, a_1=-2 $ και $ a_0=1 $. Το χαρακτηριστικό πολυώνυμο της αντίστοιχης ομογενούς εξίσωσης θα είναι το \[ P(\lambda)=\lambda^2-2\lambda+1 \]
το οποίο έχει ρίζα την $ \lambda=1 $ με πολλαπλότητα $ 2 $. Συνεπώς οι συναρτήσεις 
\[ z_1(t)=e^t\quad\textrm{και}\quad z_2(t)=te^t\ ,\ t\in\mathbb{R} \]
οι οποίες είναι γραμμικά ανεξάρτητες θα αποτελούν ένα βασικό σύνολο λύσεων της ομογενούς εξίσωσης. Όλες οι λύσεις αυτής θα δίνονται από τον τύπο
\[ \tilde{z}(t)=c_1z_1(t)+c_2z_2(t)=c_1\cdot e^t+c_2\cdot te^t \]
όπου $ c_1,c_2 $ είναι αυθαίρετες σταθερές. Μένει να βρεθεί μια μερική λύση $ z_\mu(t) $ της μη ομογενούς εξίσωσης. Έτσι θέτουμε όπου $ z=ue^{2t} $ και θα έχουμε
\begin{gather}
z'=u'e^{2t}+2ue^{2t}=\left( u'+2u\right) e^{2t}\\
z''=\left( u''+2u'\right) e^{2t}+2\left( u'+2u\right) e^{2t}=\left(u''+4u'+4u\right)e^{2t} 
\end{gather}
Αντικαθιστώντας τις σχέσεις αυτές στη μή ομογενή εξίσωση θα προκύψει :
\begin{gather*}
\left(u''+4u'+4u\right)e^{2t}-\left(2u'+4u\right)e^{2t}+ue^{2t}=te^{2t}\Rightarrow\\
u''+2u'+u=t\ ,\ t\in\mathbb{R}
\end{gather*}
Μια μερική λύση της τελευταίας εξίσωσης είναι η $ u_\mu(t)=d_1t+d_0 $. Έτσι $ \forall t\in\mathbb{R} $ θα έχουμε $ u'_\mu(t)=d_1 $ και $ u''_\mu(t)=0 $. Με αντικατάσταση των σχέσεων αυτών η εξίσωση θα γίνει :
\[ 2d_1+d_1t+d_0=t\Rightarrow \ccases{d_1=1\\2d_1+d_0=0\Rightarrow d_0=-2} \]
Οι συντελεστές αυτοί μας δίνουν τη μερική λύση $ u_\mu(t)=t-2 $. Συνεπώς η μερική λύση της αρχικής μη ομογενούς θα έχει τη μορφή : 
\[ z_\mu(t)=u_\mu(t)e^{2t}=(t-2)e^{2t}\ ,\ t\in\mathbb{R} \]
Οπότε όλες οι λύσεις της διαφορικής εξίσωσης \eqref{b41:3} θα δίνονται από τον τύπο
\[ z(t)=c_1\cdot e^t+c_2\cdot te^t+(t-2)e^{2t}\ ,\ t\in\mathbb{R} \]
ο οποίος θέτοντας διαδοχικά όπου $ t=\log{x} $ 
και στη συνέχεια $ z=\left( 1+x^2\right)y $ μας δίνει τη γενική λύση της αρχικής διαφορικής εξίσωσης η οποία θα είναι:
\[ y(x)=c_1\frac{x}{1+x^2}+c_2\cdot \frac{x\log{x}}{1+x^2}+(\log{x}-2)\frac{x^2}{1+x^2}\ ,\ x>0 \]
όπου $ c_1,c_2 $ είναι αυθαίρετες σταθερές.\epask
\begin{Askhshs}[B]
\bmath{Με τη βοήθεια ενός μετασχηματισμού της μορφής $ t=x^a $ (όπου $ a $ είναι κατάλληλος αριθμός που θα πρέπει να βρεθεί), να επιλυθεί η γραμμική διαφορική εξίσωση
\[ x^4y''+2x^2(1+x)y'+y=\frac{1}{x^2} \ \,\ \ x>0 \]}
\end{Askhshs}\mbox{}\\\\
\lysh
Χρησιμοποιώντας το μετασχηματισμό $ t=x^a $, η πρώτη και η δεύτερη παράγωγος της συνάρτησης $ y $ θα έχουν ως εξής:
\[ y'=\dfrac{dy}{dx}=ax^{a-1}\frac{dy}{dt}\ \ \textrm{καθώς και}\ \ y''=\frac{d^2y}{dx^2}=a(a-1)x^{a-2}\frac{dy}{dt}+a^2 x^{2a-2}\frac{d^2y}{dt^2} \]
Αντικαθιστούμε τις παραγώγους αυτές στη διαφορική εξίσωση και αυτή θα γίνει
\begin{gather*}
x^4\left( a(a-1)x^{a-2}\frac{dy}{dt}+a^2 x^{2a-2}\frac{d^2y}{dt^2}\right) +2x^2(1+x)ax^{a-1}\frac{dy}{dt}+y=\frac{1}{x^2}\Rightarrow\\
a(a-1)x^{a+2}\frac{dy}{dt}+a^2 x^{2a+2}\frac{d^2y}{dt^2} +2a(1+x)x^{a+1}\frac{dy}{dt}+y=\frac{1}{x^2}\Rightarrow\\
a^2 x^{2a+2}\frac{d^2y}{dt^2} +\left(a^2x+ax+2a\right) x^{a+1}\frac{dy}{dt}+y=\frac{1}{x^2}\Rightarrow
\end{gather*}
Επιλέγοντας στην παραπάνω εξίσωση όπου $ a=-1 $ παίρνουμε το μετασχηματισμό $ t=\frac{1}{x} $ ενώ η τελευταία εξίσωση θα μετατραπεί σε μια γραμμική διαφορική εξίσωση 2\tss{ης} τάξης με σταθερούς συντελεστές και είναι η ακόλουθη:
\[ \frac{d^2y}{dt^2}-2\frac{dy}{dt}+y=t^2\ \ ,\ \ t>0 \]
Η αντίστοιχη ομογενής εξίσωση έχει χαρακτηριστικό πολυώνυμο το $ P(\lambda)=\lambda^2-2\lambda+1 $ με μοναδική διπλή ρίζα την $ \lambda=1 $ οπότε ένα βασικό σύνολο λύσεων είναι το $ \{y_1(t)=e^t, y_2(t)=te^t\}\ t>0 $. Όλες οι λύσεις της ομογενούς εξίσωσης θα δίνονται από τον τύπο
\[ \tilde{y}(t)=c_1e^t+c_2te^t\ ,\ t>0 \]
όπου $ c_1,c_2 $ είναι αυθαίρετες σταθερές. Θα αναζητήσουμε μια μερική λύση της μη ομογενούς εξίσωσης η οποία θα είναι της μορφής $ y_\mu(t)=at^2+\beta t+\gamma $. Αυτή θα μας δώσει $ y'_\mu(t)=2at+\beta $ και $ y''_\mu(t)=2a $ και έτσι παίρνουμε τη σχέση:
\[ 2a-4at-2\beta+at^2+\beta t+\gamma=t^2 \]
από την οποία προκύπτει $ a=1.\beta=4,\gamma=6 $. Η ζητούμενη μερική λύση θα είναι λοιπόν η $ y_\mu(t)=t^2+4t+6,\ t>0 $. Έτσι όλες οι λύσεις της αρχικής διαφορικής εξίσωσης θα δίνονται από τον παρακάτω τύπο:
\begin{gather*}
y(t)=c_1e^t+c_2te^t+t^2+4t+6\ \ ,\ \ t>0\ \ \ \textrm{ή ισοδύναμα}\\
y(x)=c_1e^{1/x}+c_2\frac{e^{1/x}}{x}+\left(\frac{1}{x}\right)^2+\frac{4}{x}+6\ \ ,\ \ x>0
\end{gather*}
όπου $ c_1,c_2 $ είναι αυθαίρετες σταθερές.\epask
\begin{Askhshs}[B]
\bmath{Ας είναι $ b $ και $ c $ πραγματικές σταθερές και έστω $ y $ μια λύση της ομογενούς γραμμικής διαφορικής εξίσωσης
\[ y''-2by'+cy=0 \] τέτοια ώστε $ y(0)=y(1)=0 $. Να αποδειχθεί ότι $ y(n)=0 $ για κάθε ακέραιο $ n $.}
\end{Askhshs}\mbox{}\\\\
\lysh
Το χαρακτηριστικό πολυώνυμο της ομογενούς διαφορικής εξίσωσης είναι το $ P(\lambda)=\lambda^2-2b\lambda+c $ με διακρίνουσα $ \varDelta=4b^2-4c=4(b^2-c) $. Διακρίνουμε τις εξής περιπτώσεις:
\begin{enumerate}
\item Αν $ \varDelta>0 $ τότε οι ρίζες του πολυωνύμου $ P(\lambda) $ θα είναι οι
\[ \lambda_1=b+\sqrt{b^2-c}\ \ \textrm{και}\ \ \lambda_2=b-\sqrt{b^2-c} \]
Έτσι οι συναρτήσεις που σχηματίζουν ένα βασικό σύνολο λύσεων της εξίσωσης θα είναι οι $ y_1(x)=e^{\lambda_1 x},\ x\in\mathbb{R} $ και $ y_2(x)=e^{\lambda_2 x},\ x\in\mathbb{R} $. Όλες οι λύσεις της ομογενούς εξίσωσης θα δίνονται από τον τύπο:
\[ y(x)=c_1e^{\lambda_1 x}+c_2 e^{\lambda_2 x}\ \ ,\ \ x\in\mathbb{R} \]
με $ c_1,c_2 $ αυθαίρετες σταθερές. Από την υπόθεση όμως έχουμε τις σχέσεις $ y(0)=0 $ και $ y(1)=0 $ οι οποίες μας δίνουν αντίστοιχα:
\[ \ccases{c_1+c_2=0\\c_1e^{\lambda_1}+c_2e^{\lambda_2}}\Rightarrow c_1=c_2=0 \]
Η λύση λοιπόν της εξίσωσης θα είναι η $ y(x)=0,\ x\in\mathbb{R} $ οπότε $ y(n)=0 $ για κάθε ακέραιο $ n $.
\item Ας υποθέσουμε ότι $ \varDelta=0 $. Τότε παίρνουμε μοναδική ρίζα του πολυωνύμου την $ \lambda_0=b $ πολλαπλότητας $ 2 $. Οι συναρτήσεις που αποτελούν ένα βασικό σύνολο λύσεων θα είναι οι $ y_1(x)=e^{bx},\ x\in\mathbb{R} $ και $ y_2(x)=xe^{bx},\ x\in\mathbb{R} $ και έτσι όλες οι λύσεις θα δίνονται από τη συνάρτηση:
\[ y(x)=c_1 e^{bx}+c_2 xe^{bx}\ \ ,\ \ x\in\mathbb{R} \]
με $ c_1,c_2 $ αυθαίρετες σταθερές. Από τις αρχικές σχέσεις $ y(0)=0 $ και $ y(1)=0 $ παίρνουμε ότι:
\[ \ccases{c_1=0\\c_1e^{b}+c_2e^{b}}\Rightarrow c_1=c_2=0 \]
Συνεπώς ισχύει $ y(x)=0,\ x\in\mathbb{R} $ οπότε $ y(n)=0 $ για κάθε ακέραιο $ n\in\mathbb{Z} $.
\item Τέλος αν $ \varDelta<0 $ τότε οι μιγαδικές ρίζες του πολυωνύμου θα είναι οι
\[ \lambda_1=b+i\sqrt{c-b^2}\ \ \textrm{και}\ \ \lambda_2=b-i\sqrt{c-b^2} \]
ενώ οι συναρτήσεις $ y_1(x)=e^{bx}\cos{\left(\sqrt{c-b^2}x \right) },\ x\in\mathbb{R} $ και $ y_2(x)=e^{bx}\sin{\left(\sqrt{c-b^2}x \right) },\ x\in\mathbb{R} $ αποτελούν βασικό σύνολο λύσεων της εξίσωσης. Όλες οι λύσεις της θα δίνονται από τον παρακάτω τύπο:
\[ y(x)=c_1e^{bx}\cos{\left(\sqrt{c-b^2}x \right) }+c_2e^{bx}\sin{\left(\sqrt{c-b^2}x \right) }\ \ ,\ \ x\in\mathbb{R} \]
με $ c_1,c_2 $ αυθαίρετες σταθερές. Από τις αρχικές σχέσεις τώρα της άσκησης παίρνουμε 
\[ \ccases{c_2=0\\c_1e^{b}\cos{\left(\sqrt{c-b^2} \right) }+c_2e^{b}\sin{\left(\sqrt{c-b^2} \right) }}\Rightarrow \ccases{c_2=0\\c_1=0\ \ \textrm{ή}\ \ \sqrt{c-b^2}=\kappa\pi} \]
Σε κάθε περίπτωση προκύπτει ότι $ y(x)=0,\ x\in\mathbb{R} $ οπότε $ y(n)=0 $ για κάθε ακέραιο $ n\in\mathbb{Z} $.
\end{enumerate}\mbox{}\\\\\\
\begin{Askhshs}[B]
\bmath{Ας είναι $ a,b,c $ και $ k $ θετικοί αριθμοί τέτοιοι ώστε να ισχύει $ bk\neq ak^2+c $. Να αποδειχθεί ότι όλες οι λύσεις της γραμμικής διαφορικής εξίσωσης
\[ ay''+by'+cy=xe^{-kx}\ \ ,\ \ x\in\mathbb{R} \] τείνουν προς το μηδέν για $ x\to\infty $.}
\end{Askhshs}\mbox{}\\\\
\lysh
Σχηματίζουμε την αντίστοιχη ομογενή εξίσωση $ ay''+by'+cy=0 $ της αρχικής και παίρνουμε το χαρακτηριστικό πολυώνυμο της το $ P(\lambda)=a\lambda^2_b\lambda+c $ με διακρίνουσα $ \varDelta=b^2-4ac $. Εξετάζουμε στη συνέχεια τις εξής περιπτώσεις:
\begin{enumerate}
\item Αν $ \varDelta>0 $ τότε το πολυώνυμο έχει ρίζες τις $ \lambda_1=\frac{-b+\sqrt{\varDelta}}{2a}\ \ \textrm{και}\ \ \lambda_2=\frac{-b-\sqrt{\varDelta}}{2a} $.
\item Αν $ \varDelta=0 $ τότε το πολυώνυμο έχει διπλή ρίζα την $ \lambda_0=\frac{-b}{2a} $.
\item Αν $ \varDelta>0 $ τότε το πολυώνυμο έχει ρίζες τις $ \lambda_1=\frac{-b+i\sqrt{-\varDelta}}{2a}\ \ \textrm{και}\ \ \lambda_2=\frac{-b-i\sqrt{-\varDelta}}{2a} $.
\end{enumerate}
Σε καθεμία από τις παραπάνω περιπτώσεις τα πραγματικά μέρη των ριζών είναι αρνητικά δηλαδή ισχύει $ Re(\lambda_1)<0,Re(\lambda_2)<0 $ και $ Re(\lambda_0)<0 $ και κατά συνέπεια όλες οι λύσεις της ομογενούς εξίσωσης θα τείνουν στο μηδέν καθώς $ x\to\infty $. Μένει να βρούμε μια λύση της αρχικής εξίσωσης και να δείξουμε ότι κι αυτή τείνει στο μηδέν για $ x\to\infty $. Θέτουμε $ y=ze^{kx} $ και παραγωγίζοντας παίρνουμε:
\[ y'=(z'-kz)e^{-kx}\ \ \textrm{και}\ \ y''=\left(z''-2kz'+k^2z \right)e^{-kx} \]
Με αντικατάσταση στην αρχική εξίσωση θα προκύψει:
\begin{gather}
a\left(z''-2kz'+k^2z \right)+b(z'-kz)+cy=x\Rightarrow\nonumber\\
az''-2akz'+ak^2z+bz'-bkz+cy=x\Rightarrow\nonumber\\
az''+(b-2ak)z'+\left(ak^2-bk+c \right)z=x\ ,\ x\in\mathbb{R} \label{b44:1}
\end{gather}
Από την υπόθεση γνωρίζουμε ότι ισχύει $ ak^2-bk+c\neq0 $. Έτσι θα αναζητήσουμε μια μερική λύση της \eqref{b44:1} της μορφής $ z_\mu=Ax+B $. Με παραγώγιση παίρνομε $ z_\mu'=A $ και $ z''_\mu=0 $ οπότε η τελευταία θα γίνει:
\begin{gather*}
(b-2ak)A+\left(ak^2-bk+c \right)(Ax+B)=x\Rightarrow\\
A=\frac{1}{ak^2-bk+c}\ \ \textrm{και}\ \ B=\frac{2ak-b}{\left(ak^2-bk+c\right)^2}
\end{gather*}
Η μερική λύση λοιπόν της \eqref{b44:1} θα είναι η $ z_\mu(x)=\frac{x}{a\kappa^2-b\kappa+c}+\frac{2ak-b}{\left(ak^2-bk+c\right)^2},\ x\in\mathbb{R} $ η οποία ύστερα από αντικατάσταση μας δίνει την αντίστοιχη μερική λύση $ y_\mu(x) $ της αρχικής εξίσωσης η οποία είναι:
\[ y_\mu(x)=\left( \frac{x}{a\kappa^2-b\kappa+c}+\frac{2ak-b}{\left(ak^2-bk+c\right)^2}\right) \cdot e^{-\kappa x}\ ,\ x\in\mathbb{R} \]
με $ a,b,c $ και $ \kappa $ θετικούς αριθμούς. Για τη λύση αυτή, καθώς $ x\to\infty $ θα έχουμε:
\[ \lim_{x\to\infty}{y_\mu(x)}=\lim_{x\rightarrow\infty}{\frac{Ax+B}{e^{kx}}}\eq{DLH}\lim_{x\rightarrow\infty}{\frac{A}{ke^{kx}}}=\frac{1}{a\kappa^2-b\kappa+c}\lim_{x\rightarrow\infty}\frac{1}{ke^{kx}}=0 \]
Άρα όλες οι λύσεις της αρχικής διαφορικής εξίσωσης θα τείνουν προς το μηδέν καθώς $ x\to\infty $.\epask
\begin{Askhshs}[B]
\bmath{Έστω $ \kappa $ ένας θετικός ακέραιος. Να βρεθούν οι τιμές τις πραγματικής παραμέτρου $ c $ έτσι ώστε η ομογενής γραμμική διαφορική εξίσωση 
\[ y''-2cy'+y=0 \]
να έχει μια μη μηδενική λύση $ y $ τέτοια ώστε $ y(0)=y(2\kappa\pi)=0 $.}
\end{Askhshs}\mbox{}\\\\
\lysh
Το χαρακτηριστικό πολυώνυμο της διαφορικής μας εξίσωσης είναι το $ P(\lambda)=\lambda^2-2c\lambda+1 $ του οποίου η διακρίνουσα ισούται με $ \varDelta=4\left( c^2-1\right) $. Διακρίνουμε τις εξής περιπτώσεις:
\begin{rlist}
\item Αν $ \varDelta>0 $ δηλαδή αν $ c^2>1 $ τότε το πολυώνυμο έχει δύο πραγματικές ρίζες τις
\[ \lambda_1=c+\sqrt{c^2-1}\ \ \textrm{και}\ \ \lambda_2=c-\sqrt{c^2-1} \]
Έτσι οι συναρτήσεις $ y_1(x)=e^{\lambda_1 x}\ ,\ x\in\mathbb{R} $ και $ y_2(x)=e^{\lambda_2 x}\ ,\ x\in\mathbb{R} $ αποτελούν ένα βασικό σύνολο λύσεων της αρχικής εξίσωσης. Συνεπώς όλες οι λύσεις της θα δίνονται από τον τύπο
\[ y(x)=c_1y_1(x)+c_2y_2(x)\ \ ,\ \ x\in\mathbb{R} \]
όπου $ c_1,c_2 $ είναι αυθαίρετες σταθερές. Επιπλέον από την υπόθεση της άσκησης έχουμε τις συνθήκες $ y(0)=y(2\kappa\pi)=0 $ οι οποίες μας δίνουν
\begin{gather*}
\ccases{
y(0)&=0\ \\y(2\kappa\pi)&=0\ } \Rightarrow 
\ccases{
c_1+c_2&=0 \\c_1 e^{2\kappa\pi\lambda_1}+c_2 e^{2\kappa\pi\lambda_2}&=0\ }\Rightarrow 
\ccases{
c_1&=-c_2 \\c_1 \left( e^{2\kappa\pi\lambda_1}- e^{2\kappa\pi\lambda_2}\right) &=0\ }\Rightarrow \\
c_1=0\ \textrm{ή}\ \ e^{2\kappa\pi\lambda_1}=e^{2\kappa\pi\lambda_2}\Rightarrow \lambda_1=\lambda_2
\end{gather*}
Η σχέση $ \lambda_1=\lambda_2 $ δεν ισχύει καθώς μας δίνει $ c^2-1=0 $ ενώ έχουμε $ c^2-1>0 $. Έτσι παίρνουμε $ c_1=c_2=0 $. Καταλήγουμε όμως στη μηδενική λύση $ y(x)=0 $ που δεν είναι δεκτή άρα δεν υπάρχει $ c\in\mathbb{R} $ έτσι ώστε $ \varDelta>0 $ που να πληροί τις συνθήκες της άσκησης.
\item Αν $ \varDelta=0\Rightarrow c=\pm 1 $ τότε το χαρακτηριστικό πολυώνυμο $ P(\lambda) $ έχει διπλή ρίζα την $ \lambda_0=c $. Έτσι οι συναρτήσεις $ y_1(x)=e^{\lambda_0 x}\ ,\ x\in\mathbb{R} $ και $ y_2(x)=xe^{\lambda_0 x}\ ,\ x\in\mathbb{R} $ αποτελούν ένα βασικό σύνολο λύσεων της εξίσωσης. Όλες οι λύσεις της θα δίνονται από τον παρακάτω τύπο:
\[ y(x)=c_1y_1(x)+c_2y_2(x)=e^{\lambda_0 x}(c_1+c_2 x)\ \ ,\ \ x\in\mathbb{R} \]
με $ c_1,c_2 $ να είναι αυθαίρετες σταθερές. Όπως και προηγουμένως, από τις συνθήκες της άσκησης θα έχουμε:
\begin{gather*}
\ccases{
y(0)&=0\ \\y(2\kappa\pi)&=0\ } \Rightarrow 
\ccases{
c_1&=0 \\e^{2\kappa\pi\lambda_0 }(c_1+2\kappa\pi c_2)&=0\ }\Rightarrow c_1=c_2=0
\end{gather*}
Έτσι παίρνουμε πάλι τη μηδενική λύση $ y(x)=0 $ και άρα και στην περίπτωση όπου $ \varDelta=0 $ δεν υπάρχει πραγματικός $ c\in\mathbb{R} $ που να πληροί τις συνθήκες της άσκησης.
\item Τέλος εξετάζουμε την περίπτωση να ισχύει $ \varDelta<0 $ που μας δίνει, για το πολυώνυμο $ P(\lambda) $, τις συζυγείς μιγαδικές ρίζες
\[ \lambda_1=c+i\sqrt{1-c^2}\ \ \textrm{και}\ \ \lambda_2=c-i\sqrt{1-c^2} \]
Σχηματίζονται έτσι οι συναρτήσεις $ y_1(x)=e^{cx}\cos{\left(\sqrt{1-c^2}x \right) }\ ,\ x\in\mathbb{R} $ και $ y_2(x)=e^{cx}\sin{\left(\sqrt{1-c^2}x \right) }$, $ x\in\mathbb{R} $ οι οποίες αποτελούν ένα βασικό σύνολο λύσεων της ομογενούς εξίσωσης. Όλες οι λύσεις της θα δίνονται από τον τύπο
\[ y(x)=c_1y_1(x)+c_2y_2(x)=c_1e^{cx}\cos{\left(\sqrt{1-c^2}x \right) }+c_2e^{cx}\sin{\left(\sqrt{1-c^2}x \right)}\ ,\ x\in\mathbb{R} \]
με $ c_1,c_2 $ να είναι αυθαίρετες σταθερές. Οι συνθήκες της άσκησης μας δίνουν ξανά:
\begin{gather*}
\ccases{
y(0)&=0\ \\y(2\kappa\pi)&=0\ } \Rightarrow 
\ccases{
c_1&=0 \\c_1e^{2\kappa\pi c}\cos{\left(2\kappa\pi\sqrt{1-c^2} \right) }+c_2e^{2\kappa\pi c}\sin{\left(2\kappa\pi\sqrt{1-c^2} \right)}&=0\ }\Rightarrow\\
c_2e^{2\kappa\pi c}\sin{\left(2\kappa\pi\sqrt{1-c^2} \right)}=0\Rightarrow c_2=0\ \ \textrm{ή}\ \ e^{2\kappa\pi c}\sin{\left(2\kappa\pi\sqrt{1-c^2} \right)}=0\Rightarrow\\
c_2=0\ \ \textrm{ή}\ \ 2\kappa\pi\sqrt{1-c^2}=\mu\pi\ \ ,\ \mu\in\mathbb{N}\Rightarrow 
c_2=0\ \ \textrm{ή}\ \ c=\pm\frac{\sqrt{4\kappa^2-\mu^2}}{2\kappa}\ \ ,\ \mu\in\mathbb{N}
\end{gather*}
Η περίπτωση $ c_1=c_2=0 $ απορρίπτεται διότι μας οδηγεί ξανά στη μηδενική λύση άρα οι ζητούμενες τιμές της παραμέτρου $ c $ θα είναι
\[ c=\pm\frac{\sqrt{4\kappa^2-\mu^2}}{2\kappa}\ \ ,\ \kappa,\mu\in\mathbb{N} \]
\end{rlist}\mbox{}\epask
\begin{Askhshs}[B]
\bmath{Δίνεται η γραμμική διαφορική εξίσωση
\begin{equation}\label{b46:1}\tag{$ E $}
y''+2ay'+\beta y=\Phi
\end{equation}
όπου $ a $ και $ \beta $ είναι πραγματικοί αριθμοί με $ a>0 $ και $ \beta>a^2 $ και $ \Phi $ είναι μια συνεχής πραγματική συνάρτηση στο διάστημα $ [0,+\infty) $. Να αποδειχθεί ότι αν $ \lim_{x\to\infty}{\Phi(x)}=0 $, τότε για κάθε λύση $ y $ της \eqref{b46:1} ισχύει
\[ \lim_{x\to\infty}{y(x)}=\lim_{x\to\infty}{y'(x)}=0 \]}
\end{Askhshs}\mbox{}\\\\
\lysh
Η αντίστοιχη ομογενής διαφορική εξίσωση θα είναι η
\begin{equation}\label{b46:2}\tag{$ E_0 $}
y''+2ay'+\beta y=0\ \ ,\ \ x\geq0
\end{equation}
της οποίας το χαρακτηριστικό πολυώνυμο είναι το $ P(\lambda)=\lambda^2+2a\lambda+\beta $ με διακρίνουσα $ \varDelta=4a^2-4\beta=4\left( a^2-\beta\right)<0 $ αφού από την υπόθεση γνωρίζουμε ότι $ \beta>a^2 $. Οι ρίζες 
\[ \lambda_1=-a+i\sqrt{\beta-a^2}\ \ \textrm{και}\ \ \lambda_2=-a-i\sqrt{b-a^2} \] 
μας δίνουν τις συναρτήσεις $ y_1(x)=e^{-ax}\cos\left(\sqrt{\beta-a^2}x \right)\ ,\ x\geq0 $ και $ y_2(x)=e^{-ax}\sin\left(\sqrt{\beta-a^2}x \right)\ ,\ x\geq0 $ οι οποίες αποτελούν ένα βασικό σύνολο λύσεων της ομογενούς εξίσωσης \eqref{b46:2}. Όλες οι λύσεις της θα δίνονται από τον τύπο:
\[ \tilde{y}(x)=c_1e^{-ax}\cos\left(\sqrt{\beta-a^2}x \right)+c_2e^{-ax}\sin\left(\sqrt{\beta-a^2}x \right)\ \ ,\ \ x\geq0 \]
όπου $ c_1,c_2 $ είναι αυθαίρετες σταθερές. Έτσι θα έχουμε ότι:
\[ \lim_{x\to\infty}{\tilde{y}(x)}=\lim_{x\to\infty}\left[c_1e^{-ax}\cos\left(\sqrt{\beta-a^2}x \right)+c_2e^{-ax}\sin\left(\sqrt{\beta-a^2}x \right)\right]=0 \]
μιας και η παραπάνω συνάρτηση αποτελείται από γινόμενα της μηδενικής $ e^{-ax} $ επί των φραγμένων συναρτήσεων $ \cos $ και $ \sin $. Εύκολα παρατηρούμε επίσης ότι $ \lim_{x\to\infty}{\tilde{y}'(x)}=0 $. Αρκεί να βρούμε μια μερική λύση $ y_\mu $ της μη ομογενούς εξίσωσης \eqref{b46:1} και να δείξουμε ότι $ \lim_{x\to\infty}{y_\mu(x)}=0 $ καθώς και $ \lim_{x\to\infty}{y'_\mu(x)}=0 $. Έτσι λοιπόν για κάθε $ x\in[0,+\infty) $ θα έχουμε:
\begin{gather*}
W(y_1,y_2)(x)=\begin{vmatrix}
e^{\lambda_1x} & e^{\lambda_2x}\\\lambda_1e^{\lambda_1x} & \lambda_2e^{\lambda_2x}
\end{vmatrix}=(\lambda_2-\lambda_1)e^{(\lambda_1+\lambda_2)x}\neq 0\\
W_1(y_1,y_2)(x)=\begin{vmatrix}
0 & e^{\lambda_2x}\\1 & \lambda_2e^{\lambda_2x}
\end{vmatrix}=-e^{\lambda_2x}\ \ \textrm{και}\ \ W_2(y_1,y_2)(x)=\begin{vmatrix}
e^{\lambda_1x} & 0\\\lambda_1e^{\lambda_1x} & 1
\end{vmatrix}=e^{\lambda_1x}
\end{gather*}
Επομένως η ζητούμενη μερική λύση θα είναι η
\begin{align*}
y_\mu(x)&=y_1(x)\int_{0}^{x}{\frac{W_1(y_1,y_2)(t)}{W(y_1,y_2)(t)}\cdot\Phi(t)\d t}+y_2(x)\int_{0}^{x}{\frac{W_2(y_1,y_2)(t)}{W(y_1,y_2)(t)}\cdot\Phi(t)\d t}=\\
&=e^{\lambda_1x}\int_{0}^{x}{\frac{-e^{\lambda_2t}}{(\lambda_2-\lambda_1)e^{(\lambda_1+\lambda_2)t}}\cdot\Phi(t)\d t}+e^{\lambda_2x}\int_{0}^{x}{\frac{e^{\lambda_1t}}{(\lambda_2-\lambda_1)e^{(\lambda_1+\lambda_2)t}}\cdot\Phi(t)\d t}=\\
&=\frac{1}{\lambda_2-\lambda_1}\left[ -e^{\lambda_1x}\int_{0}^{x}{e^{-\lambda_1t}\cdot\Phi(t)\d t}+e^{\lambda_2x}\int_{0}^{x}{e^{-\lambda_2t}\cdot\Phi(t)\d t}\right] 
\end{align*}
Έτσι λοιπόν για κάθε $ x\geq0 $ θα έχουμε ότι:
\begin{align*}
|y_\mu(x)(\lambda_2-\lambda_1)|&=\left| -e^{\lambda_1x}\int_{0}^{x}{e^{-\lambda_1t}\cdot\Phi(t)\d t}+e^{\lambda_2x}\int_{0}^{x}{e^{-\lambda_2t}\cdot\Phi(t)\d t}\right|\leq\\
&\leq \left| -e^{\lambda_1x}\right| \int_{0}^{x}{\left| e^{-\lambda_1t}\right| \cdot\left| \Phi(t)\right| \d t}+\left| e^{\lambda_2x}\right| \int_{0}^{x}{\left| e^{-\lambda_2t}\right| \cdot\left| \Phi(t)\right| \d t}=\\
&=e^{\mathrm{Re}(\lambda_1)x} \int_{0}^{x}{e^{-\mathrm{Re}(\lambda_1)t} \cdot\left| \Phi(t)\right| \d t}+e^{\mathrm{Re}(\lambda_2)x} \int_{0}^{x}{e^{-\mathrm{Re}(\lambda_2)t} \cdot\left| \Phi(t)\right|\d t}=\\
&=e^{-ax} \int_{0}^{x}{e^{at} \cdot\left| \Phi(t)\right| \d t}+e^{-ax} \int_{0}^{x}{e^{at} \cdot\left| \Phi(t)\right|\d t}=2\frac{\int_{0}^{x}{e^{at} \cdot\left| \Phi(t)\right|\d t}}{e^{ax}}\ ,\ \textrm{με }a>0
\end{align*}
Στο σημείο αυτό εξετάζουμε τις εξής περιπτώσεις για την ποσότητα $ \int_{0}^{x}{e^{at} \cdot\left| \Phi(t)\right|\d t} $:
\begin{itemize}
\item Αν $ \int_{0}^{x}{e^{at} \cdot\left| \Phi(t)\right|\d t}<+\infty $, τότε παίρνουμε $ \lim_{x\to\infty}{\frac{\int_{0}^{x}{e^{at} \cdot\left| \Phi(t)\right|\d t}}{e^{ax}}}=0 $.
\item Αν $ \int_{0}^{x}{e^{at} \cdot\left| \Phi(t)\right|\d t}=+\infty $, τότε εφαρμόζοντας τον κανόνα του De L'Hospital θα έχουμε ότι  \[ \lim_{x\to\infty}{\frac{\int_{0}^{x}{e^{at} \cdot\left| \Phi(t)\right|\d t}}{e^{ax}}}=\lim_{x\to\infty}{\frac{e^{ax} \cdot\left| \Phi(x)\right|}{e^{ax}}}=\lim_{x\to\infty}{\left| \Phi(x)\right|}=0 \]
αφού σε κάθε περίπτωση ισχύει ότι $ \lim_{x\to\infty}{\Phi(x)}=0 $ σύμφωνα με τα δεδομένα της υπόθεσης.
\end{itemize}
Επομένως $ \lim_{x\to\infty}{y_\mu(x)}=0 $ οπότε αποδεικνύεται ότι όλες οι λύσεις της \eqref{b46:1} τείνουν στο μηδέν καθώς $ x\to\infty $. Για την παράγωγο $ y'_{\mu}(x) $ της μερικής λύσης θα ισχύει για κάθε $ x\geq 0 $ ότι:
\begin{align*}
y'_{\mu}(x)&=\frac{1}{\lambda_2-\lambda_1}\left[ -\lambda_1 e^{\lambda_1x}\int_{0}^{x}{e^{-\lambda_1t}\cdot\Phi(t)\d t}-e^{-\lambda_1x}\cdot e^{-\lambda_1x}\cdot\Phi(x)+\right.\\
&\phantom{\frac{1}{\lambda_2-\lambda_1}[]+}+\left.\lambda_2 e^{\lambda_2x}\int_{0}^{x}{e^{-\lambda_2t}\cdot\Phi(t)\d t}+e^{-\lambda_2x}\cdot e^{-\lambda_2x}\cdot\Phi(x)\right]=\\
&=\frac{1}{\lambda_2-\lambda_1}\left[ -\lambda_1 e^{\lambda_1x}\int_{0}^{x}{e^{-\lambda_1t}\cdot\Phi(t)\d t}+\lambda_2 e^{\lambda_2x}\int_{0}^{x}{e^{-\lambda_2t}\cdot\Phi(t)\d t}\right]
\end{align*}
Έτσι αν εργαστούμε με τον ίδιο τρόπο όπως πριν παίρνουμε ότι και $ \lim_{x\to\infty}{y'_\mu(x)}=0 $ επομένως οι παράγωγοι $ y' $ όλων των λύσεων της \eqref{b46:1} τείνουν προς το μηδέν για $ x\to\infty $.\\\\
\begin{Askhshs}[B]
\bmath{Δίνεται η γραμμική διαφορική εξίσωση
\begin{equation}\label{b47:1}\tag{$ E $}
y''+2ay'+\beta y=\Phi
\end{equation}
όπου $ a $ και $ \beta $ είναι πραγματικοί αριθμοί με $ a>0 $ και $ \beta>a^2 $ και $ \Phi $ είναι μια συνεχής και φραγμένη συνάρτηση στο διάστημα $ [0,+\infty) $. Να αποδειχθεί ότι όλες οι λύσεις της \eqref{b47:1} είναι φραγμένες στο $ [0,+\infty) $. Επίσης να αποδειχθεί ότι οι παράγωγοι των λύσεων της \eqref{b47:1} είναι φραγμένες στο $ [0,+\infty) $.}
\end{Askhshs}\mbox{}\\\\
\lysh
Αποδείξαμε στην άσκηση Β.46 ότι όλες οι λύσεις της αντίστοιχης ομογενούς τείνουν στο μηδέν όταν $ x\to\infty $ συνεπώς είναι και φραγμένες στο διάστημα $ [0,+\infty) $. Αρκεί λοιπόν να δείξουμε ότι η και η μερική λύση $ y_\mu $ της άσκησης Β.46 είναι φραγμένη, με την προϋπόθεση βέβαια ότι η $ \Phi $ είναι φραγμένη στο $ [0,+\infty) $. Θεωρούμε λοιπόν ότι $ |\Phi(x)|\leq M,\ M>0 $ οπότε $ \forall x\geq 0 $ θα έχουμε:
\begin{align*}
|\lambda_2-\lambda_1|\cdot|y_\mu(x)|&\leq 2e^{-ax} \int_{0}^{x}{e^{at} \cdot\left| \Phi(t)\right|\d t}\leq 2Me^{-ax} \int_{0}^{x}{e^{at}\d t}=\\
&=2Me^{-ax}\cdot\frac{1}{a}\left[e^{ax} \right]_{0}^{x} =\frac{2M}{a}e^{-ax}\left( e^{ax}-1\right)=\\
&=\frac{2M}{a}-\frac{2M}{a}e^{-ax}\leq \frac{2M}{a}
\end{align*}
Επομένως η λύση $ y_\mu $ είναι φραγμένη στο διάστημα $ [0,\infty) $ και άρα όλες οι λύσεις της αρχικής εξίσωσης $ \eqref{b47:1} $ είναι φραγμένες στο $ [0,\infty) $. Όσον αφορά την $ y'_\mu(x) $ αποδείξαμε στην άσκηση Β.46 ότι $ \lim_{x\to\infty}{\tilde{y}'(x)}=0 $ οπότε η συνάρτηση $ \tilde{y}'(x) $ είναι φραγμένη στο $ [0,+\infty) $. Έτσι αρκεί να δείξουμε ότι και η $ y'_\mu(x) $ είναι φραγμένη στο $ [0,+\infty) $. Εργαζόμενοι όπως πριν καταλήγουμε στο ζητούμενο.\epask
\begin{Askhshs}[B]
\bmath{Έστω η γραμμική διαφορική εξίσωση
\begin{equation}\label{b48:1}\tag{$ E $}
ay''+by'+cy=\Phi
\end{equation}
όπου $ a,b,c $ είναι θετικοί πραγματικοί αριθμοί και $ \Phi $ είναι μια συνεχής συνάρτηση στο διάστημα $ [0,\infty) $ τέτοια ώστε $ \lim_{x\to\infty}{\Phi(x)}=0 $. Να αποδειχθεί ότι όλες οι λύσεις της \eqref{b48:1} τείνουν προς το μηδέν για $ x\to\infty $.}
\end{Askhshs}\mbox{}\\\\
\lysh
Η αντίστοιχη ομογενής διαφορική εξίσωση είναι η
\begin{equation}\label{b48:2}\tag{$ E_0 $}
ay''+by'+cy=0\ \ ,\ \ χ\geq 0
\end{equation}
της οποίας το χαρακτηριστικό πολυώνυμο είναι το $ P(\lambda)=a\lambda^2+b\lambda+c $ με διακρίνουσα $ \varDelta=b^2-4ac $. Διακρίνουμε τώρα τις εξής περιπτώσεις:
\begin{rlist}
\item Αν $ \varDelta>0 $, τότε το $ P(\lambda) $ έχει ρίζες τις
\[ \lambda_1=\frac{-b+\sqrt{\varDelta}}{2a}\ \ \textrm{και}\ \ \lambda_2=\frac{-b-\sqrt{\varDelta}}{2a} \]
Αυτές μας δίνουν τις συναρτήσεις $ y_1(x)=e^{\lambda_1x},\ x\geq0 $ και $ y_2(x)=e^{\lambda_2x},\ x\geq0 $ οι οποίες αποτελούν ένα βασικό σύνολο λύσεων της \eqref{b48:2}. Όλες οι λύσεις της θα δίνονται από τον τύπο:
\[ \tilde{y}(x)=c_1e^{\lambda_1x}+c_2e^{\lambda_2x}\ \ ,\ \ x\geq0 \]
όπου $ c_1,c_2 $ είναι αυθαίρετες σταθερές.
\item Αν $ \varDelta>0 $, τότε το $ P(\lambda) $ έχει διπλή ρίζα την $ \lambda_0=-\frac{b}{2a} $ και έτσι όλες οι λύσεις της \eqref{b48:2} θα δίνονται από τον τύπο
\[ \tilde{y}(x)=c_1e^{\lambda_0x}+c_2xe^{\lambda_0x}\ \ ,\ \ x\geq0 \]
όπου $ c_1,c_2 $ είναι αυθαίρετες σταθερές.
\item Τέλος αν $ \varDelta<0 $ το πολυώνυμο έχει συζυγείς μιγαδικές ρίζες τις
\[ \lambda_1=\frac{-b+i\sqrt{-\varDelta}}{2a}\ \ \textrm{και}\ \ \lambda_2=\frac{-b-i\sqrt{-\varDelta}}{2a} \]
και έτσι όλες οι λύσεις της ομογενούς εξίσωσης δίνονται από τον τύπο
\[ \tilde{y}(x)=c_1e^{\lambda_1x}+c_2e^{\lambda_2x}\ \ ,\ \ x\geq0 \]
όπου $ c_1,c_2 $ είναι αυθαίρετες σταθερές.
\end{rlist}
Σε κάθε περίπτωση οι ρίζες του $ P(\lambda) $ έχουν αρνητικό πραγματικό μέρος και άρα όλες οι λύσεις $ \tilde{y}(x) $ τείνουν στο μηδέν καθώς $ x\to\infty $. Αρκεί να δείξουμε ότι υπάρχει μια μερική λύση $ y_\mu $ της \eqref{b48:1} τέτοια ώστε $ \lim_{x\to\infty}{y_\mu(x)}=0 $. Εργαζόμαστε όπως και στην άσκηση Β.46 και φτάνουμε στο ζητούμενο.\epask
\begin{Askhshs}[B]
\bmath{Να προσδιοριστούν όλοι οι πραγματικοί αριθμοί $ L>1 $, έτσι ώστε η ομογενής γραμμική διαφορική εξίσωση 
\[ x^2y''+y=0\ \ ,\ \ 1\leq x\leq L \]
να έχει μη μηδενικές λύσεις $ y $ που να πληρούν τις συνθήκες $ y(1)=y(L)=0 $.}
\end{Askhshs}\mbox{}\\\\
\lysh
Η αρχική διαφορική εξίσωση είναι μια εξίσωση Euler. Χρησιμοποιώντας το μετασχηματισμό $ t=\log{x},\ x\in[1,L] $ παίρνουμε, ύστερα από παραγώγιση
\begin{align}
x\diff{y}{x}&=x\diff{y}{t}\cdot\diff{t}{x}=x\diff{y}{t}\cdot\frac{1}{x}=\diff{y}{t}\\
x^2\diff[2]{y}{x}&=x^2\frac{d}{dx}\left( \frac{dy}{dx}\right)=x^2\frac{d}{dx}\left( \frac{1}{x}\frac{dy}{dt}\right)=x^2\left[ -\frac{1}{x^2}\frac{dy}{dt}+\frac{1}{x}\frac{d}{dx}\left( \frac{dy}{dt}\right) \right]=\nonumber\\
&=x^2\left[ -\frac{1}{x^2}\frac{dy}{dt}+\frac{1}{x}\frac{d}{dt}\left( \frac{dy}{dx}\right) \right]=
x^2\left[ -\frac{1}{x^2}\frac{dy}{dt}+\frac{1}{x^2} \frac{d^2y}{dt^2} \right]=-\frac{dy}{dt}+\frac{d^2y}{dt^2}\label{b49:1}
\end{align}
Αντικαθιστούμε την \eqref{b49:1} στην αρχική εξίσωση αυτή μετατρέπεται στην ακόλουθη ομογενή γραμμική διαφορική εξίσωση 2\tss{ης} τάξης με σταθερούς συντελεστές:
\begin{equation}\label{b49:2}
\diff[2]{y}{t}-\diff{y}{t}+y=0
\end{equation}
με $ t\in[0,\log{L}] $ και $ L>1 $. Το χαρακτηριστικό πολυώνυμο της τελευταίας εξίσωσης είναι το $ P(\lambda)=\lambda^2-\lambda+1 $ με ρίζες τις
\[ \lambda_1=\frac{1+i\sqrt{3}}{2}\ \ \textrm{και}\ \ \lambda_2=\frac{1-i\sqrt{3}}{2} \]
Σχηματίζονται έτσι οι συναρτήσεις $ y_1(t)=e^{\frac{t}{2}}\cos{\left( \frac{\sqrt{3}}{2}t\right) },\ t\in[0,\log{L}] $ και $ y_2(t)=e^{\frac{t}{2}}\sin{\left( \frac{\sqrt{3}}{2}t\right) },\ t\in[0,\log{L}] $ οι οποίες αποτελούν βασικό σύνολο λύσεων της \eqref{b49:2}. Όλες οι λύσεις της θα δίνονται από τον τύπο:
\[ y(t)=c_1e^{\frac{t}{2}}\cos{\left( \frac{\sqrt{3}}{2}t\right) }+c_2e^{\frac{t}{2}}\sin{\left( \frac{\sqrt{3}}{2}t\right) }\ \ ,\ \ t\in[0,\log{L}] \]
με $ L>1 $ και $ c_1,c_2 $ να είναι αυθαίρετες σταθερές. Χρησιμοποιώντας ξανά τον αρχικό μετασχηματισμό, η παραπάνω συνάρτηση γράφεται ισοδύναμα στη μορφή
\[ y(x)=c_1\sqrt{x}\cos{\left( \frac{\sqrt{3}}{2}\log{x}\right) }+c_2\sqrt{x}\sin{\left( \frac{\sqrt{3}}{2}\log{x}\right) }\ \ ,\ \ x\in[1,L] \]
Σύμφωνα τώρα με την υπόθεση της άσκησης, οι συνοριακές συνθήκες $ y(1)=y(L)=0 $ μας δίνουν:
\begin{align*}
\ccases{y(1)=0\\y(L)=0}&\!\!\!\!\Rightarrow \ccases{c_1=0\\c_1\sqrt{L}\cos{\left( \frac{\sqrt{3}}{2}\log{L}\right) }+c_2\sqrt{L}\sin{\left( \frac{\sqrt{3}}{2}\log{L}\right) }=0}\!\!\!\!\Rightarrow\\
&\!\!\!\!\Rightarrow\ccases{c_1=0\\c_2=0\ \textrm{ή}\ \sin{\left( \frac{\sqrt{3}}{2}\log{L}\right) }=0}\!\!\!\!\Rightarrow\ccases{c_1=0\\c_2=0\ \textrm{ή}\ \left( \frac{\sqrt{3}}{2}\log{L}\right)=\kappa\pi}\!\!\!\!\Rightarrow\\
&\!\!\!\!\Rightarrow\ccases{c_1=0\\c_2=0\ \textrm{ή}\ \log{L}=\frac{2\kappa\pi}{\sqrt{3}}}\!\!\!\!\Rightarrow\ccases{c_1=0\\c_2=0\ \textrm{ή}\ L=e^{\frac{2\kappa\pi}{\sqrt{3}}}}\ \kappa\in\{1,2,3,\ldots\}
\end{align*}
Παρατηρούμε όμως ότι ο συνδυασμός $ c_1=c_2=0 $ μας οδηγεί στη μηδενική λύση άρα απορρίπτεται. Έτσι οι ζητούμενοι αριθμοί $ L>1 $ είναι της μορφής $ L=e^{\frac{2\kappa\pi}{\sqrt{3}}}\ \kappa\in\{1,2,3,\ldots\} $.\epask
\begin{Askhshs}[B]
\bmath{Δίνεται η γραμμική διαφορική εξίσωση
\begin{equation}\label{b50:1}\tag{$ E $}
y''+2ay'+a^2y=b
\end{equation}
όπου $ a\in\mathbb{R} $ με $ a>0 $ και $ b $ είναι μια συνεχής συνάρτηση στο διάστημα $ [0,\infty) $. Ας είναι $ y $ μια τυχούσα λύση της \eqref{b50:1} και έστω $ Y(x)=\frac{1}{x}y(x) $ για $ x>0 $. Να αποδειχθεί ότι:
\begin{brlist}
\item Αν η $ b $ είναι φραγμένη, τότε η $ Y $ είναι φραγμένη.
\item Αν $ \lim_{x\to\infty}{b(x)}=0 $, τότε $ \lim_{x\to\infty}{Y(x)}=0 $.
\end{brlist}}
\end{Askhshs}\mbox{}\\\\
\lysh
Η αντίστοιχη ομογενής εξίσωση της \eqref{b50:1} θα είναι η
\begin{equation}\label{b50:2}\tag{$ E_0 $}
y''+2ay'+a^2y=0\ \ ,\ \ x\geq 0
\end{equation}
Το χαρακτηριστικό πολυώνυμο της είναι το $ P(\lambda)=\lambda^2+2a\lambda+a^2 $ με διακρίνουσα $ \varDelta=4a^2-4a^2=0 $ και έτσι θα έχει μοναδική διπλή ρίζα την $ \lambda_0=-a $. Οι συναρτήσεις $ y_1(x)-e^{-ax},\ x\geq 0 $ και $ y_2(x)=xe^{-ax},\ x\geq 0 $ αποτελούν βασικό σύνολο λύσεων και έτσι όλες οι λύσεις της \eqref{b50:2} θα δίνονται από τον τύπο:
\[ \tilde{y}(x)=c_1e^{-ax}+c_2xe^{-ax}\ \ ,\ \ x\geq0 \]
με $ c_1,c_2 $ αυθαίρετες σταθερές. Στη συνέχεια θα αναζητήσουμε μια μερική λύση της μη ομογενούς \eqref{b50:1}. Έχουμε λοιπόν $ \forall x\geq 0 $ ότι:
\begin{gather*}
W(y_1,y_2)(x)=\begin{vmatrix}
e^{-ax} & xe^{-ax}\\ae^{-ax} & (1-ax)e^{-ax}
\end{vmatrix}=e^{-2ax}\neq 0\\
W_1(y_1,y_2)(x)=\begin{vmatrix}
0 & xe^{-ax}\\1 & (1-ax)e^{-ax}
\end{vmatrix}=-xe^{-ax}\ \ \textrm{και}\ \ W_2(y_1,y_2)(x)=\begin{vmatrix}
e^{-ax} & 0\\ae^{-ax} & 1
\end{vmatrix}=e^{-ax}
\end{gather*}
Επομένως η ζητούμενη μερική λύση θα είναι η
\begin{align*}
y_\mu(x)&=y_1(x)\int_{1}^{x}{\frac{W_1(y_1,y_2)(t)}{W(y_1,y_2)(t)}\cdot b(t)\d t}+y_2(x)\int_{1}^{x}{\frac{W_2(y_1,y_2)(t)}{W(y_1,y_2)(t)}\cdot b(t)\d t}=\\
&=e^{-ax}\int_{1}^{x}{\frac{-te^{-at}}{e^{-2at}}\cdot b(t)\d t}+xe^{-ax}\int_{1}^{x}{\frac{e^{-at}}{e^{-2at}}\cdot b(t)\d t}=\\
&=-e^{-ax}\int_{1}^{x}{te^{at}\cdot b(t)\d t}+xe^{-ax}\int_{1}^{x}{e^{at}\cdot b(t)\d t}\ \ ,\ \ x\geq 0\ ,\ a>0
\end{align*}
Τελικά όλες οι λύσεις της μη ομογενούς εξίσωσης \eqref{b50:1} δίνονται από τον τύπο
\[ y(x)=c_1e^{-ax}+c_2xe^{-ax}-e^{-ax}\int_{1}^{x}{te^{at}\cdot b(t)\d t}+xe^{-ax}\int_{1}^{x}{e^{at}\cdot b(t)\d t}\ \ ,\ \ x\geq 0\ ,\ a>0 \]
\begin{rlist}
\item Θεωρούμε ότι η συνάρτηση $ b $ είναι φραγμένη. Τότε θα υπάρχει θετικός $ M $ τέτοιος ώστε $ |b(x)|\leq M $. Για την $ Y(x)=\frac{1}{x}y(x),\ x\geq 0 $ τότε θα ισχύει:
\[ Y(x)=c_1\frac{1}{x}e^{-ax}+c_2e^{-ax}-\frac{e^{-ax}}{x}\int_{1}^{x}{te^{at}\cdot b(t)\d t}+e^{-ax}\int_{1}^{x}{e^{at}\cdot b(t)\d t}\ \ ,\ \ x\geq 0\ ,\ a>0 \]
Η συνάρτηση $ \tilde{Y}(x)=c_1\frac{1}{x}e^{-ax}+c_2e^{-ax} $ είναι φραγμένη αφού για οποιεσδήποτε σταθερές $ c_1,c_2 $, ισχύει ότι $ \lim_{x\to\infty}{e^{-ax}}=\lim_{x\to\infty}{\frac{e^{-ax}}{x}}=0 $ και άρα $ \lim_{x\to\infty}{\tilde{Y}(x)}=0 $. Αρκεί να δείξουμε ότι και η $ Y_\mu(x) $ με 
\[ Y_\mu(x)=e^{-ax}\int_{1}^{x}{e^{at}\cdot b(t)\d t}-\frac{e^{-ax}}{x}\int_{1}^{x}{te^{at}\cdot b(t)\d t}\ \ ,\ \ x>0\ ,\ a>0 \]
είναι φραγμένη. Έχουμε λοιπόν για κάθε $ x>0 $ ότι:
\begin{align*}
|Y_\mu(x)|&\leq e^{-ax}\int_{1}^{x}{e^{at}\cdot |b(t)|\d t}+\left| \frac{e^{-ax}}{x}\right| \int_{1}^{x}{te^{at}\cdot |b(t)|\d t}\leq\\
&\leq Me^{-ax}\cdot\frac{1}{a}\left(e^{ax}-e^a \right)+M\frac{1}{x}e^{-ax}\int_{1}^{x}{te^{at}\d t}=\\
&=Me^{-ax}\cdot\frac{1}{a}\left(e^{ax}-e^a \right)+M\frac{1}{x}e^{-ax}\left[\frac{1}{a}te^{at}-\frac{1}{a^2}e^{at} \right]_{1}^{x}=\\
&=Me^{-ax}\cdot\frac{1}{a}\left(e^{ax}-e^a \right)+M\frac{1}{x}e^{-ax}\left(\frac{1}{a}xe^{ax}-\frac{1}{a^2}e^{ax}-\frac{1}{a}e^{a}+\frac{1}{a^2}e^{a} \right)\leq \frac{2M}{a}+\frac{M}{a^2}\cdot\frac{1}{x}e^{a(1-x)}
\end{align*}
\item Έστω ότι $ \lim_{x\to\infty}{b(x)}=0 $. Στο προηγούμενο ερώτημα αποδείξαμε ότι $ \lim_{x\to\infty}{\tilde{Y}(x)}=0 $. Αρκεί λοιπόν να δείξουμε ότι και $ \lim_{x\to\infty}{Y_\mu(x)}=0 $. Έχουμε για κάθε $ x>0 $:
\[ |Y_\mu(x)|\leq e^{-ax}\int_{1}^{x}{e^{at}\cdot |b(t)|\d t}+\frac{e^{-ax}}{x} \int_{1}^{x}{te^{at}\cdot |b(t)|\d t} \]
Διακρίνουμε τις εξής περιπτώσεις:
\begin{itemize}
\item Αν $ \int_{1}^{x}{e^{at}\cdot |b(t)|\d t}<+\infty $ και $ \int_{1}^{x}{te^{at}\cdot |b(t)|\d t}<+\infty $ τότε $ \lim_{x\to\infty}{Y_\mu(x)}=0 $.
\item Αν $ \int_{1}^{x}{e^{at}\cdot |b(t)|\d t}=+\infty $ και $ \int_{1}^{x}{te^{at}\cdot |b(t)|\d t}=+\infty $ τότε εφαρμόζοντας τον κανόνα De L'Hospital προκύπτει ότι:
\begin{align*}
&\lim_{x\to\infty}{\left( \frac{\int_{1}^{x}{e^{at}\cdot |b(t)|\d t}}{e^{ax}}+\frac{\int_{1}^{x}{te^{at}\cdot |b(t)|\d t}}{xe^{ax}}\right) }\eq{DLH}\lim_{x\to\infty}{\left( \frac{e^{ax}\cdot |b(x)|}{ae^{ax}}+ \frac{xe^{ax}\cdot |b(x)|}{e^{ax}+axe^{ax}}\right)= }\\=&\lim_{x\to\infty}{\left( \frac{ |b(x)|}{a}+ \frac{|b(x)|}{\frac{1}{x}+a}\right) }=0
\end{align*}
\end{itemize}
Αποδεικνύεται λοιπόν ότι $ \lim_{x\to\infty}{Y_\mu(x)}=0 $ άρα τελικά παίρνουμε $ \lim_{x\rightarrow\infty}{Y(x)}=0 $.
\end{rlist}\mbox{}\\\\\\
\begin{Askhshs}[B]
\bmath{Ας είναι $ b $ μια συνεχής συνάρτηση στο $ [0,\infty) $, για την οποία υπάρχει μια σταθερά $ C\geq 0 $ έτσι ώστε
\[ \int_{x}^{x+1}{|b(t)|\d t}\leq C\ \ ,\ \textrm{για όλα τα }x\geq 0 \]
Να αποδείξετε ότι ισχύει
\[ e^{-x}\int_{0}^{x}{e^t}|b(t)|\d t\leq C\cdot\frac{e}{e-1}\ \ ,\ \textrm{για όλα τα }x\geq 0 \] Στη συνέχεια να αποδειχθεί ότι όλες οι λύσεις της γραμμικής διαφορικής εξίσωσης
\[ y''+2y'+2y=b \] είναι φραγμένες στο διάστημα $ [0,\infty) $.}
\end{Askhshs}\mbox{}\\\\
\lysh
Χρησιμοποιούμε την αλλαγή μεταβλητής $u = x-t$ και η ολοκληρωτική σχέση μετατρέπεται ως εξής:
\[ e^{-x}\int_{0}^{x}{e^t}|b(t)|\d t=e^{-u-t}\int_{x}^{0}{e^t}|b(x-u)|(-\d u)=\int_0^x e^{-u} \lvert f(x-u)\rvert\,\d u \]
Θεωρούμε τώρα ένα τυχαίο $ k\in[0,\infty) $ έτσι ώστε $k\leq\lfloor x\rfloor$ και ξεχωρίζουμε τις εξής περιπτώσεις. Αν $ k<\lfloor x \rfloor $ τότε:
\[ \int_k^{k+1} e^{-u}\lvert f(x-u)\rvert\,\d u \leq e^{-k} \int_k^{k+1} \lvert f(x-u)\rvert\,\d u \leq C e^{-k} \]
ενώ αν ισχύει $k = \lfloor x\rfloor$, τότε παίρνουμε:
\[\int_k^x e^{-u}\lvert f(x-u)\rvert\,\d u \leq e^{-k} \int_k^x \lvert f(x-u)\rvert\,\d u\leq e^{-k} \int_k^{k+1} \lvert f(x-u)\rvert\,\d u \leq C e^{-k} \]
Σε κάθε περίπτωση λοιπόν θα προκύπτει ότι
\[ \int_0^x e^{-u} \lvert f(x-u)\rvert\,\d u\leq \sum_{k=0}^{\infty}\int_k^{k+1} e^{-u}\lvert f(x-u)\rvert\,\d u \leq C\cdot \sum_{k = 0}^{\infty} e^{-k} = C\frac{e}{e-1} \]
Για την εφαρμογή είναι:\\
Το χαρακτηριστικό πολυώνυμο της αντίστοιχης ομογενούς εξίσωσης $ y''+2y'+2y=0 $ είναι το $ P(\lambda)=\lambda^2+2\lambda+2 $ το οποίο έχει μιγαδικές ρίζες τις $ \lambda_1=-1+i $ και $ \lambda_2=-1-i $. Σχηματίζονται έτσι οι συναρτήσεις  \[ y_1(x)=e^{-x}\cos{x }\ \ \textrm{και}\ \  y_2(x)=e^{-x}\sin{x }\ \ ,\ \ x\geq 0\] 
οι οποίες αποτελούν ένα βασικό σύνολο λύσεων της ομογενούς εξίσωσης, όλες οι λύσεις της οποίας θα δίνονται από τον παρακάτω τύπο:
\[ \tilde{y}(x)=c_1e^{-x}\cos{x }+c_2e^{-x}\sin{x }\ \ ,\ \ x\geq 0 \]
όπου $ c_1,c_2 $ είναι αυθαίρετες σταθερές. Θα αναζητήσουμε στη συνέχεια μια μερική λύση της αρχικής, μη ομογενούς εξίσωσης. Για κάθε $ x\geq0 $ έχουμε:
\begin{gather*}
W(y_1,y_2)(x)=\begin{vmatrix}
e^{-x}\cos{x } & e^{-x}\sin{x }\\
-e^{-x}\left( \cos{x }+\sin{x }\right)  & e^{-x}\left(\cos{x }-\sin{x }\right)
\end{vmatrix}=e^{-2x}\neq 0\\
W_1(y_1,y_2)(x)=\begin{vmatrix}
0 & e^{-x}\sin{x }\\
1 & e^{-x}\left(\cos{x }-\sin{x }\right) 
\end{vmatrix}=-e^{-x}\sin{x }\\
W_2(y_1,y_2)(x)=\begin{vmatrix}
e^{-x}\cos{x } & 0\\
-e^{-x}\left(\cos{x }+\sin{x }\right)  & 1
\end{vmatrix}=e^{-x}\cos{x }
\end{gather*}
Συνεπώς η ζητούμενη μερική λύση θα είναι η παρακάτω:
\begin{align*}
y_\mu(x)&=y_1(x)\int_{0}^{x}{\frac{W_1(y_1,y_2)(t)}{W(y_1,y_2)(t)}\cdot b(t)\d t}+y_2(x)\int_{0}^{x}{\frac{W_2(y_1,y_2)(t)}{W(y_1,y_2)(t)}\cdot b(t)\d t}=\\
&=e^{-x}\cos{x }\int_{0}^{x}{\frac{-e^{-t}\sin{t}}{e^{-2t}}\cdot b(t)\d t}+e^{-x}\sin{x }\int_{0}^{x}{\frac{e^{-t}\cos{t }}{e^{-2t}}\cdot b(t)\d t}=\\
&=-e^{-x}\cos{x }\int_{0}^{x}{e^{t}\sin{t}\cdot b(t)\d t}+e^{-x}\sin{x }\int_{0}^{x}{e^{t}\cos{t }\cdot b(t)\d t}
\end{align*}
Όλες οι λύσεις της μη ομογενούς εξίσωσης θα δίνονται από τον τύπο:
\begin{gather*}
y(x)=c_1e^{-x}\cos{x }+c_2e^{-x}\sin{x }-e^{-x}\cos{x }\int_{0}^{x}{e^{t}\sin{t}\cdot b(t)\d t}+e^{-x}\sin{x }\int_{0}^{x}{e^{t}\cos{t }\cdot b(t)\d t}\ \ ,\ \ x\geq 0
\end{gather*}
όπου $ c_1,c_2 $ είναι αυθαίρετες σταθερές και $ \omega>0 $. Θα αποδείξουμε στη συνέχεια ότι οι λύσεις της αρχικής εξίσωσης είναι φραγμένες στο $ [0,\infty) $. Σύμφωνα με την υπόθεση της άσκησης θα έχουμε
\[ \lim_{x\to\infty}{\tilde{y}(x)}=\lim_{x\to\infty}{\left(c_1e^{-x}\cos{x }+c_2e^{-x}\sin{x } \right) }=0 \]
ως γινόμενα μηδενικών επί φραγμένων συναρτήσεων, άρα όλες οι λύσεις της ομογενούς εξίσωσης είναι φραγμένες. Αρκεί να δείξουμε ότι και η μερική λύση $ y_\mu $ είναι επίσης φραγμένη. Για κάθε $ x\geq 0 $ έχουμε ότι:
\begin{align*}
|y_\mu(x)|&\leq e^{-x}|\cos{x }|\int_{0}^{x}{e^{t}|\sin{t}|\cdot |b(t)|\d t}+e^{-x}|\sin{x }|\int_{0}^{x}{e^{t}|\cos{t }|\cdot |b(t)|\d t}\leq\\
&\leq e^{-x}\int_{0}^{x}{e^{t}\cdot |b(t)|\d t}+e^{-x}\int_{0}^{x}{e^{t}\cdot |b(t)|\d t}\leq 2C\cdot\frac{e}{e-1}
\end{align*}
Συνεπώς η $ y_\mu $ είναι φραγμένη και άρα όλες οι λύσεις της αρχικής διαφορικής εξίσωσης είναι επίσης φραγμένες στο διάστημα $ [0,\infty) $.\epask
\begin{Askhshs}[B]
\bmath{Έστω η γραμμική διαφορική εξίσωση
\begin{equation}\label{b52:1}\tag{$ E $}
y'''+\omega^2y'=b
\end{equation}
όπου $ \omega>0 $ είναι μια σταθερά και $ b $ είναι μια συνεχής μη μηδενική συνάρτηση στο διάστημα $ [0,\infty) $. Να βρεθούν όλες οι λύσεις της \eqref{b52:1}. Ειδικά να βρεθεί η λύση $ y_0 $ της \eqref{b52:1} που πληροί τις αρχικές συνθήκες $ y(0)=y'(0)=0 $ και $ y''(0)=1 $. Στη συνέχεια να αποδειχθεί ότι η συνθήκη $ \int_{0}^{\infty}{|b(x)|\d x}<\infty $ είναι μια ικανή συνθήκη ώστε όλες οι λύσεις της \eqref{b52:1} να είναι φραγμένες στο $ [0,\infty) $. Τέλος να δοθεί ένα παράδειγμα μιας φραγμένης συνάρτησης $ b $ στο $ [0,\infty) $ έτσι ώστε η λύση $ y_0 $ της \eqref{b52:1} να είναι μη φραγμένη στο $ [0,\infty) $.}
\end{Askhshs}\mbox{}\\\\
\lysh
Η αντίστοιχη ομογενής εξίσωση της \eqref{b52:1} θα είναι η
\begin{equation}\label{b52:2}\tag{$ E_0 $}
y'''+\omega^2y=0
\end{equation}
της οποίας το χαρακτηριστικό πολυώνυμο είναι το $ P(\lambda)=\lambda^3+\omega^2\lambda $. Οι ρίζες του πολυωνύμου θα είναι οι $ \lambda_1=0,\ \lambda_2=i\omega $ και $ \lambda_3=-i\omega $ άρα ένα βασικό σύνολο λύσεων θα αποτελείται από τις συναρτήσεις $ y_1(x)=1,\ y_2(x)=\cos{(\omega x)},\ x\geq 0 $ και $ y_3(x)=\sin{(\omega x)},\ x\geq 0 $. Έτσι όλες οι λύσεις της \eqref{b52:2} θα δίνονται από τον τύπο:
\[ \tilde{y}(x)=c_1+c_2\cos{(\omega x)}+c_3\sin{(\omega x)}\ \ ,\ \ x\geq0 \]
όπου $ c_1,c_2,c_3 $ είναι αυθαίρετες σταθερές. Στη συνέχεια θα βρούμε μια μερική λύση της μη ομογενούς εξίσωσης \eqref{b52:1}. Έχουμε λοιπόν για κάθε $ x\geq0 $ ότι:
\begin{align}
&W(y_1,y_2,y_3)(x)=\begin{vmatrix}
1 & \cos{(\omega x)} & \sin{(\omega x)}\\
0 & -\omega\sin{(\omega x)} & \omega\cos{(\omega x)}\\
0 & -\omega^2\cos{(\omega x)} & -\omega^2\sin{(\omega x)}\\
\end{vmatrix}=\omega^3\neq 0\\ 
&W_1(y_1,y_2,y_3)(x)=\begin{vmatrix}
0 & \cos{(\omega x)} & \sin{(\omega x)}\\
0 & -\omega\sin{(\omega x)} & \omega\cos{(\omega x)}\\
1 & -\omega^2\cos{(\omega x)} & -\omega^2\sin{(\omega x)}\\
\end{vmatrix}=\omega\\
&W_2(y_1,y_2,y_3)(x)=\begin{vmatrix}
1 & 0 & \sin{(\omega x)}\\
0 & 0 & \omega\cos{(\omega x)}\\
0 & 1 & -\omega^2\sin{(\omega x)}\\
\end{vmatrix}=-\omega\cos{(\omega x)}\\ 
&W_1(y_1,y_2,y_3)(x)=\begin{vmatrix}
1 & \cos{(\omega x)} & 0\\
0 & -\omega\sin{(\omega x)} & 0 \\
0 & -\omega^2\cos{(\omega x)} & 1\\
\end{vmatrix}=-\omega\sin{(\omega x)}\\
\end{align}
Συνεπώς η ζητούμενη μερική λύση θα δίνεται από τον τύπο:
\begin{multline*}
y_\mu(x)=y_1(x)\int_{0}^{x}{\frac{W_1(y_1,y_2,y_3)(t)}{W(y_1,y_2,y_3)(t)}\cdot b(t)\d t}+\\+y_2(x)\int_{0}^{x}{\frac{W_2(y_1,y_2,y_3)(t)}{W(y_1,y_2,y_3)(t)}\cdot b(t)\d t}+y_3(x)\int_{0}^{x}{\frac{W_3(y_1,y_2,y_3)(t)}{W(y_1,y_2,y_3)(t)}\cdot b(t)\d t}=
\end{multline*}
\vspace{-4mm}
\begin{align*}
&\ \ \ =\int_{0}^{x}{\frac{\omega}{\omega^3}\cdot b(t)\d t}+\cos{(\omega x)}\int_{0}^{x}{\frac{-\omega\cos{(\omega t)}}{\omega^3}\cdot b(t)\d t}+\sin{(\omega x)}\int_{0}^{x}{\frac{-\omega\sin{(\omega t)}}{\omega^3}\cdot b(t)\d t}=\\
&\ \ \ =\frac{1}{\omega^2}\int_{0}^{x}{b(t)\d t}-\frac{1}{\omega^2}\cos{(\omega x)}\int_{0}^{x}{\cos{(\omega t)}\cdot b(t)\d t}-\frac{1}{\omega^2}\sin{(\omega x)}\int_{0}^{x}{\sin{(\omega t)}\cdot b(t)\d t}
\end{align*}
Έτσι όλες οι λύσεις της αρχικής διαφορικής εξίσωσης θα δίνονται από τον παρακάτω τύπο:
\begin{multline*}
y(x)=c_1+c_2\cos{(\omega x)}+c_3\sin{(\omega x)}+\\+\frac{1}{\omega^2}\left[ \int_{0}^{x}{b(t)\d t}-\cos{(\omega x)}\int_{0}^{x}{\cos{(\omega t)}\cdot b(t)\d t}-\sin{(\omega x)}\int_{0}^{x}{\sin{(\omega t)}\cdot b(t)\d t}\right]
\end{multline*}
όπου $ x\geq 0\textrm{ και }\omega>0 $ και $ c_1,c_2,c_3 $ αυθαίρετες σταθερές. Απαιτούμε τώρα η συνάρτηση αυτή να πληροί τις αρχικές συνθήκες $ y(0)=y'(0)=0 $ και $ y''(0)=1 $. Έτσι θα έχουμε για κάθε $ x\geq 0 $ ότι:
\begin{align*}
y'(x)&=-\omega c_2\sin{(\omega x)}+\omega c_3\cos{(\omega x)}+\frac{1}{\omega^2}\left[b(x)-\cos^2{(\omega x)}b(x)-\sin^2{(\omega x)}b(x)\right]=\\
&=-\omega c_2\sin{(\omega x)}+\omega c_3\cos{(\omega x)}\qquad\textrm{ και επιπλέον}\\
y''(x)&=-\omega^2 c_2\cos{(\omega x)}+\omega^2 c_3\sin{(\omega x)}
\end{align*}
Από τις λοιπόν αρχικές συνθήκες παίρνουμε τις παρακάτω σχέσεις:
\[ \ccases{y(0)=0\\y'(0)=0\\y''(0)=1}\!\!\!\!\Rightarrow\ccases{c_1+c_2=0\\\omega c_3=0\\-\omega^2c_2=0}\!\!\!\!\Rightarrow c_1=\frac{1}{\omega^2}\ ,\ c_2=-\frac{1}{\omega^2}\ \ \textrm{και}\ \ c_3=0 \]
Συνεπώς η λύση $ y_0 $ θα είναι η:
\begin{gather*}
y_0(x)=\frac{1}{\omega^2}\left[ 1-\cos{(\omega x)}+\int_{0}^{x}{b(t)\d t}-\cos{(\omega x)}\int_{0}^{x}{\cos{(\omega t)}\cdot b(t)\d t}-\sin{(\omega x)}\int_{0}^{x}{\sin{(\omega t)}\cdot b(t)\d t}\right]
\end{gather*}
Έχουμε τώρα για κάθε $ x\geq 0 $ ότι:
\[ |y(x)|\leq |c_1|+|c_2|+|c_3|+\frac{3}{\omega^2}\int_{0}^{x}{|b(t)|\d t}\leq |c_1|+|c_2|+|c_3|+\frac{3}{\omega^2}\int_{0}^{\infty}{|b(t)|\d t} \]
Συνεπώς αν $ \int_{0}^{\infty}{|b(x)|\d t}<\infty $, όπως ζητάει η υπόθεση, τότε η $ y(x) $ είναι φραγμένη στο $ [0,\infty) $ και έτσι όλες οι λύσεις είναι φραγμένες. Όσον αφορά το παράδειγμα, αρκεί να βρούμε μια φραγμένη συνάρτηση $ b(x) $ στο $ [0,\infty) $ έτσι ώστε $ \int_{0}^{\infty}{|b(x)|\d x}=\infty $. Επιλέγοντας όπου $ b(x)=\sin{x},\ x\geq 0 $ παρατηρούμε ότι όντως ισχύει
\[ \int_{0}^{\infty}{|\sin{x}|\d x}=\infty \]\epask
\begin{Askhshs}[B]
\bmath{Να αποδειχθεί ότι υπάρχουν πραγματικές σταθερές $ a $ και $ \delta $ έτσι ώστε, για κάθε λύση $ y $ της γραμμικής διαφορικής εξίσωσης
\[ y''+8y'+25y=2\cos{x}\ \ ,\ \ x\in\mathbb{R} \]
να ισχύει
\[ \lim_{x\to\infty}{\left[y(x)-a\cos{(x-\delta)} \right] }=0 \]}
\end{Askhshs}\mbox{}\\\\
\lysh
Το χαρακτηριστικό πολυώνυμο της αντίστοιχης ομογενούς εξίσωσης είναι το $ P(\lambda)=\lambda^2+8\lambda+25 $ με μιγαδικές ρίζες τις $ \lambda_1=-4+3i $ και $ \lambda_2=-4-3i $. Αυτές μας δίνουν τις συναρτήσεις $ y_1(x)=e^{-4x}\cos{(3x)},\ x\in\mathbb{R} $ και $ y_1(x)=e^{-4x}\sin{(3x)},\ x\in\mathbb{R} $ που αποτελούν βασικό σύνολο λύσεων της εξίσωσης. Όλες οι λύσεις της θα δίνονται από τον τύπο:
\[ \tilde{y}(x)=c_1e^{-4x}\cos{(3x)}+c_2e^{-4x}\sin{(3x)}\ \ ,\ \ x\in\mathbb{R} \]
όπου $ c_1,c_2 $ είναι αυθαίρετες σταθερές. Θα αναζητήσουμε στη συνέχεια μια μερική λύση $ v_\mu $. Θεωρούμε τη διαφορική εξίσωση
\begin{equation}\label{b53:1}
y''+8y'+25y=2e^{ix}\ \ ,\ \ x\in\mathbb{R}
\end{equation}
Σ' αυτήν θέτουμε όπου $ y=ze^{ix},\ x\in\mathbb{R} $ και ύστερα από παραγώγιση παίρνουμε
\[ y'=(z'+iz)e^{ix}\ \ \textrm{και}\ \ y''=\left(z''+2iz'-z \right)e^{ix}  \]
Αντικαθιστούμε τις παραπάνω σχέσεις στη διαφορική εξίσωση \eqref{b53:1} και αυτή μετασχηματίζεται στη μορφή
\begin{gather*}
z''+2iz'-z+8z'+8iz+25z=2\Rightarrow\\
z''+(8+2i)z'+(24+8i)z=2
\end{gather*}
Η μερική λύση $ z_\mu $ της τελευταίας εξίσωσης θα είναι της μορφής $ z_\mu(x)=a $ και παίρνουμε $ z_\mu'=z_\mu''=0 $ οπότε η εξίσωση θα μας δώσει:
\[ (24+8i)a=2\Rightarrow a=\frac{1}{12+4i} \]
Άρα η μερική λύση είναι $ z_\mu(x)=\frac{1}{12+4i}=\frac{3}{40}-i\frac{1}{40} $. Με τη βοήθεια αυτής, παίρνουμε τη μερική λύση $ y_\mu $ η οποία θα είναι:
\begin{align*}
y_\mu(x)&=\left( \frac{3}{40}-i\frac{1}{40}\right)e^{ix}=\left( \frac{3}{40}-i\frac{1}{40}\right)\cdot(\cos{x}+i\sin{x})=\\
&=\frac{1}{40}(3\cos{x}+\sin{x})+\frac{1}{40}(3\sin{x}-\cos{x})\ \ ,\ \ x\in\mathbb{R}
\end{align*}
Η ζητούμενη μερική λύση της αρχικής μη ομογενούς διαφορικής εξίσωσης θα δίνεται από τον τύπο $ v_{\mu}(x)=\mathrm{Re}(y_\mu(x))=\frac{1}{40}(3\cos{x}+\sin{x}) $. Επομένως όλες οι λύσεις της θα δίνονται από τον τύπο
\[ y(x)=c_1e^{-4x}\cos{(3x)}+c_2e^{-4x}\sin{(3x)}+\frac{1}{40}(3\cos{x}+\sin{x})\ \ ,\ \ x\in\mathbb{R} \]
Παρατηρούμε ότι $ \lim_{x\to\infty}{\left( e^{-4x}\cos{x}\right) }=\lim_{x\to\infty}{\left( e^{-4x}\sin{x}\right) }=0 $ ωε μηδενικές επί φραγμένες. Έτσι αρκεί να δείξουμε ότι υπάρχουν πραγματικές σταθερές $ a $ και $ \delta $ έτσι ώστε να ισχύει:
\begin{gather*}
\lim_{x\to\infty}{\left[v(x)-a\cos{(x-\delta)} \right] }=0\Rightarrow
\lim_{x\to\infty}{\left[\frac{1}{40}(3\cos{x}+\sin{x})-a\cos{(x-\delta)} \right] }=0\Rightarrow\\
\frac{3}{40}\cos{x}+\frac{1}{40}\sin{x}=a\cos{(x-\delta)}\Rightarrow\\
\frac{3}{40}\cos{x}+\frac{1}{40}\sin{x}=a\cos{x}\cos{\delta}+a\sin{x}\sin{\delta}\Rightarrow\\
40a\cos{\delta}=3\ \ \textrm{και}\ \ 40a\sin{\delta}=1\Rightarrow
\end{gather*}
Σχηματίζεται έτσι το παρακάτω σύστημα από το οποίο παίρνουμε τις τιμές των σταθερών $a$ και $\delta$. Για την $ a $ έχουμε:
\[ \ccases{40a\cos{\delta}=3\\
40a\sin{\delta}=1}\!\!\!\!\Rightarrow\ccases{1600a^2\cos^2{\delta}=9\\
1600a^2\sin^2{\delta}=1}\!\!\!\!\xRightarrow{(1)+(2)}1600a^2=10\Rightarrow a=\pm\frac{\sqrt{10}}{40} \]
αλλά και για τη σταθερά $ \delta $ είναι:
\[ \ccases{40a\cos{\delta}=3\\
40a\sin{\delta}=1}\!\!\!\!\xRightarrow{(2)\div(1)}\tan{\delta}=\frac{1}{3}\Rightarrow \delta=\kappa\pi+\arctan{\frac{1}{3}}\ ,\ \kappa\in\mathbb{Z} \]
\begin{Askhshs}[B]
\bmath{Έστω η μη ομογενής διαφορική εξίσωση
\begin{equation}\label{b54:1}\tag{$ E $}
y''-y'\cos{x}+y\sin{x}=\sin{x}\ \ ,\ \ x\in\mathbb{R}
\end{equation}
και ας συμβολίσουμε με $ E_0 $ την αντίστοιχη ομογενή διαφορική εξίσωση αυτής. Αφού διαπιστωθεί ότι η $ y_1(x)=e^{\sin{x}} $ είναι μια μερική λύση της $ E_0 $, να βρεθούν δύο γραμμικά ανεξάρτητες λύσεις της $ E_0 $. Στη συνέχεια να βρεθεί η μερική λύση $ y_\mu $ της \eqref{b54:1} που πληροί τις αρχικές συνθήκες
\[ y_\mu(0)=y'_\mu(0)=0 \]
Τέλος να βρεθεί η λύση $ y $ της \eqref{b54:1} που πληροί τις αρχικές συνθήκες
\[ y(0)=1\ \ \textrm{και}\ \ y'(0)=0 \]}
\end{Askhshs}\mbox{}\\\\
\lysh
Η αντίστοιχη ομογενής διαφορική εξίσωση $ Ε_0 $ της \eqref{b54:1} είναι η
\begin{equation}\label{b54:2}\tag{$ E_0 $}
y''-y'\cos{x}+y\sin{x}=0\ \ ,\ \ x\in\mathbb{R}
\end{equation}
Εύκολα παρατηρούμε ότι η $ y_1 $ αποτελεί λύση της ομογενούς εξίσωσης υπολογίζοντας τις παραγώγους της έως δεύτερης τάξης. Θα είναι:
\begin{align*}
y_1'(x)&=e^{\sin{x}}\cos{x}\ \ \textrm{και}\\
y''_1(x)&=e^{\sin{x}}\cos^2{x}-e^{\sin{x}}\sin{x}=e^{\sin{x}}\left( \cos^2{x}-\sin{x}\right) 
\end{align*}
Αντικαθιστώντας τες στην \eqref{b54:2} παίρνουμε
\begin{gather*}
y''_1-y'_1\cos{x}+y_1\sin{x}=0\Rightarrow\\
e^{\sin{x}}\left( \cos^2{x}-\sin{x}\right)-e^{\sin{x}}\cos^2{x}+e^{\sin{x}}\sin{x}=0\Rightarrow\\
e^{\sin{x}}\cos^2{x}-e^{\sin{x}}\sin{x}-e^{\sin{x}}\cos^2{x}+e^{\sin{x}}\sin{x}=0\Rightarrow 0=0
\end{gather*}
Στη συνέχεια εκτελούμε το μετασχηματισμό $ y=zy_1 $ ώστε να βρούμε μια δεύτερη λύση της \eqref{b54:2} γραμμικά ανεξάρτητη με την $ y_1 $. Παραγωγίζοντας παίρνουμε:
\begin{align*}
y'&=z'y_1+zy_1'=e^{\sin{x}}\left(z'+z\cos{x}\right) \ \ \textrm{και}\\
y''&=z''y_1+2z'y_1'+zy_1''=e^{\sin{x}}\left[ z''+2z'\cos{x}+z\left( \cos^2{x}-\sin{x}\right) \right]
\end{align*}
Οπότε με αντικατάσταση στην εξίσωση \eqref{b54:2} θα πάρουμε μια νέα διαφορική εξίσωση ως προς τη συνάρτηση $ z $.
\begin{gather*}
e^{\sin{x}}\left[ z''+2z'\cos{x}+z\left( \cos^2{x}-\sin{x}\right) \right]-e^{\sin{x}}\left(z'+z\cos{x}\right)\cos{x}+ze^{\sin{x}}\sin{x}=0\Rightarrow\\
z''+2z'\cos{x}+z\left( \cos^2{x}-\sin{x}\right)-\left(z'+z\cos{x}\right)\cos{x}+z\sin{x}=0\Rightarrow\\
z''+2z'\cos{x}+z \cos^2{x}-z\sin{x}-z'\cos{x}-z\cos^2{x}+z\sin{x}=0\Rightarrow\\
z''+z'\cos{x}=0
\end{gather*}
Με νέα αντικατάσταση $ u=z'\Rightarrow u'=z'' $ θα μετατρέψουμε την τελευταία σε πρώτης τάξης εξίσωση κάνοντας υποβιβασμό. Στη συνέχεια με αναδρομική αντικατάσταση θα καταλήξουμε στη ζητούμενη συνάρτηση $ y_2(x) $.
\begin{gather*}
u'+u\cos{x}=0\Rightarrow\\
u(x)=e^{-\int{\cos{x}\d x}}=e^{-\sin{x}}\Rightarrow\\
z(x)=\int_{0}^{x}{e^{-\sin{t}}\d t}\Rightarrow y_2(x)=e^{\sin{x}}\int_{0}^{x}{e^{-\sin{t}}\d t}
\end{gather*}
Με τη χρήση της ορίζουσας Wronski θα ελέγξουμε εάν όντως οι λύσεις είναι γραμμικά ανεξάρτητες.
\[ W(y_1,y_2)(x)=\begin{vmatrix}
e^{\sin{x}} & e^{\sin{x}}\int_{0}^{x}{e^{-\sin{t}}\d t}\\
e^{\sin{x}}\cos{x} & e^{\sin{x}}\cos{x}\int_{0}^{x}{e^{-\sin{t}}\d t}+1
\end{vmatrix}=e^{\sin{x}}>0 \]
οπότε φτάσαμε στο ζητούμενο. Μένει να υπολογίσουμε μια μερική λύση της μη ομογενούς εξίσωσης \eqref{b54:1}. Αυτή θα δίνεται από τον τύπο
\begin{align*}
y_\mu(x)&=\int_{0}^{x}{\frac{y_1(t)y_2(x)-y_2(t)y_1(x)}{y_1'(t)y_2(t)-y_2'(t)y_1(t)}\cdot\frac{b(t)}{a_2(t)}\d t}=\\
&=\int_{0}^{x}{\frac{e^{\sin{t}}e^{\sin{x}}\int_{0}^{x}{e^{-\sin{s}}\d s}-e^{\sin{x}}e^{\sin{t}}\int_{0}^{t}{e^{-\sin{s}}\d s}}{e^{\sin{t}}}\cdot\sin{t}\d t}=\\
&=\int_{0}^{x}{\frac{e^{\sin{t}}e^{\sin{x}}\int_{0}^{x}{e^{-\sin{s}}\d s}-e^{\sin{x}}e^{\sin{t}}\int_{0}^{t}{e^{-\sin{s}}\d s}}{e^{\sin{t}}}\cdot\sin{t}\d t}=\ldots
\end{align*}
\epask
\begin{Askhshs}[B]
\bmath{Δίνεται η γραμμική διαφορική εξίσωση
\begin{equation}\label{a55:1}\tag{$ E_1 $}
\left( 1-\frac{1}{x}\right)y''+\left( \frac{2}{x}-\frac{2}{x^2}-\frac{1}{x^3}\right)y'-\frac{1}{x^4}y=\frac{2}{x}-\frac{2}{x^2}-\frac{2}{x^3}\ \ ,\ \ x>1
\end{equation}
Να αποδειχθεί ότι η αντικατάσταση $ t=\frac{1}{x} $ μετασχηματίζει την \eqref{a55:1} σε μια μη ομογενή γραμμική διαφορική εξίσωση $ (E_2) $. Να βρεθεί μια μερική λύση $ y_\mu $ της $ (E_2) $ της μορφής $ y_\mu(t)=t^m,\ t<1\ m$ ακέραιος. Ας είναι $ (E_3) $ η αντίστοιχη ομογενής εξίσωση της $ (E_2) $. Να βρεθούν δύο λύσεις $ y_1 $ και $ y_2 $ της $ (E_3) $ των μορφών $ y_1(t)=at+\beta,\ t<1 $ και $ y_2(t)=e^{\gamma t},\ t<1\ (a,\beta,\gamma\ \textrm{σταθερές}) $. Τέλος να βρεθεί η γενική λύση της \eqref{a55:1}.}
\end{Askhshs}\mbox{}\\\\
\lysh
Με τη χρήση του μετασχηματισμού $ t=\frac{1}{x} $ αρχικά θα υπολογίσουμε τις παραγώγους $ y' $ και $ y'' $ ως προς τη μεταβλητή $ t $. Θα είναι
\begin{align*}
y'&=\diff{y}{x}=\diff{y}{t}\diff{t}{x}=-\frac{1}{x^2}\diff{y}{t}\\
y''&=\diff[2]{y}{x}=\diff{}{x}\left( \diff{y}{x}\right)=\diff{}{x}\left( -\frac{1}{x^2}\diff{y}{t}\right)=\frac{2}{x^3}\diff{y}{t}-\frac{1}{x^2}\diff{}{x}\left( \diff{y}{t}\right)=\\
&=\frac{2}{x^3}\diff{y}{t}-\frac{1}{x^2}\diff{}{t}\left( \diff{y}{x}\right)=\frac{2}{x^3}\diff{y}{t}+\frac{1}{x^4}\diff[2]{y}{t}
\end{align*}
Σύμφωνα λοιπόν με τα παραπάνω, η \eqref{a55:1} μετασχηματίζεται στη ζητούμενη μη ομογενή γραμμική εξίσωση:
\begin{gather}
(1-t)\left( 2t^3\diff{y}{t}+t^4\diff[2]{y}{t}\right)+\left(2t-2t^2-t^3\right)\left( -t^2\diff{y}{t}\right) -t^4y=2t-2t^2-2t^3\Rightarrow\nonumber\\
2t^3\diff{y}{t}+t^4\diff[2]{y}{t}-2t^4\diff{y}{t}-t^5\diff[2]{y}{t}-2t^3\diff{y}{t}+2t^4\diff{y}{t}+t^5\diff{y}{t}-t^4y=2t-2t^2-2t^3\Rightarrow\nonumber\\
t^4\diff[2]{y}{t}-t^5\diff[2]{y}{t}+t^5\diff{y}{t}-t^4y=2t-2t^2-2t^3\Rightarrow\\
t^4(1-t)\diff[2]{y}{t}+t^5\diff{y}{t}-t^4y=2t-2t^2-2t^3\ \ ,\ \ 0<t<1\label{a55:2}\tag{$ E_2 $}
\end{gather}
Αναζητούμε μια μερική λύση της \eqref{a55:2} της μορφής $ y_\mu(t)=t^m,\ 0<t<1\ (m \textrm{ ακέραιος}) $. Με αντικατάσταση στην εξίσωση θα βρούμε την τιμή του $ m $ ώστε η $ y_\mu $ να την επαληθεύει. Έχουμε
\begin{gather*}
t^4(1-t)m(m-1)t^{m-2}+t^5mt^{m-1}-t^4t^m=2t-2t^2-2t^3\Rightarrow\\
(1-t)m(m-1)t^{m+2}+mt^{m+3}-t^{m+4}=2t-2t^2-2t^3\Rightarrow
\end{gather*}
Η τελευταία ικανοποιείται αν και μόνο αν $ m=-1 $ συνεπώς η μερική λύση θα είναι η $ y_\mu(t)=\frac{1}{t},\ 0<t<1 $. Σχηματίζουμε τώρα την αντίστοιχη ομογενή της \eqref{a55:2} η οποία θα είναι
\begin{equation}\label{a55:3}\tag{$ E_3 $}
t^4(1-t)\diff[2]{y}{t}+t^5\diff{y}{t}-t^4y=0
\end{equation}
Απαιτούμε η $ y_1(t)=at+\beta $ να είναι λύση της \eqref{a55:3} άρα πρέπει
\begin{gather*}
t^4(1-t)\diff[2]{y_1}{t}+t^5\diff{y_1}{t}-t^4y_1=0\Rightarrow\\
at^5-(at+\beta)t^4=0\Rightarrow\\
\beta t^4=0\Rightarrow \beta=0
\end{gather*}
Επομένως η λύση θα έχει τη μορφή $ y_1(t)=at $ για οποιαδήποτε σταθερά $ a $. Ζητούμε όμως η λύση αυτή να είναι συγχρόνως γραμμικά ανεξάρτητη με την $ y_2(t)=e^{\gamma t} $, την οποία θα υπολογίσουμε παρακάτω, άρα θα απορρίψουμε την περίπτωση $ a=0 $ διότι μας οδηγεί σε γραμμική εξάρτηση. Επιλέγοντας δίχως βλάβη της γενικότητας $ a=1 $ η ζητούμενη λύση της εξίσωσης είναι $ y_1(t)=t,\ 0<t<1 $. Στη συνέχεια, η $ y_2(t) $ είναι λύση της \eqref{a55:3} αν και μόνο αν $ \forall t\in(0,1) $ ισχύει
\begin{gather*}
t^4(1-t)\diff[2]{y_2}{t}+t^5\diff{y_2}{t}-t^4y_2=0\Rightarrow\\
t^4(1-t)\gamma^2e^{\gamma t}+t^5\gamma e^{\gamma t}-t^4e^{\gamma t}=0\Rightarrow\\
t^4(1-t)\gamma^2+t^5\gamma -t^4=0\Rightarrow\\
t^4\gamma^2-t^5\gamma^2+t^5\gamma -t^4=0\Rightarrow\\
\left( \gamma-\gamma^2\right)t^5+\gamma^2t^4=t^4
\end{gather*}
Αρκεί λοιπόν να ισχύει
\[ \ccases{\gamma-\gamma^2=0\\\gamma^2=1}\!\!\!\!\Rightarrow\gamma=1 \]
Η δεύτερη ζητούμενη λύση της εξίσωσης $ \eqref{a55:3} $ έτσι θα είναι η $ y_2(t)=e^t,\ 0<t<1 $ η οποία εύκολα αποδεικνύεται ότι είναι γραμμικά ανεξάρτητη της $ y_1 $ με τη χρήση της ορίζουσας Wronski. Πράγματι:
\[ W(y_1,y_2)(t)=\begin{vmatrix}
t & e^t\\1 & e^t
\end{vmatrix}=e^t(t-1)\neq 0\ \ ,\ \ \forall t\in(0,1) \]
Φτάσαμε έτσι σε ένα βασικό σύνολο λύσεων της ομογενούς \eqref{a55:3} το $ \{y_1,y_2\}=\left\lbrace t,e^t\right\rbrace $. Γνωρίζοντας επιπλέον μια μερική λύση της, έτσι όπως βρέθηκε παραπάνω, τότε όλες οι λύσεις της μη ομογενούς \eqref{a55:2} θα δίνονται από τον τύπο
\[ y(t)=c_1t+c_2e^t+\frac{1}{t}\ \ ,\ \ 0<t<1 \]
Χρησιμοποιώντας τον αρχικό μετασχηματισμό η παραπάνω συνάρτηση θα μας δώσει όλες τις λύσεις της αρχικής διαφορικής εξίσωσης \eqref{a55:1} οι οποίες θα είναι
\[ y(x)=c_1\frac{1}{x}+c_2e^{\frac{1}{x}}+x\ \ ,\ \ x>1 \]
όπου $ c_1,c_2 $ είναι αυθαίρετες σταθερές.\epask
\begin{Askhshs}[B]
\bmath{Ας είναι $ \gamma $ και $ \omega $ δύο θετικοί πραγματικοί αριθμοί με $ \gamma\neq\omega $. Ακόμα ας είναι $ f $ μια θετική πραγματική συνάρτηση στο διάστημα $ [0,\infty) $ με $ \lim_{x\to\infty}{f}=L $, όπου $ L $ έιναι μια πραγματική σταθερά. Να αποδείχθεί ότι για κάθε λύση $ y $ της γραμμικής διαφορικής εξίσωσης \[ y''+2\gamma y+\omega^2 y=f \]
ισχύει
\[ \lim_{x\to\infty}{y(x)}=\frac{L}{\omega^2}\qquad\textrm{και}\qquad\lim_{ x\rightarrow\infty }{y'(x)}=0 \]}
\end{Askhshs}\mbox{}\\\\
\lysh
Χρησιμοποιούμε το μετασχηματισμό $ z=y-\frac{L}{\omega^2} $ ο οποίος θα φέρει την αρχική διαφορική εξίσωση στην ακόλουθη μορφή:
\begin{equation}\label{a56:1}
z''+2\gamma z'+\omega ^2z=f-L\equiv b
\end{equation}
θέτοντας $ b=f-L $. Το χαρακτηριστικό πολυώνυμο της αντίστοιχης ομογενούς διαφορικής εξίσωσης είναι το $ p(\lambda)=\lambda^2+2\gamma\lambda+\omega^2 $ με διακρίνουσα
\[ \Delta=(2\gamma)^2-4\omega=4\left( \gamma^2-\omega^2\right)\neq0 \]
διότι από την υπόθεση της άσκησης έχουμε $ \gamma\neq\omega $. Διακρίνουμε έτσι τις ακόλουθες περιπτώσεις:
\begin{rlist}
\item Αν $ \Delta>0 $ τότε το πολυώνυμο έχει δύο πραγματικές ρίζες τις $ \lambda_{1,2}=-\gamma\pm\sqrt{\gamma^2-\omega^2} $ οπότε ένα βασικό σύνολο λύσεων της ομογενούς εξίσωσης θα αποτελείται από τις συναρτήσεις $ z_1(x)=e^{\lambda_1 x},\ x\in[0,\infty) $ και $ z_2(x)=e^{\lambda_2 x},\ x\in[0,\infty) $. Όλες οι λύσεις της θα δίνονται από τον τύπο:
\[ \tilde{z}(x)=c_1e^{\lambda_1 x}+c_2e^{\lambda_2 x}\ ,\ x\in[0,\infty) \]
με $ c_1,c_2 $ αυθαίρετες σταθερές.
\item Αν $ \Delta<0 $ τότε το πολυώνυμο $ p(\lambda) $ έχει δύο συζυγείς μιγαδικές ρίζες τις $ \lambda_{1,2}=-\gamma\pm i\sqrt{\omega^2-\gamma^2} $. Συνεπώς οι συναρτήσεις $ z_1(x)=e^{-\gamma x}\cos{\left( \sqrt{\omega^2-\gamma^2} x\right) },\ x\in[0,\infty) $ και $ z_2(x)=e^{-\gamma x}\sin{\left( \sqrt{\omega^2-\gamma^2} x\right) },\ x\in[0,\infty) $ αποτελούν ένα βασικό σύνολο λύσεων της ομογενούς και έτσι οι λύσεις θα είναι
\[ \tilde{z}(x)=c_1e^{-\gamma x}\cos{\left( \sqrt{\omega^2-\gamma^2} x\right)}+c_2e^{-\gamma x}\sin{\left( \sqrt{\omega^2-\gamma^2} x\right)}\ ,\ x\in[0,\infty) \]
\end{rlist}
με $ c_1,c_2 $ αυθαίρετες σταθερές. Παρατηρούμε σε κάθε περίπτωση ότι το πραγματικό μέρος των ριζών του πολυωνύμου είναι αρνητικό δηλαδή $ \textrm{Re}(\lambda_{1,2})<0 $ οπότε παίρνουμε $ \lim_{ x\rightarrow\infty }{\tilde{z}(x)}=0 $. Μένει να υπολογίσουμε μια μερική λύση $ z_\mu $ της αρχικής μη ομογενούς εξίσωσης. Αυτό θα γίνει με τη χρήση των οριζουσών Wronski οι οποίες θα είναι:
\begin{align*}
W(z_1,z_2)(x)&=\begin{vmatrix}
e^{\lambda_1 x} & e^{\lambda_2 x}\\
\lambda_1e^{\lambda_1 x} & \lambda_2e^{\lambda_2 x}
\end{vmatrix}=(\lambda_2-\lambda_1)e^{(\lambda_1+\lambda_2)x}\neq 0\ ,\forall x\in[0,\infty).\\
W_1(z_1,z_2)(x)&=\begin{vmatrix}
0 & e^{\lambda_2 x}\\
1 & \lambda_2e^{\lambda_2 x}
\end{vmatrix}=-e^{\lambda_2x}\ ,\forall x\in[0,\infty).\\
W_2(z_1,z_2)(x)&=\begin{vmatrix}
e^{\lambda_1 x} & 0\\
\lambda_1e^{\lambda_1 x} & 1
\end{vmatrix}=e^{\lambda_1x}\ ,\forall x\in[0,\infty).
\end{align*}
Τότε για την \eqref{a56:1} η μερική λύση που ζητούμε θα δίνεται από τον τύπο:
\begin{align*}
z_\mu(x)&=z_1(x)\int_{0}^{x}{\frac{W_1(z_1,z_2)(t)}{W(z_1,z_2)(t)}\cdot b(t)\d t}+z_2(x)\int_{0}^{x}{\frac{W_2(z_1,z_2)(t)}{W(z_1,z_2)(t)}\cdot b(t)\d t}=\\
&=e^{\lambda_1 x}\int_{0}^{x}{\frac{-e^{\lambda_2 t}}{(\lambda_2-\lambda_1)e^{(\lambda_1+\lambda_2)t}}\cdot b(t)\d t}+e^{\lambda_2 x}\int_{0}^{x}{\frac{e^{\lambda_1 t}}{(\lambda_2-\lambda_1)e^{(\lambda_1+\lambda_2)t}}\cdot b(t)\d t}=\\
&=\frac{1}{\lambda_2-\lambda_1}\left[ -e^{\lambda_1 x}\int_{0}^{x}{e^{-\lambda_1 t}\cdot b(t)\d t}+e^{\lambda_2 x}\int_{0}^{x}{e^{-\lambda_2 t}\cdot b(t)\d t}\right]\ \ ,\ \ \forall x\in[0,\infty)
\end{align*}
Αρκεί να δείξουμε ότι $ \lim_{ x\rightarrow\infty }{e^{\lambda_1 x}\int_{0}^{x}{e^{-\lambda_1 t}\cdot b(t)\d t}}=0 $. Έχουμε λοιπόν:
\[ L=\lim_{ x\rightarrow\infty }{e^{\lambda_1 x}\int_{0}^{x}{e^{-\lambda_1 t}\cdot b(t)\d t}}=\lim_{ x\rightarrow\infty }{\frac{\int_{0}^{x}{e^{-\lambda_1 t}\cdot b(t)\d t}}{e^{-\lambda_1 x}}} \]
Διακρίνουμε τις εξής περιπτώσεις:
\begin{rlist}
\item Αν $ \int_{0}^{\infty}{e^{-\lambda_1 t}\cdot b(t)\d t}<\infty $ τότε
\[ L=\lim_{ x\rightarrow\infty }{\frac{\int_{0}^{x}{e^{-\lambda_1 t}\cdot b(t)\d t}}{e^{-\lambda_1 x}}}=\frac{\textrm{πεπερασμένο}}{\infty}=0 \]
\item Αν $ \int_{0}^{\infty}{e^{-\lambda_1 t}\cdot b(t)\d t}\to\infty $ τότε
\[ L=\lim_{ x\rightarrow\infty }{\frac{\int_{0}^{x}{e^{-\lambda_1 t}\cdot b(t)\d t}}{e^{-\lambda_1 x}}}\eq{\frac{\infty}{\infty}}\lim_{ x\rightarrow\infty }{\frac{e^{-\lambda_1 x}\cdot b(x)}{-\lambda_1 e^{-\lambda_1 x}}}=\lim_{ x\rightarrow\infty }{\frac{b(x)}{-\lambda_1}}=\frac{L-L}{-\lambda_1}=0 \]
\end{rlist}
Αποδεικνύεται λοιπόν σε κάθε περίπτωση ότι
\[ \lim_{ x\rightarrow\infty }{z(x)}=0\Rightarrow\lim_{ x\rightarrow\infty }{\left(y-\frac{L}{\omega^2} \right) }=0\Rightarrow\lim_{ x\rightarrow\infty }{y(x)}=\frac{L}{\omega^2} \]
Επιπλέον έχουμε $ z'=y' $ και συγκεκριμένα
\begin{align*}
\hspace{-2mm}y'(x)&=\frac{1}{\lambda_2-\lambda_1}\left[ -\lambda_1e^{\lambda_1 x}\int_{0}^{x}{e^{-\lambda_1 t} b(t)\d t}-e^{(\lambda_1-\lambda_1) x}b(x)+\lambda_2e^{\lambda_2 x}\int_{0}^{x}{e^{-\lambda_2 t} b(t)\d t}+e^{(\lambda_2-\lambda_2)x}b(x)\right]=\\
&=\frac{1}{\lambda_2-\lambda_1}\left[ -\lambda_1e^{\lambda_1 x}\int_{0}^{x}{e^{-\lambda_1 t}\cdot b(t)\d t}+\lambda_2e^{\lambda_2 x}\int_{0}^{x}{e^{-\lambda_2 t}\cdot b(t)\d t}\right]
\end{align*}
όπου αν εργαστούμε όπως παραπάνω παίρνουμε ομοίως ότι
\[ \lim_{ x\rightarrow\infty }{z'(x)}=\lim_{ x\rightarrow\infty }{y'(x)}=0 \]\epask
\begin{Askhshs}[B]
\bmath{Έστω η γραμμική διαφορική εξίσωση
\begin{equation}\label{a57:1}\tag{$ E $}
y''+2ay'+\omega^2y=c\cos{(\omega x)}
\end{equation}	 
όπου $ a\geq0,\ \omega>0 $ και $ c\neq0 $ είναι πραγματικές σταθερές. Να βρεθούν όλες οι πραγματικές λύσεις της \eqref{a57:1}. Στη συνέχεια να βρεθεί η λύση $ y_0 $ της \eqref{a57:1} που πληροί τις αρχικές συνθήκες
\[ y_0(0)=y'(0)=0. \]}
\end{Askhshs}\mbox{}\\\\
\lysh
Το χαρακτηριστικό πολυώνυμο της αντίστοιχης ομογενούς εξίσωσης είναι το $ p(\lambda)=\lambda^2+2a\lambda+\omega^2 $ με διακρίνουσα \[ \varDelta=(2a)^2-4\omega^2=4a^2-4\omega^2=4\left( a^2-\omega^2\right) \]
Διακρίνουμε τις εξής περιπτώσεις:
\begin{enumerate}
\item Αν $ \varDelta>0\Rightarrow a^2>\omega^2 $ τότε το πολυώνυμο $ p(\lambda) $ έχει πραγματικές ρίζες τις $ \lambda_1=-a+\sqrt{a^2-\omega^2} $ και $ \lambda_2=-a-\sqrt{a^2-\omega^2} $ και έτσι οι συναρτήσεις $ y_1(x)=e^{\lambda_1x},\ x\in\mathbb{R} $ και $ y_2(x)=e^{\lambda_2x},\ x\in\mathbb{R} $ αποτελούν ένα βασικό σύνολο λύσεων της ομογενούς. Όλες οι λύσεις της θα δίνονται από τον τύπο
\[ \tilde{y}(x)=c_1e^{\lambda_1x}+c_2e^{\lambda_2x}\ ,\ x\in\mathbb{R} \]
όπου $ c_1,c_2 $ είναι αυθαίρετες σταθερές.
\item Αν $ \varDelta=0\Rightarrow a=\omega $ τότε το πολυώνυμο $ p(\lambda) $ έχει μια ρίζα πολλαπλότητας $ 2 $ την $ \lambda_0=-a $. Το βασικό σύνολο λύσεων λοιπόν θα είναι $ \left\lbrace e^{-ax},xe^{-ax} \right\rbrace, x\in\mathbb{R} $ και οι λύσεις δίνονται από τον τύπο
\[ \tilde{y}(x)=c_1e^{-ax}+c_2xe^{-ax}\ ,\ x\in\mathbb{R} \]
όπου $ c_1,c_2 $ είναι αυθαίρετες σταθερές.
\item Εάν τέλος $ \varDelta<0\Rightarrow a^2<\omega^2 $ τότε το πολυώνυμο έχει συζυγείς μιγαδικές ρίζες τις $ \lambda_{1,2}=-a\pm i\sqrt{\omega^2-a^2} $ και έτσι το βασικό σύνολο λύσεων αποτελείται από τις συναρτήσεις \[  y_1(x)=e^{-ax}\cos{\left(\sqrt{\omega^2-a^2} \right) },\ x\in\mathbb{R} \ \ \textrm{και}\ \  y_2(x)=e^{-ax}\sin{\left(\sqrt{\omega^2-a^2} \right) },\ x\in\mathbb{R} \]
Οι λύσεις της ομογενούς εξίσωσης θα είναι:
\[ \tilde{y}(x)=e^{-ax}\left( c_1\cos{\left(\sqrt{\omega^2-a^2} \right)}+c_2\sin{\left(\sqrt{\omega^2-a^2} \right)}\right) \ \ ,\ \ x\in\mathbb{R} \]
όπου $ c_1,c_2 $ είναι αυθαίρετες σταθερές.
\end{enumerate}
Στη συνέχεια θα αναζητήσουμε μια μερική λύση $ u_\mu $ της μη ομογενούς εξίσωσης. Θεωρώντας τη διαφορική εξίσωση
\begin{equation}\label{a57:2}
y''+2ay'+\omega^2y=ce^{i\omega x}\ \ ,\ \ x\in\mathbb{R}\ ,\ a\geq 0\ ,\ \omega>0\ ,\ c>0
\end{equation}
Σ' αυτήν θα χρησιμοποιήσουμε το μετασχηματισμό $ y=ze^{i\omega x}\ ,\ x\in\mathbb{R} $ και θα υπολογίσουμε τις παραγώγους $ y' $ και $ y'' $. Για κάθε $ x\in\mathbb{R} $ έχουμε
\begin{align*}
y'&=z'e^{i\omega x}+i\omega ze^{i\omega x}=(z'+i\omega z)e^{i\omega x}\\
y''&=(z''+i\omega z')e^{i\omega x}+i\omega (z'+i\omega z)e^{i\omega x}=\\&=\left(z''+2i\omega z'+\omega^2\right)e^{i\omega x} 
\end{align*}
Αυτές θα αντικατασταθούν στην \eqref{a57:2} και θα προκύψει
\begin{gather*}
y''+2ay'+\omega^2y=ce^{i\omega x}\Rightarrow\\
\left(z''+2i\omega z'+\omega^2\right)e^{i\omega x}+2a(z'+i\omega z)e^{i\omega x}+\omega^2 y=ce^{i\omega x}\Rightarrow\\
z''+(2a+2i\omega)z'+2ai\omega z=c\ \ ,\ \ x\in\mathbb{R}
\end{gather*}
Για τη μερική λύση $ z_\mu $ της τελευταίας εξίσωσης θα διακρίνουμε τις εξής περιπτώσεις για την παράμετρο $ a $:
\begin{rlist}
\item Αν $ a>0 $ τότε αναζητούμε μια λύση της μορφής $ z_\mu=d $ άρα με αντικατάσταση θα πάρουμε
\[ 2ai\omega d=c\Rightarrow d=-i\frac{c}{2a\omega} \]
επομένως $ z_\mu(x)=-i\frac{c}{2a\omega}\ ,\ x\in\mathbb{R} $. Αυτή με τη σειρά της μας δίνει $ y_\mu(x)=-i\frac{c}{2a\omega}e^{i\omega x}\ ,\ x\in\mathbb{R} $ από την οποία θα χρειαστούμε το πραγματικό μέρος της λύσης μιας και ο μη ομογενής όρος της αρχικής είναι όρος συνημιτόνου. Έτσι
\begin{align*}
u_\mu(x)&=\Re(y_\mu(x))=\\&=\Re\left(-i\frac{c}{2a\omega}(\cos{\omega x}+i\sin{\omega x})\right)=\\&=\frac{c}{2a\omega}\sin{\omega x}\ \ ,\ \ x\in\mathbb{R} 
\end{align*}
\item Αν $ a=0 $ τότε η εξίσωση θα γίνει
\[ z''+2i\omega z'=c \]
συνεπώς πρέπει $ z_\mu'(x)=d $ και με αντικατάσταση στην εξίσωση θα προκύψει
\[ 2i\omega d=c\Rightarrow d=-i\frac{c}{2\omega} \]
άρα θα είναι $ z_\mu(x)=-i\frac{c}{2\omega}x\ ,\ x\in\mathbb{R} $. Όπως και προηγουμένως η ζητούμενη μερική λύση αποτελεί το πραγματικό μέρος της $ y_\mu $ η οποία θα είναι
\begin{align*}
u_\mu(x)&=\Re(y_\mu(x))=\\&=\Re\left(-i\frac{cx}{2a\omega}(\cos{\omega x}+i\sin{\omega x})\right)=\\&=\frac{cx}{2a\omega}\sin{\omega x}\ \ ,\ \ x\in\mathbb{R} 
\end{align*}
\end{rlist}
Συγκεντρωτικά λοιπόν οι λύσεις της αρχικής μη ομογενούς εξίσωσης, για κάθε τιμή της παραμέτρου $ a $ θα δίνονται από τον παρακάτω τύπο σε καθεμία από τις παρακάτω περιπτώσεις
\begin{rlist}
\item Για $ a>0 $ θα είναι
\[ y(x)=\ccases{c_{1} e^{(-a+\sqrt{a^{2}-\omega^{2}}) x}+c_{2} e^{(-a-\sqrt{a^{2}-\omega^{2}}) x}+\frac{c}{2 a \omega} \sin \omega x \quad , \quad \textrm{αν} \quad a^{2}>\omega^{2} \\ c_{1} e^{-a x}+c_{2} x e^{-a x}+\frac{c}{2 a \omega} \sin \omega x \quad, \quad \textrm{αν} \quad a^{2}=\omega^{2} \\ c_{1} e^{-a x} \cos (\sqrt{\omega^{2}-a^{2}} x)+c_{2} e^{-a x} \sin (\sqrt{\omega^{2}-a^{2}} x)+\frac{c}{2 a \omega} \sin \omega x, \textrm { αν } a^{2}<\omega^{2}} \]
με $ c_1,c_2 $ αυθαίρετες σταθερές και $ x\in\mathbb{R} $.
\item Για $ a=0 $ θα έχουμε
\[ y(x)=c_1\cos{\omega x}+c_2\sin{\omega x}+\frac{c}{2\omega}x\sin{\omega x}\ \ ,\ \ x\in\mathbb{R} \]
με $ c_1,c_2 $ αυθαίρετες σταθερές. Εδώ οι περιπτώσεις 1 και 2 δεν ισχύουν καθώς το πολυώνυμο ισούται με $ P(\lambda)=\lambda^2+\omega^2 $ και έχει αρνητική διακρίνουσα. 
\end{rlist}
Συνεχίζουμε στην εύρεση της μερικής λύσης $ y_0 $ που ικανοποιεί τις αρχικές συνθήκες $ y_0(0)=y_0'(0)=0 $. Σε καθεμιά από τις προηγούμενες περιπτώσεις θα έχουμε
\begin{rlist}
\item Αν $ a>0 $
\begin{alist}
\item και $ a^2>\omega^2 $ τότε $ \forall x\in\mathbb{R} $ είναι:
\begin{align*}
y'(x)=c_1(-a+\sqrt{a^2-\omega^2})e^{(-a+\sqrt{a^2+\omega^2})x}+c_2(-a-\sqrt{a^2-\omega^2})e^{(-a-\sqrt{a^2-\omega^2})x}+\frac{c}{2a}\cos{\omega x}
\end{align*}
Επομένως για $ x=0 $ θα πάρουμε
\[ \ccases{c_1+c_2=0\\
c_1(-a+\sqrt{a^2-\omega^2})+c_2(-a-\sqrt{a^2-\omega^2})+\frac{c}{2a}=0}\Rightarrow c_1=-c_2=-\frac{c}{4a\sqrt{a^2-\omega^2}}\]
\item και $ a^2=\omega^2 $ τότε $ \forall x\in\mathbb{R} $ έχουμε
\[ y'(x)=-c_1ae^{-ax}-c_2ae^{-ax}x+c_2e^{-ax}+\frac{c}{2a}\cos{\omega x} \]
από τις αρχικές συνθήκες λοιπόν θα προκύψει
\[ \ccases{c_1=0\\-ac_1+c_2+\frac{c}{2a}=0}\Rightarrow c_1=0\ \textrm{και}\ c_2=-\frac{c}{2a} \]
\item Τέλος για $ a^2<\omega^2 $ και $ x\in\mathbb{R} $ θα είναι
\begin{align*}
y'(x)=&-c_1ae^{-ax}\cos{(\sqrt{\omega^2+a^2}x)}-\sqrt{\omega^2-a^2}c_1\sin{(\sqrt{\omega^2-a^2}x)}-\\&=-c_2ae^{-ax}\sin{(\sqrt{\omega^2-a^2}x)}+\sqrt{\omega^2-a^2}c_2e^{-ax}\cos{(\sqrt{\omega^2-a^2}x)}+\frac{c}{2a}\cos{\omega x}
\end{align*}
Έτσι λοιπόν θα ισχύει
\[ \ccases{c_1=0\\-ac_1+\sqrt{\omega^2-a^2}c_2+\frac{c}{2a}=0}\Rightarrow c_1=0\ \textrm{και}\ c_2=-\frac{c}{2a\sqrt{\omega^2-a^2
}} \]
\end{alist}
\item Για $ a=0 $ θα ισχύει
\[ y'(x)=-c_1\omega\sin{\omega x}+c_2\omega\cos{\omega x}+\frac{c}{2\omega}\sin{\omega x}+\frac{c}{2}x\cos{\omega x}\ ,\ x\in\mathbb{R} \]
Επομένως θέτοντας όπου $ x=0 $ παίρνουμε
\[ \ccases{c_1=0\\\omega c_2=0}\Rightarrow c_1=c_2=0 \]
\end{rlist}
Συγκεντρώνουμε όλα τα παραπάνω συμπεράσματα και σχηματίζουμε για κάθε τιμή της παραμέτρου $ a $ τη λύση $ y_0 $ της εξίσωσης. Για $ a>0 $ θα είναι
\[ y_0(x)=\ccases{-\frac{c}{4a\sqrt{a^2-\omega^2}} e^{(-a+\sqrt{a^{2}-\omega^{2}}) x}+\frac{c}{4a\sqrt{a^2-\omega^2}} e^{(-a-\sqrt{a^{2}-\omega^{2}}) x}+\frac{c}{2 a \omega} \sin \omega x \quad , \quad \textrm{αν} \quad a^{2}>\omega^{2} \\ -\frac{c}{2a}x e^{-a x}+\frac{c}{2 a \omega} \sin \omega x \quad, \quad \textrm{αν} \quad a^{2}=\omega^{2} \\ -\frac{c}{2a\sqrt{\omega^2-a^2
}} e^{-a x} \sin (\sqrt{\omega^{2}-a^{2}} x)+\frac{c}{2 a \omega} \sin \omega x\ \ ,\ \ \textrm { αν } a^{2}<\omega^{2}} \]
Αντίστοιχα για $ a=0 $ η λύση θα δίνεται από τον τύπο
\[ y_0(x)=\frac{c}{2\omega}x\sin{\omega x}\ \ ,\ \ x\in\mathbb{R} \]
\begin{Askhshs}[B]
\bmath{Έστω η μη ομογενής γραμμική διαφορική εξίσωση
\begin{equation}\label{a58:1}\tag{$ E $}
y'''+a_2y''+a_1y'+a_0y=b
\end{equation}
όπου $ a_0,a_1,a_2 $ και $ b $ είναι μια σταθερή συνάρτηση στο διάστημα $ [0,\infty) $.
\begin{rlist}
\item Ας είναι $ y_0 $ η λύση της ομογενούς γραμμικής διαφορικής εξίσωσης
\begin{equation}\label{a58:2}\tag{$ E_0 $}
y'''+a_2y''+a_1y'+a_0y=0
\end{equation}
η οποία πληροί τις αρχικές συνθήκες \[ y_0(0)=y'_0(0)=0\ \ \textrm{και}\ \ y''_0(0)=1 \]
Να αποδειχθεί ότι μια μερική λύση της \eqref{a58:1} είναι η \[ y\mu(x)=\int_{0}^{x}{y_0(x-t)\cdot b(t)\d t}\ \ \textrm{για}\ x\geq 0 \]
\item Ας υποθέσουμε ότι το πολυώνυμο $ p(\lambda)\lambda^3+a_2\lambda^2+a_1\lambda+a_0 $ έχει μια ρίζα $ \lambda_1 $ με $ \Re{\lambda_1}\leq 0 $ και μια διπλή ρίζα $ \lambda_2 $ με $ \Re{\lambda_2}<0 $. Ακόμα ας υποθέσουμε ότι \[ \int_{0}^{\infty}{|b(x)|\d x}<\infty \]
Να αποδειχθεί ότι οι λύσεις της \eqref{a58:1} είναι φραγμένες στο διάστημα $ [0,\infty) $.
\end{rlist}}
\end{Askhshs}\mbox{}\\\\
\lysh\vspace{-5mm}
\begin{rlist}
\item Γνωρίζουμε ότι ισχύει ο εξής τύπος \[ \diff{}{x}\left[ \int_{a(x)}^{\beta(x)}{F(x,t)\d t}\right]=\int_{a(x)}^{\beta(x)}{F(x,t)\d t}+F(x,\beta(x))\cdot\diff{}{x} \beta(x)-F(x,a(x))\cdot\diff{}{x} a(x)  \]
Έτσι λοιπόν έχουμε $ \forall x\geq 0 $ τις τρεις πρώτες παραγώγους της $ y_\mu $ να είναι:
\[ y_\mu(x)=\int_{0}^{x}{y_0(x-t)\cdot b(t)\d t} \]
\[ y'_\mu(x)=\int_{0}^{x}{y_0'(x-t)\cdot b(t)\d t}+y_0(0)b(x)-y_0(x)\cdot b(0)\cdot 0=\int_{0}^{x}{y'_0(x-t)\cdot b(t)\d t} \]
\[ y''_\mu(x)=\int_{0}^{x}{y_0''(x-t)\cdot b(t)\d t}+y'_0(0)\cdot b(x)\cdot 1-y'_0(x)\cdot b(0)\cdot 0=\int_{0}^{x}{y_0''(x-t)\cdot b(x)\d t} \]
και ομοίως για την τρίτη παράγωγο
\[ y'''_\mu(x)=\int_{0}^{x}{y'''_0(x-t)\cdot b(t)\d t}+y''_0(0)\cdot b(x)\cdot 1-y_0''(x)\cdot b(0)\cdot 0=\int_{0}^{x}{y_0'''(x-t)\cdot b(x)\d t}+b(x) \]
Με αντικατάσταση των παραγώγων αυτών στη μη ομογενή εξίσωση \eqref{a58:1} θα δούμε ότι η $ y_\mu(x) $ αποτελεί όντως λύση της. Θα είναι πράγματι
\begin{gather*}
y'''_\mu(x)+a_2y''_\mu(x)+a_1y'_\mu(x)+a_0y_\mu(x)=b(x)\Rightarrow
\end{gather*}
\begin{multline}
\int_{0}^{x}{y_0'''(x-t)\cdot b(x)\d t}+b(x)+a_2\int_{0}^{x}{y_0''(x-t)\cdot b(x)\d t}+\\+a_1\int_{0}^{x}{y_0'(x-t)\cdot b(x)\d t}+a_0\int_{0}^{x}{y_0(x-t)\cdot b(x)\d t}=b(x)\Rightarrow
\end{multline}
\begin{gather*}
b(x)+\int_{0}^{x}{\left[ y_0'''(x-t)+a_2y_0''(x-t)+a_1y_0'(x-t)+a_0y_0(x-t)\right]\cdot b(x)\d t }=b(x)\Rightarrow\\
b(x)=b(x)
\end{gather*}
\item Οι συναρτήσεις $ y_1(x)=e^{\lambda_1x}\ ,\ x\geq 0,\ y_2(x)=e^{\lambda_2 x}\ ,\ x\geq 0 $ και $ y_3(x)=xe^{\lambda_2x}\ ,\ x\geq 0 $ αποτελούν ένα βασικό σύνολο λύσεων \eqref{a58:2} συνεπώς η γενική λύση της θα δίνεται από τον τύπο
\[ \tilde{y}(x)=c_1e^{\lambda_1x}+c_2e^{\lambda_2x}+c_3xe^{\lambda_2 x}\ \ ,\ \ x\geq 0 \]
με $ c_1,c_2,c_3 $ αυθαίρετες σταθερές. Αφού $ \Re{\lambda_1}\geq 0 $ και $ \Re{\lambda_2}<0 $ τότε θεωρώντας ότι $ \lambda_1=a_1+i\beta_1 $ και $ \lambda_2=a_2+i\beta_2 $ παίρνουμε ότι
\[ y_1(x)=e^{a_1x}\left( \cos{\beta_1 x}+i\sin{\beta_1 x}\right)\xRightarrow{x\to\infty}=0 \]
ως μηδενική επί φραγμένη όπως ομοίως και η $ y_2(x) $. Παράλληλα ισχύει ότι \[ \lim_{x\to\infty}{(xe^{\lambda_2x})}=0 \]
σύμφωνα με τον κανόνα του Del Hospital. Καταλήγουμε λοιπόν στο συμπέρασμα ότι αφού οι συναρτήσεις $ y_1,y_2,y_3 $ είναι φραγμένες τότε θα είναι και όλες οι λύσεις της \eqref{a58:2} φραγμένες στο διάστημα $ [0,\infty) $. Μένει να δείξουμε ότι η μερική λύση $ y_\mu $ της \eqref{a58:1} είναι και αυτή φραγμένη στο ίδιο διάστημα. Καθώς αποδείξαμε στο προηγούμενο ερώτημα ότι η $ y_0 $ είναι φραγμένη, θα υπάρχει θετική σταθερά $ K $ τέτοια ώστε να ισχύει $ |y_0(x)|\leq K\ ,\forall x\geq 0 $. Έτσι θα έχουμε
\begin{align*}
|y_\mu(x)|=\left| y_0(x-t)\cdot b(t)\d t\right|&\leq \int_{0}^{x}{|y_0(x-t)|\cdot|b(t)|\d t}\leq\\
&\leq K\int_{0}^{x}{|b(t)|\d t}\leq K\int_{0}^{\infty}{|b(t)|\d t}<\infty
\end{align*}
Καθώς λοιπόν αποδείχθηκε και η $ y_\mu $ φραγμένη τότε όλες οι λύσεις της \eqref{a58:1} είναι φραγμένες στο διάστημα $ [0,\infty) $.
\end{rlist}\mbox{}\\\\
\begin{Askhshs}[B]
\bmath{Δίνεται η γραμμική διαφορική εξίσωση
\begin{equation}\label{a59}\tag{$ E $}
a_2y''+a_1y'+a_0y=b
\end{equation}
όπου $ a_0,a_1 $ και $ a_2\neq 0 $ είναι σταθερές και $ b $ είναι μια συνεχής συνάρτηση στο διάστημα $ [0,\infty) $. Ας είναι $ \lambda_1 $ και $ \lambda_2 $ οι ρίζες του πολυωνύμου $ a_2\lambda^2+a_1\lambda+a_0 $ με $ \lambda_1\neq\lambda_2 $ και ας υποθέσουμε ότι $ \Re{\lambda_1}<0 $ και $ \Re{\lambda_2}<0 $. Επίσης, ας υποθέσουμε ότι υπάρχει σταθερά $ C\geq 0 $ τέτοια ώστε να ισχύει
\[ \int_{x}^{x+1}|b(t)|\d t\leq C\ \ ,\ \ \textrm{για κάθε }x\geq 0 \]
Να αποδειχθεί ότι όλες οι λύσεις της \eqref{a59} είναι φραγμένες στο διάστημα $ [0,\infty) $.\\
\textit{Υπόδειξη}: να αποδειχθεί και στη συνέχεια να χρησιμοποιηθεί το ακόλουθο: Ας είναι $ \Phi $ μια συνεχής μη αρνητική συνάρτηση στο διάστημα $ [0,\infty) $ τέτοια ώστε, για κάποια σταθερά $ C\geq 0 $, να ισχύει
\[ \int_{x}^{x+1}{\Phi(t)\d t}\leq C\ \ ,\ \ \textrm{για κάθε} x\geq 0 \]
Τότε για οποιοδήποτε $ a>0 $, ισχύει
\[ e^{-ax}\int_{0}^{x}{e^{at}\Phi(t)\d t}\leq \frac{C}{1-e^{-a}}\ \ ,\ \ \forall x\in\mathbb{R} \]}
\end{Askhshs}\mbox{}\\\\
\lysh
Το χαρακτηριστικό πολυώνυμο της αντίστοιχης ομογενούς εξίσωσης είναι το $ P(\lambda)=a_2\lambda^2+a_1\lambda+a_0 $. Από την υπόθεση γνωρίζουμε ότι το $ P(\lambda) $ έχει διακεκριμένες ρίζες, έστω $ \lambda_1,\lambda_2 $. Έτσι οι συναρτήσεις $ y_1(x)=e^{\lambda_1x},\ x\geq 0 $ και $ y_2(x)=e^{\lambda_2x},\ x\geq 0 $ αποτελούν ένα βασικό σύνολο λύσεων της. Η γενική της λύση θα δίνεται από τον τύπο
\[ \tilde{y}(x)=c_1e^{\lambda_1x}+c_2e^{\lambda_2x}\ \ ,\ \ x\geq 0 \]
όπου $ c_1,c_2 $ αυθαίρετες σταθερές. Ας υποθέσουμε τώρα ότι $ \lambda_1=a_1+i\beta_1 $ και $ \lambda_2=a_2+i\beta_2 $ είναι η μορφή των ριζών του πολυωνύμου. Τότε οι γραμμικά ανεξάρτητες λύσεις που αναφέραμε θα πάρουν τη μορφή
\begin{gather*}
e^{\lambda_1x}=e^{a_1x}\left(\cos{\beta_1 x}+i\sin{\beta_1x}\right)\\
e^{\lambda_2x}=e^{a_2x}\left(\cos{\beta_2 x}+i\sin{\beta_2x}\right)
\end{gather*}
Από την υπόθεση γνωρίζουμε ότι $ Re(\lambda_1)<0 $ και $ Re(\lambda_2)<0 $ άρα παίρνουμε
\[ \lim_{x\to\infty}{e^{\lambda_1x}}=\lim_{x\to\infty}{e^{\lambda_2x}}=0 \]
κάτι που δηλώνει πως οι συναρτήσεις $ e^{\lambda_ix},\ i=1,2 $ είναι φραγμένες και κατά συνέπεια φραγμένες θα είναι και όλες οι λύσεις της ομογενούς.
Θα βρούμε μια μερική λύση $ y_\mu $ της μη ομογενούς εξίσωσης. Έχουμε $ \forall x\geq 0 $
\begin{gather*}
W(y_1,y_2)(x)=\begin{vmatrix}
e^{\lambda_1x} & e^{\lambda_2x}\\\lambda_1e^{\lambda_1x} & \lambda_2e^{\lambda_2x}
\end{vmatrix}=(\lambda_2-\lambda_1)e^{(\lambda_1+\lambda_2)x}\\
W_1(y_1,y_2)(x)=\begin{vmatrix}
0 & e^{\lambda_2x}\\1 & \lambda_2e^{\lambda_2x}
\end{vmatrix}=-e^{\lambda_2x}\ \ ,\ \ W_2(y_1,y_2)(x)=\begin{vmatrix}
e^{\lambda_1x} & 0\\\lambda_1e^{\lambda_1x} & 1
\end{vmatrix}=e^{\lambda_1x}
\end{gather*}
Έτσι θα είναι
\begin{align*}
y_\mu(x)&=y_1(x)\int_{0}^{x}{\frac{W_1(y_1,y_2)(t)}{W(y_1,y_2)(t)}\cdot b(t)\d t}+y_2(x)\int_{0}^{x}{\frac{W_2(y_1,y_2)(t)}{W(y_1,y_2)(t)}\cdot b(t)\d t}=\\&=e^{\lambda_1x}\int_{0}^{x}{\frac{-e^{\lambda_2t}}{(\lambda_2-\lambda_1)e^{(\lambda_1+\lambda_2)t}}\cdot b(t)\d t}+e^{\lambda_2x}\int_{0}^{x}{\frac{e^{\lambda_1t}}{(\lambda_2-\lambda_1)e^{(\lambda_1+\lambda_2)t}}\cdot b(t)\d t}=\\&=
\frac{1}{\lambda_2-\lambda_1}\left[ -e^{\lambda_1x}\int_{0}^{x}{e^{-\lambda_1 t}b(t)\d t}+e^{\lambda_2x}\int_{0}^{x}{e^{-\lambda_2 t}b(t)\d t} \right]\ \ ,\ \ x\geq 0
\end{align*}
Για κάθε $ x\geq 0 $ λοιπόν θα έχουμε
\begin{align*}
|\lambda_2-\lambda_1|\cdot|y_\mu(x)|&\leq e^{a_1x}\int_{0}^{x}{e^{-a_1t}|b(t)|\d t}+e^{a_2x}\int_{0}^{x}{e^{-a_2t}|b(t)|\d t}\leq\\
&\leq\frac{C}{1-e^{a_1}}+\frac{C}{1-e^{a_2}}
\end{align*}
Συνεπώς και η $ y_\mu $ είναι φραγμένη στο $ [0,\infty) $ οπότε όλες οι λύσεις της αρχικής εξίσωσης θα είναι φραγμένες στο διάστημα $ [0,\infty) $.
\section{C - Δυναμοσειρές λύσεις γραμμικών διαφορικών εξισώσεων 2\tssL{ης} τάξης}
\begin{Askhshs}[C]
\bmath{Να βρεθούν οι δυναμοσειρές λύσεις, γύρω από το σημείο $ x_0 $, για τις ακόλουθες δύο διαφορετικές εξισώσεις
\begin{multicols}{2}
\begin{rlist}
\item $ 2y''+xy'+y=0 $
\item $ 2y''-3xy'-6y=0 $
\end{rlist}
\end{multicols}}
\end{Askhshs}\mbox{}\\
\lysh
\begin{rlist}
\item Έχουμε διαδοχικά τους συντελεστές τις εξίσωσης $ a_2(x)=2,\ x\in\mathbb{R}, a_1(x)=x,\ x\in\mathbb{R} $ και $ a_0(x)=1,\ x\in\mathbb{R} $. Το σημείο $ x_0=0 $ είναι ένα ομαλό σημείο της διαφορικής εξίσωσης αφού $ a_2(0)=2\neq 0 $ και οι συναρτήσεις $ a_2,a_1,a_0 $ είναι πολυωνυμικές.
Έχουμε λοιπόν $ \forall x\in\mathbb{R} $:
\[ \frac{a_1(x)}{a_2(x)}=\frac{x}{2}=\sum_{n=0}^{\infty}{p_n x^n}\ \ \textrm{και}\ \ \frac{a_0(x)}{a_2(x)}=\frac{1}{2}=\sum_{n=0}^{\infty}{q_n x^n} \]
καθώς οι συναρτήσεις αυτές είναι αναλυτικές στο $ x_0=0 $ με ακτίνες σύγκλισης $ R_1=\infty $ και $ R_2=\infty $ αντίστοιχα. Έστω $ R=\min\{R_1,R_2\}=\infty $. Θεωρούμε επιπλέον ότι οι λύσεις της εξίσωσης έχουν τη μορφή δυναμοσειράς
\[ y(x)=\sum_{n=0}^{\infty}{c_nx^n} \]
Θα ισχύει λοιπόν $ \forall x\in\mathbb{R} $
\[ y'(x)=\sum_{n=1}^{\infty}{nc_nx^{n-1}}\ \ \textrm{και}\ \ y''(x)=\sum_{n=2}^{\infty}{n(n-1)c_nx^{n-2}} \]
Αντικαθιστούμε τις παραγώγους αυτές στην εξίσωση ώστε να προσδιορίσουμε τους συντελεστές $ c_n,\ n=0,1,2,\ldots $. Για κάθε $ x\in\mathbb{R} $ θα είναι
\begin{gather*}
2y''+xy'+y=0\Rightarrow\\
2\sum_{n=2}^{\infty}{n(n-1)c_nx^{n-2}}+x\sum_{n=1}^{\infty}{nc_nx^{n-1}}+\sum_{n=0}^{\infty}{c_nx^{n}}=0\Rightarrow\\
\sum_{n=2}^{\infty}{2n(n-1)c_nx^{n-2}}+\sum_{n=1}^{\infty}{nc_nx^{n}}+\sum_{n=0}^{\infty}{c_nx^{n}}=0\Rightarrow\\
\sum_{n=0}^{\infty}{2(n+2)(n+1)c_{n+2}x^{n}}+\sum_{n=1}^{\infty}{nc_nx^{n}}+\sum_{n=0}^{\infty}{c_nx^{n}}=0\Rightarrow\\
4c_2+c_0+\sum_{n=1}^{\infty}{[2(n+2)(n+1)c_{n+2}+(n+1)c_{n}]x^n}=0
\end{gather*}
Θα πρέπει λοιπόν να ισχύει
\[ 4c_2+c_0=0\Rightarrow c_2=-\frac{1}{4}c_0 \] και συγχρόνως
\[ 2(n+2)(n+1)c_{n+2}+(n+1)c_{n}=0\Rightarrow c_{n+2}=-\frac{c_n}{2(n+2)}\ ,\ n=0,1,2,\ldots \]
Οι δείκτες έχουν διαφορά 2 μονάδες για τους συντελεστές με άρτιο δείκτη θα ισχύει
\[ c_{2n}=-\frac{c_{2n-2}}{2\cdot 2n} \]
Έχουμε διαδοχικά τώρα για τις διάφορες τιμές του $ n $ ότι
\[ \begin{rcases}
n=1\to & c_2=-\frac{c_0}{2\cdot 2\cdot 1}\\
n=2\to & c_4=-\frac{c_2}{2\cdot 2\cdot 2}\\
& \textrm{και γενικά}\\
n=n\to & c_{2n}=-\frac{c_{2n-2}}{2\cdot 2\cdot n}
\end{rcases}\Rightarrow c_{2n}=\frac{(-1)^n}{2^{2n}\cdot n!}c_0\ ,\ n=1,2,\ldots \]
και επιπλέον οι συντελεστές με περιττούς δείκτες θα έχουν τη μορφή $ c_{2n+1}=-\frac{c_{2n-1}}{2(2n+1)} $ και έτσι έχουμε
\[ \begin{rcases}
n=1\to & c_3=-\frac{c_1}{2\cdot 3}\\
n=2\to & c_5=-\frac{c_3}{2\cdot 5}\\
& \textrm{και γενικά}\\
n=n\to & c_{2n+1}=-\frac{c_{2n-1}}{2(2n+1)}
\end{rcases}\Rightarrow c_{2n+1}=\frac{(-1)^n}{2^{n}\cdot (3\cdot 5\cdot\ldots(2n+1))}c_1\ ,\ n=1,2,\ldots \]
Σχηματίζουμε σ' αυτό το σημείο τη λύση της εξίσωσης χωρίζοντας τους όρους της σειράς σε άρτιους και περιττούς και σύμφωνα με τα παραπάνω θα γραφτεί ως γραμμικός συνδυασμός δύο γραμμικά ανεξάρτητων συναρτήσεων
\begin{align*}
y(x)=\sum_{n=0}^{\infty}{c_nx^n}&=\sum_{n=0}^{\infty}{c_{2n}x^{2n}}+\sum_{n=0}^{\infty}{c_{2n+1}x^{2n+1}}=\\&=c_0+c_1x+\sum_{n=1}^{\infty}{c_{2n}x^{2n}}+\sum_{n=1}^{\infty}{c_{2n+1}x^{2n+1}}=\\
&=c_0+c_1x+\sum_{n=1}^{\infty}{\frac{(-1)^n}{2^{2n}\cdot n!}c_0x^{2n}}+\sum_{n=1}^{\infty}{\frac{(-1)^n}{2^{n}\cdot (3\cdot 5\cdot\ldots(2n+1))}c_1x^{2n+1}}=\\
&=c_0\left[ 1+\sum_{n=1}^{\infty}{\frac{(-1)^n}{2^{2n}\cdot n!}x^{2n}}\right]+c_1\left[x+\sum_{n=1}^{\infty}{\frac{(-1)^n}{2^{n}\cdot (3\cdot 5\cdot\ldots(2n+1))}x^{2n+1}}\right]
\end{align*}
όπου $ c_0,c_1 $ είναι αυθαίρετες σταθερές.
\item Όπως και προηγουμένως, οι συντελεστές τις εξίσωσης είναι $ a_2(x)=2,\ x\in\mathbb{R}, a_1(x)=-3x,\ x\in\mathbb{R} $ και $ a_0(x)=-6,\ x\in\mathbb{R} $ ενώ επιπλέον το σημείο $ x_0=0 $ είναι ένα ομαλό σημείο της διαφορικής εξίσωσης.
Μπορούμε λοιπόν $ \forall x\in\mathbb{R} $ να αναπτύξουμε τις συναρτήσεις αυτές σε δυναμοσειρές :
\[ \frac{a_1(x)}{a_2(x)}=-\frac{3x}{2}=\sum_{n=0}^{\infty}{p_n x^n}\ \ \textrm{και}\ \ \frac{a_0(x)}{a_2(x)}=-3=\sum_{n=0}^{\infty}{q_n x^n} \]
καθώς είναι αναλυτικές στο $ x_0=0 $ με ακτίνες σύγκλισης $ R_1=\infty $ και $ R_2=\infty $ αντίστοιχα. Έστω $ R=\min\{R_1,R_2\}=\infty $. Έστω επιπλέον ότι οι λύσεις της εξίσωσης έχουν τη μορφή δυναμοσειράς με ακτίνα $ R $ 
\[ y(x)=\sum_{n=0}^{\infty}{c_nx^n} \]
Οι παράγωγοι έως και 2\tss{ης} τάξης $ \forall x\in\mathbb{R} $ θα είναι αντίστοιχα
\[ y'(x)=\sum_{n=1}^{\infty}{nc_nx^{n-1}}\ \ \textrm{και}\ \ y''(x)=\sum_{n=2}^{\infty}{n(n-1)c_nx^{n-2}} \]
Αντικαθιστούμε αυτές στην εξίσωση ώστε να προσδιορίσουμε τους συντελεστές $ c_n,\ n=0,1,2,\ldots $. Για κάθε $ x\in\mathbb{R} $ θα είναι
\begin{gather*}
2y''-3xy'-6y=0\Rightarrow\\
2\sum_{n=2}^{\infty}{n(n-1)c_nx^{n-2}}-3x\sum_{n=1}^{\infty}{nc_nx^{n-1}}-6\sum_{n=0}^{\infty}{c_nx^{n}}=0\Rightarrow\\
\sum_{n=2}^{\infty}{2n(n-1)c_nx^{n-2}}-\sum_{n=1}^{\infty}{3nc_nx^{n}}-\sum_{n=0}^{\infty}{6c_nx^{n}}=0\Rightarrow\\
\sum_{n=0}^{\infty}{2(n+2)(n+1)c_{n+2}x^{n}}-\sum_{n=1}^{\infty}{3nc_nx^{n}}-\sum_{n=0}^{\infty}{6c_nx^{n}}=0\Rightarrow\\
4c_2-6c_0+\sum_{n=1}^{\infty}{[2(n+2)(n+1)c_{n+2}-3(n+2)c_{n}]x^n}=0
\end{gather*}
Θα πρέπει λοιπόν να ισχύει
\[ 4c_2-6c_0=0\Rightarrow c_2=\frac{3}{2}c_0 \] και συγχρόνως
\[ 2(n+2)(n+1)c_{n+2}-3(n+2)c_{n}=0\Rightarrow c_{n+2}=\frac{3c_n}{2(n+1)}\ ,\ n=0,1,2,\ldots \]
Παρατηρούμε ότι οι δείκτες έχουν διαφορά 2 μονάδες οπότε για τους συντελεστές με άρτιο δείκτη θα είναι
\[ c_{2n}=-\frac{c_{2n-2}}{2\cdot(2n-1)} \]
Για τις διάφορες τιμές του $ n $ θα προκύψει
\[ \begin{rcases}
n=1\to & c_2=\frac{3c_0}{2\cdot 1}\\
n=2\to & c_4=\frac{3c_2}{2\cdot 3}\\
& \textrm{και γενικά}\\
n=n\to & c_{2n}=\frac{3c_{2n-2}}{2\cdot(2n-1)}
\end{rcases}\Rightarrow c_{2n}=\frac{3^n}{2^{n}\cdot (1\cdot3\cdot 5\cdot\ldots(2n-1))}c_0\ ,\ n=1,2,\ldots \]
και επιπλέον οι συντελεστές με περιττούς δείκτες θα έχουν τη μορφή $ c_{2n+1}=\frac{3c_{2n-1}}{2\cdot 2n} $ και έτσι έχουμε
\[ \begin{rcases}
n=1\to & c_3=\frac{3c_1}{2\cdot 2\cdot1}\\
n=2\to & c_5=\frac{3c_3}{2\cdot 2\cdot2}\\
& \textrm{και γενικά}\\
n=n\to & c_{2n+1}=\frac{3c_{2n-1}}{2\cdot2n}
\end{rcases}\Rightarrow c_{2n+1}=\frac{3^n}{2^{2n}\cdot n!}c_1\ ,\ n=1,2,\ldots \]
Θα χωρίσουμε τους όρους της σειράς σε άρτιους και περιττούς και με όσα αποδείξαμε, η λύση θα γραφτεί ως γραμμικός συνδυασμός δύο γραμμικά ανεξάρτητων συναρτήσεων. Έχουμε πράγματι
\begin{align*}
y(x)=\sum_{n=0}^{\infty}{c_nx^n}&=\sum_{n=0}^{\infty}{c_{2n}x^{2n}}+\sum_{n=0}^{\infty}{c_{2n+1}x^{2n+1}}=\\&=c_0+c_1x+\sum_{n=1}^{\infty}{c_{2n}x^{2n}}+\sum_{n=1}^{\infty}{c_{2n+1}x^{2n+1}}=\\
&=c_0+c_1x+\sum_{n=1}^{\infty}{\frac{3^n}{2^{n}\cdot (1\cdot3\cdot 5\cdot\ldots(2n-1))}c_0x^{2n}}+\sum_{n=1}^{\infty}{\frac{3^n}{2^{2n}\cdot n!}c_1x^{2n+1}}=\\
&=c_0\left[ 1+\sum_{n=1}^{\infty}{\frac{3^n}{2^{n}\cdot (1\cdot3\cdot 5\cdot\ldots(2n-1))}x^{2n}}\right]+c_1\left[x+\sum_{n=1}^{\infty}{\frac{3^n}{2^{2n}}x^{2n+1}}\right]
\end{align*}
όπου $ c_0,c_1 $ είναι αυθαίρετες σταθερές.\epask
\end{rlist}
\begin{Askhshs}[C]
\bmath{Να επιλυθεί η γραμμική διαφορική εξίσωση
\[ y''-2(x-1)y'-2y=0 \]
Ειδικά να βρεθεί η λύση $ y_0 $ αυτής που πληροί τις αρχικές συνθήκες
\[ y_0(1)=1\ \ ,\ \ y'_0(1)=0 \]}
\end{Askhshs}\mbox{}\\\\
\lysh
Οι συντελεστές των παραγώγων είναι $ a_2(x)=1,x\in\mathbb{R},\ a_1(x)=-2(x-1),x\in\mathbb{R} $ και $ a_0(x)=2 x\in\mathbb{R} $. Το σημείο $ x_0=1 $ ένα ομαλό σημείο της διαφορική εξίσωσης αφού $ a_2(1)=1\neq 0 $ και οι συναρτήσεις $ a_2,a_1,a_0 $ είναι πολυώνυμα. Έχουμε \[ \frac{a_1(x)}{a_2(x)}=-2(x-1)=\sum_{n=0}^{\infty}{p_n(x-1)^n}\ \ \textrm{και}\ \ \frac{a_0(x)}{a_2(x)}=-2=\sum_{n=0}^{\infty}{q_n(x-1)^n} \]
για κάθε $ |x-1|<R_1 $ και $ |x-1|<R_2 $ όπου $ R_{1,2} $ είναι οι ακτίνες σύγκλισης των σειρών. Θέτοντας $ R=\min\{R_1,R_2\} $ τότε η ζητούμενη λύση της εξίσωσης μπορεί να έχει τη μορφή δυναμοσειράς
\[ y(x)=\sum_{n=0}^{\infty}{c_n(x-1)^n}\ \ ,\ \ \textrm{για }|x-1|<R \]
Έχουμε λοιπόν για κάθε $ x\in\mathbb{R} $
\begin{gather*}
y''(x)-2(x-1)y'(x)-2y(x)=0\Rightarrow\\
\sum_{n=2}^{\infty}{n(n-1)c_n(x-1)^{n-2}}-2(x-1)\sum_{n=1}^{\infty}{nc_n(x-1)^{n-1}}-2\sum_{n=0}^{\infty}{c_n(x-1)^n}=0\Rightarrow\\
\sum_{n=2}^{\infty}{n(n-1)c_n(x-1)^{n-2}}-\sum_{n=1}^{\infty}{2nc_n(x-1)^{n}}-\sum_{n=0}^{\infty}{2c_n(x-1)^n}=0\Rightarrow\\
\sum_{n=0}^{\infty}{(n+2)(n+1)c_{n+2}(x-1)^{n}}-\sum_{n=0}^{\infty}{2nc_n(x-1)^{n}}-\sum_{n=0}^{\infty}{2c_n(x-1)^n}=0\Rightarrow\\
\sum_{n=0}^{\infty}{[(n+2)(n+1)c_{n+2}-2(n+1)c_n]}(x-1)^n=0
\end{gather*}
Παίρνουμε έτσι την αναδρομική σχέση των συντελεστών
\[ (n+2)(n+1)c_{n+2}-2(n+1)c_n=0\Rightarrow c_{n+2}=\frac{2}{n+2}c_0\ ,\ n=0,1,2,\ldots \]
Παρατηρούμε ότι οι δείκτες διαφέρουν κατά $ 2 $ μονάδες άρα θα ξεκινώντας με τους συντελεστές άρτιου δείκτη παίρνουμε τη σχέση
\[ c_{2n}=\frac{1}{n}c_{2n-2}\ \ ,\ \ n=1,2,\ldots \] 
Αν εργαστούμε όπως στην προηγούμενη άσκηση επιλέγοντας διαδοχικά τις τιμές του $ n $ τότε εκφράζονται οι συντελεστές αυτοί με τη βοήθεια του $ c_0 $ από τη σχέση
\[ c_{2n}=\frac{1}{n!}c_0\ \ ,\ \ n=1,2,\ldots \]
Εντελώς ανάλογα, για τους συντελεστές με περιττό δείκτη, η αναδρομική σχέση καθώς και τη τελική τους έκφραση ως προς $ c_1 $ θα είναι
\[ c_{2n+1}=\frac{2}{2n+1}c_{2n-1}\Rightarrow c_{2n+1}=\frac{2^n}{[3\cdot 5\cdot\ldots\cdot(2n+1)]}c_1\ \ ,\ \ n=1,2\ldots \]
Η γενική λύση της εξίσωσης θα γραφτεί ως γραμμικός συνδυασμός άρτιων και περιττών δυνάμεων της παράστασης $ x-1 $ και σύμφωνα με τα παραπάνω θα ισούται με
\begin{align*}
y(x)&=\sum_{n=0}^{\infty}{c_n(x-1)^n}=\sum_{n=0}^{\infty}{c_{2n}(x-1)^{2n}}+\sum_{n=0}^{\infty}{c_{2n+1}(x-1)^{2n+1}}=\\
&=c_0+c_1x+\sum_{n=1}^{\infty}{\frac{1}{n!}c_0(x-1)^{2n}}+\sum_{n=1}^{\infty}{\frac{2^n}{[3\cdot 5\cdot\ldots\cdot(2n+1)]}c_1(x-1)^{2n-1}}=\\&=
c_0\left[ 1+\sum_{n=1}^{\infty}{\frac{1}{n!}(x-1)^{2n}}\right] +c_1\left[ x+\sum_{n=1}^{\infty}{\frac{2^n}{[3\cdot 5\cdot\ldots\cdot(2n+1)]}(x-1)^{2n-1}}\right] 
\end{align*}
όπου $ c_0,c_1 $ αυθαίρετες σταθερές. Από τις αρχικές συνθήκες που δίνονται μπορούμε να προσδιορίσουμε αυτές τις σταθερές. Θα έχουμε συγκεκριμένα ότι
\[ \ccases{y(1)=1\Rightarrow c_0+c_1=1\\y'(1)=0\Rightarrow c_1=0}\Rightarrow c_0=1 \]
Έτσι η λύση του προβλήματος αρχικών τιμών δίνεται ως δυναμοσειρά από τον τύπο
\[ y(x)=1+\sum_{n=1}^{\infty}{\frac{1}{n!}(x-1)^{2n}}\ \ ,\ \ \textrm{για }|x-1|<R \]\epask
\begin{Askhshs}[C]
\bmath{Να επιλυθούν τα παρακάτω προβλήματα αρχικών τιμών:
\begin{rlist}
\item $ y''-2xy'-2y=0\ \ ,\ y(0)=1,\ y'(0)=0 $
\item $ y''+xy'+3y=0\ \ ,\ y(0)=-2,\ y'(0)=6 $
\item $ y''+x^2y'+2xy=0\ \ ,\ y(0)=1,\ y'(0)=0 $
\end{rlist}}
\end{Askhshs}\mbox{}\\\\
\lysh
\begin{rlist}
\item Οι συντελεστές της εξίσωσης είναι οι συναρτήσεις $ a_2(x)=1\neq 0, a_1(x)=-2x $ και $ a_0(x)=-2 $. Όπως βλέπουμε, το σημείο $ x=0 $ είναι ομαλό σημείο της εξίσωσης καθώς οι συναρτήσεις 
\[ \frac{a_1(x)}{a_2(x)}=-2x\ \ \textrm{και}\ \ \frac{a_0(x)}{a_2(x)}=-2 \]
είναι αναλυτικές στο σημείο αυτό. Κατά συνέπεια αναπτύσσονται σε δυναμοσειρές ως
\[ \frac{a_1(x)}{a_2(x)}=-2x=\sum_{n=0}^{\infty}{p_nx^n}\ \ \textrm{και}\ \ \frac{a_1(x)}{a_2(x)}=-2=\sum_{n=0}^{\infty}{q_nx^n} \]
με ακτίνες σύγκλισης $ R_1=R_2=\infty $. Θέτουμε λοιπόν $ R=\min\{R_1,R_2\}=\infty $ και θεωρούμε ότι οι λύσεις της εξίσωσης γράφονται σε μορφή δυναμοσειράς 
\[ y(x)=\sum_{n=0}^{\infty}{c_nx^n}\ \ ,\ \ x\in\mathbb{R} \]
Θα έχουμε για κάθε $ x\in\mathbb{R} $ τις δύο πρώτες παραγώγους να ισούνται με
\[ y'(x)=\sum_{n=1}^{\infty}{nc_nx^{n-1}}\ \ \textrm{και}\ \ y''(x)=\sum_{n=2}^{\infty}{n(n-1)c_nx^{n-2}} \]
τις οποίες, μαζί με την $ y $ θα αντικαταστήσουμε στην εξίσωση και θα πάρουμε έτσι
\begin{gather*}
y''(x)-2xy'-2y=0\Rightarrow\\
\sum_{n=2}^{\infty}{n(n-1)c_nx^{n-2}}-2x\sum_{n=1}^{\infty}{nc_nx^{n-1}}-2\sum_{n=0}^{\infty}{c_nx^{n}}=0\Rightarrow\\
\sum_{n=2}^{\infty}{n(n-1)c_nx^{n-2}}-\sum_{n=1}^{\infty}{2nc_nx^{n}}-\sum_{n=0}^{\infty}{2c_nx^{n}}=0\Rightarrow\\
\sum_{n=0}^{\infty}{(n+2)(n+1)c_{n+2}x^{n}}-\sum_{n=1}^{\infty}{2nc_nx^{n}}-\sum_{n=0}^{\infty}{2c_nx^{n}}=0\Rightarrow\\
\sum_{n=0}^{\infty}{(n+2)(n+1)c_{n+2}x^{n}}-\sum_{n=1}^{\infty}{2nc_nx^{n}}-\sum_{n=0}^{\infty}{2c_nx^{n}}=0\Rightarrow\\
2c_2-2c_0+\sum_{n=1}^{\infty}{[(n+2)(n+1)c_{n+2}-2(n+1)c_n]x^n}=0
\end{gather*}
Συνεπώς πρέπει να ισχύει
\[ 2c_2-2c_0=0\Rightarrow c_2=c_0 \]
καθώς και \[ (n+2)(n+1)c_{n+2}-2(n+1)c_n\Rightarrow c_{n+2}=\frac{c_n}{n+2}\ \ ,\ \ n=1,2,\ldots \]
Οι δείκτες έχουν διαφορά όπως βλέπουμε, $ 2 $ μονάδων. Για τους συντελεστές με άρτιο δείκτη θα προκύπτει η αναδρομική σχέση 
\[ c_{2n}=\frac{1}{n}c_{2n-2} \]
όπου διαδοχικά για τις διάφορες τιμές του $ n $ καταλήγουμε στον τύπο
\[ c_{2n}=\frac{1}{n!}c_0\ \ ,\ \ n=1,2,\ldots \]
Αντίστοιχα για τους συντελεστές με περιττούς δείκτες ο αναδρομικός τύπος
\[ c_{2n+1}=\frac{2}{2n-1}c_{2n-1} \]
μας δίνει την έκφρασή τους ως προς το συντελεστή $ c_1 $
\[ c_{2n+1}=\frac{2^n}{3\cdot5\cdot\ldots\cdot(2n+1)}c_1=\frac{2^n}{(2n+1)!!}c_1\ \ ,\ \ n=1,2,\ldots \]
Η λύση, όπως είδαμε και σε προηγούμενα παραδείγματα, θα γραφτεί ως γραμμικός συνδυασμός των άρτιων και περιττών δυνάμεων του $ x $. Θα πάρουμε τελικά ότι
\[ y(x)=c_0\left[1+\sum_{n=1}^{\infty}{\frac{1}{n!}x^{2n}}\right]+c_1\left[x+\sum_{n=1}^{\infty}{\frac{2^n}{(2n+1)!!}x^{2n+1}}\right] \]
Στο σημείο αυτό αν εφαρμόσουμε στη λύση τις αρχικές συνθήκες του προβλήματος θα προκύψει
\[ \ccases{y(0)=1\Rightarrow c_0=1\\y'(0)=0\Rightarrow c_1=0} \]
άρα καταλήγουμε στη λύση του προβλήματος αρχικών τιμών που θα είναι 
\[ y(x)=1+\sum_{n=1}^{\infty}{\frac{1}{n!}x^{2n}} \]
\item Οι συντελεστές της εξίσωσης είναι $ a_2(x)=1\neq 0, a_1(x)=x $ και $ a_0(3) $ για $ x\in\mathbb{R} $. Βλέπουμε ότι οι συναρτήσεις
\[ \frac{a_1(x)}{a_2(x)}=x\ \ \textrm{και}\ \ \frac{a_0(x)}{a_2(x)}=3 \]
άρα το σημείο $ x_0=0 $ είναι ομαλό σημείο της διαφορικής εξίσωσης. Καθεμία απ' αυτές λοιπόν μπορεί να αναπτυχθεί σε δυναμοσειρά γύρω από το σημείο αυτό ως
\[ \frac{a_1(x)}{a_2(x)}=x=\sum_{n=0}^{\infty}{p_nx^n}\ \ \textrm{και}\ \ \frac{a_0(x)}{a_2(x)}=3=\sum_{n=0}^{\infty}{q_nx^n} \]
με ακτίνες σύγκλισης $ R_1=R_2=\infty $. Θέτουμε έτσι $ R=\min\{R_1,R_2\}=\infty $ οπότε με ακτίνα σύγκλισης $ R $, οι λύσεις της εξίσωσης γράφονται με μορφή δυναμοσειράς
\[ y(x)=\sum_{n=0}^{\infty}{c_nx^n}\ \ ,\ \ x\in\mathbb{R} \]
Οι δύο πρώτες παράγωγοι αυτής θα δίνονται από τους τύπους
\[ y'(x)=\sum_{n=1}^{\infty}{nc_nx^{n-1}}\ \ \textrm{και}\ \ y''(x)=\sum_{n=2}^{\infty}{n(n-1)c_nx^{n-2}} \]
και έτσι με αντικατάσταση στην εξίσωση θα πάρουμε μια αναδρομική σχέση που συνδέει τους συντελεστές $ c_n $ των όρων της σειράς. Πράγματι θα είναι
\begin{gather*}
y''+xy'+3y=0\Rightarrow\\
\sum_{n=2}^{\infty}{n(n-1)c_nx^{n-2}}+x\sum_{n=1}^{\infty}{nc_nx^{n-1}}+3\sum_{n=0}^{\infty}{c_nx^{n}}=0\Rightarrow\\
\sum_{n=2}^{\infty}{n(n-1)c_nx^{n-2}}+\sum_{n=1}^{\infty}{nc_nx^{n}}+\sum_{n=0}^{\infty}{3c_nx^{n}}=0\Rightarrow\\
\sum_{n=0}^{\infty}{(n+2)(n+1)c_{n+2}x^{n}}+\sum_{n=1}^{\infty}{nc_nx^{n}}+\sum_{n=0}^{\infty}{3c_nx^{n}}=0\Rightarrow\\
2c_2+3c_0+\sum_{n=1}^{\infty}{[(n+2)(n+1)c_{n+2}+(n+3)c_n]}x^n=0
\end{gather*}
Πρέπει λοιπόν να ισχύουν οι σχέσεις
\[ 2c_2+3c_0=0\Rightarrow c_2=-\frac{3c_0}{2} \]
καθώς και
\[ (n+2)(n+1)c_{n+2}+(n+3)c_n\Rightarrow c_{n+2}=-\frac{n+3}{(n+2)(n+1)}c_{n}\ \ ,\ \ n=1,2,\ldots \]
Στην τελευταία αναδρομική σχέση βλέπουμε ότι οι δείκτες έχουν διαφορά $ 2 $ μονάδων. Για τους συντελεστές με άρτιους δείκτες η σχέση αυτή παίρνει τη μορφή
\[ c_{2n}=-\frac{2n+1}{2n(2n-1)}c_{2n-2}\ \ ,\ \ n=1,2,\ldots \]
από την οποία, για τις διάφορες τιμές του $ n $ προκύπτει ο τύπος
\[ c_{2n}=\frac{(-1)^n3\cdot 5\cdot\ldots\cdot(2n+1)}{2^nn!}c_0=\frac{(-1)^n(2n+1)}{2^nn!}c_0\ \ ,\ \ n=1,2,\ldots \]
Ομοίως για τους συντελεστές με περιττό δείκτη η αναδρομική σχέση παίρνει τη μορφή
\[ c_{2n+1}=-\frac{n+1}{n(2n+1)}c_{2n-1}\ \ ,\ \ n=1,2,\ldots \]
η οποία θα μας δώσει
\[ c_{2n+1}=\frac{(-1)^n(n+1)}{(2n+1)!!}c_1\ \ ,\ \ n=1,2,\ldots \]
Γράφουμε λοιπόν έτσι τη λύση της εξίσωσης ως γραμμικό συνδυασμών δύο δυναμοσειρών με άρτιες και περιττές δυνάμεις του $ x $ αντίστοιχα
\begin{align*}
y(x)&=\sum_{n=0}^{\infty}{c_nx^n}=\sum_{n=0}^{\infty}{c_{2n}x^{2n}}+\sum_{n=0}^{\infty}{c_{2n+1}x^{2n+1}}=\\&=
c_0+c_1x+\sum_{n=1}^{\infty}{\frac{(-1)^n(2n+1)}{2^nn!}c_0x^{2n}}+\sum_{n=1}^{\infty}{\frac{(-1)^n(n+1)}{(2n+1)!!}c_1x^{2n+1}}
\end{align*}
Οι αρχικές συνθήκες στο σημείο αυτό θα μας δώσουν τις τιμές των παραμέτρων $ c_0 $ και $ c_1 $. Έχουμε λοιπόν
\[ \ccases{y(0)=-2\Rightarrow c_0=-2\\y'(0)=6\Rightarrow c_1=6} \]
και έτσι η λύση του προβλήματος αρχικών τιμών θα ισούται με
\[ y(x)=-2+6x-2\sum_{n=1}^{\infty}{\frac{(-1)^n(2n+1)}{2^nn!}x^{2n}}+6\sum_{n=1}^{\infty}{\frac{(-1)^n(n+1)}{(2n+1)!!}x^{2n+1}}\ \ ,\ \ x\in\mathbb{R
} \]
\item Οι συντελεστές της εξίσωσης είναι $ a_2(x)=1\neq 0,\ a_1(x)=x^2,\ x\in\mathbb{R} $ και $ a_0(x)=2x,\ x\in\mathbb{R} $. Εφόσον $ a_2(0)=1\neq 0 $ και οι συναρτήσεις 
\[ \frac{a_1(x)}{a_2(x)}=x^2\ \ \textrm{και}\ \ \frac{a_0(x)}{a_2(x)}=2x \]
είναι αναλυτικές στο σημείο αυτό, τότε το $ x_0=0 $ είναι ομαλό σημείο της εξίσωσης. Καθεμιά απ' αυτές αναπτύσσεται σε δυναμοσειρά γύρω από το ομαλό σημείο
\[ \frac{a_1(x)}{a_2(x)}=x^2=\sum_{n=0}^{\infty}{p_nx^n}\ \ \textrm{και}\ \ \frac{a_0(x)}{a_2(x)}=2x=\sum_{n=0}^{\infty}{q_nx^n} \]
με ακτίνες σύγκλισης $ R_1=R_2=\infty $. Με ακτίνα σύγκλισης λοιπόν $ R=\min\{R_1,R_2\}=\infty $ αναζητούμε τη λύση δυναμοσειρά της διαφορικής εξίσωσης με μορφή
\[ y(x)=\sum_{n=0}^{\infty}{c_nx^n}\ \ ,\ \ x\in\mathbb{R} \]
Αντικαθιστούμε τη συνάρτηση αυτή καθώς και τις παραγώγους της στην αρχική εξίσωση και καταλήγουμε σε μια αναδρομική ακολουθία των συντελεστών $ c_n $ της σειράς. Θα έχουμε λοιπόν
\begin{gather*}
y''+x^2y'+2xy=0\Rightarrow\\
\sum_{n=2}^{\infty}{n(n-1)c_nx^{n-2}}+x^2\sum_{n=1}^{\infty}{nc_nx^{n-1}}+2x\sum_{n=0}^{\infty}{c_nx^n}=0\Rightarrow\\
\sum_{n=2}^{\infty}{n(n-1)c_nx^{n-2}}+\sum_{n=1}^{\infty}{nc_nx^{n+1}}+\sum_{n=0}^{\infty}{2c_nx^{n+1}}=0\Rightarrow\\
\sum_{n=0}^{\infty}{(n+2)(n+1)c_{n+2}x^{n}}+\sum_{n=2}^{\infty}{(n-1)c_{n-1}x^{n}}+\sum_{n=1}^{\infty}{2c_{n-1}x^{n}}=0\Rightarrow\\
\sum_{n=0}^{\infty}{(n+2)(n+1)c_{n+2}x^{n}}+\sum_{n=1}^{\infty}{(n-1)c_{n-1}x^{n}}+\sum_{n=1}^{\infty}{2c_{n-1}x^{n}}=0\Rightarrow\\
4c_2+\sum_{n=1}^{\infty}{\left[(n+2)(n+1)c_{n+2}+(n+1)c_{n-1}\right]x^n}=0
\end{gather*}
Πρέπει λοιπόν να ισχύει $ 4c_2=0\Rightarrow c_2=0 $ καθώς και 
\[ (n+2)(n+1)c_{n+2}+(n+1)c_{n-1}\Rightarrow c_{n+2}=-\frac{1}{n+2}c_{n-1}\ \ ,\ \ n=1,2,\ldots \]
Στη σχέση αυτή οι δείκτες έχουν διαφορά $ 3 $ μονάδες. Κατά συνέπεια θα διαχωρίσουμε τους συντελεστές σε κλάσεις της μορφής $ 3n,3n+1 $ και $ 3n+2 $. Θέτοντας $ n\to 3n-2 $ παίρνουμε τον αναδρομικό τύπο
\[ c_{3n}=-\frac{1}{3n}c_{3n-3}\ \ ,\ \ n=1,2,\ldots \]
όπου για τις διάφορες διαδοχικές τιμές του $ n $ θα καταλήξουμε στην επόμενη σχέση
\[ c_{3n}=\frac{(-1)^n}{3^nn!}c_0\ \ ,\ \ n=1,2,\ldots \]
Ομοίως για $ n\to 3n-1 $ παίρνουμε αντίστοιχη σχέση για τους συντελεστές με δείκτη $ 3n+1 $. Θα είναι λοιπόν
\[ c_{3n+1}=\frac{1}{3n+1}c_{3n-2}\ \ ,\ \ n=1,2,\ldots \]
η οποία θα μας δώσει\footnote{Ο ορισμός του πολλαπλού παραγοντικού που χρησιμοποιείται για την απλοποίηση των συντελεστών, δίνεται από τον τύπο
\[ n!^{(k)}=\prod_{m=0}^{\lfloor\frac{n}{k} \rfloor}{(n-mk)} \]}
\[ c_{3n+1}=\frac{(-1)^n}{1\cdot4\cdot7\cdot\ldots\cdot(3n+1)}c_1=\frac{(-1)^n}{(3n+1)!!!}c_1 \]
%\rule{\linewidth}{0.3mm}
%\textit{Σημείωση:}
%\rule{\linewidth}{0.3mm}
Θέτοντας τέλος $ n\to 3n $ θα πάρουμε έναν αντίστοιχο αναδρομικό τύπο για τους συντελεστές με δείκτη $ 3n+2 $. Εφόσον όμως αυτοί θα εκφραστούν στη συνέχεια συναρτήσει του συντελεστή $ c_2 $, μηδενίζονται για κάθε $ n=1,2,\ldots $ σύμφωνα με τη συνθήκη $ c_2=0 $. Κάτι τέτοιο ήταν αναμενόμενο καθώς σε αντίθετη περίπτωση θα παίρναμε τη λύση μιας δεύτερης τάξης διαφορικής εξίσωσης γραμμένη ως γραμμικό συνδυασμό τριών γραμμικώς ανεξάρτητων συναρτήσεων. Έτσι η λύση θα γραφτεί
\[ y(x)=\sum_{n=0}^{\infty}{\frac{(-1)^n}{3^nn!}c_0x^{3n}}+\sum_{n=0}^{\infty}{\frac{(-1)^n}{(3n+1)!!!}c_1x^{3n+1}} \]
Οι αρχικές συνθήκες με τη σειρά τους μας δίνουν
\[ \ccases{y(0)=1\Rightarrow c_0=1\\y'(0)=0\Rightarrow c_1=0} \]
οπότε η λύση του προβλήματος αρχικών τιμών θα ισούται με
\[ y(x)=\sum_{n=0}^{\infty}{\frac{(-1)^n}{3^nn!}x^{3n}}\ \ ,\ \ x\in\mathbb{R} \]
\end{rlist}\mbox{}\\
\begin{Askhshs}[C]
\bmath{Για τη διαφορική εξίσωση
\[ (1-x^2)y''-6xy'-4y=0 \]
να βρεθούν οι δυναμοσειρές λύσεις γύρω από το σημείο $ x_0=0 $.}
\end{Askhshs}\mbox{}\\\\
\lysh
Οι συντελεστές της εξίσωσης είναι οι $ a_2(x)=1-x^2 x\in \mathbb{R},\ a_1(x)=-6x,\ x\in\mathbb{R} $ και $ a_0(x)=-4,\ x\in\mathbb{R} $. Ισχύει ότι $ a_2(0)=1\neq 0 $ και επιπλέον
\[ \frac{a_1(x)}{a_2(x)}=-\frac{6x}{1-x^2}=-6\sum_{n=0}^{\infty}{x^{2n+1}}\ \ \textrm{και}\ \ \frac{a_0(x)}{a_2(x)}=-\frac{4}{1-x^2}=-4\sum_{n=0}^{\infty}{x^{2n}}\ \ ,\ \textrm{για κάθε }|x|<1 \]
Καθώς οι συναρτήσεις αυτές είναι αναλυτικές στο $ x_0=0 $ τότε αυτό είναι ένα ομαλό σημείο της εξίσωσης. Με ακτίνα σύγκλισης $ R=\min\{R_1,R_2\}=1 $, όπου $ R_1,R_2 $ είναι οι ακτίνες σύγκλισης των παραπάνω σειρών, οι λύσεις της εξίσωσης γράφονται στη μορφή \[ y(x)=\sum_{n=0}^{\infty}{c_nx^n}\ \ ,\ \ |x|<1 \]
Αντικαθιστούμε τη συνάρτηση αυτή στην εξίσωση
\begin{gather*}
(1-x^2)\sum_{n=2}^{\infty}{n(n-1)c_nx^{n-2}}-6x\sum_{n=1}^{\infty}{nc_nx^{n-1}}-4\sum_{n=0}^{\infty}{c_nx^n}=0\Rightarrow\\
\sum_{n=2}^{\infty}{n(n-1)c_nx^{n-2}}-\sum_{n=2}^{\infty}{n(n-1)c_nx^n}-6\sum_{n=1}^{\infty}{nc_nx^n}-4\sum_{n=0}^{\infty}{c_nx^n}=0\Rightarrow\\
\sum_{n=0}^{\infty}{(n+2)(n+1)c_{n+2}x^n}-\sum_{n=0}^{\infty}{n(n-1)c_nx^n}-6\sum_{n=0}^{\infty}{nc_nx^n}-4\sum_{n=0}^{\infty}{c_nx^n}=0\Rightarrow\\
\sum_{n=0}^{\infty}{[(n+2)(n+1)c_{n+2}-(n+1)(n+4)c_n]x^n}=0
\end{gather*}
Σύμφωνα με την τελευταία ισότητα θα πρέπει να ισχύει
\[ (n+2)(n+1)c_{n+2}-(n+1)(n+4)c_n=0\Rightarrow c_{n+2}=\frac{n+4}{n+2}c_n\ \ ,\ \ n=0,1,2,\ldots \]
Οι δείκτες των συντελεστών αυτών έχουν διαφορά $ 2 $ μονάδες. Θα τους διαχωρίσουμε έτσι σ' αυτούς με άρτιο και περιττό δείκτη. Για $ n\to 2n-2 $ έχουμε τον αναδρομικό τύπο
\[ c_{2n}=\frac{2n+2}{2n}c_{2n-2} \]
από τον οποίο παίρνουμε \[ c_{2n}=(n+1)c_0\ \ ,\ \ n=1,2,\ldots \]
Ομοίως για τους συντελεστές με περιττό δείκτη ο αναδρομικός τύπος \[ c_{2n+1}=\frac{2n+3}{2n+1}c_{2n-1} \] μας οδηγεί στη σχέση \[ c_{2n+1}=\frac{2n+3}{3}c_1\ \ ,\ \ n=1,2,\ldots \]
Σύμφωνα με τα παραπάνω λύση θα γραφτεί ως γραμμικός συνδυασμός στη μορφή
\[ y(x)=\sum_{n=0}^{\infty}{c_{2n}x^{2n}}+\sum_{n=0}^{\infty}{c_{2n+1}x^{2n+1}}=\sum_{n=0}^{\infty}{(n+1)c_0x^{2n}}+\sum_{n=0}^{\infty}{\frac{2n+3}{3}c_1x^{2n+1}}\ \ ,\ \ |x|<1 \]
\begin{Askhshs}[C]
\bmath{Για τη διαφορική εξίσωση \[ (1+x^2)y''+3xy'+y=0 \]
να βρεθούν οι δυναμοσειρές λύσεις γύρω από το σημείο $ x_0=0 $.}
\end{Askhshs}\mbox{}\\\\
\lysh
Οι συντελεστές της εξίσωσης είναι $ a_2(x)=1,\ x\in\mathbb{R}, a_1(x)=3x,\ x\in\mathbb{R} $ και $ a_0(x)=1,\ x\in\mathbb{R} $. Παρατηρούμε ότι οι συναρτήσεις
\[\frac{a_1(x)}{a_2(x)}=\frac{3x}{1+x^2}\ \ \textrm{και}\ \ \frac{a_0(x)}{a_2(x)}=\frac{1}{1+x^2}\] είναι αναλυτικές στο σημείο $ x_0=0 $ οπότε το σημείο αυτό είναι ομαλό σημείο της εξίσωσης. Η λύση της μπορεί να πάρει τη μορφή δυναμοσειράς με ακτίνα σύγκλισης $ R=\infty $ συγκεκριμένα \[ y(x)=\sum_{n=0}^{\infty}{c_nx^n}\ \ ,\ \ x\in\mathbb{R} \]
Αντικαθιστούμε τη συνάρτηση αυτή καθώς και τις παραγώγους της στην εξίσωση και θα πάρουμε
\begin{gather*}
(1+x^2)\sum_{n=2}^{\infty}{n(n-1)c_nx^{n-2}}+3x\sum_{n=1}^{\infty}{nc_nx^{n-1}}+\sum_{n=0}^{\infty}{c_nx^n}=0\Rightarrow\\
\sum_{n=2}^{\infty}{n(n-1)c_nx^{n-2}}+\sum_{n=2}^{\infty}{n(n-1)c_nx^n}+3\sum_{n=1}^{\infty}{nc_nx^n}+\sum_{n=0}^{\infty}{c_nx^n}=0\Rightarrow\\
\sum_{n=0}^{\infty}{(n+2)(n+1)c_{n+2}x^{n}}+\sum_{n=0}^{\infty}{n(n-1)c_nx^n}+3\sum_{n=0}^{\infty}{nc_nx^n}+\sum_{n=0}^{\infty}{c_nx^n}=0\Rightarrow\\
\sum_{n=0}^{\infty}{[(n+2)(n+1)c_{n+2}+(n+1)^2c_n]x^n}=0
\end{gather*}
Άρα θα πρέπει να ισχύει
\[ (n+2)(n+1)c_{n+2}+(n+1)^2c_n=0\Rightarrow c_{n+2}=-\frac{n+1}{n+2}c_n\ \ ,\ \ n=0,1,2,\ldots \]
Θέτοντας όπου $ n\to 2n-2 $ οι συντελεστές με άρτιους δείκτες ικανοποιούν τη σχέση
\[ c_{2n}=-\frac{2n-1}{2n}c_{2n-2}\ \ ,\ \ n=1,2,\ldots \] από την οποία προκύπτει ο τύπος
\[ c_{2n}=\frac{(-1)^n[1\cdot 3\cdot 5\cdot\ldots\cdot(2n-1)]}{2\cdot 4\cdot 6\cdot\ldots\cdot 2n}c_0=\frac{(-1)^n(2n+1)!!}{(2n)!!}c_0\ \ ,\ \ n=1,2,\ldots \]
Ομοίως για τους συντελεστές με περιττό δείκτη θέτουμε $ n\to 2n-1 $ και παίρνουμε αρχικά τη σχέση
\[ c_{2n+1}=-\frac{2n}{2n+1}c_{2n-1}\ \ ,\ \ n=1,2,\ldots \]
από την οποία προκύπτει
\[ c_{2n+1}=\frac{(-1)^n(2n)!!}{(2n+1)!!}c_1\ \ ,\ \ n=1,2,\ldots \]
Η λύση θα γραφτεί ως γραμμικός συνδυασμός άρτιων και περιττών δυνάμεων του $ x $ στη μορφή
\begin{align*}
y(x)&=\sum_{n=0}^{\infty}{c_{2n}x^{2n}}+\sum_{n=0}^{\infty}{c_{2n+1}x^{2n+1}}=\\
&=\sum_{n=0}^{\infty}{\frac{(-1)^n(2n+1)!!}{(2n)!!}c_0x^{2n}}+\sum_{n=0}^{\infty}{\frac{(-1)^n(2n)!!}{(2n+1)!!}c_1x^{2n+1}}=\\
&=c_0\sum_{n=1}^{\infty}{\frac{(-1)^n(2n+1)!!}{(2n)!!}x^{2n}}+c_1\sum_{n=1}^{\infty}{\frac{(-1)^n(2n)!!}{(2n+1)!!}x^{2n+1}}+c_0+c_1x\ \ ,\ \ |x|<1
\end{align*}
με $ c_0,c_1 $ αυθαίρετες σταθερές.\epask
\begin{Askhshs}[C]
\bmath{Να επιλυθεί το πρόβλημα αρχικών τιμών
\[ x(2-x)y''-6(x-1)y'-4y=0\ \ ,\ \ y(1)=1,\ y'(1)=3 \]}
\end{Askhshs}\mbox{}\\\\
\lysh
Το $ x_0=1 $ είναι ομαλό σημείο της διαφορικής μας εξίσωσης και οι λύσεις της θα γραφτούν με τη μορφή δυναμοσειράς ως
\[ y(x)=\sum_{n=0}^{\infty}{c_n(x-1)^n}\ \ ,\ \ |x-1|<R \]
όπου $ 0<R\leq\infty $ η ακτίνα σύγκλισης. Έτσι έχουμε
\begin{gather*}
x(2-x)y''-6(x-1)y'-4y=0\Rightarrow\\
\left(2x-x^2\right)\sum_{n=0}^{\infty}{n(n-1)c_nx^{n-2}}-6(x-1)\sum_{n=0}^{\infty}{c_nx^{n-1}}-4\sum_{n=0}^{\infty}{c_nx^n}=0\Rightarrow\\
\sum_{n=0}^{\infty}{(n+2)(n+1)c_{n+2}(x-1)^n}-\sum_{n=0}^{\infty}{n(n-1)c_{n}(x-1)^n}-\sum_{n=0}^{\infty}{6nc_n(x-1)^n}-\sum_{n=0}^{\infty}{4c_n(x-1)^n}=0\Rightarrow\\
\sum_{n=0}^{\infty}{\left[(n+2)(n+1)c_{n+2}-(n+1)(n+4)c_n\right]x^n}=0
\end{gather*}
Από την τελευταία ισότητα προκύπτει η επόμενη σχέση μεταξύ των συντελεστών
\[ (n+2)(n+1)c_{n+2}-(n+1)(n+4)c_n=0\Rightarrow c_{n+2}=\frac{n+4}{n+2}c_n\ \ ,\ \ n=0,1,2,\ldots \]
όπου παρατηρούμε ότι οι δείκτες τους διαφέρουν κατά δύο μονάδες. Έτσι για συντελεστές με άρτιους δείκτες η σχέση αυτή μας δίνει
\[ c_{2n}=\frac{2n+2}{2n}c_{2n-2}\Rightarrow c_{2n}=(n+1)c_0\ \ ,\ \ n=1,2,\ldots \]
ενώ για τους συντελεστές με περιττούς δείκτες παίρνουμε ομοίως τη σχέση
\[ c_{2n+1}\frac{2n+3}{2n+1}c_{2n-1}\Rightarrow c_{2n+1}=\frac{2n+3}{3}c_1\ \ ,\ \ n=1,2,\ldots \]
Σύμφωνα με τα παραπάνω, η λύση θα γραφτεί ως γραμμικός συνδυασμός δύο συναρτήσεων άρτιων και περιττών δυνάμεων της παράστασης $ x-1 $
\begin{align*}
y(x)&=\sum_{n=0}^{\infty}{c_{2n}(x-1)^{2n}}+\sum_{n=0}^{\infty}{c_{2n+1}(x-1)^{2n+1}}=\\
&=c_0\sum_{n=0}^{\infty}{(n+1)(x-1)^{2n}}+c_1\sum_{n=0}^{\infty}{\frac{2n+3}{3}(x-1)^{2n+1}}=\\
&=c_0+c_1(x-1)+c_0\sum_{n=1}^{\infty}{(n+1)(x-1)^{2n}}+c_1\sum_{n=1}^{\infty}{\frac{2n+3}{3}(x-1)^{2n+1}}
\end{align*}
Στο σημείο αυτό, με τη βοήθεια των αρχικών συνθηκών θα υπολογίσουμε τις τιμές των συντελεστών $ c_0,c_1 $. Ισχύει λοιπόν ότι
\[ y(1)=1\Rightarrow c_0=1\ \ \textrm{και}\ \ y'(1)=3\Rightarrow c_1=3 \]
επομένως παίρνουμε τη λύση του προβλήματος αρχικών τιμών που δίνεται από τον τύπο
\[ y(x)=-2+3x+\sum_{n=1}^{\infty}{(n+1)(x-1)^{2n}}+\sum_{n=1}^{\infty}{(2n+3)(x-1)^{2n+1}}\ \ ,\ \ \textrm{για }|x-1|<R \]
\begin{Askhshs}[C]
\bmath{Να επιλυθεί το πρόβλημα αρχικών τιμών
\[ \left(x^2-2x\right)y''(x)+5(x-1)y'(x)+3y(x)=0\ \ ,\ \ y(1)=2\ ,\ y'(1)=1 \]}
\end{Askhshs}\mbox{}\\\\
\lysh
Οι συντελεστές της εξίσωσης είναι οι $ a_2(x)=x^2-2x,\ x\in\mathbb{R}, a_1(x)=5(x-1),\ x\in\mathbb{R} $ και $ a_0(x)=3,\ x\in\mathbb{R} $ σύμφωνα με τους οποίους προκύπτει ότι οι συναρτήσεις
\[\frac{a_1(x)}{a_2(x)}=\frac{5(x-1)}{x^2-2x}=5\sum_{n=0}^{\infty}{(x-1)^{2n+1}}\ \ \textrm{και}\ \ \frac{a_0(x)}{a_2(x)}=\frac{3}{x^2-2x}=3\sum_{n=0}^{\infty}{(x-1)^{2n}}\] είναι αναλυτικές στο σημείο $ x_0=1 $ οπότε το σημείο αυτό είναι ομαλό σημείο της εξίσωσης. Αν $ R_1,R_2 $ είναι οι ακτίνες σύγκλισης των παραπάνω δυναμοσειρών τότε για κάθε $ x\in\mathbb{R} $ με $ |x-1|<R=\min\{R_1,R_2\}=1 $ η γενική λύση της εξίσωσης γράφεται ως δυναμοσειρά γύρω από το σημείο $ x=1 $ ως 
\[ y(x)=\sum_{n=0}^{\infty}{c_n(x-1)^n} \]
Με αντικατάσταση στην εξίσωση θα καταλήξουμε σε μια σχέση των συντελεστών $ c_n $. Έχουμε αναλυτικά ότι
\begin{gather*}
\left(x^2-2x\right)y''(x)+5(x-1)y'(x)+3y(x)=0\Rightarrow\\
\left(x^2-2x\right)\sum_{n=2}^{\infty}{n(n-1)c_n(x-1)^{n-2}}+5(x-1)\sum_{n=1}^{\infty}{nc_n(x-1)^{n-1}}+3\sum_{n=0}^{\infty}{c_n(x-1)^n}=0\Rightarrow\\
\left[(x-1)^2-1\right]\sum_{n=2}^{\infty}{n(n-1)c_n(x-1)^{n-2}}+5\sum_{n=1}^{\infty}{nc_n(x-1)^{n}}+3\sum_{n=0}^{\infty}{c_n(x-1)^n}=0\Rightarrow\\
\sum_{n=2}^{\infty}{n(n-1)c_n(x-1)^{n}}-\sum_{n=2}^{\infty}{n(n-1)c_n(x-1)^{n-2}}+5\sum_{n=1}^{\infty}{nc_n(x-1)^{n}}+3\sum_{n=0}^{\infty}{c_n(x-1)^n}=0\Rightarrow\\
\sum_{n=1}^{\infty}{n(n-1)c_n(x-1)^{n}}-\sum_{n=0}^{\infty}{(n+2)(n+1)c_{n+2}(x-1)^{n}}+5\sum_{n=1}^{\infty}{nc_n(x-1)^{n}}+3\sum_{n=0}^{\infty}{c_n(x-1)^n}=0\Rightarrow\\
2c_2+3c_0+\sum_{n=0}^{\infty}{\left[c_n(n+1)(n+3)-(n+2)(n+1)c_{n+2}\right](x-1)^n}=0
\end{gather*}
Σύμφωνα λοιπόν με την τελευταία εξίσωση απαιτούμε να ισχύει
\[ c_2=\frac{3}{2}c_0 \]
και συγχρόνως
\[ c_n(n+1)(n+3)-c_{n+2}(n+2)(n+1)=0\Rightarrow c_{n+2}=\frac{n+3}{n+2}c_{n}\ \ ,\ \ n=1,2,\ldots \]
Και εδώ θα διαχωρίσουμε τους συντελεστές σ' αυτούς με άρτιο και περιττό δείκτη όπου παίρνουμε για κάθε περίπτωση μια αντίστοιχη σχέση. Έχουμε λοιπόν
\[ c_{2n}=\frac{3\cdot 5\cdot \ldots\cdot(2n+1)}{2^n\cdot n!}c_0=\frac{(2n+1)!!}{2^n\cdot n!}c_0\ ,\ n=1,2,\ldots \]
για τους άρτιους, ενώ για τους περιττούς δείκτες
\[ c_{2n+1}=\frac{4\cdot 6\cdot\ldots\cdot (2n+2)}{3\cdot 5\cdot\ldots\cdot (2n+1)}c_1=\frac{2^{n}(n+1)!}{(2n+1)!!}c_1\ ,\ n=1,2,\ldots \]
Η γενική λύση της εξίσωσης θα γραφτεί ως γραμμικός συνδυασμός σειρών με άρτιες και περιττές δυνάμεις της παράστασης $ x-1 $:
\begin{align*}
y(x)&=\sum_{n=0}^{\infty}{c_{2n}(x-1)^{2n}}+\sum_{n=0}^{\infty}{c_{2n+1}(x-1)^{2n+1}}=\\
&=c_0+c_1(x-1)+c_0\sum_{n=0}^{\infty}{\frac{(2n+1)!!}{2^n\cdot n!}(x-1)^{2n}}+c_1\sum_{n=0}^{\infty}{\frac{2^{n}(n+1)!}{(2n+1)!!}(x-1)^{2n+1}}
\end{align*}
Με τη χρήση των αρχικών συνθηκών του προβλήματος υπολογίζουμε τις τιμές των σταθερών $ c_0=1 $ και $ c_1=-1 $ και παίρνουμε τη λύση του προβλήματος αρχικών τιμών η οποία δίνεται από τον τύπο
\[ y(x)=2-x+\sum_{n=1}^{\infty}{\frac{(2n+1)!!}{2^n\cdot n!}(x-1)^{2n}}-\sum_{n=1}^{\infty}{\frac{2^{n}(n+1)!}{(2n+1)!!}(x-1)^{2n+1}}\ \ ,\ \ |x-1|<1 \]
\begin{Askhshs}[C]
Έστω δεύτερης τάξης ομογενής γραμμική διαφορική εξίσωση 
\[ \left(1-x^2\right)y''-xy'+p^2y=0\ \ ,\ \ -1<x<1 \]
όπου $ p $ είναι μια μη αρνητική σταθερά. Να βρείτε ένα βασικό σύνολο λύσεων αυτής.
\end{Askhshs}\mbox{}\\\\
\lysh
Το $ x_0=1 $ είναι ένα κανονικό ανώμαλο σημείο της διαφορικής εξίσωσης. Θεωρώ $ \mathscr{A}_1 $ και $ \mathscr{A}_0 $ συναρτήσεις που είναι αναλυτικές στο $ x_0=1 $ τέτοιες ώστε
\[ (x-x_0)a_1(x)=a_2(x)\mathscr{A}_1(x)\ \ \textrm{και}\ \ (x-x_0)^2a_0=a_2(x)\mathscr{A}_0(x) \]
Αυτές γράφονται
\[ \mathscr{A}_1(x)=\frac{-x(x-1)}{1-x^2}=\frac{x}{x+1}\ \ \textrm{και}\ \ \mathscr{A}_0(x)=\frac{(x-1)^2p^2}{1-x^2}=p^2\frac{1-x}{1+x} \]
απ' όπου βλέπουμε ότι πράγματι είναι αναλυτικές στο $ x=1 $.  Άρα καθεμιά απ' αυτές παίρνει τη μορφή δυναμοσειράς γύρω από το σημείο $ x_0=1 $:
\[ \mathscr{A}_0(x)=\sum_{n=0}^{\infty}{p_n(x-1)^n}\ \ ,\ \ |x-1|<R_1 \ \ \textrm{όπου}\ \ p_n=\frac{n}{n+1}\ \textrm{και}\ p_0=\frac{1}{2} \]
και αντίστοιχα
\[  \mathscr{A}_1(x)=\sum_{n=0}^{\infty}{q_n(x-1)^n}\ \ ,\ \ |x-1|<R_2 \ \ \textrm{όπου}\ \ q_n=p^2\frac{1-x}{x+1}\ \textrm{και}\ q_0=0 \]
Ας είναι $ P(\lambda)=\lambda^2+(p_0-1)\lambda+q_0=0 $ η ενδεικτική εξίσωση της ........ όπου σύμφωνα με τα παραπάνω γράφεται
\[ P(\lambda)=\lambda^2-\frac{1}{2}\lambda=0 \]
οπότε έχει ρίζες τις $ \lambda_1=0 $ και $ \lambda_2=\frac{1}{2} $. Η διαφορική εξίσωση έχει μια λύση $ y_1 $ της μορφής
\[ y_1(x)=|x-1|^{\lambda_1}\sum_{n=0}^{\infty}{c_n(x-1)^n}=\sum_{n=0}^{\infty}{c_n(x-1)^n}\ \ ,\ \ 0<|x-1|<R \]
με $ c_0=1 $. Ελέγχω ότι η διαφορά $ \lambda_2-\lambda_1=\frac{1}{2} $ δεν είναι ακέραιος οπότε υπάρχει μια άλλη λύση $ y_2 $ της (Ε) τέτοια ώστε οι $ y_1,y_2 $ να είναι γραμμικά ανεξάρτητες και θα έχει τη μορφή:
\[ y_2(x)=|x-1|^{\lambda_2}\sum_{n=0}^{\infty}{d_n(x-1)^n}=|x-1|^{\frac{1}{2}}\sum_{n=0}^{\infty}{d_n(x-1)^n}\ \ ,\ \ 0<|x-1|<R \]
με $ d_0=1 $. Θα προσδιορίσουμε στη συνέχεια τους συντελεστές $ c_n,d_n $ για $ n=1,2,\ldots $, αντικαθιστώντας τις λύσεις $ y_1 $ και $ y_2 $ στη διαφορική εξίσωση. Για κάθε $ 0<|x-1|<R $ με τη λύση $ y_1 $ παίρνουμε
\begin{gather*}
\left(1-x^2\right)y_1''(x)-xy_1'(x)+p^2y_1(x)=0\Rightarrow\\
\left(1-x^2\right)\sum_{n=2}^{\infty}{n(n-1)c_n(x-1)^{n-2}}-x\sum_{n=1}^{\infty}{nc_n(x-1)^{n-1}}+p^2\sum_{n=0}^{\infty}{c_n(x-1)^n}=0\Rightarrow\\
\sum_{n=2}^{\infty}{n(n-1)c_n(x-1)^{n-2}}-\sum_{n=2}^{\infty}{n(n-1)c_n(x-1)^{n}}-\sum_{n=1}^{\infty}{nc_n(x-1)^n}+p^2\sum_{n=0}^{\infty}{c_n(x-1)^n}=0\Rightarrow\\
\sum_{n=0}^{\infty}{(n+2)(n+1)c_{n+2}(x-1)^{n}}-\sum_{n=0}^{\infty}{n(n-1)c_n(x-1)^n}-\sum_{n=0}^{\infty}{nc_n(x-1)^n}+p^2\sum_{n=0}^{\infty}{c_n(x-1)^n}=0\\
\sum_{n=0}^{\infty}{[(n+2)(n+1)c_{n+2}-\left(n^2-p^2\right)c_n](x-1)^n}=0
\end{gather*}
Οπότε θα πρέπει να ισχύει
\[ (n+2)(n+1)c_{n+2}-\left(n^2-p^2\right)c_n\Rightarrow c_{n+2}=\frac{n^2-p^2}{(n+2)(n+1)}c_n\ \ ,\ \ n=0,1,2,\ldots \]
Εφόσον οι δείκτες των συντελεστών έχουν διαφορά $ 2 $ μονάδων θα τους διαχωρίσουμε όπως έχουμε δει σε άρτιους και περιττούς. Για τους μεν άρτιους θέτοντας $ n\to 2n-2 $ προκύπτει η σχέση
\[ c_{2n}=\frac{(2n-2)^2-p^2}{(2n-1)2n}c_{2n-2}\ \ ,\ \ n=1,2,\ldots \]
από την οποία παίρνουμε
\[ c_{2n}=\frac{(-p^2)\left(4-p^2\right)\left(16-p^2\right)\ldots\left[(2n-2)^2-p^2\right]}{[1\cdot 3\cdot 5\cdot \ldots\cdot(2n-1)]2^nn!}c_0=\frac{\prod_{k=1}^{n}{\left[(2k-2)^2-p^2\right]}}{(2n)!} \]
Αντίστοιχα για τους περιττούς η αναδρομική σχέση, θέτοντας $ n\to 2n-1 $ θα είναι
\begin{equation}\label{c8:syntart}
c_{2n+1}=\frac{(2n-1)^2-p^2}{2n(2n+1)}c_{2n-1}\ \ ,\ \ n=1,2,\ldots
\end{equation}
που μας δίνει
\begin{equation}\label{c8:syntper}
c_{2n+1}=\frac{\left(1-p^2\right)\left(9-p^2\right)\left[(2n-1)^2-p^2\right]}{2^nn![1\cdot3\cdot5\cdot(2n+1)]}c_1=\frac{\prod_{k=1}^{n}{\left[(2k-1)^2-p^2\right]}}{(2n+1)!}c_1
\end{equation}
Σύμφωνα με όλα τα παραπάνω, η λύση $ y_1 $ θα γραφτεί ως γραμμικός συνδυασμός δυναμοσειρών άρτιων και περιττών δυνάμεων της $ x-1 $:
\[ y_1(x)=\sum_{n=0}^{\infty}{c_{2n}(x-1)^{2n}}+\sum_{n=0}^{\infty}{c_{2n+1}(x-1)^{2n+1}} \]
όπου $ c_{2n},c_{2n+1} $ είναι οι συντελεστές που δίνονται από τους τύπους \eqref{c8:syntart} και \eqref{c8:syntper} αντίστοιχα. Στη συνέχεια θα εργαστούμε αναλόγως και για τη λύση $ y_2 $. Οι δύο πρώτοι παράγωγοι αυτής θα είναι
\begin{align*}
y_2'(x)&=\frac{1}{2}(x-1)^{-\frac{1}{2}}\sum_{n=0}^{\infty}{d_n(x-1)^n}+(x-1)^{\frac{1}{2}}\sum_{n=1}^{\infty}{nd_n(x-1)^{n-1}}=\\
&=\frac{1}{2}(x-1)^{-\frac{1}{2}}\sum_{n=0}^{\infty}{d_n(x-1)^n}+(x-1)^{-\frac{1}{2}}{nd_n(x-1)^n}=\\
&=(x-1)^{-\frac{1}{2}}\sum_{n=0}^{\infty}{\left(n+\frac{1}{2}\right)d_n(x-1)^n}
\end{align*}
και
\begin{align*}
y''_2(x)&=-\frac{1}{2}(x-1)^{-\frac{3}{2}}\sum_{n=0}^{\infty}{\left(n+\frac{1}{2}\right)d_n(x-1)^n}+(x-1)^{-\frac{1}{2}}\sum_{n=1}^{\infty}n\left(n+\frac{1}{2}\right)d_n(x-1)^{n-1}=\\
&=(x-1)^{-\frac{3}{2}}\sum_{n=0}^{\infty}{\left(n^2-\frac{1}{4}\right)d_n(x-1)^n}
\end{align*}
Όπως και προηγουμένως λοιπόν, με αντικατάσταση στην εξίσωση θα προκύψει
\begin{gather*}
\left(1-x^2\right)y_2''-xy_2'+p^2y_2=0\Rightarrow\\
\left(1-x^2\right)(x-1)^{-\frac{3}{2}}\sum_{n=0}^{\infty}{\left(n^2-\frac{1}{4}\right)d_n(x-1)^n}-x(x-1)^{-\frac{1}{2}}\sum_{n=0}^{\infty}{\left(n+\frac{1}{2}\right)d_n(x-1)^n}+p^2(x-1)^{\frac{1}{2}}\sum_{n=0}^{\infty}d_n(x-1)^n=0\Rightarrow\\
-(x+1)(x-1)^{-\frac{1}{2}}\sum_{n=0}^{\infty}{\left(n^2-\frac{1}{4}\right)d_n(x-1)^{n}}-x(x-1)^{-\frac{1}{2}}\sum_{n=0}^{\infty}{\left(n+\frac{1}{2}\right)d_n(x-1)^n}+p^2(x-1)^{-\frac{1}{2}}\sum_{n=0}^{\infty}{d_n(x-1)^{n+1}}=0\Rightarrow\\
-(x+1)(x-1)^{-\frac{1}{2}}\sum_{n=0}^{\infty}{\left(n^2-\frac{1}{4}\right)d_n(x-1)^{n}}-x(x-1)^{-\frac{1}{2}}\sum_{n=0}^{\infty}{\left(n+\frac{1}{2}\right)d_n(x-1)^n}+p^2(x-1)^{-\frac{1}{2}}\sum_{n=1}^{\infty}{d_{n-1}(x-1)^{n}}=0\Rightarrow\\
-
\end{gather*}
\begin{Askhshs}[C]
Να επιλυθεί η δεύτερης τάξης ομογενής γραμμική διαφορική εξίσωση
\begin{equation}\label{c9:eq}\tag{E_0}
\left(1-x^2\right)y''-2xy'+p(p+1)y=0
\end{equation}
στο διάστημα $ (-1,1) $, όπου $ p $ είναι μια πραγματική σταθερά.
\end{Askhshs}\mbox{}\\
\lysh
Οι συντελεστές της εξίσωσης είναι $ a_2(x)=1-x^2,\ a_1(x)=-2x $ και $ a_0(x)=p(p+1) $ και παίρνουμε ότι οι συναρτήσεις
\[  \frac{a_1(x)}{a_2(x)}=\frac{-2x}{1-x^2}=-2\sum_{n=0}^{\infty}{x^{2n+1}}\ \ ,\ \ |x|<R_1=1 \]
και
\[ \frac{a_0(x)}{a_2(x)}=\frac{p(p+1)}{1-x^2}=p(p+1)\sum_{n=0}^{\infty}{x^{2n}}\ \ ,\ \ |x|<R_2=1 \]
είναι αναλυτικές στο σημείο $ x_0 $ το οποίο κατά συνέπεια είναι ομαλό σημείο της εξίσωσης. Η γενική λύση, με μορφή δυναμοσειράς θα είναι
\[ y(x)=\sum_{n=0}^{\infty}{c_nx^n}\ \ ,\ \ |x|<R=\min\{R_1,R_2\}=1 \]
την οποία και θα αντικαταστήσουμε στην ομογενή εξίσωση
\begin{gather*}
\left(1-x^2\right)y''-2xy'+p(p+1)y=0\Rightarrow\\
\sum_{n=2}^{\infty}{n(n-1)c_nx^{n-2}}-\sum_{n=2}^{\infty}{n(n-1)c_nx^n}-2x\sum_{n=1}^{\infty}{nc_nx^{n-1}}+p(p+1)\sum_{n=0}^{\infty}{c_nx^n}=0\Rightarrow\\
\sum_{n=0}^{\infty}{(n+2)(n+1)c_{n+2}x^n}-\sum_{n=0}^{\infty}{n(n-1)c_nx^n}-2\sum_{n=0}^{\infty}{nc_nx^{n}}+p(p+1)\sum_{n=0}^{\infty}{c_nx^n}=0\\
\sum_{n=0}^{\infty}{\left[(n+2)(n+1)c_{n+2}-\left(n^2+n-p(p+1)\right)c_n\right]x^n}=0
\end{gather*}
Από την τελευταία εξίσωση καταλήγουμε στην ακόλουθη σχέση μεταξύ των συντελεστών $ c_n $
\[ c_{n+2}=\frac{n^2+n-p(p+1)}{(n+2)(n+1)}c_n\ \ ,\ \ n=0,1,2,\ldots \]
Θέτοντας όπου $ n\to 2n-2 $, για τους συντελεστές με άρτιο δείκτη, η παραπάνω σχέση θα μας δώσει τον αναδρομικό τύπο 
\[ c_{2n}=\frac{(2n-2)(2n-1)-p(p+1)}{2n(2n-1)}c_{2n-2}\ \ ,\ \ n=1,2,\ldots \]
από τον οποίο, για τις διαδοχικές τιμές $ n=1,2,\ldots $ παίρνουμε την έκφρασή τους συναρτήσει του $ c_0 $
\[ c_{2n}=\frac{\prod_{k=1}^{n}{\left[(2k-2)(2k-1)-p(p+1)\right]}}{2^nn!(2n-1)!!}c_0=\frac{\prod_{k=1}^{n}{\left[(2k-2)(2k-1)-p(p+1)\right]}}{(2n)!}c_0\ \ ,\ \ n=1,2,\ldots \]
αντίστοιχα για τους συντελεστές με περιττό δείκτη θέτουμε $ n\to 2n-1 $ και παίρνουμε
\[ c_{2n+1}=\frac{\prod_{k=1}^{n}{\left[(2k-1)2k-p(p+1)\right]}}{(2n)!}c_1\ \ ,\ \ n=1,2,\ldots \]
Έτσι η ζητούμενη λύση θα δίνεται από τη δυναμοσειρά
\[ y(x)=\sum_{n=0}^{\infty}{c_nx^n}=c_0+c_1x+\sum_{n=1}^{\infty}{c_{2n}x^{2n}}+\sum_{n=0}^{\infty}{c_{2n+1}x^{2n+1}}\ \ ,\ \ |x|<1 \]
\begin{Askhshs}[C]
Με τη βοήθεια του μετασχηματισμού $ y=ze^{-x^2/4} $, να επιλυθεί η ομογενής γραμμική διαφορική εξίσωση
\begin{equation}\label{c10:eq}
y''+\left(1+\frac{x^2}{4}\right)y=0
\end{equation}
\textit{Υπόδειξη} : Για την ομογενή γραμμική διαφορική εξίσωση που θα προκύψει με το μετασχηματισμό $ y=ze^{-x^2/4} $, θα πρέπει να βρεθούν οι δυναμοσειρές λύσεις γύρω από το $ x_0=0 $.
\end{Askhshs}\mbox{}\\
\lysh
Χρησιμοποιώντας το μετασχηματισμό $ y=ze^{-x^2/4}\ ,\ x\in\mathbb{R} $ θα είναι \[ y'=\left(z'-\frac{x}{2}z\right)e^{-x^2/4} \text{ και } y''=\left[z''-xz'-\left(\frac{1}{2}-\frac{x^2}{4}\right)z\right]e^{-x^2/4} \] 
Ύστερα από αντικατάσταση η μετασχηματισμένη εξίσωση θα είναι
\[ z''-xz'+\frac{1}{2}z=0\ \ ,\ \ x\in\mathbb{R} \]
Το σημείο $ x=0 $ είναι ένα ομαλό σημείο της εξίσωσης οπότε για $ z(x)=\sum_{n=0}^{\infty}{c_nx^n} $ θα είναι 
\begin{align*}
z''(x)-xz'(x)+\frac{1}{2}z(x)&=\sum_{n=2}^{\infty}{n(n-1)c_{n}x^{n-2}}-x\sum_{n=1}^{\infty}{nc_nx^{n-1}}+\frac{1}{2}\sum_{n=0}^{\infty}{c_nx^n}=\\
&=\sum_{n=0}^{\infty}{(n+2)(n+1)c_{n+2}x^{n}}-x\sum_{n=0}^{\infty}{nc_nx^{n}}+\frac{1}{2}\sum_{n=0}^{\infty}{c_nx^n}=\\
&=\sum_{n=0}^{\infty}{\left[(n+2)(n+1)c_{n+2}-\left(n-\frac{1}{2}c_n\right)c_n\right]x^n}=0
\end{align*}
Επομένως θα πρέπει να ισχύει
\[ (n+2)(n+1)c_{n+2}-\left(n-\frac{1}{2}c_n\right)c_n=0\Rightarrow c_{n+2}=\frac{n-\frac{1}{2}}{(n+2)(n+1)}c_n\ ,\ n=0,1,2,\ldots \]
Οι δείκτες διαφέρουν κατά $ 2 $ μονάδες άρα θα διαχωριστούν ξανά σε άρτιους και περιττούς δείκτες και καταλήγουμε στους τύπους
\[ c_{2n}=\frac{2n-\frac{5}{2}}{2n(2n-1)}c_{2n-2}\ \text{ και }\ c_{2n+1}=\frac{2n-\frac{3}{2}}{(2n+1)2n}c_{2n-1}\ ,\ n=1,2,\ldots \]
Διαδοχικά για τις τιμές του $ n=0,1,2,\ldots $ οι παραπάνω συντελεστές εκφράζονται συναρτήσει των $ c_0,c_1 $ αντίστοιχα ως
\[ c_{2n}=\frac{\prod_{k=1}^{n}(4k-5)}{2^{n}(2n)!}c_{0}\ \text{ και }\ c_{2n+1}=\frac{\prod_{k=1}^{n}(4k-3)}{2^{n}(2n+1)!}c_{1}\ ,\ n=1,2,\ldots \]
Αρχικά, η δυναμοσειρά λύση της μετασχηματισμένης εξίσωσης θα δίνεται από τον τύπο 
\[ z(x)=\sum_{n=0}^{\infty}{c_{2n}x^{2n}}+\sum_{n=0}^{\infty}{c_{2n+1}x^{2n+1}} \]
Σύμφωνα τώρα με τον αρχικό μας μετασχηματισμό έχουμε $ z=ye^{x^2/4} $ και άρα η λύση της αρχικής εξίσωσης ως δυναμοσειρά θα γίνει
\begin{align*}
y(x)e^{x^2/4}&=\sum_{n=0}^{\infty}{c_{2n}x^{2n}}+\sum_{n=0}^{\infty}{c_{2n+1}x^{2n+1}}\\
y(x)&=\sum_{n=0}^{\infty}{\frac{\prod_{k=1}^{n}(4k-5)}{2^{n}(2n)!}c_{0}x^{2n}e^{-x^2/4}}+\sum_{n=0}^{\infty}{\frac{\prod_{k=1}^{n}(4k-3)}{2^{n}(2n+1)!}c_{1}x^{2n+1}e^{-x^2/4}}\ ,\ n=0,1,2,\ldots
\end{align*}
\begin{Askhshs}[C]
Να επιλυθεί γύρω από το σημείο $ x_0=0 $ η διαφορική εξίσωση 
\[ xy''+y'+xy=0 \]
\end{Askhshs}\mbox{}\\
\lysh
Το $ x_0 $ δεν είναι ομαλό σημείο αφού $ a_2(0)=0 $. Έχουμε λοιπόν τις συναρτήσεις
\[ A_1(x)=\frac{(x-0)a_1(x)}{a_2(x)}=\frac{x\cdot 1}{x}=1\ \ x\in\mathbb{R} \]
και 
\[ A_0(x)=\frac{x^2\cdot x}{x}=x^2\ \ x\in\mathbb{R} \]
οι οποίες είναι αναλυτικές στο $ x_0=0 $ οπότε αυτό είναι ένα κανονικό ανώμαλο σημείο. Είναι
\[ A_1(x)=1=\sum_{n=0}^{\infty}{p_n x^n}\ ,\ \text{για }|x|<R_1=\infty\ \text{ όπου }p_n=\ccases{1&,n=0\\0&,n\neq 0}\text{ και } \]
\[ A_0(x)=x^2=\sum_{n=0}^{\infty}{q_nx^n}\ ,\text{ για }|x|<R_2=\infty \]
\begin{Askhshs}[C]
Για την ομογενή διαφορική εξίσωση
\[ 2xy''+y'-2y=0 \]
να βρεθούν οι δυναμοσειρές λύσεις γύρω από το σημείο $ x_0 $.
\end{Askhshs}\mbox{}\\
\lysh
Το $ x_0=0 $ είναι ανώμαλο σημείο της διαφορικής εξίσωσης αφού $ a_2(0)=0 $. Οι συναρτήσεις
\[ A_1(x)=\frac{(x-0)\cdot 1}{2x}=\frac{1}{2}\ ,\ x\in\mathbb{R} \]
και 
\[ A_0(x)=\frac{(x_0)^2\cdot(-2)}{2x}=-x\ ,\ x\in\mathbb{R} \]
είναι αναλυτικές στο $ x_0=0 $. Οπότε η ενδεικτική εξίσωση θα είναι $ P(\lambda)=\lambda^2-\frac{1}{2}\lambda=0 $ με ρίζες $ \lambda_1=0 $ και $ \lambda_2=\frac{1}{2} $ άρα $ \lambda_1-\lambda_2=-\frac{1}{2}\notin\mathbb{Z} $. Έτσι, δύο ανεξάρτητες λύσεις της εξίσωσης θα είναι οι
\[ y_1(x)=|x|^{\frac{1}{2}}\sum_{n=0}^{\infty}{c_nx^n}\ \ \text{με }c_0=1 \]
και 
\[ y_2(x)=\sum_{n=0}^{\infty}{d_nx^n}\ \ \text{ με }d_0=1 \]
Θα προσδιορίσουμε τώρα τους συντελεστές $ c_n,d_n $. Έχουμε για τη συνάρτηση $ y_1 $ ότι
\begin{align*}
y_1(x)&=|x|^{\frac{1}{2}}\sum_{n=0}^{\infty}{c_nx^n}\\
y'_1(x)&=\frac{1}{2}|x|^{-\frac{1}{2}}\sum_{n=0}^{\infty}{c_nx^n}+|x^{\frac{1}{2}}|\sum_{n=1}^{\infty}{nc_nx^{n-1}}=\\
&=\frac{1}{2}x^{-\frac{1}{2}}\sum_{n=0}^{\infty}{c_nx^n}+x^{-\frac{1}{2}}\sum_{n=0}^{\infty}{nc_nx^n}=x^{-\frac{1}{2}}\sum_{n=0}^{\infty}{\left(n+\frac{1}{2}\right)c_nx^n}\\
y_1''(x)&=-\frac{1}{2}x^{-\frac{3}{2}}\sum_{n=0}^{\infty}{\left(n+\frac{1}{2}\right)c_nx^n}+x^{-\frac{1}{2}}\sum_{n=1}^{\infty}{n\left(n+\frac{1}{2}\right)c_nx^{n-1}}=\\
&=-\frac{1}{2}x^{-\frac{3}{2}}\sum_{n=0}^{\infty}{\left(n+\frac{1}{2}\right)c_nx^n}+x^{-\frac{3}{2}}\sum_{n=1}^{\infty}{n\left(n+\frac{1}{2}\right)c_nx^n}=\\&=x^{-\frac{3}{2}}\sum_{n=0}^{\infty}{\left(n+\frac{1}{2}\right)\left(n-\frac{1}{2}\right)c_nx^n}
\end{align*}
Αντικαθιστώντας στην εξίσωση παίρνουμε
\begin{gather*}
2xy''_1(x)+y'_1(x)-2y_1(x)=0\Rightarrow\\
2x^{-\frac{1}{2}}\sum_{n=0}^{\infty}{\left(n-\frac{1}{2}\right)\left(n+\frac{1}{2}\right)c_nx^n}+x^{-\frac{1}{2}}\sum_{n=0}^{\infty}{\left(n+\frac{1}{2}\right)c_nx^n}-2x^{\frac{1}{2}}\sum_{n=0}^{\infty}{c_nx^n}=0\Rightarrow\\
x^{-\frac{1}{2}}\sum_{n=0}^{\infty}{n(2n+1)c_nx^n}-2x^{-\frac{1}{2}}\sum_{n=0}^{\infty}{c_nx^{n+1}}=0\Rightarrow\\
x^{-\frac{1}{2}}\sum_{n=1}^{\infty}{n(2n+1){c_nx^n}}-2x^{-\frac{1}{2}}\sum_{n=1}^{\infty}{c_{n-1}x^n}=0\Rightarrow\\
x^{-\frac{1}{2}}\left[\sum_{n=1}^{\infty}\left[n(2n+1)c_n-2c_{n-1}\right]x^n\right]=0
\end{gather*}
Προκύπτει λοιπόν ότι
\[ c_0=1\ \text{και}\ n(2n+1)c_n-2c_{n-1}=0\Rightarrow c_n=\frac{2}{n(2n+1)}c_{n-1} \]
Ο τελευταίος τύπος για τις διαδοχικές τιμές του $ n $ μας δίνει τους συντελεστές $ c_n $
\[ c_n=\frac{2^n}{n!(2n+1)!!} \ \ ,\ \ n=0,1,2,\ldots\]
Άρα η λύση $ y_1(x) $ θα δίνεται από τον τύπο
\[ y_1(x)=|x|^{\frac{1}{2}}\sum_{n=0}^{\infty}{\frac{2^n}{n!(2n+1)!!}x^n} \]
Θα εργαστούμε στη συνέχεια αναλόγως για τη συνάρτηση $ y_2(x) $. Οι παράγωγοι της έως 2\tss{ης} τάξης θα είναι
\[
y_2'(x)=\sum_{n=1}^{\infty}{nd_{n}x^{n-1}}\ \ \text{και}\ \ y_2''(x)=\sum_{n=2}^{\infty}{n(n-1)d_nx^{n-2}}
\]
Αντικαθιστούμε στην εξίσωση και παίρνουμε
\begin{gather*}
2xy_2''(x)+y_2'(x)-2y_2(x)=0\Rightarrow\\
2x\sum_{n=2}^{\infty}{n(n-1)d_nx^{n-2}}+\sum_{n=1}^{\infty}{nd_nx^{n-1}}-2\sum_{n=0}^{\infty}{d_nx^n}=0\Rightarrow\\
2\sum_{n=2}^{\infty}{n(n-1)d_nx^{n-1}}+\sum_{n=0}^{\infty}{(n+1)d_{n+1}x^n}-2\sum_{n=0}^{\infty}{d_nx^n}=0\Rightarrow\\
2\sum_{n=1}^{\infty}{n(n+1)d_{n+1}x^{n}}+\sum_{n=0}^{\infty}{(n+1)d_{n+1}x^n}-2\sum_{n=0}^{\infty}{d_nx^n}=0\Rightarrow\\
2\sum_{n=0}^{\infty}{n(n+1)d_{n+1}x^{n}}+\sum_{n=0}^{\infty}{(n+1)d_{n+1}x^n}-2\sum_{n=0}^{\infty}{d_nx^n}=0\Rightarrow\\
\sum_{n=0}^{\infty}{\left[(n+1)(2n+1)d_{n+1}-2d_n\right]x^n}=0
\end{gather*}
Θα πρέπει λοιπόν να ισχύει
\[ d_0=1\ \text{ και }\ d_{n+1}=\frac{2}{(n+1)(2n+1)}d_n\Rightarrow d_n=\frac{2}{n(2n-1)}d_{n-1}\ ,\ n=1,2,\ldots \]
Οι συντελεστές $ d_n $ θα δίνονται από τη σχέση $ d_n=\frac{2^n}{n!(2n-1)!!}\ n=0,1,2,\ldots $ και άρα η λύση $ y_2(x) $ θα δίνεται από τον τύπο
\[ y_2(x)=\sum_{n=0}^{\infty}{\frac{2^n}{n!(2n-1)!!}x^n} \]
Οι λύσεις της αρχικής εξίσωσης θα είναι 
\[ y(x)=c_1y_1(x)+c_2y_2(x) \]
όπου $ c_1,c_2 $ αυθαίρετες σταθερές.\epask
\begin{Askhshs}[C]
Να επιλυθεί, γύρω από το σημείο $ x_0=0 $ η ομογενής γραμμική διαφορική εξίσωση
\[ xy''+(1-x^2)y'+4xy=0 \]
\end{Askhshs}\mbox{}\\
\lysh
Έχουμε τις συναρτήσεις $ a_2(x)=x,a_1(x)=1-x^2 $ και $ a_0(x)=4x $ για τις οποίες ισχύει $ a_2(0)=0 $ άρα το $ x_0=0 $ δεν είναι ομαλό σημείο. Στη συνέχεια ορίζουμε τις συναρτήσεις
\[ A_1(x)=\frac{x(1-x^2)}{x}=1-x^2\ ,\ x\in\mathbb{R}\text{ και }A_0(x)=\frac{x^2\cdot 4x}{x}=4x^2\ ,\ x\in\mathbb{R} \] οι οποίες είναι αναλυτικές στο $ x_0=0 $ οπότε το σημείο αυτό είναι ένα κανονικό ανώμαλο σημείο. Αυτές γράφονται ως δυναμοσειρές στη μορφή
\[ A_1(x)=1-x^2=\sum_{n=0}^{\infty}{p_nx^n}\ ,\ \forall x\in\mathbb{R},\ |x|<R_1 \]
όπου $ p_0=1,p_2=-1 $ και $ p_n=0,n\neq 0,2 $ και
\[ A_2(x)=4x^2=\sum_{n=0}^{\infty}{q_nx^n}\ \forall x\in\mathbb{R}\ ,\ |x|<R_2 \]
όπου $ q_2=4 $ και $ q_n=0,n\neq 2 $. Οι ακτίνες σύγκλισης των δυναμοσειρών είναι αντίστοιχα 
\[ R_1=\frac{1}{\limsup{\sqrt[n]{p_n}}}=+\infty\ \ \text{και}\ \ R_2=\frac{1}{\limsup{\sqrt[n]{q_n}}}=+\infty \]
άρα $ R=\min\{R_1,R_2\}=+\infty $. Θεωρούμε την εξίσωση $ \lambda^2+(1-1)\lambda+0=0\Rightarrow \lambda_1=0\ ,\ \lambda_2=0 $ επομένως οι λύσεις θα είναι
\[ y_1(x)=|x|\sum_{n=0}^{\infty}{c_nx^n}\ ,\ c_0=1\ \ \text{και} \]
\[ y_2(x)=y_1(x)\log|x|+|x|^0\sum_{n=0}^{\infty}{d_nx^n}=y_1(x)\log|x|+\sum_{n=0}^{\infty}{d_nx^n} \]
Για την $ y_1 $ έχουμε αντίστοιχα τις παραγώγους $ y_1'=\sum_{n=1}^{\infty}{nc_nx^{n-1}} $ και $ y_1''(x)=\sum_{n=2}^{\infty}{n(n-1)c_nx^{n-2}} $ τις οποίες αντικαθιστούμε στην αρχική εξίσωση και παίρνουμε
\begin{gather*}
x\sum_{n=2}^{\infty}{n(n-1)c_nx^{n-2}}+(1-x^2)\sum_{n=1}^{\infty}{nc_nx^{n-1}}+4x\sum_{n=0}^{\infty}{c_nx^n}=0\Rightarrow\\
\sum_{n=2}^{\infty}{n(n-1)c_nx^{n-1}}+\sum_{n=1}^{\infty}{nc_nx^{n-1}}-\sum_{n=1}^{\infty}{nc_nx^{n+1}}+4\sum_{n=0}^{\infty}{c_nx^{n+1}}=0\Rightarrow\\
\sum_{n=1}^{\infty}{n(n+1)c_{n+1}x^{n}}+\sum_{n=0}^{\infty}{(n+1)c_{n+1}x^{n}}-\sum_{n=2}^{\infty}{(n-1)c_{n-1}x^{n}}+4\sum_{n=1}^{\infty}{c_{n-1}x^{n}}=0\Rightarrow\\
\sum_{n=1}^{\infty}{n(n+1)c_{n+1}x^{n}}+\sum_{n=0}^{\infty}{(n+1)c_{n+1}x^{n}}-\sum_{n=1}^{\infty}{(n-1)c_{n-1}x^{n}}+4\sum_{n=1}^{\infty}{c_{n-1}x^{n}}=0\Rightarrow\\
c_1+\sum_{n=1}^{\infty}{[c_{n+1}(n+1)^2+c_{n-1}(n+3)]x^n}=0
\end{gather*}
οπότε παίρνουμε
\[ c_1=0\ \ \text{και}\ \ c_{n+1}(n+1)^2+c_{n-1}(n+3)=0\Rightarrow c_{n+1}=-\frac{n+3}{(n+1)^2}c_{n-1},\ \forall n\neq 1 \]
Οι δείκτες έχουν διαφορά δύο μονάδες άρα χωρίζονται σε άρτιους και περιττούς. Για τους μεν άρτιους έχουμε τον αναδρομικό τύπο
\[ c_{2n}=-\frac{2n+2}{(2n)^2}c_{2n-2} \]
από τον οποίο προκύπτει
\[ c_{2n}=\frac{(-1)^n(n+1)}{2^nn!}c_0 \]
ενώ για τους περιττούς προκύπτει ομοίως
\[ c_{2n+1}=-\frac{2n+3}{(2n+1)^2}c_{2n-1}\Rightarrow c_{2n+1}=\frac{(-1)^n}{(2n+1)!!}c_1=0 \]
Η λύση $ y_1 $ γράφεται τελικά στη μορφή
\[ y_1(x)=\sum_{n=0}^{\infty}{c_nx^n}=\sum_{n=0}^{\infty}{c_{2n}x^{2n}}+\sum_{n=0}^{\infty}{c_{2n+1}x^{2n+1}}=c_0+\sum_{n=1}^{\infty}{\frac{(-1)^n}{2^nn!}}c_0x^{2n} \]
Εργαζόμαστε ομοίως για τη λύση $ y_2 $ και έχουμε
\[ y_2'(x)=y_1'\log|x|+y_1\cdot\frac{1}{x}+\sum_{n=1}^{\infty}{nd_nx^{n-1}} \]
και
\begin{align*}
y_2''(x)&=y_1''\log|x|+y_1'\cdot\frac{1}{x}-y_1\cdot\frac{1}{x^2}+y_1'\cdot\frac{1}{x}+\sum_{n=2}^{\infty}{n(n-1)d_{n}x^{n-2}}=\\
&=y_1''\log|x|+\frac{2}{x}y_1'+\sum_{n=2}^{\infty}{n(n-1)d_nx^{n-2}}
\end{align*}
Τις αντικαθιστούμε και παίρνουμε
\begin{multline}
x\left[y_1''\log|x|+\frac{2}{x}y_1'+\sum_{n=2}^{\infty}{n(n-1)d_nx^{n-2}}\right]+\\+(1-x^2)\left[y_1'\log|x|+y_1\cdot\frac{1}{x}+\sum_{n=1}^{\infty}{nd_nx^{n-1}}\right]+4x\left[y_1\log|x|+\sum_{n=0}^{\infty}{d_nx^n}\right]=0\Rightarrow
\end{multline}
\vspace{-5mm}
\begin{multline}
y_1''x\log|x|+2y_1'+\sum_{n=2}^{\infty}{n(n-1)d_nx^{n-1}}+\\+y_1'\log|x|+y_1\cdot\frac{1}{x}+\sum_{n=1}^{\infty}{nd_nx^{n-1}}-y_1'x^2\log|x|-xy_1-\\-\sum_{n=1}^{\infty}{nd_nx^{n+1}}+4y_1x\log|x|+4\sum_{n=0}^{\infty}{d_nx^{n+1}}=0\Rightarrow
\end{multline}
\vspace{-7mm}
\begin{multline}
\log|x|\left[xy_1''+(1-x^2)y_1'+4xy_1\right]+2y_1'-xy_1+\sum_{n=1}^{\infty}{n(n-1)d_nx^{n}}+\\+\sum_{n=1}^{\infty}{nd_nx^{n-1}}+\sum_{n=1}^{\infty}{nd_nx^{n-1}}+4\sum_{n=0}^{\infty}{d_nx^{n+1}}=0\Rightarrow
\end{multline}
\vspace{-7mm}
\begin{gather*}
2y_1'-xy_1+\sum_{n=1}^{\infty}{n(n-1)d_nx^{n}}+\sum_{n=0}^{\infty}{(n+1)d_{n+1}x^{n}}+\sum_{n=2}^{\infty}{(n-1)d_{n-1}x^{n}}+4\sum_{n=1}^{\infty}{d_{n-1}x^{n}}=0\Rightarrow\\
2y_1'-xy_1+d_1+2d_2x+4d_0x+\sum_{n=1}^{\infty}{\left[n(n-1)d_{n}+(n+1)d_{n+1}+(n+3)d_{n-1}\right]x^n}=0\Rightarrow
\end{gather*}
\begin{Askhshs}[C]
Ας είναι $ p $ μια πραγματική σταθερά και έστω $ p $ δεν είναι ακέραιος. Να βρεθούν οι δυναμοσειρές λύσεις, γύρω από το σημείο $ x_0=0 $, της δεύτερης τάξης ομογενούς γραμμικής διαφορικής εξίσωσης
\[ x(1-x)y''+\left[p-\left(p+\frac{3}{2}\right)x\right]y'-\frac{p}{2}y=0\]
\end{Askhshs}\mbox{}\\
\lysh
Το $ x_0=0 $ είναι ανώμαλο σημείο της εξίσωσης. Θεωρούμε τις συναρτήσεις
\[ A_1(x)=\frac{(x-0)\left[p-\left(p+\frac{3}{2}\right)x\right]}{x(1-x)}=\frac{p-\left(p+\frac{3}{2}\right)x}{1-x} \]
και 
\[ A_2(x)=\frac{(x-0)^2\left(-\frac{p}{2}\right)}{x(1-x)}=-\frac{x\frac{p}{2}}{1-x} \]
οι οποίες είναι αναλυτικές στο $ x_0=0 $ άρα το σημείο αυτό είναι κανονικό ανώμαλο σημείο. Θεωρούμε την εξίσωση $ P(\lambda)=\lambda^2+(p-1)\lambda=0\Rightarrow \lambda_1=0 $ και $ \lambda_2=1-p\neq0 $. Εφόσον η παράσταση $ \lambda_1-\lambda_2 $ δεν είναι ακέραιος τότε οι δύο ανεξάρτητες λύσεις θα είναι 
\[ y_1(x)=x^{1-p}\sum_{n=0}^{\infty}{c_nx^n}\ \ ,\ \ c_0=1 \text{ και } y_2(x)=\sum_{n=0}^{\infty}{d_nx^n}\ \ ,\ \ d_0=1 \]
Για την $ y_1 $ έχουμε
\[ y_1'(x)=(1-p)x^{-p}\sum_{n=0}^{\infty}{c_nx^n}+x^{1-p}\sum_{n=1}^{\infty}{nc_nx^{n-1}}=x^{-p}\sum_{n=0}^{\infty}{(1-p+n)c_nx^n} \]
και
\[ y_1''(x)=-px^{-(p+1)}\sum_{n=0}^{\infty}{(1-p+n)c_nx^n}+x^{-p}\sum_{n=1}^{\infty}{n(1-p+n)c_nx^{n-1}}=x^{-(p+1)}\sum_{n=0}^{\infty}{(1-p+n)(n-p)c_nx^n} \]
Με αντικατάσταση στην εξίσωση θα προκύψει
\begin{multline*}
x(1-x)x^{-(p+1)}\sum_{n=0}^{\infty}{(1-p+n)(n-p)c_nx^n}+\\+\left[p-\left(p+\frac{3}{2}\right)x\right]x^{-p}\sum_{n=0}^{\infty}{(1-p+n)c_nx^n}-\frac{p}{2}x^{1-p}\sum_{n=0}^{\infty}{c_nx^n}=0\Rightarrow
\end{multline*}
\vspace{-7mm}
\begin{multline*}
x^{-(p+1)}\sum_{n=0}^{\infty}{(1-p+n)(n-p)c_nx^{n+1}}-x^{-(p+1)}\sum_{n=0}^{\infty}{(1-p+n)(n-p)c_nx^{n+2}}+\\+px^{-p}\sum_{n=0}^{\infty}{(1-p+n)c_nx^n}-\left(p+\frac{3}{2}\right)xx^{-p}\sum_{n=0}^{\infty}{(1-p+n)c_nx^n}-\frac{p}{2}x^{1-p}\sum_{n=0}^{\infty}{c_nx^n}=0\Rightarrow
\end{multline*}
\vspace{-7mm}
\begin{multline*}
x^{-p}\sum_{n=0}^{\infty}{(1-p+n)(n-p)c_nx^{n}}-x^{-p}\sum_{n=0}^{\infty}{(1-p+n)(n-p)c_nx^{n+1}}+\\+px^{-p}\sum_{n=0}^{\infty}{(1-p+n)c_nx^n}-\left(p+\frac{3}{2}\right)x^{-p}\sum_{n=0}^{\infty}{(1-p+n)c_nx^{n+1}}-\frac{p}{2}x^{1-p}\sum_{n=0}^{\infty}{c_nx^n}=0\Rightarrow
\end{multline*}
\vspace{-7mm}
\begin{gather*}
x^{-p}\sum_{n=0}^{\infty}{(1-p+n)nc_nx^{n}}-x^{-p}\sum_{n=0}^{\infty}{\left[(1-p+n)(n+\frac{3}{2})-\frac{p}{2}\right]c_nx^{n+1}}=0\Rightarrow\\
x^{-p}\sum_{n=0}^{\infty}{\left[(1-p+n)nc_n-(n-p)\left[n-\frac{5}{2}(1+p)\right]c_{n-1}\right]x^n}=0
\end{gather*}
Από την τελευταία σχέση προκύπτει ο αναδρομικός τύπος
\[ c_n=\frac{(n-p)\left[n-\frac{5}{2}(1+p)\right]}{n(1-p+n)}c_{n-1}\ ,\ n=1,2,\ldots \]
που μας δίνει
\[ c_n=\prod_{i=1}^{n}\frac{(i-p)\left[i-\frac{5}{2}(1+p)\right]}{i(1-p+i)}c_0 \]
Η λύση $ y_1 $ θα δίνεται από τον τύπο 
\[ y_1(x)=x^{1-p}\sum_{n=0}^{\infty}{\prod_{i=1}^{n}\frac{(i-p)\left[i-\frac{5}{2}(1+p)\right]}{i(1-p+i)}c_0}x^n \]
Για τη λύση $ y_2 $ θα έχουμε
\begin{gather*}
x(1-x)y_2''(x)+\left[p-\left(p+\frac{3}{2}\right)x\right]y_2'(x)-\frac{p}{2}y_2(x)=0\Rightarrow\\
x(1-x)\sum_{n=2}^{\infty}{n(n-1)d_nx^{n-2}}+\left[p-\left(p+\frac{3}{2}\right)x\right]\sum_{n=1}^{\infty}{nd_nx^{n-1}}-\frac{p}{2}\sum_{n=0}^{\infty}{d_nx^n}=0\Rightarrow\\
\sum_{n=2}^{\infty}{n(n-1)d_nx^{n-1}}-\sum_{n=2}^{\infty}{n(n-1)d_nx^{n}}+p\sum_{n=1}^{\infty}{nd_nx^{n-1}}-\left(p+\frac{3}{2}\right)\sum_{n=1}^{\infty}{nd_nx^{n}}-\frac{p}{2}\sum_{n=0}^{\infty}{d_nx^n}=0\Rightarrow\\
\sum_{n=0}^{\infty}{n(n+1)d_{n+1}x^{n}}-\sum_{n=0}^{\infty}{n(n-1)d_nx^{n}}+p\sum_{n=0}^{\infty}{(n+1)d_{n+1}x^{n}}-\left(p+\frac{3}{2}\right)\sum_{n=0}^{\infty}{nd_nx^{n}}-\frac{p}{2}\sum_{n=0}^{\infty}{d_nx^n}=0\Rightarrow\\
\sum_{n=0}^{\infty}{\left[(n+1)(n+p)d_{n+1}-\left[n(n-1)+\frac{3}{2}(p+1)\right]d_n\right]x^n}=0
\end{gather*}
Παίρνουμε λοιπόν τον τύπο
\[ d_{n+1}=\frac{n(n+1)+\frac{3}{2}(p+1)}{(n+1)(n+p)}d_n\ ,\ n=0,1,\ldots \]
ο οποίος μας δίνει
\[  \]
\begin{Askhshs}[C]
Με τη βοήθεια της αντικατάστασης $ x=e^t $, να επιλυθεί η ομογενής γραμμική διαφορικής εξίσωση
\begin{equation}\label{eq:c15}
2\frac{d^2y}{dt^2}-\frac{dy}{dt}+e^{2t}y=0\ ,\ t\in\mathbb{R}\tag{E}
\end{equation}
\textit{Υπόδειξη : Για την ομογενή γραμμική διαφορική εξίσωση που θα προκύψει από την αντικατάσταση $ x=e^t $, θα πρέπει να βρεθούν οι δυναμοσειρές λύσεις για $ x>0 $ (γύρω από το σημείο $ x_0=0 $)}.
\end{Askhshs}\mbox{}\\
\lysh
Για το μετασχηματισμό $ x=e^t $ έχουμε
\[ x=e^t\Rightarrow \frac{dx}{dt}=e^t \]
καθώς και
\[ \frac{dy}{dt}=\frac{dy}{dx}\frac{dx}{dt}=e^t\frac{dy}{dx}\ \text{ και }\ \frac{d^2y}{dt^2}=\frac{d}{dt}\left[e^t\frac{dy}{dx}\right]=e^t\frac{dy}{dx}+e^{2t}\frac{d^2y}{dx^2} \]
Με αντικατάσταση στην \eqref{eq:c15} θα πάρουμε
\begin{gather}
2\left[e^t\frac{dy}{dx}+e^{2t}\frac{d^2y}{dx^2}\right]-e^t\frac{dy}{dx}+e^{2t}y=0\Rightarrow\nonumber\\
2e^{2t}y''+e^ty'+e^{2t}y=0\Rightarrow\nonumber\\
2e^ty''+y'+e^ty=0\xRightarrow[]{x=e^t}\nonumber\\
2xy''+y'+xy=0\label{eq:c15_2}\tag{$ E^* $}
\end{gather}
Από την εξίσωση \eqref{eq:c15_2} ορίζουμε τις συναρτήσεις $ a_2(x)=x,a_1(x)=1 $ και $ a_0(x)=x $ με $ x\in\mathbb{R} $ και παρατηρούμε ότι $ a_2(0)=0 $ οπότε το σημείο $ x_0=0 $ είναι μή ομαλό σημείο. Στη συνέχεια οι συναρτήσεις 
\[ A_1(x)=\frac{x\cdot 1}{2x}=\frac{1}{2}\ \ \text{και}\ \ A_0(x)=\frac{x^2\cdot x}{2x}=\frac{x^2}{2}\ ,\ x\in\mathbb{R} \]
είναι αναλυτικές στο $ x_0=0 $ άρα αυτό είναι ένα κανονικό ανώμαλο σημείο. Οι $ A_1,A_0 $ γράφονται ως δυναμοσειρές στη μορφή
\[ A_1(x)=\sum_{n=0}^{\infty}{p_nx^n}\ \ ,\ \  x\in\mathbb{R}\ ,\ |x|<R_1 \]
όπου $ p_0=\frac{1}{2} $ και $ p_n=0,n\neq 0 $ με ακτίνα σύγκλισης $ R_1=\frac{1}{\limsup{\sqrt[n]{p_n}}}=+\infty $ και αντίστοιχα
\[ A_0(x)=\sum_{n=0}^{\infty}{q_nx^n}\ \ ,\ \ x\in\mathbb{R}\ ,\ |x|<R_2 \]
όπου $ q_2=\frac{1}{2} $ και $ q_n=0,n\neq 2 $ με ακτίνα σύγκλισης $ R_2=\frac{1}{\limsup{\sqrt[n]{q_n}}}=+\infty $. Έχουμε λοιπόν $ R=\min\{R_1,R_2\}=+\infty $. Θεωρούμε την εξίσωση $ P(\lambda)=\lambda^2+\left(\frac{1}{2}-1\right)\lambda+0=0\Rightarrow\lambda^2-\frac{\lambda}{2}=0\Rightarrow \lambda_1=0,\lambda_2=\frac{1}{2} $. Έτσι οι δύο ανεξάρτητες λύσεις θα γραφτούν με τη μορφή δυναμοσειράς ως
\[ y_1(x)=\sum_{n=0}^{\infty}{c_nx^n}\ \ \text{και}\ \ y_2(x)=|x|^{\frac{1}{2}}\sum_{n=0}^{\infty}{d_nx^n} \]
με $ c_0=d_0=1 $. Για τη λύση $ y_1(x) $ με αντικατάσταση στην εξίσωση \eqref{eq:c15_2} θα πάρουμε
\begin{gather*}
2x\sum_{n=2}^{\infty}{n(n-1)c_nx^{n-2}}+\sum_{n=1}^{\infty}{nc_nx^{n-1}}+x\sum_{n=0}^{\infty}{c_nx^n}=0\Rightarrow\\
2\sum_{n=2}^{\infty}{n(n-1)c_nx^{n-1}}+\sum_{n=1}^{\infty}{nc_nx^{n-1}}+\sum_{n=0}^{\infty}{c_nx^{n+1}}=0\Rightarrow\\
2\sum_{n=1}^{\infty}{n(n+1)c_{n+1}x^{n}}+\sum_{n=0}^{\infty}{(n+1)c_{n+1}x^{n}}+\sum_{n=1}^{\infty}{c_{n-1}x^{n}}=0\Rightarrow\\
c_1+\sum_{n=0}^{\infty}{\left[c_{n+1}(n+1)(2n+1)+c_{n-1}\right]x^n}=0\Rightarrow\\
c_1=0\ \ \text{και}\ \ c_{n+1}(n+1)(2n+1)+c_{n-1}=0
\end{gather*}
Από τον τελευταίο τύπο παίρνουμε
\[ c_{n+1}=-\frac{1}{(n+1)(2n+1)}c_{n-1} \]
Παρατηρούμε ότι οι δείκτες έχουν διαφορά δύο μονάδων οπότε τους χωρίζουμε σε άρτιους και περιττούς. Για τους μεν άρτιους έχουμε
\[ c_{2n}=-\frac{1}{2n(4n-1)}c_{2n-2}\ ,\ n=1,2,\ldots\xRightarrow{c_0=1} c_{2n}=\frac{(-1)^n}{2^nn!(4n-1)!!} \]
ενώ για τους περιττούς ομοίως προκύπτει
\[ c_{2n+1}=-\frac{1}{(2n+1)(4n+1)}c_{2n-1}\ \ ,\ \ n=1,2,\ldots\xRightarrow{c_1=0}c_{2n+1}=0 \]
Τελικά η λύση $ y_1 $ θα έχει τη μορφή
\[ y_1(x)=\sum_{n=0}^{\infty}{\frac{(-1)^n}{2^nn!(4n-1)!!}x^{2n}} \]
Εργαζόμαστε αναλόγως για τη λύση $ y_2 $. Παραγωγίζουμε και παίρνουμε για $ x>0 $
\begin{align*}
y_2'(x)&=\frac{1}{2}x^{-\frac{1}{2}}\sum_{n=0}^{\infty}{d_nx^n}+x^{\frac{1}{2}}\sum_{n=1}^{\infty}{nd_nx^{n-1}}\\
y_2''(x)&=-\frac{1}{4}x^{-\frac{3}{2}}\sum_{n=0}^{\infty}{d_nx^n}+\frac{1}{2}x^{-\frac{1}{2}}\sum_{n=1}^{\infty}{nd_nx^{n-1}}+\frac{1}{2}x^{-\frac{1}{2}}\sum_{n=1}^{\infty}{nd_nx^{n-1}}+x^{\frac{1}{2}}\sum_{n=2}^{\infty}{n(n-1)d_nx^{n-2}}=\\
&=-\frac{1}{4}x^{-\frac{3}{2}}\sum_{n=0}^{\infty}{d_nx^n}+x^{-\frac{1}{2}}\sum_{n=1}^{\infty}{nd_nx^{n-1}}+x^{\frac{1}{2}}\sum_{n=2}^{\infty}{n(n-1)d_nx^{n-2}}
\end{align*}
Με αντικατάσταση στην \eqref{eq:c15_2} παίρνουμε
\begin{multline*}
2x\left[-\frac{1}{4}x^{-\frac{3}{2}}\sum_{n=0}^{\infty}{d_nx^n}+x^{-\frac{1}{2}}\sum_{n=1}^{\infty}{nd_nx^{n-1}}+x^{\frac{1}{2}}\sum_{n=2}^{\infty}{n(n-1)d_nx^{n-2}}\right]+\\+\frac{1}{2}x^{-\frac{1}{2}}\sum_{n=0}^{\infty}{d_nx^n}+x^{\frac{1}{2}}\sum_{n=1}^{\infty}{nd_nx^{n-1}}+x\cdot x^{\frac{1}{2}}\sum_{n=0}^{\infty}{d_nx^n}=0\Rightarrow
\end{multline*}
\vspace{-7mm}
\begin{multline*}
-\frac{1}{2}x^{-\frac{1}{2}}\sum_{n=0}^{\infty}{d_nx^n}+2x^{\frac{1}{2}}\sum_{n=1}^{\infty}{nd_nx^{n-1}}+2x^{\frac{3}{2}}\sum_{n=2}^{\infty}{n(n-1)d_nx^{n-2}}+\\+\frac{1}{2}x^{-\frac{1}{2}}\sum_{n=0}^{\infty}{d_nx^n}+x^{\frac{1}{2}}\sum_{n=1}^{\infty}{nd_nx^{n-1}}+x^{\frac{3}{2}}\sum_{n=0}^{\infty}{d_nx^n}=0\Rightarrow
\end{multline*}
\vspace{-7mm}
\begin{multline*}
-\frac{1}{2}\sum_{n=0}^{\infty}{d_nx^{n-\frac{1}{2}}}+2\sum_{n=1}^{\infty}{nd_nx^{n-\frac{1}{2}}}+2\sum_{n=2}^{\infty}{n(n-1)d_nx^{n-\frac{1}{2}}}+\\+\frac{1}{2}\sum_{n=0}^{\infty}{d_nx^{n-\frac{1}{2}}}+\sum_{n=1}^{\infty}{nd_nx^{n-\frac{1}{2}}}+x^{\frac{3}{2}}\sum_{n=0}^{\infty}{d_nx^n}=0\Rightarrow
\end{multline*}
\vspace{-7mm}
\[ \sum_{n=}^{max} \]
\begin{Askhshs}[C]
Με τη βοήθεια της αντικατάστασης $ x=e^t $ να επιλυθεί η ομογενής γραμμική διαφορική εξίσωση
\[ 2\frac{d^2y}{dt^2}-(1+e^t)\frac{dy}{dt}-y=0\ \ ,\ \ t\in\mathbb{R} \]
\end{Askhshs}\mbox{}\\
\lysh
Χρησιμοποιώντας το μετασχηματισμό $ x=e^t $ έχουμε
\[ x=e^t\Rightarrow \frac{dx}{dt}=e^t \]
καθώς και
\[ \frac{dy}{dt}=\frac{dy}{dx}\frac{dx}{dt}=e^t\frac{dy}{dx}\ \text{ και }\ \frac{d^2y}{dt^2}=\frac{d}{dt}\left[e^t\frac{dy}{dx}\right]=e^t\frac{dy}{dx}+e^{2t}\frac{d^2y}{dx^2} \]
Αντικαθιστούμε τις παραγώγους στην αρχική εξίσωση
\begin{gather*}
2\frac{d^2y}{dt^2}-(1+e^t)\frac{dy}{dt}-y=0\Rightarrow\\
2\left[e^t\frac{dy}{dx}+e^{2t}\frac{d^2y}{dx^2}\right]-(1+e^t)\left[e^t\frac{dy}{dx}\right]-y=0\Rightarrow\\
2e^t\frac{dy}{dx}+2e^{2t}\frac{d^2y}{dx^2}-e^t\frac{dy}{dx}-e^{2t}\frac{dy}{dx}-y=0\Rightarrow\\
2x^2y''+(x-x^2)y'-y=0
\end{gather*}
Το $ x_0=0 $ είναι ανώμαλο σημείο της τελευταίας εξίσωσης και επιπλέον οι συναρτήσεις
\[ A_1(x)=\frac{1}{2}-\frac{1}{2}x\ \ \text{ και }\ \ A_0(x)=-\frac{1}{2}\ ,\ x\in\mathbb{R} \]
είναι αναλυτικές στο $ x_0 $ για κάθε $ x\in\mathbb{R} $ άρα αυτό είναι ένα κανονικό ανώμαλο σημείο. Οι $ A_1,A_0 $ γράφονται ως δυναμοσειρές στη μορφή
\[ A_1(x)=\sum_{n=0}^{\infty}{p_nx^n}\ \ ,\ \  x\in\mathbb{R}\ ,\ |x|<R_1 \]
όπου $ p_0=\frac{1}{2},p_1=-\frac{1}{2} $ και $ p_n=0,n\neq 0,1 $ με ακτίνα σύγκλισης $ R_1=\frac{1}{\limsup{\sqrt[n]{p_n}}}=+\infty $ και αντίστοιχα
\[ A_0(x)=\sum_{n=0}^{\infty}{q_nx^n}\ \ ,\ \ x\in\mathbb{R}\ ,\ |x|<R_2 \]
όπου $ q_0=-\frac{1}{2} $ και $ q_n=0,n\neq 0 $ με ακτίνα σύγκλισης $ R_2=\frac{1}{\limsup{\sqrt[n]{q_n}}}=+\infty $. Έχουμε λοιπόν $ R=\min\{R_1,R_2\}=+\infty $. Θεωρούμε την εξίσωση $ P(\lambda)=\lambda^2-\frac{1}{2}\lambda-\frac{1}{2}=0\Rightarrow \lambda_1=1,\lambda_2=-\frac{1}{2} $. Καθώς η διαφορά $ \lambda_1-\lambda_2=\frac{3}{2} $ δεν είναι ακέραια τότε οι δύο ανεξάρτητες λύσεις θα γραφτούν με τη μορφή δυναμοσειράς ως
\[ y_1(x)=|x|\sum_{n=0}^{\infty}{c_nx^n}\stackrel{x>0}{=}\sum_{n=0}^{\infty}{c_nx^{n+1}}\ \ \text{και}\ \ y_2(x)=x^{-\frac{1}{2}}\sum_{n=0}^{\infty}{d_nx^n} \]
με $ c_0=d_0=1 $. Για τη λύση $ y_1(x) $ θα έχουμε
\[
y_1'(x)=\sum_{n=0}^{\infty}{(n+1)c_nx^n}\ \ \text{και}\ \ 
y_1''(x)=\sum_{n=1}^{\infty}{n(n+1)c_nx^{n-1}}
\]
άρα με αντικατάσταση στην εξίσωση παίρνουμε
\begin{gather*}
2x^2\sum_{n=1}^{\infty}{n(n+1)c_nx^{n-1}}+(x-x^2)\sum_{n=0}^{\infty}{(n+1)c_nx^n}-\sum_{n=0}^{\infty}{c_nx^{n+1}}=0\Rightarrow\\
x\left[\sum_{n=1}^{\infty}{2n(n+1)c_nx^{n}}+\sum_{n=0}^{\infty}{(n+1)c_nx^{n}}-\sum_{n=0}^{\infty}{(n+1)c_nx^{n+1}}-\sum_{n=0}^{\infty}{c_nx^{n}}\right]=0\Rightarrow\\
x\left[\sum_{n=1}^{\infty}{2n(n+1)c_nx^{n}}+\sum_{n=0}^{\infty}{(n+1)c_nx^n}-\sum_{n=1}^{\infty}{nc_{n-1}x^{n}}-\sum_{n=0}^{\infty}{c_nx^{n}}\right]=0\Rightarrow\\
x\left[\sum_{n=1}^{\infty}{n\left[(2n+3)c_n-c_{n-1}\right]x^n}\right]=0
\end{gather*}
Από την τελευταία ισότητα προκύπτει ο αναδρομικός τύπος
\[ (2n+3)c_n-c_{n-1}\Rightarrow c_n=\frac{1}{2n+3}c_{n-1}\ ,\ n=1,2,\ldots \]
από τον οποίο παίρνουμε για $ c_0=1 $
\[ c_n=\frac{3}{(2n+3)!!} \]
και έτσι η λύση $ y_1 $ θα πάρει τη μορφή
\[ y_1(x)=\sum_{n=0}^{\infty}{\frac{3}{(2n+3)!!}x^n} \]
Εργαζόμαστε αναλόγως για τη λύση $ y_2 $ και έχουμε αρχικά τις παραγώγους
\begin{align*}
y_2'(x)&=-\frac{1}{2}x^{-\frac{3}{2}}\sum_{n=0}^{\infty}{d_nx^n}+x^{-\frac{1}{2}}\sum_{n=1}^{\infty}{nd_nx^{n-1}}=x^{-\frac{3}{2}}\sum_{n=0}^{\infty}{\left(n-\frac{1}{2}\right)d_nx^n}\\
y_2''(x)&=-\frac{3}{2}x^{-\frac{5}{2}}\sum_{n=0}^{\infty}{\left(n-\frac{1}{2}\right)d_nx^{n}}+x^{-\frac{3}{2}}\sum_{n=1}^{\infty}{n\left(n-\frac{1}{2}\right)d_nx^{n-1}}=x^{-\frac{5}{2}}\sum_{n=0}^{\infty}{\left(n-\frac{1}{2}\right)\left(n-\frac{3}{2}\right)d_nx^n}
\end{align*}
Αντικαθιστούμε στην εξίσωση
\begin{gather*}
2x^2x^{-\frac{5}{2}}\sum_{n=0}^{\infty}{\left(n-\frac{1}{2}\right)\left(n-\frac{3}{2}\right)d_nx^n}+(x-x^2)x^{-\frac{3}{2}}\sum_{n=0}^{\infty}{\left(n-\frac{1}{2}\right)d_nx^n}-x^{-\frac{1}{2}}\sum_{n=0}^{\infty}{d_nx^n}=0\Rightarrow\\
2x^{-\frac{1}{2}}\sum_{n=0}^{\infty}{\left(n-\frac{1}{2}\right)\left(n-\frac{3}{2}\right)d_nx^n}+(1-x)x^{-\frac{1}{2}}\sum_{n=0}^{\infty}{\left(n-\frac{1}{2}\right)d_nx^n}-x^{-\frac{1}{2}}\sum_{n=0}^{\infty}{d_nx^n}=0\Rightarrow\\
x^{-\frac{1}{2}}\left[2\sum_{n=0}^{\infty}{\left(n-\frac{1}{2}\right)\left(n-\frac{3}{2}\right)d_nx^n}+x^{-\frac{1}{2}}\sum_{n=0}^{\infty}{\left(n-\frac{1}{2}\right)d_nx^n}-x^{-\frac{1}{2}}\sum_{n=0}^{\infty}{\left(n-\frac{1}{2}\right)d_nx^{n+1}}-x^{-\frac{1}{2}}\sum_{n=0}^{\infty}{d_nx^n}\right]=0\Rightarrow\\
x^{-\frac{1}{2}}\left[\sum_{n=0}^{\infty}{2n\left(n-\frac{3}{2}\right)d_nx^n}-\sum_{n=1}^{\infty}{\left(n-\frac{3}{2}\right)d_{n-1}x^n}\right]=0\Rightarrow\\
x^{-\frac{1}{2}}\left[\sum_{n=0}^{\infty}{\left(n-\frac{3}{2}\right)(2nd_n-d_{n-1})x^n}\right]=0
\end{gather*}
Επομένως $ 2d_n-d_{n-1}=0\Rightarrow d_n=\frac{1}{2n}d_{n-1}\ , \ n=1,2,\ldots $. Από τον αναδρομικό τύπο αυτό προκύπτει
\[ d_n=\frac{1}{2^nn!}\ ,\ n=1,2,\ldots \]
και έτσι η λύση $ y_2 $ θα δίνεται από τον τύπο
\[ y_2(x)=x^{\frac{1}{2}}\sum_{n=0}^{\infty}{\frac{1}{2^nn!}x^n}\ \ ,\ \ x\neq 0 \]
Η γενική λύση θα δίνεται ως γραμμικός συνδυασμός των ανεξάρτητων $ y_1,y_2 $ στη μορφή $ y(x)=c_1y_1(x)+c_2y_2(x) $ όπου $ c_1,c_2 $ είναι αυθαίρετες σταθερές.
\begin{Askhshs}[C]
Να αποδειχθεί ότι η γραμμική διαφορική εξίσωση
\[ 2xy''+y'+xy=0 \]
δέχεται μια λύση της μορφής
\[ y(x)=|x|^{\lambda}\left(1+\sum_{n=1}^{\infty}{c_nx^n}\right)\ \ ,\ \ \text{για }x\neq 0 \]
όπου $ \lambda\neq 0 $ και $ c_n,\ n-1,2,\ldots $ πραγματικοί αριθμοί. Να βρεθεί η λύση αυτή.
\end{Askhshs}\mbox{}\\\\
\lysh
Το $ x_0=0 $ είναι ανώμαλο σημείο της διαφορικής εξίσωσης. Θεωρούμε τις συναρτήσεις
\[ A_1(x)=\frac{(x-0)\cdot 1}{2x}=\frac{1}{2}\ \ \text{και}\ \ A_2(x)=\frac{(x-0)^2x}{2x}=\frac{x^2}{2} \]
οι οποίες είναι αναλυτικές στο $ x_0=0 $. Επομένως το σημείο αυτό είναι κανονικό ανώμαλο σημείο και η ενδεικτική εξίσωση της διαφορικής θα είναι 
\[ P(\lambda)=\lambda^2-\left(p_0-1\right)\lambda-q_0=0\Rightarrow \lambda^2-\frac{\lambda}{2}=0 \]
με ρίζες $ \lambda_1=0 $ και $ \lambda_2=\frac{1}{2} $. Επομένως η διαφορική εξίσωση θα έχει μία λύση της μορφής
\[ y(x)=|x|^{\lambda}\sum_{n=0}^{\infty}{c_nx^n}\ \ \text{με}\ \ c_0=1 \]
άρα
\[ y(x)=|x|^{\lambda}\left(c_0+\sum_{n=1}^{\infty}{c_nx^n}\right)=|x|^{\lambda}\left(1+\sum_{n=1}^{\infty}{c_nx^n}\right) \]
Για την εύρεση της λύσης παίρνουμε τη ρίζα $ \lambda=0 $ και αυτή γράφεται
\[ y(x)=\sum_{n=0}^{\infty}{c_nx^n} \]
Αντικαθιστούμε την $ y(x) $ καθώς και τις παραγώγους της στη διαφορική και έχουμε
\begin{gather*}
2x\sum_{n=2}^{\infty}{n(n-1)c_{n}x^{n-2}}+\sum_{n=1}^{\infty}{nc_nx^{n-1}}+x\sum_{n=0}^{\infty}{c_nx^n}=0\Rightarrow\\
2\sum_{n=2}^{\infty}{n(n-1)c_{n}x^{n-1}}+\sum_{n=1}^{\infty}{nc_nx^{n-1}}+\sum_{n=0}^{\infty}{c_nx^{n+1}}=0\Rightarrow\\
2\sum_{n=1}^{\infty}{n(n+1)c_{n+1}x^{n}}+\sum_{n=0}^{\infty}{(n+1)c_{n+1}x^{n}}+\sum_{n=1}^{\infty}{c_{n-1}x^{n}}=0\Rightarrow\\
c_1+\sum_{n=1}^{\infty}{\left[(n+1)(2n+1)-c_{n-1}\right]x^n}=0
\end{gather*}
Από την τελευταία σχέση παίρνουμε $ c_1=0 $ καθώς και τον αναδρομικό τύπο 
\[ c_{n+1}=-\frac{1}{(n+1)(2n+1)}c_{n-1}\ \ ,\ \ n=1,2,\ldots \]
Χωρίζουμε τους συντελεστές σε αυτούς με περιττό και άρτιο δείκτη και έχουμε για τους άρτιους
\[ c_{2n}=\frac{-1}{2n(4n-1)}c_{2n-2}\xRightarrow{c_0=1} c_{2n}=\frac{(-1)^n}{2^nn!(4n-1)!_{(4)}} \]
ενώ για τους περιττούς ομοίως παίρνουμε
\[ c_{2n+1}=\frac{-1}{(2n+1)(4n+1)}c_1\xRightarrow{c_1=0}c_{2n+1}=0\ ,\ n=1,2,\ldots \]
Η ζητούμενη λύση έτσι θα γραφτεί ως δυναμοσειρά στη μορφή
\[ y(x)=\sum_{n=0}^{\infty}{c_{2n}x^{2n}}=1+\sum_{n=1}^{\infty}{\frac{(-1)^n}{2^nn!(4n-1)!_{(4)}}x^{2n}} \]
\begin{Askhshs}[C]
Να αποδειχθεί ότι η γραμμική διαφορική εξίσωση
\[ x^2y''+x(x-3)y'+3y=0 \]
δέχεται μία λύση της μορφής
\[ y_1(x)=x^{\lambda}\left(1+\sum_{n=1}^{\infty}c_nx^n\right)\ \ \text{για}\ \ x>0 \]
όπου $ \lambda>0 $ και $ c_n,n=1,2,\ldots $ πραγματικοί αριθμοί. Να βρεθεί η λύση αυτή. Στη συνέχεια να βρεθεί μια λύση $ y_2 $ της εξίσωσης ώστε $ y_1,y_2 $ γραμμικά ανεξάρτητες.
\end{Askhshs}\mbox{}\\\\
\lysh
Το σημείο $ x_0=0 $ είναι ανώμαλο σημείο της εξίσωσης. Θεωρούμε τις συναρτήσεις
\[ A_1(x)=\frac{(x-0)x(x-3)}{x^2}=x-3\ \ \text{και}\ \ A_0(x)=\frac{x^23}{x^2}=3 \]
οι οποίες είναι αναλυτικές στο $ x_0=0 $ και έτσι το σημείο αυτό θα είναι κανονικό ανώμαλο σημείο της εξίσωσης. Η ενδεικτική εξίσωση στο $ x_0 $ θα είναι $ P(\lambda)=\lambda^2+(p_0-1)\lambda+q_0=0\Rightarrow \lambda^2-4\lambda+3=0 $ με ρίζες $ \lambda_1=3 $ και $ \lambda_2=1 $. Μια λύση λοιπόν της διαφορικής εξίσωσης θα είναι
\[ y_1(x)=|x|^{\lambda}\sum_{n=0}^{\infty}{c_nx^n}\stackrel{c_0=1}{=}|x|^{\lambda}\left(1+\sum_{n=0}^{\infty}{c_nx^n}\right) \]
Για την εύρεση της λύσης $ y_1 $ επιλέγουμε $ \lambda=3 $ και αυτή γράφεται
\[ y(x)=|x|^3\left(1+\sum_{n=0}^{\infty}{c_nx^n}\right)\stackrel{x>0}{=}x^3\sum_{n=0}^{\infty}{c_nx^n} \]
Για αυτήν θα έχω
\begin{align*}
y_1'(x)&=3x^2\sum_{n=0}^{\infty}{c_nx^n}+x^3\sum_{n=1}^{\infty}{nc_nx^{n-1}}=\\
&=3x^2\sum_{n=0}^{\infty}{c_nx^n}+x^2\sum_{n=1}^{\infty}{nc_nx^{n}}=x^2\sum_{n=0}^{\infty}{(n+3)c_nx^n}
\end{align*}
και αντίστοιχα
\begin{align*}
y_1''(x)&=2x\sum_{n=0}^{\infty}{(n+3)c_nx^n}+x^2\sum_{n=1}^{\infty}{n(n+3)c_nx^{n-1}}=\\
&=2x\sum_{n=0}^{\infty}{(n+3)c_nx^n}+x\sum_{n=1}^{\infty}{n(n+3)c_nx^{n}}=x\sum_{n=0}^{\infty}{(n+2)(n+3)c_nx^n}
\end{align*}
Με αντικατάσταση των παραγώγων στη διαφορική εξίσωση παίρνουμε
\begin{gather*}
x^2y_1''(x)+x(x-3)y_1'(x)+3y_1(x)=0\Rightarrow\\
x^2\cdot x\sum_{n=0}^{\infty}{(n+2)(n+3)c_nx^n}+x(x-3)x^2\sum_{n=0}^{\infty}{(n+3)c_nx^n}+3x^3\sum_{n=0}^{\infty}{c_nx^n}=0\Rightarrow\\
x^3\left[\sum_{n=0}^{\infty}{(n+2)(n+3)c_nx^n}+(x-3)\sum_{n=0}^{\infty}{(n+3)c_nx^n}+3\sum_{n=0}^{\infty}{c_nx^n}\right]=0\Rightarrow\\
x^3\left[\sum_{n=0}^{\infty}{(n+2)(n+3)c_nx^n}+\sum_{n=0}^{\infty}{(n+3)c_nx^{n+1}}-3\sum_{n=0}^{\infty}{(n+3)c_nx^n}+3\sum_{n=0}^{\infty}{c_nx^n}\right]=0\Rightarrow\\
x^3\left[\sum_{n=0}^{\infty}{\left[n(n+2)c_n+(n+2)c_{n-1}\right]x^n}\right]=0
\end{gather*}
Επομένως καταλήγουμε στον αναδρομικό τύπο $ c_n=\frac{-1}{n}c_{n-1} $ από τον οποίο προκύπτει $ c_n=\frac{(-1)^n}{n!}\ ,\ n=1,2,\ldots $. Η λύση λοιπόν θα είναι
\[ y(x)=x^3\left[1+\sum_{n=0}^{\infty}{\frac{(-1)^n}{n!}}x^n\right]=x^3e^{-x} \]
Εφόσον $ \lambda_2-\lambda_1=2 $ θετικός ακέραιος, η λύση $ y_2 $, γραμμικά ανεξάρτητη της $ y_1 $, θα είναι της μορφής
\[ y_2(x)=cy_1(x)\log{|x|}+|x|\sum_{n=0}^{\infty}{d_nx^n}\ \ ,\ \ \text{για }x\neq0 \]
όπου $ d_0=1 $. Θέτουμε $ z(x)=cy_1(x)\log{|x|} $ άρα η $ y_2 $ γράφεται $ y_2(x)=cy_1(x)\log{|x|}+|x|z(x) $ και έχουμε διαδοχικά τις παραγώγους
\begin{align*}
y_2'(x)&=cy_1'(x)\log{|x|}+\frac{c}{x}y_1(x)+z(x)+xz'(x)\\
y_2''(x)&=cy_1''(x)\log{|x|}+\frac{2c}{x}y_1'(x)-\frac{1}{x^2}cy_1(x)+2z'(x)+xz''(x)
\end{align*}
Αντικαθιστούμε στην εξίσωση και παίρνουμε
\[ x^2y_2''(x)+x(x-3)y_2'(x)+3y_2(x)=0\Rightarrow \]
\vspace{-9mm}
\begin{multline*}
x^2cy_1''(x)\log{|x|}+2cxy_1'(x)-cy_1(x)+2z'(x)+x^3z''(x)+\\+x(x-3)cy_1'(x)\log{|x|}+c(x-3)y_1(x)+x(x-3)z(x)+\\+x^2(x-3)z'(x)+3cy_1(x)\log{|x|}+3xz(x)=0\Rightarrow
\end{multline*}
\vspace{-9mm}
\begin{gather*}
x^2z''(x)+x^2(x-1)z'(x)+x^2z(x)+c\left[2xy_1'(x)+(x-4)y_1(x)\right]=0\Rightarrow\\
x^2\left[x\sum_{n=2}^{\infty}{n(n-1)d_nx^{n-2}}+(x-1)\sum_{n=1}^{\infty}{nd_nx^{n-1}}+\sum_{n=0}^{\infty}{d_nx^n}\right]+c\left[6x^3e^{-x}-2x^4e^{-x}+x^4e^{-x}-4x^3e^{-x}\right]=0\Rightarrow\\
x^2\left[\sum_{n=2}^{\infty}{n(n-1)d_nx^{n-1}}+\sum_{n=1}^{\infty}{nd_nx^{n}}-\sum_{n=1}^{\infty}{nd_nx^{n-1}}+\sum_{n=0}^{\infty}{d_nx^n}\right]+c\left[2x^3e^{-x}-x^4e^{-x}\right]=0\Rightarrow\\
x^2\left[\sum_{n=1}^{\infty}{n(n+1)d_{n+1}x^{n}}+\sum_{n=1}^{\infty}{nd_nx^{n}}-\sum_{n=0}^{\infty}{(n+1)d_{n+1}x^{n}}+\sum_{n=0}^{\infty}{d_nx^n}\right]+c\left[2x^3e^{-x}-x^4e^{-x}\right]=0\Rightarrow\\
x^2\left[\sum_{n=1}^{\infty}{\left[(n-1)(n+1)d_{n+1}+(n+1)d_{n}\right]x^{n}}\right]+c\left[2x^3e^{-x}-x^4e^{-x}\right]=0
\end{gather*}
Επομένως προκύπτει
\[ c=0\ \ \text{και}\ \ d_{n+1}=-\frac{1}{n-1}d_n\ ,\ n=0,1,\ldots \]
Από τον αναδρομικό τύπο προκύπτει $ d_0=1,d_1=\frac{1}{2} $ και $ d_n=0,n=2,3,\ldots $ άρα η λύση $ y_2 $ θα είναι
\[ y_2(x)=d_0x+d_1x^2=x+\frac{x^2}{2} \]
\begin{Askhshs}[C]
Έστω η διαφορική εξίσωση
\begin{equation}\label{c:19_1}
y''+\left(p+\frac{1}{2}-\frac{1}{4}x^2\right)y=0\tag{$ E $}
\end{equation}
όπου $ p $ σταθερά.
\begin{rlist}
\item Να βρεθούν οι δυναμοσειρές λύσεις της \eqref{c:19_1} γύρω από το σημείο $ x_0=0 $.
\item Να αποδειχθεί ότι η αντικατάσταση $ y=ze^{-\frac{x^2}{4}} $ μετασχηματίζει την \eqref{c:19_1} στην $ z''-xz'+pz=0\ \ {E^*} $. Να βρεθούν οι δυναμοσειρές λύσεις γύρω από σο σημείο $ x_0=0 $ της $ E^* $.
\item Να βρεθούν όλες οι λύσεις της \eqref{c:19_1}.
\end{rlist}
\end{Askhshs}\mbox{}\\\\
\lysh
\begin{rlist}
\item Το σημείο $ x_0=0 $ είναι ομαλό σημείο της \eqref{c:19_1}. Οι λύσεις της θα έχουν τη μορφή δυναμοσειράς
\[ y(x)=\sum_{n=0}^{\infty}{c_nx^n}\ \ ,\ \ \forall x\in\mathbb{R} \]
Έχουμε λοιπόν με αντικατάσταση
\begin{gather*}
y''(x)+\left(p+\frac{1}{2}-\frac{1}{4}x^2\right)y=0\Rightarrow\\
\sum_{n=2}^{\infty}{n(n-1)c_nx^{n-2}}+\left(p+\frac{1}{2}-\frac{1}{4}x^2\right)\sum_{n=0}^{\infty}{c_nx^n}=0\Rightarrow\\
\sum_{n=2}^{\infty}{n(n-1)c_nx^{n-2}}+\left(p+\frac{1}{2}\right)\sum_{n=0}^{\infty}{c_nx^n}-\frac{1}{4}\sum_{n=0}^{\infty}{c_nx^{n+2}}=0\Rightarrow\\
\sum_{n=0}^{\infty}{(n+1)(n+2)c_{n+2}x^{n}}+\left(p+\frac{1}{2}\right)\sum_{n=0}^{\infty}{c_nx^n}-\frac{1}{4}\sum_{n=2}^{\infty}{c_{n-2}x^{n}}=0\Rightarrow\\
2c_2+\left(p+\frac{1}{2}\right)c_0+\left[6c_3+\left(p+\frac{1}{2}\right)c_1\right]x+\sum_{n=2}^{\infty}{\left[(n+1)(n+2)c_{n+2}+\left(p+\frac{1}{2}\right)c_n-\frac{1}{4}c_{n-2}\right]x^n}=0
\end{gather*}
Επομένως παίρνουμε
\begin{align*}
&2c_2+\left(p+\frac{1}{2}\right)c_0=0\Rightarrow c_2=-\frac{p+\frac{1}{2}}{2}c_0\\ &6c_3+\left(p+\frac{1}{2}\right)c_1=0\Rightarrow c_3=-\frac{p+\frac{1}{2}}{6}c_1\ \ \text{και}\\
&(n+1)(n+2)c_{n+2}+\left(p+\frac{1}{2}\right)c_n-\frac{1}{4}c_{n-2}=0\ \ ,\ \ n=2,3,\ldots
\end{align*}
\item Αν θέσουμε $ y=ze^{-\frac{x^2}{4}} $ τότε παίρνουμε τις παραγώγους
\[ y'=\left(z'-\frac{x}{2}z\right)e^{-\frac{x^2}{4}}\ \ \text{και}\ \ y''=\left[z''-xz'-\left(\frac{1}{2}-\frac{x^2}{4}\right)z\right]e^{-\frac{x^2}{4}} \]
Αντικαθιστώντας στην \eqref{c:19_1} θα έχουμε
\begin{gather*}
y''(x)+\left(p+\frac{1}{2}-\frac{x^2}{4}\right)y=0\Rightarrow\\
\left[z''-xz'-\left(\frac{1}{2}-\frac{x^2}{4}\right)z\right]e^{-\frac{x^2}{4}}+\left(p+\frac{1}{2}-\frac{x^2}{4}\right)ze^{-\frac{x^2}{4}}=0\Rightarrow\\
\left[z''-xz'-\frac{1}{2}z+\frac{x^2}{4}z+pz+\frac{1}{2}z-\frac{x^2}{4}z\right]e^{-\frac{x^2}{4}}=0\Rightarrow\\
z''-xz'+pz=0
\end{gather*}
Το $ x_0=0 $ είναι ομαλό σημείο της $ E^* $ και οι λύσεις της γύρω απ' αυτό θα έχουν τη μορφή
\[ z(x)=\sum_{n=0}^{\infty}{d_nx^n}\ ,\ x\in\mathbb{R} \]
Θα έχουμε λοιπόν
\begin{gather*}
z''(x)-xz'(x)+pz(x)=0\Rightarrow\\
\sum_{n=2}^{\infty}{n(n-1)d_nx^{n-2}}-x\sum_{n=1}^{\infty}{nd_nx^{n-1}}+p\sum_{n=0}^{\infty}{d_nx^n}=0\Rightarrow\\
\sum_{n=0}^{\infty}{(n+1)(n+2)d_{n+2}x^{n}}-\sum_{n=0}^{\infty}{nd_nx^{n}}+p\sum_{n=0}^{\infty}{d_nx^n}=0\Rightarrow\\
\sum_{n=0}^{\infty}{\left[(n+1)(n+2)d_{n+2}+(p-n)d_n\right]x^n}=0
\end{gather*}
Παίρνουμε τον αναδρομικό τύπο
\[ d_{n+2}=\frac{n-p}{(n+1)(n+2)}d_n\ ,\ n=0,1,2,\ldots \]
Χωρίζουμε τους συντελεστές $ d_n $ σε άρτιους και περιττούς δείκτες και προκύπτουν αντίστοιχα για τους μεν άρτιους
\[ d_{2n}=\frac{(2n-2)-p}{2n(2n-1)}d_{2n-2}\Rightarrow d_{2n}=\frac{\prod_{k=0}^{2n-2}{(k-p)}}{2^nn!(2n-1)!!}d_0\ ,\ n=1,2,\ldots \]
ενώ για τους περιττούς
\[ d_{2n+1}=\frac{(2n-1)-p}{2n(2n+1)}d_1\Rightarrow d_{2n+1}=\frac{\prod_{k=0}^{2n-1}{(k-p)}}{2^nn!(2n+1)!!}d_1\ ,\ n=1,2,\ldots \]
Η γενική λύση λοιπόν της $ E^* $ θα δίνεται από τον τύπο
\begin{align*}
z(x)&=d_0+d_1x+\sum_{n=1}^{\infty}{d_{2n}x^{2n}}+\sum_{n=1}^{\infty}{d_{2n+1}x^{2n+1}}=\\
&=d_0\left[\undercbrace{1+\sum_{n=1}^{\infty}{\frac{\prod_{k=0}^{2n-2}{(k-p)}}{2^nn!(2n-1)!!}x^{2n}}}_{z_1(x)}\right]+d_1\left[\undercbrace{1+\sum_{n=1}^{\infty}{\frac{\prod_{k=0}^{2n-1}{(k-p)}}{2^nn!(2n+1)!!}x^{2n+1}}}_{z_2(x)}\right]
\end{align*}
όπου $ z_1(x),z_2(x) $ δύο ανεξάρτητες λύσεις της $ E^* $.
\item Οι συναρτήσεις $ y_1(x)=z_1(x)e^{-\frac{x^2}{4}} $ και $ y_2(x)=z_2e^{-\frac{x^2}{4}} $ είναι γραμμικά ανεξάρτητες λύσεις της \eqref{c:19_1} άρα όλες οι λύσεις της θα δίνονται από τον τύπο
\[ y(x)=c_1y_1(x)+c_2y_2(x) \]
όπου $ c_1,c_2 $ αυθαίρετες σταθερές.
\end{rlist}
\begin{Askhshs}[C]
Να αποδειχθεί ότι η διαφορική εξίσωση του Bessel τάξης $ \frac{1}{2} $
\begin{equation}\label{c:20_1}
x^2y''+xy'+\left(x^2-\frac{1}{4}\right)y=0\ ,\ x>0\tag{$ E_0 $}
\end{equation}
δέχεται μία λύση της μορφής
\[ y_1(x)=x^{\lambda}\left(1+\sum_{n=1}^{\infty}{c_nx^n}\right)\ ,\ x>0 \]
όπου $ \lambda>0 $ και $ c_n,n=1,2,\ldots $ πραγματικοί αριθμοί. Να βρεθεί αυτή η λύση. Δίνεται $ \sum_{n=0}^{\infty}{\frac{(-1)^nx^{2n+1}}{(2n+1)!}}=\sin{x} $. Στη συνέχεια να βρεθεί λύση $ y_2 $ της \eqref{c:20_1} ώστε $ y_1,y_2 $ γραμμικά ανεξάρτητες.
\end{Askhshs}\mbox{}\\\\
\lysh
Το $ x_0=0 $ είναι ανώμαλο σημείο της \eqref{c:20_1}. Θεωρούμε τις συναρτήσεις
\[ A_1(x)=\frac{x\cdot x}{x^2}=1\ \ \text{και}\ \ A_0(x)=\frac{x^2\left(x^2-\frac{1}{4}\right)}{x^2}=x^2-\frac{1}{4} \]
οι οποίες είναι αναλυτικές στο $ x_0=0 $ άρα το σημείο αυτό είναι κανονικό ανώμαλο σημείο της \eqref{c:20_1}. Η ενδεικτική εξίσωση της είναι $ P(\lambda)=\lambda^2-\frac{\lambda}{4}=0 $ με ρίζες $ \lambda_1=\frac{1}{2} $ και $ \lambda_2=-\frac{1}{2} $. Η εξίσωση θα έχει δύο γραμμικά ανεξάρτητες λύσεις $ y_1,y_2 $ όπου
\begin{align*}
y_1(x)&=|x|^{\frac{1}{2}}\sum_{n=0}^{\infty}{c_nx^n}\ \ ,\ \ x\in\mathbb{R}\ \text{και}\ c_0=1\\
y_2(x)&=|x|^{-\frac{1}{2}}\sum_{n=0}^{\infty}{d_nx^n}\ \ ,\ \ x\neq0\ \text{και}\ d_0=1
\end{align*}
Για τη λύση $ y_1 $ θα έχουμε
\begin{align*}
y_1(x)&=|x|^{\frac{1}{2}}\sum_{n=0}^{\infty}{c_nx^n}\\
y'_1(x)&=\frac{1}{2}|x|^{-\frac{1}{2}}\sum_{n=0}^{\infty}{c_nx^n}+|x|^{\frac{1}{2}}\sum_{n=1}^{\infty}{nc_nx^{n-1}}=\\
&=\frac{1}{2}x^{-\frac{1}{2}}\sum_{n=0}^{\infty}{c_nx^n}+x^{-\frac{1}{2}}\sum_{n=0}^{\infty}{nc_nx^n}=x^{-\frac{1}{2}}\sum_{n=0}^{\infty}{\left(n+\frac{1}{2}\right)c_nx^n}\\
y_1''(x)&=-\frac{1}{2}x^{-\frac{3}{2}}\sum_{n=0}^{\infty}{\left(n+\frac{1}{2}\right)c_nx^n}+x^{-\frac{1}{2}}\sum_{n=1}^{\infty}{n\left(n+\frac{1}{2}\right)c_nx^{n-1}}=\\
&=-\frac{1}{2}x^{-\frac{3}{2}}\sum_{n=0}^{\infty}{\left(n+\frac{1}{2}\right)c_nx^n}+x^{-\frac{3}{2}}\sum_{n=1}^{\infty}{n\left(n+\frac{1}{2}\right)c_nx^n}=\\&=x^{-\frac{3}{2}}\sum_{n=0}^{\infty}{\left(n+\frac{1}{2}\right)\left(n-\frac{1}{2}\right)c_nx^n}
\end{align*}
Έτσι με αντικατάσταση θα πάρουμε
\begin{gather*}
x^2y_1''(x)+xy_1'(x)+\left(x^2-\frac{1}{4}\right)=0\Rightarrow\\
x^2x^{-\frac{3}{2}}\sum_{n=0}^{\infty}{\left(n+\frac{1}{2}\right)\left(n-\frac{1}{2}\right)c_nx^n}+xx^{-\frac{1}{2}}\sum_{n=0}^{\infty}{\left(n+\frac{1}{2}\right)c_nx^n}+\left(x^2-\frac{1}{4}\right)x^{\frac{1}{2}}\sum_{n=0}^{\infty}{c_nx^n}=0\Rightarrow\\
x^{\frac{1}{2}}\left[\sum_{n=0}^{\infty}{\left(n+\frac{1}{2}\right)^2c_nx^n}+\sum_{n=0}^{\infty}{c_nx^{n+2}-\frac{1}{4}\sum_{n=0}^{\infty}{c_nx^n}}\right]=0\Rightarrow\\
x^{\frac{1}{2}}\left[\sum_{n=0}^{\infty}{n(n+1)c_nx^n}+\sum_{n=2}^{\infty}{c_{n-2}x^n}\right]=0\Rightarrow\\
x^{\frac{1}{2}}\left[2c_1x+\sum_{n=0}^{\infty}{\left[n(n+1)c_n+c_{n-2}\right]x^n}\right]=0
\end{gather*}
Επομένως προκύπτει
\[ c_1=0\ \text{και}\ c_n=-\frac{1}{n(n+1)}c_{n-2}\ ,\ n=2,3,\ldots \]
Σύμφωνα με τον αναδρομικό τύπο προκύπτει ότι όλοι οι συντελεστές με περιττό δείκτη είναι μηδενικοί : $ c_{2n+1}=0 $ ενώ γι αυτούς με άρτιο δείκτη παίρνουμε
\[ c_{2n}=\frac{-1}{2n(2n+1)}c_{2n-2}\Rightarrow c_{2n}=\frac{(-1)^n}{2^{n}n!(2n+1)!!}=\frac{(-1)^n}{(2n+1)!}\ \ ,\ \ n=1,2,\ldots \]
Έτσι η λύση $ y_1 $ θα δίνεται από τον τύπο
\[ y_1(x)=x^{\frac{1}{2}}\sum_{n=0}^{\infty}{c_{2n}x^{2n}}=x^{-\frac{1}{2}}\sum_{n=0}^{\infty}{\frac{(-1)^n}{(2n+1)!}x^{2n+1}}=x^{-\frac{1}{2}}\sin{x}\ ,\ x>0 \]
\end{document}