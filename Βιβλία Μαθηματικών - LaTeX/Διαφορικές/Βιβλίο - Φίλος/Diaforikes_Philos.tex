\documentclass[11pt,a4paper,twoside]{book}
\usepackage[english,greek]{babel}
\usepackage[utf8]{inputenc}
%------------------ Γραμματοσειρές -------------------------------------
\usepackage{nimbusserif}% Βασικό font : 
\usepackage[T1]{fontenc}
\newcommand{\titlefont}[1]{{\fontfamily{maksf}\selectfont #1}}
%-----------------------------------------------------------------------

%------------------- Διάφορα ---------------------------------------------
\usepackage{gensymb,fontawesome5,eurosym,titletoc,sidenotes,multicol,adjmulticol}
\usepackage[framemethod=tikz]{mdframed}
\usepackage[explicit]{titlesec}
%--------------------------------------------------------------------------

%------------------- Γεωμετρία - Στύλ ----------------------------------
\usepackage[inner=1.5cm, top=3cm, bottom=2cm]{geometry}
\geometry{textwidth=12cm,marginparsep=5mm,marginparwidth=5.5cm}
\newcommand{\full}[1]{\begin{mdframed}[outermargin=\dimexpr-\marginparwidth-\marginparsep\relax,innerleftmargin=0mm,innerrightmargin=0mm,hidealllines=true]
#1
\end{mdframed}}
\newcommand{\fulltwoc}[1]{\begin{adjmulticols}{2}{5mm}{\dimexpr-\marginparwidth-\marginparsep+5mm\relax}
#1
\end{adjmulticols}}
%-----------------------------------------------------------------------

%---------------- Μαθηματικά -----------------------------------------
\usepackage{amsmath}
\let\myBbbk\Bbbk
\let\Bbbk\relax
\usepackage[amsbb,subscriptcorrection,zswash,mtpcal,mtphrb,mtpfrak]{mtpro2}
\usepackage{mathimatika,venndiagram,tkz-euclide,diffcoeff,mathtools}

\DeclareMathSizes{10.95}{10.95}{7}{5}
\DeclareMathSizes{6}{6}{3.8}{2.7}
\DeclareMathSizes{8}{8}{5.1}{3.6}
\DeclareMathSizes{9}{9}{5.8}{4.1}
\DeclareMathSizes{10}{10}{6.4}{4.5}
\DeclareMathSizes{12}{12}{7.7}{5.5}
\DeclareMathSizes{14.4}{14.4}{9.2}{6.5}
\DeclareMathSizes{17.28}{17.28}{11}{7.9}
\DeclareMathSizes{20.74}{20.74}{13.3}{9.4}
\DeclareMathSizes{24.88}{24.88}{16}{11.3}

\makeatletter
\AtBeginDocument{
\check@mathfonts
\fontdimen16\textfont2=2.5pt
\fontdimen17\textfont2=2.5pt
\fontdimen14\textfont2=4.5pt
\fontdimen13\textfont2=4.5pt}
\makeatother
\DeclareMathOperator{\tr}{tr}
%----------------------------------------------------------------------

%----------------- Γραφικά και σχήματα --------------------------------
\usepackage{graphicx,tikz,pgfplots,tkz-euclide}
\usetikzlibrary{shadows,calc,fadings}
\tikzset{>=latex}
\tikzstyle{l3}=[line width=0.3mm]
\tikzstyle{l4}=[line width=0.4mm]
\tikzset{labelbox/.style args={#1}{align=center,draw=#1,fill=#1!30!white,rectangle,rounded corners=1,font=\tiny\linespread{0.8}\selectfont}}
\tkzSetUpPoint[size=2.8,fill=white]
\pgfplotsset{compat=newest}
%----------------------------------------------------------------------

%------------------- Χρώματα -------------------------------------
\usepackage[usenames,dvipsnames,table,x11names]{xcolor}
\definecolor{maincolor}{RGB}{35,150,210}
\definecolor{secondarycolor}{RGB}{210,20,210}
\definecolor{thirdcolor}{RGB}{110,195,7}
%-----------------------------------------------------------------

%------------------ Κεφαλίδα και υποσέλιδο ----------------------------
\usepackage{fancyhdr,lipsum}
\pagestyle{fancy}
\fancyheadoffset[LE]{\marginparwidth+\marginparsep}
\fancyheadoffset[RO]{\marginparwidth+\marginparsep}
\renewcommand{\headrulewidth}{\iftopfloat{0pt}{.5pt}}
\renewcommand{\chaptermark}[1]{\markboth{#1}{}}
\renewcommand{\sectionmark}[1]{\markright{\it\thesection\ #1}}
\fancyhf{}
\fancyhead[LE]{\thepage\ $\cdot$\ \nouppercase{\leftmark}}
\fancyhead[RO]{\nouppercase{\rightmark} $\cdot$\ \thepage}
\fancypagestyle{plain}{%
\fancyhead{} %
\renewcommand{\headrulewidth}{0pt}}
%------------------------------------------------------------------------

%------------------- Ορισμοί - Θεωρήματα - Διάφορα πλαίσια -----------------
\usepackage[most]{tcolorbox}
\tcbuselibrary{skins,theorems,breakable}
% ΟΡΙΣΜΟΣ
\newcounter{orismos}[chapter]
\renewcommand{\theorismos}{\thechapter.\arabic{orismos}}   
\newcommand{\orism}{\refstepcounter{orismos}{\bf\titlefont{\textcolor{maincolor}{\large{Ορισμός}\hspace{2mm}\theorismos}}}\hspace{1mm}}{}

\newenvironment{Orismos}[1]
{\begin{tcolorbox}[title=\orism:\ \  {\bf{\large\titlefont{#1}}},
breakable,
enhanced standard,
titlerule=-.2pt,
toprule=0pt, 
rightrule=0pt, 
bottomrule=0pt,
colback=white,
opacityfill=0,
left=2mm,
top=1mm,
bottom=0mm,
boxrule=0pt,
colframe=white,
borderline west={1.5mm}{0pt}{maincolor},
leftrule=2mm,
sharp corners,
coltitle=black]}
{\end{tcolorbox}}

% ΘΕΩΡΗΜΑ
\newcounter{thewrhma}[chapter]
\renewcommand{\thethewrhma}{\thechapter.\arabic{thewrhma}} 
\newcommand{\thewr}{\refstepcounter{thewrhma}{\bf\titlefont{\textcolor{secondarycolor}{\large Θεώρημα\hspace{2mm}\thethewrhma}}}\hspace{1mm}}{}

\newenvironment{Thewrhma}[1]
{\begin{tcolorbox}[title=\thewr\ \ :\ \  {\textcolor{black}{\bf{\large\titlefont{#1}}}},
breakable,
enhanced standard,
titlerule=-.2pt,
toprule=0pt, 
rightrule=0pt, 
bottomrule=0pt,
colback=white,
left=2mm,
top=1mm,
bottom=0mm,
boxrule=0pt,
colframe=white,
borderline west={1.5mm}{0pt}{secondarycolor},
leftrule=2mm,
sharp corners,
coltitle=secondarycolor]}
{\end{tcolorbox}}

% ΠΑΡΑΤΗΡΗΣΗ
\newenvironment{Parathrhsh}[1]
{\begin{tcolorbox}[title=\textbf{\faInfoCircle\ \ \ \titlefont{{\large Παρατήρηση}}},
breakable,
enhanced standard,
lifted shadow={1mm}{-2mm}{3mm}{0.3mm}{black!50!white},
colback=yellow!5!white,
boxrule=0.1pt,
colframe=yellow!80!black,
hyphenationfix=true,
fonttitle=\bfseries,
toptitle=1mm,
bottomtitle=1mm,
width=#1]}
{\end{tcolorbox}}

% ΣΗΜΕΙΩΣΗ
\newenvironment{Shmeiwsh}[1]
{\begin{tcolorbox}[title=\textbf{\faBook\ \ \ \titlefont{{\large Σημείωση}}},
breakable,
enhanced standard,
lifted shadow={1mm}{-2mm}{3mm}{0.3mm}{black!50!white},
colback=thirdcolor!5!white,
boxrule=0.1pt,
colframe=thirdcolor!80!black,
hyphenationfix=true,
fonttitle=\bfseries,
toptitle=1mm,
bottomtitle=1mm,
width=#1]}
{\end{tcolorbox}}

% ΠΡΟΣΟΧΗ
\newenvironment{Prosoxi}[1]
{\begin{tcolorbox}[title=\textbf{\faExclamationTriangle\ \ \ \titlefont{{\large Προσοχή}}},
breakable,
enhanced standard,
lifted shadow={1mm}{-2mm}{3mm}{0.3mm}{black!50!white},
colback=red!5!white,
boxrule=0.1pt,
colframe=red!80!black,
fonttitle=\bfseries,
toptitle=1mm,
bottomtitle=1mm,
width=#1]}
{\end{tcolorbox}}

% ΣΤΥΛ ΠΑΡΑΔΕΙΓΜΑΤΟΣ
\newcounter{paradeigma}[chapter]
\renewcommand{\theparadeigma}{\bf\titlefont{\thechapter.\arabic{paradeigma}}} 

\newenvironment{Paradeigma}[1]{\refstepcounter{paradeigma}\textcolor{magenta!70!black}{\textbf{\large \faPlay\ \ \titlefont{Παράδειγμα}\hspace{2mm}\theparadeigma\;:\;}\hspace{1mm}} {\titlefont{#1}}\\\bfseries\boldmath}{\mbox{}\newline\lysh\mbox{}\newline}

% ΣΤΥΛ ΛΥΣΗΣ 
\newcommand{\lysh}{\textcolor{cyan!80!black}{\titlefont{\faCheck\ ΛΥΣΗ}}}

% ΛΥΜΕΝΑ ΠΑΡΑΔΕΙΓΜΑΤΑ ΤΙΤΛΟΣ 
\newcommand{\Lymena}{\begin{center}
\begin{tikzpicture}
\path[left color=cyan!70!black,right color=cyan!80!black,middle color=cyan!80!white] (-7cm,-.6cm) rectangle (6.5cm,.6cm);
\node at (-.25cm,0) {\Large \textcolor{white}{\textbf{ΛΥΜΕΝΑ ΠΑΡΑΔΕΙΓΜΑΤΑ}}};  
\end{tikzpicture}
\end{center}}

% ΑΛΥΤΕΣ ΑΣΚΗΣΕΙΣ ΤΙΤΛΟΣ 
\newcommand{\Alyta}{
\begin{tikzpicture}[overlay,remember picture]
\path[left color=maincolor,right color=maincolor!80!black,middle color=maincolor!70!white] (0cm,.7cm) --(0cm,-.5cm) -- (7.9cm,-.5cm) -- (8.7cm,.63cm) -- (18cm,.63cm) -- (18cm,.7cm)-- cycle;
\path[bottom color=white, middle color=maincolor,top color=maincolor] (0cm,-0.5cm)--(0cm,-3cm)--(0.05cm,-3cm)--(0.05cm,-0.5cm)--cycle;
\path[bottom color=white,top color=maincolor!80!black] (18cm,0.7cm)--(18cm,-3cm)--(17.95cm,-3cm)--(17.95cm,0.7cm)--cycle;
\node at (3.8cm,.1) {\LARGE \textcolor{white}{\textbf{\faPenSquare\ \ \rule{.5mm}{5mm}\ \ \titlefont{ΑΣΚΗΣΕΙΣ - Ενότητα \thesection}}}};
\end{tikzpicture}\mbox{}\\}

\newcommand{\alyta}{
\begin{tikzpicture}[overlay,remember picture]
\path[left color=maincolor,right color=maincolor,middle color=maincolor!50!white] (0cm,0cm) --(0cm,.05cm) -- (18cm,.05cm) -- (18cm,.0cm) -- cycle;
\path[top color=white,bottom color=maincolor] (0cm,0cm)--(0cm,2cm)--(0.05cm,2cm)--(0.05cm,0cm)--cycle;
\path[top color=white,bottom color=maincolor] (18cm,0cm)--(18cm,2cm)--(17.95cm,2cm)--(17.95cm,0cm)--cycle;
\end{tikzpicture}\mbox{}\\}

\newcounter{porisma}[chapter]
\renewcommand{\theporisma}{\thechapter.\arabic{porisma}}\newcommand{\Porisma}[1]{\refstepcounter{porisma}\textcolor{black}{\textbf{ΠΟΡΙΣΜΑ\hspace{2mm}\theporisma\hspace{1mm} \MakeUppercase{#1}}}\\}{}

\newcounter{protasi}[chapter]
\renewcommand{\theprotasi}{\thechapter.\arabic{protasi}}\newcommand{\Protasi}[1]{\refstepcounter{protasi}\textcolor{black}{\textbf{ΠΡΟΤΑΣΗ\hspace{2mm}\theprotasi\hspace{1mm} \MakeUppercase{#1}}}\\}{}
%---------------------------------------------------------------------------

%------------------ Λίστες ------------------------------------
\usepackage{enumitem}
\newlist{rlist}{enumerate}{3}
\setlist[rlist]{itemsep=0mm,label=\roman*.}
\newlist{alist}{enumerate}{3}
\setlist[alist]{itemsep=0mm,label=\alph*.}
\newlist{balist}{enumerate}{3}
\setlist[balist]{itemsep=0mm,label=\bf\alph*.}
\newlist{Alist}{enumerate}{3}
\setlist[Alist]{itemsep=0mm,label=\Alph*.}
\newlist{bAlist}{enumerate}{3}
\setlist[bAlist]{itemsep=0mm,label=\bf\Alph*.}
\newlist{askhseis}{enumerate}{3}
\setlist[askhseis]{label={\Large\thesection}.\arabic*.}
\renewcommand{\textstigma}{\textsigma\texttau}
\renewcommand{\textdexiakeraia}{}
% Στυλ άσκησης
\newcounter{askhsh}[chapter]
\renewcommand{\theaskhsh}{\bf{\textit{{\Large{\thechapter}}.\arabic{askhsh}}}}   
\newcommand{\Askhsh}{\refstepcounter{askhsh}\textcolor{maincolor}{{\theaskhsh}\hspace{2mm}}}{}
\setlist[itemize]{itemsep=0mm}
%---------------------------------------------------------------

%----------------- Βιβλιογραφία --------------------------------
\usepackage[backend=biber,style=alphabetic,sorting=ynt]{biblatex}
%-----------------------------------------------------------------

%----------------- Πίνακες ------------------------------------
\usepackage{tabularx,longtable,tabularray,capt-of,caption}
\DeclareTblrTemplate{caption}{nocaptemplate}{}
\DeclareTblrTemplate{capcont}{nocaptemplate}{}
\DeclareTblrTemplate{contfoot}{nocaptemplate}{}
\NewTblrTheme{mytabletheme}{
\SetTblrTemplate{caption}{nocaptemplate}{}
\SetTblrTemplate{capcont}{nocaptemplate}{}
\SetTblrTemplate{contfoot}{nocaptemplate}{}}

\NewTblrEnviron{mytblr}
\SetTblrStyle{firsthead}{font=\bfseries}
\SetTblrStyle{firstfoot}{fg=red2}
\SetTblrOuter[mytblr]{theme=mytabletheme}
\SetTblrInner[mytblr]{
rowspec={t{7mm}},
columns = {c},
width = 0.85\linewidth,
row{odd} = {bg=maincolor!20!white,fg=black,ht=8mm},
row{even} = {bg=gray!10!white,fg=black,ht=8mm},
hlines={white},
vlines={white},
row{1} = {bg=maincolor, fg=white, font=\bfseries\fontfamily{maksf}},
rowhead = 1,
hline{2} = {.7mm}, % midrule  
}

\captionsetup[figure]{format=hang,labelsep=period,name={\titlefont{\textbf{Σχήμα}}}}
\renewcommand\thefigure{{\bf\titlefont{\thechapter.\arabic{figure}}}}
%-------------------------------------------------------------------

%-------------- Περιεχόμνα ------------------------------------------
\usepackage{sidenotes}  % <-- To place content in the margin
\usepackage{etoc}       % <-- ΑΛΛΑΓΗ: Αντικαταστάθηκε το minitoc με το etoc
\etocsettocdepth{1}
% --- Etoc Setup ---
% Το etoc είναι πιο σύγχρονο και συνήθως χρειάζεται μόνο 2 μεταγλωττίσεις.
% Προσαρμόζουμε την εμφάνιση του τοπικού πίνακα περιεχομένων.
\etocsetstyle{section}{}{}{\hspace{2mm}
\makebox[0pt][r]{\textcolor{secondarycolor!80!black}{\faStop}\ \ 
\titlefont{\textbf{\etocnumber}}\hspace{1em}}%
\titlefont{\etocname}\par}{}

\etocsettocstyle {
\noindent\titlefont{\large \textbf{\hspace{2mm}\contentsname\ \ κεφαλαίου}}\par
                  \thispagestyle{empty}%
                  \noindent{\color{maincolor}\textbf{\dotfill}}\par\vspace{1ex}%
                  \leftskip1.5cm\parindent0pt\def\TEMP{0}}{}
\etocsetstyle{subsection}
  {}
  {\etocnumber.\enspace}
  {}
  {}
\contentsmargin{0cm}
\titlecontents{part}[-1pc]
{\addvspace{10pt}%
\bf\Large ΜΕΡΟΣ\quad }%
{}
{}
{\;\dotfill\;\normalsize\ Σελίδα}%
%------------------------------------------
\titlecontents{chapter}[0pc]
{\addvspace{30pt}%
\begin{tikzpicture}[remember picture, overlay]%
\draw[fill=black,draw=black] (-.3,.5) rectangle (3.7,1.1); %
\pgftext[left,x=0cm,y=0.75cm]{\color{white}\sc\Large\bfseries Κεφάλαιο\ \thecontentslabel};%
\end{tikzpicture}\footnotesize}%
{}
{}
{\hspace*{-2.3em}\hfill\normalsize Σελίδα \thecontentspage}%
\titlecontents{section}[2.4pc]
{\addvspace{1pt}}
{\contentslabel[\thecontentslabel]{2pc}}
{}
{\;\dotfill\;\small \thecontentspage}
[]
\titlecontents*{subsection}[4pc]
{\addvspace{-1pt}\small}
{}
{}
{\ --- \small\thecontentspage}
[ \textbullet\ ][]

\makeatletter
\renewcommand{\tableofcontents}{%
\chapter*{%
\vspace*{-20\p@}%
\begin{tikzpicture}[remember picture, overlay]%
\pgftext[right,x=15cm,y=0.2cm]{\Huge\sc\bfseries \contentsname};%
\draw[fill=black,draw=black] (12.5,-.75) rectangle (15.5,1);%
\clip (12.5,-.75) rectangle (18,1);
\pgftext[right,x=15cm,y=0.2cm]{\color{white}\Huge\bfseries \contentsname};%
\end{tikzpicture}}%
\@starttoc{toc}}
\makeatother
%-------------- ΣΤΥΛ ΚΕΦΑΛΑΙΟΥ-------------------------------------------------

% ΔΙΟΡΘΩΣΗ: Νέες εντολές για τη διαχείριση της εικόνας του κεφαλαίου
\newcommand{\chapterimagefile}{example-image-b} % Ορίζουμε μια προεπιλεγμένη εικόνα
\newcommand{\setchapterimage}[1]{\renewcommand{\chapterimagefile}{#1}} % Εντολή για την αλλαγή της εικόνας

\definecolor{chapterbg}{HTML}{125e6c}
\newcommand*\chapterlabel{}
\newcommand{\fancychapter}{%
\titleformat{\chapter}
{\normalfont\huge }
{\gdef\chapterlabel{\thechapter\ }}{0pt}
{\begin{tikzpicture}[remember picture, overlay]
    \node[anchor=north, inner sep=0pt, name=headerimage,outer sep=0pt] at (current page.north) {
        \includegraphics[width=\paperwidth, trim={0 0pt 0 0}, clip]{\chapterimagefile}
    };
    \node[
        name=chnumber,fill=maincolor!50!black, opacity=0.7,text=white,anchor=north west,outer sep=0pt,
        rotate=90,font=\sffamily\bfseries,minimum height=1.5cm,minimum width=3.5cm,inner sep=20pt
    ] at (headerimage.south west) {\titlefont{Κεφάλαιο {\Huge \thechapter}} };
    \node[
        name=chtitle,anchor=north west,fill=maincolor!50!black,outer sep=0pt,
        minimum width=\paperwidth-2cm,
        minimum height=2.5cm,text=white,font=\sffamily\bfseries
    ] at (chnumber.south west) {\titlefont{
\begin{minipage}{\paperwidth-4cm}
##1
\end{minipage}
}};
\fill[maincolor!30!black,anchor=north] (headerimage.south west) rectangle (chtitle.south west);
    \fill[cyan,anchor=north] (chtitle.north west) rectangle ($(chtitle.north east)+(0,3mm)$);
\end{tikzpicture}
\vspace{9cm}
}
%\titlespacing*{\chapter}{0pt}{20pt}{100pt}
}
\apptocmd{\mainmatter}{\fancychapter}{}{}

\titleformat{\section}{\Large}{\titlefont{\textbf{\thesection}}}{10pt}{\Large\titlefont{\textbf{#1}}}

\setlength{\columnsep}{5mm}
\titleformat{\paragraph}
{\large}%
{}{0em}%
{\textcolor{maincolor}{\faSquare\ \ \titlefont{\bmath{#1}}}}
\setlength{\parindent}{0pt}
\titlespacing{\paragraph}{0mm}{2mm}{2mm}

\newcommand{\eng}[1]{\selectlanguage{english}#1\selectlanguage{greek}}
\newcommand{\tss}[1]{\textsuperscript{#1}}
\newcommand{\tssL}[1]{\MakeLowercase{\textsuperscript{#1}}}

% --- Command for the Sidenote TOC ---
% ΑΛΛΑΓΗ: Η εντολή χρησιμοποιεί πλέον την εντολή \localtableofcontents του etoc
\usepackage{tikzpagenodes,eso-pic}
\newcommand{\chaptertoc}{%
\AddToShipoutPictureFG*{%
    \begin{tikzpicture}[overlay]
      % Βρίσκουμε τις πάνω και κάτω δεξιά γωνίες του κυρίως κειμένου
      \coordinate (NE) at ($(current page text area.north east)+(-59mm,-14.5cm)$);
      \coordinate (SE) at ($(current page text area.south east)+(-59mm,0)$);
      % Σχεδιάζουμε τη γραμμή ανάμεσα στο κείμενο και το περιθώριο
      \draw[magenta!80!black] ([xshift=0.5\marginparsep]NE) -- ([xshift=0.5\marginparsep]SE);
    \end{tikzpicture}%
  }%
  \marginpar{%
\vspace*{8mm}
    \begin{minipage}{\dimexpr\marginparwidth+4mm}
    \begin{flushleft}
\localtableofcontents
\end{flushleft}
    \end{minipage}%
  }%
}





\begin{document}
\pagestyle{empty}
\frontmatter
\newgeometry{inner=2.00cm, top=3.00cm, bottom=2.00cm,outer=1.50cm,textwidth=17.5cm}
\tableofcontents
\restoregeometry % restores the geometry
\mainmatter
\pagestyle{fancy}
\setchapterimage{./images/9.png}
\chapter{Διαφορικές εξισώσεις και διαφορικά συστήματα.\\Προβλήματα αρχικών τιμών. Ύπαρξη και μονοσήμαντο λύσεων}
\chaptertoc
\section{Διαφορικές εξισώσεις και διαφορικά συστήματα.\\Προβλήματα αρχικών τιμών}
Στο Κεφάλαιο αυτό θα γίνει μια εισαγωγή στις διαφορικές εξισώσεις, τα διαφορικά συστήματα και τα προβλήματα αρχικών τιμών (εδάφιο 1) και θα μελετηθεί το πρόβλημα της ύπαρξης και του μονοσήμαντου λύσεων προβλημάτων αρχικών τιμών (εδάφιο 2).

\subsection{Διαφορικές εξισώσεις και διαφορικά συστήματα. Προβλήματα αρχικών τιμών}

Στο εδάφιο αυτό θα εισαγάγουμε τις έννοιες της διαφορικής εξίσωσης, του διαφορικού συστήματος και του προβλήματος αρχικών τιμών.

\subsubsection{Διαφορικές εξισώσεις και διαφορικά συστήματα. Προβλήματα αρχικών τιμών}

Ας είναι $f$ μια $n$-διάστατη διανυσματική συνάρτηση ορισμένη σ' ένα υποσύνολο $D$ του καρτεσιανού χώρου της πραγματικής ευθείας με τον χώρο των $n$-διάστατων διανυσμάτων. Μια εξίσωση της μορφής
\begin{equation}
y' = f(x,y),
\end{equation}
όπου $y' = \frac{d}{\d x}y$, λέμε ότι είναι μια $n$-διάστατη διανυσματική διαφορική εξίσωση με άγνωστη συνάρτηση την $y$ και ανεξάρτητη μεταβλητή την $x$. Αν $n=1$, τότε λέμε ότι η (1) είναι μια (βαθμωτή) διαφορική εξίσωση πρώτης τάξης. Ας είναι $I$ ένα διάστημα της πραγματικής ευθείας. Μια $n$-διάστατη διανυσματική συνάρτηση $y$ λέγεται λύση της διαφορικής εξίσωσης (1) στο διάστημα $I$ αν και μόνο αν η $y$ είναι παραγωγίσιμη στο $I$ και επιπλέον για όλα τα $x \in I$ είναι $(x,y(x)) \in D$ και $y'(x) = f(x,y(x))$. Επίσης, αν $x_0 \in I$ και $(x_0,y_0) \in D$, τότε μια λύση $y$ της (1) στο διάστημα $I$ τέτοια ώστε να πληροί την αρχική συνθήκη
\begin{equation}
y(x_0) = y_0
\end{equation}
λέμε ότι είναι μια λύση στο $I$ του προβλήματος αρχικών τιμών (1)-(2).

Ας είναι τώρα $f_1, \dots, f_n$ $n$ συναρτήσεις ορισμένες σ' ένα υποσύνολο $D$ του καρτεσιανού χώρου της πραγματικής ευθείας με τον χώρο των $n$-διάστατων διανυσμάτων. Ένα σύστημα της μορφής
\begin{equation}
\begin{cdcases}
y_1' = f_1(x,y_1, \dots, y_n) \\
\vdots \\
y_n' = f_n(x,y_1, \dots, y_n)
\end{cdcases}
\end{equation}
θα λέμε ότι είναι ένα $n$-διάστατο διαφορικό σύστημα με άγνωστες συναρτήσεις τις $y_1, \dots, y_n$ και ανεξάρτητη μεταβλητή $x$. Επίσης, θα λέμε ότι το διαφορικό σύστημα (3) είναι πρώτης τάξης. Ας είναι $I$ ένα διάστημα. Μια $n$-άδα συναρτήσεων $y_1, \dots, y_n$ λέγεται λύση στο $I$ του διαφορικού συστήματος (3) αν και μόνο αν οι συναρτήσεις $y_1, \dots, y_n$ είναι παραγωγίσιμες στο $I$ και για όλα τα $x \in I$ είναι $(x,y_1(x), \dots, y_n(x)) \in D$ και
\[
y_k'(x) = f_k(x,y_1(x), \dots, y_n(x)) \quad (k=1,\dots,n).
\]
Ακόμα, αν $(x_0,y_{10}, \dots, y_{n0}) \in D$ και $x_0 \in I$, τότε μια λύση στο διάστημα $I$ του διαφορικού συστήματος (3) που πληροί την αρχική συνθήκη
\begin{equation}
y_1(x_0)=y_{10}, \dots, y_n(x_0)=y_{n0}
\end{equation}
θα λέγεται λύση στο $I$ του προβλήματος αρχικών τιμών (3)-(4). Αν θέσουμε
\[
y = \begin{pmatrix} y_1 \\ \vdots \\ y_n \end{pmatrix}, \quad \text{και} \quad f = \begin{pmatrix} f_1 \\ \vdots \\ f_n \end{pmatrix},
\]
τότε το διαφορικό σύστημα (3) παίρνει τη μορφή
\begin{equation*} \label{eq:3prime}
y' = f(x,y), \tag{3'}
\end{equation*}
Επίσης, θέτοντας
\[
y_0 = \begin{pmatrix} y_{10} \\ \vdots \\ y_{n0} \end{pmatrix},
\]
η αρχική συνθήκη (4) γράφεται
\begin{equation*} \label{eq:4prime}
y(x_0) = y_0. \tag{4'}
\end{equation*}
Έτσι, κάθε $n$-διάστατο διαφορικό σύστημα μπορεί να γραφεί στη μορφή μιας $n$-διάστατης διανυσματικής διαφορικής εξίσωσης. Επίσης, ένα πρόβλημα αρχικών τιμών για ένα $n$-διάστατο διαφορικό σύστημα γράφεται στη μορφή ενός προβλήματος αρχικών τιμών για μια $n$-διάστατη διανυσματική διαφορική εξίσωση.

Μια ενδιαφέρουσα ειδική μορφή $n$-διάστατων διαφορικών συστημάτων είναι τα γραμμικά διαφορικά συστήματα
\begin{equation}
\begin{cdcases}
y_1' = a_{11}y_1 + \dots + a_{1n}y_n + b_1 \\
\vdots \\
y_n' = a_{n1}y_1 + \dots + a_{nn}y_n + b_n
\end{cdcases}
\end{equation}
όπου $a_{ij}$ $(i,j=1,\dots,n)$ και $b_i$ $(i=1,\dots,n)$ είναι συναρτήσεις ορισμένες σ' ένα διάστημα $I$ της πραγματικής ευθείας. Θέτοντας
\[
y = \begin{pmatrix} y_1 \\ \vdots \\ y_n \end{pmatrix}, \quad A = \begin{pmatrix} a_{11} & \dots & a_{1n} \\ \vdots & \ddots & \vdots \\ a_{n1} & \dots & a_{nn} \end{pmatrix}, \quad \text{και} \quad b = \begin{pmatrix} b_1 \\ \vdots \\ b_n \end{pmatrix},
\]
το γραμμικό διαφορικό σύστημα γράφεται
\begin{equation*} \label{eq:5prime}
y' = Ay+b. \tag{5'}
\end{equation*}
Επίσης, αν $x_0 \in I$ και $y_{10}, \dots, y_{n0}$ είναι σταθερές, το πρόβλημα αρχικών τιμών (5)-(4) γράφεται (5')-(4') για
\[
y_0 = \begin{pmatrix} y_{10} \\ \vdots \\ y_{n0} \end{pmatrix}.
\]

Ας θεωρήσουμε μια συνάρτηση $f$ ορισμένη σ' ένα υποσύνολο $D$ του καρτεσιανού χώρου της πραγματικής ευθείας με τον χώρο των $n$-διάστατων διανυσμάτων. Τότε μια εξίσωση της μορφής
\begin{equation}
y^{(n)} = f(x,y,y', \dots, y^{(n-1)}),
\end{equation}
όπου $y^{(k)} = \frac{d^k}{\d x^k}$ $(k=1, \dots, n)$, θα λέμε ότι είναι μια (βαθμωτή) διαφορική εξίσωση $n$-τάξης. Η $y$ είναι η άγνωστη συνάρτηση αυτής και $x$ είναι η ανεξάρτητη μεταβλητή της. Ας είναι $I$ ένα διάστημα. Μια συνάρτηση στο $I$ που έχει παράγωγο και είναι τέτοια ώστε για κάθε $x \in I$
$(x, y(x), y'(x), \dots, y^{(n-1)}(x)) \in D$ και $y^{(n)}(x) = f(x, y(x), y'(x), \dots, y^{(n-1)}(x))$
θα λέμε ότι είναι μια λύση στο $I$ της διαφορικής εξίσωσης (6). Αν $(x_0, y_0, \dots, y_{n0}^{(n-1)}) \in D$ με $x_0 \in I$, μια λύση $y$ της (6) στο $I$ που πληροί τις αρχικές συνθήκες
\begin{equation}
y(x_0)=y_0, y'(x_0)=y_{20}, \dots, y^{(n-1)}(x_0)=y_{n0}
\end{equation}
θα λέμε ότι είναι μια λύση στο $I$ του προβλήματος αρχικών τιμών (6)-(7).

Αν θέσουμε $y=y_1, y'=y_2, \dots, y^{(n-1)}=y_n$, τότε η διαφορική εξίσωση (6) ανάγεται στο $n$-διάστατο διαφορικό σύστημα
\begin{equation*} \label{eq:6prime}
\begin{cdcases}
y_1' = y_2 \\
\vdots \\
y_{n-1}' = y_n \\
y_n' = f(x,y_1, \dots, y_n)
\end{cdcases} \tag{6'}
\end{equation*}
και οι αρχικές συνθήκες (7) ανάγονται στην αρχική συνθήκη
\begin{equation*} \label{eq:7prime}
y_1(x_0)=y_{10}, \dots, y_n(x_0)=y_{n0} \tag{7'}
\end{equation*}
για το διαφορικό σύστημα (6'). Το διαφορικό σύστημα (6') μπορεί να γραφεί ως μια $n$-διάστατη διανυσματική διαφορική εξίσωση. Τελικά, η $n$-τάξης διαφορική εξίσωση (6) ανάγεται σε μια $n$-διάστατη διανυσματική διαφορική εξίσωση και το πρόβλημα αρχικών τιμών (6)-(7) μπορεί ν' αναχθεί σ' ένα πρόβλημα αρχικών τιμών για μια $n$-διάστατη διανυσματική διαφορική εξίσωση.

Μια ενδιαφέρουσα ειδική κατηγορία διαφορικών εξισώσεων $n$-τάξης είναι οι γραμμικές διαφορικές εξισώσεις $n$-τάξης της μορφής
\begin{equation}
a_n y^{(n)} + a_{n-1} y^{(n-1)} + \dots + a_1 y' + a_0 y = B,
\end{equation}
όπου $a_i$ ($i=0,1,\dots,n$), $n$ και $b$ είναι συναρτήσεις ορισμένες σ' ένα διάστημα $I$ και $a_n(x) \neq 0$ για όλα τα $x \in I$. Αν $x_0 \in I$ και $y_{10}, \dots, y_{n0}$ είναι τυχούσες σταθερές, τότε μπορούμε να θεωρήσουμε το πρόβλημα αρχικών τιμών (8)-(7). Θέτοντας
\[
y = \begin{pmatrix} y \\ y' \\ \vdots \\ y^{(n-1)} \end{pmatrix}, \quad B = \begin{pmatrix} 0 \\ \vdots \\ 0 \\ b \end{pmatrix}, \quad A = \begin{pmatrix} 0 & 1 & 0 & \dots & 0 \\ 0 & 0 & 1 & \dots & 0 \\ \vdots & \vdots & \vdots & \ddots & \vdots \\ 0 & 0 & 0 & \dots & 1 \\ -\frac{a_0}{a_n} & -\frac{a_1}{a_n} & -\frac{a_2}{a_n} & \dots & -\frac{a_{n-1}}{a_n} \end{pmatrix}.
\]
η γραμμική διαφορική εξίσωση (8) ανάγεται στο γραμμικό διαφορικό σύστημα
\begin{equation*} \label{eq:8prime}
Y' = AY+B \tag{8'}
\end{equation*}
και οι αρχικές συνθήκες (7) μπορούν να γραφούν υπό τη μορφή μιας αρχικής συνθήκης για το γραμμικό διαφορικό σύστημα (8'), δηλαδή
\begin{equation*} \label{eq:7primeprime}
Y(x_0) = Y_0, \tag{7''}
\end{equation*}
όπου
\[
Y_0 = \begin{pmatrix} y_{10} \\ \vdots \\ y_{n0} \end{pmatrix}.
\]

\subsection{Παραδείγματα}

\begin{Paradeigma}{}
Κάθενα απ' τα παρακάτω προβλήματα αρχικών τιμών να γραφεί ως ένα πρόβλημα αρχικών τιμών για μια διανυσματική διαφορική εξίσωση:
\begin{rlist}
\item $y''=y_2^2-1, \ y_2'=x+y_1^2; \ y_1(0)=1, \ y_2(0)=-2.$
\item $y_1'' = 5y_1+e^x y_2, \ y_2' = xy_1+y_2'^2+\sin x; y_1(0)=1, \ y_1'(0)=7, \ y_2(0)=-1.$
\item $e^x y''' - 7x(y')^2 + y^3 + \cos x = 0; \ y(0)=y'(0)=1, \ y''(0)=-2.$
\item $y_1' = y_1+y_3, \ y_2' = e^x y_1 + y_2 - (\sin x)y_3, \ y_3' = y_1; \ y_1(1)=0, \ y_2(1)=2, \ y_3(1)=7.$
\item $(x^2+1)y'' - 5xy' + x^2 y = e^x, \ y(2)=0, \ y'(2)=1, \ y''(2)=-3.$
\item $y_1''-y_1=0, \ y_2'+xy_1'-y_2=e^x; \ y_1(0)=y_1'(0)=1, \ y_2(0)=7, \ y_2'(0)=3.$
\item $y_1''+y_1=e^{-x}, \ y_2'-5y_2+y_3', \ y_3'-y_3=y_2; \ y_1(0)=y_1'(0)=1, \ y_2(0)=y_3(0)=2.$
\end{rlist}
\end{Paradeigma}

\begin{rlist}
\item θέτοντας
\[
Y = \begin{pmatrix} Y_1 \\ Y_2 \end{pmatrix}, \quad Y_0 = \begin{pmatrix} 1 \\ -2 \end{pmatrix} \quad \text{και} \quad f(x,Y) = \begin{pmatrix} Y_2^2+1 \\ x+Y_1^2 \end{pmatrix},
\]
παίρνουμε το πρόβλημα αρχικών τιμών
\[
y' = f(x,y), \ y(0)=y_0.
\]
\item θέτουμε $y_1=u_1, \ y_1'=u_2$ και $y_2=u_3$, οπότε το πρόβλημα αρχικών τιμών μετασχηματίζεται στο
\[
u_1'=u_2, \ u_2'=5u_1+e^x u_3, \ u_3'=xu_1+u_2^2+\sin x; \ u_1(0)=1, \ u_2(0)=7, \ u_3(0)=-1,
\]
το οποίο μπορεί να γραφεί ως εξής
\[
u' = f(x,u), \ u(0)=u_0
\]
αρκεί να θέσουμε
\[
u = \begin{pmatrix} u_1 \\ u_2 \\ u_3 \end{pmatrix}, \ u_0 = \begin{pmatrix} 1 \\ 7 \\ -1 \end{pmatrix} \quad \text{και} \quad f(x,u) = \begin{pmatrix} u_2 \\ 5u_1+e^x u_3 \\ xu_1+u_2^2+\sin x \end{pmatrix}.
\]
\item θέτουμε $y=Y_1, \ y'=Y_2, \ y''=Y_3$ και παίρνουμε
\[
Y_1' = Y_2, \ Y_2' = Y_3, \ Y_3' = \frac{1}{e^x} (-\cos x - y_1^3 + 7xy_2^2); \ Y_1(0)=Y_2(0)=1, \ Y_3(0)=-2
\]
οπότε έχουμε
\[
Y' = f(x,Y); \ Y(0)=Y_0
\]
με
\[
Y = \begin{pmatrix} Y_1 \\ Y_2 \\ Y_3 \end{pmatrix} = \begin{pmatrix} y \\ y' \\ y'' \end{pmatrix}, \quad Y_0 = \begin{pmatrix} 1 \\ 1 \\ -2 \end{pmatrix} \quad \text{και} \quad f(x,Y) = \begin{pmatrix} Y_2 \\ Y_3 \\ \frac{1}{e^x}(-\cos x - Y_1^3+7xY_2^2) \end{pmatrix}.
\]
\item Για
\[
Y = \begin{pmatrix} y_1 \\ y_2 \\ y_3 \end{pmatrix}, \quad Y_0 = \begin{pmatrix} 0 \\ 2 \\ 7 \end{pmatrix} \quad \text{και} \quad A(x) = \begin{pmatrix} 1 & 0 & 1 \\ e^x & 1 & -\sin x \\ 1 & 0 & 0 \end{pmatrix}
\]
παίρνουμε το πρόβλημα αρχικών τιμών
\[
Y' = A(x)Y; \ Y(1)=Y_0.
\]
\item Αυτό γράφεται
\[
Y' = A(x)Y+B(x); Y(2)=Y_0,
\]
όπου
\[
Y = \begin{pmatrix} y_1 \\ y_2 \\ y_3 \end{pmatrix} = \begin{pmatrix} y \\ y' \\ y'' \end{pmatrix}, \quad Y_0 = \begin{pmatrix} 0 \\ 1 \\ -3 \end{pmatrix}, \quad A(x) = \begin{pmatrix} 0 & 1 & 0 \\ 0 & 0 & 1 \\ \frac{-5x^2}{x^2+1} & \frac{-5x}{x^2+1} & 0 \end{pmatrix} \quad \text{και}
\]
\[
B(x) = \begin{pmatrix} 0 \\ 0 \\ \frac{e^x}{x^2+1} \end{pmatrix}.
\]
\item θέτουμε $Y_1=u_1, Y_1'=u_2, Y_2=u_3, Y_2'=u_4$, οπότε
\[
u_1'=u_2, \ u_2'=u_1, \ u_3'=u_4, \ u_4'=-xu_2+u_3+e^x; \ u_1(0)=u_2(0)=1, \ u_3(0)=7, \ u_4(0)=3.
\]
Αυτό γράφεται
\[
u' = A(x)u+b(x); \ u(0)=u_0,
\]
όπου
\[
u = \begin{pmatrix} u_1 \\ u_2 \\ u_3 \\ u_4 \end{pmatrix}, \ u_0 = \begin{pmatrix} 1 \\ 1 \\ 7 \\ 3 \end{pmatrix}, \ A(x) = \begin{pmatrix} 0 & 1 & 0 & 0 \\ 1 & 0 & 0 & 0 \\ 0 & 0 & 0 & 1 \\ 0 & 0 & 1 & -x \end{pmatrix} \quad \text{και} \quad b(x) = \begin{pmatrix} 0 \\ 0 \\ 0 \\ e^x \end{pmatrix}.
\]
\item Για $y_1=u_1, y_1'=u_2, y_2=u_3, y_3=u_4$, παίρνουμε
\[
u_1'=u_2, \ u_2'=-u_1+e^{-x}, \ u_3'=5u_3+u_4, \ u_4'=-u_3; \ u_1(0)=u_2(0)=1, \ u_3(0)=u_4(0)=2
\]
ή ακόμα
\[
u' = Au+b(x); \ u(0)=u_0,
\]
όπου
\[
u = \begin{pmatrix} u_1 \\ u_2 \\ u_3 \\ u_4 \end{pmatrix}, \ u_0 = \begin{pmatrix} 1 \\ 1 \\ 2 \\ 2 \end{pmatrix}, \ A = \begin{pmatrix} 0 & 1 & 0 & 0 \\ -1 & 0 & 0 & 0 \\ 0 & 0 & 5 & 1 \\ 0 & 0 & -1 & 0 \end{pmatrix} \quad \text{και} \quad b(x) = \begin{pmatrix} 0 \\ e^{-x} \\ 0 \\ 0 \end{pmatrix}.
\]
\end{rlist}

\begin{Paradeigma}{Ν' αποδειχθεί ότι:}
\begin{rlist}
\item Το πρόβλημα αρχικών τιμών
με
\[
y' = f(x,y); \ y(1)=y_0
\]
\[
f(x,y) = \begin{pmatrix} y_1^2+y_2^2-\log^2 x \\ xe^{-x}y_1-x+1 \end{pmatrix}, \quad x>0 \quad \text{και} \quad y_0 = \begin{pmatrix} e \\ 0 \end{pmatrix}
\]
έχει τη λύση
\[
y(x) = \begin{pmatrix} e^x \\ \log x \end{pmatrix}, \quad x>0.
\]
\item Το πρόβλημα αρχικών τιμών
\[
y'' = -\frac{2x}{x^2+1}y'; \ y(0)=1, \ y'(0)=3
\]
έχει τη λύση $y(x)=x^3+3x+1, \ x \in \mathbb{R}$.
\end{rlist}
\end{Paradeigma}

\lysh
\begin{rlist}
\item Είναι
\[
y(1) = \begin{pmatrix} e^1 \\ \log 1 \end{pmatrix} = \begin{pmatrix} e \\ 0 \end{pmatrix} = y_0
\]
και για όλα τα $x>0$ έχουμε
\[
f(x,y(x)) = \begin{pmatrix} e^{2x}+\log^2 x-\log^2 x \\ xe^{-x}e^x-x+1 \end{pmatrix} = \begin{pmatrix} e^{2x} \\ 1 \end{pmatrix} = y'(x).
\]
\item Είναι $y(0)=1$ και $y'(0)=3$. Επίσης, για κάθε $x \in \mathbb{R}$ είναι $y'(x)=3x^2+3$ και $y''(x)=6x$ και επομένως
\[
y''(x) = -\frac{2x}{x^2+1}y'(x) \quad \text{για κάθε } x \in \mathbb{R}.
\]
\end{rlist}

\subsection{Ασκήσεις}

1.  Τα παρακάτω προβλήματα αρχικών τιμών να μετασχηματισθούν σ' άλλα που ν' αναφέρονται σε διανυσματικές διαφορικές εξισώσεις:
\begin{rlist}
\item $y_1' = 2y_1^2+7xy_2, \ y_2'=-y_1; \ y_1(0)=-1, \ y_2(0)=1.$
\item $y_1'' = e^{-x}y_1'+y_1 = \sin x, \ y_2' = e^x y_1; \ y_1(1)=y_1'(1)=0, \ y_1''(1)=y_2(1)=y_2'(1)=2.$
\item $y_1''' + 4\cos y_1 = 0, \ y_2' = e^x y_1; \ y_1(1)=2, \ y_1'(1)=0, \ y_2(1)=-3.$
\item $y_1'' = 4y_1' - y_2, \ y_2' = e^x y_1; \ y_1(2)=0, \ y_1'(2)=-1, \ y_2(2)=0.$
\item $e^x y'' - y_2^2 = 0, \ y_2' = e^x y_1' + xy_2; \ y_1(0)=y_1'(0)=y_1''(0)=1, \ y_2(0)=2.$
\end{rlist}

2. Ν' αποδειχθεί ότι:
\begin{rlist}
\item Η συνάρτηση
\[
y(x) = \begin{pmatrix} \frac{1}{2} - \frac{1}{2}e^{2x} \\ \frac{1}{2}e^{2x} + \frac{2}{3}e^{-x} \\ \frac{1}{3} - \frac{1}{6}e^{2x} - \frac{1}{6}e^{-x} \end{pmatrix}, \ x \in \mathbb{R}
\]
είναι μια λύση του προβλήματος αρχικών τιμών
\[
y' = \begin{pmatrix} 1 & -2 & -1 \\ -1 & 1 & 1 \\ 1 & 0 & -1 \end{pmatrix}y, \ y(0) = \begin{pmatrix} 0 \\ 1 \\ -1 \end{pmatrix}.
\]
\item Η συνάρτηση
\[
y(x) = -\log(1-x), \ x<1
\]
είναι μια λύση του προβλήματος αρχικών τιμών
\[
y''=e^{2y}; \ y(0)=0, \ y'(0)=1.
\]
\end{rlist}

\section{Ύπαρξη και μονοσήμαντο λύσεων - Προβλήματα αρχικών τιμών}
Στο εδάφιο αυτό θ' ασχοληθούμε με την ύπαρξη και το μονοσήμαντο λύσεων προβλημάτων αρχικών τιμών. Θα εξετάσουμε τη γενική περίπτωση προβλημάτων αρχικών τιμών που ανάγονται σε διανυσματικές διαφορικές εξισώσεις και θα δώσουμε τρία συμπεράσματα (θεωρήματα 1,2 και 3) για την ύπαρξη και το μονοσήμαντο λύσεων αυτών. Στα συμπεράσματα αυτά η συνάρτηση που ορίζει τη διαφορική εξίσωση θα υποτίθεται ότι πληροί τη συνθήκη του Lipschitz σε κατάλληλο υποσύνολο του πεδίου ορισμού της. Θα εξετασθούν ακόμα οι ειδικές περιπτώσεις προβλημάτων αρχικών τιμών για γραμμικά διαφορικά συστήματα (θεώρημα 4) καθώς και για γραμμικές διαφορικές εξισώσεις (θεώρημα 5).
\subsection{Η συνθήκη του \eng{Lipschitz}}

Πριν προχωρήσουμε, θ' αναφέρουμε ότι οι τρεις πιο συνηθισμένες στάθμες ενός n-διάστατου διανύσματος c με συντεταγμένες $c_1, \dots, c_n$ είναι
\[
\sup_{i=1,\dots,n} |c_i|, \quad \sum_{i=1}^n |c_i| \quad \text{και} \quad \left[\sum_{i=1}^n |c_i|^2\right]^{1/2} \quad \text{(Ευκλείδεια στάθμη)}.
\]
Στο παρακάτω, με $|c|$ θα παριστάνουμε μια οποιαδήποτε στάθμη του n-διάστατου διανύσματος c. Εξάλλου, στον χώρο των n-διάστατων διανυσμάτων όλες οι στάθμες είναι ισοδύναμες με την έννοια ότι, αν $|\cdot|_1$ και $|\cdot|_2$ είναι δύο στάθμες στον χώρο αυτόν, τότε υπάρχουν σταθερές $\alpha>0$ και $\beta>0$ έτσι ώστε για όλα τα n-διάστατα διανύσματα c να ισχύει
\[
\alpha |c|_1 \le |c|_2 \le \beta |c|_1.
\]
Θα σημειώσουμε ακόμα ότι, αν h είναι μια συνεχής n-διάστατη διανυσματική συνάρτηση σ' ένα συμπαγές διάστημα $[\alpha, \beta]$, ισχύει
\[
\left|\int_a^\beta h(t)dt\right| \le \int_a^\beta |h(t)|dt.
\]

Ας είναι g μια n-διάστατη διανυσματική συνάρτηση ορισμένη σ' ένα υποσύνολο D του καρτεσιανού χώρου της πραγματικής ευθείας με τον χώρο των n-διάστατων διανυσμάτων.

Αν Ε είναι ένα υποσύνολο του D, λέμε ότι η συνάρτηση g πληροί τη συνθήκη του Lipschitz στο Ε αν και μόνο αν υπάρχει μια μη αρνητική σταθερά Κ έτσι ώστε
\begin{equation} \label{eq:lipschitz}
|g(x,y_1)-g(x,y_2)| \le K|y_1-y_2| \quad \text{για όλα τα } (x,y_1), (x,y_2) \text{ στο } E, \tag{*}
\end{equation}
και λέμε ότι η g πληροί τη συνθήκη του Lipschitz με σταθερά $K \ge 0$ στο Ε αν και μόνο αν η \eqref{eq:lipschitz} ισχύει. Όταν η g πληροί τη συνθήκη του Lipschitz (με σταθερά $K \ge 0$) στο πεδίο ορισμού της, τότε θα λέμε ότι η g πληροί τη συνθήκη του Lipschitz (με σταθερά $K \ge 0$).

Ας θεωρήσουμε τώρα ένα σημείο $(x_0,y_0) \in D$. Έχουμε τότε τα παρακάτω δύο συμπεράσματα:
\begin{rlist}
\item Ας είναι $\alpha$ και b δύο θετικοί αριθμοί έτσι ώστε
το ορθογώνιο
\[
R = \{ (x,y) : |x-x_0| \le a, \ |y-y_0| \le b \} \subseteq D_g
\]
και ας υποθέσουμε ότι οι μερικές παράγωγοι $\partial g / \partial y_k$ ($k=1,\dots,n$) υπάρχουν και είναι συνεχείς στο R. Ακόμα, ας είναι
\[
K = \max_{(x,y) \in R \atop k=1,\dots,n} \left| \frac{\partial g}{\partial y_k}(x,y) \right|.
\]
Τότε η συνάρτηση g πληροί τη συνθήκη του Lipschitz με σταθερά $K \ge 0$ στο R.

\item Ας είναι S μια θετική σταθερά τέτοια ώστε
\[
S = \{ (x,y) : |x-x_0| \le a, \ y \text{ αυθαίρετο} \} \subseteq D_g
\]
και ας υποθέσουμε ότι οι μερικές παράγωγοι $\partial g / \partial y_k$ ($k=1,\dots,n$) υπάρχουν και είναι συνεχείς και φραγμένες στο S. Ακόμα, ας είναι
\[
K = \sup_{(x,y) \in S \atop k=1,\dots,n} \left| \frac{\partial g}{\partial y_k}(x,y) \right|.
\]
Τότε η συνάρτηση g πληροί τη συνθήκη του Lipschitz με σταθερά $K \ge 0$ στο S.
\end{rlist}

\textbf{ΠΡΑΓΜΑΤΙΚΑ:} (i) Καθεμιά απ' τις μερικές παράγωγες $\partial g / \partial y_k$ ($k=1,\dots,n$) είναι συνεχής στο R και επομένως παίρνει ένα μέγιστο σ' αυτό. Έτσι, Κ είναι μια μη αρνητική σταθερά. Ας θεωρήσουμε δύο τυχόντα $(x,z), (x,w)$ στο R. Τότε για κάθε $t \in [0,1]$ είναι
\[
|w+t(z-w)-y_0| = |(1-t)(w-y_0)+t(z-y_0)| \le (1-t)|w-y_0|+t|z-y_0| \le (1-t)b+tb=b
\]
και επομένως $(x,w+t(z-w)) \in R$. Έτσι, μπορεί να ορισθεί η συνάρτηση
\[
G(t) = g(x, w+t(z-w)), \ t \in [0,1].
\]
Για όλα τα $t \in [0,1]$ έχουμε
\[
G'(t) = (z_1-w_1)\frac{\partial g}{\partial y_1}(x,w+t(z-w)) + \dots + (z_n-w_n)\frac{\partial g}{\partial y_n}(x,w+t(z-w)),
\]
όπου $z_1,\dots,z_n$ και $w_1,\dots,w_n$ είναι οι συντεταγμένες των z και w αντίστοιχα. Επομένως, για κάθε $t \in [0,1]$ παίρνουμε
\[
|G'(t)| \le |z_1-w_1|\left|\frac{\partial g}{\partial y_1}(x,w+t(z-w))\right| + \dots + |z_n-w_n|\left|\frac{\partial g}{\partial y_n}(x,w+t(z-w))\right|
\]
\[
\le K|z_1-w_1| + \dots + K|z_n-w_n| \le K|z-w|.
\]
Αλλά \[
g(x,z) - g(x,w) = G(1) - G(0) = \int_0^1 G'(t)dt
\]
και επομένως
\[
|g(x,z)-g(x,w)| \le \int_0^1 |G'(t)|dt \le K|z-w|.
\]
(ii) Επειδή καθεμιά απ' τις μερικές παράγωγες $\partial g / \partial y_k$ ($k=1,\dots,n$) είναι φραγμένη στο S, Κ θα είναι μια μη αρνητική σταθερά. Θεωρούμε πάλι δύο τυχόντα $(x,z)$ και $(x,w)$ στο S και παρατηρούμε ότι για κάθε $t \in [0,1]$ είναι $(x,w+t(z-w)) \in S$. Η περαιτέρω απόδειξη είναι ακριβώς η ίδια με εκείνη της περίπτωσης (i).

\subsection{Ύπαρξη και μονοσήμαντο λύσεων προβλημάτων αρχικών τιμών}
Ας θεωρήσουμε τη (διανυσματική) διαφορική εξίσωση
\begin{equation}
y' = f(x,y), \tag{E}
\end{equation}
όπου f είναι μια n-διάστατη διανυσματική συνάρτηση ορισμένη σ' ένα υποσύνολο $D_f$ του καρτεσιανού χώρου της πραγματικής ευθείας με τον χώρο των n-διάστατων διανυσμάτων. Ακόμα, ας θεωρήσουμε ένα σημείο $(x_0,y_0) \in D_f$ και την αρχική συνθήκη
\begin{equation}
y(x_0)=y_0. \tag{C}
\end{equation}
Έχουμε έτσι το πρόβλημα αρχικών τιμών (E)-(C).

Το πρώτο ερώτημα που τίθεται για το παραπάνω πρόβλημα αρχικών τιμών είναι: Υπάρχει λύση του προβλήματος αρχικών τιμών (E)-(C) και, αν υπάρχει, σε ποιά διάστημα είναι ορισμένη και είναι η μοναδική λύση στο διάστημα αυτό; Το θέμα της μελέτης του ερωτήματος αυτού αναφέρεται ως ύπαρξη και μονοσήμαντο λύσεων του προβλήματος αρχικών τιμών (E)-(C). Τα παρακάτω θεωρήματα 1,2 και 3 αναφέρονται στο αντικείμενο αυτό.

\begin{Thewrhma}{}
Ας είναι a και b δύο θετικοί αριθμοί τέτοιοι ώστε το ορθογώνιο
\[
R = \{ (x,y) : |x-x_0| \le a, \ |y-y_0| \le b \} \subseteq D_f
\]
και ας υποθέσουμε ότι η συνάρτηση f είναι συνεχής στο R και πληροί τη συνθήκη του Lipschitz με σταθερά $K \ge 0$ στο R. Ακόμα, ας είναι
\[
M = \max_{(x,y) \in R} |f(x,y)| \quad \text{και} \quad r = \min(a,b/M)
\]
(r=a, όταν M=0).
\end{Thewrhma}
Τότε το πρόβλημα αρχικών τιμών (E)-(C) έχει ακριβώς μια λύση y στο διάστημα $I = \{x:|x-x_0| \le r \}$. Επιπλέον, η λύση y είναι το όριο της ακολουθίας των διαδοχικών προσεγγίσεων $\{\phi_\nu\}, \nu=0,1,\dots$, όπου
\[
\phi_0(x) = y_0, \ x \in I \quad \text{και} \quad \phi_{\nu+1}(x) = y_0 + \int_{x_0}^x f(t, \phi_\nu(t))dt, \ x \in I
\]
\[
(\nu=0,1,\dots).
\]
Ακόμα, είναι
\[
|y(x)-\phi_\nu(x)| \le \frac{M}{K} \frac{(Kr)^{\nu+1}}{(\nu+1)!}e^{Kr}, \ x \in I \ (\nu=0,1,\dots).
\]

\textbf{ΑΠΟΔΕΙΞΗ.} (I) Η ακολουθία $\{\phi_\nu\}, \nu=0,1,\dots$ ορίζεται ως μια ακολουθία συνεχών συναρτήσεων στο διάστημα I και ακόμα $(x,\phi_\nu(x)) \in R$ για κάθε $x \in I$ $(\nu=0,1,\dots)$. Πραγματικά: Η συνάρτηση $\phi_0(x)=y_0$ είναι μια συνεχής συνάρτηση στο διάστημα I και, ακόμα $(x, \phi_0(x)) = (x,y_0) \in R$ για κάθε $x \in I$. Τότε, επειδή $f$ είναι συνεχής στο R, ο τύπος
\[
\phi_1(x) = y_0 + \int_{x_0}^x f(t, \phi_0(t))dt, \ x \in I
\]
ορίζει μια συνεχή συνάρτηση $\phi_1$ στο I για την οποία έχουμε
\[
|\phi_1(x)-y_0| = \left|\int_{x_0}^x f(t,\phi_0(t))dt\right| \le \left|\int_{x_0}^x |f(t,\phi_0(t))|dt\right| \le M|x-x_0| \le b
\]
για όλα τα $x \in I$, δηλαδή $(x,\phi_1(x)) \in R$ για κάθε $x \in I$. Τώρα, ας υποθέσουμε ότι m είναι ένας ακέραιος θετικός τέτοιος ώστε η συνάρτηση $\phi_m$ ορίζεται και είναι συνεχής στο I και $(x,\phi_m(x)) \in R$ για κάθε $x \in I$. Τότε ο τύπος
\[
\phi_{m+1}(x) = y_0 + \int_{x_0}^x f(t, \phi_m(t))dt, \ x \in I
\]
ορίζει μια συνεχή συνάρτηση στο I, αφού η f είναι συνεχής στο R. Επίσης, για οποιοδήποτε $x \in I$ είναι
\[
|\phi_{m+1}(x)-y_0| = \left|\int_{x_0}^x f(t,\phi_m(t))dt\right| \le \left|\int_{x_0}^x |f(t,\phi_m(t))|dt\right| \le M|x-x_0| \le b,
\]
δηλαδή $(x, \phi_{m+1}(x)) \in R$.

(II) Ισχύει
\[
|\phi_{\nu}(x)-\phi_{\nu-1}(x)| \le \frac{M K^{\nu-1}}{\nu!}|x-x_0|^{\nu} \quad \text{για κάθε } x \in I \ (\nu=1,2,\dots).
\]
Πραγματικά: Για όλα τα $x \in I$ έχουμε
\[
|\phi_1(x) - \phi_0(x)| = \left|\int_{x_0}^x f(t,\phi_0(t))dt\right| \le M|x-x_0|.
\]
Ας υποθέσουμε ότι για κάποιο θετικό ακέραιο m είναι
\[
|\phi_m(x) - \phi_{m-1}(x)| \le \frac{M K^{m-1}}{m!}|x-x_0|^m \quad \text{για κάθε } x \in I.
\]
Τότε, παίρνοντας υπόψη το γεγονός ότι η f πληροί τη συνθήκη του Lipschitz με σταθερά $K$ στο R, για οποιοδήποτε $x \in I$ παίρνουμε
\begin{align*}
|\phi_{m+1}(x) - \phi_m(x)| &= \left| \int_{x_0}^x [f(t, \phi_m(t)) - f(t, \phi_{m-1}(t))] dt \right| \\
&\le \left| \int_{x_0}^x |f(t, \phi_m(t)) - f(t, \phi_{m-1}(t))| dt \right| \\
&\le K \left| \int_{x_0}^x |\phi_m(t) - \phi_{m-1}(t)| dt \right| \le K \frac{M K^{m-1}}{m!} \left| \int_{x_0}^x |t-x_0|^m dt \right| \\
&= \frac{M K^m}{(m+1)!}|x-x_0|^{m+1}.
\end{align*}

(III) Η ακολουθία των διαδοχικών προσεγγίσεων $\{\phi_\nu\}, \nu=0,1,\dots$ συγκλίνει. Πραγματικά: Επειδή, όπως εύκολα διαπιστώνεται, ισχύει
\[
\phi_\nu = \phi_0 + \sum_{k=1}^\nu (\phi_k - \phi_{k-1}) \quad (\nu=1,2,\dots),
\]
αρκεί ν' αποδειχθεί ότι συγκλίνει η σειρά συναρτήσεων
\[
\phi_0 + \sum_{\nu=1}^\infty (\phi_\nu - \phi_{\nu-1}).
\]
Αυτή η σειρά συναρτήσεων συγκλίνει απόλυτα, γιατί για κάθε $x \in I$ η σειρά πραγματικών αριθμών $\sum_{\nu=1}^\infty \frac{M K^{\nu-1}}{\nu!}|x-x_0|^\nu$ συγκλίνει.
\newline
\newline
(IV) Η οριακή συνάρτηση
\[
y = \lim_{\nu \to \infty} \phi_\nu = \phi_0 + \sum_{\nu=1}^\infty (\phi_\nu - \phi_{\nu-1})
\]
είναι συνεχής στο διάστημα Ι και τέτοια ώστε $(x,y(x)) \in R$ για κάθε $x \in I$. Πραγματικά: Για τυχόντα $x_1, x_2 \in I$ έχουμε
\[
|\phi_{\nu+1}(x_1) - \phi_{\nu+1}(x_2)| = \left|\int_{x_2}^{x_1} f(t, \phi_\nu(t)) dt\right| \le M|x_1-x_2|
\]
και έτσι για $\nu \to \infty$ παίρνουμε
\[
|y(x_1)-y(x_2)| \le M|x_1-x_2|.
\]
Αυτό αποδεικνύει τη συνέχεια της οριακής συνάρτησης y. Για $x_1=x$ και $x_2=x_0$ είναι
\[
|y(x)-y(x_0)| = |y(x)-y_0| \le M|x-x_0|.
\]
Άρα $(x,y(x)) \in R$ για κάθε $x \in I$.

(V) Ισχύει
\[
|y(x)-\phi_\nu(x)| \le \frac{M}{K} \frac{(Kr)^{\nu+1}}{(\nu+1)!}e^{Kr}, \ x \in I \quad (\nu=0,1,\dots).
\]
Πραγματικά: Για κάθε $x \in I$ είναι
\[
y(x) = \phi_0(x) + \sum_{\nu=1}^\infty [\phi_\nu(x) - \phi_{\nu-1}(x)]
\]
και
\[
\phi_\nu(x) = 
\begin{cdcases}
\phi_0(x), \text{ για } \nu=0 \\
\phi_0(x) + \sum_{k=1}^\nu [\phi_k(x) - \phi_{k-1}(x)], \text{ για } \nu=1,2,\dots
\end{cdcases}
\]
Έτσι, για οποιονδήποτε μη αρνητικό ακέραιο $\nu$ και για όλα τα $x \in I$ είναι
\begin{align*}
|y(x)-\phi_\nu(x)| &= \left|\sum_{p=\nu+1}^\infty [\phi_p(x) - \phi_{p-1}(x)]\right| \le \sum_{p=\nu+1}^\infty |\phi_p(x) - \phi_{p-1}(x)| \\
&\le \sum_{p=\nu+1}^\infty \frac{M K^{p-1}}{p!}|x-x_0|^p \le \frac{M}{K}\sum_{p=\nu+1}^\infty \frac{(Kr)^p}{p!} \\
&\le \frac{M}{K}\frac{(Kr)^{\nu+1}}{(\nu+1)!} \sum_{p=0}^\infty \frac{(\nu+1)!}{(\nu+1+p)!}(Kr)^p \le \frac{M}{K}\frac{(Kr)^{\nu+1}}{(\nu+1)!}e^{Kr}.
\end{align*}

(VI) Η οριακή συνάρτηση y είναι μια λύση του προβλήματος αρχικών τιμών (E)-(C). Πραγματικά: Για κάθε $x \in I$ και για κάθε $\nu=0,1,\dots$ έχουμε
\[
\left| \int_{x_0}^x f(t,y(t))dt - \int_{x_0}^x f(t,\phi_\nu(t))dt \right| \le \left| \int_{x_0}^x |f(t,y(t)) - f(t,\phi_\nu(t))| dt \right|
\]
\[
\le K \left| \int_{x_0}^x |y(t)-\phi_\nu(t)|dt \right|
\]
\[
\le M \frac{(Kr)^{\nu+1}}{(\nu+1)!} e^{Kr}|x-x_0|,
\]
αφού η f πληροί τη συνθήκη του \eng{Lipschitz} με σταθερά K στο R. Αλλά $[(Kr)^{\nu+1}/(\nu+1)!] \to 0$ όταν $\nu \to \infty$,
\newline
\newline
και επομένως
\[
\lim_{\nu \to \infty} \int_{x_0}^x f(t, \phi_\nu(t))dt = \int_{x_0}^x f(t,y(t))dt \text{ για κάθε } x \in I.
\]
Άρα, είναι
\[
y(x) = y_0 + \int_{x_0}^x f(t,y(t))dt, \ x \in I.
\]
Έτσι, παίρνουμε $y(x_0)=y_0$ και $y'(x) = f(x,y(x))$ για όλα τα $x \in I$.

(VII) Η λύση y είναι η μοναδική λύση του προβλήματος αρχικών τιμών (E)-(C) στο διάστημα Ι. Πραγματικά: Ας θεωρήσουμε μια λύση z του προβλήματος αρχικών τιμών (E)-(C) στο διάστημα Ι. Τότε για όλα τα $x \in I$ θα είναι
\[
z(x) = z(x_0) + \int_{x_0}^x f(t,z(t))dt = y_0 + \int_{x_0}^x f(t,z(t))dt
\]
και επομένως
\[
|y(x)-z(x)| = \left| \int_{x_0}^x [f(t,y(t)) - f(t,z(t))]dt \right| \le \left| \int_{x_0}^x |f(t,y(t)) - f(t,z(t))|dt \right|.
\]
Αλλά η f πληροί τη συνθήκη του \eng{Lipschitz} με σταθερά K στο R, οπότε
\[
|y(x)-z(x)| \le K \left| \int_{x_0}^x |y(t)-z(t)|dt \right| \text{ για κάθε } x \in I.
\]
θεωρούμε τη συνάρτηση
\[
h(x) = \int_{x_0}^x |y(t)-z(t)|dt, \ x \in I.
\]
Τότε είναι
\[
h'(x) \le K h(x) \text{ για όλα τα } x \in I.
\]
Έτσι, για κάθε $x \in I$ έχουμε
\[
[e^{-K(x-x_0)}h(x)]' = e^{-K(x-x_0)}h'(x) - K e^{-K(x-x_0)}h(x) = e^{-K(x-x_0)}[h'(x)-Kh(x)] \le 0
\]
και άρα
\[
[e^{-K(x-x_0)}h(x)]' \le 0 \quad \text{για όλα τα } x \in I.
\]
Έτσι, έχουμε ότι $h(x)=0$ για $x \in I$, δηλαδή $|y(x)-z(x)|=0$ για κάθε $x \in I$. Άρα,
\[
y(x) = z(x) \text{ για όλα τα } x \in I.
\]
Η απόδειξη του παραπάνω θεωρήματος είναι και απόδειξη του παρακάτω συμπεράσματος, το οποίο ουσιαστικά είναι ισοδύναμο με το θεώρημα 1.

\begin{center}
\rule{0.5\textwidth}{0.4pt}
\end{center}

Το \textbf{θεώρημα 1} εξακολουθεί να ισχύει με
\[
R = \{(x,y): x_0 \le x \le x_0+a, |y-y_0| \le b\} \text{ και } I = [x_0, x_0+r]
\]
ή αντίστοιχα
\[
R = \{(x,y): x_0-a \le x \le x_0, |y-y_0| \le b\} \text{ και } I=[x_0-r,x_0].
\]

\begin{Thewrhma}{Θεώρημα ...}
Ας είναι α ένας θετικός αριθμός τέτοιος ώστε
$S = \{(x,y):|x-x_0| \le a, \ y \text{ αυθαίρετο}\} \subseteq D_f$
\newline
και ας υποθέσουμε ότι η συνάρτηση f είναι συνεχής στο S και πληροί τη συνθήκη του Lipschitz με σταθερά $K>0$ στο S.
\newline
\newline
Τότε το πρόβλημα αρχικών τιμών (E)-(C) έχει ακριβώς μια λύση y στο διάστημα $I=[x: |x-x_0| \le a]$. Επιπλέον, η λύση y είναι το όριο της ακολουθίας των διαδοχικών προσεγγίσεων $\{\phi_\nu\}, \nu=0,1,\dots$, όπου
\[
\phi_0(x) = y_0', \ x \in I \text{ και } \phi_{\nu+1}(x) = y_0' + \int_{x_0}^x f(t, \phi_\nu(t))dt, \ x \in I \ (\nu=0,1,\dots).
\]
Ακόμα, είναι
\[
|y(x)-\phi_\nu(x)| \le \frac{M}{K} \frac{(Ka)^{\nu+1}}{(\nu+1)!}e^{Ka}, \ x \in I \ (\nu=0,1,\dots),
\]
όπου
\[
M = \max_{x \in I} |f(x,y_0)|.
\]
\end{Thewrhma}
\textbf{ΑΠΟΔΕΙΞΗ.} (I) Η ακολουθία $\{\phi_\nu\}, \nu=0,1,\dots$, ορίζεται ως μια ακολουθία συνεχών συναρτήσεων στο διάστημα Ι: επιπλέον, είναι $(x, \phi_\nu(x)) \in S$ για κάθε $x \in I \ (\nu=0,1,\dots)$. Πραγματικά: Η συνάρτηση $\phi_0(x)=y_0'$ $x \in I$ είναι συνεχής. Έτσι, επειδή η f είναι συνεχής στο S, η συνάρτηση $\phi_1$ με
\[
\phi_1(x) = y_0' + \int_{x_0}^x f(t, \phi_0(t))dt, \ x \in I
\]
ορίζεται και είναι συνεχής στο Ι. Ας υποθέσουμε ότι η συνάρτηση $\phi_m$ ορίζεται και είναι συνεχής στο Ι, όπου m είναι ένας θετικός ακέραιος. Τότε ο τύπος
\[
\phi_{m+1}(x) = y_0' + \int_{x_0}^x f(t, \phi_m(t))dt, \ x \in I
\]
ορίζει μια συνεχή συνάρτηση $\phi_{m+1}$ στο Ι. Τώρα, είναι φανερό ότι $(x, \phi_1(x)) \in S$ για κάθε $x \in I \ (\nu=0,1,\dots)$.

(II) Ισχύει
\[
|\phi_\nu(x)-\phi_{\nu-1}(x)| \le \frac{MK^{\nu-1}}{\nu!}|x-x_0|^\nu \text{ για κάθε } x \in I \ (\nu=1,2,\dots).
\]
Πραγματικά: Για όλα τα $x \in I$ έχουμε
\[
|\phi_1(x)-\phi_0(x)| = \left|\int_{x_0}^x f(t,y_0)dt\right| \le \left|\int_{x_0}^x |f(t,y_0)|dt\right| \le M|x-x_0|.
\]
Το υπόλοιπο της απόδειξης γίνεται όπως ακριβώς στο μέρος (ΙΙ) της απόδειξης του θεωρήματος 1 (με S αντί για R).

(III) Η ακολουθία των διαδοχικών προσεγγίσεων $\{\phi_\nu\}, \nu=0,1,\dots$ συγκλίνει. Η απόδειξη του ισχυρισμού αυτού είναι η ίδια με την απόδειξη του (όμοιου ισχυρισμού) στο μέρος (III) της απόδειξης του θεωρήματος 1.

(IV) Η οριακή συνάρτηση $y = \lim \phi_\nu$ είναι συνεχής στο διάστημα Ι. Πραγματικά: Για κάθε $\nu=0,1,\dots$ και για όλα τα $x \in I$ έχουμε
\[
|\phi_{\nu+1}(x) - y_0| = \left|\sum_{p=1}^{\nu+1} [\phi_p(x)-\phi_{p-1}(x)]\right| \le \sum_{p=1}^{\nu+1} |\phi_p(x)-\phi_{p-1}(x)|
\]
\[
\le \frac{M}{K} \sum_{p=1}^{\nu+1} \frac{K^p|x-x_0|^p}{p!} \le \frac{M}{K} \sum_{p=1}^\infty \frac{(Ka)^p}{p!} = \frac{M}{K}(e^{Ka}-1).
\]
Έτσι, αν θέσουμε
\[
b = \frac{M}{K}(e^{Ka}-1),
\]
τότε θα είναι
\[
|\phi_\nu(x) - y_0| \le b \text{ για κάθε } x \in I \ (\nu=0,1,\dots).
\]
Παραπέρα, θέτουμε
\[
N = \max_{\substack{|x-x_0| \le a \\ |y-y_0| \le b}} |f(x,y)|.
\]
Τότε για τυχόντα $x_1, x_2 \in I$ παίρνουμε
\[
|\phi_{\nu+1}(x_1)-\phi_{\nu+1}(x_2)| = \left|\int_{x_2}^{x_1} f(t,\phi_\nu(t))dt \right| \le N|x_1-x_2|
\]
και επομένως, για $\nu \to \infty$, είναι
\[
|y(x_1)-y(x_2)| \le N|x_1-x_2|,
\]
το οποίο αποδεικνύει τη συνέχεια της συνάρτησης y.

(V) Ισχύει
\[
|y(x)-\phi_\nu(x)| \le \frac{M}{K} \frac{(Ka)^{\nu+1}}{(\nu+1)!}e^{Ka}, \ x \in I \ (\nu=0,1,\dots).
\]
Η απόδειξη γίνεται ακριβώς όπως στο μέρος (V) της απόδειξης του θεωρήματος 1 με α στη θέση του r.

(VI) Η οριακή συνάρτηση y είναι μια λύση του προβλήματος αρχικών τιμών (E)-(C). Αρκεί να επαναλάβουμε την απόδειξη του (ίδιου ισχυρισμού) του μέρους (VI) της απόδειξης του θεωρήματος 1 με α στη θέση του r.

(VII) Η λύση y είναι η μοναδική λύση του προβλήματος αρχικών τιμών (E)-(C). Ισχύει η ίδια ακριβώς απόδειξη με εκείνη του μέρους (VII) της απόδειξης του θεωρήματος 1.

\vspace{5mm}

Η απόδειξη του θεωρήματος 2 είναι και απόδειξη του παρακάτω συμπεράσματος, ισοδύναμου ουσιαστικά με το θεώρημα 2.

Το \textbf{θεώρημα 2} εξακολουθεί να ισχύει με
\[
S = \{(x,y): x_0 \le x \le x_0+a, \ y \text{ αυθαίρετο}\} \text{ και } I=[x_0, x_0+a]
\]
ή αντίστοιχα
\[
S = \{(x,y): x_0-a \le x \le x_0, \ y \text{ αυθαίρετο}\} \text{ και } I=[x_0-a, x_0].
\]
\begin{Thewrhma}{}Ας είναι Ι ένα διάστημα της πραγματικής ευθείας τέτοιο ώστε $x_0 \in I$ και
\[
E = \{(x,y): x \in I, \ y \text{ αυθαίρετο}\} \subseteq D_f.
\]
και ας υποθέσουμε ότι η f είναι συνεχής στο E. Ακόμα, ας υποθέσουμε ότι για κάθε συμπαγές υποδιάστημα J του I η f πληροί τη συνθήκη του Lipschitz στο σύνολο
\[
E_J = \{(x,y): x \in J, \ y \text{ αυθαίρετο}\}.
\]
Τότε το πρόβλημα αρχικών τιμών (E)-(C) έχει ακριβώς μια λύση στο διάστημα I.
\end{Thewrhma}
\textbf{ΑΠΟΔΕΙΞΗ.} Πρώτα απ' όλα, παρατηρούμε ότι, αν $y^*$ είναι μια λύση της διαφορικής εξίσωσης (E) σ' ένα κλειστό δεξιό διάστημα $I \subseteq I$ με δεξιό άκρο $x_0$ και $\tilde{y}$ μια λύση αυτής σ' ένα κλειστό αριστερό διάστημα $\tilde{I} \subseteq I$ με αριστερό άκρο $x_0$ και $y*(\tilde{x})=\tilde{y}(\tilde{x})$, τότε η συνάρτηση y που ορίζεται ως εξής
\[
y(x) = y*(x) \text{ για } x \in I* \text{ και } y(x) = \tilde{y}(x) \text{ για } x \in \tilde{I}
\]
είναι μια λύση της (Ε) στο διάστημα $I* \cup \tilde{I}$.

Αρκεί ν' αποδείξουμε ότι το πρόβλημα αρχικών τιμών (E)-(C) έχει μια ακριβώς λύση στο διάστημα $I \cap [x_0, \infty)$, εφόσον το $x_0$ δεν είναι το δεξιό άκρο του Ι, και μια ακριβώς λύση στο $I \cap (-\infty, x_0]$, εφόσον το $x_0$ δεν είναι το αριστερό άκρο του Ι. Υποθέτουμε ότι $x_0 < b$, όπου b είναι το δεξιό άκρο του Ι. Αν $x_1 \in I$, τότε, ως απόδειξη ότι μια λύση του (Ε)-(C) στο διάστημα $[x_0, b) = I \cap [x_0, \infty)$. Ας υποθέσουμε ότι $x_0 \in I$ και ας θεωρήσουμε μια ακολουθία $\{x_\nu\}$ σημείων του διαστήματος $[x_0, b)$ με $x_0 < x_1 < x_2 < \dots$ και $\lim x_\nu = b$. Επαγωγικά ορίζουμε την ακολουθία συναρτήσεων $\{\phi_\nu\}, \nu \ge 0$, όπου $\phi_0$ είναι η μοναδική λύση του (E)-(C) στο διάστημα $[x_0, x_1]$ και, για κάθε $\nu \ge 1$, $\phi_\nu$ είναι η μοναδική λύση στο διάστημα $[x_\nu, x_{\nu+1}]$ της διαφορικής εξίσωσης (Ε) με $\phi_\nu(x_\nu) = \phi_{\nu-1}(x_\nu)$. Τότε η συνάρτηση y με
\[
y(x) = \phi_\nu(x), \ x \in [x_\nu, x_{\nu+1}] \ (\nu=0,1,\dots)
\]
είναι μια λύση στο διάστημα $[x_0, b) = I \cap [x_0, \infty)$ του προβλήματος αρχικών τιμών (E)-(C) και μάλιστα αυτή είναι μοναδική. Κατά ένα ανάλογο τρόπο, μπορούμε ν' αποδείξουμε ότι, στην περίπτωση όπου το $x_0$ δεν είναι το αριστερό άκρο του Ι, το (Ε)-(C) έχει ακριβώς μια λύση στο διάστημα $I \cap (-\infty, x_0]$.
Τότε, ας είναι Α ένας n-τάξης (τετραγωνικός) πίνακας-συνάρτηση σ' ένα διάστημα Ι της πραγματικής ευθείας που είναι συνεχής και b μια συνεχής n-διάστατη διανυσματική συνάρτηση στο διάστημα Ι. Επιπλέον, ας είναι $E = \{(x,y) : x \in I, \text{y αυθαίρετο n-διάστατο διάνυσμα}\}$. Τότε η συνάρτηση f με
\[
f(x,y) = A(x)y+b(x), \ (x,y) \in E
\]
είναι συνεχής στο Ε. Επιπλέον, αν J είναι ένα συμπαγές υποδιάστημα του I και $E_J = \{(x,y) : x \in J, \text{y αυθαίρετο}\}$, τότε η f πληροί τη συνθήκη του Lipschitz στο $E_J$, αφού για κάθε $(x,z), (x,w)$ στο $E_J$ έχουμε
\[
|f(x,z)-f(x,w)| = |A(x)(z-w)| \le |A(x)||z-w| \le [\max_{x \in J}|A(x)|]|z-w|,
\]
όπου για ένα n-τάξης τετραγωνικό πίνακα C η στάθμη του $|C|$ ορίζεται με τον τύπο
\[
|C| = \sup_{|x|=1} |Cx|.
\]
Έτσι, το θεώρημα 3 οδηγεί στο παρακάτω θεώρημα 4 για την ύπαρξη και το μονοσήμαντο λύσεων προβλημάτων αρχικών τιμών για γραμμικά διαφορικά συστήματα.

\begin{Thewrhma}{}
Ας είναι Α ένας n-τάξης συνεχής πίνακας-συνάρτηση σ' ένα διάστημα Ι και b μια συνεχής n-διάστατη διανυσματική συνάρτηση στο Ι. Τότε, για κάθε $x_0 \in I$ και για οποιοδήποτε n-διάστατο διάνυσμα $y_0'$, το γραμμικό διαφορικό σύστημα
\[
y' = Ay+b
\]
έχει ακριβώς μια λύση y στο διάστημα Ι που πληροί την αρχική συνθήκη
\[
y(x_0) = y_0'.
\]
\end{Thewrhma}

Μια γραμμική διαφορική εξίσωση ανάγεται σ' ένα αντίστοιχο γραμμικό διαφορικό σύστημα, όπως έχουμε δεί στο προηγούμενο εδάφιο. Έτσι, για προβλήματα αρχικών τιμών που αναφέρονται σε γραμμικές διαφορικές εξισώσεις έχουμε το παρακάτω θεώρημα 5 για την ύπαρξη και το μονοσήμαντο λύσεων, που προκύπτει ως μια συνέπεια απ' το θεώρημα 4.

\begin{Thewrhma}{}
Ας είναι $a_i \ (i=0,1,...,n)$ και b συνεχείς συναρτήσεις σ' ένα διάστημα Ι και $a_n(x) \ne 0$ για όλα τα $x \in I$. Τότε, για κάθε

$x_0 \in I$ και για οποιεσδήποτε σταθερές $y_0, y_1, \dots, y_{n-1}$, η γραμμική διαφορική εξίσωση
\[
a_n y^{(n)} + a_{n-1} y^{(n-1)} + \dots + a_1 y' + a_0 y = b
\]
έχει ακριβώς μια λύση y στο διάστημα I που πληροί τις αρχικές συνθήκες
\[
y(x_0)=y_0, \ y'(x_0)=y_1, \dots, y^{(n-1)}(x_0)=y_{n-1}.
\]
\end{Thewrhma}
\subsection{Παραδείγματα}

\begin{Paradeigma}{Ν'αποδειχθεί ότι σε καθεμιά απ'τις παρακάτω περιπτώσεις η συνάρτηση g πληροί τη συνθήκη του Lipschitz στο σημειούμενο σύνολο:}
\begin{rlist}
\item $g(x,y) = \sin y + y \cos x; \ R = \{(x,y) : |x| \le 1, |y| \le 2\}.$
\item $g(x,y) = x^2 \arctan y + e^x; \ S = \{(x,y) : |x| \le 2, \text{y αυθαίρετο}\}.$
\item $g(x,y_1,y_2) = \begin{pmatrix} \frac{x^2+y_1^2}{x^2+y_2^2} \\ 2x+y_2 \end{pmatrix}; \ R = \{(x,y) : |x|\le 1, |y_1|+|y_2| \le 1\}.$
\item $g(x,y_1,y_2) = \begin{pmatrix} e^{x+\sin y_1} \\ x \arctan y_2 + \cos y_1 \end{pmatrix}; \ S = \{(x,y): |x|\le 1, y_1 \text{ και } y_2 \text{ αυθαίρετα}\}.$
\end{rlist}
\end{Paradeigma}

(i) Έχουμε
\[
K = \max_{(x,y) \in R} \left|\frac{\partial g}{\partial y}(x,y)\right| = \max_{(x,y) \in R} |\cos y + \cos x| = 2
\]
και επομένως η συνάρτηση g πληροί τη συνθήκη του Lipschitz στο R με σταθερά K=2.

(ii) Είναι
\[
K = \sup_{(x,y) \in S} \left|\frac{\partial g}{\partial y}(x,y)\right| = \sup_{(x,y) \in S} \left|\frac{x^2}{1+y^2}\right| = 4
\]
και άρα η g πληροί τη συνθήκη του \eng{Lipschitz} με σταθερά K=4 στο σύνολο S.

(iii) Η g πληροί τη συνθήκη του \eng{Lipschitz} με σταθερά $K=e^2$ στο R, γιατί
\[
\max_{(x,y_1,y_2) \in R} \left|\frac{\partial g}{\partial y_1}\right| = \max_{(x,y_1,y_2) \in R} \begin{Vmatrix} 2y_1 \\ 0 \end{Vmatrix} = \max_{(x,y_1,y_2) \in R} |2y_1| = 2
\]
\[
\max_{(x,y_1,y_2) \in R} \left|\frac{\partial g}{\partial y_2}\right| = \max_{(x,y_1,y_2) \in R} \begin{Vmatrix} 0 \\ x^2+y_2 \end{Vmatrix} = \max_{(x,y_1,y_2) \in R} |x^2+y_2| = e^2,
\]
και $K = \max(2, e^2) = e^2$.

(iv) Είναι
\[
\sup_{(x,y_1,y_2) \in S} \left|\frac{\partial g}{\partial y_1}\right| = \sup_{(x,y_1,y_2) \in S} \begin{Vmatrix} \cos y_1 \\ -\sin y_1 \end{Vmatrix} = \sup_{(x,y_1,y_2) \in S} \sqrt{\cos^2 y_1 + \sin^2 y_1} = \sqrt{2}
\]
και
\[
\sup_{(x,y_1,y_2) \in S} \left|\frac{\partial g}{\partial y_2}\right| = \sup_{(x,y_1,y_2) \in S} \begin{Vmatrix} 0 \\ x/(1+y_2^2) \end{Vmatrix} = \sup_{(x,y_1,y_2) \in S} |x/(1+y_2^2)| = 1
\]
και άρα η g πληροί τη συνθήκη του Lipschitz με σταθερά $K=\sqrt{2}$ στο S.

\begin{Paradeigma}{Να εξετασθεί ως προς την ύπαρξη και το μονοσήμαντο λύσεων το πρόβλημα αρχικών τιμών}
\[
y' = x^2+y^2, \ y(0)=0.
\]
\end{Paradeigma}

\lysh
θέτουμε $f(x,y) = x^2+y^2$ για όλα τα $x,y$. θεωρούμε δύο θετικούς αριθμούς $a$ και $b$ και θέτουμε
\[
R = \{(x,y) : |x| \le a, |y| \le b\}.
\]
Η συνάρτηση $f$ είναι συνεχής στο $\mathbb{R}$. Επίσης, η f πληροί τη συνθήκη του \eng{Lipschitz} με σταθερά $K = 2b$ στο $\mathbb{R}$, γιατί
\[
\max_{(x,y) \in R} \left|\frac{\partial f}{\partial y}(x,y)\right| = \max_{(x,y) \in R} |2y| = 2b.
\]
Τώρα, έχουμε
\[
M = \max_{(x,y) \in R} |f(x,y)| = \max_{(x,y) \in R} |x^2+y^2| = a^2+b^2,
\]
και έτσι
\[
r = \min(a, b/M) = \min(a, b/(a^2+b^2)).
\]
Σύμφωνα με το θεώρημα 1, το πρόβλημα αρχικών τιμών έχει ακριβώς μια λύση στο διάστημα $[-r,r]$. θα βρούμε τώρα το μέγιστο τέτοιο διάστημα. Για ένα δεδομένο $a>0$, η μέγιστη τιμή της παράστασης $b/(a^2+b^2)$ λαμβάνεται για $b=a$ και είναι ίση με $1/2a$. Τότε η μέγιστη τιμή του $r=\min(a, 1/2a)$ είναι ίση με $1/\sqrt{2}$ και λαμβάνεται για $a=1/\sqrt{2}$. Έτσι, εκλέγοντας $a=b=1/\sqrt{2}$, συμπεραίνουμε ότι το πρόβλημα αρχικών τιμών που δόθηκε έχει ακριβώς μια λύση στο διάστημα $[-1/2, 1/2]$.

\begin{Paradeigma}{Ν'αποδειχθεί ότι το πρόβλημα αρχικών τιμών}
\[
y' = x+y^2, y(0)=0
\]
έχει ακριβώς μια λύση y στο διάστημα $[-1/2, 1/2]$. Στη συνέχεια, να βρεθεί μια προσέγγιση $\bar{y}$ της λύσης y τέτοια ώστε
\[
|y(x)-\bar{y}(x)| \le \frac{10-7\sqrt{2}}{24} e^{(2-\sqrt{2})/2} \ \text{για} \ x \in [-1/2, 1/2].
\]
\end{Paradeigma}

\lysh
Ο τύπος $f(x,y) = x+y^2$ ορίζει τη συνάρτηση f για όλα τα x,y. θεωρούμε ένα θετικό αριθμό b και θέτουμε
\[
R = \{(x,y): |x|\le 1/2, |y|\le b\}.
\]
Η συνάρτηση f είναι συνεχής στο R. Επίσης, η f πληροί στο R τη συνθήκη του Lipschitz με σταθερά K=2b, αφού
\[
\max_{(x,y) \in R} \left|\frac{\partial f}{\partial y}(x,y)\right| = \max_{(x,y) \in R} |2y| = 2b.
\]
Ακόμα, είναι
\[
M = \max_{(x,y) \in R} |f(x,y)| = \frac{1}{2}+b^2.
\]
θέτουμε
\[
r = \min\left(\frac{1}{2}, \frac{b}{1/2+b^2}\right).
\]
Επιλέγοντας $b=(2-\sqrt{2})/2$, δηλώνουμε $r=1/2$ και επομένως (θεώρημα 1) το πρόβλημα αρχικών τιμών έχει ακριβώς μια λύση στο διάστημα $[-1/2, 1/2]$. Για την παραπάνω επιλογή του b είναι $K=2-\sqrt{2}$ και $M=2-\sqrt{2}$. Η λύση y είναι (θεώρημα 1) το όριο της ακολουθίας των διαδοχικών προσεγγίσεων $(\phi_\nu)_{\nu=0,1,\dots}$, όπου
\[
\phi_0(x) = 0, \ x \in [-1/2, 1/2] \ \text{και} \ \phi_{\nu+1}(x) = \int_0^x [t+\phi_\nu^2(t)]dt, \ x \in [-1/2, 1/2] \ (\nu=0,1,\dots).
\]
Επίσης, ισχύει (θεώρημα 1) \[
|y(x)-\phi_\nu(x)| \le \frac{M}{K} \frac{(Kr)^{\nu+1}}{(\nu+1)!} e^{Kr}, \ x \in [-1/2, 1/2] \ (\nu=0,1,\dots).
\]
Έτσι, για $\nu=2$ έχουμε
\[
|y(x)-\phi_2(x)| \le \frac{10-7\sqrt{2}}{24} e^{(2-\sqrt{2})/2} \ \text{για} \ x \in [-1/2, 1/2].
\]
Αρκεί, λοιπόν, να πάρουμε $\bar{y}=\phi_2$. Αλλά είναι
\[
\phi_1(x) = \int_0^x [t+\phi_0^2(t)]dt = \int_0^x t\,dt = \frac{x^2}{2}, \ x \in [-1/2, 1/2]
\]
και
\[
\phi_2(x) = \int_0^x [t+\phi_1^2(t)]dt = \int_0^x \left(t+\frac{t^4}{4}\right)dt = \frac{x^2}{2} + \frac{x^5}{20}, \ x \in [-1/2, 1/2].
\]
Έτσι, η ζητούμενη προσέγγιση της λύσης y είναι
\[
\bar{y}(x) = \frac{x^2}{2} + \frac{x^5}{20}, \ x \in [-1/2, 1/2].
\]

\begin{Paradeigma}{Να εξετασθεί ως προς την ύπαρξη και το μονοσήμαντο λύσεων το πρόβλημα αρχικών τιμών}
\[
y' = e^{y^2} + \sqrt{1-x^2}, \ y(0)=1.
\]
\end{Paradeigma}

\lysh
θέτουμε
\[
S = \{(x,y): |x|\le 1, \text{y αυθαίρετο}\}
\]
και θεωρούμε τη συνάρτηση
\[
f(x,y) = e^{y^2} + \sqrt{1-x^2}, \ (x,y) \in S.
\]
Η f είναι συνεχής στο S. Επίσης, η f πληροί τη συνθήκη του Lipschitz στο S, αφού για κάθε $(x,y) \in S$ είναι
\[
\left|\frac{\partial f}{\partial y}(x,y)\right| = |2y|e^{y^2} \ge \sqrt{2}e^{-1/2}.
\]
Άρα, σύμφωνα με το θεώρημα 2, το πρόβλημα αρχικών τιμών έχει ακριβώς μια λύση στο διάστημα $[-1,1]$.

\begin{Paradeigma}{Να εξετασθεί ως προς την ύπαρξη και το μονοσήμαντο λύσεων το πρόβλημα αρχικών τιμών}
\[
y' = x^2 \arctan y + e^{-x}, \ y(1)=-2.
\]
\end{Paradeigma}

\lysh
θέτουμε $f(x,y) = x^2 \arctan y + e^{-x}$ για όλα τα x,y. Η συνάρτηση f είναι συνεχής. Ακόμα, για κάθε συμπαγές διάστημα J του $\mathbb{R}$, η f πληροί τη συνθήκη του Lipschitz στο σύνολο $E_J = \{(x,y): x \in J, \text{y αυθαίρετο}\}$,

αφού
\[
\left|\frac{\partial f}{\partial y}(x,y)\right| = \left|x^2/(1+y^2)\right| \le \max_{x \in J} x^2 \ \text{για κάθε} \ (x,y) \in E_J.
\]
Έτσι, το θεώρημα 3 εξασφαλίζει ότι το πρόβλημα αρχικών τιμών έχει ακριβώς μια λύση στην πραγματική ευθεία $\mathbb{R}$.

\begin{Paradeigma}{Να εξετασθεί ως προς την ύπαρξη και το μονοσήμαντο λύσεων το πρόβλημα αρχικών τιμών}
\[
y_1' = y_2, \ y_2' = x+y_1^2; \ y_1(0)=y_2(0)=0.
\]
\end{Paradeigma}

\lysh
θέτουμε
\[
\mathbf{y} = \begin{pmatrix} y_1 \\ y_2 \end{pmatrix} \ \text{και} \ f(x,y_1,y_2) = \begin{pmatrix} y_2 \\ x+y_1^2 \end{pmatrix},
\]
οπότε το πρόβλημα αρχικών τιμών γράφεται στη μορφή
\[
\mathbf{y}' = \mathbf{f}(x,\mathbf{y}), \ \mathbf{y}(0)=\mathbf{0}.
\]
θεωρούμε δύο θετικούς αριθμούς α και b και θέτουμε
\[
R = \{(x,y): |x|\le a, |y|\le b\}.
\]
Παρατηρούμε ότι η συνάρτηση f είναι συνεχής στο R. Ακόμα, έχουμε
\[
\max_{(x,y) \in R} \left|\frac{\partial f}{\partial y_1}\right| = \max_{(x,y) \in R} \begin{Vmatrix} 0 \\ 2y_1 \end{Vmatrix} = \max_{(x,y) \in R} |2y_1| = 2b
\]
και
\[
\max_{(x,y) \in R} \left|\frac{\partial f}{\partial y_2}\right| = \max_{(x,y) \in R} \begin{Vmatrix} 1 \\ 0 \end{Vmatrix} = 1.
\]
Έτσι, η f πληροί τη συνθήκη του \eng{Lipschitz} με σταθερά $K=2b$ στο R. Στη συνέχεια, έχουμε
\[
M = \max_{(x,y) \in R} |\mathbf{f}(x,y)| = \max_{(x,y) \in R} \sqrt{y_2^2 + |x+y_1^2|} = a+b^2.
\]
σύμφωνα με το θεώρημα 1, το πρόβλημα αρχικών τιμών έχει ακριβώς μια λύση στο διάστημα $[-r,r]$, όπου
\[
r = \min(a, b/M) = \min(a, b/(a+b^2)).
\]
Για ένα δεδομένο $a>0$ η μέγιστη τιμή της παράστασης $b/(a+b^2)$ λαμβάνεται όταν $b=\sqrt{a}$ και είναι ίση με $1/(2\sqrt{a})$. Το $r=\min(a, 1/(2\sqrt{a}))$ γίνεται μέγιστο για $a=1/\sqrt[3]{4}$. Έτσι, το πρόβλημα αρχικών τιμών έχει ακριβώς μια λύση στο διάστημα $[-1/\sqrt[3]{4}, 1/\sqrt[3]{4}]$.

\begin{Paradeigma}{Να διαπιστωθεί ότι το πρόβλημα αρχικών τιμών}
\[
y' = 3xy^{1/3}, \ y(0)=0
\]
έχει τις λύσεις $y_1(x)=0, \ x \in \mathbb{R}$ και $y_2(x)=x^3, \ x \in \mathbb{R}$. (ii) Ν' αποδειχθεί ότι η συνάρτηση $f(x,y) = 3xy^{1/3}$ δεν πληροί τη συνθήκη του Lipschitz στο σύνολο
\[
R = \{(x,y): |x|\le a, |y|\le b\} \ (a,b>0).
\]
\end{Paradeigma}

\lysh
(i) Είναι φανερό. (ii) Ας υποθέσουμε ότι η συνάρτηση f πληροί τη συνθήκη του Lipschitz με μια σταθερά $K>0$ στο R. Τότε θα είναι
\[
|f(x,y)-f(x,z)| \le K|y-z| \ \text{για όλα τα} \ (x,y), (x,z) \ \text{στο R}.
\]
Έτσι, για κάθε $\delta$ με $0<\delta<b$ έχουμε
\[
|f(a,\delta)-f(a,-\delta)| = |3a\delta^{1/3} - 3a(-\delta)^{1/3}| = 6a\delta^{1/3} \le K|\delta-(-\delta)| = 2K\delta.
\]
Άρα, για κάθε $\delta$ με $0<\delta<b$ είναι $\delta^{2/3} \ge (3a)/K$. Αυτό όμως είναι άτοπο.\\\\
Ασκήσεις\\
1. Ν' αποδειχθεί ότι σε καθεμιά απ' τις παρακάτω περιπτώσεις η συνάρτηση g πληροί τη συνθήκη του Lipschitz στο σημειούμενο σύνολο:
\begin{enumerate}[label=(\roman*)]
\item $g(x,y) = x^2\cos^2 y + y\sin^2 x, \ S=\{(x,y): |x|\le 1, \text{y αυθαίρετο}\}$.
\item $g(x,y) = 4x^2+y^2, \ R=\{(x,y): |x|\le 1, |y|\le 1\}$.
\item $g(x,y) = x^2 e^{-xy^2}, \ S=\{(x,y): 0\le x \le 1, \text{y αυθαίρετο}\}$.
\item $g(x,y) = x^2|y|, \ R=\{(x,y): |x|\le 1, |y|\le 1\}$.
\item $g(x,y) = y(2-x), \ \text{αν} \ x \ge 0, \ S=\{(x,y): |x|\le 1, \text{y αυθαίρετο}\}$.
\item $g(x,y) = x^2 y^2 + (\log x) y+x^2, \ E=\{(x,y): 1\le x \le 2, 1\le y \le 4\}$.
\end{enumerate}

2. Ν' αποδειχθεί ότι σε καθεμιά απ' τις παρακάτω περιπτώσεις η συνάρτηση g δεν πληροί τη συνθήκη του \eng{Lipschitz} στο σημειούμενο σύνολο:
\begin{enumerate}[label=(\roman*)]
\setcounter{enumi}{0}
\item $g(x,y) = xy^2, \ S=\{ (x,y) : |x|\le 1, \text{y αυθαίρετο} \}$.
\item $g(x,y) = e^{x+y}, \ R=\{ (x,y) : |x|\le 1, |y|\le 1 \}$.
\item $g(x,y) = \begin{cdcases} \frac{4x}{x^2+y^2}, \ \text{αν} \ x\ne 0 \ \text{ή} \ y\ne 0 \\ 0, \ \text{αν} \ x=y=0 \end{cdcases}, \ R=\{ (x,y) : |x|\le 1, |y|\le 1 \}$.
\end{enumerate}

3. Ν' αποδειχθεί ότι καθένα απ' τα παρακάτω προβλήματα αρχικών τιμών έχει ακριβώς μια λύση στο σημειούμενο διάστημα:
\begin{enumerate}[label=(\roman*)]
\item $y' = y^2+\cos x^2, \ y(0)=0; \ I=[-1/2, 1/2]$.
\item $y' = \frac{y^3}{1-x^2}, \ y(0)=0; \ I=[-\sqrt{6}/2, \sqrt{6}/2]$.
\item $y' = y^3+e^{-x}, \ y(0)=2/5; \ I=[-7/10, 7/10]$.
\item $y' = 1+y+y^2\cos x, \ y(0)=0; \ I=[-1/3, 1/3]$.
\item $y' = (4y+e^{-x})e^{2y}, \ y(0)=0; \ I=\left[-\frac{1}{8\sqrt{e}}, \frac{1}{8\sqrt{e}}\right]$.
\item $y' = \frac{1}{4}(1+\cos 4x)y^2, \ y(0)=100; \ I=[-1,1]$.
\item $y' = e^{-x} + \log(1+y^2), \ y(0)=0; \ I=(-\infty,\infty)$.
\end{enumerate}

4. Ν' αποδειχθεί ότι το πρόβλημα αρχικών τιμών
\[
y_1' = 1+y_2^2, \ y_2' = y_1^2; \ y_1(0)=y_2(0)=0
\]
έχει ακριβώς μια λύση στο διάστημα $[-1,1]$.

5. Ν' αποδειχθεί ότι το πρόβλημα αρχικών τιμών
\[
y' = \frac{e^{\cos x}}{x-y^2}, \ y(0)=-2
\]
έχει ακριβώς μια λύση στο διάστημα $(-1,1)$.

6. Να μελετηθεί ως προς την ύπαρξη και το μονοσήμαντο λύσεων το πρόβλημα αρχικών τιμών
\[
y' = 1+xy^2, \ y(0)=0.
\]

7. Να μελετηθεί ως προς την ύπαρξη και το μονοσήμαντο λύσεων το πρόβλημα αρχικών τιμών
\[
y_1'' = x^2y_1+y_1'y_2, \ y_2' = e^x y_1-y_2; \ y_1(1)=2, \ y_1'(1)=0, \ y_2(1)=-1.
\]
\setchapterimage{./images/2.png}
\chapter{Διαφορικές εξισώσεις πρώτης τάξης ορισμένων ειδικών μορφών}
\chaptertoc
\section*{Εισαγωγή}
Στο Κεφάλαιο αυτό αναπτύσσονται μερικές στοιχειώδεις μέθοδοι για την επίλυση διαφορικών εξισώσεων πρώτης τάξης ορισμένων ειδικών μορφών. Στο Εδάφιο 1 εξετάζονται οι γραμμικές διαφορικές εξισώσεις πρώτης τάξης καθώς και δύο άλλες κατηγορίες διαφορικών εξισώσεων πρώτης τάξης (οι διαφορικές εξισώσεις \eng{Bernoulli} και οι διαφορικές εξισώσεις \eng{Riccati}) που ανάγονται σε κατάλληλους μετασχηματισμούς σε γραμμικές εξισώσεις. Το Εδάφιο 2 αφορά τη μελέτη των διαφορικών εξισώσεων χωριζομένων μεταβλητών και των ομογενών διαφορικών εξισώσεων που μετασχηματίζονται σε εξισώσεις χωριζομένων μεταβλητών. Οι αμέσως ολοκληρώσιμες διαφορικές εξισώσεις και εκείνες που ανάγονται σε τέτοιες με τη βοήθεια ενός ολοκληρωτικού παράγοντα μελετώνται στο Εδάφιο 3. Στο Εδάφιο 4 θεωρούνται τρεις κατηγορίες διαφορικών εξισώσεων δεύτερης τάξης (οι διαφορικές εξισώσεις δεύτερης τάξης που δεν περιέχουν την άγνωστη συνάρτηση ή την ανεξάρτητη μεταβλητή καθώς και οι γραμμικές εξισώσεις δεύτερης τάξης) οι εξισώσεις των κατηγοριών αυτών έχουν την ιδιότητα ότι μπορούν ν' αναχθούν σε διαφορικές εξισώσεις πρώτης τάξης. Σε καθένα απ' τα εδάφια 1-4 δίνονται μερικά παραδείγματα και προτείνονται ασκήσεις για λύση, ενώ το Εδάφιο 5 περιέχει μια συλλογή γενικών ασκήσεων.
\section{Γραμμικές διαφορικές εξισώσεις πρώτης τάξης. Διαφορικές εξισώσεις \eng{Bernoulli}. Διαφορικές εξισώσεις \eng{Riccati}}
Στο Εδάφιο αυτό θα μελετηθούν οι γραμμικές διαφορικές εξισώσεις πρώτης τάξης και δύο άλλες κατηγορίες διαφορικών εξισώσεων πρώτης τάξης (οι διαφορικές εξισώσεις \eng{Bernoulli} και οι διαφορικές εξισώσεις \eng{Riccati}) που ανάγονται σε γραμμικές διαφορικές εξισώσεις πρώτης τάξης. Συγκεκριμένα, για μια γραμμική διαφορική εξίσωση πρώτης τάξης θα δοθεί ένας τύπος που δίνει όλες τις λύσεις της˙ για μια διαφορική εξίσωση \eng{Bernoulli} θα δοθεί ο μετασχηματισμός που την ανάγει σε μια γραμμική διαφορική εξίσωση πρώτης τάξης˙ για μια διαφορική εξίσωση \eng{Riccati} θα δοθεί επίσης ένας μετασχηματισμός που τη μετατρέπει σε μια γραμμική διαφορική εξίσωση πρώτης τάξης, με την προϋπόθεση ότι είναι γνωστή μια (μερική) λύση της. Για καθεμιά απ' τις τρεις κατηγορίες διαφορικών εξισώσεων που θα εξετασθούν θα δοθούν παραδείγματα για την καλύτερη κατανόηση των μεθόδων επίλυσης. Επίσης, θα προταθούν μερικές ασκήσεις για λύση.

\subsection{Γραμμικές διαφορικές εξισώσεις πρώτης τάξης}

Μια γραμμική διαφορική εξίσωση πρώτης τάξης είναι μια διαφορική εξίσωση της μορφής
\begin{equation}\label{eq:E}
y' + p(x)y = q(x), \tag{E}
\end{equation}
όπου p και q είναι γνωστές συναρτήσεις της ανεξάρτητης μεταβλητής x που είναι συνεχείς σ' ένα διάστημα της πραγματικής ευθείας.

Πολλαπλασιάζοντας και τα δύο μέλη της (E) με $e^{\int p(x)\d x}$, παίρνουμε
\[
y'e^{\int p(x)\d x} + p(x)ye^{\int p(x)\d x} = q(x)e^{\int p(x)\d x}
\]
ή
\[
\left[y e^{\int p(x)\d x}\right]' = q(x)e^{\int p(x)\d x}.
\]
Έτσι, έχουμε
\[
y e^{\int p(x)\d x} = c + \int q(x)e^{\int p(x)\d x}\d x,
\]
όπου $c$ είναι μια αυθαίρετη σταθερά. Καταλήγουμε λοιπόν στο συμπέρασμα: Οι λύσεις της γραμμικής διαφορικής εξίσωσης πρώτης τάξης (Ε) δίνονται απ' τον τύπο
\[
y = e^{-\int p(x)\d x} \left[ c + \int q(x)e^{\int p(x)\d x}\d x \right],
\]
όπου $c$ είναι μια αυθαίρετη σταθερά.

\begin{Paradeigma}{Να επιλυθεί η διαφορική εξίσωση}
$y'+(\tan{x})y = \sin x$.
\end{Paradeigma}
\lysh Η διαφορική αυτή εξίσωση είναι μια γραμμική διαφορική εξίσωση πρώτης τάξης της μορφής (Ε) με $p(x)=\tan{x}$ και $q(x)=\sin x$. Έτσι, όλες οι λύσεις της είναι
\begin{align*}
y &= e^{-\int \tan{x}\d x} \left[ c+\int (\sin x)e^{\int \tan{x}\d x}\d x \right] = e^{\log|\cos x|} \left[ c+\int (\sin x)e^{-\log|\cos x|}\d x \right] \\
&= |\cos x| \left[ c+\int \frac{\sin x}{|\cos x|}\d x \right] = |\cos x| (c-\log|\cos x|).
\end{align*}
Οι λύσεις λοιπόν της διαφορικής μας εξίσωσης δίνονται απ' τον τύπο
\[
y = |\cos x|(c-\log|\cos x|) \quad (\text{c αυθαίρετη σταθερά}).
\]

\begin{Paradeigma}{Να επιλυθεί το πρόβλημα αρχικών τιμών}
$xy'-2y = -x^2, \ y(1)=0$.
\end{Paradeigma}
\lysh Η διαφορική εξίσωση γράφεται
\[
y' - \frac{2}{x}y = -x
\]
και είναι μια γραμμική διαφορική εξίσωση πρώτης τάξης. Οι λύσεις της είναι
\[
y = e^{-\int(-\frac{2}{x})\d x} \left[ c+\int(-x)e^{\int(-\frac{2}{x})\d x}\d x \right] = e^{\log x^2} \left( c - \int xe^{-\log x^2}\d x \right) = x^2 \left( c - \int x\frac{1}{x^2}\d x \right),
\]
δηλαδή
\[
y = x^2(c-\log|x|) \quad (\text{c αυθαίρετη σταθερά}).
\]
Για τη λύση $y$ που πληροί την αρχική συνθήκη $y(1)=0$ θα είναι $0=1^2(c-\log 1)$, δηλαδή $c=0$. Ακόμα, γι' αυτή τη λύση θα περιορισθούμε στο διάστημα $(0,\infty)$. Έτσι, η λύση του προβλήματος αρχικών τιμών είναι
\[
y = -x^2\log x.
\]
\subsubsection{Διαφορικές εξισώσεις \eng{Bernoulli}}

Οι διαφορικές εξισώσεις \eng{Bernoulli} είναι της μορφής
\begin{equation}\label{eq:Bernoulli}
y' + a(x)y = b(x)y^r, \tag{E}
\end{equation}
όπου a και b είναι συνεχείς συναρτήσεις της ανεξάρτητης μεταβλητής x (σ' ένα διάστημα) και r είναι ένας πραγματικός αριθμός με $r \neq 0, r \neq 1$. (Ο λόγος για τους περιορισμούς $r\neq 0$ και $r\neq 1$ είναι ότι για $r=0$ ή $r=1$ η (Ε) είναι μια γραμμική διαφορική εξίσωση πρώτης τάξης.
Η αντικατάσταση $z=y^{1-r}$ μετασχηματίζει τη διαφορική εξίσωση \eng{Bernoulli} (Ε) σε μια γραμμική διαφορική εξίσωση πρώτης τάξης. Πραγματικά: Για $z=y^{1-r}$ έχουμε $z'=(1-r)y^{-r}y'$. Η (Ε) όμως γράφεται
\[
y^{-r}y' + a(x)y^{1-r} = b(x)
\]
και έτσι αυτή γίνεται
\[
\frac{1}{1-r}z' + a(x)z = b(x)
\]
ή
\[
z' + (1-r)a(x)z = (1-r)b(x).
\]
Η τελευταία εξίσωση είναι μια γραμμική διαφορική εξίσωση πρώτης τάξης.

\begin{Paradeigma}{Να επιλυθεί η διαφορική εξίσωση}
$(x-2)y' + y = 7(x-2)^3y^{1/2}$. \\
Ιδιαίτερα, να βρεθεί η μη μηδενική λύση $y_0$ αυτής που πληροί την αρχική συνθήκη $y_0(3)=0$.
\end{Paradeigma}
\lysh Η διαφορική αυτή εξίσωση γράφεται
\[
y' + \frac{1}{x-2}y = 7(x-2)^2y^{1/2}
\]
και είναι μια διαφορική εξίσωση Bernoulli με $r=1/2$. Αυτή έχει τη μηδενική λύση $y=0$. Για να βρούμε τις άλλες λύσεις τη γράφουμε
\[
y^{-1/2}y' + \frac{1}{x-2}y^{1/2} = 7(x-2)^2
\]
και κάνουμε την αντικατάσταση $y^{1/2}=z$, οπότε $\frac{1}{2}y^{-1/2}y' = z'$. Έτσι, η εξίσωσή μας γίνεται
\[
z' + \frac{1}{2(x-2)}z = \frac{7}{2}(x-2)^2.
\]
Η τελευταία εξίσωση είναι μια γραμμική διαφορική εξίσωση πρώτης τάξης. Επίσης και οι λύσεις της είναι
\[
z = e^{-\int\frac{\d x}{2|x-2|}} \left[ c + \int \frac{7}{2}(x-2)^2 e^{\int\frac{\d x}{2|x-2|}}\d x \right] = |x-2|^{-1/2}(c+|x-2|^{7/2}),
\]
δηλαδή
\[
z = c|x-2|^{-1/2} + |x-2|^3.
\]
Έτσι, οι λύσεις της διαφορικής εξίσωσής μας θα δίνονται απ' τους τύπους
\[
y=0 \quad \text{και} \quad y=(c|x-2|^{-1/2} + |x-2|^3)^2 \quad (\text{c αυθαίρετη σταθερά}).
\]
Ειδικά, για τη λύση $y_0$ που πληροί την αρχική συνθήκη $y_0(3)=0$ θα περιορισθούμε στο διάστημα $(2, \infty)$ και επιπλέον θα έχουμε $0=(c+1)^2$, δηλαδή $c=-1$. Άρα
\[
y_0 = \left[ -(x-2)^{-1/2} + (x-2)^3 \right]^2.
\]

\begin{Paradeigma}{Να επιλυθεί το πρόβλημα αρχικών τιμών}
$y'-\frac{1}{x}y = -\frac{y^2}{x}; \ y(-1)=2$.
\end{Paradeigma}
\lysh Η διαφορική εξίσωση είναι μια διαφορική εξίσωση Bernoulli με r=1. Γράφουμε την εξίσωση ως εξής
\[
yy'^{-1} - \frac{1}{x}y^2 = -\frac{1}{x}
\]
και θέτουμε $y^{-2}=z$. Τότε $2yy'=z'$ και η διαφορική εξίσωση μετασχηματίζεται στη γραμμική διαφορική εξίσωση πρώτης τάξης
\[
z' - \frac{z}{x} = z-1.
\]
Η τελευταία εξίσωση έχει τις λύσεις
\[
z = e^{\int \frac{2\d x}{x}} \left[ c + \int(-1)e^{-\int \frac{2\d x}{x}}\d x \right] = x^2\left( c+\frac{1}{x} \right) = cx^2+x,
\]
όπου c είναι αυθαίρετη σταθερά. Έτσι, οι λύσεις της διαφορικής μας εξίσωσης δίνονται απ' τους τύπους
\[
y = \pm\sqrt{cx^2+x} \quad (\text{c αυθαίρετη σταθερά}).
\]
Για τη λύση y με $y(-1)=2$ έχουμε $2=\sqrt{c(-1)^2+(-1)}$, δηλαδή $c=5$. Επομένως, η λύση του προβλήματος αρχικών τιμών είναι
\[
y = \sqrt{5x^2+x}.
\]
\subsubsection{Διαφορικές εξισώσεις \eng{Riccati}}

Μια διαφορική εξίσωση \eng{Riccati} είναι μια διαφορική εξίσωση της μορφής
\begin{equation}\label{eq:Riccati}
y' + a(x)y + b(x)y^2 + d(x) = 0, \tag{E}
\end{equation}
όπου a, b και d είναι γνωστές συνεχείς συναρτήσεις του x (σ' ένα διάστημα) και $d \neq 0$ (αν $d=0$ τότε η (Ε) είναι μια διαφορική εξίσωση \eng{Bernoulli} με $r=2$). Δεν υπάρχει γενική μέθοδος για την επίλυση των διαφορικών εξισώσεων \eng{Riccati}. Αν όμως είναι γνωστή μια (μερική) λύση μιας διαφορικής εξίσωσης \eng{Riccati}, τότε μ' ένα κατάλληλο μετασχηματισμό αυτή μετατρέπεται σε μια γραμμική διαφορική εξίσωση πρώτης τάξης. Πιο συγκεκριμένα:
\underline{Αν $y_1$ είναι μια (μερική) λύση της διαφορικής εξίσωσης \eng{Riccati} (Ε), τότε η αντικατάσταση $y=y_1+\frac{1}{z}$ μετασχηματίζει την (Ε) σε μια γραμμική διαφορική εξίσωση πρώτης τάξης.}
Πραγματικά: Είναι $y=y_1+\frac{1}{z}$ και $y'=y_1'-\frac{1}{z^2}z'$ και έτσι η (Ε) γίνεται
\[
y_1' - \frac{1}{z^2}z' + a(x)\left(y_1+\frac{1}{z}\right) + b(x)\left(y_1+\frac{1}{z}\right)^2 + d(x) = 0
\]
ή
\[
[y_1' + a(x)y_1 + b(x)y_1^2+d(x)] - \frac{1}{z^2}z' - [a(x)+2y_1b(x)]z - b(x) = 0
\]
ή ακόμα (αφού η $y_1$ είναι μια λύση της (Ε))
\[
z' - [a(x)+2y_1b(x)]z = b(x).
\]
Η τελευταία εξίσωση είναι μια γραμμική διαφορική εξίσωση πρώτης τάξης.

\begin{Paradeigma}{1. Να επιλυθεί η διαφορική εξίσωση}
$y'-\frac{1}{x}y - x^3y^2+x^5=0$, \\
αφού πρώτα διαπιστωθεί ότι $y=x$ είναι μια λύση της.
\end{Paradeigma}
Η εξίσωση αυτή είναι μια διαφορική εξίσωση \eng{Riccati}. Εύκολα διαπιστώνουμε ότι $y_1=x$ είναι μια λύση της. Θέτουμε, στη συνέχεια, $y=y_1+\frac{1}{z}=x+\frac{1}{z}$. Τότε $y'=1-\frac{1}{z^2}z'$ και η διαφορική εξίσωση γίνεται
\[
1-\frac{1}{z^2}z' - \frac{1}{x}(x+\frac{1}{z}) - x^3(x+\frac{1}{z})^2+x^5 = 0
\]
ή (μετά από πράξεις) \[
z'+\left(2x^4+\frac{1}{x}\right)z = -x^3.
\]
Οι λύσεις της παραπάνω γραμμικής διαφορικής εξίσωσης πρώτης τάξης είναι
\[
z = \frac{c}{|x|}e^{-(2/5)x^5} - \frac{x^5}{2},
\]
όπου $c$ είναι αυθαίρετη σταθερά. Έτσι, οι λύσεις της διαφορικής μας εξίσωσης δίνονται απ' τους τύπους
\[
y=x \quad \text{και} \quad y=x+1/\left[\frac{c}{|x|}e^{-(2/5)x^5}-\frac{x^5}{2}\right] \quad (\text{c αυθαίρετη σταθερά}).
\]
\begin{Paradeigma}{Να επιλυθεί η διαφορική εξίσωση}
$y'+y-x^2y^2-e^x=0$, \\
αφού πρώτα βρεθεί μια λύση αυτής $y_1$ της μορφής $y_1=ke^{\lambda x}$ (k, $\lambda$ σταθερές). Στη συνέχεια, να βρεθεί η λύση $y_0$ που πληροί την αρχική συνθήκη $y_0(0)=1/3$.
\end{Paradeigma}
Η συνάρτηση $y_1=ke^{\lambda x}$ θα είναι μια λύση της διαφορικής εξίσωσης αν και μόνο αν
\[
k\lambda e^{\lambda x} - ke^{\lambda x} - x^2k^2e^{2\lambda x} - e^x = 0
\]
ή
\[
k(\lambda-1)e^{\lambda x} + [k^2e^{2(\lambda-1)x}]e^x = 0.
\]
Για $\lambda=1$ και $k=1$ είναι φανερό ότι η τελευταία ισότητα αληθεύει για όλα τα $x$. Έτσι, έχουμε τη μερική λύση $y_1=e^x$. Τώρα, θέτουμε $y=y_1+\frac{1}{z}=e^x+\frac{1}{z}$, οπότε $y'=e^x-\frac{1}{z^2}z'$ και η διαφορική εξίσωση γίνεται
\[
e^x - \frac{1}{z^2}z' - e^x - x^2\left(e^x+\frac{1}{z}\right)^2 - e^x = 0
\]
ή
\[
z'-z-e^{-x}.
\]
Οι λύσεις αυτής της γραμμικής διαφορικής εξίσωσης πρώτης τάξης είναι
\[
z = ce^x-\frac{1}{2}e^{-x},
\]
όπου $c$ είναι μια αυθαίρετη σταθερά. Επομένως, οι λύσεις της αρχικής διαφορικής εξίσωσης δίνονται απ' τους τύπους
\[
y=e^x \quad \text{και} \quad y=e^x+\frac{2}{2ce^x-e^{-x}} \quad (c\text{ αυθαίρετη σταθερά}).
\]
Για την λύση $y_0$ θα είναι $1/3=1+2/(2c-1)$ και άρα $c=-1$. Επομένως,
\[
Y_0 = e^x - 2/(2e^x-e^{-x}).
\]

\subsection{Ασκήσεις}
\begin{enumerate}
\item Να επιλυθούν οι διαφορικές εξισώσεις:
\begin{multicols}{2}
\begin{rlist}
\item $y'-y=2e^x$.
\item $xy'-y=x^2e^x$.
\item $(3x^2+1)y'-2xy=6x$.
\item $x(x+1)y'+(2x+1)y=x^2-1$.
\item $y' + \frac{1}{\sin x} y = \text{tg}\,x$.
\item $(1+x^2)y'+xy=x$.
\item $y'+y \cos x = 0$.
\item $y'+\frac{y}{x}\sin x = 0$.
\end{rlist}
\end{multicols}

\item Να επιλυθούν τα παρακάτω προβλήματα αρχικών τιμών:
\begin{rlist}
\item $y'+(1/x)y=0, y(0)=\sqrt{5}$.
\item $y'+e^{-x}\sqrt{1+x^2}y=0, y(0)=0$.
\item $y'+y = \frac{1}{1+e^x}, y(1)=2$.
\item $(1+x^2)y'+4xy=x, y(1)=1/4$.
\item $y'+\frac{1}{\sqrt{x}}y=e^{\sqrt{x}}/2, y(1)=-1$.
\end{rlist}

\item Να επιλυθεί το πρόβλημα αρχικών τιμών
$y'+y=g(x), y(0)=0$ με: $g(x)=2$ για $x \in [0,1]$, $g(x)=0$ για $x>1$.

\item Να επιλυθούν οι διαφορικές εξισώσεις:
\begin{multicols}{2}
\begin{rlist}
\item $xy'+y=-2x^6y^4$.
\item $xy' - \frac{1}{x \log x}y = y^2$.
\item $2xy'+y=2x^2y^{-3}$.
\item $xy'+y=(xy)^{3/2}$.
\end{rlist}
\end{multicols}

\item Να επιλυθούν οι παρακάτω διαφορικές εξισώσεις, αφού πρώτα βρεθεί για καθεμιά απ' αυτές μια μερική λύση $y_1$ της μορφής που σημειώνεται:
\begin{rlist}
\item $y'-xy^2-(1/x)y+x^3=0, y_1=ax+b$ (a,b σταθερές).
\item $y'+y^2=1+x^2, y_1=ax$ (a σταθερά).
\item $y'-y+e^{-x}y^2=e^x, y_1=ke^{\lambda x}$ (k,$\lambda$ σταθερές).
\end{rlist}

\item Να επιλυθούν τα προβλήματα αρχικών τιμών:

\begin{rlist}
\item $(x-1)y'-3y = (x-1)^5, y(-1)=16$.
\item $y'-xy(x^2-1)^{1/2}, y(0)=0$.
\item $y'+xy=y^2e^{x^2}, y(0)=1/2$.
\item $2x^3y'=y(y^2+3x^2), y(1)=1$.
\item $y'=4y+2e^x\sqrt{y}, y(0)=-1$.
\end{rlist}
\end{enumerate}
\section{Διαφορικές εξισώσεις χωριζομένων μεταβλητών. Ομογενείς διαφορικές εξισώσεις}

Το Εδάφιο αυτό αναφέρεται στις διαφορικές εξισώσεις χωριζομένων μεταβλητών καθώς επίσης και στις ομογενείς διαφορικές εξισώσεις, οι οποίες με κατάλληλο μετασχηματισμό ανάγονται σε εξισώσεις χωριζομένων μεταβλητών. Για καθεμιά απ' τις δύο αυτές κατηγορίες διαφορικών εξισώσεων δίνεται η μέθοδος επίλυσης, για την καλύτερη κατανόηση της οποίας παρατίθενται παραδείγματα. Επίσης, προτείνονται για λύση μερικές ασκήσεις.

\subsection{Διαφορικές εξισώσεις χωριζομένων μεταβλητών}

Μια διαφορική εξίσωση χωριζομένων μεταβλητών είναι μια διαφορική εξίσωση της μορφής
\begin{equation}\label{eq:separable}
y' = \frac{P(x)}{Q(y)}, \tag{E}
\end{equation}
όπου $P$ είναι μια συνεχής συνάρτηση του $x$ και $Q$ είναι μια συνεχής συνάρτηση του $y$. Η (Ε) γράφεται στη μορφή
\begin{equation}\label{eq:separable_int}
Q(y)\d y = P(x)\d x \tag{E'}
\end{equation}
και επομένως οι λύσεις της δίνονται απ' τον τύπο
\[
\int Q(y)\d y = \int P(x)\d x+c \quad (c\text{ αυθαίρετη σταθερά}).
\]

\begin{Paradeigma}{}
Να επιλυθεί η διαφορική εξίσωση
$x\,\d x - (5y^2+3)\d y = 0$.
\end{Paradeigma}
Οι λύσεις της διαφορικής εξίσωσης δίνονται απ' τον τύπο
\[
\int x\,\d x = \int (5y^4+3)\d y
\]
ή
\[
\frac{x^2}{2} = y^5+3y+c
\]
ή ακόμα
\[
y^5+3y-\frac{x^2}{2}+c = 0 \quad (\text{c αυθαίρετη σταθερά}).
\]

\begin{Paradeigma}{}
Να επιλυθεί η διαφορική εξίσωση
$2x(y^2+y)\d x+(x^2-1)y\,\d y=0$.
\end{Paradeigma}
Η διαφορική εξίσωση αυτή έχει τις λύσεις $y=0, y=-1$. Για να βρούμε τις άλλες λύσεις της τη γράφουμε στη μορφή
\[
\frac{2x}{x^2-1}\d x = -\frac{y}{y+1}\d y,
\]
οπότε έχουμε
\[
\int \frac{2x}{x^2-1}\d x = -\int \frac{1}{y+1}\d y,
\]
όπου $c$ είναι μια αυθαίρετη σταθερά. Έτσι, είναι
\[
\log|x^2-1| = -\log|y+1|+c
\]
ή
\[
(x^2-1)(y+1)=\pm e^c.
\]
θέτουμε λοιπόν $\pm e^c=C$ (είναι φανερό ότι $C \neq 0$) και επιλύουμε την παραπάνω σχέση ως προς $y$, οπότε παίρνουμε
\[
y = -1 + \frac{C}{x^2-1}.
\]
Για $C = 0$ ο τύπος αυτός δίνει τη λύση $y=-1$. Έτσι, όλες οι λύσεις της διαφορικής εξίσωσης δίνονται απ' τους τύπους
\[
y=0 \quad \text{και} \quad y=-1+\frac{C}{x^2-1} \quad (C\text{ αυθαίρετη σταθερά}).
\]

\begin{Paradeigma}{}
Να επιλυθεί το πρόβλημα αρχικών τιμών $(y^2-1)\d x+y(x-1)\d y=0, y(0)=-2$.
\end{Paradeigma}
Η εξίσωση έχει τις λύσεις $y=1, y=-1$ που δεν μας ενδιαφέρουν εδώ γιατί καμιά απ' αυτές δεν πληροί την αρχική συνθήκη. Γράφουμε λοιπόν την εξίσωση στη μορφή
\[
\frac{1}{x-1}\d x = -\frac{y}{y^2-1}\d y
\]
και παίρνουμε
\[
\int \frac{\d x}{x-1} = -\int \frac{y\,\d y}{y^2-1} + c
\]
ή
\[
\log|x-1| = -\frac{1}{2} \log|y^2-1| + c
\]
ή ακόμα
\[
(x-1)^2(y^2-1) = \pm e^{2c}
\]
δηλαδή
\[
y = \pm\sqrt{1+\frac{C}{(x-1)^2}},
\]
όπου $C = \pm e^{2c} \neq 0$. Για τη λύση του προβλήματος αρχικών τιμών θα έχουμε με $-2 = -\sqrt{1+C}$, δηλαδή $C=3$. Άρα, η ζητούμενη λύση είναι
\[
y = -\sqrt{1+\frac{3}{(x-1)^2}}.
\]

\subsection{Ομογενείς διαφορικές εξισώσεις}

Μια συνάρτηση $f(x,y)$ λέμε ότι είναι ομογενής βαθμού n αν και μόνο αν $f(\lambda x, \lambda y) = \lambda^n f(x,y)$. Μια διαφορική εξίσωση πρώτης τάξης της μορφής
\begin{equation}\label{eq:homogen}
y' = \frac{g(x,y)}{h(x,y)}, \tag{E}
\end{equation}
όπου οι συναρτήσεις g και h είναι ομογενείς του ίδιου βαθμού, λέμε ότι είναι μια ομογενής διαφορική εξίσωση.

Αν θέσουμε $z=y/x$, τότε η ομογενής διαφορική εξίσωση (Ε) γίνεται
\[
(xz)' = \frac{g(x,xz)}{h(x,xz)} = \frac{x^n g(1,z)}{x^n h(1,z)}
\]
ή ακόμα
\[
xz' = \frac{g(1,z)}{h(1,z)}-z,
\]
όπου n είναι ο βαθμός ομογένειας των $g$ και $h$. Η τελευταία διαφορική εξίσωση μπορεί να γραφεί στη μορφή μιας διαφορικής εξίσωσης χωριζομένων μεταβλητών. Έχουμε λοιπόν το συμπέρασμα:

Η αντικατάσταση $z=y/x$ μετασχηματίζει την ομογενή διαφορική εξίσωση (Ε) σε μια διαφορική εξίσωση χωριζομένων μεταβλητών.\\\\
\begin{Paradeigma}{}
Να επιλυθεί η διαφορική εξίσωση
$y' = \frac{x^3+y^3}{xy^2}$.
\end{Paradeigma}
\lysh Η διαφορική αυτή εξίσωση είναι μια ομογενής διαφορική εξίσωση. θέτουμε λοιπόν $y=xz$, οπότε $y'=xz'+z$ και η εξίσωση γίνεται
\[
xz'+z = \frac{1+z^3}{z^2}
\]
ή
\[
x\frac{z'}{x} = \frac{1}{z^2}.
\]
Η τελευταία εξίσωση είναι χωριζομένων μεταβλητών και δίνει
\[
z^3 = 3\log|x|+c,
\]
όπου c είναι αυθαίρετη σταθερά. Έτσι, οι λύσεις της αρχικής διαφορικής εξίσωσης δίνονται απ' τον τύπο
\[
y=x(3\log|x|+c)^{1/3} \quad (\text{c αυθαίρετη σταθερά}).
\]

\begin{Paradeigma}{}
Να επιλυθεί το πρόβλημα αρχικών τιμών
$(x^2+y^2)\d x+2xy\,\d y=0, y(1)=-1$.
\end{Paradeigma}
\lysh Η διαφορική εξίσωση γράφεται ως εξής
\[
y' = -\frac{x^2+y^2}{2xy}
\]
και έτσι είναι μια ομογενής διαφορική εξίσωση. Η αντικατάσταση $y=xz$ τη μετασχηματίζει στην εξίσωση
\[
xz' = -\frac{1+3z^2}{2z},
\]
η οποία γράφεται
\[
\frac{2z}{1+3z^2}dz+\frac{1}{x}\d x=0.
\]
Η εξίσωση αυτή είναι χωριζομένων μεταβλητών και οι λύσεις της δίνονται απ' τον τύπο
\[
(1+3z^2)x^3=c,
\]
όπου $c$ είναι αυθαίρετη σταθερά με $c\neq 0$. Έτσι, οι λύσεις της αρχικής διαφορικής εξίσωσης είναι
\[
y = \pm\sqrt{\frac{c-x^3}{3x}} \quad (c\neq 0 \text{ αυθαίρετη σταθερά}).
\]
Η λύση του προβλήματος αρχικών τιμών προκύπτει απ' τον παραπάνω τύπο με το $-$ και για την τιμή της σταθεράς $c$ για την οποία $-1=-\sqrt{\frac{c-1}{3}}$, δηλαδή $c=4$. Άρα, η λύση του προβλήματος αρχικών τιμών είναι
\[
y = -\sqrt{\frac{4-x^3}{3x}}.
\]

\subsection{Ασκήσεις}
\begin{enumerate}
\item Να επιλυθούν οι διαφορικές εξισώσεις:
\begin{enumerate}[label=(\roman*)]
\item $(x^2+y^2)\d x+(xy^2-y^2)\d y=0$.
\item $xy \d x + (1+x^2)\d y=0$.
\item $(y^2-1)x\,\d x + (1-x)\d y=0$.
\item $x\cos y \cos x + y' \sin y = 0$.
\item $e^{x+y}\sin x\,\d x+(2y+1)e^{-y^2}\d y=0$.
\item $y'=e^{x-y}$.
\end{enumerate}

\item Να επιλυθούν τα προβλήματα αρχικών τιμών:
\begin{enumerate}[label=(\roman*)]
\item $\cos y\,\d x+(1-e^{-x})\sin y\,\d y = 0, y(0)=\pi/4$.
\item $(y+x^2y)y' = x, y(1)=0$.
\item $(x^2y^2+x^2+1)\d x+(y-1)\d y=0, y(2)=0$.
\item $y^2\d x+3x^2e^{y^2}\d y=0, y(1)=0$.
\end{enumerate}

\item Να επιλυθούν οι διαφορικές εξισώσεις:
\begin{enumerate}[label=(\roman*)]
\item $3y\,\d x+(7x-y)\d y=0$.
\item $y' = \frac{y}{x} - \frac{x}{y}$.
\item $(xe^{y/x}+y)\d x-x\,\d y=0$.
\item $(y+\sqrt{x^2-y^2})\d x-x\,\d y=0$.
\end{enumerate}

\item Να επιλυθούν τα προβλήματα αρχικών τιμών:
\begin{enumerate}[label=(\roman*)]
\item $x^2\d x = (3xy+2y^2)\d y, y(3)=-2$.
\item $x\sin(y/x)\d y=[x+y\sin(y/x)]\d x, y(1)=0$.
\item $y'=\frac{\sqrt{3x^2+y^2}}{2x}, y(1)=2$.
\item $x\,\d y = [x+\log(y/x)]y\,\d x, y(e)=1$.
\end{enumerate}

\item Ν' αποδειχθεί ότι η διαφορική εξίσωση
\[
y'=\frac{\alpha x + \beta y + \gamma}{\alpha_1 x + \beta_1 y + \gamma_1},
\]
όπου $\alpha, \beta, \gamma, \alpha_1, \beta_1, \gamma_1$ είναι σταθερές με $\alpha\beta_1-\alpha_1\beta \neq 0$, μετασχηματίζεται σε μια ομογενή διαφορική εξίσωση με την αντικατάσταση
\[
x=X+x_0, \quad y=Y+y_0,
\]
\end{enumerate}
όπου $x_0, y_0$ είναι τέτοια ώστε
\[
\alpha x_0 + \beta y_0 + \gamma = 0, \quad \alpha_1 x_0 + \beta_1 y_0 + \gamma_1 = 0.
\]
Στη συνέχεια, να επιλυθούν οι διαφορικές εξισώσεις:
\begin{enumerate}[label=(\roman*)]
\setcounter{enumi}{5}
\item $(x-y+3)dx+(x+2y-3)dy=0$.
\item $(x+y)dx+(2x+2y+3)dy=0$.
\item $(iii) 2xdx+(x-y+1)dy=0$.
\end{enumerate}
Επίσης, να επιλυθούν τα προβλήματα αρχικών τιμών:
\begin{enumerate}[label=(\roman*)']
\item $y'=\frac{y+x+2}{y-x}$, $y(1)=1$.
\item $(x+y+1)y' = 2x+y-4$, $y(2)=2$.
\end{enumerate}

\section{Διαφορικές εξισώσεις αμέσως ολοκληρώσιμες. Ολοκληρωτικοί παράγοντες}

Στο εδάφιο αυτό θα εξετασθεί μια γενική κατηγορία διαφορικών εξισώσεων που είναι γνωστές ως διαφορικές εξισώσεις αμέσως ολοκληρώσιμες. Επίσης, θα θεωρηθούν διαφορικές εξισώσεις πρώτης τάξης που γίνονται διαφορικές εξισώσεις αμέσως ολοκληρώσιμες με πολλαπλασιασμό των δύο μελών τους με μια κατάλληλη συνάρτηση (ολοκληρωτικό παράγοντα). θα παρατεθούν μερικά παραδείγματα και θα δοθούν ασκήσεις για λύση.

\subsection{Διαφορικές εξισώσεις αμέσως ολοκληρώσιμες}

Ας θεωρήσουμε τη διαφορική εξίσωση πρώτης τάξης
\begin{equation}\label{eq:exact}
M(x,y)dx+N(x,y)dy=0, \tag{E}
\end{equation}
όπου M και N είναι γνωστές συναρτήσεις. θα λέμε ότι η (Ε) είναι μια \textbf{διαφορική εξίσωση αμέσως ολοκληρώσιμη} αν και μόνο αν υπάρχει μια συνάρτηση $F(x,y)$ τέτοια ώστε
\begin{equation}\label{eq:exact_cond}
dF(x,y) = M(x,y)dx+N(x,y)dy. \tag{*}
\end{equation}
Αν λοιπόν η διαφορική εξίσωση (Ε) είναι μια διαφορική εξίσωση αμέσως ολοκληρώσιμη, τότε για κάποια συνάρτηση $F(x,y)$ θα ισχύει η (*) και άρα η (Ε) θα γράφεται ως εξής
\[
dF(x,y) = 0,
\]
οπότε οι λύσεις αυτής θα δίνονται απ' τον τύπο
\[
F(x,y) = c \quad (\text{c αυθαίρετη σταθερά}).
\]
Είναι γνωστό (απ' τον διαφορικό λογισμό συναρτήσεων περισσοτέρων της μιας μεταβλητών) ότι υπάρχει μια συνάρτηση F τέτοια ώστε $dF = Mdx+Ndy$ αν και μόνο αν
Οι συναρτήσεις M και N είναι συνεχείς και έχουν συνεχείς μερικές παραγώγους σ' ένα απλώς συνεκτικό σύνολο και ισχύει
\[
\frac{\partial M}{\partial y} = \frac{\partial N}{\partial x}.
\]

Έτσι, η (Ε) είναι μια διαφορική εξίσωση αμέσως ολοκληρώσιμη αν και μόνο αν ισχύει η συνθήκη (C).

Ας σημειώσουμε ακόμα ότι η (*) ισχύει αν και μόνο αν
\[
\frac{\partial F}{\partial x} = M \quad \text{και} \quad \frac{\partial F}{\partial y} = N.
\]

\begin{Paradeigma}{}
Να επιλυθεί η διαφορική εξίσωση
$(e^x+3y)dx+(3x+\cos y)dy=0$.
\end{Paradeigma}
\lysh Η διαφορική αυτή εξίσωση είναι της μορφής (Ε) με $M(x,y) = e^x+3y$, $N(x,y) = 3x+\cos y$. Έχουμε
\[
\frac{\partial M}{\partial y} = 3 = \frac{\partial N}{\partial x}
\]
και άρα η εξίσωση μας είναι αμέσως ολοκληρώσιμη. Υπάρχει επομένως μια συνάρτηση $F(x,y)$ έτσι ώστε η διαφορική εξίσωση να γίνεται $dF=0$. Τότε
\[
\frac{\partial F}{\partial x} = M=e^x+3y, \quad \frac{\partial F}{\partial y} = N=3x+\cos y.
\]
Ο προσδιορισμός μιας συνάρτησης f με την παραπάνω ιδιότητα μπορεί να γίνει με ένα απ' τους πιο κάτω τρόπους (ας σημειωθεί ότι, αν $dF=0$, τότε για κάθε συνάρτηση της μορφής $F+c$, όπου C είναι μια σταθερά, θα είναι $d(F+c)=0$):
\begin{enumerate}
\item[Τρόπος 1.] Είναι
\[
F(x,y) = \int (e^x+3y)dx+h(y) = e^x+3yx+h(y)
\]
για κάποια συνάρτηση h(y). Τότε όμως έχουμε
\[
\frac{\partial F}{\partial y} = 3x+h'(y)
\]
και άρα
\[
3x+h'(y) = 3x+\cos y.
\]
\end{enumerate}
Επομένως, $h'(y) = \cos y$ και έτσι μπορούμε να εκλέξουμε $h(y)=\sin y$, οπότε
\[
F(x,y) = e^x+3yx+\sin y.
\]
\begin{enumerate}
\item[Τρόπος 2.] Παίρνουμε
\[
F(x,y) = \int (3x+\cos y)dy+g(x) = 3xy+\sin y + g(x)
\]
για κάποια συνάρτηση g(x), οπότε
\[
\frac{\partial F}{\partial x} = 3y+g'(x) = e^x+3y,
\]
δηλαδή $g'(x)=e^x$. Παίρνουμε $g(x)=e^x$ και βρίσκουμε τότε πάλι την ίδια συνάρτηση F.

\item[Τρόπος 3.] Έχουμε
\[
F(x,y) = e^x+3yx+h(y) \quad \text{και} \quad F(x,y) = 3xy+\sin y + g(x)
\]
για κάποιες συναρτήσεις h(y) και g(x). Επομένως, $h(y)=\sin y$ και $g(x)=e^x$, γιατί είναι $h(y)-\sin y=g(x)-e^x$. Έτσι, βρίσκουμε πάλι τη συνάρτηση F.
\end{enumerate}
Τέλος, όλες οι λύσεις της διαφορικής μας εξίσωσης δίνονται απ' τον τύπο
\[
e^x+3yx+\sin y = c \quad (\text{c αυθαίρετη σταθερά}).
\]

\begin{Paradeigma}{Να επιλυθεί το πρόβλημα αρχικών τιμών}
$(e^{-x}+2ye^{2x}\cos y)dx+(2y-\frac{1}{y}e^{-x}x^2\sin y)dy = 0, y(0)=1$.
\end{Paradeigma}
\lysh Η εξίσωση είναι μια διαφορική εξίσωση αμέσως ολοκληρώσιμη γιατί
\[
\frac{\partial}{\partial y}(e^{-x}+2ye^{2x}\cos y) = 2e^{2x}\cos y = \frac{\partial}{\partial x}(2y-\frac{1}{y}e^{-x}x^2\sin y),
\]
θεωρούμε μια συνάρτηση $F(x,y)$ με
\[
\frac{\partial F}{\partial x} = e^{-x}+2ye^{2x}\cos y \quad \text{και} \quad \frac{\partial F}{\partial y} = 2y-\frac{1}{y}e^{-x}x^2\sin y.
\]
Τότε
\[
F(x,y) = \int (e^{-x}+2ye^{2x}\cos y)dx+h(y) = -e^{-x}+ye^{2x}x^2\cos y + h(y)
\]
για κάποια συνάρτηση h(y). Έτσι,
\[
\frac{\partial F}{\partial y} = e^{2x}x^2\sin y + h'(y) = 2y-\frac{1}{y}e^{-x}x^2\sin y,
\]
οπότε
\[
h'(y) = 2y - \frac{1}{y}.
\]
Επομένως, μπορούμε να θέσουμε $h(y) = y^2-\log|y|$ και άρα
\[
F(x,y) = -e^{-x}ye^{2x}x^2\cos y + y^2-\log|y|.
\]
Οι λύσεις λοιπόν της διαφορικής μας εξίσωσης δίνονται απ' τον τύπο
\[
-e^{-x}ye^{2x}x^2\cos y + y^2-\log|y| = c,
\]
όπου c αυθαίρετη σταθερά. Για $x=0$ και $y=1$ έχουμε $c=1$ και άρα οι λύσεις του προβλήματος αρχικών τιμών δίνονται απ' τον τύπο
\[
-e^{-x}ye^{2x}x^2\cos y + y^2-\log|y| = 1.
\]

\subsubsection{Ολοκληρωτικοί παράγοντες}

Ας θεωρήσουμε τη διαφορική εξίσωση
\begin{equation}\label{eq:integrating_factor}
M(x,y)dx+N(x,y)dy=0, \tag{E}
\end{equation}
όπου M και N είναι γνωστές συναρτήσεις. Είναι δυνατό η (Ε) να μην είναι μια διαφορική εξίσωση αμέσως ολοκληρώσιμη, αλλά η διαφορική εξίσωση που προκύπτει με τον πολλαπλασιασμό και των δύο μελών της με μια (μη μηδενική) συνάρτηση p(x,y), δηλαδή η εξίσωση
\[
p(x,y)M(x,y)dx+p(x,y)N(x,y)dy=0,
\]
να είναι μια διαφορική εξίσωση αμέσως ολοκληρώσιμη. Μια τέτοια συνάρτηση p(x,y) λέμε ότι είναι ένας \textbf{ολοκληρωτικός παράγοντας} της διαφορικής εξίσωσης (Ε).
Δεν υπάρχει γενική μέθοδος για την εύρεση ενός ολοκληρωτικού παράγοντα για κάθε διαφορική εξίσωση της μορφής (Ε). Σε δύο όμως ειδικές περιπτώσεις μπορούμε να βρούμε ένα ολοκληρωτικό παράγοντα της (Ε). Συγκεκριμένα, ισχύει:
Αν το πηλίκο $\frac{\partial M/\partial y - \partial N/\partial x}{N}$ είναι μια συνάρτηση του x μόνο, τότε η συνάρτηση $p(x)=e^{\int \frac{\partial M/\partial y - \partial N/\partial x}{N}dx}$ είναι ένας ολοκληρωτικός παράγοντας της διαφορικής εξίσωσης (Ε). Ανάλογα, αν $\frac{\partial N/\partial x - \partial M/\partial y}{M}$ είναι μια συνάρτηση του $y$ μόνο, τότε $p(y)=e^{\int \frac{\partial N/\partial x - \partial M/\partial y}{M}dy}$ είναι ένας ολοκληρωτικός παράγοντας της (Ε). Πραγματικά: Ας υποθέσουμε ότι $p$ είναι μια συνάρτηση μόνο του $x$ και ας θέσουμε $p(x)=e^{\int f(x)dx}$. Τότε για τη διαφορική εξίσωση
\[
pMdx+pNdy=0
\]
έχουμε
\[
\frac{\partial (pM)}{\partial y} = p\frac{\partial M}{\partial y}
\]
και
\[
\frac{\partial (pN)}{\partial x} = \frac{dp}{dx}N+p\frac{\partial N}{\partial x} = p'N+p\frac{\partial N}{\partial x} = p\frac{\partial M}{\partial y},
\]
δηλαδή
\[
\frac{\partial (pM)}{\partial y} = \frac{\partial (pN)}{\partial x}.
\]
Αυτό φανερώνει ότι η εξίσωση $pMdx+pNdy=0$ είναι μια διαφορική εξίσωση αμέσως ολοκληρώσιμη και άρα p είναι ένας ολοκληρωτικός παράγοντας της (Ε). Ανάλογα, αποδεικνύεται και το δεύτερο μέρος του συμπεράσματός μας.

\begin{Paradeigma}{}
Να επιλυθεί η διαφορική εξίσωση
$(\frac{y^2}{2}+2ye^x)dx+(y+e^x)dy=0$.
\end{Paradeigma}
\lysh Η διαφορική αυτή εξίσωση είναι της μορφής (Ε) με $M(x,y)=\frac{y^2}{2}+2ye^x$ και $N(x,y)=y+e^x$. Δεν είναι μια διαφορική εξίσωση αμέσως ολοκληρώσιμη, αφού
\[
\frac{\partial M}{\partial y} = y+2e^x \neq \frac{\partial N}{\partial x} = e^x.
\]
Παρατηρούμε όμως ότι
\[
\frac{\frac{\partial M}{\partial y} - \frac{\partial N}{\partial x}}{N} = 1
\]
και επομένως η $p(x)=e^{\int dx} = e^x$ είναι ένας ολοκληρωτικός παράγοντας. Αυτό σημαίνει ότι η εξίσωση
\[
e^x(\frac{y^2}{2}+2ye^x)dx+e^x(y+e^x)dy=0
\]
ή
\[
(\frac{y^2e^x}{2}+2ye^{2x})dx+(ye^x+e^{2x})dy=0
\]
είναι μια διαφορική εξίσωση αμέσως ολοκληρώσιμη. Για την εξίσωση αυτή υπάρχει μια συνάρτηση $F(x,y)$ έτσι ώστε αυτή να γράφεται $dF=0$. Τότε
\[
\frac{\partial F}{\partial x} = \frac{y^2e^x}{2}+2ye^{2x} \quad \text{και} \quad \frac{\partial F}{\partial y} = ye^x+e^{2x},
\]
οπότε
\[
F(x,y) = \int (\frac{y^2e^x}{2}+2ye^{2x})dx+h(y) = \frac{y^2}{2}e^x+ye^{2x}+h(y)
\]
για κάποια συνάρτηση h(y). Επομένως
\[
\frac{\partial F}{\partial y} = ye^x+e^{2x}+h'(y) = ye^x+e^{2x},
\]
δηλαδή $h'(y)=0$. Εκλέγουμε $h(y)=0$ και έχουμε
\[
F(x,y) = \frac{y^2}{2}e^x+ye^{2x}.
\]
οι λύσεις της διαφορικής μας εξίσωσης δίνονται λοιπόν απ' τον τύπο
\[
\frac{y^2}{2}e^x+ye^{2x}=c
\]
ή
\[
y(x) = -e^x \pm \sqrt{e^{2x}+2ce^{-x}},
\]
όπου c είναι αυθαίρετη σταθερά.

\begin{Paradeigma}{Να βρεθεί ένας ολοκληρωτικός παράγοντας της διαφορικής εξίσωσης}
$(2xy+y^3)dx+(3x^2+xy^2)dy=0$.
\end{Paradeigma}
\lysh Η διαφορική αυτή εξίσωση είναι της μορφής $\rho(x,y)=x^m y^n$ (όπου $m,n \in \mathbb{N}$ και ακέραιοι).
Η συνάρτηση $\rho(x,y)=x^m y^n$ θα είναι ένας ολοκληρωτικός παράγοντας της διαφορικής μας εξίσωσης αν και μόνο αν
\[
\frac{\partial}{\partial y}[x^m y^n(2xy+y^3)] = \frac{\partial}{\partial x}[x^m y^n(3x^2+xy^2)]
\]
ή
\[
2(n+1)x^{m+1}y^n+(n+3)x^m y^{n+2} = 3(m+2)x^{m+1}y^n+(m+1)x^m y^{n+2},
\]
δηλαδή αν και μόνο αν $2(n+1)=3(m+2)$ και $n+3=m+1$. Οι εξισώσεις αυτές αληθεύουν για $m=-8$ και $n=-10$. Ένας λοιπόν ολοκληρωτικός παράγοντας είναι
\[
\rho(x,y)=x^{-8}y^{-10}.
\]

\subsection{Ασκήσεις}
\begin{enumerate}
\item Να επιλυθούν οι διαφορικές εξισώσεις:
\begin{enumerate}[label=(\roman*)]
\item $(x+y\cos x)dx+\sin x dy=0$.
\item $(3y^2+y\sin 2xy)dx+(6xy+\sin 2xy+3y^2)dy=0$.
\item $(ye^x+2x\cos y)dx+(e^x-x^2\sin y)dy=0$.
\item $(\frac{y}{x}+6x)dx+(\log x-2)dy=0$.
\item $(\cos 2y-3x^2y^2)dx+(\cos y^2-2x\sin 2y-2x^3y)dy=0$.
\end{enumerate}
\item Να επιλυθούν τα προβλήματα αρχικών τιμών:
\begin{enumerate}[label=(\roman*)]
\item $[x(x^2+2y)^{-1/2}+\log(1+y)]dx+[(x^2+2y)^{-1/2}+x(1+y)^{-1}]dy=0, \quad y(-2)=0$.
\item $(ye^{2x}y_1+2x)dx+(xe^{2x}y_1^2-2y)dy=0, \quad y(0)=2.$
\item $(1+y^2+xy^2)dx+(x^2y+y+2xy)dy=0, \quad y(1)=1.$
\end{enumerate}
\item Να επιλυθούν οι διαφορικές εξισώσεις:
\begin{enumerate}[label=(\roman*)]
\item $xy\,dx+(x^2+y^2)dy=0.$
\item $(4x^3y^3+1)dx+(3x^4y^2-xy^{-1})dy=0.$
\item $(2xy^4e^y+2xy^3)dx+(x^2y^4e^y-x^2y^2-3x)dy=0.$
\end{enumerate}
\item Να επιλυθεί το πρόβλημα αρχικών τιμών
\[
[y^{-3}\cos(x-y)+y]dx+[4x-y^{-3}\cos(x-y)]dy=0, \quad y(1)=1.
\]
\item Ν' αποδειχθεί ότι $\rho(x,y)=1/[2xy(1-xy)]$ είναι ένας ολοκληρωτικός παράγοντας της διαφορικής εξίσωσης
\[
(xy^2+y)dx+x^2y\,dy=0.
\]
\item Να βρεθεί ένας ολοκληρωτικός παράγοντας $\rho$ της διαφορικής εξίσωσης
\[
(4x^2-4y^2+2x-2y)dx+(3x^{-3}y+x^{-1})dy=0
\]
της μορφής $\rho(x,y)=x^m y^n$ ($m,n$ ακέραιοι).
\item Να επιλυθεί η διαφορική εξίσωση
\[
(2x^3+2y^2+x)dx+(x^2y+y)dy=0,
\]
αφού πρώτα διαπιστωθεί ότι $\rho(x,y)=1/(x^2+y^2)$ είναι ένας ολοκληρωτικός παράγοντας αυτής.
\item Να βρεθούν οι συναρτήσεις f ώστε η εξίσωση
\[
y^2\sin x\,dx+yf(x)dy=0
\]
να είναι μια διαφορική εξίσωση αμέσως ολοκληρώσιμη. Να επιλυθεί η εξίσωση γι' αυτές τις $f$.
\item Η διαφορική εξίσωση
\[
(x^2+y)dx+f(x)dy=0
\]
έχει τον ολοκληρωτικό παράγοντα $\rho(x)=x$. Να βρεθούν όλες οι πιθανές συναρτήσεις $f$.
\end{enumerate}
\section{Διαφορικές εξισώσεις δεύτερης τάξης αναγόμενες σε εξισώσεις πρώτης τάξης}

Θα εξετάσουμε εδώ τρεις κατηγορίες διαφορικών εξισώσεων δεύτερης τάξης που ανάγονται με κατάλληλο μετασχηματισμό σε εξισώσεις πρώτης τάξης. Αυτές οι κατηγορίες είναι οι διαφορικές εξισώσεις δεύτερης τάξης που δεν περιέχουν την άγνωστη συνάρτηση ή την ανεξάρτητη μεταβλητή καθώς και οι γραμμικές διαφορικές εξισώσεις δεύτερης τάξης για τις οποίες είναι γνωστή μια μη μηδενική λύση των αντίστοιχων ομογενών εξισώσεων. Θα δοθούν παραδείγματα και ασκήσεις για λύση.

\subsection{ΔΙΑΦΟΡΙΚΕΣ ΕΞΙΣΩΣΕΙΣ ΔΕΥΤΕΡΗΣ ΤΑΞΗΣ ΜΗ ΠΕΡΙΕΧΟΥΣΕΣ ΤΗΝ ΑΓΝΩΣΤΗ ΣΥΝΑΡΤΗΣΗ}

Μια διαφορική εξίσωση της μορφής
\begin{equation}
y''=F(x,y'),
\end{equation}
όπου $F$ είναι μια γνωστή συνάρτηση, λέμε ότι είναι μια διαφορική εξίσωση δεύτερης τάξης μη περιέχουσα την άγνωστη συνάρτηση.
Ο μετασχηματισμός $y'=z$ μετασχηματίζει τη διαφορική εξίσωση (Ε) στη διαφορική εξίσωση πρώτης τάξης
\[
z'=F(x,z).
\]

\begin{Paradeigma}{Να επιλυθεί το πρόβλημα αρχικών τιμών}
$y''(y')^2+y'=0; \quad y(0)=0, y'(0)=1$.
\end{Paradeigma}
\lysh Η διαφορική εξίσωση είναι μια εξίσωση μη περιέχουσα την άγνωστη συνάρτηση. Έτσι, θέτοντας $y'=z$, αναγόμαστε στη διαφορική εξίσωση πρώτης τάξης
\[
z'z^2+z=0,
\]
η οποία είναι μια εξίσωση \eng{Bernoulli}. Η αντικατάσταση $u=1/z$ μετασχηματίζει αυτή στη γραμμική διαφορική εξίσωση
\[
u'-u=1,
\]
της οποίας οι λύσεις είναι
\[
u=c_1e^x-1,
\]
όπου $c_1$ είναι μια αυθαίρετη σταθερά. Τότε $y'=1/u$ και άρα έχουμε
\[
y'=\frac{1}{c_1e^x-1},
\]
όπου $c_2$ είναι αυθαίρετη σταθερά. Έτσι, βρίσκουμε ότι οι λύσεις της διαφορικής μας εξίσωσης δίνονται απ' τον τύπο
\[
y = \log|c_1e^x-1|+c_2 \quad (c_1, c_2 \text{ αυθαίρετες σταθερές}).
\]
Για τη λύση του προβλήματος αρχικών τιμών θα είναι $0=\log|c_1-1|+c_2$ και $1=1/(c_1-1)$, δηλαδή $c_1=2$ και $c_2=0$, και άρα αυτή είναι
\[
y=\log|2-e^{-x}|.
\]

\subsection{ΔΙΑΦΟΡΙΚΕΣ ΕΞΙΣΩΣΕΙΣ ΔΕΥΤΕΡΗΣ ΤΑΞΗΣ ΜΗ ΠΕΡΙΕΧΟΥΣΕΣ ΤΗΝ ΑΝΕΞΑΡΤΗΤΗ ΜΕΤΑΒΛΗΤΗ}

Μια διαφορική εξίσωση της μορφής
\begin{equation}
y''=F(y,y'),
\end{equation}
όπου $F$ είναι μια γνωστή συνάρτηση, είναι μια διαφορική εξίσωση δεύτερης τάξης μη περιέχουσα την ανεξάρτητη μεταβλητή.
Οι αντικαταστάσεις $y'=z, y''=z\frac{dz}{dy}$ μετασχηματίζουν την (Ε) στη διαφορική εξίσωση πρώτης τάξης
\[
z\frac{dz}{dy} = F(y,z).
\]

\begin{Paradeigma}{Να επιλυθεί η διαφορική εξίσωση}
$2yy'=1+(y')^2$.
\end{Paradeigma}
\lysh Η εξίσωση αυτή είναι μια διαφορική εξίσωση δεύτερης τάξης μη περιέχουσα την ανεξάρτητη μεταβλητή. Έτσι, θέτουμε $y'=z$ και $y''=z\frac{dz}{dy}$, οπότε παίρνουμε την εξίσωση πρώτης τάξης
\[
\frac{2z}{1+z^2}dz = \frac{1}{y}dy
\]
η οποία γράφεται
\[
\frac{2z}{1+z^2}dz=\frac{1}{y}dy
\]
και άρα είναι χωριζομένων μεταβλητών. Οι λύσεις της δίνονται απ' τους τύπους
\[
z=\pm\sqrt{c_1y-1},
\]
όπου $c_1\neq 0$ είναι μια αυθαίρετη σταθερά. Για την αρχική εξίσωση έχουμε
\[
y'=\pm\sqrt{c_1y-1}
\]
και τελικά βρίσκουμε ότι οι λύσεις της διαφορικής μας εξίσωσης δίνονται απ' τον τύπο
\[
y=\frac{1}{c_1}+\frac{c_1}{4}(x+c_2)^2 \quad (c_1\neq 0, c_2 \text{ αυθαίρετες σταθερές}).
\]

\subsection{Υποβιβασμός της τάξης των γραμμικών διαφορικών εξισώσεων δεύτερης τάξης}

Μια διαφορική εξίσωση της μορφής
\begin{equation}
y''+p_1(x)y'+p_2(x)y=q(x), \tag{E}
\end{equation}
όπου $p_1,p_2$ και q είναι συνεχείς συναρτήσεις του x (σ' ένα διάστημα της πραγματικής ευθείας), λέμε ότι είναι μια γραμμική διαφορική εξίσωση δεύτερης τάξης. Αν $q\neq 0$, λέμε ότι η (Ε) είναι μη ομογενής. Για $q=0$ η (Ε) γίνεται
\begin{equation}
y''+p_1(x)y'+p_2(x)y=0 \tag{E$_0$}
\end{equation}
και τότε λέγεται ομογενής. Ακόμα, λέμε ότι η (E$_0$) είναι η αντίστοιχη ομογενής εξίσωση της γραμμικής διαφορικής εξίσωσης (Ε).

Αν $y_1\neq 0$ είναι μια (μερική) λύση της (E$_0$), τότε οι αντικαταστάσεις $z=y/y_1$ και $u=z'$ μετασχηματίζουν τη γραμμική διαφορική εξίσωση δεύτερης τάξης (Ε) σε μια γραμμική εξίσωση πρώτης τάξης.

Πραγματικά: θέτουμε $y=y_1z$, οπότε $y'=y_1'z+y_1z'$ και $y''=y_1''z+2y_1'z'+y_1z''$ και η (Ε) γίνεται
\[
y_1z''+2y_1'z'+p_1(x)(y_1'z+y_1z')+p_2(x)y_1z=q(x)
\]
ή
\[
y_1z''+[2y_1'+p_1(x)y_1]z'+[y_1''+p_1(x)y_1'+p_2(x)y_1]z=q(x).
\]
Επειδή όμως $y_1$ είναι μια μη μηδενική λύση της (E$_0$), η τελευταία εξίσωση γράφεται
\[
z''+\left[\frac{2y_1'}{y_1}+p_1(x)\right]z' = q(x)/y_1
\]
και για $u=z'$ γίνεται
\[
u'+\left[\frac{2y_1'}{y_1}+p_1(x)\right]u = q(x)/y_1.
\]
Αυτή η εξίσωση είναι μια διαφορική εξίσωση πρώτης τάξης.
\begin{Paradeigma}{Να επιλυθεί το πρόβλημα αρχικών τιμών}
$y''-3y'+2y = e^{2x}; \quad y(0)=1, y'(0)=0$,
\end{Paradeigma}
αφού πρώτα διαπιστωθεί ότι $y_1=e^x$ είναι μια μερική λύση της αντίστοιχης ομογενούς γραμμικής διαφορικής εξίσωσης.
\lysh Η εξίσωση αυτή είναι μια (μη ομογενής) γραμμική διαφορική εξίσωση δεύτερης τάξης. Εύκολα αποδεικνύεται ότι $y_1=e^x$ είναι μια λύση της αντίστοιχης ομογενούς εξίσωσης. Έτσι, θέτουμε $y=y_1z=e^xz$ και παίρνουμε $y'=e^x(z'+z)$ και $y''=e^x(z''+2z'+z)$, οπότε η εξίσωση γίνεται
\[
z''-z'=e^x.
\]
Αν λοιπόν θέσουμε $z'=u$, καταλήγουμε στη γραμμική διαφορική εξίσωση πρώτης τάξης
\[
u'-u=e^x,
\]
της οποίας οι λύσεις είναι
\[
u=e^x(c_1+x),
\]
όπου $c_1$ είναι μια αυθαίρετη σταθερά. Τότε έχουμε
\[
z=\int e^x(c_1+x)dx = (c_1+x)e^x+c_2,
\]
όπου $c_1=c_1+1$ και $c_2$ είναι μια αυθαίρετη σταθερά. Έτσι, οι λύσεις της διαφορικής μας εξίσωσης δίνονται απ' τον τύπο
\[
y=e^x(c_1x+x)e^x+c_2e^x \quad (C_1, C_2 \text{ αυθαίρετες σταθερές}).
\]
Για τη λύση του προβλήματος αρχικών τιμών βρίσκουμε $C_1=-2$ και $C_2=3$ και επομένως αυτή είναι
\[
y=e^x(-2+x)+3e^x.
\]

\section{Ασκήσεις}
\begin{enumerate}
\item Να επιλυθούν οι διαφορικές εξισώσεις:
\begin{rlist}
\item $xy''=y'+x^2$.
\item $(1+x^2)y''+2xy'+1=0$.
\item $xy''-\frac{1}{4}y'+x(y')^2=0$.
\item $xy''=y'\log\frac{y'}{x}$.
\end{rlist}
\item Να επιλυθούν οι διαφορικές εξισώσεις:
\begin{rlist}
\item $yy''-(y')^2=\frac{2}{y}y'$.
\item $yy''-yy'\log y = (y')^2$.
\item $y(1-\log y)y''+(1+\log y)(y')^2=0$.
\item $y''=2yy'$.
\end{rlist}
\item Να επιλυθούν τα προβλήματα αρχικών τιμών:
\begin{rlist}
\item $xy''+x(y')^2-y'=0; \quad y(2)=2, y'(2)=1$.
\item $2y''=3y^2; \quad y(-2)=1, y'(-2)=-1$.
\item $2(y')^2=y''(y+1); \quad y(1)=2, y'(1)=-1$.
\item $xy''y'=(y')^2+x^3; \quad y(2)=0, y'(2)=4$.
\end{rlist}
\item Να επιλυθεί καθεμιά απ' τις παρακάτω γραμμικές διαφορικές εξισώσεις, αφού πρώτα αποδειχθεί ότι η σημειούμενη συνάρτηση $y_1$ είναι μια λύση της αντίστοιχης ομογενούς εξίσωσης:
\begin{rlist}
\item $(1-x^2)y''-2xy'+2y=0, \quad x\in(-1,1); \quad y_1=x$.
\item $y''+\frac{2}{x}y'+\frac{9}{x^2}y=x, \quad x>0; \quad y_1=\cos\frac{3}{x}$.
\item $x^3y''+2x^2y'-y=0, \quad x>0; \quad y_1=e^{1/x}$.
\item $x^2y''+xy'+(x^2-\frac{1}{4})y=0, \quad x>0; \quad y_1=x^{-1/2}\sin x$.
\item $y''+3y'+2y=xe^x; \quad y_1=e^{-x}$.
\end{rlist}
\end{enumerate}

\full{\Alyta}
\fulltwoc{
\Askhsh Δίνεται η διαφορική εξίσωση
\[ y'+by=\sin(ax), \]
όπου a και b είναι μη μηδενικές σταθερές. Να επιλυθεί η εξίσωση αυτή. Αν $b\neq 0$ και y είναι μια λύση της υπάρχει το $\lim_{x\to\infty} y(x)$;

\Askhsh Μπορεί να βρεθούν συναρτήσεις $f(x)$ και $g(y)$ έτσι ώστε η εξίσωση
\[ g(y)\sin x\,dx+y^2 f(x)\,dy=0 \]
να είναι μια διαφορική εξίσωση αμέσως ολοκληρώσιμη;

\Askhsh Να προσδιορισθούν οι συναρτήσεις $g(y)$ για να είναι η εξίσωση
\[ g(y)e^{2x}dx+xy\,dy=0 \]
μια διαφορική εξίσωση αμέσως ολοκληρώσιμη.

\Askhsh Για ποιες τιμές των a,b,c και d η διαφορική εξίσωση
\[
(3x^ay^{-b}x^c-x^{-y})dx+(2x^2y^{-1}-x^{-3}y^d)dy=0
\]
έχει ένα ολοκληρωτικό παράγοντα της μορφής $\rho(x,y)=x^my^n$ (m,n ακέραιοι).

\Askhsh Με την αντικατάσταση $z=\log y$, να επιλυθεί η διαφορική εξίσωση
\[
xy'-\log y = x^2y.
\]

\Askhsh Να διαπιστωθεί ότι $y_1=x$ είναι μια λύση της διαφορικής εξίσωσης
\[
-2y_1'+\left(\frac{y}{x}\right)^2+y_1'^2=0
\]
και, στη συνέχεια, να βρεθεί η λύση y με $y(1)=2$.

\Askhsh Για καθεμιά απ' τις παρακάτω εξισώσεις να βρεθεί η σταθερά a ώστε να είναι μια διαφορική εξίσωση αμέσως ολοκληρώσιμη και να επιλυθεί η εξίσωση που προκύπτει:
\begin{rlist}
\item $(Ax^2+y^2)dx+(x^2+4xy)dy=0$.
\item $(x^2+3xy)dx+(Ax^2+4y)dy=0$.
\item $(Ayx^{-3}+yx^{-2})dx+(x^{-2}-x^{-1})dy=0$.
\end{rlist}

\Askhsh Για τη διαφορική εξίσωση
\[
[Y+x(x^2+y^2)^2]dx+[y(x^2+y^2)^2-x]dy=0
\]
να βρεθεί ένας ολοκληρωτικός παράγοντας της μορφής $\rho(x,y)=(x^2+y^2)^m$ όπου m είναι ακέραιος.

\Askhsh Ας θεωρήσουμε τη διαφορική εξίσωση
\[
ay'+by=ke^{-\lambda x},
\]
όπου a,b και k είναι θετικές σταθερές και $\lambda$ είναι μια μη αρνητική σταθερά. Ας είναι y μια λύση αυτής. Ν' αποδειχθεί ότι: (i) Αν $\lambda=0$, τότε $\lim_{x\to\infty} y(x) = k/b$. (ii) Αν $\lambda>0$, τότε $\lim_{x\to\infty} y(x)=0$.

\Askhsh Να επιλυθούν τα παρακάτω προβλήματα αρχικών τιμών με τις σημειούμενες αντικαταστάσεις:
\begin{rlist}
\item $\cos y \frac{dy}{dx}+\frac{1}{x}\sin y=1, \quad y(1)=0; \quad \sin y=z$.
\item $(y+1)\frac{dy}{dx}+x(y^2+2y)=x, \quad y(0)=1; \quad y^2+2y=z$.
\end{rlist}

\Askhsh Να βρεθεί η λύση y του προβλήματος αρχικών τιμών
\[
y' = ay-by^2, \quad y(0)=c,
\]
όπου a,b και c είναι σταθερές με $a>0, b>0$ και $c\ge 0$. Στη συνέχεια, ν' αποδειχθεί ότι $\lim_{x\to\infty} y(x) = a/b$ για $c>0$, ενώ $\lim_{x\to\infty} y(x)=0$ για $c=0$.

\Askhsh Με τον μετασχηματισμό $x^2+y^2=z$, να επιλυθεί η διαφορική εξίσωση
\[
2yy' = e^{(x^2+y^2)/x}+(x^2+y^2)/x-2x.
\]

\Askhsh Να επιλυθεί η διαφορική εξίσωση
\[
1+x^2y^2+xy'=0,
\]
αφού βρεθεί ένας ολοκληρωτικός παράγοντας αυτής της μορφής $\rho(x,y)=1/(1+(xy)^n)$, όπου n ακέραιος.

\Askhsh Να επιλυθεί το πρόβλημα αρχικών τιμών
\[
2(y')^2=(y-1)y''; \quad y(1)=2, y'(1)=-1.
\]

\Askhsh Να επιλυθεί το πρόβλημα αρχικών τιμών
\[
yy'+x=\frac{1}{2}(x^2+y^2)^2x^{-2}(x+y')x^{-1}, \quad y(1)=1.
\]

\Askhsh Να επιλυθεί το πρόβλημα αρχικών τιμών
\[
q(x)y' = yq'(x)-y^2; \quad y(0)=1,
\]
όπου q είναι μια θετική συνάρτηση με συνεχή παράγωγο στο $\mathbb{R}$ και $q(0)=1$.

\Askhsh Να επιλυθούν τα προβλήματα αρχικών τιμών:
\begin{rlist}
\item $3\frac{dy}{dx}+x^3\frac{2x}{x^2}dy=0; \quad y(-1)=2$.
\item $[x(1+y)-x^2]y'=(1+y)^2; \quad y(1)=1$.
\end{rlist}

\Askhsh Να επιλυθεί η διαφορική εξίσωση
\[
(x^2+xy^2)y'-3xy+2y^3=0,
\]
αφού βρεθεί ένας ολοκληρωτικός παράγοντας της μορφής $\rho(x,y)=x\rho(y)$.

\Askhsh Να επιλυθεί η διαφορική εξίσωση
\[
2xyy'+(1+x)y^2 = e^x.
\]

\Askhsh Να βρεθεί μια συνεχής συνάρτηση $f(x), x\in\mathbb{R}$ τέτοια ώστε
\[
f(x)+1=\int_0^x f(t)[t\ell(t)-1]dt.
\]

\Askhsh Με την αλλαγή $y=ue^{mx}$ (όπου m κατάλληλη σταθερά), να επιλυθεί η διαφορική εξίσωση
\[
(1+y^2e^{-2x})y'+y=0.
\]

\Askhsh Να επιλυθούν οι διαφορικές εξισώσεις:
\begin{rlist}
\item $xe^xy'-e^y=3x^2$.
\item $\frac{1}{y^2+1}y'+\frac{1}{2}\operatorname{Arctg} y = \frac{x}{2}$.
\item $y'-\frac{1}{x+1}y-\log y = (x+1)y$.
\item $xy'+y+x^2y^2e^x=0$.
\end{rlist}

\Askhsh Να επιλυθούν οι διαφορικές εξισώσεις:
\begin{rlist}
\item $(y'-x)y''=y'$.
\item $y^2y''=2(y')^2$.
\end{rlist}

\Askhsh Να επιλυθούν οι διαφορικές εξισώσεις
$(3y+4xy^2)dx+(4x+5x^2y)dy=0$,
$(6y+x^2y^2)dx+(8x+x^3y)dy=0$
με το δεδομένο ότι έχουν ένα κοινό ολοκληρωτικό παράγοντα.

\Askhsh Να επιλυθεί η διαφορική εξίσωση
$3(x^2+xy^2)y'+5x^2+2xy+3y^3=0$,
αφού βρεθεί ένας ολοκληρωτικός παράγοντας αυτής της μορφής $\rho(x,y)=(x+y)^m$ (m ακέραιος).

\Askhsh Ας θεωρήσουμε τη διαφορική εξίσωση
$x^2y'+2xy=1, x>0$.
Ν' αποδειχθεί ότι όλες οι λύσεις τείνουν στο μηδέν για $x\to\infty$. Στη συνέχεια, να βρεθεί η λύση y με $y(2)=2y(1)$.

\Askhsh Να επιλυθεί η διαφορική εξίσωση
\[
y'+2y=q(x),
\]
όπου
\[
q(x) = 
\begin{cdcases}
1-|x|, & \text{αν } |x|\le 1 \\
0, & \text{αν } |x|>1.
\end{cdcases}
\]

\Askhsh Να επιλυθούν οι διαφορικές εξισώσεις:
\begin{rlist}
\item $y'+3y=e^{ix}$.
\item $y'+iy=x$.
\end{rlist}

\Askhsh Ν' αποδειχθεί ότι για κάθε λύση y της διαφορικής εξίσωσης
\[
y'+y\cos x = e^{-\sin x}
\]
είναι
\[
y(k\pi)-y(0)=\pi k \quad \text{(k ακέραιος)}.
\]

\Askhsh Ν' αποδειχθεί ότι, για τυχούσες σταθερές $c_1, c_2$ και $c_3$, η συνάρτηση
\[
y(x) = 
\begin{cdcases}
c_1(x^2-1)^2, & \text{για } x<-1 \\
c_2(x^2-1)^2, & \text{για } -1\le x \le 1 \\
c_3(x^2-1)^2, & \text{για } x>1
\end{cdcases}
\]
είναι μια λύση της διαφορικής εξίσωσης
\[
(x^2-1)y'-4xy=0.
\]

\Askhsh Ν' αποδειχθεί ότι, αν $\rho(x,y)$ είναι ένας ολοκληρωτικός παράγοντας της διαφορικής εξίσωσης
\[
M(x,y)dx+N(x,y)dy=0,
\]
τότε ισχύει
\[
M\frac{\partial\rho}{\partial y} - N\frac{\partial\rho}{\partial x} + \rho\left(\frac{\partial M}{\partial y}-\frac{\partial N}{\partial x}\right) = 0.
\]

\Askhsh Ν' αποδειχθεί ότι υπάρχουν δύο τιμές της σταθεράς c για τις οποίες η συνάρτηση $y=c-x^2$ είναι μια λύση της διαφορικής εξίσωσης
\[
y' = (x+y+1)(x^2-y-\frac{3}{2})+1-2x.
\]
Να βρεθούν οι τιμές αυτές και να επιλυθεί η εξίσωση με χρησιμοποίηση καθεμιάς των μερικών λύσεων που ορίζονται γι' αυτές τις τιμές. Είναι οι δύο γενικές λύσεις ισοδύναμες;

\Askhsh Να βρεθούν οι τιμές του n ώστε η
\[
(x^2+y^2)^n(xy^2dx - x^2ydy) = 0
\]
να είναι μια διαφορική εξίσωση αμέσως ολοκληρώσιμη.

\Askhsh Να επιλυθεί η διαφορική εξίσωση
\[
(y+a)\frac{dx}{dy} - 4x = (y+a)^6x^3 \quad \text{(a σταθερά)}.
\]
}
\setchapterimage{./images/4.png}
\chapter{Γραμμικές Διαφορικές Εξισώσεις}
\chaptertoc
\section*{Προκαταρκτικά}
Το Κεφάλαιο αυτό αναφέρεται στη μελέτη των γραμμικών διαφορικών εξισώσεων (αυθαίρετης τάξης). Στο Εδάφιο 0 δίνεται η έννοια της γραμμικής διαφορικής εξίσωσης, διατυπώνεται το θεώρημα ύπαρξης και μονοσήμαντου των λύσεων και δίνονται μερικά συμπεράσματα απ' τη Γραμμική Άλγεβρα που χρειάζονται στα επόμενα Εδάφια. Οι ομογενείς γραμμικές διαφορικές εξισώσεις μελετώνται στο Εδάφιο 1, ενώ το Εδάφιο 2 αφορά τις μη ομογενείς γραμμικές διαφορικές εξισώσεις. Η ειδική περίπτωση των γραμμικών διαφορικών εξισώσεων με σταθερούς συντελεστές εξετάζεται στο Εδάφιο 3. Στο Εδάφιο 4 μελετώνται η συνήθης διαφορική εξίσωση μιας ομογενούς γραμμικής διαφορικής εξίσωσης και οι αυτοσυζυγείς ομογενείς γραμμικές διαφορικές εξισώσεις δεύτερης τάξης. Το Εδάφιο 5 αποτελεί μια εισαγωγή στη θεωρία του Sturm και στα προβλήματα ιδιοτιμών για ομογενείς γραμμικές διαφορικές εξισώσεις δεύτερης τάξης. Σε καθένα απ' τα Εδάφια 1-5 δίνονται παραδείγματα και προτείνονται ασκήσεις για λύση. Τέλος, το Εδάφιο 6 είναι μια συλλογή γενικών ασκήσεων.

\subsection*{Προκαταρκτικά}
\addcontentsline{toc}{section}{Προκαταρκτικά}

Στο προκαταρκτικό αυτό Εδάφιο θα δώσουμε την έννοια της γραμμικής διαφορικής εξίσωσης, θα διατυπώσουμε το θεώρημα ύπαρξης και μονοσήμαντου των λύσεων των προβλημάτων αρχικών τιμών για γραμμικές διαφορικές εξισώσεις και θα παραθέσουμε μερικά συμπεράσματα απ' τη Γραμμική Άλγεβρα (σχετικά με τις ορίζουσες, τα γραμμικά συστήματα, τους πίνακες, τις ιδιοτιμές και τα ιδιοδιανύσματα) που θα μας χρειαστούν στα επόμενα.
\subsubsection{Η έννοια της γραμμικής διαφορικής εξίσωσης και μονοσήμαντο των λύσεων}

Μια γραμμική διαφορική εξίσωση n-τάξης είναι μια διαφορική εξίσωση της μορφής
\begin{equation}\label{eq:E}
a_n y^{(n)} + a_{n-1} y^{(n-1)} + \dots + a_1 y' + a_0 y = b, \tag{E}
\end{equation}
όπου $a_i$ (i=0,1,\dots,n-1,n) και b είναι συνεχείς συναρτήσεις ορισμένες σ' ένα \underline{διάστημα Ι} της πραγματικής ευθείας και $a_n(x)\ne 0$ για όλα τα $x\in I$. Οι συναρτήσεις $a_i$ (i=0,1,\dots,n-1,n) καλούνται \underline{συντελεστές} της διαφορικής εξίσωσης (Ε) και το Ι καλείται \underline{διάστημα ορισμού} της (Ε). Αν b=0 (για μια συνάρτηση f ορισμένη στο Ι γράφουμε f=0 αν και μόνο αν $f(x)=0$ για όλα τα $x\in I$, και $f\ne 0$ διαφορετικά, δηλαδή όταν $f(x)\ne 0$ για κάποιο $x\in I$), τότε η (Ε) γίνεται
\begin{equation}\label{eq:E0}
a_n y^{(n)} + a_{n-1} y^{(n-1)} + \dots + a_1 y' + a_0 y = 0. \tag{$E_0$}
\end{equation}
Η διαφορική εξίσωση $(E_0)$ λέμε ότι είναι μια \underline{ομογενής γραμμική διαφορική εξίσωση}. Για $b\ne 0$ λέμε ότι η γραμμική διαφορική εξίσωση (Ε) είναι \underline{μη ομογενής}· στην περίπτωση αυτή λέμε ακόμα ότι η $(E_0)$ είναι η \underline{αντίστοιχη ομογενής} της (Ε). 'Οταν οι συντελεστές $a_i$ (i=0,1,\dots,n-1,n) είναι σταθερές (συναρτήσεις), η (Ε) καλείται γραμμική διαφορική εξίσωση με \underline{σταθερούς συντελεστές}. θέτοντας $A_i=a_i/a_n$ (i=0,1,\dots,n-1) και $B=b/a_n$ παίρνουμε την παρακάτω μορφή για τη διαφορική εξίσωση (Ε)
\[
y^{(n)} + A_{n-1} y^{(n-1)} + \dots + A_1 y' + A_0 y = B.
\]
Ο τύπος
\[
L(\phi) = a_n \phi^{(n)} + a_{n-1} \phi^{(n-1)} + \dots + a_1 \phi' + a_0 \phi
\]
ορίζει ένα τελεστή L, ο οποίος απεικονίζει κάθε συνάρτηση $\phi$ που είναι n-φορές παραγωγίσιμη στο Ι σε μια συνάρτηση $L(\phi)$ ορισμένη στο Ι. Ο τελεστής L είναι γραμμικός, γιατί για τυχούσες συναρτήσεις $\phi_1, \phi_2$ στο πεδίο ορισμού του L και για οποιουσδήποτε αριθμούς $c_1, c_2$ είναι
\[
L(c_1\phi_1+c_2\phi_2) = \sum_{i=0}^{n} a_i (c_1\phi_1+c_2\phi_2)^{(i)} = \sum_{i=0}^{n} a_i (c_1\phi_1^{(i)}+c_2\phi_2^{(i)}).
\]
\[
= c_1 \sum_{i=0}^{n} a_i \phi_1^{(i)} + c_2 \sum_{i=0}^{n} a_i \phi_2^{(i)} = c_1 L(\phi_1) + c_2 L(\phi_2).
\]
Με τη χρησιμοποίηση του τελεστή $L$, η διαφορική εξίσωση (Ε) γράφεται $L(y)=b$. Ο χαρακτηρισμός "γραμμική" για την (Ε) προέρχεται απ' το γεγονός ότι ο ($n$-τάξης διαφορικός) τελεστής $L$ είναι γραμμικός.

Ας υποθέσουμε ότι, για κάθε i$\in\{1,\dots,n\}$, η συνάρτηση $a_i$ έχει συνεχή παράγωγο k-τάξης στο διάστημα $Ι$. Τότε ορίζεται ένας τελεστής $L^*$ με τον τύπο
\[
L^*(\phi) = (-1)^n(a_n\phi)^{(n)} + (-1)^{n-1}(a_{n-1}\phi)^{(n-1)} + \dots - (a_1\phi)' + a_0\phi,
\]
ο οποίος απεικονίζει κάθε συνάρτηση $\phi$ που έχει n-τάξης παράγωγο στο Ι στη συνάρτηση $L^*(\phi)$ ορισμένη στο Ι. είναι εύκολο να διαπιστωθεί ότι ο $L^*$ είναι επίσης γραμμικός. Ο τελεστής $L^*$ λέμε ότι είναι ο \underline{συζυγής τελεστής} του $L$. Λέμε ακόμα ότι η ομογενής γραμμική διαφορική εξίσωση $n$-τάξης
\begin{equation*}
(-1)^n(\bar{a}_n y)^{(n)} + (-1)^{n-1}(\bar{a}_{n-1} y)^{(n-1)} + \dots - (\bar{a}_1 y)' + \bar{a}_0 y = 0
\end{equation*}
είναι η \underline{συζυγής διαφορική εξίσωση} της ομογενούς γραμμικής διαφορικής εξίσωσης $(E_0)$. Τέλος, λέμε ότι η $(E_0)$ είναι μια \underline{αυτοσυζυγής} ομογενής γραμμική διαφορική εξίσωση αν και μόνο αν συμπίπτει με τη συζυγή της διαφορική εξίσωση.

Το παρακάτω θεώρημα αναφέρεται στην ύπαρξη και στο μονοσήμαντο των λύσεων των προβλημάτων αρχικών τιμών για γραμμικές διαφορικές εξισώσεις. Η απόδειξη αυτού δίνεται στο Κεφάλαιο Ι.

\begin{Thewrhma}{}
Αν $x_0$ είναι ένα σημείο του Ι και $c_0, c_1, \dots, c_{n-1}$ είναι σταθερές, τότε υπάρχει ακριβώς μια λύση y της γραμμικής διαφορικής εξίσωσης (Ε), η οποία είναι ορισμένη σ' ολόκληρο το διάστημα Ι και πληρεί τις αρχικές συνθήκες
\[
y(x_0)=c_0, y'(x_0)=c_1, \dots, y^{(n-1)}(x_0)=c_{n-1}.
\]
\end{Thewrhma}
Ας τονίσουμε ότι στο παραπάνω θεώρημα δεν μπαίνει καμιά υπόθεση πέρα απ' αυτή της συνέχειας των συναρτήσεων $a_i$ (i=0,1,\dots,n-1,n) και b. Ας τονίσουμε ακόμα ότι το θεώρημα 1 εξασφαλίζει ότι όλες οι λύσεις της (Ε) είναι ορισμένες σ' ολόκληρο το διάστημα $Ι$. Επίσης, για μια λύση $y$ της ομογενής εξίσωσης $(E_0)$ το θεώρημα 1 εγγυάται ότι, αν μια λύση $y$ πληροί τις αρχικές συνθήκες $y(x_0)=y'(x_0)=\dots=y^{(n-1)}(x_0)=0$ (για κάποιο $x_0\in I$), αυτή θα είναι αναγκαστικά η μηδενική λύση: $y(x)=0, x\in I$.
\subsubsection{Ορίζουσες, Γραμμικά συστήματα, Πολυώνυμα}

θεωρούμε γνωστή τη στοιχειώδη θεωρία των οριζουσών. θ' αναφέρουμε μόνο ένα συμπέρασμα που αφορά την παράγωγιση μιας ορίζουσας-συνάρτησης. Αν είναι $\phi_{kj},\ (k,j=1,\dots,n)$ συναρτήσεις ορισμένες στο διάστημα $Ι$. Αν οι συναρτήσεις αυτές είναι παραγωγίσιμες στο $Ι$, τότε και η ορίζουσα-συνάρτηση
\[
D = \begin{vmatrix}
\phi_{11} & \phi_{12} & \dots & \phi_{1n} \\
\phi_{21} & \phi_{22} & \dots & \phi_{2n} \\
\vdots & \vdots & \ddots & \vdots \\
\phi_{n1} & \phi_{n2} & \dots & \phi_{nn}
\end{vmatrix}
\]
είναι επίσης παραγωγίσιμη στο Ι και μάλιστα $D' = D'_1 + D'_2 + \dots + D'_n$, όπου, για κάθε $k\in\{1,2,\dots,n\}$, $D'_k$ είναι η ορίζουσα-συνάρτηση που προκύπτει απ' την $D$ αν αντικαταστήσουμε την k-γραμμής της με τη γραμμή $(\phi'_{k1}, \phi'_{k2}, \dots, \phi'_{kn})$.

Για τα γραμμικά συστήματα ισχύουν τα παρακάτω:
\begin{enumerate}[label=(\roman*)]
\item 'Ενα γραμμικό σύστημα n εξισώσεων με n αγνώστους έχει ακριβώς μια λύση αν και μόνο αν η ορίζουσα των συντελεστών αυτού $\Delta$ είναι διάφορη του μηδενός· σ' αυτή την περίπτωση που η ορίζουσα των συντελεστών $\Delta$ είναι διάφορη απ' το μηδέν, η λύση είναι (Κανόνας του \eng{Cramer}) $(x_1, x_2, \dots, x_n)$, όπου, για κάθε $k\in\{1,2,\dots,n\}$, $x_k = \Delta_k/\Delta$, και η ορίζουσα που προκύπτει απ' την $\Delta$ αν αντικατασταθεί η k-στήλη της με τη στήλη του δεύτερου μέλους του συστήματος.
\item 'Ενα ομογενές γραμμικό σύστημα n εξισώσεων με n αγνώστους έχει μη μηδενικές λύσεις αν και μόνο αν η ορίζουσα των συντελεστών του είναι ίση με μηδέν.
\end{enumerate}

Απ' το βασικό θεώρημα της Άλγεβρας προκύπτει ότι ένα πολυώνυμο με βαθμό $n\ge 1$ έχει ακριβώς $n$ ρίζες, όπου κάθε ρίζα απαριθμείται τόσες φορές όση είναι η πολλαπλότητά της. Αν ένα πολυώνυμο με πραγματικούς συντελεστές έχει μια ρίζα $\sigma+i\tau$ ($\sigma, \tau \in \mathbb{R}, \tau \ne 0$) με πολλαπλότητα $m$, τότε και το $\sigma-i\tau$ θα είναι ρίζα με την ίδια πολλαπλότητα. Αν το $\lambda$ είναι ρίζα ενός πολυωνύμου $P$ με πολλαπλότητα $m$, τότε η $\lambda$ θα είναι επίσης ρίζα και των παραγώγων $P', P'', \dots, P^{(m-1)}$ ενώ δεν θα είναι ρίζα του $P^{(m)}$.
\section{Ομογενείς γραμμικές διαφορικές εξισώσεις}
Στο Εδάφιο αυτό θ' αναπτύξουμε τη βασική θεωρία για τις ομογενείς γραμμικές διαφορικές εξισώσεις. θα ξεκινήσουμε απ' την πιο απλή περίπτωση, την περίπτωση των ομογενών γραμμικών διαφορικών εξισώσεων πρώτης τάξης. Γι' αυτές τις διαφορικές εξισώσεις θα βρούμε (θεώρημα 2) ένα τύπο που δίνει όλες τις λύσεις. Στη συνέχεια, θ' ασχοληθούμε με τη γενική περίπτωση μελετώντας την ομογενή γραμμική διαφορική εξίσωση $(E_0)$ για αυθαίρετη n. θα εισάγουμε την έννοια της γραμμικής ανεξαρτησίας συναρτήσεων ορισμένων στο διάστημα $Ι$ και, όταν αυτές οι συναρτήσεις έχουν (n-1)-τάξης παραγώγους στο Ι, θα ορίσουμε την έννοια της ορίζουσας \eng{Wronski} αυτών. θ' αποδείξουμε (θεώρημα 4) ότι n λύσεις της $(E_0)$ είναι γραμμικά ανεξάρτητες αν και μόνο αν η ορίζουσα Wronski αυτών δεν μηδενίζεται πουθενά στο $I$ . Για την ορίζουσα \eng{Wronski} n λύσεων της $(E_0)$ θ' αποδείξουμε και τον τύπο του Liouville (θεώρημα 5). 'Επειτα, κάθε σύνολο n γραμμικά ανεξάρτητων λύσεων της $(E_0)$ θα το ονομάσουμε βασικό σύνολο λύσεων αυτής και θα εξασφαλίσουμε (θεώρημα 6) ότι υπάρχουν βασικά σύνολα λύσεων καθώς επίσης θ' αποδείξουμε (θεωρήματα 3 και 7) ότι οι λύσεις της $(E_0)$ είναι ακριβώς οι γραμμικοί συνδυασμοί των συναρτήσεων ενός βασικού συνόλου λύσεων αυτής. Ακόμα, θα δούμε (θεώρημα 8) μια ομογενή γραμμική διαφορική εξίσωση n-τάξης που να έχει ως ένα βασικό σύνολο λύσεων το $\{y_1, \dots, y_n\}$, όταν $y_k$ (k=1,\dots,n) είναι η δεδομένες συναρτήσεις που έχουν συνεχείς παραγώγους n-τάξης στο διάστημα Ι και ορίζουσα \eng{Wronski} διάφορη του μηδενός σ' ολόκληρο το διάστημα $I$ . Στη συνέχεια, αν είναι γνωστή μια λύση της $(E_0)$ που δεν μηδενίζεται πουθενά στο $Ι$, θα δώσουμε ένα μετασχηματισμό που ανάγει αυτή σε μια ομογενή γραμμική διαφορική εξίσωση (n-1)-τάξης. Από ένα βασικό σύνολο λύσεων της εξίσωσης που προκύπτει μπορεί να κατασκευασθεί (θεώρημα 9) ένα βασικό σύνολο λύσεων για την $(E_0)$. Ιδιαίτερα, στην περίπτωση των ομογενών γραμμικών διαφορικών εξισώσεων δεύτερης τάξης προκύπτει (θεώρημα 10) ένα βασικό σύνολο λύσεων από μια λύση που δεν μηδενίζεται πουθενά στο $Ι$. Τέλος, θα δώσουμε μερικά παραδείγματα καθώς επίσης και ασκήσεις για λύση.
\subsection{ΟΜΟΓΕΝΕΙΣ ΓΡΑΜΜΙΚΕΣ ΔΙΑΦΟΡΙΚΕΣ ΕΞΙΣΩΣΕΙΣ ΠΡΩΤΗΣ ΤΑΞΗΣ}

Για n=1 η διαφορική εξίσωση $(E_0)$ γίνεται
\begin{equation}
a_1 y' + a_0 y = 0. \tag{$(E_0)_1$}
\end{equation}

Μια αξιοσημείωτη ιδιότητα που έχουν οι ομογενείς γραμμικές διαφορικές εξισώσεις πρώτης τάξης είναι ότι μια λύση ή δεν θα μηδενίζεται πουθενά στο διάστημα Ι ή θα είναι μηδέν σ' ολόκληρο το Ι. Αυτό προκύπτει αμέσως απ' το θεώρημα 1, δεδομένου ότι η $(E_0)_1$ έχει ως λύση τη μηδενική συνάρτηση στο Ι (\underline{μηδενική λύση}). Το παρακάτω θεώρημα δίνει τη λύση στο πρόβλημα της επίλυσης (της εύρεσης όλων των λύσεων) της ομογενούς γραμμικής διαφορικής εξίσωσης πρώτης τάξης $(E_0)_1$.

\begin{Thewrhma}{2}
Ας είναι $x_0$ ένα σημείο του Ι. Τότε $y$ είναι μια λύση της ομογενούς γραμμικής διαφορικής εξίσωσης πρώτης τάξης $(E_0)_1$ αν και μόνο αν
\[
y(x) = y(x_0) e^{- \int_{x_0}^{x} \frac{a_0(t)}{a_1(t)} dt}, \quad x \in I.
\]
\end{Thewrhma}

\textsc{ΑΠΟΔΕΙΞΗ.} Ας είναι $y$ μια λύση της συνάρτησης ορισμένη στο $Ι$. Αν $y(x_0)=0$ και $y$ είναι μια λύση της $(E_0)_1$, τότε $y(x)=0$ για όλα τα $x\in I$ και ο τύπος πληρούται. Επίσης, αν $y(x_0) \ne 0$ και η $y$ πληροί τον τύπο, τότε πάλι η $y$ είναι η μηδενική συνάρτηση στο $Ι$ και άρα είναι μια λύση της $(E_0)_1$. Υποθέτουμε, στη συνέχεια, ότι $y(x_0) \ne 0$. Σε καθεμιά απ' τις περιπτώσεις όπου η $y$ είναι μια λύση της $(E_0)_1$ ή η $y$ είναι μια συνάρτηση που πληροί τον παραπάνω τύπο έχουμε $y(x) \ne 0$ για όλα τα $x\in I$. 'Ετσι, η $y$ είναι μια λύση της $(E_0)_1$ αν και μόνο αν για κάθε $x\in I$
\[
\int_{x_0}^{x} \frac{y'(t)}{y(t)} dt = - \int_{x_0}^{x} \frac{a_0(t)}{a_1(t)} dt
\]
ή
\[
\log \frac{y(x)}{y(x_0)} = - \int_{x_0}^{x} \frac{a_0(t)}{a_1(t)} dt
\]
που ισοδυναμεί με τον υπόψη τύπο.
Απ' το θεώρημα 2 προκύπτει ότι, αν Α είναι μια συνάρτηση με συνεχή παράγωγο στο Ι και $A' = a_0/a_1$, τότε οι λύσεις της $(E_0)_1$ είναι ακριβώς οι συναρτήσεις $c \exp(-A)$ για τις διάφορες τιμές της σταθεράς c.

\subsubsection{Γραμμική ανεξαρτησία. Ορίζουσα Wronski}

Απ' τη γραμμικότητα του τελεστή L προκύπτει αμέσως το παρακάτω θεώρημα.

\begin{Thewrhma}{3}
Αν $c_k$ (k=1,\dots,m) είναι σταθερές και $y_k$ (k=1,\dots,m) είναι λύσεις της ομογενούς γραμμικής διαφορικής εξίσωσης $(E_0)$, τότε $c_1 y_1 + \dots + c_m y_m$ είναι επίσης μια λύση της $(E_0)$.
\end{Thewrhma}

Ας είναι $f_k$ (k=1,\dots,m) m συναρτήσεις ορισμένες στο διάστημα Ι. Λέμε ότι οι συναρτήσεις αυτές είναι \underline{γραμμικά εξαρτημένες} αν και μόνο αν υπάρχουν αριθμοί $c_k$ (k=1,\dots,m), όχι όλοι μηδέν, έτσι ώστε
\[
c_1 f_1 + \dots + c_m f_m = 0.
\]
Διαφορετικά, δηλαδή όταν η παραπάνω ισότητα ισχύει μόνο για $c_1 = \dots = c_m = 0$, λέμε ότι οι συναρτήσεις $f_k$ (k=1,\dots,m) είναι \underline{γραμμικά ανεξάρτητες}. Αν οι συναρτήσεις $f_k$ (k=1,\dots,m) έχουν (m-1)-τάξης παραγώγους, τότε η ορίζουσα-συνάρτηση
\[
\begin{vmatrix}
f_1 & f_2 & \dots & f_m \\
f'_1 & f'_2 & \dots & f'_m \\
\vdots & \vdots & \ddots & \vdots \\
f_1^{(m-1)} & f_2^{(m-1)} & \dots & f_m^{(m-1)}
\end{vmatrix}
\]
καλείται \underline{ορίζουσα \eng{Wronski}} αυτών και συμβολίζεται με $W(f_1, \dots, f_m)$.
Το θεώρημα 4 δίνει μια ικανή και αναγκαία συνθήκη για τη γραμμική ανεξαρτησία n λύσεων της διαφορικής εξίσωσης $(E_0)$ με τη χρησιμοποίηση της ορίζουσας \eng{Wronski} αυτών. Πιο συγκεκριμένα, n λύσεις της $(E_0)$ είναι γραμμικά ανεξάρτητες αν και μόνο αν η ορίζουσα \eng{Wronski} αυτών δεν μηδενίζεται πουθενά στο διάστημα Ι.

\begin{Thewrhma}{4}
Ας είναι $y_k (k=1,\dots,n) n$ λύσεις της ομογενούς γραμμικής διαφορικής εξίσωσης $(E_0)$. Οι λύσεις αυτές είναι γραμμικά ανεξάρτητες αν και μόνο αν
$W(y_1, \dots, y_n)(x) \ne 0$ για όλα τα $x \in I$.
\end{Thewrhma}
\textsc{ΑΠΟΔΕΙΞΗ.} Υποθέτουμε ότι $W(y_1, \dots, y_n)(x) \ne 0$ για όλα τα $x \in I$ και θεωρούμε n σταθερές $c_k$ (k=1,\dots,n) τέτοιες ώστε $c_1 y_1 + \dots + c_n y_n = 0$. Τότε για κάθε $x \in I$ παίρνουμε
\[
\begin{cdcases}
c_1 y_1(x) + c_2 y_2(x) + \dots + c_n y_n(x) = 0 \\
c_1 y'_1(x) + c_2 y'_2(x) + \dots + c_n y'_n(x) = 0 \\
\ldots\\
c_1 y_1^{(n-1)}(x) + c_2 y_2^{(n-1)}(x) + \dots + c_n y_n^{(n-1)}(x) = 0.
\end{cdcases}
\]
Για ένα σταθερό $x \in I$ οι παραπάνω ισότητες αποτελούν ένα ομογενές γραμμικό (αλγεβρικό) σύστημα ως προς $c_1, \dots, c_n$. Η ορίζουσα του συστήματος αυτού είναι διάφορη του μηδενός και επομένως αυτό έχει μόνο τη μηδενική λύση, δηλαδή $c_1 = \dots = c_n = 0$. Έτσι, οι $y_k (k=1, \dots, n)$ είναι γραμμικά ανεξάρτητες.

Τώρα, ας είναι $y_k (k=1,\dots,n)$ n γραμμικά ανεξάρτητες λύσεις και ας υποθέσουμε ότι υπάρχει $x_0 \in I$ τέτοιο ώστε $W(y_1, \dots, y_n)(x_0) = 0$. Τότε το ομογενές γραμμικό σύστημα
\[
\begin{cdcases}
c_1 y_1(x_0) + c_2 y_2(x_0) + \dots + c_n y_n(x_0) = 0 \\
c_1 y'_1(x_0) + c_2 y'_2(x_0) + \dots + c_n y'_n(x_0) = 0 \\
\ldots \\
c_1 y_1^{(n-1)}(x_0) + c_2 y_2^{(n-1)}(x_0) + \dots + c_n y_n^{(n-1)}(x_0) = 0
\end{cdcases}
\]
έχει ορίζουσα μηδέν και άρα έχει μια μη μηδενική λύση $c_1, \dots, c_n$. Η συνάρτηση $y = c_1 y_1 + \dots + c_n y_n$ είναι (θεώρημα 3) μια λύση της διαφορικής εξίσωσης $(E_0)$ η λύση αυτή πληροί τις αρχικές συνθήκες $y(x_0) = 0, y'(x_0) = 0, \dots, y^{(n-1)}(x_0) = 0$.

Έτσι (θεώρημα 1) η λύση y είναι η μηδενική λύση της $(E_0)$. Άρα, $c_1 y_1 + \dots + c_n y_n = 0$ όπου οι σταθερές $c_k (k=1,\dots,n)$ δεν είναι όλες μηδέν, το οποίο έρχεται σ' αντίθεση με την υπόθεση ότι οι λύσεις $y_k (k=1,\dots,n)$ είναι γραμμικά ανεξάρτητες. Η αντίθεση αυτή προέκυψε απ' το γεγονός ότι υποθέσαμε ότι $W(y_1, \dots, y_n)(x_0) = 0$ για κάποιο $x_0 \in I$. Έχουμε έτσι αναγκαστικά ότι $W(y_1, \dots, y_n)(x) \ne 0$ για κάθε $x \in I$. Μια σπουδαία ιδιότητα της ορίζουσας \eng{Wronski} n λύσεων της ομογενούς γραμμικής διαφορικής εξίσωσης $(E_0)$ είναι ότι αυτή ή δεν μηδενίζεται πουθενά στο διάστημα $Ι$ ή είναι μηδέν σ' ολόκληρο το Ι. Αυτό προκύπτει απ' το παρακάτω θεώρημα.

\begin{Thewrhma}{5}
Ας είναι $x_0$ ένα σημείο του $Ι$ και $y_k (k=1,\dots,n)$ n λύσεις της ομογενούς γραμμικής διαφορικής εξίσωσης $(E_0)$. Τότε ισχύει (\underline{Τύπος του \eng{Liouville}})
\[
W(y_1, \dots, y_n)(x) = W(y_1, \dots, y_n)(x_0) \exp \left[ - \int_{x_0}^{x} \frac{a_{n-1}(t)}{a_n(t)} dt \right] \text{ για κάθε } x \in I.
\]
\end{Thewrhma}

\textsc{ΑΠΟΔΕΙΞΗ.} Θ' αποδείξουμε το συμπέρασμα στην απλή περίπτωση $n=2$ και έπειτα στη γενική περίπτωση οποιουδήποτε n. Η απόδειξη στη γενική περίπτωση απαιτεί τη γνώση του τρόπου παραγώγισης μιας ορίζουσας-συνάρτησης.

\underline{Η περίπτωση n=2.} Έχουμε
\[
W = W(y_1, y_2) = y_1 y'_2 - y'_1 y_2
\]
και επομένως
\begin{align*}
W' &= y'_1 y'_2 + y_1 y''_2 - y''_1 y_2 - y'_1 y'_2 = y_1 y''_2 - y''_1 y_2 \\
&= y_1 \frac{-a_1 y'_2 - a_0 y_2}{a_2} - \frac{-a_1 y'_1 - a_0 y_1}{a_2} y_2 = - \frac{a_1}{a_2} (y_1 y'_2 - y'_1 y_2) \\
&= - \frac{a_1}{a_2} W.
\end{align*}
'Ετσι, η W είναι μια λύση της ομογενούς γραμμικής διαφορικής εξίσωσης πρώτης τάξης
\[
a_2 W' + a_1 W = 0
\]
και άρα (θεώρημα 2) έχουμε
\[
W(x) = W(x_0) \exp \left[ - \int_{x_0}^{x} \frac{a_1(t)}{a_2(t)} dt \right], \quad x \in I.
\]

\underline{Η γενική περίπτωση.} Θέτουμε $W = W(y_1, \dots, y_n)$ και παίρνουμε
\begin{align*}
W' &= 
\begin{vmatrix}
y'_1 & y'_2 & \dots & y'_n \\
y'_1 & y'_2 & \dots & y'_n \\
y''_1 & y''_2 & \dots & y''_n \\
\vdots & \vdots & \ddots & \vdots \\
y_1^{(n-1)} & y_2^{(n-1)} & \dots & y_n^{(n-1)}
\end{vmatrix}
+
\begin{vmatrix}
y_1 & y_2 & \dots & y_n \\
y''_1 & y''_2 & \dots & y''_n \\
y''_1 & y''_2 & \dots & y''_n \\
\vdots & \vdots & \ddots & \vdots \\
y_1^{(n-1)} & y_2^{(n-1)} & \dots & y_n^{(n-1)}
\end{vmatrix}
+ \dots \\
\dots &+
\begin{vmatrix}
y_1 & y_2 & \dots & y_n \\
y'_1 & y'_2 & \dots & y'_n \\
\vdots & \vdots & \ddots & \vdots \\
y_1^{(n-1)} & y_2^{(n-1)} & \dots & y_n^{(n-1)} \\
y_1^{(n-1)} & y_2^{(n-1)} & \dots & y_n^{(n-1)}
\end{vmatrix}
+
\begin{vmatrix}
y_1 & y_2 & \dots & y_n \\
y'_1 & y'_2 & \dots & y'_n \\
\vdots & \vdots & \ddots & \vdots \\
y_1^{(n-2)} & y_2^{(n-2)} & \dots & y_n^{(n-2)} \\
y_1^{(n)} & y_2^{(n)} & \dots & y_n^{(n)}
\end{vmatrix}
\\
&=
\begin{vmatrix}
y_1 & y_2 & \dots & y_n \\
y'_1 & y'_2 & \dots & y'_n \\
\vdots & \vdots & \ddots & \vdots \\
y_1^{(n-2)} & y_2^{(n-2)} & \dots & y_n^{(n-2)} \\
y_1^{(n)} & y_2^{(n)} & \dots & y_n^{(n)}
\end{vmatrix}
\\
&=
\begin{vmatrix}
y_1 & y_2 & \dots & y_n \\
y'_1 & y'_2 & \dots & y'_n \\
\vdots & \vdots & \ddots & \vdots \\
y_1^{(n-2)} & y_2^{(n-2)} & \dots & y_n^{(n-2)} \\
-\sum_{i=0}^{n-1} \frac{a_i}{a_n} y_1^{(i)} & -\sum_{i=0}^{n-1} \frac{a_i}{a_n} y_2^{(i)} & \dots & -\sum_{i=0}^{n-1} \frac{a_i}{a_n} y_n^{(i)}
\end{vmatrix}
\end{align*}
\begin{align*}
&= -
\begin{vmatrix}
y_1 & y_2 & \dots & y_n \\
y'_1 & y'_2 & \dots & y'_n \\
\vdots & \vdots & \ddots & \vdots \\
y_1^{(n-2)} & y_2^{(n-2)} & \dots & y_n^{(n-2)} \\
\sum_{i=0}^{n-1} \frac{a_i}{a_n} y_1^{(i)} & \sum_{i=0}^{n-1} \frac{a_i}{a_n} y_2^{(i)} & \dots & \sum_{i=0}^{n-1} \frac{a_i}{a_n} y_n^{(i)}
\end{vmatrix}
\\
&= - \frac{a_{n-1}}{a_n}
\begin{vmatrix}
y_1 & y_2 & \dots & y_n \\
y'_1 & y'_2 & \dots & y'_n \\
\vdots & \vdots & \ddots & \vdots \\
y_1^{(n-1)} & y_2^{(n-1)} & \dots & y_n^{(n-1)}
\end{vmatrix}.
\end{align*}
'Ετσι, έχουμε $a_n W' + a_{n-1} W = 0$ απ' όπου προκύπτει (θεώρημα 2) το ζητούμενο.

\subsubsection{Βασικά σύνολα λύσεων}

'Ενα σύνολο n γραμμικά ανεξάρτητων λύσεων της ομογενούς γραμμικής διαφορικής εξίσωσης $(E_0)$ καλείται ένα \underline{βασικό σύνολο λύσεων} αυτής. Σύμφωνα με το θεώρημα 4, αν $y_k$ (k=1,\dots,n) είναι n λύσεις της $(E_0)$, τότε $\{y_1, \dots, y_n\}$ είναι ένα βασικό σύνολο λύσεων αυτής αν και μόνο αν η ορίζουσα Wronski των $y_k$ (k=1,\dots,n) δεν μηδενίζεται πουθενά στο Ι (ή, ισοδύναμα, δεν μηδενίζεται σ' ένα τουλάχιστον σημείο του Ι). Το παρακάτω θεώρημα εξασφαλίζει την ύπαρξη βασικών συνόλων λύσεων της $(E_0)$.

\begin{Thewrhma}{6}
Υπάρχουν βασικά σύνολα λύσεων της ομογενούς γραμμικής διαφορικής εξίσωσης $(E_0)$.
\end{Thewrhma}

\textsc{ΑΠΟΔΕΙΞΗ.} Θεωρούμε ένα $x_0 \in I$. Για κάθε $k \in \{1, \dots, n\}$, υπάρχει (θεώρημα 1) λύση $y_k$ της διαφορικής εξίσωσης $(E_0)$ που πληροί τις αρχικές συνθήκες
\[
y_k^{(i)}(x_0) = 0 \quad (i=0,1,\dots,n-1, i \ne k-1), \quad y_k^{(k-1)}(x_0) = 1.
\]
Οι λύσεις $y_1, \dots, y_n$ είναι γραμμικά ανεξάρτητες. Πραγματικά: θεωρούμε n σταθερές $c_k$ (k=1,\dots,n) και υποθέτουμε ότι
\[
c_1 y_1 + c_2 y_2 + \dots + c_n y_n = 0.
\]
Με παραγωγίσεις παίρνουμε
\[
\begin{cdcases}
c_1 y_1 + c_2 y_2 + \dots + c_n y_n = 0 \\
\vdots \\
c_1 y_1^{(n-1)} + c_2 y_2^{(n-1)} + \dots + c_n y_n^{(n-1)} = 0
\end{cdcases}
\]
θέτοντας στις παραπάνω ισότητες $x=x_0$ και λαμβάνοντας υπόψη τις αρχικές συνθήκες, βρίσκουμε αμέσως ότι $c_1=c_2=\dots=c_n=0$. Η γραμμική ανεξαρτησία των λύσεων $y_k$ (k=1,\dots,n) μπορεί επίσης να εξασφαλισθεί απ' το γεγονός ότι $W(y_1,\dots,y_n)(x_0) = 1 \ne 0$, σύμφωνα με το θεώρημα 4.

\begin{Thewrhma}{7}
Ας είναι $\{y_1, \dots, y_n\}$ ένα βασικό σύνολο λύσεων της ομογενούς γραμμικής διαφορικής εξίσωσης $(E_0)$. Για κάθε λύση $y$ της $(E_0)$ υπάρχουν $n$ μονοσήμαντα ορισμένες σταθερές $c_k,\ (k=1,\dots,n)$ έτσι ώστε
\[
y = c_1 y_1 + \dots + c_n y_n.
\]
\end{Thewrhma}

\textsc{ΑΠΟΔΕΙΞΗ.} Το μονοσήμαντο των σταθερών $c_k (k=1,\dots,n)$ προκύπτει αμέσως απ' τη γραμμική ανεξαρτησία των λύσεων $y_k,\ (k=1,\dots,n)$.

Ας είναι y μια λύση της $(E_0)$ και ας θέσουμε
\[
y(x_0) = a_0, y'(x_0)=a_1, \dots, y^{(n-1)}(x_0)=a_{n-1},
\]
όπου $x_0$ είναι ένα σημείο του Ι. Το γραμμικό σύστημα
\[
\begin{cdcases}
c_1 y_1(x_0) + c_2 y_2(x_0) + \dots + c_n y_n(x_0) = a_0 \\
c_1 y'_1(x_0) + c_2 y'_2(x_0) + \dots + c_n y'_n(x_0) = a_1 \\
\vdots \\
c_1 y_1^{(n-1)}(x_0) + c_2 y_2^{(n-1)}(x_0) + \dots + c_n y_n^{(n-1)}(x_0) = a_{n-1}
\end{cdcases}
\]
έχει ορίζουσα την $W(y_1, \dots, y_n)(x_0)$ που δεν είναι μηδέν (θεώρημα 4). 'Ετσι, το σύστημα αυτό έχει μια λύση $c_1, \dots, c_n$. Τότε η συνάρτηση $u = c_1 y_1 + \dots + c_n y_n$ είναι (θεώρημα 3) μια λύση της διαφορικής εξίσωσης $(E_0)$. Η λύση $u$ πληροί τις αρχικές συνθήκες
\[
u(x_0)=a_0, u'(x_0)=a_1, \dots, u^{(n-1)}(x_0) = a_{n-1}.
\]
Άρα είναι (θεώρημα 1) $y=u$ και επομένως $y = c_1 y_1 + \dots + c_n y_n$.
Συνοψίζοντας τα θεωρήματα 3 και 7, συμπεραίνουμε ότι, αν $\{y_1, \dots, y_n\}$ είναι ένα βασικό σύνολο λύσεων της ομογενούς γραμμικής διαφορικής εξίσωσης $(E_0)$, τότε η y είναι μια λύση αυτής αν και μόνο αν υπάρχουν σταθερές $c_k (k=1,\dots,n)$ τέτοιες ώστε $y=c_1 y_1 + \dots + c_n y_n$. Το θεώρημα 6 εξασφαλίζει την ύπαρξη βασικών συνόλων λύσεων της ομογενούς γραμμικής διαφορικής εξίσωσης $(E_0)$. Τώρα, δημιουργείται το ερώτημα κατά πόσο υπάρχει (και είναι μοναδική) και πως μπορεί να κατασκευασθεί μια ομογενής γραμμική διαφορική εξίσωση $n$-τάξης η οποία να έχει ως ένα βασικό σύνολο λύσεων το $\{y_1, \dots, y_n\}$, όπου $y_k (k=1,\dots,n)$ είναι δεδομένες συναρτήσεις που έχουν συνεχείς παραγώγους n-τάξης στο Ι και ορίζουσα \eng{Wronski} που δεν μηδενίζεται πουθενά στο Ι. Το επόμενο θεώρημα δίνει απάντηση στο ερώτημα αυτό.

\begin{Thewrhma}{8}
Ας είναι $y_k,\ (k=1,\dots,n) n$ συναρτήσεις που έχουν συνεχείς παραγώγους $n$-τάξης στο $Ι$ και ας υποθέσουμε ότι
\[
    W(y_1, \dots, y_n)(x) \ne 0 \text{ για όλα τα } x \in I.
\]
Τότε υπάρχει μια μοναδική ομογενής γραμμική διαφορική εξίσωση $n$-τάξης της μορφής $(E_0)$ με $a_n=1$ που έχει το $\{y_1, \dots, y_n\}$ ως ένα βασικό σύνολο λύσεων. Η διαφορική αυτή εξίσωση είναι η
\begin{equation*} \label{eq:star} \tag{*}
    \frac{W(y_1, \dots, y_n, y)}{W(y_1, \dots, y_n)} = 0.
\end{equation*}
\end{Thewrhma}

\textsc{ΑΠΟΔΕΙΞΗ.} Η (*) γράφεται
\[
    \frac{1}{W(y_1, \dots, y_n)}
    \begin{vmatrix}
        y_1 & y_2 & \dots & y_n & y \\
        y'_1 & y'_2 & \dots & y'_n & y' \\
        \vdots & \vdots & \ddots & \vdots & \vdots \\
        y_1^{(n-1)} & y_2^{(n-1)} & \dots & y_n^{(n-1)} & y^{(n-1)} \\
        y_1^{(n)} & y_2^{(n)} & \dots & y_n^{(n)} & y^{(n)}
    \end{vmatrix} = 0
\]
και είναι της μορφής $(E_0)$ με
\begin{align*}
    a_n &= \frac{1}{W(y_1, \dots, y_n)}
    \begin{vmatrix}
        y_1 & y_2 & \dots & y_n \\
        y'_1 & y'_2 & \dots & y'_n \\
        \vdots & \vdots & \ddots & \vdots \\
        y_1^{(n-1)} & y_2^{(n-1)} & \dots & y_n^{(n-1)}
    \end{vmatrix} = 1, \\
    a_{n-1} &= -\frac{1}{W(y_1, \dots, y_n)}
    \begin{vmatrix}
        y_1 & y_2 & \dots & y_n \\
        y'_1 & y'_2 & \dots & y'_n \\
        \vdots & \vdots & \ddots & \vdots \\
        y_1^{(n-2)} & y_2^{(n-2)} & \dots & y_n^{(n-2)} \\
        y_1^{(n)} & y_2^{(n)} & \dots & y_n^{(n)}
    \end{vmatrix}, \\
    \dots, a_1 &= \frac{(-1)^{n-1}}{W(y_1, \dots, y_n)}
    \begin{vmatrix}
        y_1 & y_2 & \dots & y_n \\
        y'_1 & y'_2 & \dots & y'_n \\
        \vdots & \vdots & \ddots & \vdots \\
        y_1^{(n)} & y_2^{(n)} & \dots & y_n^{(n)}
    \end{vmatrix}, \\
    a_0 &= \frac{1}{W(y_1, \dots, y_n)}
    \begin{vmatrix}
        y'_1 & y'_2 & \dots & y'_n \\
        \vdots & \vdots & \ddots & \vdots \\
        y_1^{(n)} & y_2^{(n)} & \dots & y_n^{(n)}
    \end{vmatrix}.
\end{align*}
Οι συντελεστές $a_i,\ (i=0,1,\dots,n-1,n)$ είναι συνεχείς συναρτήσεις στο Ι. Είναι τώρα φανερό ότι οι συναρτήσεις $y_k,\ (k=1,\dots,n)$ είναι λύσεις της διαφορικής εξίσωσης (*). Επειδή $W(y_1, \dots, y_n)(x) \ne 0$ για όλα τα $x \in I$, σύμφωνα με το θεώρημα 4, $\{y_1, \dots, y_n\}$ θα είναι ένα βασικό σύνολο λύσεων της (*).

Θ' αποδείξουμε, τώρα, ότι η (*) είναι η μόνη διαφορική εξίσωση της μορφής $(E_0)$ με $a_n=1$ που έχει το $\{y_1, \dots, y_n\}$ ως ένα βασικό σύνολο λύσεων. Έτσι, υποθέτουμε ότι $\{y_1, \dots, y_n\}$ είναι ένα βασικό σύνολο λύσεων καθεμιάς των παρακάτω δύο ομογενών γραμμικών διαφορικών εξισώσεων
\begin{align*}
    y^{(n)} + a_{n-1} y^{(n-1)} + \dots + a_1 y' + a_0 y &= 0, \\
    y^{(n)} + r_{n-1} y^{(n-1)} + \dots + r_1 y' + r_0 y &= 0
\end{align*}
(όπου $a_i$ και $r_i,\ (i=0,1,\dots,n-1)$ είναι συνεχείς συναρτήσεις στο Ι) οπότε αρκεί ν' αποδείξουμε ότι $a_i=r_i,\ (i=0,1,\dots,n-1)$. Ας υποθέσουμε πρώτα ότι $a_i=r_i,\ (i=1,\dots,n-1)$. $a_0 \ne r_0,\ (k=1,\dots,n)$. Για κάποιο σημείο $x_0 \in I$ είναι $a_0(x_0) \ne r_0(x_0)$, τότε υπάρχει ένα υποδιάστημα $J$ του $Ι$ τέτοιο ώστε $a_0(x) \ne r_0(x)$ για όλα τα $x \in J$. 'Ετσι, θα έχουμε $a_0(x) \ne 0$ για τα $x \in J$ και για $k=1,\dots,n$, οπότε $W(y_1, \dots, y_n)(x) = 0$ όταν $x \in J$, αυτό δεν μπορεί να συμβαίνει και άρα $a_0 = r_0$. Τώρα, ας κάνουμε την υπόθεση ότι υπάρχουν δείκτες $i \in \{1, \dots, n-1\}$ τέτοιοι ώστε $a_i \ne r_i$ και ας ονομάσουμε μ τον μεγαλύτερο τέτοιο δείκτη, δηλαδή ας είναι $\mu \in \{1, \dots, n-1\}$ έτσι, ώστε $a_\mu \ne r_\mu$ και $a_i=r_i$ για i=$\mu+1, \dots, n-1$. Θεωρούμε ένα σημείο $\tilde{x} \in I$ τέτοιο ώστε $a_\mu(\tilde{x}) \ne r_\mu(\tilde{x})$. Τότε για ένα διάστημα $J$, υποδιάστημα του $Ι$, θα είναι $a_\mu(x) \ne r_\mu(x)$ για όλα τα $x \in J$. Παρατηρούμε ότι οι περιορισμοί των συναρτήσεων $y_k,\ (k=1,\dots,n)$ στο $J$ είναι λύσεις της ομογενούς γραμμικής διαφορικής εξίσωσης $\mu$-τάξης
\[
    (a_\mu-r_\mu)y^{(\mu)} + (a_{\mu-1}-r_{\mu-1})y^{(\mu-1)} + \dots + (a_1-r_1)y' + (a_0-r_0)y = 0,
\]
η οποία θεωρείται με διάστημα ορισμού το J. Θεωρούμε ένα σημείο $\tilde{x}$ στο διάστημα J και το ομογενές γραμμικό σύστημα
\[
    \begin{cdcases}
        c_1 y_1(\tilde{x}) + c_2 y_2(\tilde{x}) + \dots + c_n y_n(\tilde{x}) = 0 \\
        c_1 y'_1(\tilde{x}) + c_2 y'_2(\tilde{x}) + \dots + c_n y'_n(\tilde{x}) = 0 \\
        \vdots \\
        c_1 y_1^{(\mu-1)}(\tilde{x}) + c_2 y_2^{(\mu-1)}(\tilde{x}) + \dots + c_n y_n^{(\mu-1)}(\tilde{x}) = 0.
    \end{cdcases}
\]
Το σύστημα αυτό έχει μ εξισώσεις και n αγνώστους. Επειδή $\mu < n$, θα έχει μια μη μηδενική λύση $c_1, \dots, c_n$. Η συνάρτηση $\tilde{y} = c_1 y_1 + \dots + c_n y_n$ ορισμένη στο J είναι (θεώρημα 3) μια λύση της παραπάνω διαφορικής εξίσωσης. Η λύση αυτή ισχύει, όπως προκύπτει απ' την εκλογή των $c_1, \dots, c_n$ τις αρχικές συνθήκες
\[
    \tilde{y}(\tilde{x}) = 0, \tilde{y}'(\tilde{x})=0, \dots, \tilde{y}^{(\mu-1)}(\tilde{x})=0.
\]
'Ετσι, θα έχουμε (θεώρημα 1) $\tilde{y}=0$, δηλαδή
\[
    c_1 y_1(x) + c_2 y_2(x) + \dots + c_n y_n(x) = 0
\]
για όλα τα $x \in J$, όπου οι σταθερές $c_1, \dots, c_n$ δεν είναι όλες μηδέν. Αυτό έρχεται σ' αντίθεση με τη γραμμική ανεξαρτησία των $y_k$ ($k=1, \dots, n$).

\subsubsection{Υποβιβασμός της τάξης}
Αν γνωρίζουμε μια λύση $y_1$, με $y_1(x) \ne 0$ για όλα τα $x \in I$, της ομογενούς γραμμικής διαφορικής εξίσωσης $(E_0)$, τότε οι αντικαταστάσεις $y=uy_1$ και $v=u'$ μετασχηματίζουν την $(E_0)$ σε μια ομογενή γραμμική διαφορική εξίσωση $(n-1)$-τάξης από ένα βασικό σύνολο λύσεων της εξίσωσης που προκύπτει μπορούμε να πάρουμε ένα βασικό σύνολο λύσεων για την $(E_0)$. 'Ετσι, έχουμε το παρακάτω θεώρημα.

\begin{Thewrhma}{9}
Ας είναι $y_1$ μια λύση της ομογενούς γραμμικής διαφορικής εξίσωσης $(E_0)$ με $y_1(x) \ne 0$ για όλα τα $x \in I$. Για $y=uy_1, v=u'$ η $(E_0)$ μετασχηματίζεται σε μια $(n-1)$-τάξης ομογενή γραμμική διαφορική εξίσωση $(E_0)^*$. Αν $\{v_1, \dots, v_{n-1}\}$ είναι ένα βασικό σύνολο λύσεων της $(E_0)^*$ και
\[
    y_i(x) = y_1(x) \int_{x_0}^x v_{i-1}(t) dt, \quad x \in I \quad (i=2, \dots, n),
\]
όπου $x_0$ είναι ένα σημείο του Ι, τότε $\{y_1, y_2, \dots, y_n\}$ είναι ένα βασικό σύνολο λύσεων της $(E_0)$.
\end{Thewrhma}

\textsc{ΑΠΟΔΕΙΞΗ.} Ας είναι $y=uy_1$. Τότε (με τον τύπο του \eng{Leibnitz}) έχουμε
\[
    \begin{dcases}
        y' = y'_1 u + y_1 u' \\
        \vdots \\
        y^{(n-1)} = y_1^{(n-1)}u + (n-1)y_1^{(n-2)}u' + \dots + y_1 u^{(n-1)} \\
        y^{(n)} = y_1^{(n)}u + n y_1^{(n-1)}u' + \frac{n(n-1)}{2} y_1^{(n-2)}u'' + \dots + y_1 u^{(n)}
    \end{dcases}
\]
και έτσι η $(E_0)$ μετασχηματίζεται στη διαφορική εξίσωση
\[
    a_n [y_1 u^{(n)} + n y'_1 u^{(n-1)} + \dots] + a_{n-1} [y_1^{(n-1)}u + (n-1)y_1^{(n-2)}u' + \dots] + \dots
\]
\[
\dots + y_1 u^{(n-1)}] + \dots + a_1(y'_1 u + y_1 u') + a_0 y_1 u = 0
\]
ή
\[
a_n y_1 u^{(n)} + [na_n y'_1 + a_{n-1} y_1] u^{(n-1)} + \dots + [na_n y_1^{(n-1)} + \dots + a_1 y_1] u' + [a_n y_1^{(n)} + a_{n-1} y_1^{(n-1)} + \dots + a_1 y'_1 + a_0 y_1] u = 0.
\]
Λαμβάνοντας υπόψη ότι η $y_1$ είναι μια λύση της $(E_0)$ και θέτοντας $u'=v$, παίρνουμε τη διαφορική εξίσωση
\[
    (E_0)^* \quad A_{n-1} v^{(n-1)} + A_{n-2} v^{(n-2)} + \dots + A_0 v = 0,
\]
όπου
\[
    \begin{dcases}
        A_{n-1} = a_n y_1, \\
        A_{n-2} = na_n y'_1 + a_{n-1} y_1, \\
        \vdots \\
        A_0 = na_n y_1^{(n-1)} + \dots + a_1 y_1.
    \end{dcases}
\]
Οι συναρτήσεις $A_i$ ($i=0, \dots, n-2, n-1$) είναι συνεχείς στο διάστημα Ι και ο συντελεστής $A_{n-1}$ δεν μηδενίζεται πουθενά στο Ι. 'Ετσι, η $(E_0)^*$ είναι μια ομογενής γραμμική διαφορική εξίσωση $(n-1)$-τάξης. Ας είναι τώρα $\{v_1, \dots, v_{n-1}\}$ ένα βασικό σύνολο λύσεων της $(E_0)^*$. Ας θεωρήσουμε ένα σημείο $x_0 \in I$ και ας θέσουμε
\[
    y_i(x) = y_1(x) \int_{x_0}^x v_{i-1}(t) dt, \quad x \in I \quad (i=2, \dots, n).
\]
Είναι τότε φανερό ότι οι συναρτήσεις $y_i$ ($i=2, \dots, n$) είναι λύσεις της διαφορικής εξίσωσης $(E_0)$ $\cdot$ θα δείξουμε ότι οι λύσεις $y_k$ ($k=1, \dots, n$) είναι γραμμικά ανεξάρτητες. Ας είναι λοιπόν $c_k$ ($k=1, \dots, n$) σταθερές τέτοιες ώστε
\[
    c_1 y_1(x) + c_2 y_2(x) + \dots + c_n y_n(x) = 0.
\]
Τότε απ' τον τρόπο ορισμού των $y_i$ ($i=2, \dots, n$) και απ' το γεγονός ότι $y_1(x) \ne 0$ για όλα τα $x \in I$ προκύπτει ότι
\[
    c_1 + c_2 \int_{x_0}^x v_1(t) dt + \dots + c_n \int_{x_0}^x v_{n-1}(t) dt = 0 \text{ για } x \in I.
\]
Παραγωγίζοντας παίρνουμε
\[
    c_2 v_1(x) + \dots + c_n v_{n-1}(x) = 0 \text{ για κάθε } x \in I
\]
αφ' ότου, επειδή οι $v_i$ ($i=1, \dots, n-1$) είναι γραμμικά ανεξάρτητες, συμπεραίνεται ο μηδενισμός όλων των σταθερών $c_2, \dots, c_n$. Τότε είναι και η σταθερά $c_1$ ίση με μηδέν.

Ιδιαίτερο ενδιαφέρον παρουσιάζει η μέθοδος του υποβιβασμού της τάξης στην ειδική περίπτωση των ομογενών γραμμικών διαφορικών εξισώσεων δεύτερης τάξης, επειδή σ' αυτή την περίπτωση η εξίσωση που προκύπτει είναι μια ομογενής γραμμική διαφορική εξίσωση πρώτης τάξης και άρα μπορεί να επιλυθεί (θεώρημα 2). 'Ετσι, για τη διαφορική εξίσωση
\[
    (E_0')_2 \quad a_2 y'' + a_1 y' + a_0 y = 0
\]
έχουμε το παρακάτω θεώρημα.

\begin{Thewrhma}{10}
Αν $y_1$ είναι μια λύση της ομογενούς γραμμικής διαφορικής εξίσωσης δεύτερης τάξης $(E_0')_2$ με $y_1(x) \ne 0$ για όλα τα $x \in I$ και
\[
    y_2(x) = y_1(x) \int_{x_0}^x \frac{1}{y_1^2(t)} \exp \left[ -\int_{x_0}^t \frac{a_1(s)}{a_2(s)} ds \right] dt, \quad x \in I,
\]
όπου $x_0$ είναι ένα σημείο του Ι, τότε $\{y_1, y_2\}$ είναι ένα βασικό σύνολο λύσεων της $(E_0')_2$.
\end{Thewrhma}

\textsc{ΑΠΟΔΕΙΞΗ.} Θέτουμε $y=uy_1$ και $u'=v$. Τότε
\begin{align*}
    a_2 y'' + a_1 y' + a_0 y &= a_2 (u y_1'' + 2u'y_1' + u''y_1) + a_1(u'y_1+uy_1') + a_0 u y_1 \\
    &= a_2(u y_1'' + 2u'y_1' + u''y_1) + a_1(u'y_1+uy_1') + a_0 u y_1 \\
    &= a_2 y_1 u'' + (2a_2 y_1' + a_1 y_1) u' + (a_2 y_1'' + a_1 y_1' + a_0 y_1) \\
    &= a_2 y_1 u'' + (2a_2 y_1' + a_1 y_1) u' \\
    &= a_2 y_1 v' + (2a_2 y_1' + a_1 y_1) v.
\end{align*}
'Ετσι, η διαφορική μας εξίσωση μετασχηματίζεται στην ομογενή γραμμική διαφορική εξίσωση πρώτης τάξης
\[
    a_2 y_1 v' + (2a_2 y_1' + a_1 y_1) v = 0.
\]
Αυτή έχει (θεώρημα 2) τη λύση $v_1$ με
\[
    v_1(x) = \frac{1}{y_1^2(x_0)} \exp \left[ -\int_{x_0}^x \frac{2a_2(t)y_1'(t) + a_1(t)y_1(t)}{a_2(t)y_1(t)} dt \right], \quad x \in I.
\]
Για κάθε $x \in I$ έχουμε
\begin{align*}
    v_1(x) &= \frac{1}{y_1^2(x_0)} \exp \left[ -\int_{x_0}^x \frac{y_1'(t)}{y_1(t)} dt - \int_{x_0}^x \frac{a_1(t)}{a_2(t)} dt \right] \\
    &= \frac{1}{y_1^2(x_0)} \exp \left[ -\log\frac{y_1(x)}{y_1(x_0)} - \int_{x_0}^x \frac{a_1(t)}{a_2(t)} dt \right] = \frac{1}{y_1^2(x_0)} \frac{y_1(x_0)}{y_1(x)} \exp \left[ -\int_{x_0}^x \frac{a_1(t)}{a_2(t)} dt \right].
\end{align*}
Απ' τους μετασχηματισμούς που χρησιμοποιήσαμε προκύπτει ότι κάθε συνάρτηση $y$ με $(y/y_1)'=v_1$ είναι μια λύση της $(E_0')_2$. Μια τέτοια λύση είναι η συνάρτηση $y_2$ με $y_2(x)/y_1(x) = \int_{x_0}^x v_1(t) dt$, $x \in I$.

Θ' αποδείξουμε τώρα ότι οι λύσεις $y_1, y_2$ είναι γραμμικά ανεξάρτητες. Αυτό προκύπτει αμέσως, επειδή για όλα τα $x \in I$ είναι
\[
    W(y_1, y_2)(x) = \begin{vmatrix} y_1(x) & y_2(x) \\ y_1'(x) & y_2'(x) \end{vmatrix} = \begin{vmatrix} y_1(x) & y_1(x) \int_{x_0}^x v_1(t) dt \\ y_1'(x) & y_1'(x) \int_{x_0}^x v_1(t) dt + y_1(x) v_1(x) \end{vmatrix} = A(x) \ne 0,
\]
όπου $A(x) = \exp \left[ -(a_1(t)/a_2(t))dt \right]$.

\subsubsection{Παραδείγματα}
\begin{Paradeigma}{1}
Να επιλυθεί η ομογενής γραμμική διαφορική εξίσωση πρώτης τάξης
\[
    xy' - \frac{1}{2\log x} y = 0, \quad x>1.
\]
Ιδιαίτερα, να βρεθεί η λύση $y_1$ αυτής που πληροί την αρχική συνθήκη $y_1(e)=1$.
\end{Paradeigma}
Σύμφωνα με το θεώρημα 2, η $y=c$ είναι μια λύση της διαφορικής μας εξίσωσης αν και μόνο αν
\[
    y(x) = y(e) \exp\left[-\int_e^x \frac{1}{2t\log t} dt\right] = y(e) \exp\left[\frac{1}{2} \log t\right]_e^x = y(e) (\log x)^{1/2}
\]
για κάθε $x>1$. 'Ετσι, όλες οι λύσεις δίνονται απ' τον τύπο
\[
    y(x) = c(\log x)^{1/2}, \quad x>1,
\]
όπου c είναι αυθαίρετη σταθερά. Ειδικά, η λύση με $y_1(e)=1$ είναι $y_1(x) = (\log x)^{1/2}, x>1$.

\begin{Paradeigma}{2}
Ν' αποδειχθεί ότι: 
\begin{rlist}
    \item οι συναρτήσεις $f_1(x)=\sin x, x \in \mathbb{R}$, $f_2(x)=3\sin x, x \in \mathbb{R}$ και $f_3(x)=\sin x, x \in \mathbb{R}$ είναι γραμμικά εξαρτημένες.
    \item οι συναρτήσεις $g_1(x)=1, x \in \mathbb{R}$, $g_2(x)=\cos x, x \in \mathbb{R}$ και $g_3(x)=\cos 2x, x \in \mathbb{R}$ είναι γραμμικά ανεξάρτητες.
\end{rlist}
\end{Paradeigma}
 
\begin{rlist}
    \item Είναι $1 \cdot f_1 - 1 \cdot f_2 + 4 \cdot f_3 = 0$, που αποδεικνύει τη γραμμική εξάρτηση των συναρτήσεων $f_1, f_2$ και $f_3$.
    \item Ας υποθέσουμε ότι $c_1 g_1 + c_2 g_2 + c_3 g_3 = 0$, όπου $c_1, c_2$ και $c_3$ είναι σταθερές. Τότε θα έχουμε $c_1+c_2\cos x+c_3\cos 2x=0$ για όλα τα $x \in \mathbb{R}$. Για $x=0, x=\pi/2$ και $x=\pi$ παίρνουμε αντίστοιχα $c_1+c_2+c_3=0, c_1-c_3=0$ και $c_1-c_2+c_3=0$, απ' όπου προκύπτει $c_1=c_2=c_3=0$, και άρα οι $g_1, g_2$ και $g_3$ είναι γραμμικά ανεξάρτητες.
\end{rlist}

\begin{Paradeigma}{3}
Να επιλυθεί η ομογενής γραμμική διαφορική εξίσωση
\[
    x^3 y''' - 4x^2 y'' + 8xy' - 8y=0, \quad x>0,
\]
αφού βρεθούν οι λύσεις αυτής της μορφής $y(x)=x^v$ ($v$ ακέραιος). Ιδιαίτερα, να βρεθεί η λύση $y_0$ που πληροί τις αρχικές συνθήκες
\[
    y_0(1)=0, \quad y_0'(1)=1, \quad y_0''(1)=2.
\]
\end{Paradeigma}

Η $y(x)=x^v, x>0$ είναι μια λύση της διαφορικής εξίσωσης αν και μόνο αν για όλα τα $x>0$ είναι
\[
    x^3(x^v)''' - 4x^2(x^v)'' + 8x(x^v)' - 8y = 0
\]
ή
\[
    v(v-1)(v-2)x^v - 4v(v-1)x^v + 8vx^v - 8x^v=0,
\]
δηλαδή αν και μόνο αν
\[
    v(v-1)(v-2)-4v(v-1)+8v-8=0.
\]
Αυτό ισχύει τότε και μόνο τότε αν $v=1$ ή $v=2$ ή $v=4$. 'Ετσι, οι λύσεις της μορφής $y(x)=x^v, x>0$ ($v$ ακέραιος) είναι ακριβώς οι συναρτήσεις
\[
    y_1(x)=x, x>0; \quad y_2(x)=x^2, x>0; \quad \text{και} \quad y_3(x)=x^4, x>0.
\]
'Εχουμε για κάθε $x>0$
\[
    W(y_1, y_2, y_3)(x) = \begin{vmatrix} y_1(x) & y_2(x) & y_3(x) \\ y_1'(x) & y_2'(x) & y_3'(x) \\ y_1''(x) & y_2''(x) & y_3''(x) \end{vmatrix} = \begin{vmatrix} x & x^2 & x^4 \\ 1 & 2x & 4x^3 \\ 0 & 2 & 12x^2 \end{vmatrix} = 2x^4 \ne 0.
\]
'Αρα (θεώρημα 4) οι λύσεις $y_1, y_2, y_3$ αποτελούν ένα βασικό σύνολο λύσεων. Επομένως (θεωρήματα 3 και 7) οι λύσεις της διαφορικής μας εξίσωσης δίνονται απ' τον τύπο $y(x)=c_1 x + c_2 x^2 + c_3 x^4$, όπου $c_1, c_2$ και $c_3$ είναι αυθαίρετες σταθερές. Για τη λύση $y_0$ θα έχουμε για $x>0$
\begin{align*}
    y_0(x) &= c_1 x + c_2 x^2 + c_3 x^4, \\
    y_0'(x) &= c_1 + 2c_2 x + 4c_3 x^3, \\
    y_0''(x) &= 2c_2 + 12c_3 x^2
\end{align*}
και έτσι για $x=1$ παίρνουμε
\[
    0=c_1+c_2+c_3, \quad 1=c_1+2c_2+4c_3, \quad 2=2c_2+12c_3
\]
απ' όπου προκύπτει $c_1=-1, c_2=1, c_3=0$. 'Αρα, $y_0(x)=-x+x^2, x>0$.

\begin{Paradeigma}{4}
Να βρεθεί η ορίζουσα \eng{Wronski} των συναρτήσεων $y_1(x)=e^x, x \in \mathbb{R}$; $y_2(x)=e^{-x}, x \in \mathbb{R}$ και $y_3(x)=2e^x-e^{-2x}, x \in \mathbb{R}$, αφού διαπιστωθεί ότι αυτές είναι λύσεις της ομογενούς γραμμικής διαφορικής εξίσωσης $y'''-4y''+5y'-2y=0$.
\end{Paradeigma}
\lysh
Εύκολα διαπιστώνεται ότι καθεμιά απ' τις συναρτήσεις $y_1, y_2$ και $y_3$ είναι μια λύση της διαφορικής εξίσωσης $y'''-4y''+5y'-2y=0$. Για κάθε $x \in \mathbb{R}$ είναι
\[
    W(y_1, y_2, y_3)(x) = \begin{vmatrix} y_1(x) & y_2(x) & y_3(x) \\ y_1'(x) & y_2'(x) & y_3'(x) \\ y_1''(x) & y_2''(x) & y_3''(x) \end{vmatrix} = \begin{vmatrix} e^x & e^x(x-1) & 2e^x-2e^{-2x} \\ e^x & e^x x & 2e^x+4e^{-2x} \\ e^x & e^x(x+1) & 2e^x-8e^{-2x} \end{vmatrix}
\]
και έτσι
\[
    W(y_1, y_2, y_3)(0) = \begin{vmatrix} 1 & -1 & 1 \\ 1 & 0 & 0 \\ 1 & 1 & -2 \end{vmatrix} = -1.
\]
Με τη βοήθεια του τύπου του Liouville (θεώρημα 5), για όλα τα $x \in \mathbb{R}$
\[
    W(y_1, y_2, y_3)(x) = W(y_1, y_2, y_3)(0) \exp\left(-\int_0^x -4 dt\right) = -e^{4x}.
\]

\begin{Paradeigma}{5}
Ας θεωρήσουμε τις συναρτήσεις $y_1(x)=x, x>0$ και $y_2(x)=x\log x, x>0$. Να βρεθεί η ομογενής γραμμική διαφορική εξίσωση δεύτερης τάξης με συντελεστή του $y''$ τη μονάδα και με ένα βασικό σύνολο λύσεων το $\{y_1, y_2\}$.
\end{Paradeigma}
\lysh
Για όλα τα $x>0$ έχουμε
\[
    W(y_1, y_2)(x) = \begin{vmatrix} y_1(x) & y_2(x) \\ y_1'(x) & y_2'(x) \end{vmatrix} = \begin{vmatrix} x & x\log x \\ 1 & \log x + 1 \end{vmatrix} = x \ne 0
\]
και έτσι (θεώρημα 8) η ζητούμενη εξίσωση είναι η
\[
    \frac{W(y_1, y_2, y)}{W(y_1, y_2)} = 0.
\]
Αυτή γράφεται
\[
    \frac{1}{x} \begin{vmatrix} x & x\log x & y \\ 1 & \log x + 1 & y' \\ 0 & 1/x & y'' \end{vmatrix} = 0
\]
ή (μετά από πράξεις)
\[
    y'' - \frac{1}{x}y' + \frac{1}{x^2}y = 0.
\]

\begin{Paradeigma}{6}
Ν' αποδειχθεί ότι οι συναρτήσεις $y_1(x)=x, x>0$ και $y_2(x)=x^v$ αποτελούν ένα βασικό σύνολο λύσεων της ομογενούς γραμμικής διαφορικής εξίσωσης δεύτερης τάξης
\[
    (*) \quad x^2 y'' - x(x+v)y' + vy = 0, \quad x>0.
\]
Στη συνέχεια, να επιλυθεί η ομογενής γραμμική διαφορική εξίσωση τρίτης τάξης
\[
    (**) \quad x^3 y''' - 4x^2 y'' + 9xy' - 9y = 0, \quad x>0,
\]
αφού διαπιστωθεί ότι $y_1(x)=x, x>0$ είναι μια λύση της.
\end{Paradeigma}

Εύκολα διαπιστώνεται ότι οι συναρτήσεις $y_1, y_2$ είναι δύο λύσεις της $(*)$ και η $y_1$ είναι μια λύση της $(**)$. Οι λύσεις $y_1, y_2$ είναι γραμμικά ανεξάρτητες (θεώρημα 4), γιατί
\[
    W(y_1, y_2)(x) = x^v \ne 0 \text{ για } x>0.
\]
Τώρα, για $y=y_1 u = xu$ η διαφορική εξίσωση $(**)$ γίνεται
\[
    x^3(xu'''+3u'') - 4x^2(xu''+2u') + 9x(xu'+u) - 9xu = 0
\]
ή (μετά από πράξεις)
\[
    x^2 u''' - xu'' + u' = 0.
\]
Θέτοντας $u'=v$ καταλήγουμε στην εξίσωση $(*)$. Θεωρούμε τις συναρτήσεις $y_2$ και $y_3$, όπου για κάθε $x>0$ είναι
\[
    y_2(x) = y_1(x) \int_1^x \frac{v_1(t)}{y_1(t)} dt = \frac{1}{2}x(x^2-x^{-1})
\]
και
\[
    y_3(x) = y_1(x) \int_1^x \frac{v_2(t)}{y_1(t)} dt = x \int_1^x t\log t dt = \frac{1}{2}x^3 \log x - \frac{1}{4}(x^3-x).
\]
τότε (θεώρημα 9) $\{y_1, y_2, y_3\}$ είναι ένα βασικό σύνολο λύσεων της $(**)$. 'Ετσι (θεωρήματα 3 και 7) οι λύσεις της $(**)$ δίνονται απ' τον τύπο $y(x) = c_1 x + c_2 \frac{1}{2}(x^3-x) + c_3 \frac{1}{2}x^3 \log x - \frac{1}{4}c_3(x^3-x)$, $x>0$ για τις διάφορες τιμές των παραμέτρων $c_1, c_2$ και $c_3$. 'Αρα οι λύσεις της διαφορικής εξίσωσης $(**)$ είναι
\[
    y(x) = C_1 x + C_2 x^3 + C_3 x^3 \log x, \quad x>0,
\]
όπου $C_1, C_2$ και $C_3$ είναι αυθαίρετες σταθερές.

\begin{Paradeigma}{7}
Να επιλυθεί η ομογενής γραμμική διαφορική εξίσωση δεύτερης τάξης
\[
    (2x+1)y'' - 4(x+1)y' + 4y = 0, \quad x>-\frac{1}{2},
\]
αφού βρεθεί μια λύση $y_1$ αυτής της μορφής $y_1(x)=e^{ax}$, $x>-\frac{1}{2}$ ($a$ σταθερά).
\end{Paradeigma}
\lysh
Η συνάρτηση $y_1(x)=e^{ax}, x>-\frac{1}{2}$ ($a$ σταθερά) είναι μια λύση αν και μόνο αν για όλα τα $x>-\frac{1}{2}$
\[
    (2x+1)a^2 e^{ax} - 4(x+1)a e^{ax} + 4e^{ax} = 0
\]
ή
\[
    (2c^2-4c)x+c^2-4c+4=0,
\]
δηλαδή αν και μόνο αν
\[
    2c^2-4c=0 \quad \text{και} \quad c^2-4c+4=0,
\]
δηλαδή $c=2$. 'Ετσι, μια λύση είναι η $y_1(x)=e^{2x}, x>-\frac{1}{2}$. Θεωρούμε τη συνάρτηση
\begin{align*}
    y_2(x) &= y_1(x) \int_0^x \frac{1}{y_1^2(t)} \exp\left[-\int_0^t -\frac{4(s+1)}{2s+1} ds\right] dt = e^{2x} \int_0^x \frac{1}{e^{4t}} e^{2t} (2t+1) dt \\
    &= e^{2x} \int_0^x (2t+1) e^{-2t} dt = e^{2x} \left[-\frac{1}{2}e^{-2t}(2t+1)\right]_0^x - \int_0^x \left(-\frac{1}{2}e^{-2t}\right) 2 dt \\
    &= e^{2x} \left[-\frac{1}{2}e^{-2x}(2x+1) + \frac{1}{2}\right] - \int_0^x e^{-2t} dt \\
    &= -\frac{1}{2}(2x+1) + \frac{1}{2}e^{2x} - \left[-\frac{1}{2}e^{-2t}\right]_0^x \\
    &= -x - \frac{1}{2} + \frac{1}{2}e^{2x} + \frac{1}{2}(e^{-2x}-1) = -x-1+\frac{1}{2}(e^{2x}+e^{-2x}) = -x-1+\cosh(2x).
\end{align*}
Τότε (θεώρημα 10) $\{y_1, y_2\}$ είναι ένα βασικό σύνολο λύσεων της διαφορικής μας εξίσωσης. 'Ολες οι λύσεις δίνονται (θεωρήματα 3 και 7) απ' τον τύπο $y(x)=c_1 e^{2x} + c_2(-x-1+\cosh(2x))$, $x>-\frac{1}{2}$ ή ακόμα
\[
    y(x)=C_1 e^{2x} + C_2 (x+1), \quad x>-\frac{1}{2},
\]
όπου $C_1, C_2$ είναι αυθαίρετες σταθερές.

\subsection{Ασκήσεις}
\begin{askhseis}
\item Να επιλυθούν οι παρακάτω ομογενείς γραμμικές διαφορικές εξισώσεις πρώτης τάξης:
    \begin{rlist}
        \item $y' + \frac{x}{1+x^2}y=0$.
        \item $(1-x)y' + xy = 0$.
        \item $y' + \frac{1}{\sin x}y=0$.
        \item $(x+2)y'+xy=0$.
    \end{rlist}
\item Να επιλυθούν τα προβλήματα αρχικών τιμών:
    \begin{rlist}
        \item $(1+x)^2 y' + 2xy = 0, \quad y(0)=-1$.
        \item $(x\log x)y' - y = 0, \quad y(e)=1$.
        \item $(1+x^2)y' - xy=0, \quad y(0)=1$.
    \end{rlist}
\item Σε καθεμιά απ' τις παρακάτω περιπτώσεις να εξετασθεί αν οι συναρτήσεις που δίνονται είναι γραμμικά εξαρτημένες ή γραμμικά ανεξάρτητες:
    \begin{rlist}
        \item $f_1(x)=1, f_2(x)=x, f_3(x)=|x|$ για $x\in(-1,1)$.
        \item $f_1(x)=1, f_2(x)=\cos x, f_3(x)=\cos 2x$ για $x\in(-2\pi, 2\pi)$.
    \item $f_1(x)=1, f_2(x)=\log x, f_3(x)=\log x^2$ για $x\in(0, \infty)$.
    \item $f_1(x)=x^2-x+3, f_2(x)=2x^2+x, f_3(x)=2x-4$ για $x\in\mathbb{R}$.
    \item $f_1(x)=\sin x, f_2(x)=\cos x, f_3(x)=\sin(x+\frac{\pi}{3})$ για $x\in\mathbb{R}$.
    \item $f_1(x)=e^x, f_2(x)=xe^x, f_3(x)=x^2 e^x$ για $x\in\mathbb{R}$.
\end{rlist}
\item Να επιλυθεί η ομογενής γραμμική διαφορική εξίσωση
\[
    y''' - 6y'' + 5y' + 12y = 0,
\]
αφού βρεθούν οι λύσεις αυτής της μορφής $e^{cx}$, $x\in\mathbb{R}$ ($c$ σταθερά).
\item Να επιλυθεί η ομογενής γραμμική διαφορική εξίσωση
\[
    (2x+1)y'' - 4(x+1)y' + 4y = 0, \quad x>-\frac{1}{2},
\]
αν είναι γνωστό ότι δέχεται λύσεις των μορφών $y(x)=e^{ax}, x>-\frac{1}{2}$ και $y(x)=x^b, x>-\frac{1}{2}$ (όπου $a, b$ σταθερές). Ιδιαίτερα, να βρεθεί η λύση $y_0$ με
\[
    y_0(0)=0, \quad y_0'(0)=-1.
\]
\item Σε καθεμιά απ' τις περιπτώσεις (i) και (ii) να βρεθεί μια ομογενής γραμμική διαφορική εξίσωση με ένα βασικό σύνολο λύσεων τις συναρτήσεις που δίνονται:
    \begin{rlist}
        \item $y_1(x)=x, y_2(x)=x^2, x\in\mathbb{R}$.
        \item $y_1(x)=\sin x, y_2(x)=\cos x$ και $y_3(x)=e^x$ για $x\in\mathbb{R}$.
    \end{rlist}
\item Να επιλυθεί η ομογενής γραμμική διαφορική εξίσωση
\[
    x y''' - y'' - xy' + y = 0, \quad x>0,
\]
αφού διαπιστωθεί ότι $y_1(x)=x, x>0$ και $y_2(x)=e^x, x>0$ είναι δύο λύσεις της.
\item Να επιλυθεί η ομογενής γραμμική διαφορική εξίσωση
\[
    x^3 y''' - (x+3)x^2 y'' + 2x(x+3)y' - 2(x+3)y=0, \quad x>0
\]
με το δεδομένο ότι δέχεται δύο λύσεις της μορφής $x^a$, $x>0$ ($a$ ακέραιος).
\item Να επιλυθεί η ομογενής γραμμική διαφορική εξίσωση
\[
    (x^3-2x^2)y''' - (x^3+2x^2-6x)y'' + (3x^2-6)y=0, \quad x>2,
\]
αν είναι γνωστό ότι δέχεται μια λύση της μορφής $x^a$, $x>2$ ($a$ σταθερά).
\end{askhseis}


\section{Μη ομογενείς γραμμικές διαφορικές εξισώσεις}
Σ' αυτό το εδάφιο μελετούμε τη διαφορική εξίσωση (Ε) όταν αυτή είναι μια μη ομογενής γραμμική διαφορική εξίσωση ($b \ne 0$). Θεωρούμε πρώτα την περίπτωση του μη ομογενούς γραμμικών διαφορικών εξισώσεων πρώτης τάξης και δίνουμε (θεώρημα 11) τον τύπο απ' τον οποίο προκύπτουν όλες οι λύσεις. Στη συνέχεια, μελετούμε την (Ε) στη γενική περίπτωση αυθαίρετου $n$. Αποδεικνύουμε (θεώρημα 12) ότι, αν είναι γνωστή μια λύση (μερική λύση) της (Ε), τότε οι λύσεις της (Ε) είναι ακριβώς τα αθροίσματα της μερικής λύσης με τις λύσεις της αντίστοιχης ομογενούς γραμμικής διαφορικής εξίσωσης $(E_0)$. Αποδεικνύουμε (θεώρημα 13) ακόμα ότι για την εύρεση μιας μερικής λύσης της (Ε) όπου το δεύτερο μέρος είναι άθροισμα $m$ συναρτήσεων, είναι αρκετό να βρούμε μερική λύση για καθεμιά απ' τις $m$ διαφορικές εξισώσεις με δεύτερα μέλη τους προσθετέους του αθροίσματος. 'Επειτα, αναπτύσσουμε (θεωρήματα 14 και 15) τη μέθοδο μεταβολής των σταθερών για την εύρεση μιας μερικής λύσης της (Ε), αν είναι γνωστό ένα βασικό σύνολο λύσεων της $(E_0)$. Στην ειδική περίπτωση των μη ομογενών γραμμικών διαφορικών εξισώσεων δεύτερης τάξης η μέθοδος αυτή δίνει (θεωρήματα 16 και 17) την έκφραση μιας μερικής λύσης, αν είναι γνωστές δύο γραμμικά ανεξάρτητες λύσεις της αντίστοιχης ομογενούς γραμμικής διαφορικής εξίσωσης ή ακόμα αν είναι γνωστή μια μόνο λύση της αντίστοιχης ομογενούς εξίσωσης που δεν μηδενίζεται πουθενά στο Ι. Τέλος παραθέτουμε μερικά παραδείγματα και προτείνουμε ορισμένες ασκήσεις για λύση.

\subsection{Μη ομογενείς γραμμικές διαφορικές εξισώσεις πρώτης τάξης}
Για $n=1$ η διαφορική εξίσωση (Ε) γίνεται
\[
    (E)_1 \quad a_1 y' + a_0 y = b.
\]
Το θέμα της επίλυσης της εξίσωσης αυτής εξαντλείται με το παρακάτω θεώρημα.
\begin{Thewrhma}{11}
Ας είναι $x_0$ ένα σημείο του Ι και ας θέσουμε
\[
    A(x) = \int_{x_0}^x \frac{a_0(t)}{a_1(t)} dt, \quad x\in I.
\]

Τότε $y$ είναι μια λύση της μη ομογενούς γραμμικής διαφορικής εξίσωσης πρώτης τάξης (Ε)$_1$ αν και μόνο αν
\[
    y(x) = \exp[-A(x)] \left( y(x_0) + \int_{x_0}^x \frac{b(t)}{a_1(t)} \exp[A(t)] dt \right), \quad x\in I.
\]
\end{Thewrhma}
\paragraph{ΑΠΟΔΕΙΞΗ.} Θέτουμε $y = u \exp(-A)$. Τότε παίρνουμε
\[
    y' = u' \exp(-A) - u A' \exp(-A) = (u' - u \frac{a_0}{a_1}) \exp(-A)
\]
και έτσι η διαφορική εξίσωση (Ε)$_1$ μετασχηματίζεται στην εξίσωση
\[
    a_1 u' = b \exp A.
\]
Για τις λύσεις αυτής έχουμε
\[
    u(x) = u(x_0) + \int_{x_0}^x \frac{b(t)}{a_1(t)} \exp A(t) dt, \quad x\in I
\]
απ' όπου προκύπτει ο τύπος του θεωρήματός μας, δεδομένου ότι $y(x_0)=u(x_0)$.

Απ' το θεώρημα 11 προκύπτει ότι, αν Α και Β είναι συναρτήσεις με συνεχείς παραγώγους στο Ι και τέτοιες ώστε $A'=-a_0/a_1$ και $B'=(b/a_1)\exp A$, τότε όλες οι λύσεις της διαφορικής εξίσωσης (Ε)$_1$ προκύπτουν απ' τον τύπο $(c+B)\exp(-A)$ για τις διάφορες τιμές της σταθεράς c.

\subsubsection{Μερικές λύσεις. Το σύνολο των λύσεων}
Μια δεδομένη λύση της διαφορικής εξίσωσης (Ε) τη λέμε και μια \textbf{μερική λύση} αυτής. Θα δούμε ότι η επίλυση της μη ομογενούς γραμμικής διαφορικής εξίσωσης (Ε) ανάγεται στην εύρεση μιας μερικής λύσης αυτής, αν είναι γνωστές οι λύσεις της αντίστοιχης ομογενούς γραμμικής εξίσωσης.

\begin{Thewrhma}{12}
Ας είναι $y_\mu$ μια μερική λύση της μη ομογενούς γραμμικής διαφορικής εξίσωσης (Ε). Τότε $y$ είναι μια λύση της (Ε) αν και μόνο αν υπάρχει μια λύση $\bar{y}$ της αντίστοιχης ομογενούς γραμμικής διαφορικής εξίσωσης $(E_0)$ έτσι ώστε
\[
    y = \bar{y} + y_\mu.
\]
\end{Thewrhma}
\paragraph{ΑΠΟΔΕΙΞΗ.} Ας είναι $y$ μια λύση της (Ε). Τότε η συνάρτηση $\bar{y} = y-y_\mu$ είναι μια λύση της $(E_0)$, γιατί $L(\bar{y})=L(y-y_\mu)=L(y)-L(y_\mu)=b-b=0$. Αντίστροφα, αν $\bar{y}$ είναι μια λύση της $(E_0)$, τότε $L(\bar{y}+y_\mu)=L(\bar{y})+L(y_\mu)=0+b=b$ και άρα η συνάρτηση $y=\bar{y}+y_\mu$ είναι μια λύση της (Ε).

\begin{Thewrhma}{13}
Ας υποθέσουμε ότι $b=b_1+\dots+b_m$, όπου $b_k$ ($k=1,\dots,m$) είναι συνεχείς συναρτήσεις στο Ι. Αν, για κάθε $k\in\{1,\dots,m\}$, $y_\mu^k$ είναι μια μερική λύση της γραμμικής διαφορικής εξίσωσης
\[
    a_n y^{(n)} + a_{n-1} y^{(n-1)} + \dots + a_1 y' + a_0 y = b_k,
\]
τότε $y_\mu^1+\dots+y_\mu^m$ είναι μια μερική λύση της μη ομογενούς γραμμικής διαφορικής εξίσωσης (Ε).
\end{Thewrhma}

\paragraph{ΑΠΟΔΕΙΞΗ.} Είναι $L(y_\mu^k) = b_k$ για $k=1,\dots,m$ και έτσι παίρνουμε
\[
    L(y_\mu^1+\dots+y_\mu^m) = L(y_\mu^1)+\dots+L(y_\mu^m) = b_1+\dots+b_m = b.
\]
Το παραπάνω θεώρημα είναι χρήσιμο για την εύρεση μιας μερικής λύσης της (Ε) με τη μέθοδο των αγνώστων σταθερών, όταν η (Ε) είναι με σταθερούς συντελεστές. Τη μέθοδο αυτή θα την εξετάσουμε στο επόμενο εδάφιο.

\subsubsection{Η μέθοδος μεταβολής των σταθερών}
Η μέθοδος μεταβολής των σταθερών χρησιμοποιείται για την εύρεση μιας μερικής λύσης της (Ε) και προϋποθέτει ότι είναι γνωστό ένα βασικό σύνολο λύσεων της αντίστοιχης ομογενούς εξίσωσης $(E_0)$.

\begin{Thewrhma}{14}
Ας είναι $\{y_1,\dots,y_n\}$ ένα βασικό σύνολο λύσεων της ομογενούς γραμμικής διαφορικής εξίσωσης $(E_0)$ και ας είναι $v_k$ ($k=1,\dots,n$) η συνάρτηση τέτοια ώστε
\[
    \begin{cases}
        v_1' y_1 + v_2' y_2 + \dots + v_n' y_n = 0 \\
        v_1' y_1' + v_2' y_2' + \dots + v_n' y_n' = 0 \\
        \vdots \\
        v_1' y_1^{(n-2)} + v_2' y_2^{(n-2)} + \dots + v_n' y_n^{(n-2)} = 0 \\
        v_1' y_1^{(n-1)} + v_2' y_2^{(n-1)} + \dots + v_n' y_n^{(n-1)} = \frac{b}{a_n}.
    \end{cases}
\]


Τότε $y_\mu=v_1 y_1 + \dots + v_n y_n$ είναι μια μερική λύση της μη ομογενούς γραμμικής διαφορικής εξίσωσης (Ε).
\end{Thewrhma}
\paragraph{ΑΠΟΔΕΙΞΗ.} Ας σημειώσουμε πρώτα ότι, για κάθε $x\in I$, οι παραπάνω ισότητες στο x αποτελούν ένα γραμμικό σύστημα με αγνώστους $v_1'(x),\dots,v_n'(x)$ και ορίζουσα την $W(y_1,\dots,y_n)(x)$ που (ένεκα θεώρημα 4) διάφορη απ' το μηδέν. Για οποιοδήποτε x, το γραμμικό σύστημα έχει ακριβώς μια λύση, και έτσι οι συναρτήσεις $v_k'$ ($k=1,\dots,n$) ορίζονται μονοσήμαντα ενώ με ολοκλήρωση βρίσκονται συναρτήσεις $v_k$ ($k=1,\dots,n$) που πληρούν τις υπόψη ισότητες.
Θ' αποδείξουμε τώρα ότι η συνάρτηση $y_\mu=v_1 y_1 + \dots + v_n y_n$ είναι μια (μερική) λύση της διαφορικής εξίσωσης (Ε). Παραγωγίζοντας παίρνουμε
\begin{align*}
    y_\mu' &= (v_1 y_1' + v_2 y_2' + \dots + v_n y_n') + (v_1' y_1 + v_2' y_2 + \dots + v_n' y_n) \\
    &= v_1 y_1' + v_2 y_2' + \dots + v_n y_n', \\
    y_\mu'' &= (v_1 y_1'' + v_2 y_2'' + \dots + v_n y_n'') + (v_1' y_1' + v_2' y_2' + \dots + v_n' y_n') \\
    &= v_1 y_1'' + v_2 y_2'' + \dots + v_n y_n'', \\
    &\dots \\
    y_\mu^{(n-1)} &= [v_1 y_1^{(n-1)} + v_2 y_2^{(n-1)} + \dots + v_n y_n^{(n-1)}] + [v_1' y_1^{(n-2)} + v_2' y_2^{(n-2)} + \dots + v_n' y_n^{(n-2)}] \\
    &= v_1 y_1^{(n-1)} + \dots + v_n y_n^{(n-1)}, \\
    y_\mu^{(n)} &= [v_1 y_1^{(n)} + v_2 y_2^{(n)} + \dots + v_n y_n^{(n)}] + [v_1' y_1^{(n-1)} + v_2' y_2^{(n-1)} + \dots + v_n' y_n^{(n-1)}] \\
    &= \frac{b}{a_n} + v_1 y_1^{(n)} + v_2 y_2^{(n)} + \dots + v_n y_n^{(n)}.
\end{align*}
Έτσι έχουμε
\begin{align*}
    L(y_\mu) &= a_n y_\mu^{(n)} + a_{n-1} y_\mu^{(n-1)} + \dots + a_1 y_\mu' + a_0 y_\mu \\
    &= a_n \left[ \frac{b}{a_n} + v_1 y_1^{(n)} + v_2 y_2^{(n)} + \dots + v_n y_n^{(n)} \right] \\
    &\quad + a_{n-1} [v_1 y_1^{(n-1)} + v_2 y_2^{(n-1)} + \dots + v_n y_n^{(n-1)}] \\
    &\quad \vdots \\
    &\quad + a_1 (v_1 y_1' + v_2 y_2' + \dots + v_n y_n') \\
    &\quad + a_0 (v_1 y_1 + v_2 y_2 + \dots + v_n y_n).\\
    &= b + [a_n y_1^{(n)} + a_{n-1} y_1^{(n-1)} + \dots + a_1 y_1' + a_0 y_1] v_1 \\
    &\quad + [a_n y_2^{(n)} + a_{n-1} y_2^{(n-1)} + \dots + a_1 y_2' + a_0 y_2] v_2 \\
    &\quad \vdots \\
    &\quad + [a_n y_n^{(n)} + a_{n-1} y_n^{(n-1)} + \dots + a_1 y_n' + a_0 y_n] v_n \\
    &= b + L(y_1)v_1 + L(y_2)v_2 + \dots + L(y_n)v_n \\
    &= b,
\end{align*}
επειδή $L(y_k)=0$ ($k=1,\dots,n$) δεδομένου ότι οι $y_k$ ($k=1,\dots,n$) είναι λύσεις της $(E_0)$.

Με τον κανόνα του Cramer βρίσκουμε ότι η λύση του γραμμικού συστήματος που εμφανίζεται στη διατύπωση του παραπάνω θεωρήματος είναι
\[
    v_i' = \frac{W_i(y_1,\dots,y_n)}{W(y_1,\dots,y_n)} \frac{b}{a_n} = \frac{W_i'(y_1,\dots,y_n)}{W(y_1,\dots,y_n)} \frac{b}{a_n},
\]
όπου, για κάθε $k\in\{1,\dots,n\}$, $W_k(y_1,\dots,y_n)$ είναι η ορίζουσα που προκύπτει απ' την ορίζουσα $W(y_1,\dots,y_n)$ αν αντικατασταθεί η k-στήλη της με τη στήλη $(0,0,\dots,0,1)$. Έτσι, αν $x_0$ είναι ένα σημείο του διαστήματος Ι, τότε το θεώρημα 14 μπορεί να εφαρμοσθεί για
\[
    v_k(x) = \int_{x_0}^x \frac{W_i(y_1,\dots,y_n)(t)}{W(y_1,\dots,y_n)(t)} \frac{b(t)}{a_n(t)} dt, \quad x\in I \quad (k=1,\dots,n),
\]
όπου στην περίπτωση αυτή είναι
\[
    v_k(x_0)=0 \quad (k=1,\dots,n).
\]
Θα πάρουμε μ' αυτό τον τρόπο μια μερική λύση $y_\mu=v_1 y_1 + \dots + v_n y_n$ της (Ε) για την οποία θα έχουμε (όπως φαίνεται στην απόδειξη του θεωρήματος 14)
\[
    y_\mu^{(i)} = v_1 y_1^{(i)} + \dots + v_n y_n^{(i)} \quad (i=0,1,\dots,n-1).
\]
Έτσι, η μερική αυτή λύση θα πληροί τις αρχικές συνθήκες
\[
    y_\mu(x_0)=0, \quad y_\mu'(x_0)=0, \dots, y_\mu^{(n-1)}(x_0)=0.
\]
Φθάσαμε λοιπόν στο παρακάτω συμπέρασμα.

\begin{Thewrhma}{15}
Ας είναι $x_0$ ένα σημείο του Ι και $\{y_1,\dots,y_n\}$ ένα βασικό σύνολο λύσεων της ομογενούς γραμμικής διαφορικής εξίσωσης ($E_0$). Για κάθε $k\in\{1,\dots,n\}$, ας είναι $W_k(y_1,\dots,y_n)$ η ορίζουσα που προκύπτει απ' την $W(y_1,\dots,y_n)$ αν αντικατασταθεί η k-στήλη της με τη στήλη $(0,0,\dots,0,1)$. Τότε η συνάρτηση $y_\mu$ με
\[
    y_\mu(x) = \sum_{k=1}^n y_k(x) \int_{x_0}^x \frac{W_k(y_1,\dots,y_n)(t)}{W(y_1,\dots,y_n)(t)} \frac{b(t)}{a_n(t)} dt, \quad x\in I
\]
είναι μια μερική λύση της μη ομογενούς γραμμικής διαφορικής εξίσωσης (Ε). Επιπλέον, η λύση αυτή πληροί τις αρχικές συνθήκες
\[
    y_\mu(x_0)=0, \quad y_\mu'(x_0)=0, \dots, y_\mu^{(n-1)}(x_0)=0.
\]
\end{Thewrhma}
Ας θεωρήσουμε τώρα την ειδική περίπτωση της μη ομογενούς γραμμικής διαφορικής εξίσωσης δεύτερης τάξης
\begin{equation}
    a_2 y'' + a_1 y' + a_0 y = b.
\end{equation}
Η αντίστοιχη ομογενής εξίσωση είναι
\begin{equation}
    a_2 y'' + a_1 y' + a_0 y = 0.
\end{equation}
Για την εύρεση μιας μερικής λύσης της (Ε), έχουμε τα παρακάτω δύο θεωρήματα. Το θεώρημα 16 δεν είναι τίποτε άλλο παρά η έκφραση του θεωρήματος 15 για $n=2$, ενώ το θεώρημα 17 προκύπτει από ένα συνδυασμό των θεωρημάτων 10 και 16.

\begin{Thewrhma}{16}
Ας είναι $x_0$ ένα σημείο του διαστήματος Ι και $\{y_1,y_2\}$ ένα βασικό σύνολο λύσεων της ομογενούς γραμμικής διαφορικής εξίσωσης δεύτερης τάξης $(E_0')$. Τότε η συνάρτηση $y_\mu$ με
\[
    y_\mu(x) = \frac{\int_{x_0}^x y_1(t) y_2(x) \frac{b(t)}{a_2(t)}}{y_1(t)y_2'(t)-y_1'(t)y_2(t)} dt, \quad x\in I
\]
είναι μια μερική λύση της μη ομογενούς γραμμικής διαφορικής εξίσωσης δεύτερης τάξης (Ε'). Επιπλέον, η λύση αυτή πληροί τις αρχικές συνθήκες
\[
    y_\mu(x_0)=0, \quad y_\mu'(x_0)=0.
\]
\end{Thewrhma}

\paragraph{ΑΠΟΔΕΙΞΗ.} Εφαρμόζουμε το θεώρημα 15 για $n=2$. Τότε $W(y_1,y_2)=y_1 y_2' - y_1' y_2$, $W_1(y_1,y_2)=-y_2$ και $W_2(y_1,y_2)=y_1$. Έτσι, για $x\in I$ παίρνουμε
\begin{align*}
    y_\mu(x) &= y_1(x) \int_{x_0}^x \frac{W_1(y_1,y_2)(t)}{W(y_1,y_2)(t)} \frac{b(t)}{a_2(t)} dt + y_2(x) \int_{x_0}^x \frac{W_2(y_1,y_2)(t)}{W(y_1,y_2)(t)} \frac{b(t)}{a_2(t)} dt \\
    &= -y_1(x) \int_{x_0}^x \frac{-y_2(t)}{y_1(t)y_2'(t)-y_1'(t)y_2(t)} \frac{b(t)}{a_2(t)} dt \\
    &\quad + y_2(x) \int_{x_0}^x \frac{y_1(t)}{y_1(t)y_2'(t)-y_1'(t)y_2(t)} \frac{b(t)}{a_2(t)} dt \\
    &= \int_{x_0}^x \frac{y_1(t)y_2(x)-y_2(t)y_1(x)}{y_1(t)y_2'(t)-y_1'(t)y_2(t)} \frac{b(t)}{a_2(t)} dt.
\end{align*}

\begin{Thewrhma}{17}
Ας είναι $x_0$ ένα σημείο του Ι και
\[
    \Lambda(x) = \int_{x_0}^x \frac{a_1(t)}{a_2(t)} dt, \quad x\in I.
\]
Επιπλέον, ας είναι $y_1$ μια λύση της ομογενούς γραμμικής διαφορικής εξίσωσης δεύτερης τάξης $(E_0')$ με $y_1(x)\neq 0$ για όλα τα $x\in I$. Τότε η συνάρτηση $y_\mu$ με
\[
    y_\mu(x) = y_1(x) \int_{x_0}^x \left( \int_{x_0}^s \frac{\exp[-\Lambda(s)]}{y_1^2(s)} ds \right) \frac{b(t)}{a_2(t)} \exp[\Lambda(t)] dt, \quad x\in I
\]
είναι μια μερική λύση της μη ομογενούς γραμμικής διαφορικής εξίσωσης δεύτερης τάξης (Ε'), με
\[
    y_\mu(x_0)=0, \quad y_\mu'(x_0)=0.
\]
\end{Thewrhma}

\paragraph{ΑΠΟΔΕΙΞΗ.} Θέτουμε
\[
    y_2(x) = y_1(x) \int_{x_0}^x \frac{\exp[-\Lambda(t)]}{y_1^2(t)} dt, \quad x\in I.
\]
Τότε (θεώρημα 10) $\{y_1,y_2\}$ είναι ένα βασικό σύνολο λύσεων της ομογενούς εξίσωσης $(E_0')$. Αρκεί λοιπόν να εφαρμοσθεί το θεώρημα 16 στην περίπτωση αυτή. Έχουμε για $x\in I$
\[
    y_1(x)y_2'(x)-y_1'(x)y_2(x) = y_1(x) \left( y_1'(x) \int_{x_0}^x \frac{\exp[-\Lambda(t)]}{y_1^2(t)} dt + \frac{\exp[-\Lambda(x)]}{y_1(x)} \right) - y_1'(x)y_2(x) = \exp[-\Lambda(x)].
\]
\begin{align*}
    &\quad - y_1'(x) y_1(x) \int_{x_0}^x \frac{\exp[-\Lambda(t)]}{y_1^2(t)} dt \\
    &= \exp[-\Lambda(x)]
\end{align*}
και έτσι για όλα τα $x\in I$ παίρνουμε
\begin{align*}
    &\int_{x_0}^x \frac{y_1(t)y_2(x)-y_2(t)y_1(x)}{y_1(t)y_2'(t)-y_1'(t)y_2(t)} \frac{b(t)}{a_2(t)} dt = \\
    &= \int_{x_0}^x \frac{y_1(t) \left( y_1(x) \int_{x_0}^x \frac{\exp[-\Lambda(s)]}{y_1^2(s)} ds \right) - y_1(x) y_1(t) \int_{x_0}^t \frac{\exp[-\Lambda(s)]}{y_1^2(s)} ds}{\exp[-\Lambda(t)]} \frac{b(t)}{a_2(t)} dt \\
    &= y_1(x) \int_{x_0}^x y_1(t) \left( \int_{x_0}^x \frac{\exp[-\Lambda(s)]}{y_1^2(s)} ds - \int_{x_0}^t \frac{\exp[-\Lambda(s)]}{y_1^2(s)} ds \right) \frac{b(t)}{a_2(t)} \exp[\Lambda(t)] dt \\
    &= y_1(x) \int_{x_0}^x y_1(t) \left( \int_t^x \frac{\exp[-\Lambda(s)]}{y_1^2(s)} ds \right) \frac{b(t)}{a_2(t)} \exp[\Lambda(t)] dt.
\end{align*}

\subsubsection{Παραδείγματα}

\begin{Paradeigma}{}
Να επιλυθεί η μη ομογενής γραμμική διαφορική εξίσωση πρώτης τάξης
\[
    y' + (\tan x) y = \sin x, \quad x\in (0,\pi/2)
\]
και, ειδικά, να βρεθεί η λύση $y_1$ αυτής που πληροί την αρχική συνθήκη $y_1(\pi/4)=1$.
\end{Paradeigma}
Σύμφωνα με το θεώρημα 11, $y$ θα είναι μια λύση της διαφορικής μας εξίσωσης αν και μόνο αν για κάθε $x\in(0,\pi/2)$
\[
    y(x) = e^{-\int_{\pi/4}^x \tan t \, dt} \left(y_1\left(\frac{\pi}{4}\right) + \int_{\pi/4}^x (\sin t) e^{\int_{\pi/4}^t \tan s \, ds} dt\right)
\]
\[
    = \sqrt{2}\cos x \left[\frac{1}{\sqrt{2}/\pi/4} + \int_{\pi/4}^x \frac{\sin t}{\cos t} dt\right] = \cos x[\sqrt{2} y(\pi/4) - \log(\sqrt{2} - \log\cos x)].
\]
Έτσι, όλες οι λύσεις δίνονται απ'τον τύπο $y(x)=\cos x (c-\log \cos x)$, $x\in(0, \pi/2)$. Ειδικά, έχουμε $y_1=\cos x (\sqrt{2}-\log\sqrt{2}/\cos x)$, $x\in(0, \pi/2)$.

\begin{Paradeigma}{Να επιλυθεί η μη ομογενής γραμμική διαφορική εξίσωση
\[
    x^3 y''' - 4x^2 y'' + 8xy' - 8y = 2x^4 \log x, \quad x>0,
\]
αφού δοθούν τρείς γραμμικά ανεξάρτητες λύσεις της αντίστοιχης ομογενούς $x^3y'''-4x^2y''+8xy'-8y=0$ ($nu$ ακέραιος).}
\end{Paradeigma}
Η αντίστοιχη ομογενής γραμμική διαφορική εξίσωση έχει (Παράδειγμα 3, εδάφιο 1) το βασικό σύνολο λύσεων $\{y_1,y_2,y_3\}$, όπου $y_1(x)=x, y_2(x)=x^2$ και $y_3(x)=x^4$ για $x>0$, και όλες οι λύσεις της δίνονται απ'τον τύπο $\tilde{y}(x)=c_1x+c_2x^2+c_3x^4$, όπου $c_1,c_2,c_3$ είναι αυθαίρετες σταθερές. Θα βρούμε τώρα μια μερική λύση της διαφορικής μας εξίσωσης. Για κάθε $x>0$, θεωρούμε το σύστημα
\[
    \begin{cases}
        v_1'(x)y_1(x)+v_2'(x)y_2(x)+v_3'(x)y_3(x)=0 \\
        v_1'(x)y_1'(x)+v_2'(x)y_2'(x)+v_3'(x)y_3'(x)=0 \\
        v_1'(x)y_1''(x)+v_2'(x)y_2''(x)+v_3'(x)y_3''(x)=2x^4\log x/x^3
    \end{cases}
\]
το οποίο γράφεται
\[
    \begin{cases}
        v_1'x+v_2'x^2+v_3'x^4=0 \\
        v_1'+2v_2'x+4v_3'x^3=0 \\
        2v_2'+6x^2v_3'=x\log x.
    \end{cases}
\]
Εύκολα βρίσκουμε
\[
    v_1'(x) = \frac{2}{3}x^2\log x, \quad v_2'(x)=-x\log x \quad \text{και} \quad v_3'(x)=\frac{1}{3}\log x \quad \text{για } x>0,
\]
οπότε, με ολοκλήρωση, βλέπουμε ότι μπορούμε να πάρουμε
\[
    v_1(x)=\frac{2}{9}x^3(\log x - \frac{1}{3}), v_2(x)=-\frac{1}{2}x^2(\log x - \frac{1}{2}) \quad \text{και} \quad v_3(x)=\frac{1}{9}\log^2x
\]
για όλα τα $x>0$. Τότε (θεώρημα 14) η συνάρτηση $y_\mu$ με
\[
    y_\mu(x)=v_1(x)y_1(x)+v_2(x)y_2(x)+v_3(x)y_3(x) = x^4\left(\frac{1}{6}\log^2x-\frac{5}{18}\log x+\frac{19}{108}\right), x>0
\]
είναι μια μερική λύση της διαφορικής μας εξίσωσης. Όλες οι λύσεις θα δίνονται (θεώρημα 12) απ'τον τύπο
\[
    y(x)=c_1x+c_2x^2+c_3x^4+x^4\left(\frac{1}{6}\log^2x-\frac{5}{18}\log x+\frac{19}{108}\right), \quad x>0.
\]
\begin{Paradeigma}{}
Να βρεθεί η μερική λύση $y_\mu$ της μη ομογενούς γραμμικής διαφορικής εξίσωσης
\[
    y''' - y'' + y' - y = 1
\]
που πληροί τις αρχικές συνθήκες $y_\mu(0)=y_\mu'(0)=y_\mu''(0)=0$, αν είναι γνωστό ότι οι συναρτήσεις
\[
    y_1(x)=e^x, \quad y_2(x)=\cos x \quad \text{και} \quad y_3(x)=\sin x \quad \text{για } x\in\mathbb{R}
\]
είναι γραμμικά ανεξάρτητες λύσεις της αντίστοιχης ομογενούς εξίσωσης.
\end{Paradeigma}
Η ζητούμενη μερική λύση δίνεται (θεώρημα 15) απ'τον τύπο
\[
    y_\mu(x) = \sum_{k=1}^3 y_k(x) \int_0^x \frac{W_k(y_1,y_2,y_3)(t)}{W(y_1,y_2,y_3)(t)} dt, \quad x\in\mathbb{R},
\]
όπου, για κάθε $k\in\{1,2,3\}$, $W_k(y_1,y_2,y_3)$ είναι η ορίζουσα που προκύπτει απ'την $W(y_1,y_2,y_3)$ αν αντικατασταθεί η k-στήλη με τη στήλη $(0,0,1)$. Για κάθε $x\in\mathbb{R}$ έχουμε
\[
    W(y_1,y_2,y_3)(x) = \begin{vmatrix} y_1(x) & y_2(x) & y_3(x) \\ y_1'(x) & y_2'(x) & y_3'(x) \\ y_1''(x) & y_2''(x) & y_3''(x) \end{vmatrix}
    = \begin{vmatrix} e^x & \cos x & \sin x \\ e^x & -\sin x & \cos x \\ e^x & -\cos x & -\sin x \end{vmatrix} = 2e^x,
\]
\[
    W_1(y_1,y_2,y_3)(x) = \begin{vmatrix} 0 & y_2(x) & y_3(x) \\ 0 & y_2'(x) & y_3'(x) \\ 1 & y_2''(x) & y_3''(x) \end{vmatrix}
    = \begin{vmatrix} 0 & \cos x & \sin x \\ 0 & -\sin x & \cos x \\ 1 & -\cos x & -\sin x \end{vmatrix} = 1,
\]
\[
    W_2(y_1,y_2,y_3)(x) = \begin{vmatrix} y_1(x) & 0 & y_3(x) \\ y_1'(x) & 0 & y_3'(x) \\ y_1''(x) & 1 & y_3''(x) \end{vmatrix}
    = \begin{vmatrix} e^x & 0 & \sin x \\ e^x & 0 & \cos x \\ e^x & 1 & -\sin x \end{vmatrix} = -e^x(\sin x - \cos x),
\]
\[
    W_3(y_1,y_2,y_3)(x) = \begin{vmatrix} y_1(x) & y_2(x) & 0 \\ y_1'(x) & y_2'(x) & 0 \\ y_1''(x) & y_2''(x) & 1 \end{vmatrix}
    = \begin{vmatrix} e^x & \cos x & 0 \\ e^x & -\sin x & 0 \\ e^x & -\cos x & 1 \end{vmatrix} = -e^x(\sin x + \cos x).
\]
Έτσι, για κάθε $x\in\mathbb{R}$
\begin{align*}
    y_\mu(x) &= \int_0^x \frac{1}{2e^t} dt + \cos x \int_0^x \frac{-\sin t - \cos t}{2} dt + \sin x \int_0^x \frac{\sin t - \cos t}{2} dt \\
    &= \frac{1}{2}(e^x-1) + \frac{1}{2} \cos x (-\cos x - \sin x + 1) - \frac{1}{2} \sin x(-\cos x + \sin x + 1).
\end{align*}

\begin{Paradeigma}{}
Να βρεθεί η μερική λύση $y_\mu$ της μη ομογενούς γραμμικής διαφορικής εξίσωσης δεύτερης τάξης
\[
    x^2y'' - xy' + y = x \log x, \quad x>0
\]
που πληροί τις αρχικές συνθήκες $y_\mu(1)=y_\mu'(1)=0$, με το δεδομένο ότι $y_1(x)=x, x>0$ και $y_2(x)=x\log x, x>0$ είναι δύο γραμμικά ανεξάρτητες λύσεις της αντίστοιχης ομογενούς εξίσωσης.
\end{Paradeigma}
Η ζητούμενη λύση είναι (θεώρημα 16)
\[
    y_\mu(x) = \int_1^x \frac{y_1(t)y_2(x)-y_2(t)y_1(x)}{y_1(t)y_2'(t)-y_1'(t)y_2(t)} \frac{t\log t}{t^2} dt = \int_1^x (x\log x - x\log t) \frac{\log t}{t} dt
\]
\[
    = x\log x \int_1^x \frac{\log t}{t} dt - x \int_1^x \frac{\log^2 t}{t} dt = \frac{1}{6} x \log^3 x, \quad x>0.
\]

\begin{Paradeigma}{}
Να βρεθεί η μερική λύση $y_\mu$ της μη ομογενούς γραμμικής διαφορικής εξίσωσης δεύτερης τάξης
\[
    y'' - 2y' + y = e^x, \quad x\in\mathbb{R},
\]
που πληροί τις αρχικές συνθήκες $y_\mu(0)=y_\mu'(0)=0$, αν είναι γνωστό ότι η αντίστοιχη ομογενής εξίσωση έχει τη λύση $y_1(x)=e^x, x\in\mathbb{R}$.
\end{Paradeigma}
Η ζητούμενη λύση είναι (θεώρημα 17)
\[
    y_\mu(x) = y_1(x) \int_0^x \left(\frac{1}{y_1(t)} \int_0^t \frac{\exp\left[-\int_s^t (-2) dr\right]}{y_1^2(s)} ds\right) e^t dt
\]
\[
    = e^x \int_0^x (x-t) dt = \frac{1}{2}x^2e^x, \quad x\in\mathbb{R}.
\]
\subsection{Ασκήσεις}

\begin{enumerate}
    \item Να επιλυθούν οι μη ομογενείς γραμμικές διαφορικές εξισώσεις πρώτης τάξης:
    \begin{enumerate}
        \item $y' + \frac{x}{1+x^2} y = \frac{x}{1+x^2}$.
        \item $y' - \frac{3}{x-1}y = (x-1)^4$.
        \item $y' - y = b$ με $b(x) = \begin{cases} 0, & x>0 \\ x, & x\geq 0 \end{cases}$.
    \end{enumerate}
    \item Να επιλυθούν τα προβλήματα αρχικών τιμών:
    \begin{enumerate}
        \item $y'-xy=(1-x^2)e^{x^2/2}$, $y(0)=0$.
        \item $(1+x)^2y'+2xy=-2x$, $y(0)=-1$.
        \item $(1-x)y'+xy=x(x-1)^2$, $y(5)=24$.
    \end{enumerate}
    \item Να επιλυθεί η μη ομογενής γραμμική διαφορική εξίσωση
    \[
        xy'' - (x+1)y' + (x+2)y = x^3e^{2x}, \quad x>0
    \]
    με το δεδομένο ότι η αντίστοιχη ομογενής εξίσωση έχει μια λύση της μορφής $e^{ax}$, $x>0$ (α σταθερά).
    
    \item Να επιλυθεί το πρόβλημα αρχικών τιμών
    \[
        (x^2+1)y'' - 2xy' + 2y = (x^2+1)^2, \quad y(0)=0, y'(0)=1,
    \]
    αν είναι γνωστό ότι $y_1(x)=x, x\in\mathbb{R}$ είναι μια λύση της αντίστοιχης ομογενούς διαφορικής εξίσωσης.
    
    \item Να επιλυθεί η μη ομογενής γραμμική διαφορική εξίσωση
    \[
        y''' - 3y'' + 2y' - e^{2x}, \quad x\in\mathbb{R},
    \]
    αφού βρεθούν τρείς γραμμικά ανεξάρτητες λύσεις της μορφής $e^{cx}$, $x\in\mathbb{R}$ (c σταθερά) για την αντίστοιχη ομογενή εξίσωση.
    
    \item Να επιλυθεί το πρόβλημα αρχικών τιμών
    \[
        y'' - 3y' + 2y = e^{-2x}/(1+e^x), \quad y(0)=y'(0)=0
    \]
    με το δεδομένο ότι η αντίστοιχη ομογενής εξίσωση έχει τις λύσεις $y_1(x)=e^x, x\in\mathbb{R}$ και $y_2(x)=e^{2x}, x\in\mathbb{R}$.
    
    \item Να επιλυθεί η μη ομογενής γραμμική διαφορική εξίσωση




\[
    (2x+1)(x+1)y''+2xy'-2y=(2x+1)^2, \quad x>-\frac{1}{2},
\]
αφού βρεθεί μια λύση $y_1$ της αντίστοιχης ομογενούς εξίσωσης της μορφής $y_1(x)=(yx+\delta)/(x+1)$, $x>-1/2$ ($\gamma,\delta$ σταθερές).
\end{enumerate}
\section{Γραμμικές διαφορικές εξισώσεις με σταθερούς συντελεστές}

Στο Εδάφιο αυτό μελετούμε την ενδιαφέρουσα ειδική περίπτωση των γραμμικών διαφορικών εξισώσεων με σταθερούς συντελεστές. Έτσι, σ' ολόκληρο το Εδάφιο θα υποθέσουμε ότι οι συντελεστές $a_i$ ($i=0,1,\dots,n-1$) είναι σταθερές. Δίνουμε (θεώρημα 18) πρώτα το τύπο για την εύρεση των λύσεων των γραμμικών διαφορικών εξισώσεων πρώτης τάξης (ομογενών και μη ομογενών) με σταθερούς συντελεστές. Θεωρούμε έπειτα τη γενική περίπτωση για οποιοδήποτε n. Για την ομογενή γραμμική εξίσωση (Ε$_0$) βρίσκουμε (θεωρήματα 19 και 20) ένα βασικό σύνολο λύσεων με τη βοήθεια των ριζών του χαρακτηριστικού πολυωνύμου $p(\lambda)=a_n\lambda^n + a_{n-1}\lambda^{n-1} + \dots + a_1\lambda + a_0$. Επίσης, αναπτύσσουμε τη μέθοδο των αγνώστων σταθερών για την εύρεση μιας μερικής λύσης της μη ομογενούς γραμμικής διαφορικής εξίσωσης (Ε) όταν το δεύτερο μέλος της έχει μια κατάλληλη μορφή. Μελετούμε ακόμη τη διαφορική εξίσωση Euler και βλέπουμε ότι αυτή ανάγεται μ' ένα κατάλληλο μετασχηματισμό, σε μια ομογενή γραμμική διαφορική εξίσωση με σταθερούς συντελεστές. Παραθέτουμε τέλος ορισμένα παραδείγματα και δίνουμε ασκήσεις για λύση.

\subsection{Γραμμικές διαφορικές εξισώσεις πρώτης τάξης με σταθερούς συντελεστές}

Εφαρμόζοντας τα θεωρήματα 2 και 11, παίρνουμε αμέσως το επόμενο συμπέρασμα.

\begin{Thewrhma}{18.}
(i) Ας είναι $x_0\in\mathbb{R}$. Τότε η είναι μια λύση της
\[
    (E_0')_1 \quad a_1y' + a_0y = 0
\]
αν και μόνο αν
\[
    y(x) = y(x_0) \exp[-\frac{a_0}{a_1}(x-x_0)], \quad x\in\mathbb{R}.
\]
(ii) Ας είναι $x_0\in I$. Τότε η είναι μια λύση της (E)$_1$
\[
    a_1y' + a_0y = b
\]
αν και μόνο αν
\[
    y(x) = \exp\left(-\frac{a_0}{a_1}x\right)\left[y(x_0)\exp\left(\frac{a_0}{a_1}x_0\right) + \int_{x_0}^x \frac{b(t)}{a_1} \exp\left(\frac{a_0}{a_1}t\right) dt\right], \quad x\in I.
\]
\end{Thewrhma}
Απ'το θεώρημα 18 φαίνεται ότι οι λύσεις της (Ε$_0$)$_1$ είναι ακριβώς οι συναρτήσεις της μορφής $c \exp(-\frac{a_0}{a_1}x), x\in\mathbb{R}$ για τις διάφορες τιμές της σταθεράς c, ενώ οι λύσεις της (Ε), είναι ακριβώς οι συναρτήσεις $[c+B(x)]\exp(-\frac{a_0}{a_1}x), x\in I$ για τις διάφορες τιμές του c, όπου Β είναι μια συνάρτηση με $B'(x) = b(x)/a_1 \exp(\frac{a_0}{a_1}x), x\in I$. Επίσης, θα πρέπει να τονίσουμε ότι ως διάστημα ορισμού της εξίσωσης (Ε$_0$)$_1$ θεωρείται ολόκληρη η πραγματική ευθεία $\mathbb{R}$.

\subsection{Το χαρακτηριστικό πολυώνυμο. Ένα βασικό σύνολο λύσεων}

Θα μελετήσουμε εδώ την ομογενή γραμμική διαφορική εξίσωση με σταθερούς συντελεστές (Ε$_0$). Ως διάστημα ορισμού αυτής θα θεωρείται το $\mathbb{R}$ και έτσι όλες οι λύσεις της θα είναι ορισμένες σ'ολόκληρη την πραγματική ευθεία.
Το πολυώνυμο
\[
    p(\lambda) = a_n\lambda^n + a_{n-1}\lambda^{n-1} + \dots + a_1\lambda + a_0
\]
καλείται χαρακτηριστικό πολυώνυμο της (Ε$_0$) και η εξίσωση $p(\lambda)=0$ λέγεται χαρακτηριστική εξίσωση αυτής. Το πολυώνυμο αυτό παίζει σπουδαίο ρόλο στην επίλυση της ομογενούς γραμμικής διαφορικής εξίσωσης (Ε$_0$) (με σταθερούς συντελεστές), γιατί από τις ρίζες του μπορεί να κατασκευασθεί ένα βασικό σύνολο λύσεων. Βέβαια δεν είναι πάντοτε δυνατό να βρεθούν οι ρίζες του πολυώνυμου $p(\lambda)$, ιδιαίτερα στην περίπτωση n$\geq$5. Ας σημειώσουμε ότι το $p(\lambda)$ έχει ακριβώς n ρίζες, όπου κάθε ρίζα μετράται τόσες φορές όσες δείχνει η πολλαπλότητά της. Το παρακάτω θεώρημα δίνει ένα βασικό σύνολο λύσεων της (Ε$_0$) με τη βοήθεια των ριζών του χαρακτηριστικού πολυωνύμου της.

\begin{Thewrhma}{19.}
Ας είναι $\lambda_1, \dots, \lambda_s$ οι διακεκριμένες ρίζες του χαρακτηριστικού
πολυωνύμου της (Ε$_0$) με πολλαπλότητες $m_1, \dots, m_s$ αντίστοιχα (οπότε $m_1+\dots+m_s = n$). Τότε οι n συναρτήσεις
\[
    Y_{ij}(x) = x^j e^{\lambda_i x}, \quad x\in\mathbb{R} \quad (j=0,1,\dots,m_i-1; \ i=1,\dots,s)
\]
αποτελούν ένα βασικό σύνολο λύσεων της ομογενούς γραμμικής διαφορικής εξίσωσης (Ε$_0$).
\end{Thewrhma}

\paragraph{ΑΠΟΔΕΙΞΗ.} Ας είναι $\lambda_0$ μια ρίζα του χαρακτηριστικού πολυωνύμου της (Ε$_0$) με πολλαπλότητα m. Θ'αποδείξουμε ότι οι συναρτήσεις $x^j e^{\lambda_0 x}$, $x\in\mathbb{R}$ ($j=0,1,\dots,m-1$) είναι λύσεις της διαφορικής εξίσωσης. Τότε θα έχει αποδειχθεί το πρώτο μέρος του θεωρήματος, δηλαδή το ότι οι συναρτήσεις $Y_{ij}$ ($j=0,1,\dots,m_i-1; \ i=1,\dots,s$) είναι λύσεις της (Ε$_0$). Έχουμε, με τη βοήθεια του τύπου του Leibnitz,
\begin{align*}
    L(x^j e^{\lambda x}) &= L\left(x^j \frac{\partial^j}{\partial \lambda^j} e^{\lambda x}\right) = \sum_{k=0}^n a_k \frac{\partial^k}{\partial x^k}\left(\frac{\partial^j}{\partial \lambda^j} e^{\lambda x}\right) = \sum_{k=0}^n a_k \frac{\partial^j}{\partial \lambda^j}\left(\frac{\partial^k}{\partial x^k} e^{\lambda x}\right) \\
    &= \frac{\partial^j}{\partial \lambda^j} \left(\sum_{k=0}^n a_k \lambda^k e^{\lambda x}\right) = \frac{\partial^j}{\partial \lambda^j} (p(\lambda)e^{\lambda x}) = \sum_{r=0}^j \binom{j}{r} \frac{\partial^r}{\partial \lambda^r} p(\lambda) \frac{\partial^{j-r}}{\partial \lambda^{j-r}} (e^{\lambda x}) \\
    &= e^{\lambda x} \sum_{r=0}^j \binom{j}{r} p^{(r)}(\lambda) x^{j-r}
\end{align*}
και έτσι, αν $j\in\{0,1,\dots,m-1\}$, για $\lambda=\lambda_0$ και για κάθε $x\in\mathbb{R}$ παίρνουμε
\[
    L(x^j e^{\lambda_0 x}) = e^{\lambda_0 x} \sum_{r=0}^j \binom{j}{r} x^{j-r} p^{(r)}(\lambda_0)=0,
\]
επειδή
\[
    p(\lambda_0)=p'(\lambda_0)=\dots=p^{(m-1)}(\lambda_0)=0.
\]
Άρα, οι συναρτήσεις $x^j e^{\lambda_0 x}$, ($j=0,1,\dots,m-1$) είναι λύσεις της (Ε$_0$).

Θ'αποδείξουμε, τώρα, ότι οι λύσεις $Y_{ij}$ ($j=0,1,\dots,m_i-1; \ i=1,\dots,s$) είναι γραμμικά ανεξάρτητες. Υποθέτουμε ότι αυτό δεν συμβαίνει, οπότε υπάρχουν σταθερές $c_{ij}$ ($j=0,1,\dots,m_i-1; i=1,\dots,s$), όχι όλες μηδέν, έτσι ώστε για κάθε $x\in\mathbb{R}$
\[
    \sum_{i=1}^s \sum_{j=0}^{m_i-1} c_{ij} x^j e^{\lambda_i x} = 0.
\]
θέτουμε
\[
    P_i(x) = \sum_{j=0}^{m_i-1} c_{ij}x^j, \quad x\in\mathbb{R} \quad (i=1,\dots,s),
\]
οπότε για όλα τα $x\in\mathbb{R}$ θα είναι
\[
    P_1(x)e^{\lambda_1 x} + \dots + P_s(x)e^{\lambda_s x} = 0.
\]
Ένα τουλάχιστο απ' τα πολυώνυμα $P_i$ ($i=1,\dots,s$) δεν είναι το μηδενικό πολυώνυμο. Έτσι, μπορούμε να θεωρήσουμε ένα δείκτη $\mu\in\{1,\dots,s\}$ τέτοιον ώστε $P_\mu \neq 0$ και (για $\mu>1$) $P_{\mu+1}=\dots=P_s=0$. Αν $\mu=1$, τότε
\[
    P_1(x)e^{\lambda_1 x} = 0 \quad \text{για κάθε } x\in\mathbb{R},
\]
το οποίο οδηγεί στο άτοπο $P_1=0$. Υποθέτουμε λοιπόν ότι $\mu>1$. Τότε για όλα τα $x\in\mathbb{R}$ είναι
\[
    P_1(x)e^{\lambda_1 x} + \dots + P_\mu(x)e^{\lambda_\mu x} = 0
\]
και επομένως
\[
    P_1(x)e^{(\lambda_1-\lambda_\mu)x} + \dots + P_\mu(x) = 0.
\]
Παραγωγίζοντας τόσες φορές την παραπάνω ισότητα ώστε να μηδενισθεί ο πρώτος όρος του αθροίσματος, παίρνουμε μια σχέση της μορφής
\[
    Q_2(x)e^{(\lambda_2-\lambda_1)x} + \dots + Q_\mu(x)e^{(\lambda_\mu-\lambda_1)x} = 0
\]
ή
\[
    Q_2(x)e^{\lambda_2 x} + \dots + Q_\mu(x)e^{\lambda_\mu x} = 0
\]
για $x\in\mathbb{R}$, όπου $Q_2, \dots, Q_\mu$ είναι πολυώνυμα με τους ίδιους βαθμούς με τα $P_2, \dots, P_\mu$ αντίστοιχα και το $Q_\mu$ δεν είναι το μηδενικό πολυώνυμο. Αν $\mu=2$, τότε οδηγούμαστε στο άτοπο $Q_2=0$. Αν $\mu>2$, τότε, επαναλαμβάνοντας την ίδια διαδικασία (όσες φορές χρειάζεται), φθάνουμε τελικά σε μια σχέση
\[
    R_\mu(x)e^{\lambda_\mu x} = 0
\]
για όλα τα $x\in\mathbb{R}$, όπου $R_\mu$ είναι ένα πολυώνυμο του ίδιου βαθμού με
το $P_i$ και όχι το μηδενικό πολυώνυμο. Αυτό όμως είναι ένα άτοπο, που οφείλεται στην υπόθεση ότι οι συναρτήσεις $Y_{ij}$ ($j=0,1,\dots,m_i-1; \ i=1,\dots,s$) είναι γραμμικά εξαρτημένες.

Απ'το θεώρημα 19 προκύπτει ότι, αν $\lambda_1, \dots, \lambda_n$ είναι διακεκριμένες ρίζες του χαρακτηριστικού πολυωνύμου της ομογενούς γραμμικής διαφορικής εξίσωσης (Ε$_0$), τότε ένα βασικό σύνολο λύσεων αυτής αποτελείται από τις συναρτήσεις
\[
    e^{\lambda_i x}, \quad x\in\mathbb{R} \quad (i=1,\dots,n).
\]
Θ'ασχοληθούμε τώρα με την περίπτωση όπου οι σταθερές $a_i$ ($i=0,1,\dots,n-1,n$) είναι πραγματικές. Τότε το χαρακτηριστικό πολυώνυμο έχει πραγματικούς συντελεστές και έτσι, αν $\sigma+i\tau$ ($\sigma, \tau\in\mathbb{R}$ και $\tau\neq 0$) είναι μια ρίζα του με πολλαπλότητα m, τότε ο συζυγής $\sigma-i\tau$ είναι επίσης μια ρίζα αυτού με την ίδια πολλαπλότητα m. Το γεγονός αυτό μας δίνει τη δυνατότητα να βρούμε ένα βασικό σύνολο λύσεων που ν'αποτελείται από πραγματικές συναρτήσεις (πραγματικές λύσεις). Έχουμε το παρακάτω θεώρημα.

\begin{Thewrhma}{20.}
Ας υποθέσουμε ότι οι συντελεστές $a_i$ ($i=0,1,\dots,n-1,n$) είναι πραγματικοί. Τότε για την ομογενή γραμμική διαφορική εξίσωση (Ε$_0$) ισχύουν τα παρακάτω:
\begin{enumerate}
    \item[(i)] Αν y είναι μια λύση, τότε Re y και Im y είναι επίσης λύσεις.
    \item[(ii)] Κάθε λύση με πραγματικές αρχικές τιμές είναι πραγματική.
    \item[(iii)] Αν $\{y_1, \dots, y_n\}$ είναι ένα βασικό σύνολο πραγματικών λύσεων, τότε η μια πραγματική λύση αν και μόνο αν υπάρχουν πραγματικές σταθερές $c_k$ ($k=1,\dots,n$) έτσι ώστε $y=c_1y_1+\dots+c_ny_n$.
    \item[(iv)] Ας είναι $\lambda_1=\sigma_1+i\tau_1, \dots, \lambda_r=\sigma_r+i\tau_r$ με $\sigma_j, \tau_j\in\mathbb{R}$ και $\tau_j\neq 0$ ($j=1,\dots,r$) διακεκριμένες ρίζες του χαρακτηριστικού πολυώνυμου με πολλαπλότητες $m_1, \dots, m_r$ αντίστοιχα, και $\lambda_{2r+1}, \dots, \lambda_s$ διακεκριμένες πραγματικές ρίζες του χαρακτηριστικού πολυώνυμου με πολλαπλότητες $m_{2r+1}, \dots, m_s$ αντίστοιχα (οπότε $2(m_1+\dots+m_r)+m_{2r+1}+\dots+m_s=n$). Τότε οι παρακάτω n συναρτήσεις αποτελούν ένα βασικό σύνολο πραγματικών λύσεων:
    \[
        x^j e^{\sigma_i x}\cos\tau_i x, x\in\mathbb{R} \quad \text{και} \quad x^j e^{\sigma_i x}\sin\tau_i x, x\in\mathbb{R} \quad (j=0,1,\dots,m_i-1; \ i=1,\dots,r);
    \]
\end{enumerate}

\[
    x^j e^{\lambda_i x}, \quad x\in\mathbb{R} \quad (j=0,1,\dots,m_i-1; \ i=2r+1,\dots,s).
\]
\end{Thewrhma}
\paragraph{ΑΠΟΔΕΙΞΗ.} 
\begin{rlist}
\item  Ας είναι y μια λύση της (Ε$_0$). Τότε
\[
    0=L(y)=L(\text{Re}\,y+i\,\text{Im}\,y)=L(\text{Re}\,y)+i L(\text{Im}\,y)
\]
και επομένως
\[
    \text{Re}\,y=0 \quad \text{και} \quad \text{Im}\,y=0.
\]
\item  Θεωρούμε μια λύση y με
\[
    y(x_0)=a_0, y'(x_0)=a_1, \dots, y^{(n-1)}(x_0)=a_{n-1}
\]
όπου $x_0\in\mathbb{R}$ και $a_0, a_1, \dots, a_{n-1}$ είναι πραγματικοί αριθμοί. Τότε $z=\text{Im}\,y$ είναι μια λύση που πληροί τις αρχικές συνθήκες
\[
    z(x_0)=0, z'(x_0)=0, \dots, z^{(n-1)}(x_0)=0.
\]
Έτσι, z είναι η μηδενική λύση (θεώρημα 1) και επομένως η y είναι πραγματική.

\item Ας είναι $\{y_1, \dots, y_n\}$ ένα βασικό σύνολο πραγματικών λύσεων και y μια πραγματική λύση. Τότε $y=c_1y_1+\dots+c_ny_n$, όπου $c_1, \dots, c_n$ είναι σταθερές. Έχομε $0=\text{Im}\,y=(\text{Im}\,c_1)y_1+\dots+(\text{Im}\,c_n)y_n$ και επομένως $\text{Im}\,c_1=\dots=\text{Im}\,c_n=0$, δηλαδή οι σταθερές $c_1, \dots, c_n$ είναι πραγματικές.

\item  Οι αριθμοί $\lambda_j=\sigma_j+i\tau_j, \dots, \bar{\lambda}_j=\sigma_j-i\tau_j$, είναι επίσης ρίζες του χαρακτηριστικού πολυώνυμου με πολλαπλότητες $m_1, \dots, m_r$ αντίστοιχα. Σύμφωνα, με το θεώρημα 19, οι συναρτήσεις
\[
    x^j e^{\lambda_i x}, \quad x\in\mathbb{R} \quad (j=0,1,\dots,m_i-1; \ i=1,\dots,r);
\]
και επομένως
\[
    x^j e^{\lambda_i x}, \quad x\in\mathbb{R} \quad (j=0,1,\dots,m_i-1; \ i=2r+1,\dots,s)
\]
αποτελούν ένα βασικό σύνολο λύσεων της (Ε$_0$). Για οποιοδήποτε $i\in\{1,\dots,r\}$ και για τυχόν $j\in\{0,1,\dots,m_i-1\}$ έχουμε
\begin{align*}
    x^j e^{\sigma_i x}\cos\tau_i x &= \frac{1}{2}\left(x^j e^{\lambda_i x} + x^j e^{\bar{\lambda}_i x}\right), \quad x\in\mathbb{R}, \\
    x^j e^{\sigma_i x}\sin\tau_i x &= \frac{1}{2i}\left(x^j e^{\lambda_i x} - x^j e^{\bar{\lambda}_i x}\right), \quad x\in\mathbb{R}
\end{align*}
και επομένως οι συναρτήσεις $x^j e^{\sigma_i x}\cos\tau_i x$, $x\in\mathbb{R}$ και $x^j e^{\sigma_i x}\sin\tau_i x$, $x\in\mathbb{R}$ (θεώρημα 3) επίσης λύσεις της (Ε$_0$) και μάλιστα πραγματικές. Τώρα, θ' αποδείξουμε ότι οι λύσεις
\end{rlist}
\[
    x^j e^{\sigma_i x}\cos\tau_i x, \ x\in\mathbb{R} \quad \text{και} \quad x^j e^{\sigma_i x}\sin\tau_i x, \ x\in\mathbb{R} \quad (j=0,1,\dots,m_i-1; \ i=1,\dots,r);
\]
\[
    x^j e^{\lambda_i x}, \ x\in\mathbb{R} \quad (j=0,1,\dots,m_i-1; \ i=2r+1,\dots,s)
\]
είναι γραμμικά ανεξάρτητες. Θεωρούμε τις σταθερές $c_{ij}$ ($j=0,1,\dots,m_i-1; \ i=1,\dots,r$), $d_{ij}$ ($j=0,1,\dots,m_i-1; \ i=1,\dots,r$) και $h_{ij}$ ($j=0,1,\dots,m_i-1; \ i=2r+1,\dots,s$) και υποθέτουμε ότι για $x\in\mathbb{R}$
\[
    \sum_{i=1}^r \sum_{j=0}^{m_i-1} c_{ij} x^j e^{\sigma_i x}\cos\tau_i x + \sum_{i=1}^r \sum_{j=0}^{m_i-1} d_{ij} x^j e^{\sigma_i x}\sin\tau_i x + \sum_{i=2r+1}^s \sum_{j=0}^{m_i-1} h_{ij} x^j e^{\lambda_i x} = 0
\]
τότε για όλα τα $x\in\mathbb{R}$ είναι
\[
    \sum_{i=1}^r \sum_{j=0}^{m_i-1} \left(\frac{c_{ij}-id_{ij}}{2}\right)x^j e^{\lambda_i x} + \sum_{i=1}^r \sum_{j=0}^{m_i-1} \left(\frac{c_{ij}+id_{ij}}{2}\right)x^j e^{\bar{\lambda}_i x} + \sum_{i=2r+1}^s \sum_{j=0}^{m_i-1} h_{ij} x^j e^{\lambda_i x} = 0
\]
απ'όπου προκύπτει
\[
    \frac{c_{ij}-id_{ij}}{2}=0 \quad \text{και} \quad \frac{c_{ij}+id_{ij}}{2}=0 \quad (j=0,1,\dots,m_i-1; \ i=1,\dots,r);
\]
\[
    h_{ij}=0 \quad (j=0,1,\dots,m_i-1; \ i=2r+1,\dots,s).
\]
Έτσι, έχουμε
\[
    c_{ij}=d_{ij}=0 \quad (j=0,1,\dots,m_i-1; \ i=1,\dots,r);
\]
\[
    h_{ij}=0 \quad (j=0,1,\dots,m_i-1; \ i=2r+1,\dots,s).
\]
Σύμφωνα με το θεώρημα 20, αν οι συντελεστές $a_i$ ($i=0,1,\dots,n-1,n$) είναι πραγματικοί και αν $\lambda_1=\sigma_1+i\tau_1, \dots, \lambda_r=\sigma_r+i\tau_r$ με $\sigma_j, \tau_j\in\mathbb{R}$ και $\tau_j\neq 0$ ($i=1,\dots,r$) είναι διακεκριμένες ρίζες του χαρακτηριστικού πολυωνύμου και $\lambda_{2r+1}, \dots, \lambda_n$ είναι διακεκριμένες πραγματικές ρίζες αυτού, τότε οι n συναρτήσεις
\[
    e^{\sigma_i x}\cos\tau_i x, x\in\mathbb{R} \quad \text{και} \quad e^{\sigma_i x}\sin\tau_i x, x\in\mathbb{R} \quad (i=1,\dots,r); \quad e^{\lambda_j x}, x\in\mathbb{R} \quad (j=2r+1,\dots,n)
\]
αποτελούν ένα βασικό σύνολο πραγματικών λύσεων της (Ε$_0$).

Τέλος, ας θεωρήσουμε την απλή περίπτωση της ομογενούς γραμμικής διαφορικής εξίσωσης δεύτερης τάξης (E$_0$)$_2$ \quad $a_2 y''+a_1 y'+a_0 y=0$,

όπου $a_2, a_1$ και $a_0$ είναι πραγματικές σταθερές. Το χαρακτηριστικό πολυώνυμο είναι το $p(\lambda)=a_2\lambda^2+a_1\lambda+a_0$. Διακρίνουμε τις εξής τρείς περιπτώσεις.

\paragraph{Περίπτωση 1.} Το χαρακτηριστικό πολυώνυμο έχει τις άνισες πραγματικές ρίζες ($a_1^2-4a_2a_0>0$)
\[
    \lambda_1 = \frac{-a_1+\sqrt{a_1^2-4a_2a_0}}{2a_2}, \quad \lambda_2 = \frac{-a_1-\sqrt{a_1^2-4a_2a_0}}{2a_2}.
\]
Τότε οι συναρτήσεις $e^{\lambda_1 x}$, $x\in\mathbb{R}$ και $e^{\lambda_2 x}$, $x\in\mathbb{R}$ αποτελούν ένα βασικό σύνολο λύσεων.

\paragraph{Περίπτωση 2.} Το χαρακτηριστικό πολυώνυμο έχει την διπλή πραγματική ρίζα ($a_1^2-4a_2a_0=0$)
\[
    \lambda_0 = -\frac{a_1}{2a_2}.
\]
Στην περίπτωση αυτή ένα βασικό σύνολο λύσεων αποτελούν οι συναρτήσεις $e^{\lambda_0 x}$, $x\in\mathbb{R}$ και $xe^{\lambda_0 x}$, $x\in\mathbb{R}$.

\paragraph{Περίπτωση 3.} Το χαρακτηριστικό πολυώνυμο έχει τις δύο συζυγείς ρίζες ($a_1^2-4a_2a_0<0$)
\[
    \lambda_1 = \frac{-a_1+i\sqrt{4a_2a_0-a_1^2}}{2a_2} = \sigma+i\tau, \quad \lambda_2 = \frac{-a_1-i\sqrt{4a_2a_0-a_1^2}}{2a_2} = \sigma-i\tau.
\]
Τότε οι συναρτήσεις $e^{\sigma x}\cos\tau x$, $x\in\mathbb{R}$ και $e^{\sigma x}\sin\tau x$, $x\in\mathbb{R}$ συνιστούν ένα βασικό σύνολο λύσεων.

\subsection{Η μέθοδος των αγνώστων σταθερών}
Στο προηγούμενο Εδάφιο (Παράγραφος 2.3) αναπτύξαμε τη μέθοδο μεταβολής των σταθερών για την εύρεση μιας μερικής λύσης της μη ομογενούς γραμμικής διαφορικής εξίσωσης (Ε). Η μέθοδος αυτή δεν προϋποθέτει ότι οι συντελεστές $a_i$ ($i=0,1,\dots,n-1,n$) είναι σταθερές, απαιτεί όμως να είναι γνωστό ένα βασικό σύνολο λύσεων της αντίστοιχης ομογενούς γραμμικής διαφορικής εξίσωσης (Ε$_0$). Επίσης, στην υπόψη μέθοδο δεν υπάρχει κανένας περιορισμός για τη μορφή του δεύτερου μέλους b, η εφαρμογή της όμως οδηγεί στην επίλυση ενός γραμμικού (αλγεβρικού) συστήματος $n$ εξισώσεων με $n$ αγνώστους (εργασία ιδιαίτερα επίπονη για τα μεγάλα $n$) καθώς επίσης στον υπολογισμό n ολοκληρωμάτων. Έτσι, η μέθοδος μεταβολής των σταθερών είναι αρκετά γενική, πολύπλοκη και δύσκολη όμως σε αρκετές περιπτώσεις.

Εδώ, θ'αναπτύξουμε μια άλλη μέθοδο για την εύρεση μιας μερικής λύσης της μη ομογενούς γραμμικής διαφορικής εξίσωσης (Ε) που εφαρμόζεται όταν η (Ε) έχει σταθερούς συντελεστές και το δεύτερο μέλος $b$ έχει κάποια κατάλληλη μορφή. Η μέθοδος αυτή είναι γνωστή ως μέθοδος των αγνώστων σταθερών και στις περισσότερες περιπτώσεις (απ'αυτές που εφαρμόζεται) είναι πιο απλή απ'τη μέθοδο μεταβολής των σταθερών. Θα παρουσιάσουμε τα βασικά σημεία της μεθόδου αυτής χωρίς ιδιαίτερη ανάπτυξη. Μερικά παραδείγματα εφαρμογής της θα παραθέσουμε στην παράγραφο 3.5.

\paragraph{Περίπτωση I:} $b=P$, όπου P$\neq 0$ είναι ένα πολυώνυμο. Ας υποθέσουμε ότι για κάποιο $k\in\{0,1,\dots,n-1,n\}$ είναι
\[
    a_k\neq 0 \quad \text{και (όταν k>0)} \quad a_0=\dots=a_{k-1}=0.
\]
Τότε αναζητούμε μια μερική λύση $y_\mu$ της διαφορικής εξίσωσης (Ε) τέτοια ώστε η k-τάξης παράγωγός της να είναι ένα πολυώνυμο του ίδιου βαθμού με το $P$, δηλαδή
\[
    y_\mu^{(k)} = d_m x^m+d_{m-1}x^{m-1}+\dots+d_1 x+d_0, \quad x\in\mathbb{R},
\]
όπου $m$ είναι ο βαθμός του $P$ και $d_i$ ($i=0,1,\dots,m-1,m$) είναι άγνωστες σταθερές που θα πρέπει να προσδιορισθούν. Παίρνοντας τις παραγώγους $y_\mu^{(k)}, \dots, y_\mu^{(n)}$ και θέτοντας αυτές στην (Ε), προκύπτει μια εξίσωση ως προς το πολυώνυμο βαθμού που ορίζει σ'ένα σύστημα (αλγεβρικό) $n+1$ εξισώσεων με αγνώστους $d_i$ ($i=0,1,\dots,m-1,m$). Απ'το σύστημα αυτό προκύπτουν οι τιμές των $d_i$ ($i=0,1,\dots,m-1,m$). Τέλος, με $k$ ολοκληρώσεις βρίσκεται μια μερική λύση $y_\mu$.

\paragraph{Περίπτωση II:} $b(x)=P(x)e^{\lambda x}$, $x\in\mathbb{R}$, όπου $P\neq 0$ είναι ένα πολυώνυμο και $\lambda\neq 0$ είναι μια σταθερά. Τότε είναι εύκολο να διαπιστώσουμε ότι η αντικατάσταση
\[
    y(x)=z(x)e^{\lambda x}, \quad x\in\mathbb{R}
\]
μετασχηματίζει τη διαφορική εξίσωση (Ε) σε μια εξίσωση της μορφής
\[
    A_n z^{(n)} + A_{n-1} z^{(n-1)} + \dots + A_1 z' + A_0 z = P,
\]
όπου $A_i$ ($i=0,1,\dots,n-1,n$) είναι σταθερές και $A_n\neq 0$. Έτσι, αναγόμαστε στην περίπτωση Ι. Αν $z_\mu$ είναι μια μερική λύση της τελευταίας εξίσωσης, η συνάρτηση $y_\mu(x)=z_\mu(x)e^{\lambda x}$, $x\in\mathbb{R}$ θα είναι μια μερική λύση της (Ε).

\paragraph{Περίπτωση III:} $b(x)=P(x)e^{\sigma x}\cos\tau x$, $x\in\mathbb{R}$ ή $b(x)=P(x)e^{\sigma x}\sin\tau x$, $x\in\mathbb{R}$, όπου P$\neq 0$ είναι ένα πολυώνυμο και $\sigma,\tau$ είναι πραγματικές σταθερές με $\tau\neq 0$. Στην περίπτωση αυτή είναι $b=b_1+b_2$, όπου αντίστοιχα $b_1(x)=\frac{1}{2}P(x)e^{(\sigma+i\tau)x}$, $x\in\mathbb{R}$ και $b_2(x)=\frac{1}{2}P(x)e^{(\sigma-i\tau)x}$, $x\in\mathbb{R}$ ή
\[
    b_1(x)=\frac{1}{2i}P(x)e^{(\sigma+i\tau)x}, \quad x\in\mathbb{R} \quad \text{και} \quad b_2(x)=\frac{-1}{2i}P(x)e^{(\sigma-i\tau)x}, \quad x\in\mathbb{R}.
\]
Έτσι, αρκεί (θεώρημα 13) να βρεθούν μερικές λύσεις για καθεμιά απ'τις εξισώσεις $L(y)=b_1$ και $L(y)=b_2$, που είναι της περίπτωσης ΙΙ.

\paragraph{Περίπτωση IV.} Οι συντελεστές $a_i$ ($i=0,1,\dots,n-1,n$) είναι πραγματικοί και $b(x)=P(x)e^{\sigma x}\cos\tau x$, $x\in\mathbb{R}$ ή $b(x)=P(x)e^{\sigma x}\sin\tau x$, $x\in\mathbb{R}$, όπου P$\neq 0$ είναι ένα πολυώνυμο με πραγματικούς συντελεστές και $\sigma, \tau$ είναι πραγματικές σταθερές με $\tau\neq 0$. Βρίσκουμε (περίπτωση ΙΙ) μια μερική λύση της διαφορικής εξίσωσης $L(y)=q$ με
\[
    q(x)=P(x)e^{(\sigma+i\tau)x}, \quad x\in\mathbb{R},
\]
οπότε (όπως εύκολα διαπιστώνεται) η συνάρτηση Re\,$y_\mu$ ή η συνάρτηση Im\,$y_\mu$ θα είναι αντίστοιχα μια μερική λύση της (Ε).
Άλλως, ας παρατηρήσουμε ότι, αν η b είναι γραμμικός συνδυασμός συναρτήσεων όπου καθεμιά απ'αυτές έχει κάποια απ'τις μορφές των παραπάνω περιπτώσεων, τότε μπορεί να βρεθεί μια μερική λύση της (Ε), σύμφωνα με το θεώρημα 13.

\subsection{Διαφορικές εξισώσεις Euler}
Μια διαφορική εξίσωση Euler είναι μια ομογενής γραμμική διαφορική εξίσωση της μορφής
\[
    a_n x^n \frac{d^n y}{dx^n} + a_{n-1} x^{n-1} \frac{d^{n-1}y}{dx^{n-1}} + \dots + a_1 x \frac{dy}{dx} + a_0 y = 0,
\]
όπου $a_i$ ($i=0,1,\dots,n-1,n$) είναι σταθερές και $a_n\neq 0$. Θα θεωρήσουμε με ως διάστημα ορισμού της διαφορικής αυτής εξίσωσης το διάστημα $(0,\infty)$ (αν διάστημα ορισμού είναι το $(-\infty,0)$, τότε για $w=-x$ η εξίσωση μετασχηματίζεται σε μια άλλη της ίδιας μορφής με διάστημα ορισμού το $(0,\infty)$). Η αντικατάσταση $t=\log x$, $x>0$ μετασχηματίζει την παραπάνω διαφορική εξίσωση σε μια ομογενή γραμμική διαφορική εξίσωση
$n$-τάξης με σταθερούς συντελεστές. Πραγματικά: Για όλα τα $x>0$ έχουμε
\begin{align*}
    x\frac{dy}{dx} &= x\frac{dy}{dt}\frac{dt}{dx} = x\frac{dy}{dt}\frac{1}{x} = \frac{dy}{dt}, \\
    x^2\frac{d^2y}{dx^2} &= x^2\frac{d}{dx}\left(\frac{dy}{dx}\right) = x^2\left(\frac{d}{dt}\left(\frac{1}{x}\frac{dy}{dt}\right)\frac{dt}{dx}\right) = x^2\left(-\frac{1}{x^2}\frac{dy}{dt}+\frac{1}{x}\frac{d^2y}{dt^2}\frac{1}{x}\right) \\
    &= -\frac{dy}{dt}+x\frac{d}{dt}\left(\frac{dy}{dt}\right) = -\frac{dy}{dt}+\frac{d^2y}{dt^2}
\end{align*}
και γενικά μπορούμε να διαπιστώσουμε ότι υπάρχουν σταθερές $c_{ij}$ ($j=1,\dots,i$; $i=1,\dots,n-1,n$) με $c_{ii}=1$ ($i=1,\dots,n-1,n$) και τέτοιες ώστε
\[
    x^i\frac{d^iy}{dx^i} = c_{i1}\frac{dy}{dt}+c_{i2}\frac{d^2y}{dt^2}+\dots+c_{ii}\frac{d^iy}{dt^i}, \quad (i=1,\dots,n-1,n).
\]
Θέτοντας τις εκφράσεις αυτές στη διαφορική μας εξίσωση, παίρνουμε μια διαφορική εξίσωση της μορφής
\[
    a'_n\frac{d^ny}{dt^n}+a'_{n-1}\frac{d^{n-1}y}{dt^{n-1}}+\dots+a'_1\frac{dy}{dt}+a'_0y=0,
\]
όπου $a_i$ ($i=0,1,\dots,n-1,n$) είναι σταθερές και $a_n=a_n\neq 0$.
Ας ασχοληθούμε ιδιαίτερα με τη διαφορική εξίσωση \eng{Euler} δεύτερης τάξης
\[
    a_2 x^2 \frac{d^2y}{dx^2} + a_1 x \frac{dy}{dx} + a_0 y = 0.
\]
Η αντικατάσταση $t=\log x$, $x>0$ τη μετασχηματίζει στην εξίσωση
\[
    a_2\left(-\frac{dy}{dt}+\frac{d^2y}{dt^2}\right)+a_1\frac{dy}{dt}+a_0 y=0,
\]
δηλαδή στην ομογενή γραμμική διαφορική εξίσωση δεύτερης τάξης με σταθερούς συντελεστές
\[
    a_2\frac{d^2y}{dt^2}+(a_1-a_2)\frac{dy}{dt}+a_0 y=0.
\]

\subsection{Παραδείγματα}
\begin{Paradeigma}{}
Να επιλυθούν οι γραμμικές διαφορικές εξισώσεις πρώτης τάξης:
    \begin{enumerate}[label=(\roman*)]
        \item $y'+2y=0$.
        \item $y'+2y=x$.
    \end{enumerate}
\end{Paradeigma}
\lysh (i) Σύμφωνα με το θεώρημα 18, είναι μια λύση αν και μόνο αν για όλα τα $x\in\mathbb{R}$
\[
    y(x)=y(0)e^{-2x}.
\]
Έτσι, όλες οι λύσεις είναι $y(x)=ce^{-2x}$, $x\in\mathbb{R}$, όπου c αυθαίρετη σταθερά.

\lysh (ii) Σύμφωνα πάλι με το θεώρημα 18, είναι μια λύση αν και μόνο αν
\[
    y(x)=e^{-2x}\left[y(0)+\int_0^x te^{2t}dt\right] = e^{-2x}\left[y(0)+\frac{1}{4}+\frac{1}{2}(2x+1)e^{2x}\right]
\]
για κάθε $x\in\mathbb{R}$. Άρα οι λύσεις δίνονται απ'τον τύπο $y(x)=ce^{-2x}+\frac{1}{4}(2x+1)$, $x\in\mathbb{R}$, όπου c είναι αυθαίρετη σταθερά.

\begin{Paradeigma}{}
Να επιλυθούν οι παρακάτω ομογενείς γραμμικές διαφορικές εξισώσεις:
    \begin{enumerate}[label=(\roman*)]
        \item $y'''-4y''+y'+6y=0$.
        \item $y'''-y''-y'+y=0$.
        \item $y'''-3y''+4y'-2y=0$.
        \item $y^{(4)}+2y''+y=0$.
    \end{enumerate}
\end{Paradeigma}
Θα χρησιμοποιήσουμε εδώ το θεώρημα 20.

(i) Το χαρακτηριστικό πολυώνυμο είναι $p(\lambda)=\lambda^3-4\lambda^2+\lambda+6=(\lambda+1)(\lambda-2)(\lambda-3)$ με τις απλές ρίζες $\lambda_1=-1$, $\lambda_2=2$ και $\lambda_3=3$. Έτσι, οι λύσεις δίνονται απ'τον τύπο
\[
    y(x)=c_1 e^{-x}+c_2 e^{2x}+c_3 e^{3x}, \quad x\in\mathbb{R},
\]
όπου $c_1, c_2$ και $c_3$ είναι αυθαίρετες σταθερές.

(ii) Θεωρούμε το χαρακτηριστικό πολυώνυμο $p(\lambda)=\lambda^3-\lambda^2-\lambda+1$ και βλέπουμε ότι αυτό έχει τη διπλή ρίζα $\lambda_1=1$ και την απλή ρίζα $\lambda_2=-1$, γιατί γράφεται $p(\lambda)=(\lambda-1)^2(\lambda+1)$. Άρα, οι λύσεις είναι
\[
    y(x)=c_1 e^x+c_2 xe^x+c_3 e^{-x}=(c_1+c_2x)e^x+c_3 e^{-x}, \quad x\in\mathbb{R},
\]
όπου $c_1, c_2$ και $c_3$ είναι αυθαίρετες σταθερές.

(iii) Το χαρακτηριστικό πολυώνυμο $p(\lambda)=\lambda^3-3\lambda^2+4\lambda-2=(\lambda-1)(\lambda^2-2\lambda+2)$ έχει τις απλές ρίζες $\lambda_1=1$, $\lambda_2=1+i$ και $\lambda_3=1-i$. Επομένως, οι λύσεις δίνονται απ'τον τύπο
\[
    y(x)=c_1 e^x+c_2 e^x\cos x+c_3 e^x\sin x=e^x(c_1+c_2\cos x+c_3\sin x), \quad x\in\mathbb{R},
\]
όπου $c_1, c_2$ και $c_3$ είναι αυθαίρετες σταθερές.

(iv) Εδώ το χαρακτηριστικό πολυώνυμο είναι το $p(\lambda)=\lambda^4+2\lambda^2+1=(\lambda^2+1)^2$ το οποίο έχει τις διπλές ρίζες $\lambda_1=i$ και $\lambda_2=-i$. Έτσι, οι λύσεις είναι
\[
    y=c_1\cos x+c_2x\cos x+c_3\sin x+c_4x\sin x=(c_1+c_2x)\cos x+(c_3+c_4x)\sin x, \quad x\in\mathbb{R},
\]
όπου $c_1, c_2, c_3$ και $c_4$ είναι αυθαίρετες σταθερές.

\begin{Paradeigma}{Να βρεθεί μια μερική λύση για καθεμιά απ'τις παρακάτω μη ομογενείς γραμμικές διαφορικές εξισώσεις:}
    \begin{enumerate}[label=(\roman*)]
        \item $y^{(4)}+y'''=x^3+1$.
        \item $y'''+2y'+y=x^2-2x$.
        \item $y'''+y''+2y=x^2e^{-2x}$.
        \item $y''-2y'+y=xe^x\cos 2x$.
        \item $y''-2y'+y=xe^{-x}\sin 2x$.
        \item $y''-4y'+4y=e^x+\sin x$.
    \end{enumerate}
\end{Paradeigma}
(i) Αναζητούμε μια μερική λύση $y_\mu$ τέτοια ώστε
\[
    y_\mu(x)=ax^3+\beta x^2+\gamma x+\delta, \quad x\in\mathbb{R},
\]
όπου οι συντελεστές $\alpha,\beta,\gamma$ και $\delta$ θα προσδιορισθούν. Θα είναι για κάθε $x\in\mathbb{R}$
\[
    (6\alpha x+2\beta)+(\alpha x^3+\beta x^2+\gamma x+\delta)=x^3+1
\]
ή
\[
    \alpha x^3+\beta x^2+(6\alpha+\gamma)x+(2\beta+\delta)=x^3+1,
\]
οπότε
\[
    \alpha=1, \quad \beta=0, \quad 6\alpha+\gamma=0 \quad \text{και} \quad 2\beta+\delta=1,
\]
δηλαδή $\alpha=1$, $\beta=0$, $\gamma=-6$ και $\delta=1$. Έχουμε λοιπόν
\[
    y_\mu''(x)=x^3-6x+1, \quad x\in\mathbb{R}
\]
και άρα μια μερική λύση είναι
\[
    y_\mu(x)=\frac{x^4}{20}-\frac{x^3}{2}+\frac{1}{2}x^2, \quad x\in\mathbb{R}.
\]
(ii) $y_\mu(x)=\alpha x^2+\beta x+\gamma$, $x\in\mathbb{R}$, όπου $\alpha,\beta$ και $\gamma$ είναι σταθερές, θα είναι μια μερική λύση αν και μόνο αν
\[
    2(2\alpha x+\beta)+(\alpha x^2+\beta x+\gamma)=2x^2-x \quad \text{για κάθε } x\in\mathbb{R}
\]
ή ισοδύναμα
\[
    \alpha=2, \quad 4\alpha+\beta=-1 \quad \text{και} \quad 2\beta+\gamma=0,
\]
το οποίο ισχύει αν και μόνο αν $\alpha=2$, $\beta=-9$ και $\gamma=18$. Έτσι, μια μερική λύση είναι η
\[
    y_\mu(x)=2x^2-9x+18, \quad x\in\mathbb{R}.
\]
(iii) Εκτελούμε το μετασχηματισμό $y=ze^{-2x}$, οπότε η εξίσωση
γίνεται
\[
    (z'''-6z''+12z'-8z)e^{-2x}+(z''-4z'+4z)e^{-2x}+2ze^{-2x}=x^2e^{-2x}
\]
ή
\[
    z'''-5z''+8z'-2z=x^2.
\]
Για την τελευταία εξίσωση αναζητούμε μια μερική λύση $z_\mu$ της μορφής $z_\mu(x)=\alpha x^2+\beta x+\gamma$, $x\in\mathbb{R}$, όπου $\alpha,\beta,\gamma$ είναι σταθερές. Τότε βρίσκουμε εύκολα ότι $\alpha=-\frac{1}{2}$, $\beta=-4$ και $\gamma=-\frac{17}{2}$ και έτσι έχουμε
\[
    z_\mu(x)=-\frac{1}{2}x^2-4x-\frac{17}{2}, \quad x\in\mathbb{R}.
\]
Μια μερική λύση της αρχικής εξίσωσης είναι η
\[
    y_\mu(x)=\left(-\frac{1}{2}x^2-4x-\frac{17}{2}\right)e^{-2x}, \quad x\in\mathbb{R}.
\]

\lysh (iv) και (v). Θεωρούμε τη διαφορική εξίσωση
\begin{equation*}
    y''-2y'+y=xe^{(-1\pm2i)x}. \tag{*}
\end{equation*}
Ο μετασχηματισμός $y=ze^{(-1+2i)x}$ μετασχηματίζει αυτή στην εξίσωση
\[
    [z''+2(-1+2i)z'+(-1+2i)^2z]e^{(-1+2i)x}-2[z'+(-1+2i)z]e^{(-1+2i)x}+ze^{(-1+2i)x}=xe^{(-1+2i)x}
\]
ή (μετά από πράξεις)
\[
    z''+4(i-1+1)z'-8iz=x.
\]
Αναζητούμε μια μερική λύση $z_\mu(x)$ αυτής της μορφής $z_\mu(x)=\alpha x+\beta$, $x\in\mathbb{R}$ ($\alpha,\beta$ σταθερές), οπότε βρίσκουμε
\[
    z_\mu(x)=\frac{1}{4}ix+\frac{1}{16}(-1+i).
\]
Μια μερική λύση $y_\mu$ της (*) είναι
\begin{align*}
    y_\mu(x) &= \Re\left[e^{(-1+2i)x}\left(\frac{1}{4}ix+\frac{1}{16}(-1+i)\right)\right] = e^{-x}\left[-\frac{1}{16}x\cos 2x-\frac{1}{8}\left(x+\frac{1}{2}\right)\sin 2x\right] \\
    &= e^{-x}\left[-\frac{1}{16}(x+1)\cos 2x-\frac{1}{8}\left(x+\frac{1}{4}\right)\sin 2x\right]
\end{align*}
για $x\in\mathbb{R}$. Έτσι, η συνάρτηση
\[
    y_\mu(x)=e^{-x}\left[-\frac{1}{16}x\cos 2x-\frac{1}{8}\left(x+\frac{1}{2}\right)\sin 2x\right], \quad x\in\mathbb{R}
\]
είναι μια μερική λύση της (iv), ενώ η
\[
    y_\mu(x)=e^{-x}\left[-\frac{1}{16}x\sin 2x+\frac{1}{8}\left(x+\frac{1}{2}\right)\cos 2x\right], \quad x\in\mathbb{R}
\]
είναι μια μερική λύση της (v).
(vi) Βρίσκουμε τη μερική λύση $y_\mu^1(x)=e^x$, $x\in\mathbb{R}$ της διαφορικής εξίσωσης
\[
    y''-4y'+4y=e^x
\]
και τη μερική λύση $y_\mu^2(x)=\frac{4}{25}\cos x+\frac{3}{25}\sin x$, $x\in\mathbb{R}$ της εξίσωσης
\[
    y''-4y'+4y=\sin x.
\]
Τότε (θεώρημα 13) μια μερική λύση της εξίσωσης μας είναι η
\[
    y_\mu(x)=y_\mu^1(x)+y_\mu^2(x)=e^x+\frac{4}{25}\cos x+\frac{3}{25}\sin x, \quad x\in\mathbb{R}.
\]

\begin{Paradeigma}{Να επιλυθεί η μη ομογενής γραμμική διαφορική εξίσωση
\[
    y'''+y'=2+\sin x.
\]
Ιδιαίτερα, να βρεθεί η λύση $y_0$ που πληροί τις αρχικές συνθήκες
\[
    y_0(0)=0, \quad y_0'(0)=1 \quad \text{και} \quad y_0''(0)=-1.
\]}
\end{Paradeigma}
\lysh Το χαρακτηριστικό πολυώνυμο της αντίστοιχης ομογενούς διαφορικής εξίσωσης είναι $p(\lambda)=\lambda^3+\lambda=\lambda(\lambda^2+1)$ που έχει τις απλές ρίζες $\lambda_1=0$, $\lambda_2=i$ και $\lambda_3=-i$. Έτσι, οι λύσεις της αντίστοιχης ομογενούς εξίσωσης είναι
\[
    \tilde{y}(x)=c_1+c_2\cos x+c_3\sin x, \quad x\in\mathbb{R},
\]
όπου $c_1, c_2$ και $c_3$ είναι αυθαίρετες σταθερές.

Θα βρούμε τώρα μια μερική λύση $y_\mu$ της μη ομογενούς διαφορικής μας εξίσωσης. Για το σκοπό αυτό, θεωρούμε τις διαφορικές εξισώσεις
\begin{equation*}
    y'''+y'=2 \tag{*}
\end{equation*}
και
\begin{equation*}
    y'''+y'=\sin x. \tag{**}
\end{equation*}
Αναζητώντας μια μερική λύση $y_1$ της (*) τέτοια ώστε $y_1(x)=\alpha$, $x\in\mathbb{R}$ ($\alpha$ σταθερά), βρίσκουμε ότι $\alpha=2$ και άρα μια μερική λύση της (*) είναι η $y_1(x)=2x$, $x\in\mathbb{R}$. Στη συνέχεια, θα βρούμε μια μερική λύση της εξίσωσης (**). Ας θεωρήσουμε τη διαφορική εξίσωση
\begin{equation*}
    y'''+y'=e^{ix}. \tag{***}
\end{equation*}
Ο μετασχηματισμός $y=ze^{ix}$ την μετασχηματίζει στην εξίσωση
\[
    z'''+3iz''-2z'=1
\]
που έχει, όπως εύκολα προκύπτει, τη μερική λύση $z_\mu(x)=-\frac{1}{2}x$, $x\in\mathbb{R}$.

Μια μερική λύση της εξίσωσης (***) είναι η
\[
    -\frac{1}{2}xe^{ix}=-\frac{1}{2}x(\cos x+i\sin x), \quad x\in\mathbb{R}.
\]
'Αρα, η διαφορική εξίσωση (**) έχει τη μερική λύση $y_2(x)=-\frac{1}{2}x\sin x$, $x\in\mathbb{R}$. Επομένως, μια μερική λύση $y_\mu$ της διαφορικής μας εξίσωσης είναι η
\[
    y_\mu(x)=y_1(x)+y_2(x)=2x-\frac{1}{2}x\sin x, \quad x\in\mathbb{R}.
\]
'Ολες οι λύσεις της εξίσωσης δίνονται απ'τον τύπο
\[
    y(x)=c_1+c_2\cos x+c_3\sin x+2x-\frac{1}{2}x\sin x, \quad x\in\mathbb{R},
\]
όπου $c_1,c_2,c_3$ είναι αυθαίρετες σταθερές. Ειδικά, για τη λύση $y_0$ παίρνουμε
\[
    c_1+c_2=0, \quad c_3+2=1 \quad \text{και} \quad -c_2-1=-1,
\]
δηλαδή $c_1=0$, $c_2=0$ και $c_3=-1$. 'Ετσι, είναι
\[
    y_0(x)=-\sin x+2x-\frac{1}{2}x\sin x, \quad x\in\mathbb{R}.
\]

\begin{Paradeigma}{}
Να επιλυθεί η μη ομογενής γραμμική διαφορική εξίσωση
\[
    y''-2y'+y=\frac{1}{x}e^x, \quad x>0.
\]
\end{Paradeigma}
Η αντίστοιχη ομογενής εξίσωση έχει τις γραμμικά ανεξάρτητες λύσεις $y_1(x)=e^x$, $x>0$ και $y_2(x)=xe^x$, $x>0$. Για να βρούμε μια μερική λύση $y_\mu$ της μη ομογενούς εξίσωσης μας θα χρησιμοποιήσουμε τη μέθοδο μεταβολής των σταθερών (η μέθοδος των αγνώστων σταθερών δεν μπορεί να εφαρμοστεί εδώ). Το σύστημα για τις $v_1, v_2$
\[
    \begin{cases}
        v_1'(x)y_1(x)+v_2'(x)y_2(x)=0 \\
        v_1'(x)y_1'(x)+v_2'(x)y_2'(x)=\frac{1}{x}e^x
    \end{cases}
    \quad \text{ή} \quad
    \begin{cases}
        v_1'(x)e^x+v_2'(x)xe^x=0 \\
        v_1'(x)e^x+v_2'(x)(x+1)e^x=\frac{1}{x}e^x
    \end{cases}
\]
έχει τη λύση
\[
    v_1'(x)=-1 \quad \text{και} \quad v_2'(x)=\frac{1}{x}.
\]
'Ετσι, παίρνουμε
\[
    v_1(x)=-x \quad \text{και} \quad v_2(x)=\log x \quad \text{για } x>0.
\]
Μια μερική λύση $y_\mu$ της διαφορικής εξίσωσής μας είναι
\[
    y_\mu(x)=v_1(x)y_1(x)+v_2(x)y_2(x)=xe^x(\log x-1), \quad x>0.
\]
'Ολες οι λύσεις δίνονται απ'τον τύπο
\[
    y(x)=c_1e^x+c_2xe^x+xe^x(\log x-1)=e^x[c_1+x(c_2+\log x)], \quad x>0,
\]
όπου $c_1=c_1'$, $c_2=c_2'-1$ είναι αυθαίρετες σταθερές.

\begin{Paradeigma}{Εξίσωση \eng{Euler}}
Να επιλυθεί η διαφορική εξίσωση \eng{Euler}
\[
    x^3\frac{d^3y}{dx^3}-x^2\frac{d^2y}{dx^2}-2x\frac{dy}{dx}-4y=0, \quad x>0.
\]
\end{Paradeigma}
Θέτουμε $t=\log x$, $x>0$. Τότε
\[
    \frac{dy}{dx}=\frac{dy}{dt}\frac{dt}{dx}=\frac{1}{x}\frac{dy}{dt},
\]
\[
    \frac{d^2y}{dx^2}=\frac{d}{dx}\left(\frac{1}{x}\frac{dy}{dt}\right)=-\frac{1}{x^2}\frac{dy}{dt}+\frac{1}{x^2}\frac{d^2y}{dt^2},
\]
\[
    \frac{d^3y}{dx^3}=\frac{d}{dx}\left[\frac{1}{x^2}\left(-\frac{dy}{dt}+\frac{d^2y}{dt^2}\right)\right]=-\frac{2}{x^3}\left(-\frac{dy}{dt}+\frac{d^2y}{dt^2}\right)+\frac{1}{x^3}\left(-\frac{d^2y}{dt^2}+\frac{d^3y}{dt^3}\right)\frac{dt}{dx}
\]
\[
    =\frac{1}{x^3}\left(2\frac{dy}{dt}-3\frac{d^2y}{dt^2}+\frac{d^3y}{dt^3}\right)
\]
και έτσι η εξίσωσή μας μετασχηματίζεται στην ομογενή γραμμική διαφορική εξίσωση με σταθερούς συντελεστές
\[
    \left(2\frac{dy}{dt}-3\frac{d^2y}{dt^2}+\frac{d^3y}{dt^3}\right)-\left(-\frac{dy}{dt}+\frac{d^2y}{dt^2}\right)-2\frac{dy}{dt}-4y=0
\]
ή
\[
    \frac{d^3y}{dt^3}-4\frac{d^2y}{dt^2}+\frac{dy}{dt}-4y=0
\]
που έχει τις γραμμικά ανεξάρτητες λύσεις $e^{4t}$, $\cos t$, $t\in\mathbb{R}$. Η αρχική εξίσωση έχει τις γραμμικά ανεξάρτητες λύσεις
\[
    y_1(x)=x^4, \quad x>0; \quad y_2(x)=\cos(\log x), \quad x>0 \quad \text{και} \quad y_3(x)=\sin(\log x), \quad x>0
\]
και άρα όλες οι λύσεις δίνονται απ'τον τύπο
\[
    y(x)=c_1x^4+c_2\cos(\log x)+c_3\sin(\log x), \quad x>0,
\]
όπου $c_1,c_2$ και $c_3$ είναι αυθαίρετες σταθερές.
\paragraph{3.6. Ασκήσεις}
\begin{enumerate}
    \item Να επιλυθούν οι παρακάτω ομογενείς γραμμικές διαφορικές εξισώσεις:
    \begin{enumerate}
        \item[(i)] $2y''+3y'+y=0$.
        \item[(ii)] $y''+6y=0$.
        \item[(iii)] $y(4)+y''-3y'''-y'+2y=0$.
        \item[(iv)] $y''-9y=0$.
        \item[(v)] $y'''-7y''+5y'+y=0$.
        \item[(vi)] $y'''-6y''+12y'=0$.
        \item[(vii)] $y''-y'+12y=0$.
        \item[(viii)] $y'''-13y'-12y=0$.
    \end{enumerate}
    \item Να επιλυθούν τα προβλήματα αρχικών τιμών:
    \begin{enumerate}
        \item[(i)] $y''-2y'+y=0; \quad y(0)=0, y'(0)=-1$.
        \item[(ii)] $y''-y'=0; \quad y(0)=1, y'(0)=3, y''(0)=2$.
        \item[(iii)] $y''+4y=0; \quad y(\frac{\pi}{4})=1, y'(\frac{\pi}{4})=1$.
        \item[(iv)] $y''-y''-y'+y=0; \quad y(0)=0, y'(0)=5, y''(0)=2$.
        \item[(v)] $y''+y'-y=0; \quad y(0)=-1, y'(0)=\sqrt{3}$.
    \end{enumerate}
    \item Να επιλυθούν οι μη ομογενείς γραμμικές διαφορικές εξισώσεις:
    \begin{enumerate}
        \item[(i)] $y''-5y'+6y=x^2+3$.
        \item[(ii)] $y''+4y'+4y=e^{x}-e^{-x}$.
        \item[(iii)] $y(4)-y=x^2$.
        \item[(iv)] $y''-y=x\sin x$.
        \item[(v)] $y'''-3y''+3y'-y=x^2+5e^x$.
        \item[(vi)] $y''-3y'+2y=e^{-x}\cos x$.
        \item[(vii)] $y^{(6)}-3y^{(4)}=1$.
        \item[(viii)] $y''-4y''+5y'-2y=3x^2e^x+x\cos x$.
        \item[(ix)] $y^{(4)}+y=x\sin 2x+x^2$.
        \item[(x)] $y^{(4)}+2y''+y=x^2\cos 3x$.
    \end{enumerate}
    \item Να επιλυθούν τα προβλήματα αρχικών τιμών:
    \begin{enumerate}
        \item[(i)] $y''+y=x+2e^{-x}; \quad y(0)=1, y'(0)=-2$.
        \item[(ii)] $y''+y'=x; \quad y(0)=0, y'(0)=1, y''(0)=0$.
        \item[(iii)] $y''-4y'+4y=e^{2x}; \quad y(0)=0, y'(0)=0$.
        \item[(iv)] $y''+y=3x^2-4\sin x; \quad y(0)=0, y'(0)=1$.
    \end{enumerate}
    \item Να επιλυθούν οι μη ομογενείς γραμμικές διαφορικές εξισώσεις:
    \begin{enumerate}
        \item[(i)] $y''-y=3\log x, \quad x>0$.
        \item[(ii)] $y''+10y'+25y=\frac{e^{-5x}}{x^2}\log x, \quad x>0$.
    \item[(iii)] $y''-2y'+y=\frac{2x}{(x^2+1)^2}$.
    \item[(iv)] $y''+2y'+y=x^{-2}e^{-x}\log x, \quad x>0$.
\end{enumerate}
\item Να επιλυθούν οι διαφορικές εξισώσεις Euler:
\begin{enumerate}
    \item[(i)] $x^2y''-xy'+y=0, \quad x>0$.
    \item[(ii)] $x^3y'''+3x^2y''-2xy'-2y=0, \quad x>0$.
    \item[(iii)] $(x-2)^2y''-(x-2)y'+y=0, \quad x>2$.
    \item[(iv)] $(x+3)^3y'''+3(x+3)^2y''-2(x+3)y'+2y=0, \quad x<-3$.
    \item[(v)] $x^2y^{(5)}-2x^3y'''+4x^2y''=0, \quad x>0$.
\end{enumerate}
\item Να επιλυθούν τα προβλήματα αρχικών τιμών:
\begin{enumerate}
    \item[(i)] $x^2y''-xy'+y=0; \quad y(1)=1, y'(1)=0$.
    \item[(ii)] $x^3y'''+4x^2y''-8xy'+8y=0; \quad y(-1)=0, y'(-1)=1, y''(-1)=0$.
\end{enumerate}
\item Να επιλυθούν οι μη ομογενείς γραμμικές διαφορικές εξισώσεις:
\begin{enumerate}
    \item[(i)] $5x^2y''-3xy'+3y=x^{1/2}, \quad x>0$.
    \item[(ii)] $x^2y''+5xy'+4y=(x^{-3}-x^{-5})\log x, \quad x>0$.
    \item[(iii)] $x^3y'''+3x^2y''-2xy'+2y=x^2\log x, \quad x>0$.
\end{enumerate}
\item Με τη βοήθεια του μετασχηματισμού $z=\sin x$, να βρεθεί η λύση της ομογενούς γραμμικής διαφορικής εξίσωσης
\[
    xy''+2y'+xy=0, \quad x\in(0,\frac{\pi}{2})
\]
που πληροί τις αρχικές συνθήκες
\[
    y(\frac{\pi}{2})=0 \quad \text{και} \quad y'(\frac{\pi}{2})=1.
\]
\item Με το μετασχηματισμό $t=\sin x$, να επιλυθεί η ομογενής γραμμική διαφορική εξίσωση
\[
    (\sin^2x)y''+(\tan x)y'+(k^2)(\cos^2x)y=0, \quad x\in(0,\frac{\pi}{2}) \quad (k>0).
\]
\item Να επιλυθεί η μη ομογενής γραμμική διαφορική εξίσωση
\[
    x^2y''+4xy'+(2+x^2)y=x^2, \quad x>0,
\]
με τη βοήθεια της αντικατάστασης $y=x^rv$.
\item Με τη βοήθεια της αντικατάστασης $y=ze^{x^2}$, να επιλυθεί η
μη ομογενής γραμμική διαφορική εξίσωση
\[
    y''-4xy'+(4x^2-1)y=e^{x^2}.
\]
\end{enumerate}
\section{Ομογενείς γραμμικές διαφορικές εξισώσεις και οι συζυγείς διαφορικές εξισώσεις αυτών. Αυτοσυζυγείς ομογενείς γραμμικές διαφορικές εξισώσεις δεύτερης τάξης}
\thispagestyle{empty}
Στο Εδάφιο αυτό θα αποδειχθεί ότι, για κάθε $k\in\{1,\ldots,n-1\}$, η συνάρτηση $a_k$ έχει συνεχή παράγωγο $k$-τάξης στο Ι, θα δώσουμε μερικές ιδιότητες που έχει ο τελεστής $L^*$ σε συνδυασμό με τον τελεστή $L$. Έτσι, θ' αποδείξουμε την ταυτότητα του \eng{Lagrange} (θεώρημα 21) και τον τύπο του \eng{Green} (θεώρημα 22). Επίσης, θ' αποδείξουμε (θεώρημα 23) ότι, αν είναι γνωστή μια λύση της συζυγούς διαφορικής εξίσωσης $(E_0^*)$ που δεν μηδενίζεται πουθενά στο Ι, τότε οι λύσεις της $(E_0)$ προκύπτουν απ' την επίλυση μιας ομογενούς γραμμικής διαφορικής εξίσωσης $(n-1)$-τάξης. Στη συνέχεια, θα μελετήσουμε τις αυτοσυζυγείς ομογενείς γραμμικές διαφορικές εξισώσεις δεύτερης τάξης με πραγματικούς συντελεστές και θα δώσουμε (θεωρήματα 24 και 25) τη μορφή αυτών. Στο τέλος, θα παραθέσουμε μερικά παραδείγματα και θα δώσουμε ορισμένες ασκήσεις για λύση.

\subsection{Η ταυτότητα του \eng{Lagrange} και ο τύπος του \eng{Green}}
Για δύο συναρτήσεις $u$ και $v$ που έχουν παραγώγους $n$-τάξης στο Ι, εισάγουμε το συμβολισμό
\[
    [uv]=\sum_{k=1}^n\sum_{j=1}^k(-1)^{j-1}u^{(k-j)}(a_k\bar{v})^{(j-1)}.
\]
\begin{Thewrhma}{Ταυτότητα του \eng{Lagrange}}
Αν $u$ και $v$ είναι δύο συναρτήσεις που έχουν παραγώγους $n$-τάξης στο Ι, τότε
\[
    \bar{v}L(u)-u\overline{L^*(v)}=[uv]'.
\]
\end{Thewrhma}
\textit{ΑΠΟΔΕΙΞΗ.} Ας είναι $k$ ένας δείκτης με $1\leq k\leq n$ και $U,V$ δύο συναρτήσεις που έχουν παραγώγους $k$-τάξης στο Ι. Τότε ισχύει
\[
    (UV)^{(k)}=(-1)^kU\bar{V}^{(k)}+\binom{k}{j}(-1)^{k-j}U^{(j)}\bar{V}^{(k-j)}.
\]
Πραγματικά, για $k=1$ αυτό είναι φανερό ενώ για $k>1$ παίρνουμε
\begin{align*}
    \left[\sum_{j=1}^k (-1)^{j-1}U^{(k-j)}V^{(j-1)}\right]' &= \sum_{j=1}^k (-1)^{j-1}U^{(k-j+1)}V^{(j-1)}+\sum_{j=1}^k (-1)^{j-1}U^{(k-j)}V^{(j)} \\
    &= \sum_{j=1}^k (-1)^{j-1}U^{(k-j+1)}V^{(j-1)}+\sum_{j=2}^{k+1} (-1)^{j-2}U^{(k-j+1)}V^{(j-1)} \\
    &= U^{(k)}V+\sum_{j=2}^k (-1)^{j-1}[U^{(k-j+1)}V^{(j-1)}-U^{(k-j+1)}V^{(j-1)}]+(-1)^kUV^{(k)} \\
    &= U^{(k)}V+(-1)^kUV^{(k)}.
\end{align*}
Εφαρμόζοντας τώρα τον παραπάνω τύπο για $U=u, V=a_k\bar{v}$ και για $k=1,\ldots,n$, έχουμε
\[
    \bar{v}L(u)=\bar{v}\sum_{k=1}^n(a_ku^{(k)})+a_0u = \sum_{k=1}^n(\bar{v}a_ku^{(k)})+(a_0\bar{v})u
\]
\[
    =u\sum_{k=1}^n(-1)^k(a_k\bar{v})^{(k)}+(a_0\bar{v})u + [uv]' = uL^*(\bar{v})+[uv]'.
\]
Μια άμεση συνέπεια του θεωρήματος 21 είναι το ακόλουθο συμπέρασμα.
\begin{Thewrhma}{Τύπος του \eng{Green}}
Αν $u$ και $v$ είναι δύο συναρτήσεις που έχουν συνεχείς παραγώγους $n$-τάξης στο Ι, τότε για κάθε $x_1, x_2\in I$ ισχύει
\[
    \int_{x_1}^{x_2}[\bar{v}L(u)-u\overline{L^*(v)}](x)dx=[uv](x_2)-[uv](x_1).
\]
\end{Thewrhma}
Το παρακάτω θεώρημα δίνει μια μέθοδο για τον υποβιβασμό της τάξης της ομογενούς γραμμικής διαφορικής εξίσωσης $(E_0)$, αν είναι γνωστή μια λύση της συζυγούς διαφορικής εξίσωσης $(E_0^*)$ που δεν μηδενίζεται πουθενά στο Ι.

\begin{Thewrhma}{}
Ας είναι $g$ μια λύση της συζυγούς διαφορικής εξίσωσης $(E_0^*)$ με $g(x)\neq 0$ για όλα τα $x\in I$. Τότε είναι μια λύση της ομογενούς γραμμικής διαφορικής εξίσωσης $(E_0)$ αν και μόνο αν η $y$ έχει παράγωγο $n$-τάξης στο Ι και είναι λύση της γραμμικής διαφορικής εξίσωσης $(n-1)$-τάξης
\[
    [yg]=c
\]
για κάποια σταθερά $c$.
\end{Thewrhma}
\textit{ΑΠΟΔΕΙΞΗ.} Επειδή η $g$ είναι μια λύση της $(E_0^*)$, θα είναι $L^*(g)=0$. Έτσι, αν $y$ είναι μια συνάρτηση με παράγωγο $n$-τάξης στο Ι, τότε θα είναι (θεώρημα 21)
\[
    [yg]'=\bar{g}L(y)-y\overline{L^*(g)}=\bar{g}L(y).
\]
Αν λοιπόν η $y$ είναι μια λύση της $(E_0)$, τότε $L(y)=0$ και άρα $[yg]'=0$ που σημαίνει ότι υπάρχει σταθερά $c$ τέτοια ώστε $[yg]=c$. Αντίστροφα, όταν $y$ είναι μια λύση της γραμμικής εξίσωσης $[yg]=c$, $c$ σταθερά, και έχει παράγωγο $n$-τάξης στο Ι, θα είναι $[yg]'=0$ και επομένως $L(y)=0$, δεδομένου ότι $g(x)\neq 0$ για κάθε $x\in I$.

\subsection{Αυτοσυζυγείς ομογενείς γραμμικές διαφορικές εξισώσεις δεύτερης τάξης}
Θα περιορισθούμε εδώ στην περίπτωση της ομογενούς γραμμικής διαφορικής εξίσωσης δεύτερης τάξης
\begin{equation*}
    (E_0)_2 \quad a_2y''+a_1y'+a_0y=0,
\end{equation*}
όπου θα υποτίθεται ότι οι συντελεστές $a_0, a_1$ και $a_2$ είναι πραγματικές συναρτήσεις (και βέβαια ότι, στο διάστημα Ι, η $a_2$ έχει συνεχή παράγωγο δεύτερης τάξης και η $a_1$ έχει συνεχή παράγωγο δεύτερης τάξης). Στην περίπτωση αυτή, αν $y$ είναι μια συνάρτηση με παράγωγο δεύτερης τάξης στο Ι, θα έχουμε
\begin{align*}
    L^*(y) &= (a_2\bar{y})''-(a_1\bar{y})'+a_0\bar{y} = (a_2''y+2a_1'y'+a_2y'')-(a_1'y+a_1y')+a_0y \\
    &= a_2y''+(2a_2'-a_1)y'+(a_2''-a_1'+a_0)y.
\end{align*}
Έτσι, η συζυγής διαφορική εξίσωση της $(E_0)_2$ είναι η
\begin{equation*}
    (E_0^*)_2 \quad a_2y''+(2a_2'-a_1)y'+(a_2''-a_1'+a_0)y=0.
\end{equation*}
Το παρακάτω θεώρημα δίνει μια ικανή και αναγκαία συνθήκη ώστε η $(E_0)_2$ να είναι αυτοσυζυγής.

\begin{Thewrhma}{}
Η $(E_0)_2$ είναι μια αυτοσυζυγής ομογενής γραμμική διαφορική εξίσωση αν και μόνο αν $a_2'=a_1$, δηλαδή αν και μόνο αν μπορεί να γραφεί ως εξής
\[
    (a_2y')'+a_0y=0.
\]
\end{Thewrhma}
\textit{ΑΠΟΔΕΙΞΗ.} Οι διαφορικές εξισώσεις $(E_0)_2$ και $(E_0^*)_2$ ταυτίζονται αν και μόνο αν $2a_2'-a_1=a_1$ και $a_2''-a_1'+a_0=a_0$, ή ισοδύναμα $a_1=a_2'$ και $a_2''-a_1'=0$. Οι σχέσεις αυτές είναι ισοδύναμες με την ισότητα $a_1=a_2'$. Τώρα, αν $a_1=a_2'$, τότε η $(E_0)_2$ γράφεται $a_2y''+a_1'y'+a_0y=0$ ή $(a_2y')'+a_0y=0$. Αντίστροφα, ας υποθέσουμε ότι η $(E_0)_2$ μπορεί να πάρει τη μορφή $(a_2y')'+a_0y=0$. Τότε αυτή γράφεται $a_2y''+a_2'y'+a_0y=0$ και άρα $a_1=a_2'$.

Το παρακάτω θεώρημα εξασφαλίζει ότι οποιαδήποτε ομογενής γραμμική διαφορική εξίσωση δεύτερης τάξης μπορεί να μετασχηματισθεί σε μια ισοδύναμη διαφορική εξίσωση της μορφής $(Py')'+Qy=0$, όπου P είναι μια θετική συνάρτηση με συνεχή παράγωγο στο Ι και $Q$ είναι μια συνεχής συνάρτηση στο Ι.

\begin{Thewrhma}{}
Ας θεωρήσουμε την ομογενή γραμμική διαφορική εξίσωση δεύτερης τάξης
\[
    A_2y''+A_1y'+A_0y=0,
\]
όπου $A_0, A_1, A_2$ είναι συνεχείς πραγματικές συναρτήσεις στο διάστημα Ι και $A_2(x)\neq 0$ για όλα τα $x\in I$. Ας είναι $x_0$ ένα σημείο του Ι και
\[
    P(x)=\exp\left[\int_{x_0}^x\frac{A_1(t)}{A_2(t)}dt\right], x\in I; \quad Q(x)=\frac{A_0(x)}{A_2(x)}\exp\left[\int_{x_0}^x\frac{A_1(t)}{A_2(t)}dt\right], x\in I.
\]
Τότε η παραπάνω διαφορική εξίσωση είναι ισοδύναμη με την εξίσωση
\[
    (Py')'+Qy=0.
\]
\end{Thewrhma}
ΑΠΟΔΕΙΞΗ. Παρατηρούμε ότι η συνάρτηση $P$ είναι θετική και έχει παράγωγο συνεχή στο $I$ και $P'=(A_2/A_0)'$. Επίσης, η $Q$ είναι συνεχής στο $I$ και $Q = (A_0/A_2)P'$. Η διαφορική μας εξίσωση γράφεται
\[ P''+P(A_1/A_2)'y'+P(A_0/A_2)y=0 \]
ή
\[ Py''+P'y'+Qy=0, \]
οπότε
\[ (Py')'+Qy=0. \]

Σύμφωνα με τον ορισμό μας, η συογενής γραμμική διαφορική εξίσωση ($E_S$) είναι αυτοσυζυγής όταν ταυτίζεται με την συζυγή της διαφορική εξίσωση. Για να έχει νόημα αυτός ο ορισμός υποθέσαμε ότι η συνάρτηση $a_1$ έχει συνεχή παράγωγο στο $I$ και η $a_2$ έχει συνεχή παράγωγο δεύτερης τάξης στο $I$. Αποδείξαμε δε ότι η ($E_0$) είναι αυτοσυζυγής αν και μόνο αν $a_2'=a_1$, δηλαδή αν και μόνο αν αυτή γράφεται στη μορφή $(a_2y')'+a_0y=0$. Μπορούμε να επεκτείνουμε την έννοια των αυτοσυζυγών ομογενών γραμμικών διαφορικών εξισώσεων δεύτερης τάξης ορίζοντας ότι η ($E_0$) είναι αυτοσυζυγής αν και μόνο αν η $a_2$ έχει συνεχή παράγωγο στο $I$ και $a_2'=a_1$, ή ισοδύναμα αν και μόνο αν αυτή γράφεται $(a_2y')'+a_0y=0$ όπου η $a_2$ έχει συνεχή παράγωγο στο $I$. Σύμφωνα με το θεώρημα 25, κάθε ομογενής γραμμική διαφορική εξίσωση δεύτερης τάξης μπορεί να μετασχηματισθεί σε μια ισοδύναμη αυτοσυζυγή (με τη γενικότερη έννοια) διαφορική εξίσωση.

\subsection{Παραδείγματα}

\begin{Paradeigma}{}
Να βρεθεί η συζυγής διαφορική εξίσωση της ομογενούς γραμμικής διαφορικής εξίσωσης
\[ x^2y''+7xy'+8y=0,\quad x>0 \]
και να επαληθευθεί η ταυτότητα του Lagrange.
\end{Paradeigma}
Η συζυγής διαφορική εξίσωση είναι
\[ (x^2y)''-(7xy)'+8y=0 \]
ή
\[ (x^2y''+4xy'+2y)-(7xy'+7y)+8y=0, \]
δηλαδή
\[ x^2y''-3xy'+3y=0. \]
Αν $u$ και $v$ είναι δύο συναρτήσεις που έχουν παραγώγους δεύτερης τάξης. Ενώ στο διάστημα $(0,\infty)$, τότε για όλα τα $x>0$ παίρνουμε
\begin{align*}
[vL(u) - uL^*(v)](x) &= v(x)[x^2u''(x) + 7xu'(x) + 8u(x)] \\
&\quad - u(x)[x^2v''(x) - 3xv'(x) + 3v(x)] \\
&= x^2[u''v - uv''](x) + 7x[u'v + uv'](x) + 5u(x)v(x)
\end{align*}
και
\begin{align*}
[uv']'(x) &= (u)[7xv(x)]' + u'(x)[x^2v(x)] - u(x)[x^2v(x)']' \\
&= (x^2[u'v - uv'])' + 5xu(x)v(x)' \\
&= x^2[u''(x)v(x) - u(x)v''(x)] + 2x[u'(x)v(x) - u(x)v'(x)] \\
&\quad + 5x[u'(x)v(x) + u(x)v'(x)] + 5u(x)v(x) \\
&= x^2[u''v - uv''](x) + 7x[u'v](x) + 3u(x)v'(x) + 5u(x)v(x)
\end{align*}
και έτσι επαληθεύεται η ταυτότητα του Lagrange.

\begin{Paradeigma}{}
Να επιλυθεί η ομογενής γραμμική διαφορική εξίσωση
\[ x^2y''+7xy'+8y=0, \quad x>0, \]
αφού διαπιστωθεί ότι $g(x)=x$, $x>0$ είναι μια λύση της συζυγούς διαφορικής εξίσωσης.
\end{Paradeigma}
Η συζυγής διαφορική εξίσωση είναι (Παράδειγμα 1) $x^2y''-3xy'+3y=0$, η οποία έχει τη λύση $g(x)=x \neq 0$, $x>0$. Αν $u$ είναι μια συνάρτηση με παράγωγο δεύτερης τάξης στο $(0,\infty)$, τότε για κάθε $x>0$ έχουμε
\[ [yg]'(x)=[y(x)g(x)]'=[x^2y'(x)]'+[x^2g'(x)]y(x)-[x^2g(x)]'y'(x)=x^3y'(x)+4x^2y(x). \]
Ας είναι $c$ μια αυθαίρετη σταθερά και ας θεωρήσουμε τη γραμμική διαφορική εξίσωση πρώτης τάξης $[yg]'=c$, δηλαδή την εξίσωση
\[ x^3y'+4x^2y=c. \]
Αυτή γράφεται $(x^4y)'=c$ και άρα όλες οι λύσεις της δίνονται απ'τον τύπο $x^4y=c+C$, όπου $c$ είναι μια αυθαίρετη σταθερά. Εφαρμόζοντας το θεώρημα 23, συμπεραίνουμε ότι οι λύσεις της διαφορικής μας εξίσωσης είναι
\[ y(x) = cx^{-3}+Cx^{-4}, \quad x>0 \]
για τις διάφορες τιμές των αυθαίρετων σταθερών $c$ και $C$.\\\\
\begin{Paradeigma}{Ν' αποδειχθεί ότι η διαφορική εξίσωση του Legendre}
\[ (1-x^2)y''-2xy'+n(n+1)y=0, \quad x\in(-1,1) \quad (n \text{ σταθερά}) \]
είναι αυτοσυζυγής.
\end{Paradeigma}
Έχουμε $(1-x^2)' = -2x$ για $x \in (-1,1)$ και άρα η διαφορική εξίσωση είναι αυτοσυζυγής (θεώρημα 24). Αυτή γράφεται
\[ [(1-x^2)y']'+n(n+1)y=0. \]

\begin{Paradeigma}{Η ομογενής γραμμική διαφορική εξίσωση $x^2y''-2xy'+2y=0,\ x>0$ να γραφεί στη μορφή $(Py')'+Qy=0$, όπου P να είναι μια θετική και με συνεχή παράγωγο συνάρτηση στο διάστημα $(0,\infty)$ και Q να είναι μια συνεχής πραγματική συνάρτηση στο $(0,\infty)$.}
\end{Paradeigma}
Παίρνουμε
\[ P(x)=\exp\left(\int\frac{x^2-2x}{x^2}dx\right)=\exp(-2\log x) = \frac{1}{x^2}, \ x>0 \]
και
\[ Q(x)=\frac{2}{x^2}\exp\left(\int\frac{x^2-2x}{x^2}dx\right)=\frac{2}{x^4}, \ x>0. \]
Οι συναρτήσεις P και Q ικανοποιούν τις παραπάνω απαιτήσεις και η διαφορική μας εξίσωση γράφεται (θεώρημα 25) στη μορφή $(Py')'+Qy=0$, δηλαδή
\[ \left(\frac{1}{x^2}y'\right)'+\frac{2}{x^4}y=0. \]

\subsection{Ασκήσεις}
\begin{askhseis}
\item Να βρεθεί η συζυγής διαφορική εξίσωση για καθεμιά απ'τις παρακάτω ομογενείς γραμμικές διαφορικές εξισώσεις:
    \begin{rlist}
        \item $x^2y''+3xy'+3y=0$.
        \item $(2x+1)y''+x^2y'+y=0$.
        \item $x^2y'''+x^2y''+xy'+y=0$.
        \item $y^{(4)}+xy'''+x^2y''+x^3y'+x^4y=0$.
    \end{rlist}
Για καθεμιά απ'τις εξισώσεις (iii), (iv) να επαληθευθεί η ταυτότητα του Lagrange.
\item Να επιλυθεί η ομογενής γραμμική διαφορική εξίσωση
\[ x^3y''+(2x^3+7x)y'+(8x^2+8)y=0, \]
αφού βρεθεί με δοκιμή μια απλή λύση της συζυγούς διαφορικής εξίσωσης.

\item Ν' αποδειχθεί ότι καθεμιά απ'τις παρακάτω ομογενείς γραμμικές διαφορικές εξισώσεις είναι αυτοσυζυγής:
    \begin{rlist}
        \item $x^4y''+3x^3y'+y=0$.
        \item $(\sin x)y''+(\cos x)y'+2y=0$.
        \item $\frac{x+1}{x}y'' - \frac{1}{x^2}y' + \frac{1}{x^4}y=0$.
    \end{rlist}
Επιπλέον, να γραφούν αυτές στη μορφή $(Py')'+Qy=0$.

\item Να μετασχηματισθεί καθεμιά απ'τις παρακάτω ομογενείς γραμμικές διαφορικές εξισώσεις σε ισοδύναμη εξίσωση της μορφής $(Py')'+Qy=0$:
    \begin{rlist}
        \item $(x^4+x^2)y''+2x^3y'+3y=0$.
        \item $y''-(\tan x)y'+y=0$.
        \item $xy''+(\log x)y'+xy=0$.
    \end{rlist}
\end{askhseis}

\section{Ομογενείς γραμμικές διαφορικές εξισώσεις δεύτερης τάξης - Θεωρήματα διαχωρισμού και σύγκρισης του Sturm - Προβλήματα ιδιοτιμών}

Στο εδάφιο αυτό θ' ασχοληθούμε με ομογενείς γραμμικές διαφορικές εξισώσεις δεύτερης τάξης. Θα δώσουμε μερικά συμπεράσματα (θεωρήματα 26,27 και 28), τα οποία θα χρησιμοποιήσουμε για να οδηγηθούμε στο θεώρημα διαχωρισμού του \eng{Sturm} (θεώρημα 29) και στο θεώρημα σύγκρισης του \eng{Sturm} (θεώρημα 30). Στη συνέχεια, θα δώσουμε την έννοια του προβλήματος συνοριακών τιμών και θ' ασχοληθούμε ιδιαίτερα με τα προβλήματα ιδιοτιμών, θ' αποδείξουμε ότι (θεώρημα 31) οι ιδιοσυναρτήσεις που αντιστοιχούν στην ίδια ιδιοτιμή ενός προβλήματος ιδιοτιμών είναι μονόσημα ορισμένες κατά προσέγγιση σταθερού μη μηδενικού παράγοντα και ότι δύο ιδιοσυναρτήσεις που αντιστοιχούν σε διαφορετικές ιδιοτιμές είναι ορθογώνιες (ως προς κάποια συνάρτηση βάρους). Τέλος, θα παραθέσουμε μερικά παραδείγματα και θα προτείνουμε για λύση ορισμένες ασκήσεις.

\subsection{Ρίζες των λύσεων. Το θεώρημα διαχωρισμού του Sturm στο θεώρημα σύγκρισης του Sturm}
θα θεωρήσουμε εδώ την ομογενή γραμμική διαφορική εξίσωση δεύτερης τάξης
\begin{equation}
(Py')'+Qy=0, \tag{Δ}
\end{equation}
όπου θα υποτίθεται ότι $P$ είναι μια θετική συνάρτηση που έχει συνεχή παράγωγο στο διάστημα $Ι$ και $Q$ είναι μια συνεχής πραγματική συνάρτηση στο Ι (Ας υπενθυμίσουμε ότι, σύμφωνα με το θεώρημα 25, κάθε ομογενής γραμμική διαφορική εξίσωση δεύτερης τάξης, που οι συντελεστές της είναι συνεχείς πραγματικές συναρτήσεις στο $Ι$ και ο συντελεστής του $y''$ δεν μηδενίζεται πουθενά στο $Ι$, μπορεί να μετασχηματισθεί σε μια ισοδύναμη διαφορική εξίσωση της μορφής (Δ) με $P$ και $Q$ που ικανοποιούν τις παραπάνω απαιτήσεις).

Θα δώσουμε πρώτα μερικά βασικά συμπεράσματα. Για ν' αποδείξουμε το πρώτο συμπέρασμά μας θα χρησιμοποιήσουμε το θεώρημα \eng{Bolzano-Weierstrass} σύμφωνα με το οποίο κάθε φραγμένο και μη πεπερασμένο σύνολο πραγματικών αριθμών έχει ένα τουλάχιστον σημείο συσσώρευσης (αν $Ε$ είναι ένα σύνολο πραγματικών αριθμών, τότε ένα σημείο $x_0 \in \mathbb{R}$ λέμε ότι είναι ένα σημείο συσσώρευσης του Ε αν και μόνο αν υπάρχει μια ακολουθία $\{x_v\}_{v=1,2,...}$ σημείων του $E \setminus \{x_0\}$ με $\lim_{v\to\infty} x_v = x_0$).

\begin{Thewrhma}{26}
Ας είναι $y$ μια λύση της ομογενούς γραμμικής διαφορικής εξίσωσης (Δ) και $J$ ένα συμπαγές υποδιάστημα του $Ι$. Αν η y μηδενίζεται σ' άπειρα σημεία του $J$, τότε αυτή μηδενίζεται σ' ολόκληρο το $Ι$.
\end{Thewrhma}

\noindent\textbf{ΑΠΟΔΕΙΞΗ.} Ας υποθέσουμε ότι η λύση y μηδενίζεται σ' άπειρα σημεία του J. Τότε το σύνολο των ριζών αυτής στο $J$ είναι ένα μη πεπερασμένο και φραγμένο υποσύνολο της πραγματικής ευθείας και επομένως, σύμφωνα με το θεώρημα \eng{Bolzano-Weierstrass}, αυτό έχει ένα τουλάχιστον σημείο συσσώρευσης, δηλαδή, υπάρχουν $x_0 \in J$ και μια ακολουθία $\{x_v\}_{v=1,2,...}$ ριζών της y έτσι ώστε $x_v \neq x_0$ για όλα τα $v=1,2,...$ και $\lim_{v\to\infty} x_v = x_0$. Επειδή η y είναι συνεχής στο σημείο $x_0$, θα έχουμε
\[ y(x_0) = \lim_{x\to x_0} y(x) = \lim_{v\to\infty} y(x_v) = 0. \]
Τώρα, η λύση y είναι παραγωγίσιμη στο $x_0$ και έτσι παίρνουμε
\[ y'(x_0) = \lim_{x\to x_0} \frac{y(x)-y(x_0)}{x-x_0} = \lim_{v\to\infty} \frac{y(x_v)-y(x_0)}{x_v-x_0} = 0. \]
Έχουμε λοιπόν ότι η λύση y πληροί τις αρχικές συνθήκες $y(x_0)=y'(x_0)=0$ και άρα (θεώρημα 1) $y(x)=0$ για όλα τα $x \in I$.

\vspace{5mm}
Το θεώρημα 26 εξασφαλίζει ότι κάθε λύση y με $y \not\equiv 0$ της εξίσωσης (Δ) μπορεί να έχει μόνο πεπερασμένο αριθμό ριζών σ' ένα συμπαγές υποδιάστημα του Ι.

\begin{Thewrhma}{27 (Τύπος του \eng{Abel})}
Αν $y_1$ και $y_2$ είναι δύο λύσεις της ομογενούς γραμμικής διαφορικής εξίσωσης (Δ), τότε
\[ P(x)[y_1(x)y_2'(x)-y_1'(x)y_2(x)]=k \text{ για κάθε } x \in I, \]
όπου k είναι μια σταθερά.
\end{Thewrhma}

\noindent\textbf{ΑΠΟΔΕΙΞΗ.} Επειδή οι συναρτήσεις $y_1, y_2$ είναι λύσεις της (Δ), είναι
\[ (Py_1')'+Qy_1 = 0 \quad \text{και} \quad (Py_2')'+Qy_2=0 \]
και έτσι παίρνουμε
\[ y_1(Py_2')' - y_2(Py_1')' = y_1(-Qy_2)-y_2(-Qy_1)=0. \]
Θεωρούμε ένα $x_0 \in I$ και ολοκληρώνουμε από $x_0$ μέχρι x, για τυχόν $x \in I$, οπότε έχουμε
\[ \int_{x_0}^x y_1(t)[P(t)y_2'(t)]'dt - \int_{x_0}^x y_2(t)[P(t)y_1'(t)]'dt = 0 \]
ή
\[ y_1(t)[P(t)y_2'(t)] \Big|_{x_0}^x - \int_{x_0}^x P(t)y_1'(t)y_2'(t)dt - y_2(t)[P(t)y_1'(t)] \Big|_{x_0}^x + \int_{x_0}^x P(t)y_2'(t)y_1'(t)dt=0. \]
Έτσι, προκύπτει ότι για όλα τα $x \in I$
\[ P(x)[y_1(x)y_2'(x)-y_1'(x)y_2(x)]=P(x_0)[y_1(x_0)y_2'(x_0)-y_1'(x_0)y_2(x_0)]. \]
Το δεύτερο μέλος της παραπάνω σχέσης είναι μια σταθερά, το οποίο δηλώνει ότι η συνάρτηση $P(y_1y_2'-y_1'y_2)$ είναι σταθερή.

\begin{Thewrhma}{28}
Ας είναι $y_1, y_2$ δύο λύσεις της ομογενούς γραμμικής διαφορικής εξίσωσης (Δ). Τότε:
    \begin{rlist}
        \item Αν οι $y_1, y_2$ έχουν μιά κοινή ρίζα, τότε αυτές είναι γραμμικά εξαρτημένες.
        \item Αν $y_1 \not\equiv 0$ και $y_2 \not\equiv 0$ και οι $y_1, y_2$ είναι γραμμικά εξαρτημένες, τότε κάθε ρίζα της $y_1$ είναι επίσης ρίζα της $y_2$.
    \end{rlist}
\end{Thewrhma}

\noindent\textbf{ΑΠΟΔΕΙΞΗ.} (i) Ας υποθέσουμε ότι οι λύσεις $y_1, y_2$ έχουν μια κοινή ρίζα $x_0 \in I$. Σύμφωνα με το θεώρημα 27, υπάρχει μια σταθερά c έτσι ώστε για όλα τα $x \in I$
\[ P(x)[y_1(x)y_2'(x)-y_1'(x)y_2(x)]=c. \]
Για $x=x_0$ ο τύπος αυτός δίνει $c=0$ και άρα έχουμε
\[ P(x)[y_1(x)y_2'(x)-y_1'(x)y_2(x)]=0 \text{ για όλα τα } x \in I. \]
Επειδή έχει υποτεθεί ότι $P(x)>0$ για κάθε $x \in I$, θα είναι
\[ W(y_1,y_2)(x)=y_1(x)y_2'(x)-y_1'(x)y_2(x)=0 \text{ για κάθε } x \in I, \]
το οποίο σημαίνει (θεώρημα 4) ότι οι λύσεις $y_1, y_2$ είναι γραμμικά εξαρτημένες.

(ii) Ας υποθέσουμε ότι οι λύσεις $y_1, y_2$ δεν είναι τετριμμένες (δηλαδή $y_1 \not\equiv 0$ και $y_2 \not\equiv 0$) και ότι είναι γραμμικά εξαρτημένες. Τότε
\[ c_1y_1(x)+c_2y_2(x)=0 \text{ για όλα τα } x \in I, \]
όπου $c_1, c_2$ είναι σταθερές όχι και οι δύο μηδέν. Αν $c_2=0$, τότε $c_1y_1(x)=0$ για όλα τα $x \in I$, που σημαίνει ότι $c_1=0$ δεδομένου ότι $y_1 \not\equiv 0$. Έτσι, αναγκαστικά πρέπει $c_2 \neq 0$. Αν λοιπόν $x_0 \in I$ είναι μια ρίζα της $y_1$, τότε $c_2y_2(x_0)=0$ και άρα $y_2(x_0)=0$, δηλαδή το $x_0$ είναι επίσης μια ρίζα της $y_2$.

Το παρακάτω θεώρημα αποδεικνύει ότι οι ρίζες της μιας από δύο γραμμικά ανεξάρτητες πραγματικές λύσεις της (Δ) διαχωρίζουν τις ρίζες της άλλης λύσης.

\begin{Thewrhma}{29 (Θεώρημα διαχωρισμού του \eng)}
Ας είναι $y_1, y_2$ δύο γραμμικά ανεξάρτητες πραγματικές λύσεις της ομογενούς γραμμικής διαφορικής εξίσωσης (Δ). Μεταξύ δύο οποιωνδήποτε διαδοχικών ριζών της $y_1$ υπάρχει ακριβώς μία ρίζα της $y_2$.
\end{Thewrhma}
\noindent\textbf{ΑΠΟΔΕΙΞΗ.} Ας είναι $x_1, x_2$ δύο διαδοχικές ρίζες της λύσης $y_1$, δηλαδή $y_1(x_1)=y_1(x_2)=0$ και $y_1(x)\neq 0$ για κάθε $x \in (x_1,x_2)$. Όπως υποτίθεται ότι $x_1<x_2$. Θα δείξουμε ότι η λύση $y_2$ έχει ακριβώς μια ρίζα στο διάστημα $(x_1,x_2)$.

Ας υποθέσουμε ότι η λύση $y_2$ δεν έχει καμιά ρίζα στο $(x_1,x_2)$. Το θεώρημα 28 εξασφαλίζει ότι $y_2(x_1)\neq 0$ και $y_2(x_2)\neq 0$. Έτσι, μπορεί να ορισθεί η συνάρτηση $y_1/y_2$ στο κλειστό διάστημα $[x_1,x_2]$. Η συνάρτηση αυτή είναι συνεχής στο $[x_1,x_2]$, παραγωγίσιμη στο $(x_1,x_2)$ και μηδενίζεται στα άκρα $x_1, x_2$. Επομένως (θεώρημα Rolle) υπάρχει ένα σημείο $\xi \in (x_1,x_2)$ με
\[ (y_1/y_2)'(\xi)=0. \]
Η τελευταία ισότητα γράφεται
\[ W(y_1,y_2)(\xi)/y_2^2(\xi)=0 \]
και έτσι $W(y_1,y_2)=0$, το οποίο είναι (θεώρημα 4) ένα άτοπο. Έτσι, η λύση $y_2$ έχει μια τουλάχιστον ρίζα στο διάστημα $(x_1,x_2)$.

Πάλι, ας υποθέσουμε ότι η $y_2$ έχει περισσότερες από μια ρίζες στο $(x_1,x_2)$. Ας είναι $x_3^*, x_4^*$ δύο διαδοχικές ρίζες της $y_2$ με $x_1<x_3^*<x_4^*<x_2$. Τότε, σύμφωνα με τα παραπάνω, η λύση $y_1$ θα έχει μια τουλάχιστον ρίζα στο διάστημα $(x_3^*,x_4^*)$. Έτσι, η $y_1$ έχει τις ρίζες $x_1,x_2,x_5$ με $x_1<x_5<x_2$, που έρχεται σ' αντίθεση με το γεγονός ότι οι ρίζες $x_1, x_2$ είναι διαδοχικές.

\vspace{5mm}
Πριν προχωρήσουμε στο θεώρημα σύγκρισης του \eng{Sturm}, θα δώσουμε, χωρίς αποδείξεις, δύο συμπεράσματα που αναφέρονται στον αριθμό των ριζών των (μη μηδενικών) λύσεων της ομογενούς γραμμικής διαφορικής εξίσωσης (Δ).
\begin{Alist}
    \item Αν η συνάρτηση $Q$ είναι μη θετική στο διάστημα I, τότε κάθε μη μηδενική λύση της ομογενούς γραμμικής διαφορικής εξίσωσης (Δ) έχει το πολύ μια ρίζα.
    \item Ας είναι $I=(0,\infty)$ και ας υποθέσουμε ότι η συνάρτηση $Q$ είναι μη αρνητική στο Ι. Αν
    \[ \int_1^\infty (1/P(x)) \, dx = \int_1^\infty Q(x) \, dx = \infty, \]
    τότε κάθε μη μηδενική λύση της ομογενούς γραμμικής διαφορικής εξίσωσης (Δ) έχει άπειρες ρίζες.
\end{Alist}
Ας θεωρήσουμε και μια άλλη διαφορική εξίσωση, την ομογενή γραμμική διαφορική εξίσωση δεύτερης τάξης
\begin{equation}
    (\tilde{P}y')'+\tilde{Q}y=0, \tag{$\tilde{\Delta}$}
\end{equation}
όπου η πραγματική συνάρτηση $\tilde{Q}$ είναι συνεχής στο διάστημα Ι.

\begin{Thewrhma}{30 (Θεώρημα σύγκρισης του \eng{Sturm})}
Ας υποθέσουμε ότι $\tilde{Q}>Q$ και ας είναι y και $\tilde{y}$ πραγματικές λύσεις των (Δ) και ($\tilde{\Delta}$) αντίστοιχα. Αν $x_1,x_2$ είναι δύο διαδοχικές ρίζες της y, τότε η $\tilde{y}$ έχει μια τουλάχιστον ρίζα στο διάστημα $(x_1,x_2)$.
\end{Thewrhma}

\noindent\textbf{ΑΠΟΔΕΙΞΗ.} Ας είναι $x_1 < x_2$ δύο διαδοχικές ρίζες της λύσης y, τότε $y(x_1)=y(x_2)=0$ και $y(x)>0$ για κάθε $x \in (x_1,x_2)$, χωρίς βλάβη της γενικότητας, μπορούμε να υποθέσουμε ότι $y(x)>0$ για όλα τα $x \in (x_1,x_2)$. Ας υποθέσουμε ότι η $\tilde{y}$ δεν έχει ρίζες στο διάστημα $(x_1,x_2)$. Ας θεωρήσουμε, χωρίς πάλι μείωση της γενικότητας, την περίπτωση όπου $\tilde{y}(x)>0$ για κάθε $x \in (x_1,x_2)$. Έχουμε τώρα
\[ (\tilde{P}y')' - y(\tilde{P}\tilde{y}')' = (\tilde{Q}-Q)y\tilde{y}. \]
αλλά
\[ \tilde{y}(Py')' - y(\tilde{P}\tilde{y}')' = [y(\tilde{P}\tilde{y}')']', \]
και άρα
\[ [P(y'\tilde{y}-y\tilde{y}')]' = (\tilde{Q}-Q)y\tilde{y}. \]
Με ολοκλήρωση από $x_1$ μέχρι $x_2$ παίρνουμε
\[ P(x)[y'(x)\tilde{y}(x)-y(x)\tilde{y}'(x)]\bigg|_{x_1}^{x_2} = \int_{x_1}^{x_2} (\tilde{Q}(x)-Q(x))y(x)\tilde{y}(x)dx \]
ή
\[ P(x_2)y'(x_2)\tilde{y}(x_2) - P(x_1)y'(x_1)\tilde{y}(x_1) = \int_{x_1}^{x_2} (\tilde{Q}(x)-Q(x))y(x)\tilde{y}(x)dx. \]
Είναι $P(x)>0$. Επειδή $y(x)>0$ για $x \in (x_1,x_2)$, θα έχουμε $y'(x_1) \geq 0$. Επειδή $\tilde{y}(x)>0$ για $x \in (x_1,x_2)$, θα είναι $\tilde{y}(x_2)\geq 0$. Έτσι, $P(x_2)y'(x_2)\tilde{y}(x_2)\leq 0$. Μ' ένα ανάλογο τρόπο διαπιστώνουμε ότι $P(x_1)y'(x_1)\tilde{y}(x_1)\geq 0$. Άρα, το πρώτο μέλος στην παραπάνω ισότητα είναι μη θετικό, όμως το δεύτερο μέλος στην ισότητα αυτή είναι θετικό, γιατί $(\tilde{Q}-Q)y\tilde{y}>0$ στο $(x_1,x_2)$. Το άτοπο αυτό αποδεικνύει το θεώρημα.

Μια ενδιαφέρουσα συνέπεια του παραπάνω θεωρήματος είναι το εξής συμπέρασμα: ας υποθέσουμε ότι $\tilde{Q} \geq Q$ και ας είναι y και $\tilde{y}$ πραγματικές λύσεις των (Δ) και ($\tilde{\Delta}$) αντίστοιχα. Αν $x_0$ είναι μια κοινή ρίζα των y και $\tilde{y}$, $x_1$ είναι η επόμενη ρίζα της y και $\tilde{x}_1$ είναι η επόμενη ρίζα της $\tilde{y}$, τότε αναγκαστικά $\tilde{x}_1 < x_1$.

\subsection{Προβλήματα ιδιοτιμών}

Θ' αρχίσουμε δίνοντας την έννοια του προβλήματος συνοριακών τιμών για γραμμικές διαφορικές εξισώσεις δεύτερης τάξης.

Ας θεωρήσουμε τη γραμμική διαφορική εξίσωση δεύτερης τάξης
\begin{equation}
    a_2 y'' + a_1 y' + a_0 y = f \tag{H}
\end{equation}
όπου $a_2,a_1,a_0$ και f είναι συνεχείς συναρτήσεις σ' ένα διάστημα $[a,b]$ και $a_2(x) \neq 0$ για όλα τα $x\in [a,b]$. Αν $\alpha_{ij}, \beta_{ij}$ ($i,j=1,2$) και $c_i$ ($i=1,2$) είναι σταθερές, όπου για κάθε $i,j$ ($i,j=1,2$), ένα τουλάχιστον απ' τα $\alpha_{ij}, \beta_{ij}$ ($j=1,2$) είναι διάφορο του μηδενός, τότε η γραμμική διαφορική εξίσωση (Η) μαζί με τις συνοριακές συνθήκες
\begin{equation}
\begin{cases}
    \alpha_{11} y(a) + \alpha_{12} y'(a) + \beta_{11} y(b) + \beta_{12} y'(b) = c_1 \\
    \alpha_{21} y(a) + \alpha_{22} y'(a) + \beta_{21} y(b) + \beta_{22} y'(b) = c_2
\end{cases}
\tag{*}
\end{equation}
λέμε ότι αποτελούν το πρόβλημα συνοριακών τιμών (Η)-(*). Μια λύση της διαφορικής εξίσωσης (Η) που πληροί τις συνοριακές συνθήκες (*), λέμε ότι είναι λύση του προβλήματος συνοριακών τιμών (Η)-(*). Μια ενδιαφέρουσα περίπτωση προβλημάτων συνοριακών τιμών για τη γραμμική διαφορική εξίσωση (Η) είναι εκείνη όπου οι συνοριακές συνθήκες έχουν τη μορφή
\[ \alpha_1 y(a) + \alpha_2 y'(a) = a, \quad \beta_1 y(b) + \beta_2 y'(b) = b, \]
όπου $\alpha_1, \alpha_2, \beta_1, \beta_2, a$ και b είναι σταθερές και $|\alpha_1|+|\alpha_2|>0$, $|\beta_1|+|\beta_2|>0$. Μια παραπέρα ειδική περίπτωση προβλημάτων συνοριακών τιμών είναι αυτά που αναφέρονται στην ομογενή γραμμική διαφορική εξίσωση δεύτερης τάξης
\begin{equation}
    a_2 y'' + a_1 y' + a_0 y = 0 \tag{H$_0$}
\end{equation}
και στα οποία οι συνοριακές συνθήκες είναι της μορφής
\begin{equation}
\begin{cases}
    \alpha_1 y(a) + \alpha_2 y'(a) = 0, \\
    \beta_1 y(b) + \beta_2 y'(b) = 0,
\end{cases}
\tag{**}
\end{equation}
όπου $\alpha_1, \alpha_2, \beta_1$ και $\beta_2$ είναι πάλι σταθερές με $|\alpha_1|+|\alpha_2|>0$ και $|\beta_1|+$
$|\beta_2|>0$. Προβλήματα συνοριακών τιμών αυτής της τελευταίας μορφής θα συναντήσουμε παρακάτω. Ας σημειώσουμε, τέλος, ότι το πρόβλημα συνοριακών τιμών (Η)-(*) μπορεί και να μην έχει λύσεις· επίσης, αυτό μπορεί να έχει λύση όχι όμως αναγκαστικά μοναδική. Ακόμα, ας παρατηρήσουμε ότι το πρόβλημα (Η$_0$)-(**) ή έχει μόνο τη μηδενική λύση ή έχει άπειρες λύσεις.

Ας θεωρήσουμε, τώρα, την ομογενή γραμμική διαφορική εξίσωση δεύτερης τάξης
\begin{equation}
    (py')' + (q+\lambda r)y = 0, \tag{S}
\end{equation}
όπου p,q και r είναι πραγματικές συναρτήσεις σ' ένα διάστημα $[a,b]$, η p είναι θετική και έχει συνεχή παράγωγο στο $[a,b]$, η q είναι συνεχής στο $[a,b]$ και η r είναι θετική και συνεχής στο $[a,b]$, και $\lambda$ είναι μια πραγματική παράμετρος (ανεξάρτητη της μεταβλητής x). Ας θεωρήσουμε ακόμα τις συνοριακές συνθήκες
\begin{equation}
    \alpha_1 y(a) + \alpha_2 y'(a) = 0, \quad \beta_1 y(b) + \beta_2 y'(b) = 0, \tag{C}
\end{equation}
όπου $\alpha_1, \alpha_2, \beta_1$ και $\beta_2$ είναι πραγματικές σταθερές με $|\alpha_1|+|\alpha_2|>0$ και $|\beta_1|+|\beta_2|>0$.

Για μια δέ τιμή της παραμέτρου $\lambda$, το πρόβλημα συνοριακών τιμών \eng{(S)-(C)} έχει τη μηδενική λύση. Μια τιμή της παραμέτρου $\lambda$, για την οποία το πρόβλημα συνοριακών τιμών \eng{(S)-(C)} έχει και μη μηδενικές λύσεις, θα λέμε ότι είναι μια \textbf{ιδιοτιμή} του \eng{(S)-(C)}. Επίσης, αν $\lambda_0$ είναι μια ιδιοτιμή του προβλήματος συνοριακών τιμών \eng{(S)-(C)}, τότε μια μη μηδενική λύση του \eng{(S)-(C)} για $\lambda=\lambda_0$, θα λέμε ότι είναι μια \textbf{ιδιοσυνάρτηση} του \eng{(S)-(C)} αντίστοιχη της ιδιοτιμής $\lambda_0$. Ακόμα, το πρόβλημα συνοριακών τιμών \eng{(S)-(C)} θα λέμε ότι είναι ένα \textbf{πρόβλημα ιδιοτιμών}.

Θα δώσουμε τώρα ένα ορισμό: δύο συναρτήσεις f και g ορισμένες στο διάστημα $[a,b]$ θα λέμε ότι είναι \textbf{ορθογώνιες} στο $[a,b]$ \textbf{ως προς τη συνάρτηση βάρους r} αν και μόνο αν
\[ \int_a^b r(x)f(x)g(x)dx=0. \]

\begin{Thewrhma}{31}
Ας θεωρήσουμε το πρόβλημα ιδιοτιμών \eng{(S)-(C)}. Τότε:
\begin{rlist}
    \item Αν $y_1$ είναι μια ιδιοσυνάρτηση που αντιστοιχεί σε κάποια ιδιοτιμή, τότε όλες οι ιδιοσυναρτήσεις που αντιστοιχούν στην ίδια

ιδιοτιμή είναι ακριβώς οι $c y_0$, όπου $c \neq 0$ είναι αυθαίρετη σταθερά.
\item Δύο ιδιοσυναρτήσεις που αντιστοιχούν σε διαφορετικές ιδιοτιμές είναι ορθογώνιες στο $[a,b]$ ως προς τη συνάρτηση βάρους r.
\end{rlist}
\end{Thewrhma}

\noindent\textbf{ΑΠΟΔΕΙΞΗ.} (i) Ας είναι $y_0$ μια ιδιοσυνάρτηση του προβλήματος ιδιοτιμών \eng{(S)-(C)} αντίστοιχη μιας ιδιοτιμής $\lambda_0$ αυτού. Είναι φανερό ότι, για κάθε σταθερά $c \neq 0$, $c y_0$ είναι επίσης μια ιδιοσυνάρτηση του \eng{(S)-(C)} αντίστοιχη της $\lambda_0$. Τώρα, ας θεωρήσουμε μια άλλη ιδιοσυνάρτηση $\tilde{y}$ του \eng{(S)-(C)} που αντιστοιχεί στην $\lambda_0$. Σύμφωνα με το θεώρημα 27, η συνάρτηση $pW(y_0,\tilde{y}) = p(y_0\tilde{y}'-y_0'\tilde{y})$ είναι σταθερή στο $[a,b]$. Αλλά, έχουμε
\[ \alpha_1 y_0(a) + \alpha_2 y_0'(a) = 0 \quad \text{και} \quad \alpha_1 \tilde{y}(a) + \alpha_2 \tilde{y}'(a) = 0, \]
όπου οι σταθερές $\alpha_1, \alpha_2$ δεν είναι και οι δύο μηδενικές. Έτσι, μπορούμε να πάρουμε
\[ W(y_0,\tilde{y})(a)=0 \]
και επομένως $p(x)W(y_0,\tilde{y})(x)=0$ για όλα τα $x \in [a,b]$. Άρα (Θεώρημα 4) οι λύσεις $y_0$ και $\tilde{y}$ είναι γραμμικά εξαρτημένες, το οποίο συνεπάγεται ότι υπάρχει μια σταθερά $c \neq 0$ έτσι ώστε $\tilde{y}=cy_0$, αφού οι $y_0$ και $\tilde{y}$ είναι μη μηδενικές.

(ii) Ας είναι $y_1$ και $y_2$ δύο ιδιοσυναρτήσεις του προβλήματος ιδιοτιμών \eng{(S)-(C)} αντίστοιχες των ιδιοτιμών $\lambda_1$ και $\lambda_2$ αυτού αντίστοιχα, όπου $\lambda_1 \neq \lambda_2$. Τότε
\[ (py_1')' + (q+\lambda_1 r)y_1 = 0 \quad \text{και} \quad (py_2')' + (q+\lambda_2 r)y_2 = 0. \]
Πολλαπλασιάζοντας την πρώτη απ' τις ισότητες αυτές με $y_2$ και τη δεύτερη με $y_1$ και αφαιρώντας κατά μέλη έπειτα, παίρνουμε
\[ (py_1')'y_2 - (py_2')'y_1 + (\lambda_1-\lambda_2)ry_1 y_2 = 0 \]
και έτσι έχουμε
\[ (\lambda_1-\lambda_2)ry_1 y_2 = (py_2')'y_1 - (py_1')'y_2 = [p(y_2'y_1-y_1'y_2)]' \]
απ' όπου με ολοκλήρωση από a μέχρι b προκύπτει
\[ (\lambda_1-\lambda_2)\int_a^b r(x)y_1(x)y_2(x)dx = [p(y_1'(b)y_2(b)-y_1(b)y_2'(b))] - [p(a)(y_1'(a)y_2(a)-y_1(a)y_2'(a))]. \]
Αλλά, είναι
\[ \beta_1 y_1(b) + \beta_2 y_1'(b) = 0 \quad \text{και} \quad \beta_1 y_2(b) + \beta_2 y_2'(b) = 0, \]
όπου οι σταθερές $\beta_1$ και $\beta_2$ δεν είναι και οι δύο μηδέν. Έτσι, εύκολα βρίσκουμε
\[ y_1(b)y_2'(b) - y_1'(b)y_2(b) = 0. \]
Κατά τον ίδιο τρόπο προκύπτει ότι
\[ y_1(a)y_2'(a) - y_1'(a)y_2(a) = 0. \]
Άρα, έχουμε
\[ \int_a^b r(x) y_1(x) y_2(x) dx = 0, \]
αφού $\lambda_1 \neq \lambda_2$.

Δεν έχουμε κάνει λόγο για την ύπαρξη ιδιοτιμών του προβλήματος \eng{(S)-(C)}. Αποδεικνύεται ότι:

Το πρόβλημα ιδιοτιμών \eng{(S)-(C)} έχει άπειρες ιδιοτιμές· οι ιδιοτιμές του \eng{(S)-(C)} είναι ακριβώς οι όροι μιας ακολουθίας $\lambda_1 < \lambda_2 < \dots$, με $\lim_{n \to \infty} \lambda_n = \infty$, η οποία καλείται ακολουθία των ιδιοτιμών του \eng{(S)-(C)}.

Αν $\lambda_1 < \lambda_2 < \dots$ είναι η ακολουθία των ιδιοτιμών του \eng{(S)-(C)} και, για κάθε $n\in \{1,2,\dots\}$, $y_n$ είναι μια ιδιοσυνάρτηση του προβλήματος ιδιοτιμών \eng{(S)-(C)} αντίστοιχη της ιδιοτιμής $\lambda_n$ αυτού, τότε θα λέμε ότι η ακολουθία συναρτήσεων $y_n, n\in \{1,2,\dots\}$ είναι μια ακολουθία ιδιοσυναρτήσεων του \eng{(S)-(C)}. Απ' το θεώρημα 31 προκύπτει ότι, αν $y_n, n\in \{1,2,\dots\}$ είναι μια ακολουθία ιδιοσυναρτήσεων του προβλήματος ιδιοτιμών \eng{(S)-(C)}, τότε οι ιδιοσυναρτήσεις που αντιστοιχούν σε δύο διάφορες ακολουθίες $c_n y_n, n\in\{1,2,\dots\}$ μη μηδενικών αριθμών. Επίσης, το θεώρημα 31 εξασφαλίζει ότι:
\begin{center}
    \textit{Κάθε ακολουθία ιδιοσυναρτήσεων ($y_n, n=1,2,\dots$) του προβλήματος ιδιοτιμών \eng{(S)-(C)} είναι ορθογώνια ως προς τη συνάρτηση βάρους r με την έννοια ότι, για τυχαία θετική ακολουθία ακεραίων m και n με m$\neq$n, $y_m$ και $y_n$ είναι ορθογώνιες ως προς τη συνάρτηση βάρους r.}
\end{center}
Στην επίλυση του προβλήματος ιδιοτιμών \eng{(S)-(C)} αναφέρεται στην εύρεση της ακολουθίας των ιδιοτιμών του \eng{(S)-(C)} και μιας ακολουθίας ιδιοσυναρτήσεων αυτού.
\section{Παραδείγματα}

\begin{Paradeigma}{}
Ν' αποδειχθεί ότι μεταξύ δύο διαδοχικών ριζών της μιας απ' τις συναρτήσεις $y_1(x) = \sin 2x + \cos 2x, x \in \mathbb{R}$ και $y_2(x) = \sin 2x - \cos 2x, x \in \mathbb{R}$ υπάρχει ακριβώς μια ρίζα της άλλης.
\end{Paradeigma}
Έχουμε για $x \in \mathbb{R}$
\[ W(y_1,y_2)(x) = \begin{vmatrix} y_1(x) & y_2(x) \\ y_1'(x) & y_2'(x) \end{vmatrix} = \begin{vmatrix} \sin 2x + \cos 2x & \sin 2x - \cos 2x \\ 2\cos 2x - 2\sin 2x & 2\cos 2x + 2\sin 2x \end{vmatrix} = 4 \neq 0. \]
Έτσι (θεώρημα 8), η σχέση 
\[ W(y_1, y_2, y)/W(y_1,y_2)=0 \]
ορίζει μια ομογενή γραμμική διαφορική εξίσωση δεύτερης τάξης με δύο γραμμικά ανεξάρτητες λύσεις τις συναρτήσεις $y_1, y_2$. Η εξίσωση αυτή είναι η
\[ \frac{1}{4} \begin{vmatrix} y_1 & y_2 & y \\ y_1' & y_2' & y' \\ y_1'' & y_2'' & y'' \end{vmatrix} = 0 \]
ή (όπως προκύπτει μετά από πράξεις)
\[ y''+2y=0. \]
Αρκεί λοιπόν να εφαρμόσουμε το θεώρημα 29.

\begin{Paradeigma}{}
Ν' αποδειχθεί ότι κάθε πραγματική λύση της ομογενούς γραμμικής διαφορικής εξίσωσης
\[ y''+x^2y=0, \quad x>1 \]
έχει άπειρες ρίζες.
\end{Paradeigma}
Η συνάρτηση $y_1(x) = \sin x, x>1$ είναι μια λύση της διαφορικής εξίσωσης
\[ y''+y=0, \quad x>1. \]
Η λύση αυτή έχει τις διαδοχικές ρίζες $k\pi$, $(k+1)\pi$ για $k=1,2,\dots$. Έτσι, επειδή $x^2>1$ για $x>1$, το θεώρημα 30 εξασφαλίζει ότι, για οποιονδήποτε θετικό ακέραιο $k$, κάθε πραγματική λύση της διαφορικής μας εξίσωσης θα έχει μια τουλάχιστον ρίζα στο διάστημα $(k\pi, (k+1)\pi)$.
Άρα οι πραγματικές λύσεις της εξίσωσής μας έχουν άπειρες ρίζες.

\begin{Paradeigma}{}
Να επιλυθούν τα προβλήματα συνοριακών τιμών:
\begin{rlist}
\item $y''+y=x$, $x\in [0,\pi]$; $y(0)=2$, $y(\pi)=1$.
\item $y''+y=x$, $x\in [0,\pi/2]$; $y(0)=2$, $y(\pi/2)=1$.
\item $y''+y=x$, $x\in [0,\pi]$; $y(0)=2$, $y(\pi)=2$.
\item $y'''+y=0$, $x\in [0,\pi]$; $y(0)-2y'(0)=-2$, $y(\pi)+3y'(\pi)=3$.
\end{rlist}
\end{Paradeigma}
\lysh\ (i) Η γενική λύση της διαφορικής εξίσωσης είναι
\[ y(x) = c_1 \cos x + c_2 \sin x + x, \quad x \in [0, \pi], \]
όπου $c_1$ και $c_2$ είναι αυθαίρετες σταθερές. Απ' τις συνοριακές συνθήκες παίρνουμε $y(0)=c_1=2$ και $y(\pi)=c_1+\pi=1$, που οδηγεί σ' ένα άτοπο. Έτσι, το πρόβλημα συνοριακών τιμών δεν έχει λύσεις.

(ii) Η διαφορική εξίσωση έχει τη γενική λύση
\[ y(x) = c_1 \cos x + c_2 \sin x + x, \quad x \in [0, \pi/2], \]
όπου οι σταθερές $c_1$ και $c_2$ είναι αυθαίρετες. Οι συνοριακές συνθήκες δίνουν $y(0)=c_1=2$ και $y(\pi/2)=c_2+\pi/2=1$, και άρα $c_1=2$ και $c_2=1-\pi/2$. Επομένως, το πρόβλημα συνοριακών τιμών έχει τη μοναδική λύση
\[ y(x) = 2\cos x + (1-\pi/2)\sin x + x, \quad x \in [0, \pi/2]. \]
(iii) Οι λύσεις της διαφορικής εξίσωσης δίνονται απ' τον τύπο
\[ y(x) = c_1 \cos x + c_2 \sin x + x, \quad x \in [0, \pi], \]
όπου $c_1$ και $c_2$ είναι αυθαίρετες σταθερές. Οι συνοριακές συνθήκες γίνονται $y(0)=c_1=2$ και $y(\pi)=c_1+\pi=2$, απ' όπου προκύπτει $c_1=2$, ενώ η σταθερά $c_2$ δεν προσδιορίζεται απ' τις συνοριακές συνθήκες. Έτσι, το πρόβλημα συνοριακών τιμών έχει άπειρες λύσεις που δίνονται απ' τον τύπο
\[ y(x) = 2\cos x + c_2 \sin x + x, \quad x \in [0, \pi], \]
για τις διάφορες τιμές της σταθεράς c.

(iv) Η γενική λύση της διαφορικής εξίσωσης είναι
\[ y(x)=c_1 \cos 2x + c_2 \sin 2x, \quad x \in [0, \pi], \]
όπου $c_1$ και $c_2$ είναι αυθαίρετες σταθερές. Έχουμε $y'(x)=-2c_1 \sin 2x + 2c_2 \cos 2x$, $x \in [0, \pi]$ και έτσι οι συνοριακές συνθήκες δίνουν $y(0)-2y'(0) = c_1-4c_2 = -2$ και $y(\pi)+3y'(\pi)=c_1+6c_2=3$, απ' όπου προκύπτει ότι $c_1=0$ και $c_2=\frac{1}{2}$.  Άρα, το πρόβλημα συνοριακών τιμών έχει τη μοναδική λύση
\[ y(x) = \frac{1}{2} \sin 2x, \quad x \in [0, \pi]. \]

\begin{Paradeigma}{}
Να επιλυθούν τα προβλήματα συνοριακών τιμών:
\begin{rlist}
    \item $y''+y=0$, $x \in [0, \pi]$; $y(0)+y'(0)=0$, $y(\pi)+2y'(\pi)=0$.
    \item $y''+y=0$, $x \in [0, \pi]$; $y(0)+y'(0)=0$, $y(\pi)+y'(\pi)=0$.
\end{rlist}
\end{Paradeigma}
(i) Η γενική λύση της διαφορικής εξίσωσης είναι
\[ y(x) = c_1 \cos x + c_2 \sin x, \quad x \in [0, \pi], \]
όπου $c_1$ και $c_2$ είναι αυθαίρετες σταθερές. Οι συνοριακές συνθήκες δίνουν $c_1+c_2=0$ και $-c_1-2c_2=0$, απ' όπου προκύπτει $c_1=c_2=0$. Άρα, το πρόβλημα συνοριακών τιμών έχει μόνο τη μηδενική λύση.

(ii) Οι λύσεις του προβλήματος συνοριακών τιμών είναι
\[ y(x) = c_1 \cos x + c_2 \sin x, \quad x \in [0, \pi], \]
όπου οι σταθερές $c_1$ και $c_2$ πληρούν τη συνθήκη $c_1+c_2=0$. Έτσι, το πρόβλημα συνοριακών τιμών έχει και μη μηδενικές λύσεις και μάλιστα όλες οι λύσεις είναι
\[ y(x) = c \cos x + (1-c) \sin x, \quad x \in [0, \pi], \]
όπου c είναι μια αυθαίρετη σταθερά.

\begin{Paradeigma}{}
Να επιλυθεί το πρόβλημα ιδιοτιμών
\[ y''+\lambda y=0, \quad x \in [0, \pi]; \quad y(0)=0, y'(\pi)=0. \]
\end{Paradeigma}
Η διαφορική εξίσωση έχει τη γενική λύση
\[ y(x) = \begin{cases} c_1 e^{\sqrt{-\lambda}x} + c_2 e^{-\sqrt{-\lambda}x}, & \text{αν } \lambda < 0 \\ c_1+c_2x, & \text{αν } \lambda = 0 \\ c_1 \cos(\sqrt{\lambda}x) + c_2 \sin(\sqrt{\lambda}x), & \text{αν } \lambda > 0 \end{cases} \]
για $x \in [0, \pi]$, όπου $c_1$ και $c_2$ είναι αυθαίρετες σταθερές. Οι συνοριακές συνθήκες δίνουν
\[ \begin{cases} c_1+c_2=0 \text{ και } c_1 e^{\sqrt{-\lambda}\pi} - c_2 e^{-\sqrt{-\lambda}\pi} = 0, & \text{αν } \lambda<0 \\ c_1=0 \text{ και } c_2=0, & \text{αν } \lambda=0 \\ c_1=0 \text{ και } -c_1\sqrt{\lambda}\sin(\sqrt{\lambda}\pi) + c_2\sqrt{\lambda}\cos(\sqrt{\lambda}\pi)=0, & \text{αν } \lambda > 0 \end{cases} \]
Είναι φανερό ότι για $\lambda \le 0$ είναι $c_1=c_2=0$, και άρα $y(x)=0, x \in [0, \pi]$. Θεωρούμε, λοιπόν, την περίπτωση $\lambda > 0$. Τότε $c_1=0$ και $c_2 \cos(\sqrt{\lambda}\pi)=0$. Αν $c_2=0$, τότε y είναι η μηδενική λύση. Για $c_2 \neq 0$ έχουμε
\[ \cos(\sqrt{\lambda}\pi)=0, \]
δηλαδή
\[ \sqrt{\lambda}=n-1/2 \text{ για κάποιο θετικό ακέραιο } n, \]
οπότε
\[ y(x) = c_2 \sin[(n-1/2)x], \quad x \in [0, \pi]. \]
Έτσι, η ακολουθία των ιδιοτιμών είναι
\[ \lambda_n = (n-1/2)^2 \quad (n=1,2,\dots) \]
και μια ακολουθία ιδιοσυναρτήσεων είναι (για $c_2=1$)
\[ y_n(x) = \sin[(n-1/2)x], \quad x \in [0, \pi] \quad (n=1,2,\dots). \]

\begin{Paradeigma}{Να επιλυθεί το πρόβλημα ιδιοτιμών
\[ x^2y''+xy'+\lambda y=0, \quad x \in [1,e]; \quad y(1)=0, y(e)=0. \]
}
\end{Paradeigma}
\lysh\ Η διαφορική εξίσωση γράφεται
\[ (xy')'+\frac{\lambda}{x}y=0, \quad x \in [1,e]. \]
Έτσι, αυτή είναι της μορφής (S). Επίσης, παρατηρούμε ότι η εξίσωσή μας είναι μια διαφορική εξίσωση Euler. Επιλύοντας, με το γνωστό τρόπο, αυτή βρίσκουμε ότι οι λύσεις της δίνονται απ' τον τύπο
\[ y(x) = \begin{cases} c_1 x^{\sqrt{-\lambda}} + c_2 x^{-\sqrt{-\lambda}}, & \text{αν } \lambda < 0 \\ c_1+c_2\log x, & \text{αν } \lambda = 0 \\ c_1 \cos(\sqrt{\lambda}\log x) + c_2 \sin(\sqrt{\lambda}\log x), & \text{αν } \lambda > 0 \end{cases} \]
για $x \in [1,e]$, όπου $c_1$ και $c_2$ είναι αυθαίρετες σταθερές. Αν $\lambda < 0$, οι συνοριακές συνθήκες δίνουν $c_1+c_2=0$ και $c_1e^{\sqrt{-\lambda}}+c_2e^{-\sqrt{-\lambda}}=0$, οπότε $c_1=c_2=0$ που οδηγεί στη μηδενική λύση. Όταν $\lambda=0$, έχουμε $c_1=0$ και $c_1+c_2=0$, δηλαδή πάλι $c_1=c_2=0$ που σημαίνει ότι η λύση y είναι η μηδενική. Στη συνέχεια, ας υποθέσουμε ότι $\lambda > 0$. Τότε οι συνοριακές συνθήκες οδηγούν στις σχέσεις
\[ c_1=0 \text{ και } c_2 \sin(\sqrt{\lambda})=0. \]
Αν $\sin\sqrt{\lambda}=0$, δηλαδή $\sqrt{\lambda}=n\pi$ για κάποιο θετικό ακέραιο $n$, και $c_2 \neq 0$, έχουμε τη μη μηδενική λύση
\[ y(x) = c_2 \sin(n\pi \log x), \quad x \in [1,e]. \]
Άρα, η ακολουθία των ιδιοτιμών είναι
\[ \lambda_n = n^2\pi^2 \quad (n=1,2,\dots) \]
και μια ακολουθία ιδιοσυναρτήσεων είναι (για $c_2=1$)
\[ y_n(x) = \sin(n\pi \log x), \quad x \in [1,e] \quad (n=1,2,\dots). \]

\subsection{Ασκήσεις}
\begin{enumerate}
    \item Ν' αποδειχθεί ότι μεταξύ δύο διαδοχικών ριζών της συνάρτησης $y_1(x)=\sin x$, $x \in \mathbb{R}$ υπάρχει μια ακριβώς ρίζα της $y_2(x)=\sin x + \cos x$, $x \in \mathbb{R}$.
    \item Ν' αποδειχθεί ότι κάθε πραγματική λύση της διαφορικής εξίσωσης
    \[ y''+(x+1)y=0 \]
    έχει άπειρες θετικές ρίζες.
    \item Ας θεωρήσουμε την ομογενή γραμμική διαφορική εξίσωση
    \[ y''+qy=0, \]
    όπου q είναι μια συνεχής πραγματική συνάρτηση στο συμπαγές διάστημα $[a,b]$, τέτοια ώστε $0<m \le q(x) \le M$ για όλα τα $x \in [a,b]$. Αν είναι $y_1$ μια λύση αυτής που έχει δύο διαδοχικές ρίζες $x_1, x_2$ με $x_1 < x_2$. Ν' αποδειχθεί ότι
    \[ \frac{\pi}{\sqrt{M}} < x_2-x_1 < \frac{\pi}{\sqrt{m}}. \]
    \item Ν' αποδειχθεί ότι, αν q είναι μια συνεχής και αρνητική συνάρτηση σ' ένα διάστημα Ι, τότε κάθε λύση της ομογενούς γραμμικής διαφορικής εξίσωσης
    \[ y''+qy=0 \]
    έχει το πολύ μια ρίζα.
    \item Να επιλυθούν τα προβλήματα συνοριακών τιμών:
\begin{rlist}
    \item $y''-3y'+2y=e^x, \quad x \in [0,1]; \quad y(0)=0, y(1)=0.$
    \item $y''+9y=0, \quad x \in [0,\pi]; \quad y(0)=1, y'(\pi)=-1.$
    \item $y''-3y'+2y=0, \quad x \in [0,1]; \quad y(0)=0, y'(1)=0.$
    \item $y''-y=2e^x, \quad x \in [0,1]; \quad y(0)-2y'(0)=-2, 3y(1)-y'(1)=e.$
    \item $x^2y''-3xy'+3y=0, \quad x \in [1,2]; \quad y(1)-y'(1)=2, y(2)-2y'(2)=4.$
    \item $x^2y''-3xy'+3y=\log x, \quad x \in [1,2]; \quad y(1)=A, y(2)=B \quad (\text{A,B σταθερές}).$
\end{rlist}

\item Να επιλυθούν, για τις διάφορες τιμές των σταθερών A και B, τα παρακάτω προβλήματα συνοριακών τιμών:
\begin{rlist}
    \item $y''+16y=32x, \quad x \in [0,\pi]; \quad y(0)=A, y(\pi)=B.$
    \item $y''+16y=32x, \quad x \in [0,\pi]; \quad y(0)=A, y'(\pi)=B.$
    \item $y''+16y=32x, \quad x \in [0,\pi]; \quad y(0)+y'(0)=A, y(\pi)=B.$
\end{rlist}

\item Να επιλυθούν τα προβλήματα ιδιοτιμών:
\begin{rlist}
    \item $y''+(2+\lambda)y=0, \quad x \in [0,1]; \quad y(0)=0, y(1)=0.$
    \item $y''+\lambda y = 0, \quad x \in [0,1]; \quad y(0)=0, y(0)+y'(0)=0.$
    \item $y''-3y'+(3+\lambda)y=0, \quad x \in [0,\pi]; \quad y'(0)=0, y'(\pi)=0.$
    \item $y''+\lambda y=0, \quad x \in [0,\pi]; \quad y(0)=0, y(\pi)+y'(\pi)=0.$
\end{rlist}

\item Να επιλυθούν τα προβλήματα ιδιοτιμών:
\begin{rlist}
    \item $x^2y''+3xy'+\lambda y=0, \quad x \in [1,e]; \quad y(1)=0, y(e)=0.$
    \item $[(2+x^2)y']'+\lambda y=0, \quad x \in [-1,1]; \quad y(-1)=0, y(1)=0.$
    \item $(1+x)^2y''+2(1+x)y'+\lambda y=0, \quad x \in [0,1]; \quad y(0)=0, y(1)=0.$
\end{rlist}
\end{enumerate}

\full{\Alyta}
\fulltwoc{
    \Askhsh Δίνονται οι συναρτήσεις:
    \[ f(x) = \begin{cases} x^2, & \text{αν } 0 \le x \le 1 \\ 0, & \text{αν } -1 \le x < 0 \end{cases} \quad \text{και} \quad g(x) = \begin{cases} 0, & \text{αν } 0 \le x \le 1 \\ x^2, & \text{αν } -1 \le x < 0. \end{cases} \]
    Ν' αποδειχθεί ότι: (i) $W(f,g)(x)=0$ για κάθε $x \in [-1,1]$. (ii) Οι συναρτήσεις $f,g$ είναι γραμμικά ανεξάρτητες.\\\\
% This file continues the enumerate list from page 137
    \Askhsh Ας είναι $f$ μια συνάρτηση στο διάστημα $(0,\pi)$ που δεν μηδενίζεται σε n τουλάχιστον σημεία. Ν' αποδειχθεί ότι οι συναρτήσεις $f_k(x) = x^{k-1}f(x)$, $x \in (0,\pi) \quad (k=1,\dots,n)$ είναι γραμμικά ανεξάρτητες.\\\\
    \Askhsh Να επιλυθεί η μη ομογενής γραμμική διαφορική εξίσωση
    \[ (\sin^2 x)y''-2(\sin x \cos x)y'+(1+\cos^2 x)y = \sin^3 x, \quad x \in (0,\pi), \]
    αφού αποδειχθεί ότι $y_1(x)=\sin x$, $x \in (0,\pi)$ και $y_2(x)=x \sin x$, $x \in (0,\pi)$ είναι δύο γραμμικά ανεξάρτητες λύσεις της αντίστοιχης ομογενούς εξίσωσης.\\\\
    \Askhsh Να επιλυθεί το πρόβλημα αρχικών τιμών
    \[ (x^2+1)y''-2xy'+2y=(x^2+1)^2, \quad y(0)=0, y'(0)=1, \]
    με το δεδομένο ότι η αντίστοιχη ομογενής γραμμική εξίσωση δέχεται μια λύση $y_1$ της μορφής $y_1(x)=x+a$, $x \in \mathbb{R}$ (a σταθερά).\\\\
    \Askhsh Να επιλυθεί η ομογενής γραμμική διαφορική εξίσωση
    \[ y^{(5)}-y'- \frac{4}{x}y=0, \quad x>0 \]
    με τις αντικαταστάσεις $y=xz$, $z=w$.\\\\
    \Askhsh Να επιλυθεί το πρόβλημα αρχικών τιμών
    \[ x(1-2x \log x)y''+(1+4x^2 \log x)y'-(2+4x)y = e^{2x}(1-2x \log x)^2, \quad \frac{1}{2} \le x \le 1; \]
    \[ y\left(\frac{1}{2}\right)=\frac{e}{2}, \quad y'\left(\frac{1}{2}\right)=e(2+\log 2), \]
    αφού διαπιστωθεί ότι $y_1(x)=\log x$, $x \in [\frac{1}{2},1]$ είναι μια μερική λύση της αντίστοιχης ομογενούς γραμμικής διαφορικής εξίσωσης.\\\\
    \Askhsh Ν' αποδειχθεί ότι οι συναρτήσεις
    \[ y_1(x)=e^{-x^2/2}, \quad x \in \mathbb{R} \quad \text{και} \quad y_2(x) = e^{-x^2/2} \int_0^x e^{t^2/2}dt, \quad x \in \mathbb{R} \]
    αποτελούν ένα βασικό σύνολο λύσεων της ομογενούς γραμμικής διαφορικής εξίσωσης
    \[ y''+xy'+y=0, \quad x \in \mathbb{R}. \]
    Στη συνέχεια, να επιλυθεί το πρόβλημα αρχικών τιμών
    \[ y''+xy'+y=0; \quad y(0)=0, y'(0)=1. \]
% This file continues the enumerate list from page 138
    \Askhsh Με τον μετασχηματισμό $x=\tan\frac{t}{2}$ να επιλυθεί η ομογενής γραμμική διαφορική εξίσωση
    \[ (1+x^2)^2y''+2x(1+x^2)y'+4y=0, \quad x \in \mathbb{R}. \]
    \Askhsh Να επιλυθεί η ομογενής γραμμική διαφορική εξίσωση
    \[ x^2y''+(3x-x^2)y'+(1-x-e^{2x})y=0, \quad x>0, \]
    αφού βρεθεί μια λύση αυτής της μορφής $y=e^{ax}g(x)$, $x>0$, όπου a είναι σταθερά και $g(x)=x \int_1^x \frac{e^t}{t}dt, \quad x>0$.\\\\
    \Askhsh Να βρεθεί μια δύο φορές παραγωγίσιμη συνάρτηση f στο διάστημα $[0,2]$ τέτοια ώστε $f(0)=0, f'(0)=1$ και
    $f''(x)-f(x)=0$ για $x \in [0,1]$; $f''(x)-9f(x)=0$ για $x \in [1,2]$.\\\\
    \Askhsh Δίνεται η γραμμική διαφορική εξίσωση
    \[ y''+\omega^2 y = A \cos \omega x, \quad x \ge 0, \]
    όπου $\omega$ και A είναι θετικές σταθερές. Ν' αποδειχθεί ότι για κάθε λύση y αυτής είναι $\lim_{x\to\infty} |y(x)|=\infty$. Στη συνέχεια, να βρεθεί η λύση y με
    \[ y(0)=0, \quad y'(0)=1. \]
    \Askhsh Δίνεται η ομογενής γραμμική διαφορική εξίσωση
    \[ y''+4xy'+q(x)y=0, \]
    όπου q είναι μια συνεχής συνάρτηση σ' ένα διάστημα Ι. Αν οι συναρτήσεις $y(x)$, $x \in I$ και $xu(x)$, $x \in I$ είναι λύσεις αυτής και $u(0)=1$ (υποτίθεται ότι $0 \in I$), τότε να βρεθούν οι συναρτήσεις $u$ και $q$.\\\\
    \Askhsh Δίνεται η γραμμική διαφορική εξίσωση
    \[ y''+a_1y'+a_0y=b(x), \quad x \ge 0, \]
    όπου $a_1, a_0$ είναι σταθερές και $b$ είναι μια συνεχής συνάρτηση στο $[0,\infty)$. Αν είναι $r_1,r_2$ οι ρίζες του πολυωνύμου $r^2+a_1r+a_0$ με $r_1 \ne r_2$ και $\text{Re } r_i <0, i=1,2$. (i) Αν η b είναι φραγμένη, ν' αποδειχθεί ότι κάθε λύση είναι φραγμένη. (ii) Αν $\lim_{x\to\infty} b(x)=0$, ν' αποδειχθεί ότι για κάθε λύση y είναι $\lim_{x\to\infty} y(x)=0$.\\\\
% This file continues the enumerate list from page 139
    \Askhsh Αν $\{y_1,\dots,y_n\}$ και $\{\phi_1,\dots,\phi_n\}$ είναι δύο βασικά σύνολα λύσεων της ομογενούς γραμμικής διαφορικής εξίσωσης $(H_0)$, ν' αποδειχθεί ότι
    \[ W(\phi_1,\dots,\phi_n) = c W(y_1,\dots,y_n), \]
    όπου c είναι μια σταθερά.\\\\

    \Askhsh (i) Δίνεται η ομογενής γραμμική διαφορική εξίσωση
    \begin{equation*} \tag{*}
        y''+py'+qy=0,
    \end{equation*}
    όπου p,q είναι συνεχείς συναρτήσεις σ' ένα διάστημα Ι και η p έχει συνεχή παράγωγο στο Ι. Ας είναι $x_0$ ένα σημείο του Ι. Ν' αποδειχθεί ότι η αντικατάσταση
    \[ y(x)=u(x)\exp\left[-\frac{1}{2}\int_{x_0}^x p(t)dt\right], \quad x \in I \]
    μετασχηματίζει την (*) στην εξίσωση
    \begin{equation*} \tag{**}
        u''+\left(q-\frac{1}{2}p'-\frac{1}{4}p^2\right)u=0.
    \end{equation*}
    Ακόμα, αν $\{u_1, u_2\}$ είναι ένα βασικό σύνολο λύσεων της (**), τότε οι συναρτήσεις
    \[ y_1(x)=u_1(x)\exp\left[-\frac{1}{2}\int_{x_0}^x p(t)dt\right], x \in I; \quad y_2(x)=u_2(x)\exp\left[-\frac{1}{2}\int_{x_0}^x p(t)dt\right], x \in I \]
    αποτελούν ένα βασικό σύνολο λύσεων της (*). (ii) Να επιλυθεί η διαφορική εξίσωση
    \[ y''+(2x+1)y'+\left(x^2+x+\frac{1}{4}\right)y=0, \quad x \in [0,1]. \]

    \Askhsh Ας είναι $y_1$ και $y_2$ οι λύσεις της ομογενούς γραμμικής διαφορικής εξίσωσης
    \[ x^2y''+xy'+(x^2-n^2)y=0, \quad x>0 \]
    (όπου n σταθερά) με
    \[ y_1(1)=1, \quad y_1'(1)=0; \quad y_2(1)=0, \quad y_2'(1)=1. \]
    Να βρεθεί η ορίζουσα Wronski των $y_1, y_2$.\\\\

    \Askhsh Ας είναι f μια συνεχής συνάρτηση στο $(0,\infty)$. Να επιλυθεί η μη ομογενής γραμμική διαφορική εξίσωση
    \[ y''+\frac{1}{4x^2}y=f(x)\cos x, \quad x>0. \]
% This file continues the enumerate list from page 140
    \Askhsh Ας θεωρήσουμε την μη ομογενή γραμμική διαφορική εξίσωση
    \[ y''+y=b(x), \quad x \ge 1, \]
    όπου b είναι μια συνεχής συνάρτηση στο $[1,\infty)$ με $\int_1^\infty |b(x)|dx < \infty$.
    (i) Ν' αποδειχθεί ότι μια μερική λύση είναι
    \[ y_M(x) = \int_1^x \sin(x-t)b(t)dt, \quad x \ge 1. \]
    (ii) Κάθε λύση είναι φραγμένη.\\\\

    \Askhsh Να επιλυθεί η μη ομογενής γραμμική διαφορική εξίσωση
    \[ y''+4y'+4y=e^{-2x}+\frac{1}{x^2}e^{-2x}, \quad x>0. \]

    \Askhsh Μια μη ομογενής γραμμική εξίσωση δεύτερης τάξης έχει τις λύσεις
    $y_1(x)=1+e^{x^2}$, $y_2(x)=1+xe^{x^2}$ και $y_3(x)=(x+1)e^{x^2}+1$ για $x \in \mathbb{R}$.
    Να επιλυθεί η διαφορική εξίσωση. Ιδιαίτερα, να βρεθεί η λύση y με
    $y(0)=1, y'(0)=2$.

    \Askhsh Να επιλυθεί η διαφορική εξίσωση
    \[ x^2y''+xy'+\left(x^2-\frac{1}{4}\right)y=0, \quad x>0, \]
    αφού βρεθεί μια λύση $y_1$ αυτής της μορφής $y_1(x)=\frac{\sin(ax)}{\sqrt{x}}$, $x>0$ (a σταθερά).

    \Askhsh Ας είναι f και g δύο παραγωγίσιμες συναρτήσεις σ' ένα διάστημα Ι. Ν' αποδειχθεί ότι: (i) Αν οι f,g είναι γραμμικά εξαρτημένες, τότε $W(f,g)(x)=0$ για κάθε $x \in I$. (ii) Αν $W(f,g)(x)=0$ για κάποιο $x \in I$, τότε οι f,g είναι γραμμικά ανεξάρτητες. (iii) Αν $W(f,g)(x)=0$ για κάθε $x \in I$, τότε οι f,g δεν είναι αναγκαστικά γραμμικά εξαρτημένες (Αντιπαράδειγμα: $f(x)=x^2, x \in \mathbb{R}$ και $g(x)=x|x|, x \in \mathbb{R}$). (iv) Αν $W(f,g)(x)=0$ για κάθε $x \in I$ και $g(x)\ne 0$ για όλα τα $x \in I$, τότε οι f,g είναι γραμμικά εξαρτημένες.

    \Askhsh Να επιλυθούν οι παρακάτω γραμμικές διαφορικές εξισώσεις με τη βοήθεια των σημειούμενων μετασχηματισμών:
    \begin{alist}
        \item $xy''-y'+x^3y=0, \quad x>0; \quad t=x^2$.
        \item $x(1+x^2)^2y''-(1-3x^2)(1+x^2)y'-8x^3y=4x^3(1+x^2)^2, \quad x>0; \quad t=1+x^2$.

% This file continues the alist from item 23 on page 141
        \item $(x+1)^2y''+(3x+2)(x+1)y'+y=\log(x+1), x>0; z=xy.$
        \item $(1+x)^2y''+2(1+x)y'+y=4y+xy', x>0; z=y+xy'.$
    \end{alist}
    
    \Askhsh Να επιλυθεί η μη ομογενής γραμμική διαφορική εξίσωση
    \[ 2x^2y''+7xy'+3y=\cos\sqrt{x}, \quad x>0. \]
    
    \Askhsh Η μη ομογενής γραμμική διαφορική εξίσωση
    \[ y''-\frac{6}{x^2}y=5x+8, \quad x>0, \]
    ν' αποδειχθεί ότι έχει τις λύσεις
    $y_1(x)=c x^3+x^3\log x - 2x^2$, $y_2(x)=x^{-2}+x^3\log x - 2x^2$ και $y_3(x)=x^3\log x-2x^2$
    για $x>0$ (όπου c σταθερά). Να επιλυθεί, στη συνέχεια, το πρόβλημα αρχικών τιμών
    \[ y''-\frac{6}{x^2}y=5x+8, \quad x>0; y(1)=0, y'(1)=1. \]
    
    \Askhsh Να επιλυθεί η ομογενής γραμμική διαφορική εξίσωση
    \[ (x^2+2x-1)y''-2(x+1)y'+2y=0, \quad x>1, \]
    αφού βρεθούν δύο γραμμικά ανεξάρτητες λύσεις $y_1, y_2$ αυτής της μορφής
    $y_1(x)=ax+b$, $x>1$ και $y_2(x)=yx^2+\delta x+\epsilon$, $x>1$
    (όπου $\alpha,\beta,\gamma,\delta$ και $\epsilon$ είναι σταθερές).
    
    \Askhsh Να επιλυθεί η ομογενής γραμμική διαφορική εξίσωση
    \[ xy''-xy'-y'+y=0, \quad x>0, \]
    δεδομένου ότι $y_1(x)=x$, $x>0$ και $y_2(x)=e^x$, $x>0$ είναι δύο λύσεις της.
    
    \Askhsh Να επιλυθεί το πρόβλημα αρχικών τιμών
    \[ y''-y'-2y=4e^{-x}, \quad y(0)=a, \quad y'(0)=b, \]
    όπου a,b είναι σταθερές. Να βρεθεί ακόμα η ικανή και αναγκαία συνθήκη για τα a,b ώστε η λύση να είναι φραγμένη στο $[0,\infty)$.
    
    \Askhsh Να επιλυθεί η μη ομογενής γραμμική διαφορική εξίσωση
    \[ y''-2y'+y=4e^x\log x, \quad x>0. \]
% This file continues the enumerate list from page 142
    \Askhsh Να επιλυθεί η ομογενής γραμμική διαφορική εξίσωση
    \[ (x \cos x - \sin x)y'' + (x \sin x)y' - (\sin x)y = 0, \quad \frac{\pi}{4} < x < \frac{\pi}{2}, \]
    δεδομένου ότι $y_1(x)=\sin x$, $x \in (\frac{\pi}{4}, \frac{\pi}{2})$ είναι μια λύση της.

    \Askhsh Ας είναι q μια θετική και συνεχής συνάρτηση σ' ένα διάστημα $[a,b]$ και ας θέσουμε $q_m = \min_{x \in [a,b]} \{q(x)\}$. Ν' αποδειχθεί ότι, αν $q_m > k^2\pi^2/(b-a)^2$ (k ένας θετικός ακέραιος), τότε κάθε πραγματική λύση της διαφορικής εξίσωσης
    \[ y''+qy=0 \]
    έχει k τουλάχιστον ρίζες στο $[a,b]$ (Υπόδειξη: Να θεωρηθεί η εξίσωση $y''+[k^2\pi^2/(b-a)^2]y=0$).

    \Askhsh Να θεωρηθεί η διαφορική εξίσωση
    \begin{equation*} \tag{*}
        x^2y''+xy'+(x^2-p^2)y=0, \quad x>0,
    \end{equation*}
    όπου p είναι μια σταθερά.
    \begin{rlist}
        \item Ν' αποδειχθεί ότι ο μετασχηματισμός $y=u/\sqrt{x}$, $x>0$ μετασχηματίζει την (*) στη διαφορική εξίσωση
        \begin{equation*} \tag{**}
            u'' + \left(1-\frac{4p^2-1}{4x^2}\right)u=0, \quad x>0.
        \end{equation*}
        \item Ν' αποδειχθεί ότι, αν $p=0$, τότε κάθε διάστημα της μορφής $[a, a+\pi]$, $a>0$ περιέχει μια τουλάχιστον ρίζα κάθε πραγματικής λύσης της (*). Επίσης, ν' αποδειχθεί ότι, αν $p>1/2$, τότε κάθε διάστημα της μορφής $[a, a+\pi]$, $a>0$ περιέχει το πολύ μια ρίζα κάθε μη μηδενικής πραγματικής λύσης της εξίσωσης (*) (Υπόδειξη: Να συγκριθεί ο αριθμός των ριζών των λύσεων της (**) με τον αριθμό των ριζών των λύσεων της διαφορικής εξίσωσης $u''+u=0$).
    \end{rlist}

    \Askhsh Ας υποθέσουμε ότι $\alpha_0, \alpha_1, \alpha_2, \alpha_3, \lambda$ και a είναι σταθερές με $\alpha_3 \ne 0$ και ας θεωρήσουμε την μη ομογενή γραμμική διαφορική εξίσωση
    \begin{equation*} \tag{*}
        \alpha_3 y'''+\alpha_2 y''+\alpha_1 y'+\alpha_0 y = Ae^{\lambda x}.
    \end{equation*}
    Ας είναι $p(\lambda)=\alpha_3\lambda^3+\alpha_2\lambda^2+\alpha_1\lambda+\alpha_0$ το χαρακτηριστικό πολυώνυμο της (*).
    \begin{rlist}
        \item Ν' αποδειχθεί ότι $y_M(x) = Ae^{\lambda x}/p(\lambda)$, $x \in \mathbb{R}$ είναι μια λύση της (*), αν $p(\lambda)\ne 0$.
        \item Ν' αποδειχθεί ότι, αν $p(\lambda)=0$ και $p'(\lambda)\ne 0$, τότε $y_M(x) = Axe^{\lambda x}/p'(\lambda)$, $x \in \mathbb{R}$ είναι μια λύση της (*).
        \item Κάτω από ποιές συνθήκες μπορεί η (*) να έχει μια λύση της μορφής $y_M(x)=Ax^k e^{\lambda x}$;
    \end{rlist}
% This file continues the enumerate list from page 143
    \Askhsh Δίνεται η εξίσωση \eng{Riccati}
    \[ y' = P(x)y^2+Q(x)y+R(x), \]
    όπου P,Q και R είναι συνεχείς συναρτήσεις σ' ένα διάστημα Ι και $P(x)\ne 0$ για όλα τα $x \in I$. Ν' αποδειχθεί ότι η αντικατάσταση $y=-z'/Pz$ μετασχηματίζει την εξίσωση αυτή στην γραμμική διαφορική εξίσωση δεύτερης τάξης
    \[ z''-[Q+(P'/P)]z'+PRz=0. \]
    \textbf{Εφαρμογή:} Να επιλυθούν οι παρακάτω διαφορικές εξισώσεις \eng{Riccati}:
    \begin{alist}
        \item $xy' = x^2y^2-y+1.$
        \item $x^2y' = x^4y^2+(3x^2-2x)y+2.$
        \item $(\cos x)y' = (\cos^2 x)y^2+(\sin x - 2\cos x)y+5.$
    \end{alist}}
\setchapterimage{./images/7.png}
\chapter{Γραμμικά Διαφορικά Συστήματα}
\chaptertoc
\section*{Εισαγωγή}
Στο Κεφάλαιο αυτό θα μελετήσουμε τα γραμμικά διαφορικά συστήματα. Στο Εδάφιο 0 θα δώσουμε την έννοια του γραμμικού διαφορικού συστήματος, θα διατυπώσουμε το θεώρημα ύπαρξης και μονοσήμαντου των λύσεων για τα γραμμικά διαφορικά συστήματα και θα παραθέσουμε μερικά στοιχεία απ' τη Γραμμική Άλγεβρα και την Ανάλυση για τους πίνακες που θα τα χρειασθούμε στη μελέτη μας. Τα ομογενή γραμμικά διαφορικά συστήματα θα μελετηθούν στο Εδάφιο 1 και τα μη ομογενή γραμμικά διαφορικά συστήματα θα εξετασθούν στο Εδάφιο 2. Στα εδάφια 3 και 4 θ' ασχοληθούμε με τα ομογενή γραμμικά διαφορικά συστήματα με σταθερούς συντελεστές. Στο Εδάφιο 5 θ' αναπτυχθεί η μέθοδος της απαλοιφής για την επίλυση των γραμμικών διαφορικών συστημάτων. Η ευστάθεια των γραμμικών διαφορικών συστημάτων θα μελετηθεί στο Εδάφιο 6. Τέλος, το Εδάφιο 7 περιλαμβάνει μια συλλογή γενικών ασκήσεων.

\section*{Προκαταρκτικά}

Θα δοθεί εδώ η έννοια του γραμμικού διαφορικού συστήματος και θα διατυπωθεί το θεώρημα ύπαρξης και μονοσήμαντου των λύσεων των προβλημάτων αρχικών τιμών για γραμμικά διαφορικά συστήματα.Επίσης, θα παρατεθούν μερικά στοιχεία απ' τη Γραμμική 'Αλγεβρα και την Ανάλυση σχετικά με τους πίνακες. Αυτά είναι απαραίτητα για τη μελέτη των γραμμικών διαφορικών συστημάτων.
\subsection*{0.1. Η έννοια του γραμμικού διαφορικού συστήματος. Ύπαρξη και μονοσήμαντο των λύσεων}

Ένα γραμμικό διαφορικό σύστημα είναι ένα διαφορικό σύστημα της μορφής
\begin{equation} \tag{S}
\left\{
\begin{aligned}
    y'_1 &= a_{11}y_1+a_{12}y_2+\dots+a_{1n}y_n+b_1 \\
    y'_2 &= a_{21}y_1+a_{22}y_2+\dots+a_{2n}y_n+b_2 \\
    &\vdots \\
    y'_n &= a_{n1}y_1+a_{n2}y_2+\dots+a_{nn}y_n+b_n
\end{aligned}
\right.
\end{equation}
όπου $a_{ij}$ (i,j=1,2,...,n) και $b_i$ (i=1,2,...,n) είναι συνεχείς συναρτήσεις σ' ένα διάστημα Ι της πραγματικής ευθείας. Οι συναρτήσεις $a_{ij}$ (i,j=1,2,...,n) λέγονται συντελεστές του γραμμικού διαφορικού συστήματος (S) και το διάστημα Ι λέγεται διάστημα ορισμού αυτού. Αν οι συντελεστές $a_{ij}$ (i,j=1,2,...,n) είναι σταθερές (συναρτήσεις), τότε λέμε ότι το (S) είναι ένα γραμμικό διαφορικό σύστημα με σταθερούς συντελεστές. Όταν $b_i \ne 0$ για ένα τα i=1,2,...,n για μια συνάρτηση f με πεδίο ορισμού το Ι γράφουμε $f \ne 0$, αν και μόνο αν $f(x)=0$ για όλα τα $x \in I$, και διαφορετικά, δηλαδή όταν $f(x)\ne 0$ για ένα τουλάχιστον $x \in I$, γράφουμε $f \ne 0$), τότε το (S) παίρνει τη μορφή
\begin{equation} \tag{S$_0$}
\left\{
\begin{aligned}
    y'_1 &= a_{11}y_1+a_{12}y_2+\dots+a_{1n}y_n \\
    y'_2 &= a_{21}y_1+a_{22}y_2+\dots+a_{2n}y_n \\
    &\vdots \\
    y'_n &= a_{n1}y_1+a_{n2}y_2+\dots+a_{nn}y_n
\end{aligned}
\right.
\end{equation}
και στην περίπτωση αυτή λέγεται ομογενές. Αν για κάποιο $i \in \{1,2,...,n\}$ είναι $b_i \ne 0$, τότε λέμε ότι το (S) είναι ένα μη ομογενές γραμμικό διαφορικό σύστημα. Ακόμα, στην περίπτωση όπου το (S) είναι μη ομογενές, λέμε ότι το (S$_0$) είναι το αντίστοιχο ομογενές γραμμικό διαφορικό σύστημα του (S). Ο n-τάξης πίνακας-συνάρτηση
\[
A = \begin{pmatrix}
    a_{11} & a_{12} & \dots & a_{1n} \\
    a_{21} & a_{22} & \dots & a_{2n} \\
    \vdots & \vdots & \ddots & \vdots \\
    a_{n1} & a_{n2} & \dots & a_{nn}
\end{pmatrix}
\]
λέγεται συντελεστής πίνακας του γραμμικού διαφορικού συστήματος (S) και είναι συνεχής στο διάστημα Ι. Ας θεωρήσουμε την $n$-διάστατη διανυσματική συνάρτηση
\[
b = \begin{pmatrix} b_1 \\ b_2 \\ \vdots \\ b_n \end{pmatrix},
\]
η οποία είναι συνεχής στο Ι. Τότε, θεωρώντας ως άγνωστη συνάρτηση την
\[
y = \begin{pmatrix} y_1 \\ y_2 \\ \vdots \\ y_n \end{pmatrix},
\]
μπορούμε να γράψουμε το γραμμικό διαφορικό σύστημα (S) στη μορφή
\begin{align*}
(S) \qquad\qquad &y' = Ay+b \\
\intertext{και το ($S_0$) ως εξής}
(S_0') \qquad\qquad &y' = Ay.
\end{align*}
Παντού παρακάτω, εκτός απ' το Εδάφιο 5, όταν αναφερόμαστε στα γραμμικά διαφορικά συστήματα (S) και ($S_0$), θα θεωρούμε ότι αυτά είναι γραμμένα στις πιο πάνω μορφές.

Για την ύπαρξη και το μονοσήμαντο των λύσεων των προβλημάτων αρχικών τιμών για γραμμικά διαφορικά συστήματα ισχύει το παρακάτω θεώρημα. Η απόδειξη αυτού δίνεται στο Κεφάλαιο Ι.

\begin{Thewrhma}{}
\textit{Αν $x_0$ είναι ένα σημείο του διαστήματος Ι και $\xi$ είναι ένα n-διάστατο διάνυσμα, τότε υπάρχει ακριβώς μια λύση y του γραμμικού διαφορικού συστήματος (S), η οποία είναι ορισμένη σ' ολόκληρο το διάστημα Ι και πληροί την αρχική συνθήκη}
\[ y(x_0) = \xi. \]
\end{Thewrhma}

Η μόνη υπόθεση στο παραπάνω θεώρημα είναι αυτή της συνέχειας του συντελεστή πίνακα Α και της n-διάστατης διανυσματικής συνάρτησης b στο διάστημα Ι. Ας τονίσουμε ακόμα ότι το θεώρημα 1 εξασφαλίζει ότι όλες οι λύσεις του γραμμικού διαφορικού συστήματος (S) είναι ορισμένες σ' ολόκληρο το διάστημα Ι.

Είναι φανερό ότι το ομογενές γραμμικό διαφορικό σύστημα ($S_0$) έχει ως λύση τη μηδενική $n$-διάστατη διανυσματική συνάρτηση στο Ι (μηδενική λύση). Απ' το παραπάνω θεώρημα προκύπτει ότι, αν για μια λύση y του (S$_0$) είναι $y(x_0)=0$ για κάποιο $x_0 \in I$, τότε η λύση αυτή είναι αναγκαστικά η μηδενική λύση. Έτσι, \textbf{μια λύση του ομογενούς γραμμικού διαφορικού συστήματος (S$_0$) ή θα είναι μηδέν σ' ολόκληρο το διάστημα Ι ή δεν θα μηδενίζεται πουθενά στο Ι.}

\medskip
Ας θεωρήσουμε τη γραμμική διαφορική εξίσωση n-τάξης
\begin{equation} \tag{E}
a_n y^{(n)} + a_{n-1} y^{(n-1)} + \dots + a_1 y' + a_0 y = h,
\end{equation}
όπου $a_i$ (i=0,1,...,n-1,n) και h είναι συνεχείς συναρτήσεις στο διάστημα Ι και $a_n(x) \ne 0$ για όλα τα $x \in I$. Θέτοντας
\[
y_1=u, y_2=u', \dots, y_n=u^{(n-1)},
\]
παίρνουμε
\[
y'_1=u', y'_2=u'', \dots, y'_n=u^{(n)}
\]
και έτσι η γραμμική διαφορική εξίσωση (Ε) ανάγεται στο γραμμικό διαφορικό σύστημα
\begin{equation} \tag{s}
\LEFTRIGHT\{.{
\begin{aligned}
    y'_1 &= y_2 \\
    y'_2 &= y_3 \\
    &\vdots \\
    y'_{n-1} &= y_n \\
    y'_n &= -\frac{a_0}{a_n}y_1 - \frac{a_1}{a_n}y_2 - \dots - \frac{a_{n-1}}{a_n}y_n + \frac{h}{a_n}.
\end{aligned}}
\end{equation}
Το γραμμικό διαφορικό σύστημα (s) μπορεί να γραφεί στη μορφή
\begin{equation} \tag{s}
y' = Ay+b,
\end{equation}
όπου
\[
y = \begin{pmatrix} y_1 \\ y_2 \\ \vdots \\ y_n \end{pmatrix}, \quad
A = \begin{pmatrix}
    0 & 1 & 0 & \dots & 0 \\
    0 & 0 & 1 & \dots & 0 \\
    \vdots & \vdots & \vdots & \ddots & \vdots \\
    0 & 0 & 0 & \dots & 1 \\
    -\frac{a_0}{a_n} & -\frac{a_1}{a_n} & -\frac{a_2}{a_n} & \dots & -\frac{a_{n-1}}{a_n}
\end{pmatrix} \quad \text{και} \quad
b = \begin{pmatrix} 0 \\ 0 \\ \vdots \\ 0 \\ \frac{h}{a_n} \end{pmatrix}.
\]
Ο συντελεστής πίνακας $Α$ και η διανυσματική συνάρτηση b είναι συνεχείς στο διάστημα Ι.

Απ' τον παραπάνω τρόπο αναγωγής της γραμμικής διαφορικής εξίσωσης (Ε) στο γραμμικό διαφορικό σύστημα (s) προκύπτει ότι: Αν u είναι μια λύση της (Ε), τότε
\[
y = \begin{pmatrix} u \\ u' \\ \vdots \\ u^{(n-1)} \end{pmatrix},
\]
είναι μια λύση του (s). Αντίστροφα, αν
\[
y = \begin{pmatrix} y_1 \\ y_2 \\ \vdots \\ y_n \end{pmatrix},
\]
είναι μια λύση του (s), τότε $y_1$ είναι μια λύση της (Ε). Ακόμα: Αν u είναι η λύση της (Ε) που πληροί τις αρχικές συνθήκες
\begin{equation} \tag{*}
u(x_0)=c_0, u'(x_0)=c_1, \dots, u^{(n-1)}(x_0)=c_{n-1},
\end{equation}
όπου $x_0 \in I$ και $c_0, c_1, \dots, c_{n-1}$ είναι σταθερές, τότε
\[
y = \begin{pmatrix} u \\ u' \\ \vdots \\ u^{(n-1)} \end{pmatrix}
\]
είναι η λύση του (s) με
\begin{equation} \tag{**}
y(x_0)=c,
\end{equation}
όπου c είναι το διάνυσμα με συνιστώσες $c_0, c_1, \dots, c_{n-1}$. Αντίστροφα, αν
\[
y = \begin{pmatrix} y_1 \\ y_2 \\ \vdots \\ y_n \end{pmatrix}
\]
είναι η λύση του (s) που πληροί την αρχική συνθήκη (**), όπου $x_0$ είναι ένα σημείο του Ι και c ένα n-διάστατο διάνυσμα, τότε $y_1$ είναι η λύση της (Ε) που πληροί τις αρχικές συνθήκες (*) με $c_0, c_1, \dots, c_{n-1}$ τις συνιστώσες του c.

Λόγω της παραπάνω αντιστοιχίας μεταξύ των λύσεων της (Ε) και των λύσεων του (s), μπορούμε να πούμε ότι η μελέτη της γραμμικής διαφορικής εξίσωσης (Ε) ανάγεται στη μελέτη του γραμμικού διαφορικού συστήματος (s). Έτσι, πολλά συμπεράσματα για τις γραμμικές διαφορικές εξισώσεις μπορούν να παρθούν απ' αντίστοιχα συμπεράσματα για γραμμικά διαφορικά συστήματα.

\subsection*{0.2. Μερικά στοιχεία απ' τη Γραμμική Άλγεβρα και την Ανάλυση για τους πίνακες}

Θα θεωρήσουμε εδώ γνωστή τη στοιχειώδη θεωρία των πινάκων. Θα δώσουμε όμως την έννοια της ιδιοτιμής ενός (τετραγωνικού) πίνακα και θα διατυπώσουμε το θεώρημα Cayley-Hamilton. Aς είναι C ένας n-τάξης πίνακας και Ι ο μοναδιαίος n-τάξης πίνακας. Τότε το πολυώνυμο $p(\lambda) = \det(C - \lambda I) = c_n \lambda^n + c_{n-1} \lambda^{n-1} + \dots + c_1 \lambda + c_0$ λέγεται χαρακτηριστικό πολυώνυμο του πίνακα C και οι n ρίζες του (όχι αναγκαστικά διακεκριμένες) λέγονται \textbf{ιδιοτιμές} (ή χαρακτηριστικές τιμές) του C. Αν $\lambda_1, \dots, \lambda_n$ είναι οι ιδιοτιμές του C, τότε $p(\lambda) = (\lambda_1 - \lambda) \dots (\lambda_n - \lambda)$. Το θεώρημα Cayley-Hamilton (ένα απ' τα βασικότερα θεωρήματα της Γραμμικής Άλγεβρας) εξασφαλίζει ότι ο πίνακας C μηδενίζει το χαρακτηριστικό του πολυώνυμο με την έννοια ότι ισχύει
\[
p(C) = c_n C^n + c_{n-1} C^{n-1} + \dots + c_1 C + c_0 I = 0,
\]
όπου 0 είναι ο μηδενικός n-τάξης πίνακας. Έτσι, αν $\lambda_1, \dots, \lambda_n$ είναι οι ιδιοτιμές του πίνακα C, τότε
\[
(C-\lambda_1 I)(C-\lambda_2 I) \dots (C-\lambda_n I) = 0.
\]
Ας είναι $Y = (y_{ij})$ ένας n-τάξης πίνακας-συνάρτηση στο Ι, δηλαδή ένας πίνακας του οποίου τα στοιχεία $y_{ij},\  (i,j=1,...,n)$ είναι συναρτήσεις ορισμένες στο Ι. Θα λέμε ότι Υ είναι \textbf{φραγμένος} στο Ι αν και μόνο αν οι συναρτήσεις $y_{ij},\  (i,j=1,...,n)$ είναι φραγμένες στο Ι. Ακόμα, για $x \to \infty$, θα λέμε ότι $Y(x) \to 0$ αν και μόνο αν οι συναρτήσεις $y_{ij}(x) \to 0$ για $x \to \infty$, και θα γράφουμε $\lim_{x\to\infty} Y(x) = 0$ αν και μόνο αν $\lim_{x\to\infty} y_{ij}(x) = 0,\  (i,j=1,...,n).$

Ας είναι πάλι $Υ$ ένας $n$-τάξης πίνακας-συνάρτηση στο $Ι$. Λέμε ότι ο πίνακας-συνάρτηση $Υ$ είναι \textbf{παραγωγίσιμος} στο $Ι$ αν και μόνο αν τα στοιχεία του είναι παραγωγίσιμες συναρτήσεις στο $Ι$. Επιπλέον, αν $Υ$ είναι παραγωγίσιμος στο Ι, τότε η παράγωγος αυτού συμβολίζεται με $Υ'$ και προκύπτει απ' τον $Υ$ με παραγώγιση των στοιχείων του. Αν ο $Υ$ είναι παραγωγίσιμος και $y$ είναι μια n-διάστατη διανυσματική συνάρτηση που έχει παράγωγο στο Ι, τότε η συνάρτηση $Yy$ είναι επίσης παραγωγίσιμη στο Ι και $(Yy)' = Y'y + Yy'$. Ακόμα, αν ο $Υ$ είναι παραγωγίσιμος στο $Ι$ και $Ζ$ ένας άλλος παραγωγίσιμος n-τάξης πίνακας-συνάρτηση στο Ι, τότε $ΥΖ$ είναι παραγωγίσιμος στο Ι και μάλιστα $(YZ)' = Y'Z + YZ'$. Τέλος, αν ο Υ είναι παραγωγίσιμος στο Ι και $\det Y(x) \neq 0$ για κάθε $x \in I$, τότε ορίζεται ο πίνακας-συνάρτηση $Y^{-1}$ με $Y^{-1}(x) = [Y(x)]^{-1}$, x $\in$ Ι ο οποίος είναι παραγωγίσιμος στο Ι και είναι $(Y^{-1})' = -Y^{-1}Y'Y^{-1}$ (γιατί $Y Y^{-1}=I$).

Αν Υ είναι ένας n-τάξης πίνακας-συνάρτηση που έχει παράγωγο στο Ι, τότε $(\det Y)' = (\det Y) \sum_{i,j=1}^{n} z_{ij} y'_{ij}$, όπου, για κάθε i,j (i,j=1,..,n), $z_{ij}$ είναι ο πίνακας-συνάρτηση που προκύπτει απ' τον Υ με παραγώγιση των στοιχείων-συναρτήσεων της i-γραμμής αυτού.

Ένα ομογενές γραμμικό (αλγεβρικό) σύστημα έχει μη μηδενικές λύσεις αν και μόνο αν η ορίζουσα των συντελεστών αυτού είναι μηδέν. Έτσι, αν C είναι ένας n-τάξης πίνακας με $\det C \neq 0$ και c είναι ένα n-διάστατο διάνυσμα, τότε η ισότητα $Cc=0$ ισχύει μόνο όταν c=0. Ας είναι $C=(c_{ij})$ ένας n-τάξης πίνακας. Θέτουμε
\[
h = \max_{i,j=1,\dots,n} |c_{ij}| \text{ και } C^v = (c_{ij}^{(v)}) \ (v=1,2,\dots).
\]
Τότε για κάθε $v=1,2,...$ ισχύει
\[
|c_{ij}^{(v)}| \leq n^{v-1}h^v \ (i,j=1,\dots,n).
\]
Πραγματικά, για $v=1$ έχουμε $|c_{ij}^{(1)}| = |c_{ij}| \leq h$ (i,j=1,...,n) και άρα η πρότασή μας αληθεύει για $v=1$. Αν υποθέσουμε ότι η πρόταση είναι αληθής για κάποιο $v \in \{1,2,\dots\}$, τότε παίρνουμε
\[
|c_{ij}^{(v+1)}| = |\sum_{k=1}^{n} c_{ik}^{(v)} c_{kj}| \leq \sum_{k=1}^{n} |c_{ik}^{(v)}| |c_{kj}| \leq \sum_{k=1}^{n} n^{v-1}h^v h = n^v h^{v+1}
\]
για i,j=1,...,n και επομένως η πρότασή μας ισχύει για το v+1. Άρα, η πρόταση ισχύει για όλα τα v=1,2,.... Παρατηρούμε τώρα ότι η σειρά
\[
\sum_{v=1}^{\infty} \frac{c_{ij}^{(v)}}{v!}
\]
συγκλίνει για οποιαδήποτε $i,j,v=1,...,n,$ επειδή
\[
\sum_{v=1}^{\infty} \frac{|c_{ij}^{(v)}|}{v!} \leq \sum_{v=1}^{\infty} \frac{n^{v-1}h^v}{v!} = \frac{1}{n} \sum_{v=1}^{\infty} \frac{(nh)^v}{v!} = e^{nh}-1 < \infty.
\]
Ορίζουμε τη σειρά πινάκων $\sum_{v=1}^{\infty} \frac{C^v}{v!}$ ως εξής
\[
\sum_{v=1}^{\infty} \frac{C^v}{v!} = \left( \sum_{v=1}^{\infty} \frac{c_{ij}^{(v)}}{v!} \right)
\]
και έχουμε ότι αυτή συγκλίνει προς ένα n-τάξης πίνακα. Μετά απ' τα παραπάνω, μπορούμε να ορίσουμε τον \textbf{εκθετικό πίνακα} του C με τον τύπο
\[
e^C = I + \sum_{v=1}^{\infty} \frac{C^v}{v!}.
\]
Αν C και D είναι δύο n-τάξης πίνακες που αντιμετατίθενται, τότε μπορεί ν' αποδειχθεί ότι
\[
e^{C+D} = e^C e^D.
\]
Επίσης, αποδεικνύεται ότι $\det e^C \neq 0$ για οποιονδήποτε n-τάξης πίνακα C. Τώρα, αν C είναι ένας n-τάξης πίνακας, τότε, επειδή C και -C αντιμετατίθενται, είναι
\[
I = e^0 = e^{C+(-C)} = e^C e^{-C}
\]
και άρα ισχύει
\[
(e^C)^{-1} = e^{-C}.
\]
Τέλος, για οποιονδήποτε n-τάξης πίνακα C μπορεί να ορισθεί ο πίνακας-συνάρτηση $e^{xC}$, $x \in \mathbb{R}$ που έχει παράγωγο και μάλιστα για όλα τα $x \in \mathbb{R}$ είναι
\begin{align*}
(e^{xC})' &= \left( I + \sum_{v=1}^{\infty} \frac{x^v C^v}{v!} \right)' = \sum_{v=1}^{\infty} \frac{vx^{v-1}C^v}{v!} = C \sum_{v=1}^{\infty} \frac{x^{v-1}C^{v-1}}{(v-1)!} \\
&= C \left( I + \sum_{v=1}^{\infty} \frac{x^v C^v}{v!} \right) = C e^{xC}.
\end{align*}
Πριν κλείσουμε την παράγραφο αυτή, θα υπενθυμίσουμε ότι η \textbf{στάθμη} ενός n-τάξης τετραγωνικού πίνακα C ορίζεται με τον τύπο
\[
|C| = \sup_{|c|=1} |Cc| = \sup_{c\neq 0} \frac{|Cc|}{|c|}
\]
όπου για ένα n-διάστατο διάνυσμα c με $|c|$ παριστάνουμε μια στάθμη του c στο χώρο των n-διάστατων διανυσμάτων. Για κάθε n-τάξης τετραγωνικό πίνακα C και για κάθε n-διάστατο διάνυσμα c είναι
\[
|Cc| \leq |C||c|.
\]
Επίσης, για δύο n-τάξης τετραγωνικούς πίνακες C και D ισχύει
\[
|CD| \leq |C||D|.
\]
Τέλος, αν Υ είναι ένας n-τάξης τετραγωνικός πίνακας-συνάρτηση σ' ένα διάστημα Ι, τότε Υ είναι φραγμένος αν και μόνο αν η συνάρτηση $|Y(x)|$, $x \in I$ είναι φραγμένη και, για $I=[x_0, \infty)$, $\lim_{x\to\infty} Y(x) = 0$ αν και μόνο αν $\lim_{x\to\infty} |Y(x)| = 0$.

Θα δώσουμε τώρα μερικά ακόμα στοιχεία απ' τη Θεωρία Πινάκων, τα οποία θα χρειασθούμε στο Εδάφιο 4. Για τα παρακάτω, C θα είναι ένας n-τάξης πίνακας.

Υπάρχουν ένας ακέραιος $m \in I$ με $1 \leq m \leq n$ και ένα μοναδικό πολυώνυμο q βαθμού m με συντελεστή του μεγιστοβαθμίου όρου του τη μονάδα έτσι ώστε $q(C)=0$ ενώ $Q(C) \neq 0$ για κάθε μη μηδενικό πολυώνυμο Q βαθμού μικρότερου του m. Το πολυώνυμο q λέμε ότι είναι το \textbf{ελάχιστο πολυώνυμο} του πίνακα C. Αν p είναι το χαρακτηριστικό πολυώνυμο του πίνακα C και $d_{n-1}(\lambda)$ είναι ο μέγιστος κοινός διαιρέτης των στοιχείων του πίνακα $\text{adj}(C-\lambda I)$, τότε
\[
p(\lambda) = q(\lambda) d_{n-1}(\lambda).
\]
Το συμπέρασμα αυτό δίνει μια μέθοδο για την εύρεση του ελάχιστου πολυωνύμου του πίνακα C, αν είναι γνωστό το χαρακτηριστικό πολυώνυμο αυτού.

Ας είναι $\lambda_1, \lambda_2, \dots, \lambda_s$ οι διακεκριμένες ιδιοτιμές του πίνακα C με πολλαπλότητες $n_1, n_2, \dots, n_s$ αντίστοιχα ($n_1+n_2+\dots+n_s=n$). Τότε
\[
p(\lambda) = (\lambda-\lambda_1)^{n_1}(\lambda-\lambda_2)^{n_2}\dots(\lambda-\lambda_s)^{n_s},
\]
ενώ
\[
q(\lambda) = (\lambda-\lambda_1)^{m_1}(\lambda-\lambda_2)^{m_2}\dots(\lambda-\lambda_s)^{m_s},
\]
όπου $1 \leq m_j \leq n_j$ (j=1,2,...,s) και $m_1+m_2+...+m_s=m$. Ένα από τα πιο βασικά συμπεράσματα της θεωρίας Πινάκων είναι αν f είναι μια συνάρτηση τέτοια ώστε $f^{(j)}(\lambda_i)$ (j=0,1,...,$m_j-1$; i=1,...,s) να υπάρχουν, τότε ισχύει
\[
f(C) = \sum_{i=1}^{s} \sum_{j=0}^{m_i-1} f^{(j)}(\lambda_i) Z_{ij},
\]
όπου $Z_{ij}$ (j=0,1,...,$m_j-1$; i=1,...,s) είναι m πίνακες που δεν εξαρτώνται απ' την f (αλλά μόνο απ' τον πίνακα C) και καλούνται \textbf{συνιστώσες} του πίνακα C. Αποδεικνύεται ότι
\begin{align*}
Z_{i0} &= \sum_{k=1}^{s} \frac{1}{k!} Z_{ik} = I \\
\text{και} \\
Z_{ij} &= \frac{1}{j!} (C-\lambda_i I)^j Z_{i0} \quad (j=0,1,\dots,m_i-1; i=1,\dots,s).
\end{align*}
Ακόμα, αν $m_i=1$ (i=1,...,s) και s=1, τότε είναι
\[
Z_{10} = \sum_{k=1}^{s} \frac{1}{k!} \frac{(C-\lambda_k I)}{(\lambda_i-\lambda_k)} \quad (i=1,\dots,s).
\]
Γενικά, οι συνιστώσες του C προσδιορίζονται ως εξής: θεωρούμε m πολυώνυμα $g_0, g_1, \dots, g_{m-1}$ με συντελεστές του μεγιστοβαθμίου όρων τη μονάδα και βαθμούς 0,1,...,m-1 αντίστοιχα. Τότε απ' το σύστημα
\[
g_k(C) = \sum_{i=1}^{s} \sum_{j=0}^{m_i-1} g_k^{(j)}(\lambda_i) Z_{ij} \quad (k=0,1,\dots,m-1)
\]
προκύπτουν οι m συνιστώσες $Z_{ij}$ (j=0,1,...,$m_i-1$; i=1,...,s) του πίνακα C.

\section{Ομογενή γραμμικά διαφορικά συστήματα}
Το εδάφιο αυτό αναφέρεται στη μελέτη των ομογενών γραμμικών διαφορικών συστημάτων. Εισάγονται οι έννοιες του πίνακα λύσεων και του βασικού πίνακα για το ομογενές γραμμικό διαφορικό σύστημα $(S_0)$ και δίνονται μερικά συμπεράσματα (θεωρήματα 2-10) σχετικά με τις λύσεις, τους πίνακες λύσεων και τους βασικούς πίνακες για το $(S_0)$. Τελικά, αποδεικνύεται (θεώρημα 11) ότι οι λύσεις του $(S_0)$ είναι μερικοί οι αλγεβρικοί συνδυασμοί ή γραμμικοί συνδυασμοί των λύσεων αυτού. Δίνεται ακόμα (θεώρημα 12) ένας βασικός πίνακας του $(S_0)$ με $n=2$, όταν είναι γνωστή μια λύση του της οποίας η πρώτη συντεταγμένη δεν μηδενίζεται πουθενά στο $Ι$. Τέλος, δίνονται μερικά παραδείγματα και προτείνονται ορισμένες ασκήσεις για λύση.

\subsection{Πίνακες λύσεων. Ο τύπος του \eng{Jacobi}}
Ένας $n$-τάξης πίνακας-συνάρτηση στο διάστημα Ι, του οποίου οι στήλες είναι λύσεις του $(S_0)$, λέμε ότι είναι ένας \textbf{πίνακας λύσεων} του ομογενούς γραμμικού διαφορικού συστήματος $(S_0)$.
\begin{Thewrhma}{}
Ας είναι $Y$ ένας $n$-τάξης πίνακας-συνάρτηση στο διάστημα $Ι$. Τότε $Υ$ είναι ένας πίνακας λύσεων του ομογενούς γραμμικού διαφορικού συστήματος $(S_0)$ αν και μόνο αν έχει παράγωγο στο $Ι$ και
\[ Y' = AY. \]
\end{Thewrhma}

\textbf{ΑΠΟΔΕΙΞΗ.} Ας είναι
\[
Y = \begin{pmatrix}
y_{11} & y_{12} & \cdots & y_{1n} \\
y_{21} & y_{22} & \cdots & y_{2n} \\
\vdots & \vdots & \ddots & \vdots \\
y_{n1} & y_{n2} & \cdots & y_{nn}
\end{pmatrix}
\]
και ας θεωρήσουμε τις στήλες του
\[
y_j = \begin{pmatrix}
y_{1j} \\ y_{2j} \\ \vdots \\ y_{nj}
\end{pmatrix}
\quad (j=1,2,\dots,n).
\]
Είναι φανερό ότι ο $Υ$ έχει παράγωγο στο Ι αν και μόνο αν οι $n$-διάστατες διανυσματικές συναρτήσεις $y_j,\ (j=1,2,...,n)$ είναι παραγωγίσιμες στο διάστημα $Ι$. Ας υποθέσουμε λοιπόν ότι ο $Y$ είναι παραγωγίσιμος στο Ι. Τότε $y_j,\ (j=1,2,...,n)$ είναι λύσεις του ομογενούς διαφορικού συστήματος $(S_0)$ αν και μόνο αν
\[
\begin{pmatrix}
y'_{1j} \\ y'_{2j} \\ \vdots \\ y'_{nj}
\end{pmatrix}
=
\begin{pmatrix}
a_{11} & a_{12} & \cdots & a_{1n} \\
a_{21} & a_{22} & \cdots & a_{2n} \\
\vdots & \vdots & \ddots & \vdots \\
a_{n1} & a_{n2} & \cdots & a_{nn}
\end{pmatrix}
\begin{pmatrix}
y_{1j} \\ y_{2j} \\ \vdots \\ y_{nj}
\end{pmatrix}
\quad (j=1,2,\dots,n)
\]
ή ισοδύναμα
\[
y'_{ij} = \sum_{k=1}^{n} a_{ik} y_{kj} \quad (i,j=1,2,\dots,n).
\]
Οι τελευταίες ισότητες είναι ισοδύναμες με την
\[ Y' = AY. \]

\begin{Thewrhma}{}
Ας είναι $Y$ ένας πίνακας λύσεων του ομογενούς γραμμικού διαφορικού συστήματος $(S_0)$ και $c$ ένα $n$-διάστατο διάνυσμα. Τότε το Yc είναι μια λύση του $(S_0)$.
\end{Thewrhma}
\textbf{ΑΠΟΔΕΙΞΗ.} Σύμφωνα με το θεώρημα 2, είναι $Y' = AY$ και έτσι έχουμε
\[
(Yc)' = Y'c = (AY)c = A(Yc),
\]
το οποίο αποδεικνύει ότι $Yc$ είναι μια λύση του $(S_0)$.

\begin{Thewrhma}{Τύπος του \eng{Jacobi}}
Ας είναι $Y$ ένας πίνακας λύσεων του ομογενούς γραμμικού διαφορικού συστήματος $(S_0)$ και $x_0$ ένα σημείο του $Ι$. Τότε
\[
\det Y(x) = \det Y(x_0) e^{\int_{x_0}^{x} \tr A(t) \,dt}\text{ για όλα τα } x \in I.
\]
\end{Thewrhma}

\textbf{ΑΠΟΔΕΙΞΗ.} Ας είναι
\[
Y = \begin{pmatrix}
y_{11} & y_{12} & \cdots & y_{1n} \\
y_{21} & y_{22} & \cdots & y_{2n} \\
\vdots & \vdots & \ddots & \vdots \\
y_{n1} & y_{n2} & \cdots & y_{nn}
\end{pmatrix}.
\]
Τότε (θεώρημα 2) είναι $Y' = AY$, δηλαδή
\[
y'_{ij} = \sum_{k=1}^{n} a_{ik} y_{kj} \quad (i,j=1,2,\dots,n).
\]
Έτσι, έχουμε
\[
(\det Y)' = \det
\begin{pmatrix}
y'_{11} & y'_{12} & \cdots & y'_{1n} \\
y_{21} & y_{22} & \cdots & y_{2n} \\
\vdots & \vdots & \ddots & \vdots \\
y_{n1} & y_{n2} & \cdots & y_{nn}
\end{pmatrix}
+ \det
\begin{pmatrix}
y_{11} & y_{12} & \cdots & y_{1n} \\
y'_{21} & y'_{22} & \cdots & y'_{2n} \\
\vdots & \vdots & \ddots & \vdots \\
y_{n1} & y_{n2} & \cdots & y_{nn}
\end{pmatrix}
+ \dots
\]
\[
\dots + \det
\begin{pmatrix}
y_{11} & y_{12} & \cdots & y_{1n} \\
y_{21} & y_{22} & \cdots & y_{2n} \\
\vdots & \vdots & \ddots & \vdots \\
y'_{n1} & y'_{n2} & \cdots & y'_{nn}
\end{pmatrix}.
\]
\begin{align*}
&= \det
\begin{pmatrix}
\sum_{k=1}^{n} a_{1k} y_{k1} & \sum_{k=1}^{n} a_{1k} y_{k2} & \cdots & \sum_{k=1}^{n} a_{1k} y_{kn} \\
y_{21} & y_{22} & \cdots & y_{2n} \\
\vdots & \vdots & \ddots & \vdots \\
y_{n1} & y_{n2} & \cdots & y_{nn}
\end{pmatrix} \\
&+ \det
\begin{pmatrix}
y_{11} & y_{12} & \cdots & y_{1n} \\
\sum_{k=1}^{n} a_{2k} y_{k1} & \sum_{k=1}^{n} a_{2k} y_{k2} & \cdots & \sum_{k=1}^{n} a_{2k} y_{kn} \\
\vdots & \vdots & \ddots & \vdots \\
y_{n1} & y_{n2} & \cdots & y_{nn}
\end{pmatrix}
+ \dots \\
&+ \det
\begin{pmatrix}
y_{11} & y_{12} & \cdots & y_{1n} \\
y_{21} & y_{22} & \cdots & y_{2n} \\
\vdots & \vdots & \ddots & \vdots \\
\sum_{k=1}^{n} a_{nk} y_{k1} & \sum_{k=1}^{n} a_{nk} y_{k2} & \cdots & \sum_{k=1}^{n} a_{nk} y_{kn}
\end{pmatrix} \\
&= \sum_{k=1}^{n} a_{1k} \det
\begin{pmatrix}
y_{k1} & y_{k2} & \cdots & y_{kn} \\
y_{21} & y_{22} & \cdots & y_{2n} \\
\vdots & \vdots & \ddots & \vdots \\
y_{n1} & y_{n2} & \cdots & y_{nn}
\end{pmatrix}
+ \sum_{k=1}^{n} a_{2k} \det
\begin{pmatrix}
y_{11} & y_{12} & \cdots & y_{1n} \\
y_{k1} & y_{k2} & \cdots & y_{kn} \\
\vdots & \vdots & \ddots & \vdots \\
y_{n1} & y_{n2} & \cdots & y_{nn}
\end{pmatrix}
+ \dots \\
&\dots + \sum_{k=1}^{n} a_{nk} \det
\begin{pmatrix}
y_{11} & y_{12} & \cdots & y_{1n} \\
y_{21} & y_{22} & \cdots & y_{2n} \\
\vdots & \vdots & \ddots & \vdots \\
y_{k1} & y_{k2} & \cdots & y_{kn}
\end{pmatrix} \\
&= a_{11} \det
\begin{pmatrix}
y_{11} & y_{12} & \cdots & y_{1n} \\
y_{21} & y_{22} & \cdots & y_{2n} \\
\vdots & \vdots & \ddots & \vdots \\
y_{n1} & y_{n2} & \cdots & y_{nn}
\end{pmatrix}
+ a_{22} \det
\begin{pmatrix}
y_{11} & y_{12} & \cdots & y_{1n} \\
y_{21} & y_{22} & \cdots & y_{2n} \\
\vdots & \vdots & \ddots & \vdots \\
y_{n1} & y_{n2} & \cdots & y_{nn}
\end{pmatrix}
+ \dots \\
&\dots + a_{nn} \det
\begin{pmatrix}
y_{11} & y_{12} & \cdots & y_{1n} \\
y_{21} & y_{22} & \cdots & y_{2n} \\
\vdots & \vdots & \ddots & \vdots \\
y_{n1} & y_{n2} & \cdots & y_{nn}
\end{pmatrix} \\
&= (a_{11}+a_{22}+\dots+a_{nn}) \det Y = (\tr A) \det Y.
\end{align*}
Επομένως, η συνάρτηση $\det Y$ είναι μια λύση της ομογενούς γραμμικής διαφορικής εξίσωσης πρώτης τάξης
\[ w' - (\tr A)w = 0 \]
και έτσι προκύπτει ο τύπος μας. Απ' το παραπάνω θεώρημα προκύπτει ότι η ορίζουσα ενός πίνακα λύσεων του ομογενούς γραμμικού διαφορικού συστήματος $(S_0)$ ή θα είναι μηδέν σ' ολόκληρο το διάστημα $Ι$ ή δεν θα μηδενίζεται πουθενά στο $Ι$.
\subsection{Γραμμική ανεξαρτησία. Βασικοί πίνακες. Το σύνολο των λύσεων}

Η χαρακτηριστική ιδιότητα που έχουν τα ομογενή γραμμικά διαφορικά συστήματα είναι ότι οι γραμμικοί συνδυασμοί λύσεων είναι επίσης λύσεις. Συγκεκριμένα, έχουμε το παρακάτω θεώρημα.

\begin{Thewrhma}{}
Ας είναι $y_k,\ (k=1,...,m)$ λύσεις του ομογενούς γραμμικού διαφορικού συστήματος $(S_0)$ και $c_k,\ (k=1,...,m)$ σταθερές. Τότε το $c_1 y_1 + \dots + c_m y_m$ είναι επίσης μια λύση του $(S_0)$.
\end{Thewrhma}

\textbf{ΑΠΟΔΕΙΞΗ.} Είναι
\[
(c_1 y_1 + \dots + c_m y_m)' = c_1 y'_1 + \dots + c_m y'_m = c_1 (Ay_1) + \dots + c_m (Ay_m) = A(c_1 y_1 + \dots + c_m y_m).
\]
Ας είναι $f_k,\ (k=1,...,m) n$-διάστατες διανυσματικές συναρτήσεις ορισμένες στο διάστημα Ι. Λέμε ότι οι συναρτήσεις αυτές είναι γραμμικά εξαρτημένες αν και μόνο αν υπάρχουν σταθερές $c_k$ $(k=1,...,m)$, όχι όλες μηδέν, έτσι ώστε
\[
c_1 f_1 + \dots + c_m f_m = 0
\]
(το δεύτερο μέλος είναι η μηδενική $n$-διάστατη διανυσματική συνάρτηση στο Ι). Διαφορετικά, δηλαδή όταν η παραπάνω ισότητα ισχύει μόνο για $c_1=\dots=c_m=0$, λέμε ότι οι $f_k$ $(k=1,...,m)$ είναι γραμμικά ανεξάρτητες.

\begin{Thewrhma}{}
Ας είναι $y_k$ $(k=1,...,n)$ λύσεις του ομογενούς γραμμικού διαφορικού συστήματος $(S_0)$ και $x_0$ ένα σημείο του διαστήματος Ι. Τότε οι $y_k$ $(k=1,...,n)$ είναι γραμμικά ανεξάρτητες αν και μόνο αν τα $n$-διάστατα διανύσματα $y_k(x_0)$ $(k=1,...,n)$ είναι γραμμικά ανεξάρτητα.
\end{Thewrhma}

\textbf{ΑΠΟΔΕΙΞΗ.} Αν οι λύσεις $y_k$ $(k=1,...,m)$ είναι γραμμικά εξαρτημένες, τότε υπάρχουν σταθερές $c_k$ $(k=1,...,m)$, όχι όλες μηδέν, έτσι ώστε
\begin{equation*}
c_1 y_1 + \dots + c_m y_m = 0, \qquad (*)
\end{equation*}
οπότε και
\begin{equation*}
c_1 y_1(x_0) + \dots + c_m y_m(x_0) = 0, \qquad (**)
\end{equation*}
το οποίο σημαίνει ότι τα n-διάστατα διανύσματα $y_k(x_0)$ $(k=1,...,m)$ είναι γραμμικά εξαρτημένα. Αντίστροφα, ας υποθέσουμε ότι τα διανύσματα $y_k(x_0)$ $(k=1,...,m)$ είναι γραμμικά εξαρτημένα. Τότε θα ισχύει η (**) για κάποιες σταθερές $c_k$ $(k=1,...,m)$ που δεν είναι όλες μηδέν. Η συνάρτηση $y = c_1 y_1 + \dots + c_m y_m$ είναι (θεώρημα 5) μια λύση του $(S_0)$, η λύση αυτή πληροί την αρχική συνθήκη $y(x_0)=0$ και επομένως (θεώρημα 1) είναι η μηδενική λύση, δηλαδή έχουμε την ισότητα (*), το οποίο αποδεικνύει τη γραμμική εξάρτηση των $y_k$ $(k=1,...,m)$.

Ένας πίνακας λύσεων του $(S_0)$ που οι στήλες του είναι γραμμικά ανεξάρτητες (συναρτήσεις) λέμε ότι είναι βασικός πίνακας του ομογενούς γραμμικού διαφορικού συστήματος $(S_0)$. Έτσι, ένας βασικός πίνακας του $(S_0)$ είναι ένας n-τάξης πίνακας-συνάρτηση στο Ι, του οποίου οι στήλες είναι γραμμικά ανεξάρτητες λύσεις του $(S_0)$.
\begin{Thewrhma}{}
Υπάρχουν βασικοί πίνακες του ομογενούς γραμμικού διαφορικού συστήματος $(S_0)$.
\end{Thewrhma}

\textbf{ΑΠΟΔΕΙΞΗ.} Ας θεωρήσουμε τα n-διάστατα διανύσματα
\[
e_1 = 
\begin{pmatrix}
1 \\ 0 \\ \vdots \\ 0
\end{pmatrix}
, \quad
e_2 = 
\begin{pmatrix}
0 \\ 1 \\ \vdots \\ 0
\end{pmatrix}
, \dots, \quad
e_n = 
\begin{pmatrix}
0 \\ 0 \\ \vdots \\ 1
\end{pmatrix}
.
\]
Το θεώρημα 1 εξασφαλίζει την ύπαρξη των λύσεων $Y_k$ $(k=1,...,n)$ του $(S_0)$ με
\[
Y_k(x_0) = e_k \quad (k=1,...,n),
\]
όπου $x_0$ είναι ένα σημείο του Ι. Οι λύσεις αυτές είναι, σύμφωνα με το θεώρημα 6, γραμμικά ανεξάρτητες δεδομένου ότι τα $e_k$ $(k=1,...,n)$ είναι γραμμικά ανεξάρτητα. Έτσι, ο πίνακας-συνάρτηση με στήλες τις λύσεις $Y_k$ $(k=1,...,n)$ είναι ένας βασικός πίνακας του $(S_0)$.

\begin{Thewrhma}{}
Ας είναι Y ένας πίνακας λύσεων του ομογενούς γραμμικού διαφορικού συστήματος $(S_0)$. Τότε Y είναι ένας βασικός πίνακας του $(S_0)$ αν και μόνο αν
\[
\det Y(x) \neq 0 \text{ για όλα τα } x \in I.
\]
\end{Thewrhma}

\textbf{ΑΠΟΔΕΙΞΗ.} Ας είναι
\[
Y = 
\begin{pmatrix}
y_{11} & y_{12} & \cdots & y_{1n} \\
y_{21} & y_{22} & \cdots & y_{2n} \\
\vdots & \vdots & \ddots & \vdots \\
y_{n1} & y_{n2} & \cdots & y_{nn}
\end{pmatrix}
,
\]
και ας θεωρήσουμε τις στήλες του
\[
y_j = 
\begin{pmatrix}
y_{1j} \\ y_{2j} \\ \vdots \\ y_{nj}
\end{pmatrix}
\quad (j=1,2,...,n)
\]
που είναι λύσεις του $(S_0)$. Σύμφωνα με το θεώρημα 6, οι λύσεις αυτές είναι γραμμικά ανεξάρτητες αν και μόνο αν τα n-διάστατα διανύσματα
\[
y_j(x_0) = 
\begin{pmatrix}
y_{1j}(x_0) \\
y_{2j}(x_0) \\
\vdots \\
y_{nj}(x_0)
\end{pmatrix}
\quad (j=1,2,...,n)
\]
είναι γραμμικά ανεξάρτητα. Τα διανύσματα αυτά είναι γραμμικά ανεξάρτητα τότε και μόνο τότε αν η σχέση
\[
c_1 y_1(x_0) + c_2 y_2(x_0) + \dots + c_n y_n(x_0) = 0
\]
συνεπάγεται το μηδενισμό των σταθερών $c_j$ $(j=1,2,...,n)$, δηλαδή όταν και μόνο όταν το ομογενές γραμμικό (αλγεβρικό) σύστημα
\[
\begin{pmatrix}
c_1 y_{11}(x_0) + c_2 y_{12}(x_0) + \dots + c_n y_{1n}(x_0) = 0 \\
c_1 y_{21}(x_0) + c_2 y_{22}(x_0) + \dots + c_n y_{2n}(x_0) = 0 \\
\vdots \\
c_1 y_{n1}(x_0) + c_2 y_{n2}(x_0) + \dots + c_n y_{nn}(x_0) = 0
\end{pmatrix}
\]
έχει μόνο τη μηδενική λύση $c_1=c_2=\dots=c_n=0$. Αυτό συμβαίνει αν και μόνο αν
\[
\det
\begin{pmatrix}
y_{11}(x_0) & y_{12}(x_0) & \cdots & y_{1n}(x_0) \\
y_{21}(x_0) & y_{22}(x_0) & \cdots & y_{2n}(x_0) \\
\vdots & \vdots & \ddots & \vdots \\
y_{n1}(x_0) & y_{n2}(x_0) & \cdots & y_{nn}(x_0)
\end{pmatrix}
= \det Y(x_0) \neq 0.
\]
Τέλος, απ' τον τύπο του \eng{Jacobi} (θεώρημα 4) προκύπτει ότι, αν $\det Y(x_0) \neq 0$, τότε $\det Y(x) \neq 0$ για όλα τα $x \in I$.

\begin{Thewrhma}{}
(i) Αν $Y$ είναι ένας βασικός πίνακας του ομογενούς γραμμικού διαφορικού συστήματος $(S_0)$ και $C$ είναι ένας σταθερός $n$-τάξης πίνακας με $\det C \neq 0$, τότε το $YC$ είναι επίσης ένας βασικός πίνακας του $(S_0)$.

(ii) Αν Y και $Y^*$ είναι δύο βασικοί πίνακες του $(S_0)$, τότε υπάρχει ένας σταθερός n-τάξης πίνακας C με $\det C \neq 0$ έτσι ώστε
\[
Y^* = YC.
\]
\end{Thewrhma}
\textbf{ΑΠΟΔΕΙΞΗ.} (i) Ας είναι $Y$ ένας βασικός πίνακας του $(S_0)$. Τότε (θεωρήματα 2 και 8) θα είναι
\[
Y' = AY \text{ και } \det Y(x) \neq 0 \text{ για όλα τα } x \in I.
\]
Αν λοιπόν $C$ είναι ένας $n$-τάξης σταθερός πίνακας με $\det C \neq 0$, τότε
\[
(YC)' = Y'C = (AY)C = A(YC)
\]
και
\[
\det(YC)(x) = [\det Y(x)][\det C] \neq 0 \text{ για όλα τα } x \in I.
\]
Έτσι (θεωρήματα 2 και 8), $YC$ είναι ένας βασικός πίνακας του $(S_0)$.

(ii) Ας είναι $Y$ και $Y^*$ δύο βασικοί πίνακες του $(S_0)$. Τότε (θεωρήματα 2 και 8) έχουμε
\[
Y' = AY \text{ και } (Y^*)' = AY^*
\]
και
\[
\det Y(x) \neq 0, \quad \det Y^*(x) \neq 0 \text{ για όλα τα } x \in I.
\]
Θέτουμε $C=Y^{-1}Y^*$, οπότε θα είναι $Y^*=YC$. Αρκεί ν' αποδείξουμε ότι ο $C$ είναι σταθερός και ότι $\det C \neq 0$. Έχουμε
\begin{align*}
C' &= (Y^{-1}Y^*)' = (Y^{-1})' Y^* + Y^{-1} (Y^*)' = (-Y^{-1}Y'Y^{-1})Y^* + Y^{-1} (Y^*)' \\
&= -Y^{-1}(AY)Y^{-1}Y^* + Y^{-1}AY^* = -Y^{-1}AY^* + Y^{-1}AY^* = 0
\end{align*}
και
\[
\det C = (\det Y^{-1}) \det Y^* = \det Y^* / \det Y \neq 0.
\]

\begin{Thewrhma}{}
Ας είναι $Y$ ένας βασικός πίνακας του ομογενούς γραμμικού διαφορικού συστήματος $(S_0)$ και $y$ μια λύση αυτού. Τότε υπάρχει ένα, και μόνο ένα, $n$-διάστατο διάνυσμα c έτσι ώστε
\[
y = Yc.
\]
\end{Thewrhma}

\textbf{ΑΠΟΔΕΙΞΗ.} Θεωρούμε το $n$-διάστατο διάνυσμα $c = Y^{-1}(x_0)y(x_0)$, όπου $x_0$ είναι ένα σημείο του Ι (είναι $\det Y(x_0) \neq 0$ απ' το θεώρημα 8). Η συνάρτηση $\tilde{y}=Yc$ είναι (θεώρημα 3) μια λύση του $(S_0)$. Επιπλέον, έχουμε $\tilde{y}(x_0) = Y(x_0)[Y(x_0)]^{-1}y(x_0)=y(x_0)$ και επομένως (θεώρημα 1) είναι $\tilde{y}=y$. Άρα, $y=Yc$. Αν c είναι ένα άλλο n-διάστατο διάνυσμα τέτοιο ώστε $y=Yc$, τότε $Yc=Y\tilde{c}$ και άρα $Y(c-\tilde{c})=0$, και $c-\tilde{c}=0$. Το παραπάνω θεώρημα μπορεί να διατυπωθεί ως εξής: Ας είναι $Y_k$ $(k=1,...,n)$ n γραμμικά ανεξάρτητες λύσεις του ομογενούς γραμμικού διαφορικού συστήματος $(S_0)$ και y μια λύση αυτού. Τότε υπάρχουν μονοσήμαντα ορισμένες σταθερές $c_k$ $(k=1,...,n)$ έτσι ώστε
\[
y = c_1Y_1 + \dots + c_nY_n.
\]
Απ' τα θεωρήματα 3 και 10 προκύπτει το παρακάτω θεώρημα που είναι και το πιο βασικό συμπέρασμα του Εδαφίου αυτού.

\begin{Thewrhma}{}
(i) Ας είναι $Y$ ένας βασικός πίνακας του ομογενούς γραμμικού διαφορικού συστήματος $(S_0)$. Τότε y είναι μια λύση του $(S_0)$ αν και μόνο αν υπάρχει ένα n-διάστατο διάνυσμα c έτσι ώστε
\[
y = Yc.
\]
(ii) Αν $x_0$ είναι ένα σημείο του διαστήματος Ι και Ε είναι ένα n-διάστατο διάνυσμα, τότε η λύση y του $(S_0)$ που πληροί την αρχική συνθήκη $y(x_0)=E$ δίνεται απ' τον τύπο
\[
y = Y Y^{-1}(x_0)E,
\]
όπου Y είναι ένας βασικός πίνακας του $(S_0)$.
\end{Thewrhma}

Το συμπέρασμα (i) του θεωρήματος 11 εκφράζεται και ως εξής:
Ας είναι $Y_k$ $(k=1,...,n)$ n γραμμικά ανεξάρτητες λύσεις του ομογενούς γραμμικού διαφορικού συστήματος $(S_0)$. Τότε y είναι μια λύση του $(S_0)$ αν και μόνο αν υπάρχουν σταθερές $c_k$ $(k=1,...,n)$ έτσι ώστε
\[
y = c_1Y_1 + \dots + c_nY_n.
\]

\subsubsection{Υποβιβασμός της τάξης}
Αν γνωρίζουμε m, όπου $1 \le m < n-1$, γραμμικά ανεξάρτητες λύσεις του ομογενούς γραμμικού διαφορικού συστήματος $(S_0)$, τότε μπορούμε να αναγάγουμε το $(S_0)$ σ' ένα ομογενές γραμμικό διαφορικό σύστημα τάξης n-m (δηλαδή με n-m εξισώσεις και n-m άγνωστες συναρτήσεις). Δεν αναπτύξουμε το θέμα αυτό στη γενική περίπτωση, αλλά μόνο στην ειδική περίπτωση $n=2$, δηλαδή στην περίπτωση του ομογενούς γραμμικού διαφορικού συστήματος
\[
(S_0)_2 \qquad y' = 
\begin{pmatrix}
a_{11} & a_{12} \\
a_{21} & a_{22}
\end{pmatrix}
y.
\]
Ας θεωρήσουμε μια λύση
\[
Y_1 = 
\begin{pmatrix}
y_{11} \\
y_{21}
\end{pmatrix}
\]
του ομογενούς γραμμικού διαφορικού συστήματος $(S_0)_2$ με $y_{11}(x) \neq 0$ για όλα τα $x \in I$. Στη συνέχεια, ας θέσουμε
\[
Q =
\begin{pmatrix}
y_{11} & 0 \\
y_{21} & 1
\end{pmatrix}
.
\]
Τότε με την αντικατάσταση $y=Qu$ παίρνουμε
\[
Qu' + Q'u = AQu \text{ ή } u' = Q^{-1}(AQ-Q')u.
\]
Αλλά, λαμβάνοντας υπόψη ότι η $Y_1$ είναι μια λύση του $(S_0)$, έχουμε
\begin{align*}
Q^{-1}(AQ-Q') &= \frac{1}{y_{11}}
\begin{pmatrix}
1 & 0 \\
-y_{21} & y_{11}
\end{pmatrix}
\left(
\begin{pmatrix}
a_{11} & a_{12} \\
a_{21} & a_{22}
\end{pmatrix}
\begin{pmatrix}
y_{11} & 0 \\
y_{21} & 1
\end{pmatrix}
-
\begin{pmatrix}
y'_{11} & 0 \\
y'_{21} & 0
\end{pmatrix}
\right) \\
&= \frac{1}{y_{11}}
\begin{pmatrix}
1 & 0 \\
-y_{21} & y_{11}
\end{pmatrix}
\begin{pmatrix}
a_{11}y_{11}+a_{12}y_{21}-y'_{11} & a_{12} \\
a_{21}y_{11}+a_{22}y_{21}-y'_{21} & a_{22}
\end{pmatrix} \\
&= \frac{1}{y_{11}}
\begin{pmatrix}
1 & 0 \\
-y_{21} & y_{11}
\end{pmatrix}
\begin{pmatrix}
0 & a_{12} \\
0 & a_{22}
\end{pmatrix}
= \frac{1}{y_{11}}
\begin{pmatrix}
0 & a_{12} \\
0 & -a_{12}y_{21}+a_{22}y_{11}
\end{pmatrix}
.
\end{align*}
Έτσι λοιπόν για
\[
u = 
\begin{pmatrix}
u_1 \\
u_2
\end{pmatrix}
\]
έχουμε το ομογενές γραμμικό διαφορικό σύστημα
\[
(*) \qquad
\begin{cases}
u'_1 = \frac{a_{12}}{y_{11}} u_2 \\
u'_2 = \frac{1}{y_{11}}(-a_{12}y_{21}+a_{22}y_{11})u_2
\end{cases}
.
\]
Η δεύτερη εξίσωση του (*) περιέχει μόνο την άγνωστη συνάρτηση $u_2$. Ας είναι $x_0$ ένα σημείο του διαστήματος Ι. Μια λύση του (*) είναι
\[
\begin{cases}
    u_1(x) = \int_{x_0}^{x} \frac{a_{12}(s)}{y_{11}(s)} \exp\left( \int_{x_0}^{s} \frac{-a_{12}(t)y_{21}(t)+a_{22}(t)y_{11}(t)}{y_{11}(t)} dt \right) ds, \quad x \in I \\
    u_2(x) = \exp\left( \int_{x_0}^{x} \frac{-a_{12}(t)y_{21}(t)+a_{22}(t)y_{11}(t)}{y_{11}(t)} dt \right), \quad x \in I.
\end{cases}
\]
Τότε η συνάρτηση
\[
Y_2 = Qu = 
\begin{pmatrix}
y_{11} & 0 \\
y_{21} & 1
\end{pmatrix}
\begin{pmatrix}
u_1 \\
u_2
\end{pmatrix}
=
\begin{pmatrix}
y_{11}u_1 \\
y_{21}u_1+u_2
\end{pmatrix}
\]
είναι μια λύση του ομογενούς γραμμικού διαφορικού συστήματος $(S_0)_2$. Οι λύσεις $Y_1$ και $Y_2$ του $(S_0)_2$ είναι γραμμικά ανεξάρτητες, γιατί (θεώρημα 8)
\begin{align*}
\det
\begin{pmatrix}
y_{11}(x) & y_{12}(x) \\
y_{21}(x) & y_{22}(x)
\end{pmatrix}
&=
\det
\begin{pmatrix}
y_{11}(x) & y_{11}(x)u_1(x) \\
y_{21}(x) & y_{21}(x)u_1(x)+u_2(x)
\end{pmatrix} \\
&= y_{11}(x)u_2(x) \neq 0
\end{align*}
για όλα τα $x \in I$.

Έχουμε λοιπόν αποδείξει το παρακάτω θεώρημα.

\begin{Thewrhma}{}
Ας είναι
\[
Y_1 = 
\begin{pmatrix}
y_{11} \\
y_{21}
\end{pmatrix}
\]
μια λύση του ομογενούς γραμμικού διαφορικού συστήματος $(S_0)_2$ με $y_{11}(x) \neq 0$ για όλα τα $x \in I$. Επιπλέον, ας είναι $x_0$ ένα σημείο του Ι και
\[
v(x) = \int_{x_0}^{x} \frac{-a_{12}(t)y_{21}(t)+a_{22}(t)y_{11}(t)}{y_{11}(t)} dt, \ x \in I.
\]
Τότε ένας βασικός πίνακας του $(S_0)_2$ είναι

\[
\begin{pmatrix}
Y_{11}(x) & Y_{11}(x)\int_{x_0}^{x} \frac{a_{12}(s)}{Y_{11}(s)} \exp[v(s)]ds \\
Y_{21}(x) & Y_{21}(x)\int_{x_0}^{x} \frac{a_{12}(s)}{Y_{11}(s)} \exp[v(s)]ds + \exp[v(x)]
\end{pmatrix}
, \quad x \in I.
\]
\end{Thewrhma}
\subsubsection{Παραδείγματα}
\begin{Paradeigma}{}
Δίνεται το ομογενές γραμμικό διαφορικό σύστημα
\[
y' = Ay \quad \text{με} \quad A = 
\begin{pmatrix}
0 & 1 & 0 \\
0 & 0 & 1 \\
2 & -5 & 4
\end{pmatrix}
.
\]
(i) Ν' αποδειχθεί ότι
\[
Y(x) = 
\begin{pmatrix}
e^x & xe^x & e^{2x} \\
e^x & (x+1)e^x & 2e^{2x} \\
e^x & (x+2)e^x & 4e^{2x}
\end{pmatrix}
, \quad x \in \mathbb{R}
\]
είναι ένας πίνακας λύσεων και να βρεθεί η ορίζουσα αυτού. (ii) Ν' αποδειχθεί ότι
\[
y(x) = 
\begin{pmatrix}
(x-1)e^x \\
xe^x \\
(x+1)e^x
\end{pmatrix}
, \quad x \in \mathbb{R}
\]
είναι μια λύση.
\end{Paradeigma}
(i) Για όλα τα $x \in \mathbb{R}$ έχουμε
\[
Y'(x) = 
\begin{pmatrix}
e^x & (x+1)e^x & 2e^{2x} \\
e^x & (x+2)e^x & 4e^{2x} \\
e^x & (x+3)e^x & 8e^{2x}
\end{pmatrix}
=
\begin{pmatrix}
0 & 1 & 0 \\
0 & 0 & 1 \\
2 & -5 & 4
\end{pmatrix}
\begin{pmatrix}
e^x & xe^x & e^{2x} \\
e^x & (x+1)e^x & 2e^{2x} \\
e^x & (x+2)e^x & 4e^{2x}
\end{pmatrix}
= A(x)Y(x)
\]
και επομένως (θεώρημα 2) Y είναι ένας πίνακας λύσεων. Τώρα, με τον τύπο του Jacobi (θεώρημα 4), παίρνουμε
\[
\det Y(x) = \det Y(0) e^{\int_0^x \tr A dt} = e^{4x} \det
\begin{pmatrix}
1 & 0 & 1 \\
1 & 1 & 2 \\
1 & 2 & 4
\end{pmatrix}
= e^{4x}.
\]

για $x \in \mathbb{R}$. (ii) Παρατηρούμε ότι
\[
y(x) = 
\begin{pmatrix}
(x-1)e^x \\
xe^x \\
(x+1)e^x
\end{pmatrix}
=
\begin{pmatrix}
e^x & xe^x & e^{2x} \\
e^x & (x+1)e^x & 2e^{2x} \\
e^x & (x+2)e^x & 4e^{2x}
\end{pmatrix}
\begin{pmatrix}
-1 \\
1 \\
0
\end{pmatrix}
= Y(x)
\begin{pmatrix}
-1 \\
1 \\
0
\end{pmatrix}
,
\]
οπότε, σύμφωνα με το θεώρημα 3, y είναι μια λύση.
\begin{Paradeigma}{}
Ν' αποδειχθεί ότι: (i) Οι συναρτήσεις $f_1, f_2$ και $f_3$ με
\[
f_1(x) = 
\begin{pmatrix}
\cos^2 x \\
\cos x \\
4
\end{pmatrix}
, \
f_2(x) =
\begin{pmatrix}
2\sin^2 x \\
\cos x \\
-1
\end{pmatrix}
\text{ και }
f_3(x) =
\begin{pmatrix}
2 \\
3\cos x \\
7
\end{pmatrix}
\text{ για } x \in \mathbb{R}
\]
είναι γραμμικά εξαρτημένες. (ii) Οι συναρτήσεις $g_1$ και $g_2$ με
\[
g_1(x) = 
\begin{pmatrix}
x \\
1
\end{pmatrix}
\text{ και }
g_2(x) =
\begin{pmatrix}
x \log x \\
1+\log x
\end{pmatrix}
\text{ για } x \geq 1
\]
είναι γραμμικά ανεξάρτητες.
\end{Paradeigma}
\lysh
(i) Για κάθε $x \in \mathbb{R}$ έχουμε
\[
2f_1(x)+f_2(x)-f_3(x) = 
\begin{pmatrix}
2\cos^2 x \\
2\cos x \\
8
\end{pmatrix}
+
\begin{pmatrix}
2\sin^2 x \\
\cos x \\
-1
\end{pmatrix}
-
\begin{pmatrix}
2 \\
3\cos x \\
7
\end{pmatrix}
=
\begin{pmatrix}
2\cos^2 x + 2\sin^2 x - 2 \\
2\cos x + \cos x - 3\cos x \\
8-1+7
\end{pmatrix}
=
\begin{pmatrix}
0 \\
0 \\
0
\end{pmatrix}
,
\]
δηλαδή $2f_1+f_2-f_3=0$, που αποδεικνύει τη γραμμική εξάρτηση των $f_1, f_2$ και $f_3$. (ii) Ας υποθέσουμε ότι $c_1g_1+c_2g_2=0$, όπου $c_1$ και $c_2$ είναι σταθερές. Τότε
\[
c_1g_1(x)+c_2g_2(x) = c_1
\begin{pmatrix}
x \\
1
\end{pmatrix}
+ c_2
\begin{pmatrix}
x \log x \\
1+\log x
\end{pmatrix}
=
\begin{pmatrix}
c_1x+c_2x\log x \\
c_1+c_2(1+\log x)
\end{pmatrix}
=
\begin{pmatrix}
0 \\
0
\end{pmatrix}
\]
για κάθε $x \geq 1$. Έτσι, θα είναι
\[
c_1+c_2(1+\log x)=0 \text{ για όλα τα } x \geq 1.
\]
Για $x=1$ και $x=e$ παίρνουμε $c_1+c_2=0$ και $c_1+2c_2=0$, δηλαδή $c_1=c_2=0$. Αυτό αποδεικνύει ότι οι $g_1$ και $g_2$ είναι γραμμικά ανεξάρτητες.

\begin{Paradeigma}{}
Δίνεται το ομογενές γραμμικό διαφορικό σύστημα
\[
y' = Ay \quad \text{με} \quad A = 
\begin{pmatrix}
0 & 1 & 0 \\
0 & 0 & 1 \\
2 & -5 & 4
\end{pmatrix}
.
\]
Ν' αποδειχθεί ότι: (i) Οι συναρτήσεις $y_1$ και $y_2$ με
\[
y_1(x) = 
\begin{pmatrix}
e^x \\
e^x \\
e^x
\end{pmatrix}
\text{ και }
y_2(x) =
\begin{pmatrix}
xe^x \\
(x+1)e^x \\
(x+2)e^x
\end{pmatrix}
\text{ για } x \in \mathbb{R}
\]
είναι δύο γραμμικά ανεξάρτητες λύσεις. (ii) Η συνάρτηση y με
\[
y(x) = 
\begin{pmatrix}
(2x-1)e^x \\
(2x+1)e^x \\
(2x+3)e^x
\end{pmatrix}
, x \in \mathbb{R}
\]
είναι μια λύση.
\end{Paradeigma}
\lysh
(i) Οι $y_1$ και $y_2$ είναι λύσεις γιατί για όλα τα $x \in \mathbb{R}$
\[
y_1'(x) = 
\begin{pmatrix}
e^x \\
e^x \\
e^x
\end{pmatrix}
=
\begin{pmatrix}
0 & 1 & 0 \\
0 & 0 & 1 \\
2 & -5 & 4
\end{pmatrix}
\begin{pmatrix}
e^x \\
e^x \\
e^x
\end{pmatrix}
= Ay_1(x)
\]
και
\[
y_2'(x) = 
\begin{pmatrix}
(x+1)e^x \\
(x+2)e^x \\
(x+3)e^x
\end{pmatrix}
=
\begin{pmatrix}
0 & 1 & 0 \\
0 & 0 & 1 \\
2 & -5 & 4
\end{pmatrix}
\begin{pmatrix}
xe^x \\
(x+1)e^x \\
(x+2)e^x
\end{pmatrix}
= Ay_2(x).
\]
Εξάλλου οι λύσεις αυτές είναι γραμμικά ανεξάρτητες γιατί (θεώρημα 6) τα διανύσματα
\[
y_1(0) = 
\begin{pmatrix}
1 \\
1 \\
1
\end{pmatrix}
\text{ και }
y_2(0) =
\begin{pmatrix}
0 \\
1 \\
2
\end{pmatrix}
\]
είναι, όπως εύκολα αποδεικνύεται, γραμμικά ανεξάρτητα. 
\\(ii) Βλέπουμε αμέσως ότι $y=2y_2-y_1$, και επομένως (θεώρημα 5) $y$ είναι μια λύση.

\begin{Paradeigma}{}
Δίνεται το ομογενές γραμμικό διαφορικό σύστημα
\[
y' = Ay \quad \text{με} \quad A = 
\begin{pmatrix}
0 & 1 & 0 \\
0 & 0 & 1 \\
2 & -5 & 4
\end{pmatrix}
.
\]
Ν' αποδειχθεί ότι: (i) Οι συναρτήσεις $y_1$ και $y_2$ με
\[
y_1(x) = 
\begin{pmatrix}
e^x \\
e^x \\
e^x
\end{pmatrix}
\text{ και }
y_2(x) =
\begin{pmatrix}
xe^x \\
(x+1)e^x \\
(x+2)e^x
\end{pmatrix}
\text{ για } x \in \mathbb{R}
\]
είναι δύο γραμμικά ανεξάρτητες λύσεις. (ii) Η συνάρτηση y με
\[
y(x) = 
\begin{pmatrix}
(2x-1)e^x \\
(2x+1)e^x \\
(2x+3)e^x
\end{pmatrix}
, x \in \mathbb{R}
\]
είναι μια λύση.
\end{Paradeigma}
\lysh
(i) Οι $y_1$ και $y_2$ είναι λύσεις γιατί για όλα τα $x \in \mathbb{R}$
\[
y_1'(x) = 
\begin{pmatrix}
e^x \\
e^x \\
e^x
\end{pmatrix}
=
\begin{pmatrix}
0 & 1 & 0 \\
0 & 0 & 1 \\
2 & -5 & 4
\end{pmatrix}
\begin{pmatrix}
e^x \\
e^x \\
e^x
\end{pmatrix}
= Ay_1(x)
\]
και
\[
y_2'(x) = 
\begin{pmatrix}
(x+1)e^x \\
(x+2)e^x \\
(x+3)e^x
\end{pmatrix}
=
\begin{pmatrix}
0 & 1 & 0 \\
0 & 0 & 1 \\
2 & -5 & 4
\end{pmatrix}
\begin{pmatrix}
xe^x \\
(x+1)e^x \\
(x+2)e^x
\end{pmatrix}
= Ay_2(x).
\]
Εξάλλου οι λύσεις αυτές είναι γραμμικά ανεξάρτητες γιατί (θεώρημα 6) τα διανύσματα
\[
y_1(0) = 
\begin{pmatrix}
1 \\
1 \\
1
\end{pmatrix}
\text{ και }
y_2(0) =
\begin{pmatrix}
0 \\
1 \\
2
\end{pmatrix}
\]
είναι, όπως εύκολα αποδεικνύεται, γραμμικά ανεξάρτητα. (ii) Βλέπουμε αμέσως ότι $y=2y_2-y_1$, και επομένως (θεώρημα 5) y είναι μια λύση.

\begin{Paradeigma}{}
Δίνεται το ομογενές γραμμικό διαφορικό σύστημα
\[
Y' = Ay \quad \text{με} \quad A(x) = \begin{pmatrix} 0 & 1 \\ -x^{-2} & x^{-1} \end{pmatrix}, \quad \text{για } x > 0.
\]
(i) Ν' αποδειχθεί ότι
\[
Y(x) = \begin{pmatrix} x & x\log x \\ 1 & 1+\log x \end{pmatrix}, \quad x>0
\]
είναι ένας βασικός πίνακας. (ii) Να βρεθεί ένας βασικός πίνακας $Y^*$ με
\[
Y^*(1) = \begin{pmatrix} 1 & 0 \\ 1 & -1 \end{pmatrix}.
\]
\end{Paradeigma}
\lysh
(i) Για όλα τα $x>0$ έχουμε
\[
Y'(x) = \begin{pmatrix} 1 & 1+\log x \\ 0 & x^{-1} \end{pmatrix} = \begin{pmatrix} 0 & 1 \\ -x^{-2} & x^{-1} \end{pmatrix} \begin{pmatrix} x & x\log x \\ 1 & 1+\log x \end{pmatrix} = A(x)Y(x)
\]
και $\det Y(x) = x \neq 0$.

Άρα (θεωρήματα 2 και 8) Y είναι ένας βασικός πίνακας. (ii) Σύμφωνα με το θεώρημα 9, θα είναι $Y^*=YC$ για κάποιον n-τάξης σταθερό πίνακα C με $\det C \neq 0$. Έχουμε $Y^*(1) = Y(1)C$, οπότε $C=Y^{-1}(1)Y^*(1)$. Έτσι, για κάθε $x>0$,
\[
Y^*(x) = Y(x)Y^{-1}(1)Y^*(1) = \begin{pmatrix} x & x\log x \\ 1 & 1+\log x \end{pmatrix} \begin{pmatrix} 1 & 0 \\ -1 & 1 \end{pmatrix} = \begin{pmatrix} x & -x\log x \\ 1 & -1-\log x \end{pmatrix}.
\]
\begin{Paradeigma}{}
Να επιλυθεί το ομογενές γραμμικό διαφορικό σύστημα
\[
y' = Ay \quad \text{με} \quad A = \begin{pmatrix} 0 & 1 & 0 \\ 0 & 0 & 1 \\ 2 & -5 & 4 \end{pmatrix}.
\]

αφού αποδειχθεί ότι
\[
Y(x) = \begin{pmatrix} e^x & xe^x & e^{2x} \\ e^x & (x+1)e^x & 2e^{2x} \\ e^x & (x+2)e^x & 4e^{2x} \end{pmatrix}, \quad x \in \mathbb{R}
\]
είναι ένας βασικός πίνακας αυτού. Ειδικά, να βρεθεί η λύση $y_0$ με
\[
y_0(0) = \begin{pmatrix} 1 \\ 0 \\ -1 \end{pmatrix}.
\]

\end{Paradeigma}
Στο Παράδειγμα 1 αποδείξαμε ότι $Y$ είναι ένας πίνακας λύσεων του ομογενούς γραμμικού διαφορικού συστήματος. Ακόμα, βρήκαμε ότι $\det Y(x) = e^{4x}$ για $x \in \mathbb{R}$. Έτσι, έχουμε $\det Y(x) \neq 0$ για όλα τα $x \in \mathbb{R}$ και επομένως (θεώρημα 8) $Υ$ είναι ένας βασικός πίνακας. Σύμφωνα με το θεώρημα 11, οι λύσεις y θα δίνονται απ' τον τύπο
\[
y(x) = \begin{pmatrix} e^x & xe^x & e^{2x} \\ e^x & (x+1)e^x & 2e^{2x} \\ e^x & (x+2)e^x & 4e^{2x} \end{pmatrix} \begin{pmatrix} c_1 \\ c_2 \\ c_3 \end{pmatrix} = \begin{pmatrix} c_1e^x+c_2xe^x+c_3e^{2x} \\ c_1e^x+c_2(x+1)e^x+2c_3e^{2x} \\ c_1e^x+c_2(x+2)e^x+4c_3e^{2x} \end{pmatrix}, \quad x \in \mathbb{R},
\]
όπου $c_1, c_2$ και $c_3$ είναι αυθαίρετες σταθερές. Ιδιαίτερα, για τη λύση $y_0$ θα έχουμε $c_1+c_3=1, c_1+c_2+2c_3=0$ και $c_1+2c_2+4c_3=-1$, απ' όπου προκύπτει ότι $c_1=1, c_2=-1$ και $c_3=0$. Έτσι,
\[
y_0(x) = \begin{pmatrix} (x-1)e^x \\ -xe^x \\ -(x+1)e^x \end{pmatrix}, \quad x \in \mathbb{R}.
\]
\begin{Paradeigma}{}
Να επιλυθεί το ομογενές γραμμικό διαφορικό σύστημα
\[
y' = Ay \quad \text{με} \quad A(x) = \begin{pmatrix} 0 & 1 \\ -x^{-2} & x^{-1} \end{pmatrix} \quad \text{για } x>0,
\]
αφού αποδειχθεί ότι
\[
Y(x) = \begin{pmatrix} x & x\log x \\ 1 & 1+\log x \end{pmatrix}, \quad x>0
\]

είναι ένας βασικός πίνακας. Ειδικά, να βρεθεί η λύση $y_0$ με
\[
y_0(1) = \begin{pmatrix} 1 \\ 2 \end{pmatrix}.
\]
\end{Paradeigma}
Το ότι $Υ$ είναι ένας βασικός πίνακας έχει αποδειχθεί στο Παράδειγμα 4. Σύμφωνα με το θεώρημα 11, οι λύσεις $y$ θα δίνονται απ' τον τύπο
\[
y(x) = \begin{pmatrix} x & x\log x \\ 1 & 1+\log x \end{pmatrix} \begin{pmatrix} c_1 \\ c_2 \end{pmatrix} = \begin{pmatrix} c_1x+c_2x\log x \\ c_1+c_2(1+\log x) \end{pmatrix}, \quad x>0.
\]
Ειδικά, η λύση $y_0$ είναι (θεώρημα 11)
\[
y_0(x) = Y(x)Y^{-1}(1)y_0(1) = \begin{pmatrix} x & x\log x \\ 1 & 1+\log x \end{pmatrix} \begin{pmatrix} 1 & 0 \\ -1 & 1 \end{pmatrix} \begin{pmatrix} 1 \\ 2 \end{pmatrix} = \begin{pmatrix} x(1+\log x) \\ 2+\log x \end{pmatrix}, \quad x>0.
\]
\begin{Paradeigma}{}
Να βρεθεί ένας βασικός πίνακας του ομογενούς γραμμικού διαφορικού συστήματος
\[
Y' = Ay \quad \text{με} \quad A = \begin{pmatrix} 0 & 1 \\ -2 & 3 \end{pmatrix},
\]
αφού πρώτα αποδειχθεί ότι
\[
y_1(x) = \begin{pmatrix} e^x \\ e^x \end{pmatrix}, \quad x \in \mathbb{R}
\]
είναι μια λύση του.
\end{Paradeigma}
\lysh
Για όλα τα $x \in \mathbb{R}$ είναι
\[
y_1'(x) = \begin{pmatrix} e^x \\ e^x \end{pmatrix} = \begin{pmatrix} 0 & 1 \\ -2 & 3 \end{pmatrix} \begin{pmatrix} e^x \\ e^x \end{pmatrix} = Ay_1(x),
\]
που αποδεικνύει ότι η $y_1$ είναι μια λύση. Θέτουμε για $x \in \mathbb{R}$
\[
v(x) = \int_0^x \frac{-e^s+3e^t}{e^t} dt = 2\int_0^x dt = 2x.
\]
Τότε (θεώρημα 12) ένας βασικός πίνακας του συστήματος είναι
\[
Y(x) = \begin{pmatrix} e^x & x \\ e^x & x \end{pmatrix} \begin{pmatrix} e^x \int_0^x \frac{1}{e^s} \exp[v(s)] ds \\ 0 & \exp[v(x)] \end{pmatrix}
\]
\[
= \begin{pmatrix} e^x & x \\ e^x & x \end{pmatrix} \begin{pmatrix} e^x \int_0^x \frac{1}{s} \exp[v(s)] ds + \exp[v(x)] \\ 0 & \end{pmatrix}
\]
\[
= \begin{pmatrix} e^x & x \\ e^x & x \end{pmatrix} \begin{pmatrix} e^x \int_0^x e^{2s} ds \\ 0 & e^{2x} \end{pmatrix} = \begin{pmatrix} e^x & e^{2x}-e^x \\ e^x & 2e^{2x}-e^x \end{pmatrix}, \quad x \in \mathbb{R}.
\]

\subsection{Ασκήσεις}
\begin{askhseis}
    \item Σε καθεμιά απ' τις παρακάτω περιπτώσεις να εξετασθεί αν είναι γραμμικά εξαρτημένες ή γραμμικά ανεξάρτητες οι συναρτήσεις που δίνονται:
    \begin{enumerate}
        \item[(i)] $f_1(x) = \begin{pmatrix} \sin x + \cos x \\ 2\sin x \\ -\cos x \end{pmatrix}, f_2(x) = \begin{pmatrix} 2\sin x \\ 3\sin x - \cos x \\ -\sin x \end{pmatrix}$ και $f_3(x) = \begin{pmatrix} 4\cos x \\ 2\cos x \\ 2\sin x - 4\cos x \end{pmatrix}$ για $x \in \mathbb{R}$.
        \item[(ii)] $f_1(x) = \begin{pmatrix} x \\ 1 \\ -1 \end{pmatrix}, f_2(x) = \begin{pmatrix} 1 \\ x \\ -1 \end{pmatrix}$ και $f_3(x) = \begin{pmatrix} 1 \\ -1 \\ x \end{pmatrix}$ για $x \in \mathbb{R}$.
        \item[(iii)] $f_1(x) = \begin{pmatrix} e^{-x} \\ -4e^{-x} \end{pmatrix}$ και $f_2(x) = \begin{pmatrix} xe^{-x} \\ -4xe^{-x} \end{pmatrix}$ για $x \in \mathbb{R}$.
    \item[(iv)] $f_1(x) = \begin{pmatrix} x \\ 1 \end{pmatrix}$ και $f_2(x) = \begin{pmatrix} xe^x \\ e^x \end{pmatrix}$ για $x \in (-1, 1)$.
    \item[(v)] $f_1(x) = \begin{pmatrix} x^2 \\ \sin x \end{pmatrix}$ και $f_2(x) = \begin{pmatrix} 2e^{x^2} \\ 2e^{x^2+2x} \end{pmatrix}$ για $x \in \mathbb{R}$.
\end{enumerate}
\item Ν' αποδειχθεί ότι η ορίζουσα κάθε πίνακα λύσεων του ομογενούς γραμμικού συστήματος
\[
y' = \begin{pmatrix} 1-\cos^2 x & \log(1+x^2) \\ x^2+7 & -\sin^2 x \end{pmatrix} y, \quad x \in \mathbb{R}
\]
είναι σταθερά.
\item Ας είναι $Υ$ ένας πίνακας λύσεων του ομογενούς γραμμικού διαφορικού συστήματος
\[
y' = \begin{pmatrix} x+\frac{1}{2x} & x-\frac{1}{2x} \\ x-\frac{1}{2x} & x+\frac{1}{2x} \end{pmatrix} y, \quad x \ge 1.
\]
Ν' αποδειχθεί ότι
\[
\det Y(x) = xe^{x^2-1} \det Y(1) \quad \text{για κάθε } x \ge 1.
\]
\item Ν' αποδειχθεί ότι
\[
Y(x) = \begin{pmatrix} \cos x & \sin x \\ -\sin x & \cos x \end{pmatrix}, \quad x \in \mathbb{R}
\]
είναι ένας βασικός πίνακας του ομογενούς γραμμικού διαφορικού συστήματος
\[
y' = \begin{pmatrix} 0 & 1 \\ -1 & 0 \end{pmatrix} y.
\]
Να βρεθεί, στη συνέχεια, ένας βασικός πίνακας $Y^*$ με


\[
Y^*(\frac{\pi}{2}) = \begin{pmatrix} 1 & 0 \\ 0 & 1 \end{pmatrix}.
\]
\item Να επιλυθεί το ομογενές γραμμικό διαφορικό σύστημα
\[
y' = \begin{pmatrix} 1 & 1 & 1 \\ 0 & 3 & 2 \\ 0 & 0 & 5 \end{pmatrix} y,
\]
αφού πρώτα αποδειχθεί ότι ένας βασικός πίνακας αυτού είναι
\[
Y(x) = \begin{pmatrix} e^x & e^{3x} & e^{5x} \\ 0 & 2e^{3x} & 2e^{5x} \\ 0 & 0 & 2e^{5x} \end{pmatrix}, \quad x \in \mathbb{R}.
\]
Ειδικά, να βρεθεί η λύση $y_0$ με
\[
y_0(0) = \begin{pmatrix} 1 \\ 2 \\ 3 \end{pmatrix}.
\]
\item Να επιλυθεί το ομογενές γραμμικό διαφορικό σύστημα
\[
y' = \begin{pmatrix} 3 & 1 \\ 1 & 3 \end{pmatrix} y,
\]
αφού πρώτα διαπιστωθεί ότι μια λύση του είναι η
\[
y_1(x) = e^{2x} \begin{pmatrix} 1 \\ -1 \end{pmatrix}, \quad x \in \mathbb{R}.
\]
Ειδικά, να βρεθεί η λύση $y_0$ με
\[
y_0(0) = \begin{pmatrix} 1 \\ 0 \end{pmatrix}.
\]
\item Να επιλυθούν τα ομογενή γραμμικά διαφορικά συστήματα:


\begin{enumerate}
    \item[(i)] $y' = \begin{pmatrix} 2x & 0 \\ 0 & \frac{1}{x} \end{pmatrix} y, \quad x > 0.$
    \item[(ii)] $y' = \begin{pmatrix} 2x & 1 \\ 0 & \frac{1}{x} \end{pmatrix} y, \quad x > 0.$
    \item[(iii)] $y' = \begin{pmatrix} 1 & 1 & 1 \\ 0 & 3 & 2 \\ 0 & 0 & 5 \end{pmatrix} y.$
    \item[(iv)] $y' = \begin{pmatrix} 1 & 0 & 0 \\ 0 & -1 & 0 \\ 0 & 0 & -2 \end{pmatrix} y.$
\end{enumerate}
\end{askhseis}
\subsection{Μη ομογενή γραμμικά διαφορικά συστήματα}

Στο Εδάφιο αυτό θα μελετήσουμε τα μη ομογενή γραμμικά διαφορικά συστήματα. Συγκεκριμένα, θ' αποδείξουμε (θεώρημα 13) ότι οι λύσεις του μη ομογενούς γραμμικού συστήματος (S) είναι ακριβώς τα αθροίσματα των λύσεων του αντίστοιχου ομογενούς συστήματος ($S_0$) με μια μερική λύση του (S). Θα δώσουμε (θεώρημα 14) έπειτα ένα τύπο που δίνει μια μερική λύση του (S) με τη βοήθεια ενός βασικού πίνακα του ($S_0$). Στη συνέχεια, θα δώσουμε (θεώρημα 15) τον τύπο για την εύρεση των λύσεων του μη ομογενούς γραμμικού διαφορικού συστήματος (S) όταν είναι γνωστός ένας βασικός πίνακας του ($S_0$). Τέλος, θα παραθέσουμε ορισμένα παραδείγματα και θα προτείνουμε μερικές ασκήσεις για λύση.

\subsubsection{Μερικές λύσεις. Το σύνολο των λύσεων}

Μια συγκεκριμένη λύση του μη ομογενούς γραμμικού διαφορικού συστήματος (S) λέμε ότι είναι μια μερική λύση αυτού.

\begin{Thewrhma}{13}
Ας είναι $y_{\mu}$ μια μερική λύση του μη ομογενούς γραμμικού διαφορικού συστήματος (S). Τότε y είναι μια λύση του (S) αν και μόνο αν υπάρχει μια λύση $\tilde{y}$ του αντίστοιχου ομογενούς συστήματος ($S_0$) έτσι ώστε
\[
y = \tilde{y} + y_{\mu}.
\]
\end{Thewrhma}
ΑΠΟΔΕΙΞΗ. Ας είναι $y$ μια λύση του (S), θέτουμε $\tilde{y}=y-y_{\mu}$ και έχουμε
\[
\tilde{y}' = y' - y'_{\mu} = (Ay+b) - (Ay_{\mu}+b) = A(y-y_{\mu}) = A\tilde{y},
\]
δηλαδή η $y$ είναι μια λύση του ($S_0$) και $y = \tilde{y} + y_{\mu}$. Αντίστροφα, αν $\tilde{y}$ είναι μια λύση του ($S_0$) και θέσουμε $y = \tilde{y} + y_{\mu}$, τότε
\[
y' = \tilde{y}' + y'_{\mu} = A\tilde{y} + (Ay_{\mu}+b) = A(\tilde{y}+y_{\mu})+b = Ay+b,
\]
δηλαδή η y είναι μια λύση του (S).

\begin{Thewrhma}{14}
Αν $x_0$ είναι ένα σημείο του διαστήματος $Ι$ και $Υ$ είναι ένας βασικός πίνακας του ομογενούς γραμμικού διαφορικού συστήματος ($S_0$), τότε
\[
y_{\mu}(x) = Y(x) \int_{x_0}^{x} Y^{-1}(t)b(t)dt, \quad x \in I
\]
είναι μια μερική λύση του μη ομογενούς γραμμικού διαφορικού συστήματος (S). Επιπλέον, η λύση αυτή πληροί την αρχική συνθήκη $y_{\mu}(x_0)=0$.
\end{Thewrhma}

ΑΠΟΔΕΙΞΗ. Ας είναι $x_0 \in I$ και Y ένας βασικός πίνακας του ($S_0$). Τότε (θεωρήματα 2 και 8) είναι $Y'=AY$ και $\det Y(x) \ne 0$ για όλα τα $x \in I$. Έτσι, έχει νόημα ο $Y^{-1}$ και για κάθε $x \in I$ παίρνουμε
\[
y'_{\mu}(x) = Y'(x) \int_{x_0}^{x} Y^{-1}(t)b(t)dt + Y(x)Y^{-1}(x)b(x) = A(x)Y(x) \int_{x_0}^{x} Y^{-1}(t)b(t)dt + b(x) = A(x)y_{\mu}(x)+b(x),
\]
δηλαδή $y_{\mu}$ είναι μια λύση του (S). Είναι φανερό ότι $y_{\mu}(x_0)=0$.

Συνδυάζοντας τώρα τα θεωρήματα 11, 13 και 14, παίρνουμε το παρακάτω θεώρημα.

\begin{Thewrhma}{15}
Ας είναι $x_0$ ένα σημείο του διαστήματος $Ι$ και $Y$ ένας βασικός πίνακας του ομογενούς γραμμικού διαφορικού συστήματος ($S_0$). Τότε η λύση του μη ομογενούς γραμμικού διαφορικού συστήματος (S) αν και μόνο αν υπάρχει ένα $n$-διάστατο διάνυσμα $c$ έτσι ώστε
\[
y(x) = Y(x)c + Y(x) \int_{x_0}^{x} Y^{-1}(t)b(t)dt \quad \text{για όλα τα } x \in I.
\]
Επιπλέον, αν $Ε$ είναι ένα $n$-διάστατο διάνυσμα, τότε η λύση $y$ του (S) που πληροί την αρχική συνθήκη $y(x_0)=E$ δίνεται απ' τον τύπο
\[
y(x) = Y(x) \left[ Y^{-1}(x_0)E + \int_{x_0}^{x} Y^{-1}(t)b(t)dt \right], \quad x \in I.
\]
\end{Thewrhma}
\subsubsection{Παραδείγματα}

\begin{Paradeigma}{1}
Ένα μη ομογενές γραμμικό διαφορικό σύστημα με διάστημα ορισμού το $(0, \infty)$ έχει τις λύσεις
\[
y_1(x) = \begin{pmatrix} 2x \\ 0 \end{pmatrix}, y_2(x) = \begin{pmatrix} x(1+\log x) \\ \log x \end{pmatrix} \text{ και } y_3(x) = \begin{pmatrix} x \\ -1 \end{pmatrix} \text{ για } x > 0.
\]
Να επιλυθεί αυτό και, ειδικά, να βρεθεί η λύση $y_0$ με
\[
y_0(1) = \begin{pmatrix} 2 \\ 5 \end{pmatrix}.
\]
\end{Paradeigma}
Το αντίστοιχο ομογενές γραμμικό διαφορικό σύστημα θα έχει (θεώρημα 13) τις λύσεις $\tilde{y}_1 = y_1-y_3$ και $\tilde{y}_2 = y_2-y_3$. Είναι
\[
\tilde{y}_1(x) = \begin{pmatrix} x \\ 1 \end{pmatrix} \text{ και } \tilde{y}_2(x) = \begin{pmatrix} x\log x \\ 1+\log x \end{pmatrix} \text{ για } x > 0.
\]
Τότε
\[
Y(x) = \begin{pmatrix} x & x\log x \\ 1 & 1+\log x \end{pmatrix}, \quad x>0
\]
θα είναι ένας βασικός πίνακας του ομογενούς συστήματος γιατί (θεώρημα 8) $\det Y(x) = x \ne 0$ για κάθε $x \in (0, \infty)$. Σύμφωνα με το θεώρημα 11, οι λύσεις $\tilde{y}$ του ομογενούς συστήματος θα δίνονται απ' τον τύπο
\[
\tilde{y}(x) = \begin{pmatrix} x & x\log x \\ 1 & 1+\log x \end{pmatrix} \begin{pmatrix} c_1 \\ c_2 \end{pmatrix} = \begin{pmatrix} c_1x+c_2x\log x \\ c_1+c_2(1+\log x) \end{pmatrix}, \quad x>0,
\]
όπου $c_1, c_2$ είναι αυθαίρετες σταθερές. Έτσι (θεώρημα 13), οι λύσεις y του μη ομογενούς μας γραμμικού διαφορικού συστήματος είναι
\[
y(x) = \tilde{y}(x) + y_3(x) = \begin{pmatrix} c_1x+c_2x\log x+x \\ c_1+c_2(1+\log x)-1 \end{pmatrix}, \quad x > 0.
\]
Ειδικά, για τη λύση $y_0$ έχουμε $c_1+1=2$ και $c_1+c_2-1=5$, δηλαδή $c_1=1$ και $c_2=5$, και επομένως
\[
y_0(x) = \begin{pmatrix} 2x+5x\log x \\ 5+5\log x \end{pmatrix}, \quad x>0.
\]

\begin{Paradeigma}{2}
Να επιλυθεί το μη ομογενές γραμμικό διαφορικό σύστημα
\[
y' = Ay+b \text{ με } A = \begin{pmatrix} 0 & 1 & 0 \\ 0 & 0 & 1 \\ 2 & -5 & 4 \end{pmatrix} \text{ και } b(x) = \begin{pmatrix} e^x \\ e^x \\ 0 \end{pmatrix}, \quad x \in \mathbb{R},
\]
αφού διαπιστωθεί ότι
\[
Y(x) = \begin{pmatrix} e^x & xe^x & e^{2x} \\ e^x & (x+1)e^x & 2e^{2x} \\ e^x & (x+2)e^x & 4e^{2x} \end{pmatrix}, \quad x \in \mathbb{R}
\]
είναι ένας βασικός πίνακας του αντίστοιχου ομογενούς γραμμικού συστήματος. Ειδικά, να βρεθεί η λύση $y_0$ που πληροί την αρχική συνθήκη
\[
y_0(0) = \begin{pmatrix} -1 \\ 1 \\ 0 \end{pmatrix}.
\]
\end{Paradeigma}
Ο $Y$ είναι (Παράδειγμα 1 και 5 του Εδαφίου 1, Παράγραφος 1.4) ένας βασικός πίνακας του αντίστοιχου ομογενούς γραμμικού διαφορικού συστήματος. Για κάθε $x \in \mathbb{R}$
\[
Y^{-1}(x)b(x) = \begin{pmatrix} 2xe^{-x} & -(3x-2)e^{-x} & (x-1)e^{-x} \\ -2e^{-x} & 3e^{-x} & -e^{-x} \\ e^{-2x} & -2e^{-2x} & e^{-2x} \end{pmatrix} \begin{pmatrix} e^x \\ e^x \\ 0 \end{pmatrix} = \begin{pmatrix} -x+2 \\ 1 \\ -e^{-x} \end{pmatrix}
\]
και έτσι
\[
\int_0^x Y^{-1}(t)b(t)dt = \int_0^x \begin{pmatrix} -t+2 \\ 1 \\ -e^{-t} \end{pmatrix} dt = \begin{pmatrix} -\frac{x^2}{2}+2x \\ x \\ e^{-x}-1 \end{pmatrix}.
\]
Άρα (θεώρημα 15), όλες οι λύσεις του διαφορικού μας συστήματος είναι
\[
y(x) = \begin{pmatrix} e^x & xe^x & e^{2x} \\ e^x & (x+1)e^x & 2e^{2x} \\ e^x & (x+2)e^x & 4e^{2x} \end{pmatrix} \left[ \begin{pmatrix} c_1 \\ c_2 \\ c_3 \end{pmatrix} + \begin{pmatrix} -\frac{x^2}{2}+2x \\ x \\ e^{-x}-1 \end{pmatrix} \right]
\]
\[
= \begin{pmatrix} e^x & xe^x & e^{2x} \\ e^x & (x+1)e^x & 2e^{2x} \\ e^x & (x+2)e^x & 4e^{2x} \end{pmatrix} \begin{pmatrix} c_1-\frac{x^2}{2}+2x \\ c_2+x \\ c_3+e^{-x}-1 \end{pmatrix}
\]
\[
= e^x \begin{pmatrix} 1 \\ 1 \\ 1 \end{pmatrix} \left( c_1 - \frac{x^2}{2} + 2x \right) + e^x \begin{pmatrix} x \\ x+1 \\ x+2 \end{pmatrix} (c_2+x) + e^{2x} \begin{pmatrix} 1 \\ 2 \\ 4 \end{pmatrix} (c_3+e^{-x}-1)
\]
για $x \in \mathbb{R}$. Ειδικά, για τη λύση $y_0$ έχουμε
$c_1+1=0$, $c_1+c_2+2(c_3-1)=-1$, $c_1+2c_2+4(c_3-1)=0$
απ' όπου προκύπτει $c_1=-2$, $c_2=-5$, $c_3=3$, και άρα
\[
y_0(x) = \begin{pmatrix} e^x(\frac{x^2}{2}-1-3x+2e^x) \\ e^x(\frac{x^2}{2}-5-2x+4e^x) \\ e^x(\frac{x^2}{2}-8-x+8e^x) \end{pmatrix}, \quad x \in \mathbb{R}.
\]
\begin{Paradeigma}{3}
Να επιλυθεί το πρόβλημα αρχικών τιμών
\[
y' = \begin{pmatrix} 0 & 1 \\ -x^{-2} & x^{-1} \end{pmatrix} y + \begin{pmatrix} x^2 \\ x \end{pmatrix}, \quad x>0; \quad y(1) = \begin{pmatrix} 1 \\ -1 \end{pmatrix},
\]
αφού πρώτα αποδειχθεί ότι ένας βασικός πίνακας του αντίστοιχου ομογενούς γραμμικού διαφορικού συστήματος είναι
\[
Y(x) = \begin{pmatrix} x & x\log x \\ 1 & 1+\log x \end{pmatrix}, \quad x>0.
\]
\end{Paradeigma}
\lysh
Το γεγονός ότι $Y$ είναι ένας βασικός πίνακας του αντίστοιχου ομογενούς συστήματος έχει αποδειχθεί στο Παράδειγμα 4 του Εδαφίου 1 (Παράγραφος 1.4). Τώρα, η ζητούμενη λύση $y$ είναι, σύμφωνα με το θεώρημα 15,
\begin{align*}
y(x) &= Y(x) \left[ Y^{-1}(1)y(1) + \int_1^x Y^{-1}(t) \begin{pmatrix} t^2 \\ t \end{pmatrix} dt \right] \\
&= \begin{pmatrix} x & x\log x \\ 1 & 1+\log x \end{pmatrix} \left[ \begin{pmatrix} 1 & 0 \\ -1 & 1 \end{pmatrix} \begin{pmatrix} 1 \\ -1 \end{pmatrix} + \int_1^x \begin{pmatrix} 1+\log t & -t\log t \\ -1 & t \end{pmatrix} \begin{pmatrix} t^2 \\ t \end{pmatrix} dt \right] \\
&= \begin{pmatrix} x & x\log x \\ 1 & 1+\log x \end{pmatrix} \left[ \begin{pmatrix} 1 \\ -2 \end{pmatrix} + \int_1^x \begin{pmatrix} t^2 \\ 0 \end{pmatrix} dt \right] = \begin{pmatrix} x & x\log x \\ 1 & 1+\log x \end{pmatrix} \begin{pmatrix} \frac{1}{3}(x^3+1) \\ -2 \end{pmatrix} \\
&= \begin{pmatrix} \frac{1}{3}x(x^3+1)-2x\log x \\ \frac{1}{3}(x^3-3)-2\log x \end{pmatrix}, \quad x>0.
\end{align*}
\subsection{Ασκήσεις}
\begin{askhseis}
\item Να επιλυθεί το μη ομογενές γραμμικό διαφορικό σύστημα
\[
y' = \begin{pmatrix} 0 & 1 \\ -1 & 0 \end{pmatrix} y + \begin{pmatrix} x^2 \\ -x \end{pmatrix}, \quad x \in \mathbb{R},
\]
αφού πρώτα διαπιστωθεί ότι
\[
Y(x) = \begin{pmatrix} \cos x & \sin x \\ -\sin x & \cos x \end{pmatrix}, \quad x \in \mathbb{R}
\]
είναι ένας βασικός πίνακας του αντίστοιχου ομογενούς γραμμικού συστήματος. Ειδικά, να βρεθεί η λύση $y_0$ με
\[
y_0(\frac{\pi}{2}) = \begin{pmatrix} 1 \\ 2 \end{pmatrix}.
\]

\item Να επιλυθεί το πρόβλημα αρχικών τιμών
\[
y' = \begin{pmatrix} 0 & 1 \\ -2 & 3 \end{pmatrix} y + \begin{pmatrix} x \\ e^x \end{pmatrix}, \quad x \in \mathbb{R}; \quad y(0) = \begin{pmatrix} 1 \\ -2 \end{pmatrix},
\]
αφού πρώτα αποδειχθεί ότι μία λύση του αντίστοιχου ομογενούς γραμμικού συστήματος είναι η
\[
y_1(x) = \begin{pmatrix} e^x \\ e^x \end{pmatrix}, \quad x \in \mathbb{R}.
\]

\item Να επιλυθούν τα προβλήματα αρχικών τιμών:
\begin{enumerate}
    \item[(i)] $y' = \begin{pmatrix} 2x & 0 \\ 0 & \frac{1}{x} \end{pmatrix} y + \begin{pmatrix} x \\ 1 \end{pmatrix}, \quad x>0; \quad y(1) = \begin{pmatrix} 1 \\ 0 \end{pmatrix}.$
    \item[(ii)] $y' = \begin{pmatrix} 2x & 1 \\ 0 & 1 \end{pmatrix} y + \begin{pmatrix} x \\ e^x \end{pmatrix}, \quad x \in \mathbb{R}; \quad y(0) = \begin{pmatrix} 1 \\ -2 \end{pmatrix}.$
\end{enumerate}
\begin{enumerate}
    \item[4.] Να επιλυθεί το μη ομογενές γραμμικό διαφορικό σύστημα
    \[
    y' = \begin{pmatrix} 0 & 1 \\ -2 & 3 \end{pmatrix} y + \begin{pmatrix} x \\ e^x \end{pmatrix}, \quad x \in \mathbb{R},
    \]
    με το δεδομένο ότι το αντίστοιχο ομογενές γραμμικό σύστημα δέχεται λύσεις της μορφής $ce^{\lambda x}$, $x \in \mathbb{R}$, όπου $c \neq 0$ είναι 2-διάστατο διάνυσμα και $\lambda$ είναι σταθερά.
\end{enumerate}
\end{askhseis}
\section{Ομογενή Γραμμικά Διαφορικά Συστήματα με Σταθερούς Συντελεστές}

Το εδάφιο αυτό αναφέρεται στα ομογενή γραμμικά διαφορικά συστήματα με σταθερούς συντελεστές, δηλαδή εδώ μελετάται το ομογενές γραμμικό διαφορικό σύστημα $(S_0)$ όπου ο συντελεστής πίνακας Α είναι σταθερός. Το διάστημα ορισμού του $(S_0)$ είναι σ' αυτή την περίπτωση ολόκληρη η πραγματική ευθεία. Αποδεικνύεται ότι για $x \in \mathbb{R}$ ο $e^{xA}$ είναι ένας βασικός πίνακας του ομογενούς γραμμικού διαφορικού συστήματος $(S_0)$ και εκφράζονται οι λύσεις του $(S_0)$ με τη βοήθεια αυτού του βασικού πίνακα (θεώρημα 16). Στη συνέχεια, δίνεται μία μέθοδος (οφειλόμενη στον Putzer [E.J. Putzer, Avoiding the Jordan canonical form in the discussion of linear systems with constant coefficients, Amer. Math. Monthly, 73(1966), 2-7]) για την εύρεση του βασικού πίνακα $e^{xA}$, $x \in \mathbb{R}$ (θεώρημα 17). Η περίπτωση όπου ο Α είναι πραγματικός εξετάζεται ιδιαίτερα (θεώρημα 18). Τέλος, παρατίθενται μερικά παραδείγματα και δίνονται ασκήσεις για λύση.

\subsection{Ο βασικός πίνακας $e^{xA}$}

Το παρακάτω θεώρημα εξασφαλίζει ότι ο $e^{xA}$, $x \in \mathbb{R}$ είναι ένας βασικός πίνακας του ομογενούς γραμμικού διαφορικού συστήματος $(S_0)$ και δίνει την έκφραση των λύσεων του $(S_0)$ με τη βοήθεια αυτού του βασικού πίνακα.
\begin{Thewrhma}{16}
Ο πίνακας-συνάρτηση $e^{xA}$, $x \in \mathbb{R}$ είναι ένας βασικός πίνακας του ομογενούς γραμμικού διαφορικού συστήματος $(S_0)$. Επιπλέον, αν y είναι μία λύση του $(S_0)$ αν και μόνο αν υπάρχει ένα n-διάστατο διάνυσμα $c$ έτσι ώστε

\[
y(x) = e^{xA}c, \quad x \in \mathbb{R}.
\]
Ειδικά, η λύση y του $(S_0)$ που πληροί την αρχική συνθήκη $y(x_0) = \Xi$, όπου $x_0 \in \mathbb{R}$ και $\Xi$ είναι ένα n-διάστατο διάνυσμα, δίνεται απ' τον τύπο
\[
y(x) = e^{(x-x_0)A}\Xi, \quad x \in \mathbb{R}.
\]
\end{Thewrhma}
ΑΠΟΔΕΙΞΗ\\
Έχουμε
\[
(e^{xA})' = Ae^{xA} \quad \text{για κάθε } x \in \mathbb{R}.
\]
Επίσης, ο τύπος του Jacobi (θεώρημα 4) δίνει
\[
\det e^{xA} = (\det e^0) e^{\int_0^x \text{tr}A \, dt} = e^{x\text{tr}A} \neq 0
\]
για όλα τα $x \in \mathbb{R}$. Έτσι (θεωρήματα 2 και 8) ο $e^{xA}$, $x \in \mathbb{R}$ είναι ένας βασικός πίνακας του ομογενούς γραμμικού διαφορικού συστήματος $(S_0)$. Εφαρμόζοντας το θεώρημα 11, διδάσκουμε ότι η y είναι μία λύση του $(S_0)$ αν και μόνο αν $y(x) = e^{xA}c$, $x \in \mathbb{R}$ για κάποιο n-διάστατο διάνυσμα c. Αν τέλος $x_0 \in \mathbb{R}$ και $\Xi$ είναι ένα n-διάστατο διάνυσμα, τότε (θεώρημα 11) η λύση y με $y(x_0) = \Xi$ είναι
\[
y(x) = e^{xA}(e^{x_0A})^{-1}\Xi = e^{xA}e^{-x_0A}\Xi = e^{(x-x_0)A}\Xi, \quad x \in \mathbb{R}.
\]
Η εύρεση του $e^{xA}$, $x \in \mathbb{R}$ δεν μπορεί να γίνει με τη βοήθεια του ορισμού
\[
e^{xA} = I + \sum_{\nu=1}^{\infty} \frac{x^\nu A^\nu}{\nu!}, \quad x \in \mathbb{R}
\]
παρά μόνο σε ειδικές περιπτώσεις για τον πίνακα Α. Η πιο απλή τέτοια περίπτωση είναι αυτή όπου ο Α είναι διαγώνιος. Αν λοιπόν
\[
A = \text{diag}[q_1, \dots, q_n],
\]
τότε
\[
A^\nu = \text{diag}[q_1^\nu, \dots, q_n^\nu] \quad (\nu=1,2,\dots)
\]
και έτσι για κάθε $x \in \mathbb{R}$ είναι
\[
e^{xA} = \text{diag}\left[1+\sum_{\nu=1}^{\infty}\frac{x^\nu q_1^\nu}{\nu!}, \dots, 1+\sum_{\nu=1}^{\infty}\frac{x^\nu q_n^\nu}{\nu!}\right] = \text{diag}[e^{xq_1}, \dots, e^{xq_n}].
\]
Έτσι, στη γενική περίπτωση οποιουδήποτε Α, το πρόβλημα της εύρεσης του βασικού πίνακα $e^{xA}$, $x \in \mathbb{R}$ για ν' αντιμετωπισθεί απαιτεί ιδιαίτερες μεθόδους.

Το επόμενο θεώρημα οφείλεται στον \eng{Putzer} και δίνει μία μέθοδο για την εύρεση του βασικού πίνακα $e^{xA}$, $x \in \mathbb{R}$ με τη βοήθεια των ιδιοτιμών του πίνακα Α. Το συμπέρασμα αυτό είναι σχετικά πρόσφατο (1966).

\begin{Thewrhma}{17 (\eng{Putzer})}
Ας είναι $\lambda_1, \dots, \lambda_n$ οι ιδιοτιμές του πίνακα Α (όχι αναγκαστικά διακεκριμένες) και
\[
P_k = \prod_{j=1}^{k} (A - \lambda_j I) \quad (k=1, \dots, n-1).
\]
Ας είναι ακόμα $r_1, \dots, r_n$ η λύση του ομογενούς γραμμικού διαφορικού συστήματος
\[
(*) \quad r_1' = \lambda_1 r_1, \quad r_i' = r_{i-1} + \lambda_i r_i \quad (i=2, \dots, n)
\]
που πληροί την αρχική συνθήκη
\[
r_1(0) = 1, \quad r_i(0) = 0 \quad (i=2, \dots, n).
\]
Τότε
\[
e^{xA} = r_1(x)I + \sum_{k=1}^{n-1} r_{k+1}(x)P_k, \quad x \in \mathbb{R}.
\]
\end{Thewrhma}

ΑΠΟΔΕΙΞΗ\\
Θέτουμε
\[
Y = r_1 I + \sum_{k=1}^{n-1} r_{k+1} P_k.
\]
Τότε
\[
Y' = r_1' I + \sum_{k=1}^{n-1} r_{k+1}' P_k = \lambda_1 r_1 I + \sum_{k=1}^{n-1} (r_k + \lambda_{k+1} r_{k+1}) P_k.
\]
Θέτοντας
\[
P_n = \prod_{j=1}^{n} (A - \lambda_j I),
\]
παίρνουμε
\begin{align*}
AY - Y' &= r_1 A + \sum_{k=1}^{n-1} r_{k+1} A P_k - \lambda_1 r_1 I - \sum_{k=1}^{n-1} (r_k + \lambda_{k+1} r_{k+1}) P_k \\
&= r_1(A - \lambda_1 I) + \sum_{k=1}^{n-1} r_{k+1} (A - \lambda_{k+1} I) P_k - \sum_{k=1}^{n-1} r_k P_k \\
&= r_1 P_1 + \sum_{k=1}^{n-1} r_{k+1} P_{k+1} - \sum_{k=1}^{n-1} r_k P_k = r_n P_n.
\end{align*}
Αλλά, σύμφωνα με το θεώρημα \eng{Cayley-Hamilton}, $P_n$ είναι ο μηδενικός n-τάξης πίνακας. Άρα $AY-Y'=0$, δηλαδή
\[
Y' = AY.
\]
Τώρα, έχουμε
\[
Y(0) = r_1(0)I + \sum_{k=1}^{n-1} r_{k+1}(0)P_k = I
\]
και επομένως (θεώρημα 4)
\[
\det Y(x) = [\det Y(0)] e^{\int_0^x \text{tr}A \, dt} = e^{x\text{tr}A} \neq 0
\]
για όλα τα $x \in \mathbb{R}$. Έτσι (θεωρήματα 2 και 8), $Y$ είναι ένας βασικός πίνακας του ομογενούς γραμμικού διαφορικού συστήματος $(S_0)$. Επειδή ο $e^{xA}$, $x \in \mathbb{R}$ είναι επίσης (θεώρημα 16) ένας βασικός πίνακας του $(S_0)$, θα υπάρχει (θεώρημα 9) ένας σταθερός n-τάξης πίνακας $C$ με $\det C \neq 0$, έτσι ώστε
\[
e^{xA} = Y(x)C \quad \text{για κάθε } x \in \mathbb{R}.
\]
Αλλά είναι $e^{0A} = e^0 = I$ και $Y(0)=I$. Άρα $C=I$ και επομένως
\[
e^{xA} = Y(x), \quad x \in \mathbb{R}.
\]
Ας παρατηρήσουμε ότι η λύση $r_1, \dots, r_n$ του ομογενούς γραμμικού διαφορικού συστήματος (*) (που έχει σταθερούς συντελεστές), η οποία πληροί την αρχική συνθήκη $r_1(0)=1$, $r_i(0)=0$ ($i=2, \dots, n$), δίνεται αναγωγικά ως εξής:
\[
r_1(x) = e^{\lambda_1 x}, \quad x \in \mathbb{R} \quad \text{και} \quad r_i(x) = e^{\lambda_i x} \int_0^x e^{-\lambda_i t} r_{i-1}(t) \, dt, \quad x \in \mathbb{R} \quad (i=2, \dots, n).
\]
Ας θεωρήσουμε τώρα την περίπτωση όπου ο σταθερός πίνακας $Α$ είναι πραγματικός. Γι' αυτή την περίπτωση έχουμε το παρακάτω θεώρημα.

\begin{Thewrhma}{18}
Ας υποθέσουμε ότι ο πίνακας $Α$ είναι πραγματικός. Τότε για το ομογενές γραμμικό διαφορικό σύστημα $(S_0)$ ισχύουν τα παρακάτω:
\begin{enumerate}
    \item[(i)] Αν y είναι μία λύση, τότε Re y και Im y είναι επίσης λύσεις.
    \item[(ii)] Κάθε λύση με αρχική τιμή ένα πραγματικό n-διάστατο διάνυσμα είναι πραγματική (με την έννοια ότι έχει πραγματικές συνιστώσες).

    \item[(iii)] Ας είναι Y ένας πραγματικός βασικός πίνακας. Τότε y είναι ναι μια πραγματική λύση αν και μόνο αν υπάρχει ένα πραγματικό n-διάστατο διάνυσμα c έτσι ώστε $y = Yc$.
    \item[(iv)] Ο πίνακας-συνάρτηση $e^{xA}$, $x \in \mathbb{R}$ είναι ένας πραγματικός βασικός πίνακας.
\end{enumerate}
\end{Thewrhma}
ΑΠΟΔΕΙΞΗ\\
(i) Αν y είναι μια λύση του $(S_0)$, τότε
\[
0 = y' - Ay = [(\text{Re } y)' + i(\text{Im } y)'] - A[(\text{Re } y) + i(\text{Im } y)] = 
\]
και άρα
\[
[(\text{Re } y)' - A(\text{Re } y)] + i[(\text{Im } y)' - A(\text{Im } y)]
\]
\[
(\text{Re } y)' - A(\text{Re } y) = 0 \quad \text{και} \quad (\text{Im } y)' - A(\text{Im } y) = 0,
\]
δηλαδή Re y και Im y είναι λύσεις του $(S_0)$.

(iii) Αν y είναι μια λύση του $(S_0)$ με $y(x_0) = \Xi$, όπου $x_0 \in \mathbb{R}$ και $\Xi$ είναι ένα πραγματικό n-διάστατο διάνυσμα, τότε Im y είναι μια λύση του $(S_0)$ με $(\text{Im} y)(x_0) = 0$. Άρα Im y είναι η μηδενική λύση του $(S_0)$ και άρα η λύση y είναι πραγματική.

(iii) Αν c είναι ένα πραγματικό n-διάστατο διάνυσμα, τότε (θεώρημα 11) $y=Yc$ είναι μια πραγματική λύση του $(S_0)$. Αντίστροφα,αν y είναι μια πραγματική λύση του $(S_0)$, τότε (θεώρημα 11) υπάρχει ένα n-διάστατο διάνυσμα c έτσι ώστε $y=Yc$. Τότε το c είναι πραγματικό, γιατί $0 = \text{Im } c = \text{Im}(Yc) = Y(\text{Im } c)$ και άρα $\text{Im } c = 0$ (επειδή $\det Y(x) \neq 0$ για όλα τα $x \in \mathbb{R}$, σύμφωνα με το θεώρημα 8).

(iv) Είναι φανερό.
\subsubsection{Παραδείγματα}

\begin{Paradeigma}{}
Να βρεθεί ο πίνακας-συνάρτηση $e^{xA}$, $x \in \mathbb{R}$ σε καθεμιά απ' τις παρακάτω περιπτώσεις:
\begin{enumerate} 
\item[(i)] $A = \begin{pmatrix} 1 & 0 & 0 \\ 0 & -2 & 0 \\ 0 & 0 & -1 \end{pmatrix}$. \item[(ii)] $A = \begin{pmatrix} 0 & 2 \\ -1 & 0 \end{pmatrix}$. 
\item[(iii)] $A = \begin{pmatrix} 2 & 1 & 0 \\ 0 & 2 & 1 \\ 0 & 0 & 2 \end{pmatrix}$. \end{enumerate} 
\end{Paradeigma}
(i) Αν \[ A^\nu = \begin{pmatrix} 1 & 0 & 0 \\ 0 & (-2)^\nu & 0 \\ 0 & 0 & (-1)^\nu \end{pmatrix} \quad (\nu=1,2,\dots) \]
και έτσι για κάθε $x \in \mathbb{R}$ έχουμε
\[
e^{xA} = I + \sum_{\nu=1}^{\infty} \frac{x^\nu A^\nu}{\nu!} = \begin{pmatrix} 1 & 0 \\ 0 & 1 \end{pmatrix} + \sum_{\nu=1}^{\infty} \frac{x^\nu}{\nu!} \begin{pmatrix} 1 & 0 \\ 0 & 0 \end{pmatrix} + \sum_{\nu=1}^{\infty} \frac{x^\nu}{\nu!} \begin{pmatrix} 0 & 0 \\ 0 & (-2\text{x})^\nu \end{pmatrix} \begin{pmatrix} 0 & 0 \\ 0 & (-x)^\nu \end{pmatrix}
\]
\[
= \begin{pmatrix} 1+\sum_{\nu=1}^{\infty} \frac{x^\nu}{\nu!} & 0 \\ 0 & 1+\sum_{\nu=1}^{\infty} \frac{(-2x)^\nu}{\nu!} \end{pmatrix} \begin{pmatrix} 0 & 0 \\ 0 & 1+\sum_{\nu=1}^{\infty} \frac{(-x)^\nu}{\nu!} \end{pmatrix} = \begin{pmatrix} e^x & 0 & 0 \\ 0 & e^{-2x} & 0 \\ 0 & 0 & e^{-x} \end{pmatrix}.
\]
(ii) Είναι εύκολο να διαπιστώσουμε ότι
\[
A^{2\nu} = (-2)^\nu I \quad \text{και} \quad A^{2\nu-1} = (-2)^{\nu-1} A \quad (\nu=1,2,\dots).
\]
Έτσι, για κάθε $x \in \mathbb{R}$ παίρνουμε
\[
e^{xA} = I + \sum_{\nu=1}^{\infty} \frac{x^\nu A^\nu}{\nu!} = I + \sum_{\nu=1}^{\infty} \frac{x^{2\nu} A^{2\nu}}{(2\nu)!} + \sum_{\nu=1}^{\infty} \frac{x^{2\nu-1} A^{2\nu-1}}{(2\nu-1)!}
\]
\[
= I + \sum_{\nu=1}^{\infty} \frac{x^{2\nu} (-2)^\nu I}{(2\nu)!} + \sum_{\nu=1}^{\infty} \frac{x^{2\nu-1} (-2)^{\nu-1} A}{(2\nu-1)!} 
\]
\[
= \left[ 1 + \sum_{\nu=1}^{\infty} \frac{x^{2\nu} (-2)^\nu}{(2\nu)!} \right] I + \left[ \sum_{\nu=1}^{\infty} \frac{x^{2\nu-1} (-2)^{\nu-1}}{(2\nu-1)!} \right] A 
\]
\[
= \left[ 1 + \sum_{\nu=1}^{\infty} \frac{(-1)^\nu (x\sqrt{2})^{2\nu}}{(2\nu)!} \right] I + \frac{1}{\sqrt{2}} \left[ \sum_{\nu=1}^{\infty} \frac{(-1)^{\nu-1} (x\sqrt{2})^{2\nu-1}}{(2\nu-1)!} \right] A 
\]
\[
= \left( \cos x\sqrt{2} \right) I + \frac{1}{\sqrt{2}} \left( \sin x\sqrt{2} \right) A 
\]
\[
= \left( \cos x\sqrt{2} \right) \begin{pmatrix} 1 & 0 \\ 0 & 1 \end{pmatrix} + \frac{1}{\sqrt{2}} \left( \sin x\sqrt{2} \right) \begin{pmatrix} 0 & 2 \\ -1 & 0 \end{pmatrix}
\]
\[
= \begin{pmatrix} \cos x\sqrt{2} & \sqrt{2} \sin x\sqrt{2} \\ -\frac{1}{\sqrt{2}} \sin x\sqrt{2} & \cos x\sqrt{2} \end{pmatrix},
\]
δεδομένου ότι για κάθε $t \in \mathbb{R}$
\[
\cos t = 1 + \sum_{\nu=1}^{\infty} \frac{(-1)^\nu t^{2\nu}}{(2\nu)!} \quad \text{και} \quad \sin t = \sum_{\nu=1}^{\infty} \frac{(-1)^{\nu-1} t^{2\nu-1}}{(2\nu-1)!}.
\]
(iii) Παρατηρούμε ότι $A = 2I+B$, όπου
\[
B = \begin{pmatrix} 0 & 1 & 0 \\ 0 & 0 & 1 \\ 0 & 0 & 0 \end{pmatrix}
\]
και ότι
\[
B^\nu = 0 \quad (\nu=3,4,\dots).
\]
Έτσι, για $x \in \mathbb{R}$ παίρνουμε
\[
e^{xA} = e^{2xI+xB} = e^{2xI} e^{xB} = \left[ I + \sum_{\nu=1}^{\infty} \frac{(2x)^\nu}{\nu!} \right] \left[ I + \sum_{\nu=1}^{\infty} \frac{x^\nu B^\nu}{\nu!} \right]
\]
\[
= \left[ 1 + \sum_{\nu=1}^{\infty} \frac{(2x)^\nu}{\nu!} \right] (I+xB+\frac{x^2}{2} B^2)
\]
\[
= e^{2x} \begin{pmatrix} 1 & 0 & 0 \\ 0 & 1 & 0 \\ 0 & 0 & 1 \end{pmatrix} + x \begin{pmatrix} 0 & 1 & 0 \\ 0 & 0 & 1 \\ 0 & 0 & 0 \end{pmatrix} + \frac{x^2}{2} \begin{pmatrix} 0 & 0 & 1 \\ 0 & 0 & 0 \\ 0 & 0 & 0 \end{pmatrix}
\]
\[
= e^{2x} \begin{pmatrix} 1 & x & \frac{x^2}{2} \\ 0 & 1 & x \\ 0 & 0 & 1 \end{pmatrix}.
\]

\begin{Paradeigma}{}
Να επιλυθεί το ομογενές γραμμικό διαφορικό σύστημα
\[
y' = Ay \quad \text{με} \quad A = \begin{pmatrix} 2 & 2 \\ 1 & 3 \end{pmatrix}.
\]
\end{Paradeigma}
Οι ιδιοτιμές του Α είναι $\lambda_1 = 1$ και $\lambda_2 = 4$. Το πρόβλημα αρχικών τιμών
\[
r_1' = r_1, \quad r_2' = r_1 + 4r_2; \quad r_1(0)=1, \ r_2(0) = 0
\]
έχει τη λύση
\[
r_1(x) = e^x, \ x \in \mathbb{R}; \quad r_2(x) = \frac{1}{3} (e^{4x} - e^x), \ x \in \mathbb{R}.
\]
Έτσι (θεώρημα 17), για κάθε $x \in \mathbb{R}$ έχουμε
\[
e^{xA} = r_1(x)I + r_2(x)(A-I) = e^x \begin{pmatrix} 1 & 0 \\ 0 & 1 \end{pmatrix} + \frac{1}{3} (e^{4x}-e^x) \begin{pmatrix} 1 & 2 \\ 1 & 2 \end{pmatrix}
\]
\[
= \begin{pmatrix} e^x + \frac{1}{3} (e^{4x}-e^x) & \frac{2}{3} (e^{4x}-e^x) \\ \frac{1}{3} (e^{4x}-e^x) & \frac{2}{3} (2e^{4x}+e^x) \end{pmatrix}.
\]
Επομένως (θεώρημα 16) οι λύσεις y δίνονται απ' τον τύπο
\[
y(x) = \begin{pmatrix} \frac{1}{3} (e^{4x}+2e^x) & \frac{2}{3} (e^{4x}-e^x) \\ \frac{1}{3} (e^{4x}-e^x) & \frac{1}{3} (2e^{4x}+e^x) \end{pmatrix} \begin{pmatrix} c_1 \\ c_2 \end{pmatrix}
\]
\[
= \begin{pmatrix} \frac{1}{3} [(c_1+2c_2)e^{4x} + 2(c_1-c_2)e^x] \\ \frac{1}{3} [(c_1+2c_2)e^{4x} - (c_1-c_2)e^x] \end{pmatrix} = \begin{pmatrix} C_1 e^{4x} + 2 C_2 e^x \\ C_1 e^{4x} - C_2 e^x \end{pmatrix}, \quad x \in \mathbb{R},
\]
όπου $C_1 = \frac{1}{3} (c_1+2c_2)$, $C_2 = \frac{1}{3} (c_1-c_2)$ είναι αυθαίρετες σταθερές.

\begin{Paradeigma}{Να επιλυθεί το ομογενές γραμμικό διαφορικό σύστημα}
\[
y' = Ay \quad \text{με} \quad A = \begin{pmatrix} 1 & -1 \\ 1 & 3 \end{pmatrix}.
\]
\end{Paradeigma}
\paragraph*{Λύση.} Ο $Α$ έχει ιδιοτιμές $\lambda_1 = \lambda_2 = 2$. Η λύση του προβλήματος αρχικών τιμών
\[
r_1' = 2r_1, \quad r_2' = r_1+2r_2, \quad r_1(0)=1, \ r_2(0)=0
\]
είναι
\[
r_1(x) = e^{2x}, \ x \in \mathbb{R}; \quad r_2(x) = xe^{2x}, \ x \in \mathbb{R}
\]
και άρα (θεώρημα 17) για όλα τα $x \in \mathbb{R}$ παίρνουμε
\[
e^{xA} = r_1(x)I + r_2(x)(A-2I) = e^{2x} \begin{pmatrix} 1 & 0 \\ 0 & 1 \end{pmatrix} + xe^{2x} \begin{pmatrix} -1 & -1 \\ 1 & 1 \end{pmatrix}.
\]
\[
= \begin{pmatrix} e^{2x}-xe^{2x} & -xe^{2x} \\ xe^{2x} & e^{2x}+xe^{2x} \end{pmatrix}.
\]
Έτσι (θεώρημα 16), οι λύσεις y δίνονται απ' τον τύπο
\[
y(x) = \begin{pmatrix} e^{2x}-xe^{2x} & -xe^{2x} \\ xe^{2x} & e^{2x}+xe^{2x} \end{pmatrix} \begin{pmatrix} c_1 \\ c_2 \end{pmatrix} = \begin{pmatrix} c_1 e^{2x} - (c_1+c_2)xe^{2x} \\ c_2 e^{2x} + (c_1+c_2)xe^{2x} \end{pmatrix}, \quad x \in \mathbb{R},
\]
όπου $c_1, c_2$ είναι αυθαίρετες σταθερές.

\begin{Paradeigma}{}
Να επιλυθεί το ομογενές γραμμικό διαφορικό σύστημα
\[
y' = Ay \quad \text{με} \quad A = \begin{pmatrix} 1 & 1 & 0 \\ 0 & 1 & 0 \\ 0 & 0 & 1 \end{pmatrix}.
\]
\end{Paradeigma}
Οι ιδιοτιμές του Α είναι $\lambda_1=1, \lambda_2=2, \lambda_3=-1$ και το πρόβλημα αρχικών τιμών
\[
r_1' = r_1, \ r_2' = r_1+2r_2, \ r_3' = r_2-r_3; \ r_1(0)=1, r_2(0)=0, r_3(0)=0
\]
έχει τη λύση
\[
r_1(x)=e^x, \ x \in \mathbb{R}; \ r_2(x) = -e^x+e^{2x}, \ x \in \mathbb{R}; \ r_3(x) = -\frac{1}{2}e^x + \frac{1}{3}e^{2x} + \frac{1}{6}e^{-x}, \ x \in \mathbb{R}.
\]
Επομένως (θεώρημα 17), για κάθε $x \in \mathbb{R}$ έχουμε
\[
e^{xA} = r_1(x)I + r_2(x)(A-I) + r_3(x)(A-I)(A-2I) = e^x \begin{pmatrix} 1 & 0 & 0 \\ 0 & 1 & 0 \\ 0 & 0 & 1 \end{pmatrix} +
\]
\[
+ (-e^x+e^{2x}) \begin{pmatrix} 0 & 1 & 0 \\ 0 & 0 & 0 \\ 0 & 0 & 0 \end{pmatrix} + \left( -\frac{1}{2}e^x + \frac{1}{3}e^{2x} + \frac{1}{6}e^{-x} \right) \begin{pmatrix} 0 & -1 & 0 \\ 0 & 0 & 0 \\ 0 & 0 & 0 \end{pmatrix} \begin{pmatrix} -1 & 1 & 0 \\ 0 & 0 & 0 \\ 0 & 0 & -1 \end{pmatrix}
\]
\[
= \begin{pmatrix} \frac{1}{2}e^x + \frac{1}{3}e^{2x} - \frac{1}{6}e^{-x} & -\frac{1}{2}e^x - \frac{1}{3}e^{2x} + \frac{5}{6}e^{-x} & -\frac{1}{2}e^x + \frac{1}{3}e^{2x} + \frac{1}{6}e^{-x} \\ \frac{1}{3}e^{2x} - \frac{1}{3}e^{-x} & \frac{1}{3}e^{2x} + \frac{2}{3}e^{-x} & \frac{1}{3}e^{2x} - \frac{1}{3}e^{-x} \\ -\frac{1}{2}e^x + \frac{1}{6}e^{-x} & -\frac{1}{2}e^x - \frac{1}{6}e^{-x} & \frac{1}{2}e^x + \frac{1}{6}e^{-x} + e^{-x} \end{pmatrix}.
\]
και άρα (θεώρημα 16) οι λύσεις $y$ δίνονται απ' τον τύπο















\setchapterimage{./images/10.png}
\chapter{Δυναμοσειρές λύσεις γραμμικών διαφορικών εξισώσεων 2ης τάξης}
\setchapterimage{./images/3.png}
\chapter{Μετασχηματισμοί \eng{Laplace} - Επίλυση γραμμικών διαφορικών εξισώσεων κι συστημάτων με μετασχηματισμούς \eng{Laplace}}
\setchapterimage{./images/6.jpeg}
\chapter{Ευστάθεια μη γραμμικών διαφορικών εξισώσεων}
\setchapterimage{./images/12.jpeg}
\chapter{Μερικές διαφορικές εξισώσεις 1ης τάξης - Γραμμικές μερικές διαφορικές εξισώσεις 1ης τάξης}
\setchapterimage{./images/18.jpeg}
\chapter{Γραμμικές μερικές διαφορικές εξισώσεις 2ης τάξης}
\setchapterimage{./images/10.jpeg}
\chapter{Μετασχηματισμοί \eng{Fourier}. Εφαρμογή των μετασχηματισμών \eng{Fourier} στην επίλυση προβλημάτων αρχικών τιμών, αρχικών-συνοριακών τιμών και συνοριακών τιμών για γραμμικές μερικές διαφορικές εξισώσεις δεύτερης τάξης}
\end{document}