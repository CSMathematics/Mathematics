\documentclass[twoside,nofonts,internet,shmeiwseis]{thewria}
\usepackage[amsbb,subscriptcorrection,zswash,mtpcal,mtphrb,mtpfrak]{mtpro2}
\usepackage[no-math,cm-default]{fontspec}
\usepackage{amsmath}
\usepackage{xgreek}
\usepackage{fontawesome}
\newfontfamily{\FA}{fontawesome.otf}
\let\hbar\relax
\defaultfontfeatures{Mapping=tex-text,Scale=MatchLowercase}
\setmainfont[Mapping=tex-text,Numbers=Lining,Scale=1.0,BoldFont={Minion Pro Bold}]{Minion Pro}
\newfontfamily\scfont{GFS Artemisia}
\font\icon = "Webdings"
\usepackage{tikz,pgfplots}
\tkzSetUpPoint[size=7,fill=white]
\xroma{red!70!black}
%------TIKZ - ΣΧΗΜΑΤΑ - ΓΡΑΦΙΚΕΣ ΠΑΡΑΣΤΑΣΕΙΣ ----
\usepackage{tikz}
\usepackage{tkz-euclide}
\usetkzobj{all}
\usepackage[framemethod=TikZ]{mdframed}
\usetikzlibrary{decorations.pathreplacing}
\usepackage{pgfplots}
\usetkzobj{all}
%-----------------------
\usepackage{calc}
\usepackage{hhline}
\usepackage[explicit]{titlesec}
\usepackage{graphicx}
\usepackage{multicol}
\usepackage{multirow}
\usepackage{enumitem}
\usepackage{tabularx}
\usetikzlibrary{backgrounds}
\usepackage{sectsty}
\sectionfont{\centering}
\usepackage{adjustbox}
\usepackage{mathimatika,gensymb,eurosym,wrap-rl}
\usepackage{systeme,regexpatch}
%-------- ΜΑΘΗΜΑΤΙΚΑ ΕΡΓΑΛΕΙΑ ---------
\usepackage{mathtools}
%----------------------
%-------- ΠΙΝΑΚΕΣ ---------
\usepackage{booktabs}
%----------------------
%----- ΥΠΟΛΟΓΙΣΤΗΣ ----------
\usepackage{calculator}
%----------------------------
%------ ΔΙΑΓΩΝΙΟ ΣΕ ΠΙΝΑΚΑ -------
\usepackage{array}
\newcommand\diag[5]{%
\multicolumn{1}{|m{#2}|}{\hskip-\tabcolsep
$\vcenter{\begin{tikzpicture}[baseline=0,anchor=south west,outer sep=0]
\path[use as bounding box] (0,0) rectangle (#2+2\tabcolsep,\baselineskip);
\node[minimum width={#2+2\tabcolsep-\pgflinewidth},
minimum  height=\baselineskip+#3-\pgflinewidth] (box) {};
\draw[line cap=round] (box.north west) -- (box.south east);
\node[anchor=south west,align=left,inner sep=#1] at (box.south west) {#4};
\node[anchor=north east,align=right,inner sep=#1] at (box.north east) {#5};
\end{tikzpicture}}\rule{0pt}{.71\baselineskip+#3-\pgflinewidth}$\hskip-\tabcolsep}}
%---------------------------------
%---- ΟΡΙΖΟΝΤΙΟ - ΚΑΤΑΚΟΡΥΦΟ - ΠΛΑΓΙΟ ΑΓΚΙΣΤΡΟ ------
\newcommand{\orag}[3]{\node at (#1)
{$ \overcbrace{\rule{#2mm}{0mm}}^{{\scriptsize #3}} $};}
\newcommand{\kag}[3]{\node at (#1)
{$ \undercbrace{\rule{#2mm}{0mm}}_{{\scriptsize #3}} $};}
\newcommand{\Pag}[4]{\node[rotate=#1] at (#2)
{$ \overcbrace{\rule{#3mm}{0mm}}^{{\rotatebox{-#1}{\scriptsize$#4$}}}$};}
%-----------------------------------------
%------------------------------------------
\newcommand{\tss}[1]{\textsuperscript{#1}}
\newcommand{\tssL}[1]{\MakeLowercase{\textsuperscript{#1}}}
%---------- ΛΙΣΤΕΣ ----------------------
\newlist{bhma}{enumerate}{3}
\setlist[bhma]{label=\bf\textit{\arabic*\textsuperscript{o}\;Βήμα :},leftmargin=0cm,itemindent=1.8cm,ref=\bf{\arabic*\textsuperscript{o}\;Βήμα}}
\newlist{rlist}{enumerate}{3}
\setlist[rlist]{itemsep=0mm,label=\roman*.}
\newlist{brlist}{enumerate}{3}
\setlist[brlist]{itemsep=0mm,label=\bf\roman*.}
\newlist{tropos}{enumerate}{3}
\setlist[tropos]{label=\bf\textit{\arabic*\textsuperscript{oς}\;Τρόπος :},leftmargin=0cm,itemindent=2.3cm,ref=\bf{\arabic*\textsuperscript{oς}\;Τρόπος}}
% Αν μπει το bhma μεσα σε tropo τότε
%\begin{bhma}[leftmargin=.7cm]
\tkzSetUpPoint[size=7,fill=white]
\tikzstyle{pl}=[line width=0.3mm]
\tikzstyle{plm}=[line width=0.4mm]
\usepackage{etoolbox}
\makeatletter
\renewrobustcmd{\anw@true}{\let\ifanw@\iffalse}
\renewrobustcmd{\anw@false}{\let\ifanw@\iffalse}\anw@false
\newrobustcmd{\noanw@true}{\let\ifnoanw@\iffalse}
\newrobustcmd{\noanw@false}{\let\ifnoanw@\iffalse}\noanw@false
\renewrobustcmd{\anw@print}{\ifanw@\ifnoanw@\else\numer@lsign\fi\fi}
\makeatother
\ekthetesdeiktes
\setlist[enumerate]{itemsep=0mm}


\begin{document}
\twocolumn
\paragraph{Θεμέλια των μαθηματικών}
\begin{enumerate}
\item Αριθμοί
\begin{enumerate}
\item Ακέραιοι - Ρητοί - Άρρητοι - Πραγματικοί
\item Πράξεις - Ιδιότητες
\item Ισότητες
\item Διάταξη
\item Συστήματα αρίθμησης
\item Ρίζες
\item Απόλυτες τιμές
\end{enumerate}
\item Λογική – Προτασιακός λογισμός
\item Σύνολα
\item Συστήματα συντεταγμένων
\begin{enumerate}
\item Καρτεσιανό 2-δ
\item Πολικές συντετ.
\item Καρτεσιανό 3-δ
\item Κυλινδρικές συντ.
\item Σφαιρικές συντεταγμένες
\item Άλλες συντ.
\end{enumerate}
\end{enumerate}
\paragraph{Βασική Άλγεβρα}
\begin{enumerate}[resume]
\item Αλγεβρικές παραστάσεις
\begin{enumerate}
\item Γενικές
\item Πολυώνυμα
\item Ταυτότητες - Παραγοντοποίηση
\item Ρητές
\item Άρρητες
\item Εκθετικές
\item Λογάριθμοι
\end{enumerate}
\item Τριγωνομετρία
\begin{enumerate}
\item Τριγωνομετρικοί αριθμοί
\item Ταυτότητες
\item Αναγωγή στο 1ο τετ.
\item Νόμος ημιτόνων, συνημιτόνων, εφαπτομένων - Επίλυση τριγώνου
\end{enumerate}
\item Εξισώσεις – Ανισώσεις
\begin{enumerate}
\item Πρώτου βαθμού
\item Δευτέρου βαθμού
\item Τρίτου βαθμού
\item Τετάρτου βαθμού
\item Πολυωνυμικές
\item Ρητές
\item Άρρητες
\item Τριγωνομετρικές
\item Εκθετικές
\item Λογαριθμικές
\end{enumerate}
\end{enumerate}
\paragraph{Ανάλυση}
\begin{enumerate}[resume]
\item Συναρτήσεις
\begin{enumerate}
\item Βασικές έννοιες
\item Βασικά είδη συναρτήσεων
\item Ιδιότητες συναρτήσεων
\item Όρια
\item Συνέχεια
\end{enumerate}
\end{enumerate}
\paragraph{Διαφορικός - Ολοκληρωτικός Λογισμός}
\begin{enumerate}[resume]
\item Παράγωγος – Μεικτές παράγωγοι
\item Ολοκλήρωμα
\item Απειροστικός λογισμός πολλών μεταβλητών
\item Διανυσματικός λογισμός
\item Πολλαπλά Ολοκληρώματα
\item Επικαμπύλια – Επιφανειακά ολοκληρώματα
\item Ολοκληρωτικά θεωρήματα
\item Ακολουθίες
\item Σειρές
\item Γινόμενα
\end{enumerate}
\paragraph{Μιγαδική Ανάλυση}
\begin{enumerate}[resume]
\item Μιγαδικοί αριθμοί
\item Κουατέρνια
\end{enumerate}
\paragraph{Γεωμετρία}
\begin{enumerate}[resume]
\item Βασικά σχήματα
\item Ευθείες
\item Τρίγωνα
\item Βασικά τετράπλευρα
\item Κύκλος
\item Κανονικά πολύγωνα
\item Στερεομετρία
\item Κωνικές τομές
\item Διανύσματα
\item Αναλυτική Γεωμετρία
\end{enumerate}
\paragraph{Γραμμική Άλγεβρα}
\begin{enumerate}[resume]
\item Πίνακες – Συστήματα
\item Διανυσματικοί χώροι
\item Ορίζουσες
\end{enumerate}
\paragraph{Διαφορικές Εξισώσεις}
\begin{enumerate}[resume]
\item Συνήθεις Διαφορικές εξισώσεις
\item Μερικές Διαφορικές Εξισώσεις
\item Ολοκληρωτικές εξισώσεις
\item Στοχαστικές Διαφορικές Εξισώσεις
\item Συναρτησιακές Διαφορικές Εξισώσεις
\item Αλγεβρικές διαφορικές εξισώσεις
\item Υστερημένες Διαφορικές Εξισώσεις
\end{enumerate}
\paragraph{Πιθανότητες - Στατιστική}
\begin{enumerate}[resume]
\item Πιθανότητες – Στατιστική
\item Τοπολογία
\item Συναρτησιακή ανάλυση
\item Ειδικές συναρτήσεις
\item Θεωρία μέτρου
\item Θεωρία ομάδων
\item Θεωρία δακτυλίων – σωμάτων
\item Θεωρία αριθμών
\item Διαφορική Γεωμετρία
\item Θεωρία συνόλων
\item Ιστορία μαθηματικών
\item Ανάλυση Fourier
\item Αριθμητική ανάλυση
\item Μοντέρνα άλγεβρα
\item Μοντέρνα Γεωμετρία - μη ευκλείδειες
\item Συνδυαστική
\end{enumerate}
\end{document}
