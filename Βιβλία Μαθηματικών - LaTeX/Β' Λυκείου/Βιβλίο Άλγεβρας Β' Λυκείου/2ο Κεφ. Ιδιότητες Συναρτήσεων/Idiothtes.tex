\documentclass[twoside,10pt]{book}
\usepackage[amsbb,mtpfrak,zswash,mtpcal]{mtpro2}
\usepackage[no-math,cm-default]{fontspec}
\usepackage{xunicode}
\usepackage{xltxtra}
\usepackage{xgreek}
\defaultfontfeatures{Mapping=tex-text,Scale=MatchLowercase}
\setmainfont[Mapping=tex-text,Numbers=Lining,Scale=1.0,BoldFont={Minion Pro Bold}]{Minion Pro}
\defaultfontfeatures{Ligatures=TeX}
\font\kefalaio="Minion Pro Bold" at 36pt
\font\ArKef="Minion Pro Bold Italic" at 72pt
\font\OnKef="Times New Roman" at 20pt
\font\OnPar="Minion Pro Bold" at 18pt
\newfontfamily\scfont{GFS Artemisia}
\usepackage[inner=2.00cm, outer=1.50cm, top=3.00cm, bottom=2.00cm,paperwidth=17cm,paperheight=24cm]{geometry}
\usepackage{amsmath}
\usepackage[amsbb,mtpfrak,zswash,mtpcal]{mtpro2}
\usepackage{makeidx}
\usepackage{longtable,xcolor,varwidth}
\usepackage{float}
\usepackage{subfig}
\def\xrwma{cyan!70!black}
\def\xrwmath{cyan}
\usepackage{etoolbox}
\makeatletter
\newif\ifLT@nocaption
\preto\longtable{\LT@nocaptiontrue}
\appto\endlongtable{%
\ifLT@nocaption
\addtocounter{table}{\m@ne}%
\fi}
\preto\LT@caption{%
\noalign{\global\LT@nocaptionfalse}}
\makeatother
\makeindex
\usepackage{tikz,pgfplots}
\usepackage{tkz-euclide,tkz-fct}
\usetikzlibrary{fadings}
\usepackage{wrap-rl}
\usetkzobj{all}
\usepackage{calc}
\usepackage{cleveref}
\usepackage[colorlinks=false, pdfborder={0 0 0}]{hyperref}
\usepackage[framemethod=TikZ]{mdframed}
\definecolor{steelblue}{cmyk}{.7,.278,0,.294}
\definecolor{doc}{cmyk}{1,0.455,0,0.569}
\definecolor{olivedrab}{cmyk}{0.25,0,0.75,0.44}
\usepackage{capt-of}
\usepackage{titletoc}
\usepackage[explicit]{titlesec}
\usepackage{graphicx}
\usepackage{multicol}
\usepackage{multirow}
\usepackage{enumitem}
\usepackage{tabularx}
\usepackage[decimalsymbol=comma]{siunitx}
\tikzset{>=latex}
\makeatletter
\pretocmd{\@part}{\gdef\parttitle{#1}}{}{}
\pretocmd{\@spart}{\gdef\parttitle{#1}}{}{}
\makeatother
\usepackage[titletoc]{appendix}
\usepackage{fancyhdr}
\pagestyle{fancy}
\fancyheadoffset{0cm}
\renewcommand{\headrulewidth}{\iftopfloat{0pt}{.5pt}}
\renewcommand{\chaptermark}[1]{\markboth{#1}{}}
\renewcommand{\sectionmark}[1]{\markright{\it\thesection\ #1}}
\fancyhf{}
\fancyhead[LE]{\thepage\ $\cdot$\ \scfont\scshape\nouppercase{\leftmark}}
\fancyhead[RO]{\nouppercase{\rightmark} $\cdot$\ \thepage}
\fancypagestyle{plain}{%
\fancyhead{} %
\renewcommand{\headrulewidth}{0pt}}

\newcounter{thewrhma}[chapter]
\renewcommand{\thethewrhma}{\thechapter.\arabic{thewrhma}} 
\newcommand{\Thewrhma}[1]{\refstepcounter{thewrhma}{\textbf{\textcolor{\xrwmath}{{\large Θεώρημα\hspace{2mm}\thethewrhma\;}:\;}\hspace{1mm}}} \MakeUppercase{\textbf{#1}}\\}{}

\newcounter{porisma}[chapter]
\renewcommand{\theporisma}{\thechapter.\arabic{porisma}}\newcommand{\Porisma}[1]{\refstepcounter{porisma}\textcolor{black}{\textbf{ΠΟΡΙΣΜΑ\hspace{2mm}\theporisma\hspace{1mm} \MakeUppercase{#1}}}\\}{}

\newcounter{protasi}[chapter]
\renewcommand{\theprotasi}{\thechapter.\arabic{protasi}}\newcommand{\Protasi}[1]{\refstepcounter{protasi}\textcolor{black}{\textbf{ΠΡΟΤΑΣΗ\hspace{2mm}\theprotasi\hspace{1mm} \MakeUppercase{#1}}}\\}{}


\newcounter{orismos}[chapter]
\renewcommand{\theorismos}{\arabic{orismos}}   
\newcommand{\Orismos}[1]{\refstepcounter{orismos}{\textbf{\textbf{\textcolor{\xrwma}{{\large Ορισμός\hspace{2mm}\theorismos\;}:\;}}}}\hspace{1mm} \MakeUppercase{\textbf{#1}\\}}{}
\usepackage{venndiagram,mathimatika}
%-------- ΣΤΥΛ ΚΕΦΑΛΑΙΟΥ ---------
\newcommand*\chapterlabel{}
\newcommand{\fancychapter}{%
\titleformat{\chapter}
{
\normalfont\Huge}
{\gdef\chapterlabel{\thechapter\ }}{0pt}
{\begin{tikzpicture}[remember picture,overlay]
\node[yshift=-7cm] at (current page.north west)
{\begin{tikzpicture}[remember picture, overlay]
%\node[inner sep=0pt] at ($(current page.north) +			(0cm,-1.38in)$) {\includegraphics[width=17cm]{Kefalaio}};
\node[anchor=west,xshift=.1\paperwidth,yshift=.14\paperheight,rectangle]
{{\color{white}\fontsize{30}{20}\textbf{\textcolor{black}{\contour{white}{ΚΕΦΑΛΑΙΟ}}}}};
\node[anchor=west,xshift=.09\paperwidth,yshift=.08\paperheight,rectangle] {\fontsize{24}{20} {\color{black}{{\textcolor{black}{\contour{white}{\sc##1}}}}}};
%\fill[fill=black] (12.2,2) rectangle (14.8,4.7);
\node[anchor=west,xshift=.74\paperwidth,yshift=.11\paperheight,rectangle]
{{\color{white}\fontsize{80}{20}\textbf{\textit{\textcolor{white}{\contour{black}{\thechapter}}}}}};
\end{tikzpicture}
};
\end{tikzpicture}
}
\titlespacing*{\chapter}{0pt}{20pt}{30pt}
}
%------------------------------------------------


\usepackage[outline]{contour}
\newcommand{\regularchapter}{%
\titleformat{\chapter}[display]
{\normalfont\huge\bfseries}{\chaptertitlename\ \thechapter}{20pt}{\Huge##1}
\titlespacing*{\chapter}
{0pt}{-20pt}{40pt}
}

\apptocmd{\mainmatter}{\fancychapter}{}{}
\apptocmd{\backmatter}{\regularchapter}{}{}
\apptocmd{\frontmatter}{\regularchapter}{}{}

\titlespacing*{\section}
{0pt}{30pt}{0pt}
\usepackage{booktabs}
\usepackage{hhline}
\DeclareRobustCommand{\perthousand}{%
\ifmmode
\text{\textperthousand}%
\else
\textperthousand
\fi}


\contentsmargin{0cm}
\titlecontents{part}[-1pc]
{\addvspace{10pt}%
\bf\Large ΜΕΡΟΣ\quad }%
{}
{}
{\;\dotfill}%
%------------------------------------------
\titlecontents{chapter}[0pc]
{\addvspace{30pt}%
\begin{tikzpicture}[remember picture, overlay]%
\draw[fill=black,draw=black] (-.3,.5) rectangle (3.7,1.1); %
\pgftext[left,x=0cm,y=0.75cm]{\color{white}\sc\Large\bfseries Κεφάλαιο\ \thecontentslabel};%
\end{tikzpicture}\large\sc}%
{}
{}
{\hspace*{-2.3em}\hfill\normalsize Σελίδα \thecontentspage}%
\titlecontents{section}[2.4pc]
{\addvspace{1pt}}
{\contentslabel[\thecontentslabel]{2pc}}
{}
{\;\dotfill\;\small \thecontentspage}
[]
\titlecontents*{subsection}[4pc]
{\addvspace{-1pt}\small}
{}
{}
{\ --- \small\thecontentspage}
[ \textbullet\ ][]

\makeatletter
\renewcommand{\tableofcontents}{%
\chapter*{%
\vspace*{-20\p@}%
\begin{tikzpicture}[remember picture, overlay]%
\pgftext[right,x=12cm,y=0.2cm]{\Huge\sc\bfseries \contentsname};%
\draw[fill=black,draw=black] (9.5,-.75) rectangle (12.5,1);%
\clip (9.5,-.75) rectangle (15,1);
\pgftext[right,x=12cm,y=0.2cm]{\color{white}\Huge\bfseries \contentsname};%
\end{tikzpicture}}%
\@starttoc{toc}}
\makeatother

\usepackage[contents={},scale=1,opacity=1,color=black,angle=0]{background}

\newcommand\blfootnote[1]{%
\begingroup
\renewcommand\thefootnote{}\footnote{#1}%
\addtocounter{footnote}{-1}%
\endgroup
}
\usepackage{epstopdf}
\epstopdfsetup{update}
\usepackage{textcomp}

\titleformat{\section}
{\normalfont\Large\bf}%
{}{0em}%
{{\color{black}\titlerule[0pt]}\vskip-.2\baselineskip{\parbox[t]{\dimexpr\textwidth-2\fboxsep\relax}{\raggedright\strut\itshape{\LARGE{\thesection~#1}}\strut}}}[\vskip 0\baselineskip{\color{black}\titlerule[1pt]}]
\titlespacing*{\section}{0pt}{0pt}{30pt}

\newcommand{\methodologia}{\begin{center}
{\large \textbf{ΜΕΘΟΔΟΛΟΓΙΑ}}\\\vspace{-2mm}
\begin{tikzpicture}
\shade[left color=white, right color=black,] (-3cm,0) rectangle (0,.2mm);
\shade[left color=black, right color=white,] (0,0) rectangle (3cm,.2mm);   
\end{tikzpicture}
\end{center}}

\newcommand{\orismoi}{\begin{center}
\vspace{-3mm}{\large \textbf{\textcolor{\xrwma}{ΟΡΙΣΜΟΙ}}}\\\vspace{-2mm}
\begin{tikzpicture}
\shade[left color=white, right color=cyan!80!black,] (-3cm,0) rectangle (0,.2mm);
\shade[left color=cyan!80!black, right color=white,] (0,0) rectangle (3cm,.2mm);   
\end{tikzpicture}
\end{center}}
\newcommand{\thewrhmata}{\begin{center}
{\large \textbf{\textcolor{\xrwmath}{ΘΕΩΡΗΜΑΤΑ - ΠΟΡΙΣΜΑΤΑ - ΠΡΟΤΑΣΕΙΣ\\ΚΡΙΤΗΡΙΑ - ΙΔΙΟΤΗΤΕΣ}}}\\\vspace{-2mm}
\begin{tikzpicture}
\shade[left color=white, right color=\xrwmath,] (-5cm,0) rectangle (0,.2mm);
\shade[left color=\xrwmath, right color=white,] (0,0) rectangle (5cm,.2mm);   
\end{tikzpicture}
\end{center}}
\usepackage[labelfont={footnotesize,it,bf},font={footnotesize}]{caption}

%-------- ΠΙΝΑΚΕΣ ---------
\usepackage{booktabs}
%----------------------
%----- ΥΠΟΛΟΓΙΣΤΗΣ ----------
%\usepackage{calculator}
%----------------------------

%----- ΟΡΙΖΟΝΤΙΑ ΛΙΣΤΑ ------
\usepackage{xparse}
\newcounter{answers}
\renewcommand\theanswers{\arabic{answers}}
\ExplSyntaxOn
\NewDocumentCommand{\results}{m}
{
\seq_set_split:Nnn \l_results_a_seq {,}{#1}
\par\nobreak\noindent\setcounter{answers}{0}
\seq_map_inline:Nn \l_results_a_seq
{
\makebox[.18\linewidth][l]{\stepcounter{answers}\theanswers.~##1}\hfill
}
\par
}
\seq_new:N \l_results_a_seq
\ExplSyntaxOff
%----------------------------
%------ ΜΗΚΟΣ ΓΡΑΜΜΗΣ ΚΛΑΣΜΑΤΟΣ ---------
\DeclareRobustCommand{\frac}[3][0pt]{%
{\begingroup\hspace{#1}#2\hspace{#1}\endgroup\over\hspace{#1}#3\hspace{#1}}}
%----------------------------------------
\usepackage{microtype}
\usepackage{float}

\usepackage{caption}

%---- ΟΡΙΖΟΝΤΙΟ - ΚΑΤΑΚΟΡΥΦΟ - ΠΛΑΓΙΟ ΑΓΚΙΣΤΡΟ ------
\newcommand{\orag}[3]{\node at (#1)
{$ \overcbrace{\rule{#2mm}{0mm}}^{{\scriptsize #3}} $};}

\newcommand{\kag}[3]{\node at (#1)
{$ \undercbrace{\rule{#2mm}{0mm}}_{{\scriptsize #3}} $};}

\newcommand{\Pag}[4]{\node[rotate=#1] at (#2)
{$ \overcbrace{\rule{#3mm}{0mm}}^{{\rotatebox{-#1}{\scriptsize$#4$}}}$};}
%-----------------------------------------
\tikzstyle{pl}=[line width=0.3mm]
\tikzstyle{plm}=[line width=0.4mm]
%------- ΣΤΥΛ ΠΑΡΑΔΕΙΓΜΑΤΟΣ -------
\newcounter{paradeigma}[section]
\renewcommand{\theparadeigma}{\bf\thechapter.\arabic{paradeigma}}   
\newcommand{\Paradeigma}[1]{\refstepcounter{paradeigma}\textcolor{cyan}{\textbf{{\large Παράδειγμα\hspace{2mm}\theparadeigma\;:\;}\hspace{1mm}}} \MakeUppercase{\textbf{#1}}\\}{}
%-----------------------------------

%------- ΣΤΥΛ ΛΥΣΗΣ ------------------
\newcommand{\lysh}{{\textbf{ΛΥΣΗ}}}
%------------------------------------

%------ ΛΥΜΕΝΑ ΠΑΡΑΔΕΙΓΜΑΤΑ ΤΙΤΛΟΣ ---------
\newcommand{\Lymena}{\begin{center}
\begin{tikzpicture}
\path[left color=cyan!70!black,right color=cyan!80!black,middle color=cyan!80!white] (-7cm,-.6cm) rectangle (6.5cm,.6cm);
\node at (-.25cm,0) {\Large \textcolor{white}{\textbf{ΛΥΜΕΝΑ ΠΑΡΑΔΕΙΓΜΑΤΑ}}};  
\end{tikzpicture}
\end{center}}
%--------------------------------------

%--------- ΑΛΥΤΕΣ ΑΣΚΗΣΕΙΣ ΤΙΤΛΟΣ ----------
\newcommand{\Alyta}{\begin{center}
\begin{tikzpicture}
\path[left color=cyan!70!black,right color=cyan!80!black,middle color=cyan!80!white] (-7cm,-.6cm) rectangle (6.5cm,.6cm);
\node at (-.25cm,0) {\Large \textcolor{white}{\textbf{ΑΣΚΗΣΕΙΣ - ΠΡΟΒΛΗΜΑΤΑ}}};  
\end{tikzpicture}
\end{center}}
%--------------------------------------------
\usetikzlibrary{shadows,calc}
\usepackage{tcolorbox}
\tcbuselibrary{skins,theorems,breakable}
%---------- ΜΕΘΟΔΟΣ --------------
\newcounter{Methodos}[chapter]
\renewcommand{\theMethodos}{\thechapter.\arabic{Methodos}}
\newenvironment{Methodos}[2][\linewidth]
{\refstepcounter{Methodos}
\begin{tcolorbox}[breakable,
enhanced standard,
boxrule=0.7pt,titlerule=-.2pt,drop fuzzy shadow southeast=black!50,
width=\linewidth,
title style={color=white},
overlay unbroken and first={
\path[left color=cyan!70!black,right color=cyan,draw=black]
([yshift=-\pgflinewidth]frame.north west) to ([yshift=-5pt]title.south west)[rounded corners=2pt] -- ([xshift=-#2-15pt,yshift=-5pt]title.south east) to[rounded corners=2pt] ([xshift=-#2,yshift=-\pgflinewidth]frame.north east) -- cycle;
},
fonttitle=\bfseries,
before=\par\medskip\noindent,
after=\par\medskip,
toptitle=3pt,
top=11pt,topsep at break=-5pt,
colback=white,title={\large Μέθοδος \theMethodos} : {\textcolor{black}{\MakeUppercase{#1}}}]}
{\end{tcolorbox}}
%------------------------------------------
%---------- ΛΙΣΤΕΣ ----------------------
\newlist{bhma}{enumerate}{3}
\setlist[bhma]{label=\bf\textit{\arabic*\textsuperscript{o}\;Βήμα :},leftmargin=0cm,itemindent=1.5cm,ref=\bf{\arabic*\textsuperscript{o}\;Βήμα}}
\newlist{rlist}{enumerate}{3}
\setlist[rlist]{itemsep=0mm,label=\roman*.}


%----ΣΤΥΛ ΑΣΚΗΣΗΣ ----------
\newcounter{askhsh}[chapter]
\renewcommand{\theaskhsh}{\bf{{\large{\thechapter}}.\arabic{askhsh}}}   
\newcommand{\Askhsh}{\refstepcounter{askhsh}\textcolor{\xrwma}{{\theaskhsh}\hspace{1mm}}}{}
%---------------------------

\newlist{brlist}{enumerate}{3}
\setlist[brlist]{itemsep=0mm,label=\bf\roman*.}
\newlist{tropos}{enumerate}{3}
\setlist[tropos]{label=\bf\textit{\arabic*\textsuperscript{oς}\;Τρόπος :},leftmargin=0cm,itemindent=2.3cm,ref=\bf{\arabic*\textsuperscript{oς}\;Τρόπος}}
% Αν μπει το bhma μεσα σε tropo τότε
%\begin{bhma}[leftmargin=.7cm]
\newcommand{\tss}[1]{\textsuperscript{#1}}
\newcommand{\tssL}[1]{\MakeLowercase{\textsuperscript{#1}}}
%------------------------------------------
\setlength{\parindent}{0pt}
\setlist[itemize]{itemsep=0mm}
\tkzSetUpPoint[size=7,fill=white]
\newcommand{\twocolkentro}[1]{
\twocolumn[
\begin{@twocolumnfalse}
#1
\end{@twocolumnfalse}]}
\newcommand{\bcc}[1]{
\begin{center}
{\color{\xrwma}{\hrulefill}\raisebox{-2.5mm}{\rule{.4pt}{5mm}}}\hspace{1em}\raisebox{-.65ex}{\begin{varwidth}{\dimexpr0.7\textwidth-2em\relax}\centering{\textbf{\textcolor{\xrwma}{#1}}}\end{varwidth}}\hspace*{1em}{\color{\xrwma}{\raisebox{-2.5mm}{\rule{.4pt}{5mm}}\hrulefill}}
\end{center}}



\begin{document}
\mainmatter
\pagestyle{fancy}
\chapter{Ιδιότητες Συναρτήσεων}
\section{Μονοτονία - Ακρότατα}
\orismoi
\Orismos{Μονοτονία}
Μια συνάρτηση αύξουσα ή φθίνουσα, χαρακτηρίζεται ως \textbf{μονότονη}, ενώ μια γνησίως αύξουσα ή γνησίως φθίνουσα συνάρτηση ως \textbf{γνησίως μονότονη}. Οι χαρακτηρισμοί αυτοί αφορούν τη \textbf{μονοτονία} μιας συνάρτησης, μια ιδιότητα των συναρτήσεων η οποία δείχνει την αύξηση ή τη μείωση των τιμών μιας συνάρτησης σε ένα διάστημα του πεδίου ορισμού.
\begin{enumerate}[itemsep=0mm,label=\bf\arabic*.]
\item \textbf{Γνησίως αύξουσα}\\ Μια συνάρτηση $ f $ ορισμένη σε ένα διάστημα $ \varDelta $ ονομάζεται γνησίως αύξουσα στο $ \varDelta $ εαν για κάθε ζεύγος αριθμών $ x_1,x_2\in\varDelta $ με $ x_1<x_2 $ ισχύει \[ f(x_1)<f(x_2) \]
\item \textbf{Γνησίως φθίνουσα}\\ Μια συνάρτηση $ f $ ορισμένη σε ένα διάστημα $ \varDelta $ ονομάζεται γνησίως φθίνουσα στο $ \varDelta $ εαν για κάθε ζεύγος αριθμών $ x_1,x_2\in\varDelta $ με $ x_1<x_2 $ ισχύει \[ f(x_1)>f(x_2) \]
\begin{center}
\begin{tabular}{p{5cm}p{5cm}}
\begin{tikzpicture}
\draw[dashed] (3.3,1.4) node[anchor=north]{\scriptsize $x_2$} -- 
(3.3,2.58)--(1,2.58) node[left]{\scriptsize $f(x_2)$};
\draw[dashed] (2,1.4) node[anchor=north]{\scriptsize $x_1$}-- 
(2,2.08)--(1,2.08)node[left]{\scriptsize $f(x_1)$};
\begin{axis}[x=1cm,y=1cm,aks_on,xmin=-1,xmax=3,
ymin=-1.4,ymax=2,ticks=none,xlabel={\footnotesize $ x $},
ylabel={\footnotesize $ y $},belh ar]
\addplot[grafikh parastash,\xrwma,domain=-.8:3]{ln(x+1)};
\end{axis}
\tkzDrawPoint[size=7,fill=black](2,2.09)
\tkzDrawPoint[size=7,fill=black](3.3,2.59)
\node[fill=white,inner sep=.1mm] at (2.7,0.6) {\scriptsize $ x_1<x_2\Rightarrow f(x_1)<f(x_2)$};
\end{tikzpicture}\captionof{figure}{Γνησίως αύξουσα}	& \begin{tikzpicture}
\draw[dashed] (2.6,1.4) node[anchor=north]{\scriptsize $x_2$} -- 
(2.6,2.02)--(1,2.02) node[left]{\scriptsize $f(x_2)$};
\draw[dashed] (1.5,1.4) node[anchor=north]{\scriptsize $x_1$}-- 
(1.5,2.7)--(1,2.7)node[left]{\scriptsize $f(x_1)$};
\begin{axis}[x=1cm,y=1cm,aks_on,xmin=-1,xmax=3,
ymin=-1.4,ymax=2,ticks=none,xlabel={\footnotesize $ x $},
ylabel={\footnotesize $ y $},belh ar,clip=false]
\addplot[grafikh parastash,\xrwma,domain=-.6:3]{-0.2*(x+.5)^2+1.5};
\end{axis}
\tkzDrawPoint[size=7,fill=black](2.6,2.02)
\tkzDrawPoint[size=7,fill=black](1.5,2.7)
\node[fill=white,inner sep=.1mm] at (1.95,0.6) {\scriptsize $ x_1<x_2\Rightarrow f(x_1)>f(x_2)$};
\end{tikzpicture}\captionof{figure}{Γνησίως φθίνουσα} \\ 
\end{tabular} 
\end{center}
\end{enumerate}
\Orismos{Ολικά Ακρότατα}
Ακρότατα ονομάζονται οι μέγιστες ή ελάχιστες τιμές μιας συνάρτησης $ f:D_f\rightarrow\mathbb{R} $ τις οποίες παίρνει σε ένα διάστημα ή σε ολόκληρο το πεδίο ορισμού της.
\begin{enumerate}[itemsep=0mm,label=\bf\arabic*.]
\item \textbf{Ολικό μέγιστο}\\
Μια συνάρτηση $ f:D_f\rightarrow\mathbb{R} $ παρουσιάζει ολικό μέγιστο σε ένα σημείο $ x_0\in D_f $ του πεδίου ορισμού της όταν η τιμή $ f(x_0) $ είναι μεγαλύτερη από κάθε άλλη $ f(x) $ για κάθε σημείο $ x_0 $ του πεδίου ορισμού. \[ f(x)\leq f(x_0)\;\;,\;\;\textrm{για κάθε } x\in D_f \]
\item \textbf{Ολικό ελάχιστο}\\
Μια συνάρτηση $ f:D_f\rightarrow\mathbb{R} $ παρουσιάζει ολικό ελάχιστο σε ένα σημείο $ x_0\in D_f $ του πεδίου ορισμού της όταν η τιμή $ f(x_0) $ είναι μικρότερη από κάθε άλλη $ f(x) $ για κάθε σημείο $ x_0 $ του πεδίου ορισμού. \[ f(x)\geq f(x_0)\;\;,\;\;\textrm{για κάθε } x\in D_f \]
\begin{center}
\begin{tabular}{p{5cm}p{5cm}}
\begin{tikzpicture}
\begin{axis}[x=1cm,y=1cm,aks_on,xmin=-.7,xmax=3.2,
ymin=-1,ymax=2,ticks=none,xlabel={\footnotesize $ x $},
ylabel={\footnotesize $ y $},belh ar,clip=false]
\addplot[grafikh parastash,domain=-.3:2.3]{-x^2+2*x};
\end{axis}
\tkzDrawPoint[size=7,fill=black](1.7,2)
\node at (1.95,0.4) {\scriptsize $ f(x)\leq f(x_0)$};
\draw[dashed] (1.7,1) node[anchor=north]{\scriptsize $x_0$} -- 
(1.7,2)--(0.7,2) node[left]{\scriptsize $f(x_0)$};
\node at (0.5,0.8) {\footnotesize $ O $};
\end{tikzpicture}\captionof{figure}{Ολικό μέγιστο}	& \begin{tikzpicture}
\begin{axis}[x=1cm,y=1cm,aks_on,xmin=-.7,xmax=3,
ymin=-.7,ymax=2.3,ticks=none,xlabel={\footnotesize $ x $},
ylabel={\footnotesize $ y $},belh ar,clip=false]
\addplot[grafikh parastash,domain=-.3:2.3]{x^2-2*x+1.5};
\end{axis}
\tkzDrawPoint[size=7,fill=black](1.7,1.2)
\node at (2.1,0.2) {\scriptsize $ f(x)\leq f(x_0)$};
\draw[dashed] (1.7,0.7) node[anchor=north]{\scriptsize $x_0$} -- 
(1.7,1.2)--(0.7,1.2) node[left]{\scriptsize $f(x_0)$};
\node[fill=white,inner sep=.5pt] at (0.5,0.5) {\footnotesize $ O $};
\end{tikzpicture}\captionof{figure}{Ολικό ελάχιστο} \\ 
\end{tabular} 
\end{center}
\end{enumerate}
\Orismos{Άρτια - Περιττή συνάρτηση}
\vspace{-5mm}
\begin{enumerate}[itemsep=0mm,label=\bf\arabic*.]
\item \textbf{Άρτια συνάρτηση}\\ Άρτια ονομάζεται μια συνάρτηση $ f:D_f\rightarrow\mathbb{R} $ για την οποία ισχύουν οι παρακάτω συνθήκες :
\begin{enumerate}[itemsep=0mm,label=\roman*.]
\item $ \forall x\in D_f\Rightarrow -x\in D_f $
\item $ f(-x)=f(x)\;,\;\forall x\in D_f$
\end{enumerate}
\item \textbf{Περιττή συνάρτηση}\\ Περιττή ονομάζεται μια συνάρτηση $ f:D_f\rightarrow\mathbb{R} $ για την οποία ισχύουν οι παρακάτω συνθήκες :
\begin{enumerate}[itemsep=0mm,label=\roman*.]
\item $ \forall x\in D_f\Rightarrow -x\in D_f $
\item $ f(-x)=-f(x)\;,\;\forall x\in D_f$
\end{enumerate}
\end{enumerate}
\begin{center}
\begin{tabular}{p{4.5cm}p{4.5cm}}
\begin{tikzpicture}
\begin{axis}[x=2cm,y=3cm,aks_on,xmin=-1,xmax=1,ymin=-.1,ymax=0.9,ticks=none,xlabel={\footnotesize $ x $},ylabel={\footnotesize $ y $},belh ar]
\addplot[grafikh parastash,domain=-.85:.85]{(x^2)};
\draw[dashed](axis cs:.7,0)node[below]{{\footnotesize $ x $}}--(axis cs:.7,.49)--(axis cs:-.7,.49)--(axis cs:-.7,0)node[below]{{\footnotesize $ -x $}};
\end{axis}
\node[fill=white,inner sep=.1mm] at (2,2.5){\scriptsize $f(-x)=f(x)$};
\end{tikzpicture}\captionof{figure}{Άρτια συνάρτηση}	& \begin{tikzpicture}
\node at (3.4,0) {\scriptsize $f(-x)=-f(x)$};
\begin{axis}[x=2cm,y=1.8cm,aks_on,xmin=-1,xmax=1,ymin=-.9,ymax=.9,ticks=none,xlabel={\footnotesize $ x $},ylabel={\footnotesize $ y $},belh ar]
\addplot[grafikh parastash,domain=-.9:.9]{(x^3)};
\draw[dashed](axis cs:.7,0)node[below]{{\footnotesize $ x $}}--(axis cs:.7,.343)--(axis cs:0,.343)node[left]{{\footnotesize $ f(x) $}};
\draw[dashed](axis cs:-.7,0)node[above]{{\footnotesize $ -x $}}--(axis cs:-.7,-.343)--(axis cs:0,-.343)node[right]{{\footnotesize $ f(-x) $}};
\end{axis}
\end{tikzpicture}\captionof{figure}{Περιττή συνάρτηση} \\ 
\end{tabular} 
\end{center}
\begin{itemize}[itemsep=0mm]
\item Η γραφική παράσταση μιας άρτιας συνάρτησης είναι συμμετρική ως προς τον κατακόρυφο άξονα.
\item H γραφική παράσταση μιας περιττής συνάρτησης είναι συμμετρική ως προς την αρχή των αξόνων.
\item Η αρχή των αξόνων για μια περιττή συνάρτηση ονομάζεται \textbf{κέντρο συμμετρίας} της.
\end{itemize}
\section{Μετατόπιση γραφικής παράστασης}
\thewrhmata
\Thewrhma{Κατακόρυφη μετατόπιση}
Η γραφική παράσταση $ C_f $ μιας συνάρτησης $ f $ μετατοπίζεται κατακόρυφα κατά $ c $ μονάδες προς τα πάνω ή προς τα κάτω, εαν αυξήσουμε ή μειώσουμε αντίστοιχα τις τεταγμένες $ f(x) $ των σημείων της κατά $ c $ μονάδες.
\[ g(x)=f(x)\pm c\;\;,\;\;c>0 \]
Η γραφική παράσταση $ C_g $ της νέας συνάρτησης $ g(x) $ προκύπτει από κατακόρυφη μετατόπιση της $ C_f $ κατά $ c $ μονάδες.
\begin{center}
\begin{tabular}{p{5cm}cp{5cm}}
\begin{tikzpicture}
\begin{axis}[aks_on,belh ar,xlabel={\footnotesize$x$},ylabel={\footnotesize$y$}
,xmin=-2,xmax=2.,ymin=-1,ymax=3,x=1cm,y=1cm]
\addplot[clip=false,domain=-1.8:1.8,grafikh parastash]{x^2-.7};
\addplot[domain=-1.8:1.8,pl,samples=200]{x^2};
\addplot[domain=-1.5:1.5,grafikh parastash]{x^2+.7};
\draw[pl,-latex] (axis cs:.5,.25) -- (axis cs:.5,.95);
\draw[pl,-latex] (axis cs:.5,.25) -- (axis cs:.5,-.45);
\draw[pl,-latex] (axis cs:-.5,.25) -- (axis cs:-.5,-.45);
\draw[pl,-latex] (axis cs:-.5,.25) -- (axis cs:-.5,.95);
\node at (axis cs:-.25,.5) {\footnotesize$+c$};
\node at (axis cs:-.25,-.25) {\footnotesize$-c$};
\node at (axis cs:.25,.5) {\footnotesize$+c$};
\node at (axis cs:.25,-.25) {\footnotesize$-c$};
\end{axis}
\end{tikzpicture} & & \begin{tikzpicture}
\begin{axis}[clip=false,aks_on,belh ar,xlabel={\footnotesize$x$},ylabel={\footnotesize$y$}
,xmin=-2,xmax=4.2,ymin=-1,ymax=3,x=1cm,y=1cm]
\addplot[domain=-1.8:1.8,grafikh parastash]{x^2-.7};
\addplot[domain=-.8:2.8,pl,samples=200]{(x-1)^2-.7};
\addplot[domain=.2:3.8,grafikh parastash]{(x-2)^2-.7};
\draw[pl,-latex] (axis cs:2,.3) -- (axis cs:3,.3);
\draw[pl,-latex] (axis cs:2.5,1.55) -- (axis cs:3.5,1.55);
\draw[pl,-latex] (axis cs:0,.3) -- (axis cs:-1,.3);
\draw[pl,-latex] (axis cs:-.5,1.55) -- (axis cs:-1.5,1.55);
\node at (axis cs:-.5,.5) {\footnotesize$+c$};
\node at (axis cs:3,1.7) {\footnotesize$-c$};
\node at (axis cs:2.5,.5) {\footnotesize$-c$};
\node at (axis cs:-1,1.7) {\footnotesize$+c$};
\end{axis}
\end{tikzpicture} \\ 
\end{tabular} 
\end{center}
\Thewrhma{Οριζόντια μετατόπιση}
Η γραφική παράσταση $ C_f $ μιας συνάρτησης $ f $ μετατοπίζεται οριζόντια κατά $ c $ μονάδες προς τα αριστερά ή προς τα δεξιά, εαν αυξήσουμε ή μειώσουμε αντίστοιχα τις τετμημένες $ x $ των σημείων της κατά $ c $ μονάδες.
\[ g(x)=f(x\pm c)\;\;,\;\;c>0  \]
Η γραφική παράσταση $ C_g $ της νέας συνάρτησης $ g(x) $ προκύπτει από οριζόντια μετατόπιση της $ C_f $ κατά $ c $ μονάδες.
\end{document}







