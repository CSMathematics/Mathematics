\documentclass[twoside,10pt]{book}
\usepackage[amsbb,mtpfrak,zswash,mtpcal]{mtpro2}
\usepackage[no-math,cm-default]{fontspec}
\usepackage{xunicode}
\usepackage{xltxtra}
\usepackage{xgreek}
\defaultfontfeatures{Mapping=tex-text,Scale=MatchLowercase}
\setmainfont[Mapping=tex-text,Numbers=Lining,Scale=1.0,BoldFont={Minion Pro Bold}]{Minion Pro}
\defaultfontfeatures{Ligatures=TeX}
\font\kefalaio="Minion Pro Bold" at 36pt
\font\ArKef="Minion Pro Bold Italic" at 72pt
\font\OnKef="Times New Roman" at 20pt
\font\OnPar="Minion Pro Bold" at 18pt
\newfontfamily\scfont{GFS Artemisia}
\usepackage[inner=2.00cm, outer=1.50cm, top=3.00cm, bottom=2.00cm,paperwidth=17cm,paperheight=24cm]{geometry}
\usepackage{amsmath}
\usepackage[amsbb,mtpfrak,zswash,mtpcal]{mtpro2}
\usepackage{makeidx}
\usepackage{longtable,xcolor,varwidth}
\usepackage{float}
\usepackage{subfig}
\def\xrwma{cyan!70!black}
\def\xrwmath{cyan}
\usepackage{etoolbox}
\makeatletter
\newif\ifLT@nocaption
\preto\longtable{\LT@nocaptiontrue}
\appto\endlongtable{%
\ifLT@nocaption
\addtocounter{table}{\m@ne}%
\fi}
\preto\LT@caption{%
\noalign{\global\LT@nocaptionfalse}}
\makeatother
\makeindex
\usepackage{tikz,pgfplots}
\usepackage{tkz-euclide,tkz-fct}
\usetikzlibrary{fadings}
\usepackage{wrap-rl}
\usetkzobj{all}
\usepackage{calc}
\usepackage{cleveref}
\usepackage[colorlinks=false, pdfborder={0 0 0}]{hyperref}
\usepackage[framemethod=TikZ]{mdframed}
\definecolor{steelblue}{cmyk}{.7,.278,0,.294}
\definecolor{doc}{cmyk}{1,0.455,0,0.569}
\definecolor{olivedrab}{cmyk}{0.25,0,0.75,0.44}
\usepackage{capt-of}
\usepackage{titletoc}
\usepackage[explicit]{titlesec}
\usepackage{graphicx}
\usepackage{multicol}
\usepackage{multirow}
\usepackage{enumitem}
\usepackage{tabularx}
\usepackage[decimalsymbol=comma]{siunitx}
\tikzset{>=latex}
\makeatletter
\pretocmd{\@part}{\gdef\parttitle{#1}}{}{}
\pretocmd{\@spart}{\gdef\parttitle{#1}}{}{}
\makeatother
\usepackage[titletoc]{appendix}
\usepackage{fancyhdr}
\pagestyle{fancy}
\fancyheadoffset{0cm}
\renewcommand{\headrulewidth}{\iftopfloat{0pt}{.5pt}}
\renewcommand{\chaptermark}[1]{\markboth{#1}{}}
\renewcommand{\sectionmark}[1]{\markright{\it\thesection\ #1}}
\fancyhf{}
\fancyhead[LE]{\thepage\ $\cdot$\ \scfont\scshape\nouppercase{\leftmark}}
\fancyhead[RO]{\nouppercase{\rightmark} $\cdot$\ \thepage}
\fancypagestyle{plain}{%
\fancyhead{} %
\renewcommand{\headrulewidth}{0pt}}

\newcounter{thewrhma}[chapter]
\renewcommand{\thethewrhma}{\thechapter.\arabic{thewrhma}} 
\newcommand{\Thewrhma}[1]{\refstepcounter{thewrhma}{\textbf{\textcolor{\xrwmath}{{\large Θεώρημα\hspace{2mm}\thethewrhma\;}:\;}\hspace{1mm}}} \MakeUppercase{\textbf{#1}}\\}{}

\newcounter{porisma}[chapter]
\renewcommand{\theporisma}{\thechapter.\arabic{porisma}}\newcommand{\Porisma}[1]{\refstepcounter{porisma}\textcolor{black}{\textbf{ΠΟΡΙΣΜΑ\hspace{2mm}\theporisma\hspace{1mm} \MakeUppercase{#1}}}\\}{}

\newcounter{protasi}[chapter]
\renewcommand{\theprotasi}{\thechapter.\arabic{protasi}}\newcommand{\Protasi}[1]{\refstepcounter{protasi}\textcolor{black}{\textbf{ΠΡΟΤΑΣΗ\hspace{2mm}\theprotasi\hspace{1mm} \MakeUppercase{#1}}}\\}{}


\newcounter{orismos}[chapter]
\renewcommand{\theorismos}{\thechapter.\arabic{orismos}}   
\newcommand{\Orismos}[1]{\refstepcounter{orismos}{\textbf{\textbf{\textcolor{\xrwma}{{\large Ορισμός\hspace{2mm}\theorismos\;}:\;}}}}\hspace{1mm} \MakeUppercase{\textbf{#1}\\}}{}
\usepackage{venndiagram,mathimatika}
%-------- ΣΤΥΛ ΚΕΦΑΛΑΙΟΥ ---------
\newcommand*\chapterlabel{}
\newcommand{\fancychapter}{%
\titleformat{\chapter}
{
\normalfont\Huge}
{\gdef\chapterlabel{\thechapter\ }}{0pt}
{\begin{tikzpicture}[remember picture,overlay]
\node[yshift=-7cm] at (current page.north west)
{\begin{tikzpicture}[remember picture, overlay]
%\node[inner sep=0pt] at ($(current page.north) +			(0cm,-1.38in)$) {\includegraphics[width=17cm]{Kefalaio}};
\node[anchor=west,xshift=.1\paperwidth,yshift=.14\paperheight,rectangle]
{{\color{white}\fontsize{30}{20}\textbf{\textcolor{black}{\contour{white}{ΚΕΦΑΛΑΙΟ}}}}};
\node[anchor=west,xshift=.09\paperwidth,yshift=.08\paperheight,rectangle] {\fontsize{24}{20} {\color{black}{{\textcolor{black}{\contour{white}{\sc##1}}}}}};
%\fill[fill=black] (12.2,2) rectangle (14.8,4.7);
\node[anchor=west,xshift=.74\paperwidth,yshift=.11\paperheight,rectangle]
{{\color{white}\fontsize{80}{20}\textbf{\textit{\textcolor{white}{\contour{black}{\thechapter}}}}}};
\end{tikzpicture}
};
\end{tikzpicture}
}
\titlespacing*{\chapter}{0pt}{20pt}{30pt}
}
%------------------------------------------------


\usepackage[outline]{contour}
\newcommand{\regularchapter}{%
\titleformat{\chapter}[display]
{\normalfont\huge\bfseries}{\chaptertitlename\ \thechapter}{20pt}{\Huge##1}
\titlespacing*{\chapter}
{0pt}{-20pt}{40pt}
}

\apptocmd{\mainmatter}{\fancychapter}{}{}
\apptocmd{\backmatter}{\regularchapter}{}{}
\apptocmd{\frontmatter}{\regularchapter}{}{}

\titlespacing*{\section}
{0pt}{30pt}{0pt}
\usepackage{booktabs}
\usepackage{hhline}
\DeclareRobustCommand{\perthousand}{%
\ifmmode
\text{\textperthousand}%
\else
\textperthousand
\fi}


\contentsmargin{0cm}
\titlecontents{part}[-1pc]
{\addvspace{10pt}%
\bf\Large ΜΕΡΟΣ\quad }%
{}
{}
{\;\dotfill}%
%------------------------------------------
\titlecontents{chapter}[0pc]
{\addvspace{30pt}%
\begin{tikzpicture}[remember picture, overlay]%
\draw[fill=black,draw=black] (-.3,.5) rectangle (3.7,1.1); %
\pgftext[left,x=0cm,y=0.75cm]{\color{white}\sc\Large\bfseries Κεφάλαιο\ \thecontentslabel};%
\end{tikzpicture}\large\sc}%
{}
{}
{\hspace*{-2.3em}\hfill\normalsize Σελίδα \thecontentspage}%
\titlecontents{section}[2.4pc]
{\addvspace{1pt}}
{\contentslabel[\thecontentslabel]{2pc}}
{}
{\;\dotfill\;\small \thecontentspage}
[]
\titlecontents*{subsection}[4pc]
{\addvspace{-1pt}\small}
{}
{}
{\ --- \small\thecontentspage}
[ \textbullet\ ][]

\makeatletter
\renewcommand{\tableofcontents}{%
\chapter*{%
\vspace*{-20\p@}%
\begin{tikzpicture}[remember picture, overlay]%
\pgftext[right,x=12cm,y=0.2cm]{\Huge\sc\bfseries \contentsname};%
\draw[fill=black,draw=black] (9.5,-.75) rectangle (12.5,1);%
\clip (9.5,-.75) rectangle (15,1);
\pgftext[right,x=12cm,y=0.2cm]{\color{white}\Huge\bfseries \contentsname};%
\end{tikzpicture}}%
\@starttoc{toc}}
\makeatother

\usepackage[contents={},scale=1,opacity=1,color=black,angle=0]{background}

\newcommand\blfootnote[1]{%
\begingroup
\renewcommand\thefootnote{}\footnote{#1}%
\addtocounter{footnote}{-1}%
\endgroup
}
\usepackage{epstopdf}
\epstopdfsetup{update}
\usepackage{textcomp}

\titleformat{\section}
{\normalfont\Large\bf}%
{}{0em}%
{{\color{black}\titlerule[0pt]}\vskip-.2\baselineskip{\parbox[t]{\dimexpr\textwidth-2\fboxsep\relax}{\raggedright\strut\itshape{\LARGE{\thesection~#1}}\strut}}}[\vskip 0\baselineskip{\color{black}\titlerule[1pt]}]
\titlespacing*{\section}{0pt}{0pt}{30pt}

\newcommand{\methodologia}{\begin{center}
{\large \textbf{ΜΕΘΟΔΟΛΟΓΙΑ}}\\\vspace{-2mm}
\begin{tikzpicture}
\shade[left color=white, right color=black,] (-3cm,0) rectangle (0,.2mm);
\shade[left color=black, right color=white,] (0,0) rectangle (3cm,.2mm);   
\end{tikzpicture}
\end{center}}

\newcommand{\orismoi}{\begin{center}
\vspace{-3mm}{\large \textbf{\textcolor{\xrwma}{ΟΡΙΣΜΟΙ}}}\\\vspace{-2mm}
\begin{tikzpicture}
\shade[left color=white, right color=cyan!80!black,] (-3cm,0) rectangle (0,.2mm);
\shade[left color=cyan!80!black, right color=white,] (0,0) rectangle (3cm,.2mm);   
\end{tikzpicture}
\end{center}}
\newcommand{\thewrhmata}{\begin{center}
{\large \textbf{\textcolor{\xrwmath}{ΘΕΩΡΗΜΑΤΑ - ΠΟΡΙΣΜΑΤΑ - ΠΡΟΤΑΣΕΙΣ\\ΚΡΙΤΗΡΙΑ - ΙΔΙΟΤΗΤΕΣ}}}\\\vspace{-2mm}
\begin{tikzpicture}
\shade[left color=white, right color=\xrwmath,] (-5cm,0) rectangle (0,.2mm);
\shade[left color=\xrwmath, right color=white,] (0,0) rectangle (5cm,.2mm);   
\end{tikzpicture}
\end{center}}
\usepackage[labelfont={footnotesize,it,bf},font={footnotesize}]{caption}

%-------- ΠΙΝΑΚΕΣ ---------
\usepackage{booktabs}
%----------------------
%----- ΥΠΟΛΟΓΙΣΤΗΣ ----------
%\usepackage{calculator}
%----------------------------

%----- ΟΡΙΖΟΝΤΙΑ ΛΙΣΤΑ ------
\usepackage{xparse}
\newcounter{answers}
\renewcommand\theanswers{\arabic{answers}}
\ExplSyntaxOn
\NewDocumentCommand{\results}{m}
{
\seq_set_split:Nnn \l_results_a_seq {,}{#1}
\par\nobreak\noindent\setcounter{answers}{0}
\seq_map_inline:Nn \l_results_a_seq
{
\makebox[.18\linewidth][l]{\stepcounter{answers}\theanswers.~##1}\hfill
}
\par
}
\seq_new:N \l_results_a_seq
\ExplSyntaxOff
%----------------------------
%------ ΜΗΚΟΣ ΓΡΑΜΜΗΣ ΚΛΑΣΜΑΤΟΣ ---------
\DeclareRobustCommand{\frac}[3][0pt]{%
{\begingroup\hspace{#1}#2\hspace{#1}\endgroup\over\hspace{#1}#3\hspace{#1}}}
%----------------------------------------
\usepackage{microtype}
\usepackage{float}

\usepackage{caption}

%---- ΟΡΙΖΟΝΤΙΟ - ΚΑΤΑΚΟΡΥΦΟ - ΠΛΑΓΙΟ ΑΓΚΙΣΤΡΟ ------
\newcommand{\orag}[3]{\node at (#1)
{$ \overcbrace{\rule{#2mm}{0mm}}^{{\scriptsize #3}} $};}

\newcommand{\kag}[3]{\node at (#1)
{$ \undercbrace{\rule{#2mm}{0mm}}_{{\scriptsize #3}} $};}

\newcommand{\Pag}[4]{\node[rotate=#1] at (#2)
{$ \overcbrace{\rule{#3mm}{0mm}}^{{\rotatebox{-#1}{\scriptsize$#4$}}}$};}
%-----------------------------------------
\tikzstyle{pl}=[line width=0.3mm]
\tikzstyle{plm}=[line width=0.4mm]
%------- ΣΤΥΛ ΠΑΡΑΔΕΙΓΜΑΤΟΣ -------
\newcounter{paradeigma}[section]
\renewcommand{\theparadeigma}{\bf\thechapter.\arabic{paradeigma}}   
\newcommand{\Paradeigma}[1]{\refstepcounter{paradeigma}\textcolor{cyan}{\textbf{{\large Παράδειγμα\hspace{2mm}\theparadeigma\;:\;}\hspace{1mm}}} \MakeUppercase{\textbf{#1}}\\}{}
%-----------------------------------

%------- ΣΤΥΛ ΛΥΣΗΣ ------------------
\newcommand{\lysh}{{\textbf{ΛΥΣΗ}}}
%------------------------------------

%------ ΛΥΜΕΝΑ ΠΑΡΑΔΕΙΓΜΑΤΑ ΤΙΤΛΟΣ ---------
\newcommand{\Lymena}{\begin{center}
\begin{tikzpicture}
\path[left color=cyan!70!black,right color=cyan!80!black,middle color=cyan!80!white] (-7cm,-.6cm) rectangle (6.5cm,.6cm);
\node at (-.25cm,0) {\Large \textcolor{white}{\textbf{ΛΥΜΕΝΑ ΠΑΡΑΔΕΙΓΜΑΤΑ}}};  
\end{tikzpicture}
\end{center}}
%--------------------------------------

%--------- ΑΛΥΤΕΣ ΑΣΚΗΣΕΙΣ ΤΙΤΛΟΣ ----------
\newcommand{\Alyta}{\begin{center}
\begin{tikzpicture}
\path[left color=cyan!70!black,right color=cyan!80!black,middle color=cyan!80!white] (-7cm,-.6cm) rectangle (6.5cm,.6cm);
\node at (-.25cm,0) {\Large \textcolor{white}{\textbf{ΑΣΚΗΣΕΙΣ - ΠΡΟΒΛΗΜΑΤΑ}}};  
\end{tikzpicture}
\end{center}}
%--------------------------------------------
\usetikzlibrary{shadows,calc}
\usepackage{tcolorbox}
\tcbuselibrary{skins,theorems,breakable}
%---------- ΜΕΘΟΔΟΣ --------------
\newcounter{Methodos}[chapter]
\renewcommand{\theMethodos}{\thechapter.\arabic{Methodos}}
\newenvironment{Methodos}[2][\linewidth]
{\refstepcounter{Methodos}
\begin{tcolorbox}[breakable,
enhanced standard,
boxrule=0.7pt,titlerule=-.2pt,drop fuzzy shadow southeast=black,
width=\linewidth,
title style={color=white},
overlay unbroken and first={
\path[left color=cyan!70!black,right color=cyan,draw=black]
([yshift=-\pgflinewidth]frame.north west) to ([yshift=-5pt]title.south west)[rounded corners=2pt] -- ([xshift=-#2-15pt,yshift=-5pt]title.south east) to[rounded corners=2pt] ([xshift=-#2,yshift=-\pgflinewidth]frame.north east) -- cycle;
},
fonttitle=\bfseries,
before=\par\medskip\noindent,
after=\par\medskip,
toptitle=3pt,
top=11pt,topsep at break=-5pt,
colback=white,title={\large Μέθοδος \theMethodos} : {\textcolor{black}{\MakeUppercase{#1}}}]}
{\end{tcolorbox}}
%------------------------------------------
%---------- ΛΙΣΤΕΣ ----------------------
\newlist{bhma}{enumerate}{3}
\setlist[bhma]{label=\bf\textit{\arabic*\textsuperscript{o}\;Βήμα :},leftmargin=0cm,itemindent=1.5cm,ref=\bf{\arabic*\textsuperscript{o}\;Βήμα}}
\newlist{rlist}{enumerate}{3}
\setlist[rlist]{itemsep=0mm,label=\roman*.}


%----ΣΤΥΛ ΑΣΚΗΣΗΣ ----------
\newcounter{askhsh}[chapter]
\renewcommand{\theaskhsh}{\bf{{\large{\thechapter}}.\arabic{askhsh}}}   
\newcommand{\Askhsh}{\refstepcounter{askhsh}\textcolor{\xrwma}{{\theaskhsh}\hspace{1mm}}}{}
%---------------------------

\newlist{brlist}{enumerate}{3}
\setlist[brlist]{itemsep=0mm,label=\bf\roman*.}
\newlist{tropos}{enumerate}{3}
\setlist[tropos]{label=\bf\textit{\arabic*\textsuperscript{oς}\;Τρόπος :},leftmargin=0cm,itemindent=2.3cm,ref=\bf{\arabic*\textsuperscript{oς}\;Τρόπος}}
% Αν μπει το bhma μεσα σε tropo τότε
%\begin{bhma}[leftmargin=.7cm]
\newcommand{\tss}[1]{\textsuperscript{#1}}
\newcommand{\tssL}[1]{\MakeLowercase{\textsuperscript{#1}}}
%------------------------------------------
\setlength{\parindent}{0pt}
\setlist[itemize]{itemsep=0mm}
\tkzSetUpPoint[size=7,fill=white]
\newcommand{\twocolkentro}[1]{
\twocolumn[
\begin{@twocolumnfalse}
#1
\end{@twocolumnfalse}]}
\newcommand{\bcc}[1]{
\begin{center}
{\color{\xrwma}{\hrulefill}\raisebox{-2.5mm}{\rule{.4pt}{5mm}}}\hspace{1em}\raisebox{-.65ex}{\begin{varwidth}{\dimexpr0.7\textwidth-2em\relax}\centering{\textbf{\textcolor{\xrwma}{#1}}}\end{varwidth}}\hspace*{1em}{\color{\xrwma}{\raisebox{-2.5mm}{\rule{.4pt}{5mm}}\hrulefill}}
\end{center}}



\begin{document}
\mainmatter
\pagestyle{fancy}
\chapter{Πολυώνυμα}
\section{Η έννοια του πολυωνύμου}
\orismoi
\Orismos{Μεταβλητή}
Μεταβλητή ονομάζεται το σύμβολο το οποίο χρησιμοποιούμε για εκφράσουμε έναν άγνωστο αριθμό. Η μεταβλητή μπορεί να βρίσκεται μέσα σε μια εξίσωση και γενικά σε μια αλγεβική παράσταση.
Συμβολίζεται με ένα γράμμα όπως $ a,\beta,x,y,\ldots $ κ.τ.λ.\\\\
\Orismos{ΜΟΝΏΝΥΜΟ}
Μονώνυμο ονομάζεται η ακέραια αλγεβρική παράσταση η οποία έχει μεταξύ των μεταβλητών μόνο την πράξη του πολλαπλασιασμού.
\[ \textrm{{\scriptsize Συντελεστής} }\longrightarrow a\cdot \undercbrace{x^{\nu_1}y^{\nu_2}\cdot \ldots\cdot z^{\nu_\kappa}}_{\textrm{κύριο μέρος}}\;\;,\;\;\nu_1,\nu_2,\ldots,\nu_\kappa\in\mathbb{N} \]
\begin{itemize}[itemsep=0mm]
\item Το γινόμενο των μεταβλητών ενός μονωνύμου ονομάζεται \textbf{κύριο μέρος}.
\item  Ο σταθερός αριθμός με τον οποίο πολλαπλασιάζουμε το κύριο μέρος ενός μονωνύμου ονομάζεται \textbf{συντελεστής}.
\item Τα μονώνυμα μιας μεταβλητής είναι της μορφής $ ax^\nu $, όπου $ a\in\mathbb{R} $ και $ \nu\in\mathbb{N} $.
\end{itemize}
\Orismos{ΠΟΛΥΏΝΥΜΟ}	Πολυώνυμο ονομάζεται η ακέραια αλγεβρική παράσταση η οποία είναι άθροισμα
ανόμοιων μονωνύμων.
\begin{itemize}[itemsep=0mm]
\item Κάθε μονώνυμο μέσα σ' ένα πολυώνυμο ονομάζεται \textbf{όρος} του πολυωνύμου.
\item Το πολυώνυμο με 3 όρους ονομάζεται \textbf{τριώνυμο}.
\item Οι αριθμοί ονομάζονται \textbf{σταθερά πολυώνυμα} ενώ το 0 \textbf{μηδενικό πολυώνυμο}.
\item  Κάθε πολυώνυμο συμβολίζεται με ένα κεφαλαίο γράμμα όπως : $ P, Q, A, B\ldots $ τοποθετώντας δίπλα από το όνομα μια παρένθεση η οποία περιέχει τις μεταβλητές του δηλαδή : $ P(x), Q(x,y), A(z,w), B(x_1,x_2,\ldots,x_\nu) $.
\item \textbf{Βαθμός} ενός πολυωνύμου ορίζεται ως ο μεγαλύτερος εκθέτης της κάθε μεταβλητής. Ο όρος που περιέχει τη μεταβλητή με το μεγαλύτερο εκθέτη ονομάζεται \textbf{μεγιστοβάθμιος}.
\item Τα πολυώνυμα μιας μεταβλητής τα γράφουμε κατά φθίνουσες δυνάμεις της μεταβλητής δηλαδή από τη μεγαλύτερη στη μικρότερη. Έχουν τη μορφή :
\end{itemize}
\[ P(x)=a_\nu x^\nu+a_{\nu-1}x^{\nu-1}+\ldots+a_1x+a_0 \]
\Orismos{ΤΙΜΉ ΠΟΛΥΩΝΎΜΟΥ}
Τιμή ενός πολυωνύμου $ P(x)=a_\nu x^\nu+a_{\nu-1}x^{\nu-1}+\ldots+a_1x+a_0 $ ονομάζεται ο πραγματικός αριθμός που προκύπτει ύστερα από πράξεις αν αντικαταστίσουμε τη μεταβλητή του πολυωνύμου με έναν αριθμό $ x_0 $. Συμβολίζεται με $ P(x_0) $ και είναι ίση με :
\[ P(x_0)=a_\nu x_0^\nu+a_{\nu-1}x_0^{\nu-1}+\ldots+a_1x_0+a_0 \]
\Orismos{ΡΊΖΑ ΠΟΛΥΩΝΎΜΟΥ}
Ρίζα ενός πολυωνύμου $ P(x)=a_\nu x^\nu+a_{\nu-1}x^{\nu-1}+\ldots+a_1x+a_0 $ ονομάζεται κάθε πραγματικός αριθμός $ \rho\in\mathbb{R} $ ο οποίος μηδενίζει το πολυώνυμο.
\[ P(\rho)=0 \]
\thewrhmata
\Thewrhma{Βαθμός πολυωνύμου}
Έστω δύο πολυώνυμα $ A(x)=a_\nu x^\nu+a_{\nu-1}x^{\nu-1}+\ldots+a_1x+a_0 $ και $ B(x)=\beta_\mu x^\mu+\beta_{\mu-1}x^{\mu-1}+\ldots+\beta_1x+\beta_0 $ βαθμών $ \nu $ και $ \mu $ αντίστοιχα με $ \nu\geq\mu $. Τότε ισχύουν οι παρακάτω προτάσεις :
\begin{rlist}
\item Ο βαθμός του αθροίσματος ή της διαφοράς $ A(x)\pm B(x) $ είναι μικρότερος ίσος του μέγιστου των βαθμών των πολυωνύμων $ A(x) $ και $ B(x) $ : $ \textrm{βαθμός}(A(x)+B(x))\leq\max\{\nu,\mu\} $.
\item Ο βαθμός του γινομένου $ A(x)\cdot B(x) $ ισούται με το άθροισμα των βαθμών των πολυωνύμων $ A(x) $ και $ B(x) $ : $ \textrm{βαθμός}(A(x)\cdot B(x))=\nu+\mu $.
\item Ο βαθμός του πηλίκου $ \pi(x) $ της διαίρεσης $ A(x):B(x) $ ισούται με τη διαφορά των βαθμών των πολυωνύμων $ A(x) $ και $ B(x) $ : $ \textrm{βαθμός}(A(x): B(x))=\nu-\mu $.
\item Ο βαθμός της δύναμης $ [A(x)]^\kappa $ του πολυωνύμου $ A(x) $ ισούται με το γινόμενο του εκθέτη $ \kappa $ με το βαθμό του $ A(x) $ : $ \textrm{βαθμός}([A(x)]^\kappa)=\nu\cdot\kappa $.
\end{rlist}
\Thewrhma{Ίσα πολυώνυμα}
Δύο πολυώνυμα $ A(x)=a_\nu x^\nu+a_{\nu-1}x^{\nu-1}+\ldots+a_1x+a_0 $ και $ B(x)=\beta_\mu x^\mu+\beta_{\mu-1}x^{\mu-1}+\ldots+\beta_1x+\beta_0 $ βαθμών $ \nu $ και $ \mu $ αντίστοιχα με $ \nu\geq\mu $ θα είναι μεταξύ τους ίσα αν και μόνο αν οι συντελεστές των ομοβάθμιων όρων τους είναι ίσοι.
\begin{gather*}
A(x)=B(x)\Leftrightarrow a_i=\beta_i\ ,\ \textrm{για κάθε }i=0,1,2,\ldots,\mu\\
\textrm{και }a_i=0\ ,\ \textrm{για κάθε }i=\mu+1,\mu+2,\ldots,\nu
\end{gather*}
Ένα πολυώνυμο $ A(x)=a_\nu x^\nu+a_{\nu-1}x^{\nu-1}+\ldots+a_1x+a_0 $ ισούται με το μηδενικό πολυώνυμο αν και μόνο αν όλοι του οι συντελεστές είναι μηδενικοί.
\[ A(x)=0\Leftrightarrow a_i=0 \ ,\ \textrm{για κάθε }i=0,1,2,\ldots,\nu\]
\newpage
\noindent
\Lymena
\begin{Methodos}[Τιμή πολυωνύμου]{5cm}
Αν $ A(x) $ είναι ένα πολυώνυμο μιας μεταβλητής τότε προκειμένου να υπολογίσουμε την τιμή του για δοσμένη τιμή της μεταβλητής του
\begin{bhma}
\item \textbf{Αντικατάσταση τιμών}\\
Αντικαθιστούμε την τιμή της μεταβλητής $ x $ που μας δίνεται στο πολυώνυμο, οπότε μετατρέπεται από αλγεβρική σε αριθμιτική παράσταση.
\item \textbf{Πράξεις}\\
Εκτελούμε τις πράξεις μέσα στην αριθμιτική παράσταση που προέκυψε με τη γνωστή σειρά και υπολογίζουμε το αποτέλεσμα.
\end{bhma}
\end{Methodos}
\Paradeigma{Υπολογισμόσ τιμήσ}
\textbf{Να υπολογιστεί η τιμή του παρακάτω πολυωνύμου}
{\boldmath  $ A(x)=x^3+4x^2+3x-7 $}
\textbf{εαν θέσουμε όπου {\boldmath$ x=-2 $}}.\\\\
\lysh\\
Αν θέσουμε όπου $ x=-2$ τότε προκύπτει η παρακάτω αριθμητική παράσταση :
\begin{align*}
A(-2)=(-2)^3+4(-2)^2+3(-2)-7=-8+4\cdot 4+3(-2)-7=-8+16-6-7=-5
\end{align*}
Η τιμή λοιπόν του πολυωνύμου για τις δοσμένες τιμές των μεταβλητών του θα είναι ίση με $ -41 $.\\\\
\Paradeigma{Υπολογισμόσ τιμήσ}
\textbf{Να υπολογιστεί η τιμή του παρακάτω πολυωνύμου}
{\boldmath $ P(x)=5x^3-3x^2+2x-4 $}
\textbf{εαν μας δίνεται οτι {\boldmath$ x=1$}}.\\\\
\lysh\\
Το πολυώνυμο που μας δίνεται είναι μιας μεταβλητής. Θέτοντας λοιπόν όπου $ x=1 $ η τιμή του θα συμβολιστεί με $ P(1) $. Θα έχουμε λοιπόν
\begin{align*} P(x)=5x^3-3x^2+2x-4\xRightarrow{x=1}P(1)&=5\cdot 1^3-3\cdot1^2+2\cdot1-4\\
&=5\cdot 1-3\cdot1+2\cdot1-4\\
&=5-3+2-4=0 
\end{align*}
Προέκυψε λοιπόν η τιμή του πολυωνύμου $ P(1)=0 $ όποτε ο αριθμός $ 1 $ είναι ρίζα του πολυωνύμου.
\begin{Methodos}[Αλλαγή μεταβλητήσ]{5cm}
Όπως και στην προηγούμενη μέθοδο αντικαταστήσαμε στη θέση των μεταβλητών σταθερούς αριθμούς με τον ίδιο τρόπο μπορούμε να θέσουμε στη θέση των αρχικών μεταβλητών, νέες μεταβλητές.
\begin{bhma}
\item \textbf{Αντικατάσταση}\\
Αντικαθιστούμε στη θέση των αρχικών μεταβλητών τις νέες μεταβλητές που μας δίνονται.
\item \textbf{Απλοποίηση}\\
Προκύπτει τότε μια νέα αλγεβρική παράσταση την οποία απλοποιούμε εκτελώνας όλες τις δυνατές πράξεις.
\end{bhma}
\end{Methodos}
\Paradeigma{Αλλαγή μεταβλητήσ}
\textbf{Δίνεται το πολυώνυμο {\boldmath$ P(x)=2x^2-3x+5 $}. Να βρεθούν τα πολυώνυμα}
{\boldmath
\begin{multicols}{3}
\begin{brlist}
\item $ P(t) $
\item $ P(2x) $
\item $ P(-3s) $
\end{brlist}\end{multicols}}
\lysh
\begin{rlist}
\item Αντικαθιστώντας τη μεταβλητή $ t $ στη θέση της μεταβλητής $ x $ του πολυωνύμου $ P $ παρατηρούμε οτι γίνεται μόνο αλλαγή του συμβολισμού της πράγμα που σημαίνει οτι η δομή του πολυωνύμου δεν θα αλλάξει. Έχουμε λοιπόν
\[ P(x)=2x^2-3x+5\xRightarrow{x\rightarrow t}P(t)=2t^2-3t+5 \]
\item Θέτοντας στη θέση της μεταβλητής $ x $ το μονώνυμο $ 2x $ στο πολυώνυμο $ P $ θα προκύψει
\begin{align*}
 P(x)=2x^2-3x+5\xRightarrow{x\rightarrow 2x}P(2x)&=2(2x)^2-3\cdot(2x)+5\\
&=2\cdot 4x^2-6x+5=8x^2-6x+5
\end{align*}
\item Θέτοντας όπου $ x $ το μονώνυμο $ -3s $ έχουμε
\begin{align*}
 P(x)=2x^2-3x+5\xRightarrow{x\rightarrow -3s}P(-3s)&=2(-3s)^2-3\cdot(-3s)+5\\
&=2\cdot 9s^2+9s+5=18s^2+9s+5
\end{align*}
\end{rlist}
\begin{Methodos}[Ισότητα πολυωνύμων]{4cm}
Γνωρίζουμε ότι δύο πολυώνυμα είναι ίσα αν και μόνο αν οι συντελεστές των ομοβάθμιων όρων του είναι ίσοι. Έτσι για τον υπολογισμό των συντελεστών των πολυωνύμων :
\begin{bhma}
\item \textbf{Ίσοι συντελεστές}\\
Εξισώνουμε τους συντελεστές των ομοβάθμιων όρων τους. Αν κάποιο πολυώνυμο έχει μεγαλύτερο βαθμό τότε οι συντελεστές των παραπανίσιων όρων ισούνται με το $ 0 $.
\item \textbf{Εύρεση συντελεστών}\\
Λύνουμε τις εξισώσεις ή τα συστήματα εξισώσεων που θα προκύψουν οπότε προσδιορίζουμε τους ζητούμενους συντελεστές.
\end{bhma}
\end{Methodos}
\Paradeigma{Ισότητα πολυωνύμων}
\textbf{Δίνονται τα πολυώνυμα {\boldmath$ A(x)=x^3+\beta x^2-4x+\delta $} και {\boldmath$ B(x)=a x^3-3x^2+\gamma x-7 $}. Να υπολογίσετε τους πραγματικούς αριθμούς {\boldmath$ a,\beta,\gamma,\delta $} ώστε τα πολυώνυμα {\boldmath$ A(x),B(x) $} να είναι μεταξύ τους ίσα.}\\\\
\lysh\\
Για να ισχύει η ισότητα $ A(x)=B(x) $ θα πρέπει να έχουμε
\[ A(x)=B(x)\Leftrightarrow x^3+\beta x^2-4x+\delta=a x^3-3x^2+\gamma x-7\Leftrightarrow a=1\ ,\ \beta=-3\ ,\ \gamma=-4\ , \ \delta=-7 \]
\Paradeigma{Ισότητα πολυωνύμων}
\textbf{Δίνονται τα παρακάτω πολυώνυμα {\[ \boldmath A(x)=(a-2)x^4+3x^3-2x^2+(\beta+\gamma)x+2\ \textrm{και}\  B(x)=(2\beta-\gamma)x^3-\delta x^2+2 \]}Να υπολογίσετε τους πραγματικούς αριθμούς {\boldmath$ a,\beta,\gamma,\delta $} ώστε τα πολυώνυμα {\boldmath$ A(x),B(x) $} να είναι μεταξύ τους ίσα.}\\\\
\lysh\\
Προκειμένου να είναι τα δύο πολυώνυμα ίσα θα πρέπει να να έχουν ίσους συντελεστές οπότε προκύπτουν οι παρακάτω ισότητες :
\begin{gather*}
A(x)=B(x)\Leftrightarrow (a-2)x^4+3x^3-2x^2+(\beta+\gamma)x+2=(2\beta-\gamma)x^3-\delta x^2+2\Leftrightarrow\\
a-2=0\quad ,\quad \systeme[\beta\gamma]{\ 2\beta-\gamma=3, \ \beta+\gamma=0}\quad ,\quad -\delta=-2
\end{gather*}
Έπειτα από τη λύση των εξισώσεων και του γραμμικού συστήματος παίρνουμε τους αριθμούς : $ a=2\ ,\ \beta=2\ ,\ \gamma=-1 $ και $ \delta=2 $.
\begin{Methodos}[Πρόσθεση - Αφαίρεση πολυωνύμων]{3cm}
Για να προσθέσουμε ή να αφαιρέσουμε δύο ή περισσότερα πολυώνυμα μεταξύ τους εκτελούμε τις πράξεις μεταξύ των συντελεστών των όμοιων μονωνύμων τους κάνοντας αναγωγή ομοίων όρων.
\end{Methodos}
\Paradeigma{Πρόσθεση πολυωνύμων}
\textbf{Δίνονται τα πολυώνυμα {\boldmath$ A(x)=x^3-5x^2+2x+1 $} και {\boldmath$ B(x)=3x^3-x^2+5x+4$}. Να βρεθούν τα πολυώνυμα}
{\boldmath
\begin{multicols}{2}
\begin{brlist}
\item $ A(x)+B(x) $
\item $ B(x)-A(x) $
\end{brlist}
\end{multicols}}
\lysh\\
Όπως και στην πρόσθεση έτσι και στην αφαίρεση των πολυωνύμων θα χρειαστεί να ξεχωρίσουμε τους όμοιους μεταξύ τους όρους.
\begin{rlist}
\item Έχουμε λοιπόν
\begin{gather*}
A(x)+B(x)=\left( x^3-5x^2+2x+1\right) +\left( 3x^3-x^2+5x+4\right)=\\
x^3+3x^3-5x^2-x^2+2x+5x+1+4=4x^3-6x^2+7x+5
\end{gather*}
\item Για τη διαφορά των δύο πολυωνύμων θα χρειαστεί να αλλάξουμε τα πρόσημα του δεύτερου πολυωνύμου.
\begin{gather*}
B(x)-A(x)=\left( 3x^3-x^2+5x+4\right)-\left( x^3-5x^2+2x+1\right)=\\
3x^3-x^2+5x+4-x^3+5x^2-2x-1=2x^3+4x^2+3x+3
\end{gather*}
\end{rlist}
\begin{Methodos}[Πολλαπλασιασμόσ πολυωνύμων]{3cm}
Για τον πολλαπλασιασμό πολυωνύμων κάνουμε χρήση της επιμεριστικής ιδιότητας.
\begin{bhma}
\item \textbf{Πολλαπλασιασμός}\\
Για να πολλαπλασιάσουμε δύο πολυώνυμα μεταξύ τους πολλαπλασιάζουμε κάνοντας χρήση της επιμεριστικής ιδιότητας κάθε όρο του πρώτου με κάθεναν από τους όρους του δεύτερο πολυωνύμου.
\item \textbf{Αναγωγή ομοίων όρων}\\
Αφού βρεθεί το ανάπτυγμα του γινομένου προσθέτουμε αν υπάρχουν τους όμοιους όρους που θα προκύψουν μεταξύ τους ώστε να απλοποιηθεί η παράσταση.
\end{bhma}
\end{Methodos}
\Paradeigma{Πολλαπλασιασμόσ πολυωνύμων}
\textbf{Να υπολογιστεί το γινόμενο {\boldmath$ A(x)\cdot B(x) $} των πολυωνύμων {\boldmath$ A(x)=x^2-4x+3 $} και {\boldmath$ B(x)=3x+5 $}.}\\\\
\lysh\\
Το γινόμενο των πολυωνύμων θα έχει ως εξής :
\begin{gather*}
A(x)\cdot B(x)=\left( x^2-4x+3\right)\cdot(3x+5)=x^2\cdot 3x+x^2\cdot 5-4x\cdot 3x-4x\cdot 5+3\cdot 3x+3\cdot 5\\=
3x^3+5x^2-12x^2-20x+9x+15=3x^3-7x^2-11x+15
\end{gather*}
\begin{Methodos}[Βαθμός πολυωνύμου]{4cm}
Ο βαθμός ενός πολυωνύμου καθορίζεται από το μέγιστο εκθέτη της μεταβλητής του. Για να βρεθεί ο βαθμός ενός πολυωνύμου
\end{Methodos}
\section{Διαίρεση πολυωνύμων}
\orismoi
\Orismos{ευκλειδεια διαιρεση πολυωνυμων}
Ευκλείδεια διαίρεση ονομάζεται η διαδικασία με την οποία για κάθε ζεύγος πολυωνύμων $ \varDelta(x),\delta(x) $ (Διαιρετέος και διαιρέτης αντίστοιχα) προκύπτουν μοναδικά πολυώνυμα $ \pi(x),\upsilon(x) $ (πηλίκο και υπόλοιπο) για τα οποία ισχύει :
\[ \varDelta(x)=\delta(x)\cdot\pi(x)+\upsilon(x) \]
\begin{itemize}[itemsep=0mm]
\item Η παραπάνω ισότητα ονομάζεται \textbf{ταυτότητα της ευκλείδειας διαίρεσης}.
\item Εαν $ \upsilon(x)=0 $ τότε η διαίρεση ονομάζεται \textbf{τέλεια} ενώ η ταυτότητα της διαίρεσης είναι
\[ \varDelta(x)=\delta(x)\cdot\pi(x) \]
\item Στην τέλεια διαίρεση τα πολυώνυμα $ \delta(x),\pi(x) $ ονομάζονται \textbf{παράγοντες} ή \textbf{διαιρέτες}.
\end{itemize}
\thewrhmata
\Thewrhma{Διαίρεση με {\MakeLowercase{$ \mathbold{x-\rho} $}}}
Το υπόλοιπο της διαίρεσης ενός πολυωνύμου $ P(x) $ με διαρέτη ένα πολυώνυμο 1\tss{ου} βαθμού της μορφής $ x-\rho $ ισούται με την τιμή του πολυωνύμου $ P(x) $ για $ x=\rho $.
\[ \upsilon=P(\rho) \]
\Thewrhma{Ρίζα πολυωνύμου}
Ένα πολυώνυμο $ P(x) $ έχει παράγοντα ένα πολυώνυμο της μορφής $ x-\rho $ αν και μόνο αν ο πραγματικός αριθμός $ \rho $ είναι ρίζα του πολυωνύμου $ P(x) $.
\[ x-\rho\ \textrm{ παράγοντας }\ \Leftrightarrow P(\rho)=0 \]
\section{Διαίρεση με σχήμα Horner}
\section{Πολυωνυμικές εξισώσεις - Ανισώσεις}
\orismoi
\Orismos{Πολυωνυμικη εξισωση}
Πολυωνυμική εξίσωση ν-οστού βαθμού ονομάζεται κάθε πολυωνυμική εξίσωση της οποίας η αλγεβρική παράσταση είναι πολυώνυμο ν-οστού βαθμού.
\[ a_\nu x^\nu+a_{\nu-1}x^{\nu-1}+\ldots+a_1x+a_0=0 \]
όπου $ a_\kappa\in\mathbb{R}\;\;,\;\;\kappa=0,1,2,\ldots,\nu $. \textbf{Ρίζα} μιας πολυωνυμικής εξίσωσης ονομάζεται η ρίζα του πολυωνύμου της εξίσωσης.\\\\
\thewrhmata
\Thewrhma{Θεώρημα ακέραιων ριζών}
Αν ένας μη μεδενικός ακέραιος αριθμός $ \rho\neq0 $ είναι ρίζα μιας πολυωνυμικής εξίσωσης $ a_\nu x^\nu+a_{\nu-1}x^{\nu-1}+\ldots+a_1x+a_0=0 $ με ακέραιους συντελεστές $ a_\nu ,a_{\nu-1},\ldots,a_1,a_0\in\mathbb{Z} $ τότε ο αριθμός αυτός θα είναι διαιρέτης του σταθερού όρου $ a_0 $ του πολυωνύμου.
\section{Μη πολυωνυμικές εξισώσεις}
\orismoi
\Orismos{Κλασματικη εξίσωση}
Κλασματική ονομάζεται μια εξίσωση η οποία περιέχει τουλάχιστον μια ρητή αλγεβρική παράσταση. Γενικά έχει τη μορφή :
\[ \dfrac{P(x)}{Q(x)}+R(x)= 0\]
όπου $ P(x),Q(x),R(x) $ πολυώνυμα με $ Q(x)\neq0 $.\\\\
\Orismos{Άρρητη εξίσωση}
Άρρητη ονομάζεται κάθε εξίσωση που περιέχει τουλάχιστον μια άρρητη αλγεβρική παράσταση. Θα είναι
\[ \sqrt[\nu]{P(x)}+Q(x)=0 \]
όπου $ P(x),Q(x) $ πολυώνυμα με $ P(x)\geq0 $.\\\\
\end{document}







