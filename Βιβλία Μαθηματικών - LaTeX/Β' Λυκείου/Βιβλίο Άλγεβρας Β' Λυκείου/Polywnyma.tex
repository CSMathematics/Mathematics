\chapter{Πολυώνυμα}
\section{Η έννοια του πολυωνύμου}
\orismoi
\Orismos{Μεταβλητή}
Μεταβλητή ονομάζεται το σύμβολο το οποίο χρησιμοποιούμε για εκφράσουμε έναν άγνωστο αριθμό. Η μεταβλητή μπορεί να βρίσκεται μέσα σε μια εξίσωση και γενικά σε μια αλγεβική παράσταση.
Συμβολίζεται με ένα γράμμα όπως $ a,\beta,x,y,\ldots $ κ.τ.λ.\\\\
\Orismos{ΜΟΝΏΝΥΜΟ}
Μονώνυμο ονομάζεται η ακέραια αλγεβρική παράσταση η οποία έχει μεταξύ των μεταβλητών μόνο την πράξη του πολλαπλασιασμού.
\[ \textrm{{\scriptsize Συντελεστής} }\longrightarrow a\cdot \undercbrace{x^{\nu_1}y^{\nu_2}\cdot \ldots\cdot z^{\nu_\kappa}}_{\textrm{κύριο μέρος}}\;\;,\;\;\nu_1,\nu_2,\ldots,\nu_\kappa\in\mathbb{N} \]
\begin{itemize}[itemsep=0mm]
\item Το γινόμενο των μεταβλητών ενός μονωνύμου ονομάζεται \textbf{κύριο μέρος}.
\item  Ο σταθερός αριθμός με τον οποίο πολλαπλασιάζουμε το κύριο μέρος ενός μονωνύμου ονομάζεται \textbf{συντελεστής}.
\item Τα μονώνυμα μιας μεταβλητής είναι της μορφής $ ax^\nu $, όπου $ a\in\mathbb{R} $ και $ \nu\in\mathbb{N} $.
\end{itemize}
\Orismos{ΠΟΛΥΏΝΥΜΟ}	Πολυώνυμο ονομάζεται η ακέραια αλγεβρική παράσταση η οποία είναι άθροισμα
ανόμοιων μονωνύμων.
\begin{itemize}[itemsep=0mm]
\item Κάθε μονώνυμο μέσα σ' ένα πολυώνυμο ονομάζεται \textbf{όρος} του πολυωνύμου.
\item Το πολυώνυμο με 3 όρους ονομάζεται \textbf{τριώνυμο}.
\item Οι αριθμοί ονομάζονται \textbf{σταθερά πολυώνυμα} ενώ το 0 \textbf{μηδενικό πολυώνυμο}.
\item  Κάθε πολυώνυμο συμβολίζεται με ένα κεφαλαίο γράμμα όπως : $ P, Q, A, B\ldots $ τοποθετώντας δίπλα από το όνομα μια παρένθεση η οποία περιέχει τις μεταβλητές του δηλαδή : $ P(x), Q(x,y), A(z,w), B(x_1,x_2,\ldots,x_\nu) $.
\item \textbf{Βαθμός} ενός πολυωνύμου ορίζεται ως ο μεγαλύτερος εκθέτης της κάθε μεταβλητής. Ο όρος που περιέχει τη μεταβλητή με το μεγαλύτερο εκθέτη ονομάζεται \textbf{μεγιστοβάθμιος}.
\item Τα πολυώνυμα μιας μεταβλητής τα γράφουμε κατά φθίνουσες δυνάμεις της μεταβλητής δηλαδή από τη μεγαλύτερη στη μικρότερη. Έχουν τη μορφή :
\end{itemize}
\[ P(x)=a_\nu x^\nu+a_{\nu-1}x^{\nu-1}+\ldots+a_1x+a_0 \]
\Orismos{ΤΙΜΉ ΠΟΛΥΩΝΎΜΟΥ}
Τιμή ενός πολυωνύμου $ P(x)=a_\nu x^\nu+a_{\nu-1}x^{\nu-1}+\ldots+a_1x+a_0 $ ονομάζεται ο πραγματικός αριθμός που προκύπτει ύστερα από πράξεις αν αντικαταστίσουμε τη μεταβλητή του πολυωνύμου με έναν αριθμό $ x_0 $. Συμβολίζεται με $ P(x_0) $ και είναι ίση με :
\[ P(x_0)=a_\nu x_0^\nu+a_{\nu-1}x_0^{\nu-1}+\ldots+a_1x_0+a_0 \]
\Orismos{ΡΊΖΑ ΠΟΛΥΩΝΎΜΟΥ}
Ρίζα ενός πολυωνύμου $ P(x)=a_\nu x^\nu+a_{\nu-1}x^{\nu-1}+\ldots+a_1x+a_0 $ ονομάζεται κάθε πραγματικός αριθμός $ \rho\in\mathbb{R} $ ο οποίος μηδενίζει το πολυώνυμο.
\[ P(\rho)=0 \]
\thewrhmata
\Thewrhma{Βαθμός πολυωνύμου}
Έστω δύο πολυώνυμα $ A(x)=a_\nu x^\nu+a_{\nu-1}x^{\nu-1}+\ldots+a_1x+a_0 $ και $ B(x)=\beta_\mu x^\mu+\beta_{\mu-1}x^{\mu-1}+\ldots+\beta_1x+\beta_0 $ βαθμών $ \nu $ και $ \mu $ αντίστοιχα με $ \nu\geq\mu $. Τότε ισχύουν οι παρακάτω προτάσεις :
\begin{rlist}
\item Ο βαθμός του αθροίσματος ή της διαφοράς $ A(x)\pm B(x) $ είναι μικρότερος ίσος του μέγιστου των βαθμών των πολυωνύμων $ A(x) $ και $ B(x) $ : $ \textrm{βαθμός}(A(x)+B(x))\leq\max\{\nu,\mu\} $.
\item Ο βαθμός του γινομένου $ A(x)\cdot B(x) $ ισούται με το άθροισμα των βαθμών των πολυωνύμων $ A(x) $ και $ B(x) $ : $ \textrm{βαθμός}(A(x)\cdot B(x))=\nu+\mu $.
\item Ο βαθμός του πηλίκου $ \pi(x) $ της διαίρεσης $ A(x):B(x) $ ισούται με τη διαφορά των βαθμών των πολυωνύμων $ A(x) $ και $ B(x) $ : $ \textrm{βαθμός}(A(x): B(x))=\nu-\mu $.
\item Ο βαθμός της δύναμης $ [A(x)]^\kappa $ του πολυωνύμου $ A(x) $ ισούται με το γινόμενο του εκθέτη $ \kappa $ με το βαθμό του $ A(x) $ : $ \textrm{βαθμός}([A(x)]^\kappa)=\nu\cdot\kappa $.
\end{rlist}
\Thewrhma{Ίσα πολυώνυμα}
Δύο πολυώνυμα $ A(x)=a_\nu x^\nu+a_{\nu-1}x^{\nu-1}+\ldots+a_1x+a_0 $ και $ B(x)=\beta_\mu x^\mu+\beta_{\mu-1}x^{\mu-1}+\ldots+\beta_1x+\beta_0 $ βαθμών $ \nu $ και $ \mu $ αντίστοιχα με $ \nu\geq\mu $ θα είναι μεταξύ τους ίσα αν και μόνο αν οι συντελεστές των ομοβάθμιων όρων τους είναι ίσοι.
\begin{gather*}
A(x)=B(x)\Leftrightarrow a_i=\beta_i\ ,\ \textrm{για κάθε }i=0,1,2,\ldots,\mu\\
\textrm{και }a_i=0\ ,\ \textrm{για κάθε }i=\mu+1,\mu+2,\ldots,\nu
\end{gather*}
Ένα πολυώνυμο $ A(x)=a_\nu x^\nu+a_{\nu-1}x^{\nu-1}+\ldots+a_1x+a_0 $ ισούται με το μηδενικό πολυώνυμο αν και μόνο αν όλοι του οι συντελεστές είναι μηδενικοί.
\[ A(x)=0\Leftrightarrow a_i=0 \ ,\ \textrm{για κάθε }i=0,1,2,\ldots,\nu\]
\newpage
\noindent
\Lymena
\begin{Methodos}[Τιμή πολυωνύμου]{5cm}
Αν $ A(x) $ είναι ένα πολυώνυμο μιας μεταβλητής τότε προκειμένου να υπολογίσουμε την τιμή του για δοσμένη τιμή της μεταβλητής του
\begin{bhma}
\item \textbf{Αντικατάσταση τιμών}\\
Αντικαθιστούμε την τιμή της μεταβλητής $ x $ που μας δίνεται στο πολυώνυμο, οπότε μετατρέπεται από αλγεβρική σε αριθμιτική παράσταση.
\item \textbf{Πράξεις}\\
Εκτελούμε τις πράξεις μέσα στην αριθμιτική παράσταση που προέκυψε με τη γνωστή σειρά και υπολογίζουμε το αποτέλεσμα.
\end{bhma}
\end{Methodos}
\Paradeigma{Υπολογισμόσ τιμήσ}
\textbf{Να υπολογιστεί η τιμή του παρακάτω πολυωνύμου}
{\boldmath  $ A(x)=x^3+4x^2+3x-7 $}
\textbf{εαν θέσουμε όπου {\boldmath$ x=-2 $}}.\\\\
\lysh\\
Αν θέσουμε όπου $ x=-2$ τότε προκύπτει η παρακάτω αριθμητική παράσταση :
\begin{align*}
A(-2)=(-2)^3+4(-2)^2+3(-2)-7=-8+4\cdot 4+3(-2)-7=-8+16-6-7=-5
\end{align*}
Η τιμή λοιπόν του πολυωνύμου για τις δοσμένες τιμές των μεταβλητών του θα είναι ίση με $ -41 $.\\\\
\Paradeigma{Υπολογισμόσ τιμήσ}
\textbf{Να υπολογιστεί η τιμή του παρακάτω πολυωνύμου}
{\boldmath $ P(x)=5x^3-3x^2+2x-4 $}
\textbf{εαν μας δίνεται οτι {\boldmath$ x=1$}}.\\\\
\lysh\\
Το πολυώνυμο που μας δίνεται είναι μιας μεταβλητής. Θέτοντας λοιπόν όπου $ x=1 $ η τιμή του θα συμβολιστεί με $ P(1) $. Θα έχουμε λοιπόν
\begin{align*} P(x)=5x^3-3x^2+2x-4\xRightarrow{x=1}P(1)&=5\cdot 1^3-3\cdot1^2+2\cdot1-4\\
&=5\cdot 1-3\cdot1+2\cdot1-4\\
&=5-3+2-4=0 
\end{align*}
Προέκυψε λοιπόν η τιμή του πολυωνύμου $ P(1)=0 $ όποτε ο αριθμός $ 1 $ είναι ρίζα του πολυωνύμου.
\begin{Methodos}[Αλλαγή μεταβλητήσ]{5cm}
Όπως και στην προηγούμενη μέθοδο αντικαταστήσαμε στη θέση των μεταβλητών σταθερούς αριθμούς με τον ίδιο τρόπο μπορούμε να θέσουμε στη θέση των αρχικών μεταβλητών, νέες μεταβλητές.
\begin{bhma}
\item \textbf{Αντικατάσταση}\\
Αντικαθιστούμε στη θέση των αρχικών μεταβλητών τις νέες μεταβλητές που μας δίνονται.
\item \textbf{Απλοποίηση}\\
Προκύπτει τότε μια νέα αλγεβρική παράσταση την οποία απλοποιούμε εκτελώνας όλες τις δυνατές πράξεις.
\end{bhma}
\end{Methodos}
\Paradeigma{Αλλαγή μεταβλητήσ}
\textbf{Δίνεται το πολυώνυμο {\boldmath$ P(x)=2x^2-3x+5 $}. Να βρεθούν τα πολυώνυμα}
{\boldmath
\begin{multicols}{3}
\begin{brlist}
\item $ P(t) $
\item $ P(2x) $
\item $ P(-3s) $
\end{brlist}\end{multicols}}
\lysh
\begin{rlist}
\item Αντικαθιστώντας τη μεταβλητή $ t $ στη θέση της μεταβλητής $ x $ του πολυωνύμου $ P $ παρατηρούμε οτι γίνεται μόνο αλλαγή του συμβολισμού της πράγμα που σημαίνει οτι η δομή του πολυωνύμου δεν θα αλλάξει. Έχουμε λοιπόν
\[ P(x)=2x^2-3x+5\xRightarrow{x\rightarrow t}P(t)=2t^2-3t+5 \]
\item Θέτοντας στη θέση της μεταβλητής $ x $ το μονώνυμο $ 2x $ στο πολυώνυμο $ P $ θα προκύψει
\begin{align*}
 P(x)=2x^2-3x+5\xRightarrow{x\rightarrow 2x}P(2x)&=2(2x)^2-3\cdot(2x)+5\\
&=2\cdot 4x^2-6x+5=8x^2-6x+5
\end{align*}
\item Θέτοντας όπου $ x $ το μονώνυμο $ -3s $ έχουμε
\begin{align*}
 P(x)=2x^2-3x+5\xRightarrow{x\rightarrow -3s}P(-3s)&=2(-3s)^2-3\cdot(-3s)+5\\
&=2\cdot 9s^2+9s+5=18s^2+9s+5
\end{align*}
\end{rlist}
\begin{Methodos}[Ισότητα πολυωνύμων]{4cm}
Γνωρίζουμε ότι δύο πολυώνυμα είναι ίσα αν και μόνο αν οι συντελεστές των ομοβάθμιων όρων του είναι ίσοι. Έτσι για τον υπολογισμό των συντελεστών των πολυωνύμων :
\begin{bhma}
\item \textbf{Ίσοι συντελεστές}\\
Εξισώνουμε τους συντελεστές των ομοβάθμιων όρων τους. Αν κάποιο πολυώνυμο έχει μεγαλύτερο βαθμό τότε οι συντελεστές των παραπανίσιων όρων ισούνται με το $ 0 $.
\item \textbf{Εύρεση συντελεστών}\\
Λύνουμε τις εξισώσεις ή τα συστήματα εξισώσεων που θα προκύψουν οπότε προσδιορίζουμε τους ζητούμενους συντελεστές.
\end{bhma}
\end{Methodos}
\Paradeigma{Ισότητα πολυωνύμων}
\textbf{Δίνονται τα πολυώνυμα {\boldmath$ A(x)=x^3+\beta x^2-4x+\delta $} και {\boldmath$ B(x)=a x^3-3x^2+\gamma x-7 $}. Να υπολογίσετε τους πραγματικούς αριθμούς {\boldmath$ a,\beta,\gamma,\delta $} ώστε τα πολυώνυμα {\boldmath$ A(x),B(x) $} να είναι μεταξύ τους ίσα.}\\\\
\lysh\\
Για να ισχύει η ισότητα $ A(x)=B(x) $ θα πρέπει να έχουμε
\[ A(x)=B(x)\Leftrightarrow x^3+\beta x^2-4x+\delta=a x^3-3x^2+\gamma x-7\Leftrightarrow a=1\ ,\ \beta=-3\ ,\ \gamma=-4\ , \ \delta=-7 \]
\Paradeigma{Ισότητα πολυωνύμων}
\textbf{Δίνονται τα παρακάτω πολυώνυμα {\[ \boldmath A(x)=(a-2)x^4+3x^3-2x^2+(\beta+\gamma)x+2\ \textrm{και}\  B(x)=(2\beta-\gamma)x^3-\delta x^2+2 \]}Να υπολογίσετε τους πραγματικούς αριθμούς {\boldmath$ a,\beta,\gamma,\delta $} ώστε τα πολυώνυμα {\boldmath$ A(x),B(x) $} να είναι μεταξύ τους ίσα.}\\\\
\lysh\\
Προκειμένου να είναι τα δύο πολυώνυμα ίσα θα πρέπει να να έχουν ίσους συντελεστές οπότε προκύπτουν οι παρακάτω ισότητες :
\begin{gather*}
A(x)=B(x)\Leftrightarrow (a-2)x^4+3x^3-2x^2+(\beta+\gamma)x+2=(2\beta-\gamma)x^3-\delta x^2+2\Leftrightarrow\\
a-2=0\quad ,\quad \systeme[\beta\gamma]{\ 2\beta-\gamma=3, \ \beta+\gamma=0}\quad ,\quad -\delta=-2
\end{gather*}
Έπειτα από τη λύση των εξισώσεων και του γραμμικού συστήματος παίρνουμε τους αριθμούς : $ a=2\ ,\ \beta=2\ ,\ \gamma=-1 $ και $ \delta=2 $.
\begin{Methodos}[Πρόσθεση - Αφαίρεση πολυωνύμων]{3cm}
Για να προσθέσουμε ή να αφαιρέσουμε δύο ή περισσότερα πολυώνυμα μεταξύ τους εκτελούμε τις πράξεις μεταξύ των συντελεστών των όμοιων μονωνύμων τους κάνοντας αναγωγή ομοίων όρων.
\end{Methodos}
\Paradeigma{Πρόσθεση πολυωνύμων}
\textbf{Δίνονται τα πολυώνυμα {\boldmath$ A(x)=x^3-5x^2+2x+1 $} και {\boldmath$ B(x)=3x^3-x^2+5x+4$}. Να βρεθούν τα πολυώνυμα}
{\boldmath
\begin{multicols}{2}
\begin{brlist}
\item $ A(x)+B(x) $
\item $ B(x)-A(x) $
\end{brlist}
\end{multicols}}
\lysh\\
Όπως και στην πρόσθεση έτσι και στην αφαίρεση των πολυωνύμων θα χρειαστεί να ξεχωρίσουμε τους όμοιους μεταξύ τους όρους.
\begin{rlist}
\item Έχουμε λοιπόν
\begin{gather*}
A(x)+B(x)=\left( x^3-5x^2+2x+1\right) +\left( 3x^3-x^2+5x+4\right)=\\
x^3+3x^3-5x^2-x^2+2x+5x+1+4=4x^3-6x^2+7x+5
\end{gather*}
\item Για τη διαφορά των δύο πολυωνύμων θα χρειαστεί να αλλάξουμε τα πρόσημα του δεύτερου πολυωνύμου.
\begin{gather*}
B(x)-A(x)=\left( 3x^3-x^2+5x+4\right)-\left( x^3-5x^2+2x+1\right)=\\
3x^3-x^2+5x+4-x^3+5x^2-2x-1=2x^3+4x^2+3x+3
\end{gather*}
\end{rlist}
\begin{Methodos}[Πολλαπλασιασμόσ πολυωνύμων]{3cm}
Για τον πολλαπλασιασμό πολυωνύμων κάνουμε χρήση της επιμεριστικής ιδιότητας.
\begin{bhma}
\item \textbf{Πολλαπλασιασμός}\\
Για να πολλαπλασιάσουμε δύο πολυώνυμα μεταξύ τους πολλαπλασιάζουμε κάνοντας χρήση της επιμεριστικής ιδιότητας κάθε όρο του πρώτου με κάθεναν από τους όρους του δεύτερο πολυωνύμου.
\item \textbf{Αναγωγή ομοίων όρων}\\
Αφού βρεθεί το ανάπτυγμα του γινομένου προσθέτουμε αν υπάρχουν τους όμοιους όρους που θα προκύψουν μεταξύ τους ώστε να απλοποιηθεί η παράσταση.
\end{bhma}
\end{Methodos}
\Paradeigma{Πολλαπλασιασμόσ πολυωνύμων}
\textbf{Να υπολογιστεί το γινόμενο {\boldmath$ A(x)\cdot B(x) $} των πολυωνύμων {\boldmath$ A(x)=x^2-4x+3 $} και {\boldmath$ B(x)=3x+5 $}.}\\\\
\lysh\\
Το γινόμενο των πολυωνύμων θα έχει ως εξής :
\begin{gather*}
A(x)\cdot B(x)=\left( x^2-4x+3\right)\cdot(3x+5)=x^2\cdot 3x+x^2\cdot 5-4x\cdot 3x-4x\cdot 5+3\cdot 3x+3\cdot 5\\=
3x^3+5x^2-12x^2-20x+9x+15=3x^3-7x^2-11x+15
\end{gather*}
\begin{Methodos}[Βαθμός πολυωνύμου]{4cm}
Ο βαθμός ενός πολυωνύμου καθορίζεται από το μέγιστο εκθέτη της μεταβλητής του. Για να βρεθεί ο βαθμός ενός πολυωνύμου
\end{Methodos}
\section{Διαίρεση πολυωνύμων}
\orismoi
\Orismos{ευκλειδεια διαιρεση πολυωνυμων}
Ευκλείδεια διαίρεση ονομάζεται η διαδικασία με την οποία για κάθε ζεύγος πολυωνύμων $ \varDelta(x),\delta(x) $ (Διαιρετέος και διαιρέτης αντίστοιχα) προκύπτουν μοναδικά πολυώνυμα $ \pi(x),\upsilon(x) $ (πηλίκο και υπόλοιπο) για τα οποία ισχύει :
\[ \varDelta(x)=\delta(x)\cdot\pi(x)+\upsilon(x) \]
\begin{itemize}[itemsep=0mm]
\item Η παραπάνω ισότητα ονομάζεται \textbf{ταυτότητα της ευκλείδειας διαίρεσης}.
\item Εαν $ \upsilon(x)=0 $ τότε η διαίρεση ονομάζεται \textbf{τέλεια} ενώ η ταυτότητα της διαίρεσης είναι
\[ \varDelta(x)=\delta(x)\cdot\pi(x) \]
\item Στην τέλεια διαίρεση τα πολυώνυμα $ \delta(x),\pi(x) $ ονομάζονται \textbf{παράγοντες} ή \textbf{διαιρέτες}.
\end{itemize}
\thewrhmata
\Thewrhma{Διαίρεση με {\MakeLowercase{$ \mathbold{x-\rho} $}}}
Το υπόλοιπο της διαίρεσης ενός πολυωνύμου $ P(x) $ με διαρέτη ένα πολυώνυμο 1\tss{ου} βαθμού της μορφής $ x-\rho $ ισούται με την τιμή του πολυωνύμου $ P(x) $ για $ x=\rho $.
\[ \upsilon=P(\rho) \]
\Thewrhma{Ρίζα πολυωνύμου}
Ένα πολυώνυμο $ P(x) $ έχει παράγοντα ένα πολυώνυμο της μορφής $ x-\rho $ αν και μόνο αν ο πραγματικός αριθμός $ \rho $ είναι ρίζα του πολυωνύμου $ P(x) $.
\[ x-\rho\ \textrm{ παράγοντας }\ \Leftrightarrow P(\rho)=0 \]
\section{Διαίρεση με σχήμα Horner}
\section{Πολυωνυμικές εξισώσεις - Ανισώσεις}
\orismoi
\Orismos{Πολυωνυμικη εξισωση}
Πολυωνυμική εξίσωση ν-οστού βαθμού ονομάζεται κάθε πολυωνυμική εξίσωση της οποίας η αλγεβρική παράσταση είναι πολυώνυμο ν-οστού βαθμού.
\[ a_\nu x^\nu+a_{\nu-1}x^{\nu-1}+\ldots+a_1x+a_0=0 \]
όπου $ a_\kappa\in\mathbb{R}\;\;,\;\;\kappa=0,1,2,\ldots,\nu $. \textbf{Ρίζα} μιας πολυωνυμικής εξίσωσης ονομάζεται η ρίζα του πολυωνύμου της εξίσωσης.\\\\
\thewrhmata
\Thewrhma{Θεώρημα ακέραιων ριζών}
Αν ένας μη μηδενικός ακέραιος αριθμός $ \rho\neq0 $ είναι ρίζα μιας πολυωνυμικής εξίσωσης $ a_\nu x^\nu+a_{\nu-1}x^{\nu-1}+\ldots+a_1x+a_0=0 $ με ακέραιους συντελεστές $ a_\nu ,a_{\nu-1},\ldots,a_1,a_0\in\mathbb{Z} $ τότε ο αριθμός αυτός θα είναι διαιρέτης του σταθερού όρου $ a_0 $ του πολυωνύμου.
\section{Μη πολυωνυμικές εξισώσεις}
\orismoi
\Orismos{Κλασματικη εξίσωση}
Κλασματική ονομάζεται μια εξίσωση η οποία περιέχει τουλάχιστον μια ρητή αλγεβρική παράσταση. Γενικά έχει τη μορφή :
\[ \dfrac{P(x)}{Q(x)}+R(x)= 0\]
όπου $ P(x),Q(x),R(x) $ πολυώνυμα με $ Q(x)\neq0 $.\\\\
\Orismos{Άρρητη εξίσωση}
Άρρητη ονομάζεται κάθε εξίσωση που περιέχει τουλάχιστον μια άρρητη αλγεβρική παράσταση. Θα είναι
\[ \sqrt[\nu]{P(x)}+Q(x)=0 \]
όπου $ P(x),Q(x) $ πολυώνυμα με $ P(x)\geq0 $.