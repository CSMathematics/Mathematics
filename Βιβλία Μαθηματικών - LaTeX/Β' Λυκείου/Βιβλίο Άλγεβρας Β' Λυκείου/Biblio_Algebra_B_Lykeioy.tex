\documentclass[twoside,10pt]{book}
\usepackage[amsbb,subscriptcorrection,zswash,mtpcal,mtphrb]{mtpro2}
\usepackage[no-math,cm-default]{fontspec}
\usepackage{xunicode}
\usepackage{xgreek}
\defaultfontfeatures{Mapping=tex-text,Scale=MatchLowercase}
\setmainfont[Mapping=tex-text,Numbers=Lining,Scale=1.0,BoldFont={Constantia Bold}]{Constantia}
\defaultfontfeatures{Ligatures=TeX}
\font\kefalaio="Tinos Bold" at 36pt
\font\ArKef="Tinos Bold Italic" at 72pt
\font\OnKefh="Linux Libertine O C" at 10pt
\font\onoma="URWGothic-Demi" at 10pt
\font\onomaaskpar="Kerkis Sans Bold" at 20pt
\font\OnPar="Tinos Bold" at 18pt
\newfontfamily\scfont{Linux Libertine O C}
\usepackage[inner=2.00cm, outer=1.50cm, top=2.50cm, bottom=2.00cm,paperwidth=17cm,paperheight=24cm]{geometry}
\usepackage{amsmath}
\usepackage[amsbb,subscriptcorrection,zswash,mtpcal,mtphrb]{mtpro2}
\usepackage{makeidx}
\usepackage{longtable,xcolor,varwidth,fontawesome5}
\usepackage{float}
\usepackage{subfig}
\def\xrwma{red!70!black}
\def\xrwmath{orange!90!black}
\usepackage{etoolbox}
\makeatletter
\newif\ifLT@nocaption
\preto\longtable{\LT@nocaptiontrue}
\appto\endlongtable{%
\ifLT@nocaption
\addtocounter{table}{\m@ne}%
\fi}
\preto\LT@caption{%
\noalign{\global\LT@nocaptionfalse}}
\makeatother
\makeindex
\usepackage{tikz,pgfplots}
\usepackage{tkz-euclide}
\usetikzlibrary{fadings,positioning}
\usepackage{wrap-rl}
\usepackage{calc}
\usepackage[colorlinks=false, pdfborder={0 0 0}]{hyperref}
\usepackage{cleveref}
\usepackage[framemethod=TikZ]{mdframed}
\definecolor{steelblue}{cmyk}{.7,.278,0,.294}
\definecolor{doc}{cmyk}{1,0.455,0,0.569}
\definecolor{olivedrab}{cmyk}{0.25,0,0.75,0.44}
\usepackage{capt-of}
\usepackage{titletoc}
\usepackage[explicit]{titlesec}
\usepackage{graphicx,adjmulticol}
\usepackage{multicol}
\usepackage{multirow}
\usepackage{enumitem}
\usepackage{tabularx}
\usepackage[decimalsymbol=comma]{siunitx}
\tikzset{>=latex}
\makeatletter
\pretocmd{\@part}{\gdef\parttitle{#1}}{}{}
\pretocmd{\@spart}{\gdef\parttitle{#1}}{}{}
\makeatother
\usepackage[titletoc]{appendix}
\usepackage{fancyhdr}
\pagestyle{fancy}
\fancyheadoffset{0cm}
\renewcommand{\headrulewidth}{\iftopfloat{0pt}{.5pt}}
\renewcommand{\chaptermark}[1]{\markboth{#1}{}}
\renewcommand{\sectionmark}[1]{\markright{\it\thesection\ #1}}
\fancyhf{}
\fancyhead[LE]{\thepage\ $\cdot$\ \OnKefh\nouppercase{\leftmark}}
\fancyhead[RO]{\nouppercase{\rightmark} $\cdot$\ \thepage}
\fancypagestyle{plain}{%
\fancyhead{} %
\renewcommand{\headrulewidth}{0pt}}


\newcounter{orismos}[chapter]
\renewcommand{\theorismos}{\thechapter.\arabic{orismos}}   
\newcommand{\Orismos}{\refstepcounter{orismos}{\textbf{\textcolor{\xrwma}{\onoma{Ορισμός\hspace{2mm}\theorismos}\;}:\;}}}{}

\newcounter{thewrhma}[chapter]
\renewcommand{\thethewrhma}{\thechapter.\arabic{thewrhma}} 
\newcommand{\Thewrhma}{\refstepcounter{thewrhma}{\textbf{\textcolor{\xrwmath}{{\large \onoma{Θεώρημα\hspace{2mm}\thethewrhma}\;}:\;}}}}{}

\newcounter{porisma}[chapter]
\renewcommand{\theporisma}{\thechapter.\arabic{porisma}}\newcommand{\Porisma}[1]{\refstepcounter{porisma}\textcolor{black}{\textbf{ΠΟΡΙΣΜΑ\hspace{2mm}\theporisma\hspace{1mm} \MakeUppercase{#1}}}\\}{}

\newcounter{protasi}[chapter]
\renewcommand{\theprotasi}{\thechapter.\arabic{protasi}}\newcommand{\Protasi}[1]{\refstepcounter{protasi}\textcolor{black}{\textbf{ΠΡΟΤΑΣΗ\hspace{2mm}\theprotasi\hspace{1mm} \MakeUppercase{#1}}}\\}{}

\usepackage{venndiagram,mathimatika,eurosym}
%-------- ΣΤΥΛ ΚΕΦΑΛΑΙΟΥ ---------
\newcommand*\chapterlabel{}
\newcommand{\fancychapter}{%
\titleformat{\chapter}
{
\normalfont\Huge}
{\gdef\chapterlabel{\thechapter\ }}{0pt}
{\begin{tikzpicture}[remember picture,overlay]
\node[yshift=-7cm] at (current page.north west)
{\begin{tikzpicture}[remember picture, overlay]
%\node[inner sep=0pt] at ($(current page.north) +			(0cm,-1.38in)$) {\includegraphics[width=17cm]{Kefalaio}};
\node[anchor=west,xshift=.1\paperwidth,yshift=.14\paperheight,rectangle]
{{\color{white}\fontsize{30}{20}\textbf{\textcolor{black}{\contour{white}{ΚΕΦΑΛΑΙΟ}}}}};
\node[anchor=west,xshift=.09\paperwidth,yshift=.08\paperheight,rectangle] {\fontsize{24}{20} {\color{black}{{\textcolor{black}{\contour{white}{\sc##1}}}}}};
%\fill[fill=black] (12.2,2) rectangle (14.8,4.7);
\node[anchor=west,xshift=.74\paperwidth,yshift=.11\paperheight,rectangle]
{{\color{white}\fontsize{80}{20}\textbf{\textit{\textcolor{white}{\contour{black}{\thechapter}}}}}};
\end{tikzpicture}
};
\end{tikzpicture}
}
\titlespacing*{\chapter}{0pt}{20pt}{30pt}
}
%------------------------------------------------
\allowdisplaybreaks

\usepackage[outline]{contour}
\newcommand{\regularchapter}{%
\titleformat{\chapter}[display]
{\normalfont\huge\bfseries}{\chaptertitlename\ \thechapter}{20pt}{\Huge##1}
\titlespacing*{\chapter}
{0pt}{-20pt}{40pt}
}

\apptocmd{\mainmatter}{\fancychapter}{}{}
\apptocmd{\backmatter}{\regularchapter}{}{}
\apptocmd{\frontmatter}{\regularchapter}{}{}

\titlespacing*{\section}
{0pt}{30pt}{0pt}
\usepackage{booktabs}
\usepackage{hhline}
\DeclareRobustCommand{\perthousand}{%
\ifmmode
\text{\textperthousand}%
\else
\textperthousand
\fi}


\contentsmargin{0cm}
\titlecontents{part}[-1pc]
{\addvspace{10pt}%
\bf\Large ΜΕΡΟΣ\quad }%
{}
{}
{\;\dotfill}%
%------------------------------------------
\titlecontents{chapter}[0pc]
{\addvspace{30pt}%
\begin{tikzpicture}[remember picture, overlay]%
\draw[fill=black,draw=black] (-.3,.5) rectangle (3.7,1.1); %
\pgftext[left,x=0cm,y=0.75cm]{\color{white}\sc\Large\bfseries Κεφάλαιο\ \thecontentslabel};%
\end{tikzpicture}\large\sc}%
{}
{}
{\hspace*{-2.3em}\hfill\normalsize Σελίδα \thecontentspage}%
\titlecontents{section}[2.4pc]
{\addvspace{1pt}}
{\contentslabel[\thecontentslabel]{2pc}}
{}
{\;\dotfill\;\small \thecontentspage}
[]
\titlecontents*{subsection}[4pc]
{\addvspace{-1pt}\small}
{}
{}
{\ --- \small\thecontentspage}
[ \textbullet\ ][]

\makeatletter
\renewcommand{\tableofcontents}{%
\chapter*{%
\vspace*{-20\p@}%
\begin{tikzpicture}[remember picture, overlay]%
\pgftext[right,x=12cm,y=0.2cm]{\Huge\sc\bfseries \contentsname};%
\draw[fill=black,draw=black] (9.5,-.75) rectangle (12.5,1);%
\clip (9.5,-.75) rectangle (15,1);
\pgftext[right,x=12cm,y=0.2cm]{\color{white}\Huge\bfseries \contentsname};%
\end{tikzpicture}}%
\@starttoc{toc}}
\makeatother

\usepackage[contents={},scale=1,opacity=1,color=black,angle=0]{background}

\newcommand\blfootnote[1]{%
\begingroup
\renewcommand\thefootnote{}\footnote{#1}%
\addtocounter{footnote}{-1}%
\endgroup
}
\usepackage{epstopdf}
\epstopdfsetup{update}
\usepackage{textcomp}

\titleformat{\section}
{\normalfont\Large\bf}%
{}{0em}%
{{\color{black}\titlerule[0pt]}\vskip-.2\baselineskip{\parbox[t]{\dimexpr\textwidth-2\fboxsep\relax}{\raggedright\strut\itshape{\LARGE{\thesection~#1}}\strut}}}[\vskip 0\baselineskip{\color{black}\titlerule[1pt]}]
\titlespacing*{\section}{0pt}{0pt}{30pt}

\newcommand{\methodologia}{\begin{center}
{\large \textbf{ΜΕΘΟΔΟΛΟΓΙΑ}}\\\vspace{-2mm}
\begin{tikzpicture}
\shade[left color=white, right color=black,] (-3cm,0) rectangle (0,.2mm);
\shade[left color=black, right color=white,] (0,0) rectangle (3cm,.2mm);   
\end{tikzpicture}
\end{center}}

\newcommand{\orismoi}{\begin{center}
\vspace{-3mm}{\large \textbf{\textcolor{\xrwma}{ΟΡΙΣΜΟΙ}}}\\\vspace{-2mm}
\begin{tikzpicture}
\shade[left color=white, right color=red!80!black,] (-3cm,0) rectangle (0,.2mm);
\shade[left color=red!80!black, right color=white,] (0,0) rectangle (3cm,.2mm);   
\end{tikzpicture}
\end{center}}
\newcommand{\thewrhmata}{\begin{center}
{\large \textbf{\textcolor{\xrwmath}{ΘΕΩΡΗΜΑΤΑ - ΠΟΡΙΣΜΑΤΑ - ΠΡΟΤΑΣΕΙΣ\\ΚΡΙΤΗΡΙΑ - ΙΔΙΟΤΗΤΕΣ}}}\\\vspace{-2mm}
\begin{tikzpicture}
\shade[left color=white, right color=\xrwmath,] (-5cm,0) rectangle (0,.2mm);
\shade[left color=\xrwmath, right color=white,] (0,0) rectangle (5cm,.2mm);   
\end{tikzpicture}
\end{center}}
\usepackage[labelfont={footnotesize,it,bf},font={footnotesize}]{caption}
\setlength{\abovecaptionskip}{0pt}

%-------- ΠΙΝΑΚΕΣ ---------
\usepackage{booktabs}
%----------------------
%----- ΥΠΟΛΟΓΙΣΤΗΣ ----------
%\usepackage{calculator}
%----------------------------

%----- ΟΡΙΖΟΝΤΙΑ ΛΙΣΤΑ ------
\usepackage{xparse}
\newcounter{answers}
\renewcommand\theanswers{\arabic{answers}}
\ExplSyntaxOn
\NewDocumentCommand{\results}{m}
{
\seq_set_split:Nnn \l_results_a_seq {,}{#1}
\par\nobreak\noindent\setcounter{answers}{0}
\seq_map_inline:Nn \l_results_a_seq
{
\makebox[.18\linewidth][l]{\stepcounter{answers}\theanswers.~##1}\hfill
}
\par
}
\seq_new:N \l_results_a_seq
\ExplSyntaxOff
%----------------------------
%------ ΜΗΚΟΣ ΓΡΑΜΜΗΣ ΚΛΑΣΜΑΤΟΣ ---------
\DeclareRobustCommand{\frac}[3][0pt]{%
{\begingroup\hspace{#1}#2\hspace{#1}\endgroup\over\hspace{#1}#3\hspace{#1}}}
%----------------------------------------
\usepackage{microtype}
\usepackage{float}

\usepackage{caption}


\tikzstyle{pl}=[line width=0.3mm]
\tikzstyle{plm}=[line width=0.4mm]
%------- ΣΤΥΛ ΠΑΡΑΔΕΙΓΜΑΤΟΣ -------
\newcounter{paradeigma}[section]
\renewcommand{\theparadeigma}{\bf\onoma{\thechapter.\arabic{paradeigma}}}   
\newcommand{\Paradeigma}[1]{\refstepcounter{paradeigma}\textcolor{red!80!black}{\textbf{{\large \onoma{{\small \faPlay}\ \ Παράδειγμα\hspace{2mm}\theparadeigma}\;:\;}\hspace{1mm}}} \textbf{\onoma#1}\\}{}
%-----------------------------------

%------- ΣΤΥΛ ΛΥΣΗΣ ------------------
\newcommand{\lysh}{\textcolor{\xrwma}{\textbf{\faCheck\ \ \onoma{ΛΥΣΗ}}}}
%------------------------------------

%------ ΛΥΜΕΝΑ ΠΑΡΑΔΕΙΓΜΑΤΑ ΤΙΤΛΟΣ ---------
\newcommand{\Lymena}{\begin{center}
\begin{tikzpicture}
\path[left color=red!70!black,right color=red!80!black,middle color=red!80!white] (-7cm,-.6cm) rectangle (6.5cm,.6cm);
\node at (-.25cm,0) { \textcolor{white}{\onomaaskpar ΛΥΜΕΝΑ ΠΑΡΑΔΕΙΓΜΑΤΑ}};  
\end{tikzpicture}
\end{center}}
%--------------------------------------

%-------- ΠΑΡΑΤΗΡΗΣΕΙΣ -----------------
\newcommand{\Parathrhsh}[1]{\textbf{\faLightbulb\ \ -\ \ Παρατήρηση}}

%----------- ΠΑΡΑΤΗΡΗΣΗ------------------
\newenvironment{parat}[1]
{\begin{tcolorbox}[title=\Parathrhsh,
breakable,
enhanced standard,lifted shadow={1mm}{-2mm}{3mm}{0.3mm}%
{black!50!white},
colback=red!5!white,
boxrule=0.1pt,
colframe=red!80!black,
fonttitle=\bfseries,width=#1]}
{\end{tcolorbox}}
%-----------------------------------------

%--------- ΑΛΥΤΕΣ ΑΣΚΗΣΕΙΣ ΤΙΤΛΟΣ ----------
\newcommand{\Alyta}{\begin{center}
\begin{tikzpicture}
\path[left color=red!70!black,right color=red!80!black,middle color=red!80!white] (-7cm,-.6cm) rectangle (6.5cm,.6cm);
\node at (-.25cm,0) {\textcolor{white}{\onomaaskpar ΑΣΚΗΣΕΙΣ - ΠΡΟΒΛΗΜΑΤΑ}};  
\end{tikzpicture}
\end{center}}
%--------------------------------------------
\usetikzlibrary{shadows,calc}
\usepackage{tcolorbox}
\tcbuselibrary{skins,theorems,breakable}
%---------- ΜΕΘΟΔΟΣ --------------
\newcounter{Methodos}[chapter]
\renewcommand{\theMethodos}{\onoma{\thechapter.\arabic{Methodos}}}
\newenvironment{Methodos}[2][\linewidth]
{\refstepcounter{Methodos}
\begin{tcolorbox}[breakable,
enhanced standard,
boxrule=0.7pt,titlerule=-.2pt,drop fuzzy shadow southeast=black!50,
width=\linewidth,
title style={color=white},
overlay unbroken and first={
\path[left color=red!70!black,right color=red,draw=black]
([yshift=-\pgflinewidth+0.2pt,xshift=.2pt]frame.north west) to ([yshift=-5pt,xshift=-.5pt]title.south west)[rounded corners=2pt] -- ([xshift=-#2-15pt,yshift=-5pt]title.south east) to[rounded corners=2pt] ([xshift=-#2,yshift=-\pgflinewidth+0.2pt]frame.north east) -- cycle;
},
fonttitle=\bfseries,
before=\par\medskip\noindent,
after=\par\medskip,
toptitle=3pt,
top=11pt,topsep at break=-5pt,
colback=white,title={\large \onoma{Μέθοδος \theMethodos}} : {\textcolor{black}{\MakeUppercase{\onoma#1}}}]}
{\end{tcolorbox}}
%------------------------------------------
%---------- ΛΙΣΤΕΣ ----------------------
\newlist{bhma}{enumerate}{3}
\setlist[bhma]{label=\bf\textit{\arabic*\textsuperscript{o}\;Βήμα :},leftmargin=0cm,itemindent=1.5cm,ref=\bf{\arabic*\textsuperscript{o}\;Βήμα}}
\newlist{rlist}{enumerate}{3}
\setlist[rlist]{itemsep=0mm,label=\roman*.}
\newlist{alist}{enumerate}{3}
\setlist[alist]{itemsep=0mm,label=\alph*.}
\makeatletter
\renewrobustcmd{\anw@true}{\let\ifanw@\iffalse}
\renewrobustcmd{\anw@false}{\let\ifanw@\iffalse}\anw@false
\newrobustcmd{\noanw@true}{\let\ifnoanw@\iffalse}
\newrobustcmd{\noanw@false}{\let\ifnoanw@\iffalse}\noanw@false
\renewrobustcmd{\anw@print}{\ifanw@\ifnoanw@\else\numer@lsign\fi\fi}
\makeatother

%----ΣΤΥΛ ΑΣΚΗΣΗΣ ----------
\newcounter{askhsh}[chapter]
\renewcommand{\theaskhsh}{\bf{\textit{{\Large{\thechapter}}.\arabic{askhsh}}}}   
\newcommand{\Askhsh}{\refstepcounter{askhsh}\textcolor{\xrwma}{{\theaskhsh}\hspace{1mm}}}{}
%---------------------------

\newlist{brlist}{enumerate}{3}
\setlist[brlist]{itemsep=0mm,label=\bf\roman*.}
\newlist{tropos}{enumerate}{3}
\setlist[tropos]{label=\bf\textit{\arabic*\textsuperscript{oς}\;Τρόπος :},leftmargin=0cm,itemindent=2.3cm,ref=\bf{\arabic*\textsuperscript{oς}\;Τρόπος}}
% Αν μπει το bhma μεσα σε tropo τότε
%\begin{bhma}[leftmargin=.7cm]
\newcommand{\tss}[1]{\textsuperscript{#1}}
\newcommand{\tssL}[1]{\MakeLowercase{\textsuperscript{#1}}}
%------------------------------------------
\setlength{\parindent}{0pt}
\setlist[itemize]{itemsep=0mm}
\tkzSetUpPoint[size=3,fill=white]
\newcommand{\twocolkentro}[1]{
\twocolumn[
\begin{@twocolumnfalse}
#1
\end{@twocolumnfalse}]}
\newcommand{\bcc}[1]{
\begin{center}
{\color{\xrwma}{\rule{1cm}{.4pt}}\raisebox{-2.5mm}{\rule{.4pt}{5mm}}}\hspace{1em}\raisebox{-.65ex}{\begin{varwidth}{\dimexpr0.7\textwidth-2em\relax}\centering{\textbf{\textcolor{black}{\large\scfont\textsc{#1}}}}\end{varwidth}}\hspace*{1em}{\color{\xrwma}{\raisebox{-2.5mm}{\rule{.4pt}{5mm}}\hrulefill}}
\end{center}}

\newcommand{\full}[1]{\begin{mdframed}[outermargin=\dimexpr-\marginparwidth-\marginparsep\relax,innerleftmargin=0mm,innerrightmargin=0mm,hidealllines=true]
#1
\end{mdframed}}

%\newcommand{\fulltc}[1]{\begin{tcolorbox}[enhanced,breakable, colback=white, colframe=white, check odd page, toggle left and right, grow to right by=\marginparwidth+\marginparsep, toggle enlargement=evenpage]
%#1
%\end{tcolorbox}}

\newcommand{\fulltwoc}[1]{\begin{adjmulticols}{2}{0cm}{0cm}
#1
\end{adjmulticols}}


\DeclareMathSizes{10.95}{10.95}{7}{5}
\DeclareMathSizes{6}{6}{3.8}{2.7}
\DeclareMathSizes{8}{8}{5.1}{3.6}
\DeclareMathSizes{9}{9}{5.8}{4.1}
\DeclareMathSizes{10}{10}{6.4}{4.5}
\DeclareMathSizes{12}{12}{7.7}{5.5}
\DeclareMathSizes{14.4}{14.4}{9.2}{6.5}
\DeclareMathSizes{17.28}{17.28}{11}{7.9}
\DeclareMathSizes{20.74}{20.74}{13.3}{9.4}
\DeclareMathSizes{24.88}{24.88}{16}{11.3}

\makeatletter
\AtBeginDocument{
\check@mathfonts
\fontdimen16\textfont2=2.5pt
\fontdimen17\textfont2=2.5pt
\fontdimen14\textfont2=4.5pt
\fontdimen13\textfont2=4.5pt 
}
\makeatother

%----------- ΟΡΙΣΜΟΣ------------------
\newenvironment{orismos}[1]
{\begin{tcolorbox}[title=\Orismos {\textcolor{black}{\onoma#1}},breakable,bottomtitle=-1.5mm,
enhanced standard,titlerule=-.2pt,toprule=0pt, rightrule=0pt, bottomrule=0pt,
colback=white,opacityfill=0,left=2mm,top=1mm,bottom=0mm,
boxrule=0pt,
colframe=white,borderline west={1.5mm}{0pt}{\xrwma},leftrule=2mm,sharp corners,coltitle=\xrwma]}
{\end{tcolorbox}}
%-----------------------------------------
%----------- ΘΕΩΡΗΜΑ------------------
\newenvironment{thewrhma}[1]
{\begin{tcolorbox}[title=\Thewrhma
{\textcolor{black}{\onoma#1}},breakable,
enhanced standard,titlerule=-.2pt,toprule=0pt, rightrule=0pt, bottomrule=0pt,
colback=white,left=2mm,top=1mm,bottom=0mm,
boxrule=0pt,
colframe=white,borderline west={1.5mm}{0pt}{orange!90!black},leftrule=2mm,sharp corners,coltitle=orange!90!black]}
{\end{tcolorbox}}
%-----------------------------------------
\newcommand{\vfilll}[1]{\mbox{}\\\vspace{#1mm}}




\begin{document}
\title{\MakeUppercase{ΣΧΟΛΙΚΟ ΤΥΠΟΛΟΓΙΟ}}
\pagestyle{empty}
\frontmatter
\begin{titlepage}
\newgeometry{left=2.5cm,top=2.5cm} %defines the geometry for the titlepage
\pagecolor{white}
\begin{center}
{\large Σπύρος Φρόνιμος\\Μαθηματικός}
\end{center}
\noindent
\par
\noindent
\mbox{}\\\\
\begin{center}
\textbf{\fontsize{20}{40}\selectfont{ΑΛΓΕΒΡΑ}}\par\mbox{}\\\vspace{-3mm}
\textbf{\fontsize{20}{40}\selectfont{Β' ΛΥΚΕΙΟΥ}}\par\mbox{}\\
\vspace{-4mm}
\rule{12cm}{0.1mm}\\
\vspace{3mm}
{\fontsize{15}{15}\MakeUppercase{τύποι ορισμοί θεωρήματα και}}\\
\vspace{.7mm}
{\fontsize{15}{15}\MakeUppercase{βασική μεθοδολογία για την}}\\
\vspace{.7mm}
{\fontsize{15}{15}\MakeUppercase{αλγεβρα της Β' ΛΥΚΕΙΟΥ}}\\
\rule{12cm}{0.1mm}\\
\end{center}
\vspace{3cm}
\begin{flushright}
\begin{itemize}
\item 100 Ορισμοί
\item 250 Θεωρήματα
\item 400 Μέθοδοι για λύση ασκήσεων
\item 200 Λυμένα παραδείγματα
\item 500 Άλυτες ασκήσεις και προβλήματα
\item 200 Επαναληπτικά θέματα
\item Απαντήσεις ασκήσεων
\end{itemize}
\end{flushright}

\vfill
\noindent
\color{black}
\begin{center}
{\large{ΕΚΔΟΣΕΙΣ \_\_\_\_\_}\\
\large{ΚΕΡΚΥΡΑ 2015}}
\vskip\baselineskip
\end{center}
\hbox{ % Horizontal box
\hspace*{0.2\textwidth} % Whitespace to the left of the title page
\rule{1pt}{\textheight} % Vertical line
\hspace*{0.05\textwidth} % Whitespace between the vertical line and title page text
\parbox[b]{0.75\textwidth}{ % Paragraph box which restricts text to less than the width of the page

{\textbf{Αλγεβρα}\\\textbf{Β΄ Λύκείου}\\\\\noindent \textbf{Σπύρος Φρόνιμος - Μαθηματικός}\\e-mail : spyrosfronimos@gmail.com\\[0.5\baselineskip]
Σελίδες : ...\\
ΙΣΒΝ : ...\\
Εκδόσεις : ...\\
\textcopyright Copyright 2015}\\[2\baselineskip] % Title
{Φιλολογική Επιμέλεια :\\\textbf{Μαρία Πρεντουλή}}
{- e-mail : predouli@yahoo.com}\\[0.5\baselineskip]
{Επιστημονική Επιμέλεια :}{\textbf{Σπύρος Φρόνιμος}}\\[0.5\baselineskip]
{Εξώφυλλο : \\\textbf{Δημήτρης Πρεντουλής}}\\[1\baselineskip] % Tagline or further description
% Author name

\vspace{.4\textheight} % Whitespace between the title block and the publisher
{Πνευματικά Δικαιώματα : ...}\\[\baselineskip]}}
\vspace*{2\baselineskip}
\newpage
\mbox{}\\\\\\\\\\\\
\hspace*{0.75\textwidth}
\textit{{\large Στη γυναίκα μου.}}
\newpage
\mbox{}
\newpage
\mbox{}
{\LARGE \textbf{Πρόλογος}}\\\\\\\\

Το βιβλίο περιέχει συγκεντρωμένη όλη τη θεωρία των μαθηματικών όλων των τάξεων του γυμνασίου και του λυκείου γραμμένη αναλυτικά και κατανοητά.\\\\
Ειδικότερα ο αναγνώστης θα βρει\\
\vspace{-.4cm}
\begin{itemize}
\item Ορισμούς
\item Θεωρήματα
\item Τυπολόγιο
\item Μεθοδολογία
\end{itemize}
Σκοπό έχει να αποτελέσει ένα χρήσιμο βοήθημα για μικρούς ή μεγάλους μαθητές όπου μπορούν να έχουν όλη τη θεωρία της χρονιάς τους συγκεντρωμένη, χρήσιμη για επανάληψη και διαγωνίσματα, αλλά και να μπορούν εύκολα να καλύψουν τυχόν κενά από προηγούμενες τάξεις.\\\\\\
Θέλω να ευχαριστήσω όλους όσους βοήθησαν.
\newpage
\end{titlepage}
\restoregeometry % restores the geometry
\tableofcontents
\mainmatter
\pagestyle{fancy}
\documentclass[twoside,10pt]{book}
\usepackage[amsbb,subscriptcorrection,zswash,mtpcal,mtphrb]{mtpro2}
\usepackage[no-math,cm-default]{fontspec}
\usepackage{xunicode}
\usepackage{xgreek}
\defaultfontfeatures{Mapping=tex-text,Scale=MatchLowercase}
\setmainfont[Mapping=tex-text,Numbers=Lining,Scale=1.0,BoldFont={Minion Pro Bold}]{Minion Pro}
\defaultfontfeatures{Ligatures=TeX}
\font\kefalaio="Minion Pro Bold" at 36pt
\font\ArKef="Minion Pro Bold Italic" at 72pt
\font\OnKef="Times New Roman" at 20pt
\font\OnPar="Minion Pro Bold" at 18pt
\newfontfamily\scfont{Times New Roman}
\usepackage[inner=2.00cm, outer=1.50cm, top=3.00cm, bottom=2.00cm,paperwidth=17cm,paperheight=24cm]{geometry}
\usepackage{amsmath}
\usepackage[amsbb,subscriptcorrection,zswash,mtpcal,mtphrb]{mtpro2}
\usepackage{makeidx}
\usepackage{longtable,xcolor,varwidth}
\usepackage{float}
\usepackage{subfig}
\def\xrwma{cyan!70!black}
\def\xrwmath{cyan}
\usepackage{etoolbox}
\makeatletter
\newif\ifLT@nocaption
\preto\longtable{\LT@nocaptiontrue}
\appto\endlongtable{%
\ifLT@nocaption
\addtocounter{table}{\m@ne}%
\fi}
\preto\LT@caption{%
\noalign{\global\LT@nocaptionfalse}}
\makeatother
\makeindex
\usepackage{tikz,pgfplots}
\usepackage{tkz-euclide,tkz-fct}
\usetikzlibrary{fadings}
\usepackage{wrap-rl}
\usetkzobj{all}
\usepackage{calc}
\usepackage[colorlinks=false, pdfborder={0 0 0}]{hyperref}
\usepackage{cleveref}
\usepackage[framemethod=TikZ]{mdframed}
\definecolor{steelblue}{cmyk}{.7,.278,0,.294}
\definecolor{doc}{cmyk}{1,0.455,0,0.569}
\definecolor{olivedrab}{cmyk}{0.25,0,0.75,0.44}
\usepackage{capt-of}
\usepackage{titletoc}
\usepackage[explicit]{titlesec}
\usepackage{graphicx}
\usepackage{multicol}
\usepackage{multirow}
\usepackage{enumitem}
\usepackage{tabularx}
\usepackage[decimalsymbol=comma]{siunitx}
\tikzset{>=latex}
\makeatletter
\pretocmd{\@part}{\gdef\parttitle{#1}}{}{}
\pretocmd{\@spart}{\gdef\parttitle{#1}}{}{}
\makeatother
\usepackage[titletoc]{appendix}
\usepackage{fancyhdr}
\pagestyle{fancy}
\fancyheadoffset{0cm}
\renewcommand{\headrulewidth}{\iftopfloat{0pt}{.5pt}}
\renewcommand{\chaptermark}[1]{\markboth{#1}{}}
\renewcommand{\sectionmark}[1]{\markright{\it\thesection\ #1}}
\fancyhf{}
\fancyhead[LE]{\thepage\ $\cdot$\ \scfont\scshape\nouppercase{\leftmark}}
\fancyhead[RO]{\nouppercase{\rightmark} $\cdot$\ \thepage}
\fancypagestyle{plain}{%
\fancyhead{} %
\renewcommand{\headrulewidth}{0pt}}

\newcounter{thewrhma}[chapter]
\renewcommand{\thethewrhma}{\thechapter.\arabic{thewrhma}} 
\newcommand{\Thewrhma}[1]{\refstepcounter{thewrhma}{\textbf{\textcolor{\xrwmath}{{\large Θεώρημα\hspace{2mm}\thethewrhma\;}:\;}\hspace{1mm}}} \MakeUppercase{\textbf{#1}}\\}{}

\newcounter{porisma}[chapter]
\renewcommand{\theporisma}{\thechapter.\arabic{porisma}}\newcommand{\Porisma}[1]{\refstepcounter{porisma}\textcolor{black}{\textbf{ΠΟΡΙΣΜΑ\hspace{2mm}\theporisma\hspace{1mm} \MakeUppercase{#1}}}\\}{}

\newcounter{protasi}[chapter]
\renewcommand{\theprotasi}{\thechapter.\arabic{protasi}}\newcommand{\Protasi}[1]{\refstepcounter{protasi}\textcolor{black}{\textbf{ΠΡΟΤΑΣΗ\hspace{2mm}\theprotasi\hspace{1mm} \MakeUppercase{#1}}}\\}{}


\newcounter{orismos}[chapter]
\renewcommand{\theorismos}{\arabic{orismos}}   
\newcommand{\Orismos}[1]{\refstepcounter{orismos}{\textbf{\textbf{\textcolor{\xrwma}{{\large Ορισμός\hspace{2mm}\theorismos\;}:\;}}}}\hspace{1mm} \MakeUppercase{\textbf{#1}\\}}{}
\usepackage{venndiagram,mathimatika,eurosym}
%-------- ΣΤΥΛ ΚΕΦΑΛΑΙΟΥ ---------
\newcommand*\chapterlabel{}
\newcommand{\fancychapter}{%
\titleformat{\chapter}
{
\normalfont\Huge}
{\gdef\chapterlabel{\thechapter\ }}{0pt}
{\begin{tikzpicture}[remember picture,overlay]
\node[yshift=-7cm] at (current page.north west)
{\begin{tikzpicture}[remember picture, overlay]
%\node[inner sep=0pt] at ($(current page.north) +			(0cm,-1.38in)$) {\includegraphics[width=17cm]{Kefalaio}};
\node[anchor=west,xshift=.1\paperwidth,yshift=.14\paperheight,rectangle]
{{\color{white}\fontsize{30}{20}\textbf{\textcolor{black}{\contour{white}{ΚΕΦΑΛΑΙΟ}}}}};
\node[anchor=west,xshift=.09\paperwidth,yshift=.08\paperheight,rectangle] {\fontsize{24}{20} {\color{black}{{\textcolor{black}{\contour{white}{\sc##1}}}}}};
%\fill[fill=black] (12.2,2) rectangle (14.8,4.7);
\node[anchor=west,xshift=.74\paperwidth,yshift=.11\paperheight,rectangle]
{{\color{white}\fontsize{80}{20}\textbf{\textit{\textcolor{white}{\contour{black}{\thechapter}}}}}};
\end{tikzpicture}
};
\end{tikzpicture}
}
\titlespacing*{\chapter}{0pt}{20pt}{30pt}
}
%------------------------------------------------


\usepackage[outline]{contour}
\newcommand{\regularchapter}{%
\titleformat{\chapter}[display]
{\normalfont\huge\bfseries}{\chaptertitlename\ \thechapter}{20pt}{\Huge##1}
\titlespacing*{\chapter}
{0pt}{-20pt}{40pt}
}

\apptocmd{\mainmatter}{\fancychapter}{}{}
\apptocmd{\backmatter}{\regularchapter}{}{}
\apptocmd{\frontmatter}{\regularchapter}{}{}

\titlespacing*{\section}
{0pt}{30pt}{0pt}
\usepackage{booktabs}
\usepackage{hhline}
\DeclareRobustCommand{\perthousand}{%
\ifmmode
\text{\textperthousand}%
\else
\textperthousand
\fi}


\contentsmargin{0cm}
\titlecontents{part}[-1pc]
{\addvspace{10pt}%
\bf\Large ΜΕΡΟΣ\quad }%
{}
{}
{\;\dotfill}%
%------------------------------------------
\titlecontents{chapter}[0pc]
{\addvspace{30pt}%
\begin{tikzpicture}[remember picture, overlay]%
\draw[fill=black,draw=black] (-.3,.5) rectangle (3.7,1.1); %
\pgftext[left,x=0cm,y=0.75cm]{\color{white}\sc\Large\bfseries Κεφάλαιο\ \thecontentslabel};%
\end{tikzpicture}\large\sc}%
{}
{}
{\hspace*{-2.3em}\hfill\normalsize Σελίδα \thecontentspage}%
\titlecontents{section}[2.4pc]
{\addvspace{1pt}}
{\contentslabel[\thecontentslabel]{2pc}}
{}
{\;\dotfill\;\small \thecontentspage}
[]
\titlecontents*{subsection}[4pc]
{\addvspace{-1pt}\small}
{}
{}
{\ --- \small\thecontentspage}
[ \textbullet\ ][]

\makeatletter
\renewcommand{\tableofcontents}{%
\chapter*{%
\vspace*{-20\p@}%
\begin{tikzpicture}[remember picture, overlay]%
\pgftext[right,x=12cm,y=0.2cm]{\Huge\sc\bfseries \contentsname};%
\draw[fill=black,draw=black] (9.5,-.75) rectangle (12.5,1);%
\clip (9.5,-.75) rectangle (15,1);
\pgftext[right,x=12cm,y=0.2cm]{\color{white}\Huge\bfseries \contentsname};%
\end{tikzpicture}}%
\@starttoc{toc}}
\makeatother

\usepackage[contents={},scale=1,opacity=1,color=black,angle=0]{background}

\newcommand\blfootnote[1]{%
\begingroup
\renewcommand\thefootnote{}\footnote{#1}%
\addtocounter{footnote}{-1}%
\endgroup
}
\usepackage{epstopdf}
\epstopdfsetup{update}
\usepackage{textcomp}

\titleformat{\section}
{\normalfont\Large\bf}%
{}{0em}%
{{\color{black}\titlerule[0pt]}\vskip-.2\baselineskip{\parbox[t]{\dimexpr\textwidth-2\fboxsep\relax}{\raggedright\strut\itshape{\LARGE{\thesection~#1}}\strut}}}[\vskip 0\baselineskip{\color{black}\titlerule[1pt]}]
\titlespacing*{\section}{0pt}{0pt}{30pt}

\newcommand{\methodologia}{\begin{center}
{\large \textbf{ΜΕΘΟΔΟΛΟΓΙΑ}}\\\vspace{-2mm}
\begin{tikzpicture}
\shade[left color=white, right color=black,] (-3cm,0) rectangle (0,.2mm);
\shade[left color=black, right color=white,] (0,0) rectangle (3cm,.2mm);   
\end{tikzpicture}
\end{center}}

\newcommand{\orismoi}{\begin{center}
\vspace{-3mm}{\large \textbf{\textcolor{\xrwma}{ΟΡΙΣΜΟΙ}}}\\\vspace{-2mm}
\begin{tikzpicture}
\shade[left color=white, right color=cyan!80!black,] (-3cm,0) rectangle (0,.2mm);
\shade[left color=cyan!80!black, right color=white,] (0,0) rectangle (3cm,.2mm);   
\end{tikzpicture}
\end{center}}
\newcommand{\thewrhmata}{\begin{center}
{\large \textbf{\textcolor{\xrwmath}{ΘΕΩΡΗΜΑΤΑ - ΠΟΡΙΣΜΑΤΑ - ΠΡΟΤΑΣΕΙΣ\\ΚΡΙΤΗΡΙΑ - ΙΔΙΟΤΗΤΕΣ}}}\\\vspace{-2mm}
\begin{tikzpicture}
\shade[left color=white, right color=\xrwmath,] (-5cm,0) rectangle (0,.2mm);
\shade[left color=\xrwmath, right color=white,] (0,0) rectangle (5cm,.2mm);   
\end{tikzpicture}
\end{center}}
\usepackage[labelfont={footnotesize,it,bf},font={footnotesize}]{caption}

%-------- ΠΙΝΑΚΕΣ ---------
\usepackage{booktabs}
%----------------------
%----- ΥΠΟΛΟΓΙΣΤΗΣ ----------
%\usepackage{calculator}
%----------------------------

%----- ΟΡΙΖΟΝΤΙΑ ΛΙΣΤΑ ------
\usepackage{xparse}
\newcounter{answers}
\renewcommand\theanswers{\arabic{answers}}
\ExplSyntaxOn
\NewDocumentCommand{\results}{m}
{
\seq_set_split:Nnn \l_results_a_seq {,}{#1}
\par\nobreak\noindent\setcounter{answers}{0}
\seq_map_inline:Nn \l_results_a_seq
{
\makebox[.18\linewidth][l]{\stepcounter{answers}\theanswers.~##1}\hfill
}
\par
}
\seq_new:N \l_results_a_seq
\ExplSyntaxOff
%----------------------------
%------ ΜΗΚΟΣ ΓΡΑΜΜΗΣ ΚΛΑΣΜΑΤΟΣ ---------
\DeclareRobustCommand{\frac}[3][0pt]{%
{\begingroup\hspace{#1}#2\hspace{#1}\endgroup\over\hspace{#1}#3\hspace{#1}}}
%----------------------------------------
\usepackage{microtype}
\usepackage{float}

\usepackage{caption}

%---- ΟΡΙΖΟΝΤΙΟ - ΚΑΤΑΚΟΡΥΦΟ - ΠΛΑΓΙΟ ΑΓΚΙΣΤΡΟ ------
\newcommand{\orag}[3]{\node at (#1)
{$ \overcbrace{\rule{#2mm}{0mm}}^{{\scriptsize #3}} $};}

\newcommand{\kag}[3]{\node at (#1)
{$ \undercbrace{\rule{#2mm}{0mm}}_{{\scriptsize #3}} $};}

\newcommand{\Pag}[4]{\node[rotate=#1] at (#2)
{$ \overcbrace{\rule{#3mm}{0mm}}^{{\rotatebox{-#1}{\scriptsize$#4$}}}$};}
%-----------------------------------------
\tikzstyle{pl}=[line width=0.3mm]
\tikzstyle{plm}=[line width=0.4mm]
%------- ΣΤΥΛ ΠΑΡΑΔΕΙΓΜΑΤΟΣ -------
\newcounter{paradeigma}[section]
\renewcommand{\theparadeigma}{\bf\thechapter.\arabic{paradeigma}}   
\newcommand{\Paradeigma}[1]{\refstepcounter{paradeigma}\textcolor{cyan}{\textbf{{\large Παράδειγμα\hspace{2mm}\theparadeigma\;:\;}\hspace{1mm}}} \MakeUppercase{\textbf{#1}}\\}{}
%-----------------------------------

%------- ΣΤΥΛ ΛΥΣΗΣ ------------------
\newcommand{\lysh}{{\textbf{ΛΥΣΗ}}}
%------------------------------------

%------ ΛΥΜΕΝΑ ΠΑΡΑΔΕΙΓΜΑΤΑ ΤΙΤΛΟΣ ---------
\newcommand{\Lymena}{\begin{center}
\begin{tikzpicture}
\path[left color=cyan!70!black,right color=cyan!80!black,middle color=cyan!80!white] (-7cm,-.6cm) rectangle (6.5cm,.6cm);
\node at (-.25cm,0) {\Large \textcolor{white}{\textbf{ΛΥΜΕΝΑ ΠΑΡΑΔΕΙΓΜΑΤΑ}}};  
\end{tikzpicture}
\end{center}}
%--------------------------------------

%--------- ΑΛΥΤΕΣ ΑΣΚΗΣΕΙΣ ΤΙΤΛΟΣ ----------
\newcommand{\Alyta}{\begin{center}
\begin{tikzpicture}
\path[left color=cyan!70!black,right color=cyan!80!black,middle color=cyan!80!white] (-7cm,-.6cm) rectangle (6.5cm,.6cm);
\node at (-.25cm,0) {\Large \textcolor{white}{\textbf{ΑΣΚΗΣΕΙΣ - ΠΡΟΒΛΗΜΑΤΑ}}};  
\end{tikzpicture}
\end{center}}
%--------------------------------------------
\usetikzlibrary{shadows,calc}
\usepackage{tcolorbox}
\tcbuselibrary{skins,theorems,breakable}
%---------- ΜΕΘΟΔΟΣ --------------
\newcounter{Methodos}[chapter]
\renewcommand{\theMethodos}{\thechapter.\arabic{Methodos}}
\newenvironment{Methodos}[2][\linewidth]
{\refstepcounter{Methodos}
\begin{tcolorbox}[breakable,
enhanced standard,
boxrule=0.7pt,titlerule=-.2pt,drop fuzzy shadow southeast=black!50,
width=\linewidth,
title style={color=white},
overlay unbroken and first={
\path[left color=cyan!70!black,right color=cyan,draw=black]
([yshift=-\pgflinewidth]frame.north west) to ([yshift=-5pt]title.south west)[rounded corners=2pt] -- ([xshift=-#2-15pt,yshift=-5pt]title.south east) to[rounded corners=2pt] ([xshift=-#2,yshift=-\pgflinewidth]frame.north east) -- cycle;
},
fonttitle=\bfseries,
before=\par\medskip\noindent,
after=\par\medskip,
toptitle=3pt,
top=11pt,topsep at break=-5pt,
colback=white,title={\large Μέθοδος \theMethodos} : {\textcolor{black}{\MakeUppercase{#1}}}]}
{\end{tcolorbox}}
%------------------------------------------
%---------- ΛΙΣΤΕΣ ----------------------
\newlist{bhma}{enumerate}{3}
\setlist[bhma]{label=\bf\textit{\arabic*\textsuperscript{o}\;Βήμα :},leftmargin=0cm,itemindent=1.5cm,ref=\bf{\arabic*\textsuperscript{o}\;Βήμα}}
\newlist{rlist}{enumerate}{3}
\setlist[rlist]{itemsep=0mm,label=\roman*.}
\newlist{alist}{enumerate}{3}
\setlist[alist]{itemsep=0mm,label=\alph*.}
\makeatletter
\renewrobustcmd{\anw@true}{\let\ifanw@\iffalse}
\renewrobustcmd{\anw@false}{\let\ifanw@\iffalse}\anw@false
\newrobustcmd{\noanw@true}{\let\ifnoanw@\iffalse}
\newrobustcmd{\noanw@false}{\let\ifnoanw@\iffalse}\noanw@false
\renewrobustcmd{\anw@print}{\ifanw@\ifnoanw@\else\numer@lsign\fi\fi}
\makeatother

%----ΣΤΥΛ ΑΣΚΗΣΗΣ ----------
\newcounter{askhsh}[chapter]
\renewcommand{\theaskhsh}{\bf{\textit{{\large{\thechapter}}.\arabic{askhsh}}}}   
\newcommand{\Askhsh}{\refstepcounter{askhsh}\textcolor{\xrwma}{{\theaskhsh}\hspace{1mm}}}{}
%---------------------------

\newlist{brlist}{enumerate}{3}
\setlist[brlist]{itemsep=0mm,label=\bf\roman*.}
\newlist{tropos}{enumerate}{3}
\setlist[tropos]{label=\bf\textit{\arabic*\textsuperscript{oς}\;Τρόπος :},leftmargin=0cm,itemindent=2.3cm,ref=\bf{\arabic*\textsuperscript{oς}\;Τρόπος}}
% Αν μπει το bhma μεσα σε tropo τότε
%\begin{bhma}[leftmargin=.7cm]
\newcommand{\tss}[1]{\textsuperscript{#1}}
\newcommand{\tssL}[1]{\MakeLowercase{\textsuperscript{#1}}}
%------------------------------------------
\setlength{\parindent}{0pt}
\setlist[itemize]{itemsep=0mm}
\tkzSetUpPoint[size=7,fill=white]
\newcommand{\twocolkentro}[1]{
\twocolumn[
\begin{@twocolumnfalse}
#1
\end{@twocolumnfalse}]}
\newcommand{\bcc}[1]{
\begin{center}
{\color{\xrwma}{\rule{1cm}{.4pt}}\raisebox{-2.5mm}{\rule{.4pt}{5mm}}}\hspace{1em}\raisebox{-.65ex}{\begin{varwidth}{\dimexpr0.7\textwidth-2em\relax}\centering{\textbf{\textcolor{\xrwma}{\large\scfont\textsc{#1}}}}\end{varwidth}}\hspace*{1em}{\color{\xrwma}{\raisebox{-2.5mm}{\rule{.4pt}{5mm}}\hrulefill}}
\end{center}}

\DeclareMathSizes{10.95}{10.95}{7}{5}
\DeclareMathSizes{6}{6}{3.8}{2.7}
\DeclareMathSizes{8}{8}{5.1}{3.6}
\DeclareMathSizes{9}{9}{5.8}{4.1}
\DeclareMathSizes{10}{10}{6.4}{4.5}
\DeclareMathSizes{12}{12}{7.7}{5.5}
\DeclareMathSizes{14.4}{14.4}{9.2}{6.5}
\DeclareMathSizes{17.28}{17.28}{11}{7.9}
\DeclareMathSizes{20.74}{20.74}{13.3}{9.4}
\DeclareMathSizes{24.88}{24.88}{16}{11.3}

\makeatletter
\AtBeginDocument{
\check@mathfonts
\fontdimen16\textfont2=2.5pt
\fontdimen17\textfont2=2.5pt
\fontdimen14\textfont2=4.5pt
\fontdimen13\textfont2=4.5pt 
}
\makeatother





\begin{document}
\mainmatter
\pagestyle{fancy}
\chapter{Συστήματα} 
\section{Γραμμικά Συστήματα}
\orismoi
\Orismos{Γραμμική Εξίσωση}
Γραμμική εξίσωση δύο μεταβλητών, ονομάζεται κάθε πολυωνυμική εξίσωση της μορφής \[ ax+\beta y=\gamma \]
\begin{itemize}[itemsep=0mm]
\item Οι μεταβλητές της εξίσωσης είναι οι $ x,y $.
\item Οι πραγματικοί αριθμοί $ a,\beta $ λέγονται \textbf{συντελεστές} των μεταβλητών.
\item Ο πραγματικός αριθμός $ \gamma $ ονομάζεται \textbf{σταθερός όρος} της εξίσωσης.
\end{itemize}
\Orismos{Λύση γραμμικήσ εξίσωσησ}
Λύση μιας γραμμικής εξίσωσης της μορφής \[ ax+\beta y=\gamma \] ονομάζεται κάθε διατεταγμένο ζεύγος αριθμών $ \left( x_0,y_0\right)  $ το οποίο επαληθεύει την εξίσωση.\\\\
\Orismos{Γραμμικό σύστημα $ \mathbold{2\times2} $}
Γραμμικό σύστημα δύο εξισώσεων με δύο άγνωστους ονομάζεται ο συνδυασμός - σύζευξη δύο γραμμικών εξισώσεων. Είναι της μορφής :
\begin{equation}\label{systhma}
\ccases{{a}x+{\beta} y={\gamma}\\{a'}x+{\beta'} y={\gamma'} }
\end{equation}
\begin{itemize}[itemsep=0mm]
\item Οι αριθμοί $ a,a',\beta,\beta'\in\mathbb{R} $ ονομάζονται \textbf{συντελεστές} του συστήματος ενώ οι $ \gamma,\gamma'\in\mathbb{R} $ λέγονται \textbf{σταθεροί όροι}.
\item Κάθε διατεταγμένο ζεύγος αριθμών $ \left(x_0,y_0\right)  $ το οποίο επαληθεύει και τις δύο εξισώσεις ονομάζεται \textbf{λύση} του γραμμικού συστήματος.
\item Τα συστήματα τα οποία έχουν ακριβώς τις ίδιες λύσεις ονομάζονται \textbf{ισοδύναμα}.
\item Ένα σύστημα που έχει λύση λέγεται \textbf{συμβιβαστό}. Εάν δεν έχει λύση ονομάζεται \textbf{αδύνατο} ενώ αν έχει άπειρες λύσεις \textbf{αόριστο}.
\end{itemize}
\Orismos{Επαλήθευση Συστήματοσ}
Επαλήθευση ενός συστήματος εξισώσεων ονομάζεται η διαδικασία με την οποία εξετάζουμε εάν ένα ζεύγος αριθμών $ \left(x_0,y_0\right)  $ είναι λύση του, αντικαθιστώντας τους αριθμούς στη θέση των μεταβλητών.\\\\
\Orismos{Ορίζουσα Συστήματοσ {$ \mathbold{2\mathbold{\times}2} $}}
Θεωρούμε ένα γραμμικό σύστημα $ 2\times2 $ της μορφής \eqref{systhma}. Ορίζουσα των συντελεστών ονομάζεται ο αριθμός $ a\beta'-a'\beta $ και συμβολίζεται με $ D $. Μπορεί να γραφτεί στη μορφή πίνακα όπως φαίνεται παρακάτω:
\[ D=\left|\begin{array}{cc}
a & \beta \\ 
a' & \beta'
\end{array}  \right|  \]
Επίσης $ D_x,D_y $ είναι οι ορίζουσες των μεταβλητών που προκύπτουν αν αντικαταστήσουμε στην ορίζουσα $ D $ τη στήλη των συντελεστών των μεταβλητών $ x,y $ αντίστοιχα με τους σταθερούς όρους $ \gamma,\gamma' $. Θα είναι:
\[ D_x=\left|\begin{array}{cc}
\gamma & \beta \\ 
\gamma' & \beta'
\end{array}  \right|\quad,\quad D_y=\left|\begin{array}{cc}
a & \gamma \\ 
a' & \gamma'
\end{array}  \right| \]
\Orismos{Γραμμικό Σύστημα Εξισώσεων $ \mathbold{3\times3} $}
Γραμμικό σύστημα τριών εξισώσεων με τρεις άγνωστους ονομάζεται ένας συνδυασμός από τρεις γραμμικές εξισώσεις της μορφής
\[ \ccases{{a_1}x+{\beta_1} y+{\gamma_1} z=\delta_1\\{a_2}x+{\beta_2} y+{\gamma_2} z=\delta_2\\{a_3}x+{\beta_3} y+{\gamma_3} z=\delta_3} \]
με $ a_i,\beta_i,\gamma_i,\delta_i\in\mathbb{R}\;,\;i=1,2,3 $. Κάθε διατεταγμένη τριάδα αριθμών $ \left( x_0,y_0,z_0\right)  $ η οποία επαληθεύει και τις τρεις εξισώσεις ονομάζεται \textbf{λύση} του γραμμικού συστήματος $ 3\times3 $.\\\\
\Orismos{Παραμετρικό σύστημα}
Παραμετρικό ονομάζεται το γραμμικό σύστημα του οποίου οι συντελεστές ή και οι σταθεροί όροι δίνονται με τη βοήθεια μιας ή περισσότερων παραμέτρων. Η διαδικασία επίλυσης ενός παραμετρικού συστήματος ονομάζεται \textbf{διερεύνηση}.
\\\\
\thewrhmata
\Thewrhma{Γραφική παράσταση εξίσωσης}
Δίνεται η γραμμική εξίσωση με δύο μεταβλητές της μορφής $ ax+\beta y=\gamma $. Για τις διάφορες τιμές των συντελεστών $ a,\beta $ καθώς και του σταθερού όρου $ \gamma $ θα μελετήσουμε τη γραφική αναπαράσταση της εξίσωσης.\\
\wrapr{-4mm}{10}{4.5cm}{4mm}{
\begin{tikzpicture}[domain=-.2:4,y=1cm,scale=.8]
\draw[-latex] (-.5,0) -- coordinate (x axis mid) (4.4,0) node[right,fill=white] {{\footnotesize $ x $}};
\draw[-latex] (0,-.5) -- (0,3.5) node[above,fill=white] {{\footnotesize $ y $}};
\draw[domain=-.2:3.4,samples=100,line width=.4mm,\xrwma] plot function{-0.8*x+2.5};
\tkzText(2.5,2.7){$ ax+\beta y=\gamma $}
\tkzText(2.5,2.2){{\footnotesize $ a,\beta,\gamma\in\mathbb{R} $}}
\tkzText(2.5,1.7){{\footnotesize $ a\neq0 $ ή $ \beta\neq0 $}}
\tkzDefPoint(0,0){O}
\tkzLabelPoint[below left](O){$ O $}
\end{tikzpicture}}{
\begin{rlist}
\item Αν $ a\neq0 $ ή $ \beta\neq0 $ τότε η εξίσωση παριστάνει πλάγια ευθεία με συντελεστή διεύθυνσης $ \lambda=-\frac{a}{\beta} $.
\item Aν $ a=0 $ και $ \beta\neq0 $ τότε παίρνει τη μορφή $ y=\lambda $ και παριστάνει οριζόντια ευθεία με συντελεστή διεύθυνσης $ \lambda=0 $.
\item Αν $ a\neq0 $ και $ \beta=0 $ τότε η εξίσωση γράφεται στη μορφή $ x=\lambda $ και παριστάνει κατακόρυφη ευθεία. Οι ευθείες αυτές δεν έχουν συντελεστή διεύθυνσης.
\item Τέλος αν $ a=0 $ και $ \beta=0 $ τότε διακρίνουμε τις εξής περιπτώσεις:
\begin{alist}
\item  Αν $ \gamma=0 $ τότε η εξίσωση γίνεται $ 0x+0y=0 $, είναι αόριστη και αντιστοιχεί σε όλο το καρτεσιανό επίπεδο.
\item Αν $ \gamma\neq0 $ τότε η εξίσωση είναι $ 0x+0y=\gamma $, είναι αδύνατη και δεν αντιστοιχεί σε κανένα σημείο.
\end{alist}
\end{rlist}}
\Thewrhma{Σημείο σε ευθεία}
Ένα σημείο $ A(x_0,y_0) $ ανήκει σε μια ευθεία με εξίσωση $ ax+\beta y=\gamma $ αν και μόνο αν οι συντεταγμένες του επαληθεύουν την εξίσωση της.\\\\
\Thewrhma{Λύση συστήματοσ {$ \mathbold{2\mathbold{\times}2} $} με χρήση οριζουσων}
Έστω το γραμμικό σύστημα 
\[ \ccases{{a}x+{\beta} y={\gamma}\\{a'}x+{\beta'} y={\gamma'} } \]
με πραγματικούς συντελεστές και ορίζουσα συντελεστών $ D $.
\begin{rlist}
\item Αν η ορίζουσα των συντελεστών του συστήματος είναι διάφορη του μηδενός δηλαδή $ D\neq0 $ τότε το σύστημα έχει μοναδική λύση. Οι τιμές των μεταβλητών δίνονται από τις σχέσεις
\[ x=\frac{D_x}{D}\;\;,\;\;y=\frac{D_y}{D} \]
ενώ η λύση του συστήματος θα είναι $ (x,y)=\left(\frac{D_x}{D},\frac{D_y}{D} \right)  $.
\item Αν η ορίζουσα των συντελεστών του συστήματος είναι μηδενική δηλαδή $ D=0 $ τότε το σύστημα είναι είτε αόριστο είτε αδύνατο. Συγκεκριμένα
\begin{alist}
\item αν $ D_x\neq0 $ ή $ D_y\neq0 $ τότε το σύστημα είναι αδύνατο.
\item αν $ D_x=0 $ και $ D_y=0 $ τότε το σύστημα είναι αόριστο.
\end{alist}
\end{rlist}
\newpage
\noindent
\Lymena
\begin{Methodos}[Γραμμική εξίσωση - Ευθεία - Κοινά σημεία ]{1cm}
Δεδομένης μιας γραμμικής εξίσωσης μπορούμε να εξετάσουμε ποια σημεία ανήκουν στην ευθεία που παριστάνει καθώς και τα σημεία τομής της ευθείας αυτής με του άξονες.
\begin{itemize}[leftmargin=3mm]
\item \textbf{Σημεία ευθείας}\\
Για να εξετάσουμε αν κάποιο σημείο $ A(x_0,y_0) $ ανήκει στην ευθεία που ορίζει μια γραμμική εξίσωση, αντικαθιστούμε τις συντεταγμένες του $ A $ στη θέση των μεταβλητών της εξίσωσης και εξετάζουμε αν οι αριθμοί αυτοί την επαληθεύουν. Αν ναι τότε το σημείο ανήκει στην ευθεία.
\item \textbf{Σημεία τομής με τους άξονες}\\
Για να βρούμε τα σημεία τομής μιας ευθείας με τους άξονες τότε θέτουμε όπου $ x=0 $ ή $ y=0 $ στη γραμμική εξίσωση για τους άξονες $ y'y $ και $ x'x $ αντίστοιχα.
\end{itemize}
Αν αναζητούμε κοινά σημεία δύο ευθειών τότε λύνουμε το γραμμικό σύστημα που ορίζουν οι εξισώσεις των ευθειών. Την περίπτωση αυτή την εξετάζουμε στη \textbf{Μέθοδο \ref{meth:grafikh}}.
\end{Methodos}
\Paradeigma{Σημείο ευθείας}
\textbf{Να εξεταστεί αν τα σημεία {\boldmath$ A(2,3) $} και {\boldmath$ B(-1,4) $} ανήκουν στην ευθεία {\boldmath$ 2x+y=7 $}.}\\\\
\lysh\\
Αντικαθιστούμε τς συντεταγμένες του σημείου $ A(2,3) $ στην εξίσωση και έχουμε αναλυτικά :
\[ \textrm{Για }x=2\ \textrm{και }y=3\Rightarrow 2\cdot 2+3=7\Rightarrow 4+3=7\Rightarrow 7=7 \]
Η εξίσωση επαληθεύεται οπότε το σημείο $ A $ ανήκει στην ευθεία. Ομοίως για το σημείο $ B $ θα έχουμε :
\[ \textrm{Για }x=-1\ \textrm{και }y=4\Rightarrow 2\cdot (-1)+4=7\Rightarrow -2+4=7\Rightarrow 2=7 \]
Η εξίσωση δεν επαληθεύεται οπότε το σημείο $ B $ δεν ανήκει στην ευθεία.\\\\
\Paradeigma{Σημεία τομής με τους άξονες}
\textbf{Να βρεθούν τα σημεία τομής της ευθείας {\boldmath$ 3x+4y=6 $} με τους άξονες {\boldmath$ x'x,y'y $}.}\\\\
\lysh\\
Επιλέγουμε $ x=0 $ και $ y=0 $ αντίστοιχα και θα έχουμε :\\
\wrapr{-11mm}{7}{3.5cm}{-11mm}{\begin{tikzpicture}

\begin{axis}[belh ar,aks_on,xmin=-.2,xmax=3,ymin=-.2,
ymax=2.5,xlabel={\footnotesize$x$},ylabel={\footnotesize$y$},x=1cm,y=1cm]
\addplot[domain=-.3:2.5,grafikh parastash]{-3*x/4+3/2};
\coordinate (B) at (axis cs:2,0);
\coordinate (A) at (axis cs:0,1.5);
\end{axis}
\tkzDrawPoints(A,B)
\tkzLabelPoint[above right](A){\footnotesize$A\left(0,\frac{3}{2}\right)$}
\tkzLabelPoint[above right](B){\footnotesize$B(2,0)$}
\end{tikzpicture}}{
\begin{itemize}[leftmargin=4mm]
\item Για το σημείο του άξονα $ y'y $ έχουμε $ x=0 $ άρα $ 3\cdot0+4y=6\Rightarrow y=\frac{3}{2} $. Το σημείο θα είναι $ A\left( 0,\frac{3}{2}\right)  $.
\item Για το σημείο του άξονα $ x'x $ έχουμε $ y=0 $ άρα $ 3x+4\cdot0=6\Rightarrow x=2 $. Το σημείο θα είναι $ B(2,0) $.
\end{itemize}}
\begin{Methodos}[Μέθοδος της αντικατάστασης]{3cm}
Για την επίλυση ενός συστήματος με δύο μεταβλητές έστω $ x,y $ με τη μέθοδο της αντικατάστασης ακολουθούμε τα παρακάτω βήματα.
\begin{bhma}
\item \textbf{Επιλογή εξίσωσης}\\
Επιλέγουμε μια απ' τις δύο εξισώσεις ώστε να λύσουμε ως προς οποιαδήποτε μεταβλητή. Θα προκύψει μια σχέση (1) που θα μας δίνει την μεταβλητή αυτή ως συνάρτηση της άλλης. 
\item \textbf{Αντικατάσταση}\\
Τη μεταβλητή αυτή την αντικαθιστούμε στην άλλη εξίσωση του συστήματος οπότε προκύπτει μια εξίσωση με έναν άγνωστο. Λύνοντας την εξίσωση υπολογίζουμε τον άγνωστο αυτό.
\item \textbf{Υπολογισμός 2\tss{ου} αγνώστου}\\
Την τιμή που θα βρούμε για τη μια μεταβλητή λύνοντας την εξίσωση, την αντικαθιστούμε στη σχέση (1) ώστε να βρεθεί και η άλλη μεταβλητή του συστήματος.
\item \textbf{Λύση συστήματος}\\
Όταν βρεθούν οι τιμές $ x_0,y_0 $ και των δύο αγνώστων, σχηματίζουμε το διατεταγμένο ζεύγος $ (x,y)=(x_0,y_0) $ το οποίο είναι η λύση του συστήματος.
\end{bhma}
\end{Methodos}
\Paradeigma{Λύση συστήματος με αντικατάσταση}
\textbf{Να λυθεί το παρακάτω σύστημα με τη μέθοδο της αντικατάστασης}
{\boldmath\[ \systeme{2x+3y=5,x-4y=-3} \]}
\lysh\\
Παρατηρούμε οτι η 2\tss{η} εξίσωση είναι εύκολο να λυθεί ως προς $ x $ οπότε έχουμε
\begin{equation}
\systeme{2x+3y=5,x-4y=-3}\Rightarrow x=4y-3
\end{equation}
Αντικαθιστώντας το αποτέλεσμα της σχέσης (1) στην 1\tss{η} εξίσωση προκύπτει :
\begin{equation}\begin{aligned}
2x+3y=5\Rightarrow 2(4y-3)+3y=5&\Rightarrow 8y-6+3y=5\\&\Rightarrow 8y+3y=5+6\Rightarrow 11y=11\Rightarrow y=1\end{aligned}
\end{equation}
Τη λύση της εξίσωσης (2) την αντικαθιστούμε στην (1) για να υπολογίσουμε τον άγνωστο $ x $
\[ x=4y-3=4\cdot1-3=4-3=1 \]
Επομένως η λύση του συστήματος θα είναι η $ (x,y)=(1,1) $.
\begin{Methodos}[Μέθοδος των αντίθετων συντελεστών]{2cm}
Για την επίλυση ενός συστήματος με τη μέθοδο των αντίθετων συντελεστών
\begin{bhma}
\item \textbf{Επιλογή μεταβλητής}\\
Επιλέγουμε ποια από τις δύο μεταβλητές θα απαλείψουμε χρησιμοποιώντας τη μέθοδο αυτή.
\item \textbf{Πολλαπλασιασμός εξισώσεων}\\
Τοποθετούμε δίπλα από κάθε εξίσωση τους συντελεστές την μεταβλητής που επιλέξαμε \textquotedblleft χιαστί\textquotedblright\; αλλάζοντας το πρόσημο του ενός από τους δύο. Πολλαπλασιάζουμε την κάθε εξίσωση με τον αριθμό που προκύπτει.
\item \textbf{Πρόσθεση κατά μέλη}\\
Προσθέτουμε κατά μέλη τις νέες εξισώσεις οπότε προκύπτει μια εξίσωση με έναν άγνωστο τον οποίο και υπολογίζουμε λύνοντας την.
\item \textbf{Εύρεση 2\tss{ης} μεταβλητής}\\
Αντικαθιστούμε το αποτέλεσμα σε οποιαδήποτε εξίσωση του αρχικού συστήματος ώστε να υπολογίσουμε και τη δεύτερη μεταβλητή.
\item \textbf{Λύση συστήματος}\\
Όταν βρεθούν οι τιμές $ x_0,y_0 $ και των δύο αγνώστων, σχηματίζουμε το διατεταγμένο ζεύγος $ (x,y)=(x_0,y_0) $ το οποίο είναι η λύση του συστήματος.
\end{bhma}
\end{Methodos}
\Paradeigma{Λύση συστήματοσ με αντίθετουσ συντελεστέσ}
\textbf{Να λυθεί το παρακάτω σύστημα με τη μέθοδο των αντίθετων συντελεστών}
{\boldmath\[ \systeme{4x-y=5,3x+2y=12} \]}
\lysh\\
Επιλέγουμε με τη μέθοδο αυτή να απαλείψουμε τη μεταβλητή $ y $ του συστήματος. Έχουμε λοιπόν
\[ \left. \systeme{4x-y=5,3x+2y=12}\right| \synt{2}{1}\Rightarrow\systeme{8x-2y=10,3x+2y=12} \]
Οπότε προσθέτοντας τις εξισώσεις κατά μέλη προκύπτει
\begin{center}
\vspace{-5mm}
\begin{equation}\label{par:as}
\begin{tabular}{rr}
$ \displaystyle\systeme{8x-2y=10,3x+2y=12} $  &  \\ 
\hhline{-~} $ 11x=22 $ & $ \Rightarrow x=2  $
\end{tabular}
\end{equation}
\end{center}
Την τιμή αυτή της μεταβλητής $ x $ από τη σχέση \eqref{par:as} την αντικαθιστούμε σε οποιαδήποτε εξίσωση και υπολογίζουμε τη δεύτερη μεταβλητή $ y $.
\begin{equation}\begin{aligned}
3x+2y=12\Rightarrow 3\cdot2+2y=12&\Rightarrow 6+2y=12\\&\Rightarrow 2y=12-6\Rightarrow 2y=6\Rightarrow y=3\end{aligned}
\end{equation}
Από τις σχέσεις (3) και (4) παίρνουμε τη λύση του συστήματος η οποία είναι $ (x,y)=(2,3) $.
\begin{Methodos}[Μέθοδος των οριζουσών]{4cm}
Ένα γραμμικό σύστημα δύο εξισώσεων με δύο άγνωστους μπορούμε πλέον να το λύσουμε με τη χρήση οριζουσών ως εξής.
\begin{bhma}
\item \textbf{Υπολογισμός οριζουσών}\\
Υπολογίζουμε την ορίζουσα $ D $ των συντελεστών του συστήματος καθώς και τις ορίζουσες των μεταβλητών $ D_x $ και $ D_y $.
\item \textbf{Διερεύνηση - Εύρεση λύσεων}\\
Διακρίνουμε τις εξής περιπτώσεις
\begin{itemize}
\item Αν $ D\neq0 $ τότε υπολογίζουμε τις τιμές των μεταβλητών σύμφωνα με το \textbf{Θεώρημα 3} οπότε η μοναδική λύση θα είναι $ (x,y)=\left(\frac{D_x}{D},\frac{D_y}{D} \right) $.
\item Αν $ D=0 $ τότε διακρίνουμε τις εξής περιπτώσεις:
\begin{itemize}
\item Αν $ D_x\neq 0 $ ή $ D_y\neq 0 $ τότε το σύστημα είναι αδύνατο.
\item Αν $ D_x=D_y=0 $ τότε το σύστημα είναι αόριστο.
\end{itemize}
\end{itemize}
\end{bhma}
\end{Methodos}\mbox{}\\\\
\Paradeigma{Λύση συστήματοσ με τη μέθοδο οριζουσών}
\textbf{Να λυθεί το παρακάτω σύστημα}
{\boldmath\[ \systeme{x-5y=3,4x-3y=-5} \]}
\textbf{με τη μέθοδο των οριζουσών}.\\\\
\lysh\\
Η ορίζουσα των συντελεστών του συστήματος θα είναι
\[ D=\begin{vmatrix}
1&-5\\4&-3
\end{vmatrix}=1\cdot(-3)-4\cdot(-5)=-3+20=17 \]
Η ορίζουσα των συντελεστών είναι μη μηδενική οπότε το σύστημα έχει μοναδική λύση. Οι ορίζουσες των μεταβλητών θα είναι
\begin{gather*}
D_x=\begin{vmatrix}
3&-5\\-5&-3
\end{vmatrix}=3\cdot(-3)-(-5)\cdot(-5)=-34\;\textrm{ και }\\D_y=\begin{vmatrix}
1 & 3\\4&-5
\end{vmatrix}=1\cdot(-5)-4\cdot3=-17 \end{gather*}
Σύμφωνα με τα παραπάνω οι τιμές των μεταβλητών του συστήματος θα είναι 
\[ x=\frac{D_x}{D}=\frac{-34}{17}=-2\;\textrm{ και }\;y=\frac{D_y}{D}=\frac{-17}{17}=-1 \]
οι οποίες μας δίνουν τη λύση του συστήματος $ (x,y)=(-2,-1) $.\\\\
\Paradeigma{Λύση συστήματοσ με τη μέθοδο οριζουσών}
\textbf{Να λυθεί το παρακάτω σύστημα με τη μέθοδο των οριζουσών}
{\boldmath\[ \systeme{4x-8y=1,6x-12y=4} \]}
\lysh\\
Υπολογίζουμε την ορίζουσα των συντελεστών
\[ D=\begin{vmatrix}
4& -8\\6 & -12
\end{vmatrix}=4\cdot(-12)-6\cdot(-8)=-48+48=0 \]
Η μηδενική ορίζουσα μας δείχνει ότι το σύστημα είναι είτε αόριστο είτε αδύνατο. Για να προσδιορίσουμε το είδος του υπολογίζουμε τις ορίζουσες $ D_x,D_y $.
\begin{align*}
D_x&=\begin{vmatrix}
1& -8\\4 & -12
\end{vmatrix}=1\cdot(-12)-4\cdot(-8)=-12+32=20\neq 0\\
D_y&=\begin{vmatrix}
4& 1\\6 & 4
\end{vmatrix}=4\cdot 4-1\cdot 6=16-6=10\neq 0
\end{align*}
Οι ορίζουσες των μεταβλητών είναι μη μηδενικές οπότε το σύστημα είναι αδύνατο.\\\\
\Paradeigma{Λύση συστήματοσ με τη μέθοδο οριζουσών}
\textbf{Να λυθεί το παρακάτω σύστημα με τη μέθοδο των οριζουσών}
{\boldmath\[ \systeme{3x-y=5,6x-2y=10} \]}
\lysh\\
Η ορίζουσα του συστήματος θα είναι
\[ D=\begin{vmatrix}
3& -1\\6 & -2
\end{vmatrix}=3\cdot(-2)-6\cdot(-1)=-6+6=0 \]
Θα πρέπει κι εδώ να προσδιορίσουμε αν το σύστημα είναι αδύνατο ή αόριστο. Για να προσδιορίσουμε το είδος του υπολογίζουμε τις ορίζουσες $ D_x,D_y $.
\begin{align*}
D_x&=\begin{vmatrix}
5& -1\\10 & -2
\end{vmatrix}=5\cdot(-2)-(-1)\cdot 10=-10+10=0\\
D_y&=\begin{vmatrix}
3& 5\\6 & 10
\end{vmatrix}=3\cdot 10-5\cdot 6=30-30=0
\end{align*}
Οι ορίζουσες των μεταβλητών είναι μηδενικές οπότε το σύστημα είναι αόριστο. Για να βρούμε τη μορφή όλων των λύσεων τις εκφράζουμε με τη βοήθεια μιας παραμέτρου ως εξής : 
Λύνουμε την πρώτη εξίσωση ως προς $ y $ : 
\begin{equation}\label{par:oraor}
3x-y=5\Rightarrow -y=5-3x\Rightarrow y=3x-5 
\end{equation}
Θέτουμε στη δεύτερη μεταβλητή $ x=\lambda $ και παίρνουμε από τη σχέση \eqref{par:oraor} $ y=3\lambda-5 $. Επομένως οι άπειρες λύσεις δίνονται παραμετρικά από τη σχέση \[ (x,y)=(\lambda,3\lambda-5) \]
\begin{Methodos}[Γραφική επίλυση συστήματο2]{3cm}\label{meth:grafikh}
Ένα γραμμικό σύστημα μπορεί να λυθεί και γεωμετρικά με τη βοήθεια των ευθειών των εξισώσεων.
\begin{bhma}
\item \textbf{Χάραξη των ευθειών}\\
Σχεδιάζουμε τις δύο ευθείες του συστήματος βρίσκοντας για καθεμιά δύο σημεία της με τη βοήθεια της εξίσωσης της.
\item \textbf{Σχετική θέση ευθειών}\\
Διακρίνουμε τις εξής περιπτώσεις για τη σχετική θέση των δύο ευθειών
\begin{itemize}[itemsep=0mm]
\item Αν οι ευθείες \textbf{τέμνονται} σε ένα σημείο τότε οι συντεταγμένες του κοινού αυτού σημείου είναι η ζητούμενη λύση του συστήματος. Τις συντεταγμένες αυτές τις βρίσκουμε σχεδιάζοντας από το σημείο, κάθετες γραμμές προς τους άξονες $ x'x $ και $ y'y $.
\item Αν οι δύο ευθείες είναι μεταξύ τους παράλληλες τότε \textbf{δεν υπάρχουν κοινά σημεία} μεταξύ τους και κατά συνέπεια το σύστημα δεν έχει λύση οπότε είναι \textbf{αδύνατο}.
\item Τέλος αν οι ευθείες \textbf{ταυτίζονται} τότε έχουν μεταξύ τους άπειρα κοινά σημεία οπότε το σύστημα έχει άπειρες λύσεις δηλαδή είναι \textbf{αόριστο}.
\end{itemize}
\vspace{-7mm}
\begin{center}
\begin{tabular}{p{4cm}p{4cm}p{4cm}}
\begin{tikzpicture}
\begin{axis}[aks_on,belh ar,ticks=none,xlabel={\footnotesize $x$},
ylabel={\footnotesize $y$},xmin=-.3,xmax=3.5,ymin=-.3,ymax=3.5,x=.8cm,y=.8cm]
\addplot[grafikh parastash,\xrwmath,domain=-.2:3.3]{-x+2.5};
\addplot[grafikh parastash,\xrwmath,domain=-.2:2.7]{.8*x+.7};
\node (A) at (axis cs:1,1.5){};
\end{axis}
\tkzDrawPoint(A)
\tkzLabelPoint[right](A){$A(x_0,y_0)$}
\node at (0,0) {$O$};
\node at (2,3.4) {\footnotesize {Μοναδική λύση}};
\node at (2,1.9) {\footnotesize $\varepsilon_2$};
\node at (.8,2) {\footnotesize $\varepsilon_1$};
\end{tikzpicture}	& \begin{tikzpicture}
\begin{axis}[aks_on,belh ar,ticks=none,xlabel={\footnotesize $x$},
ylabel={\footnotesize $y$},xmin=-.3,xmax=3.5,ymin=-.3,ymax=3.5,x=.8cm,y=.8cm]
\addplot[grafikh parastash,\xrwmath,domain=-.2:2.5]{x+.7};
\addplot[grafikh parastash,\xrwmath,domain=-.2:3.3]{x-1};
\end{axis}
\node at (0,0) {$O$};
\node at (2,3.4) {\footnotesize {Καμία λύση}};
\node at (2,1.5) {\footnotesize $\varepsilon_2$};
\node at (1,2) {\footnotesize $\varepsilon_1$};
\end{tikzpicture} & \begin{tikzpicture}
\begin{axis}[aks_on,belh ar,ticks=none,xlabel={\footnotesize $x$},
ylabel={\footnotesize $y$},xmin=-.3,xmax=3.5,ymin=-.3,ymax=3.5,x=.8cm,y=.8cm]
\addplot[grafikh parastash,\xrwmath,domain=-.2:3.5]{0.7*x+.5};
\addplot[grafikh parastash,\xrwmath,domain=-.2:3.5,dashed]{0.7*x+.45};
\end{axis}
\node at (0,0) {$O$};
\node at (2,3.4) {\footnotesize {Άπειρες λύσεις}};
\node at (2,1.5) {\footnotesize $\varepsilon_2$};
\node at (1,1.5) {\footnotesize $\varepsilon_1$};
\end{tikzpicture} \\ 
\end{tabular} 
\end{center}
\end{bhma}
\end{Methodos}
\Paradeigma{Γραφική επίλυση συστήματοσ}
\textbf{Να λυθούν γραφικά τα παρακάτω συστήματα}
\begin{multicols}{3}
\begin{brlist}
\item {\boldmath$ \systeme{3x-2y=4,x+y=3} $}
\item {\boldmath$ \systeme{2x-y=5,4x-2y=4} $}
\item {\boldmath$ \systeme{x-3y=1,3x-9y=3} $}
\end{brlist}
\end{multicols}
\noindent
Για κάθε μια από τις εξισώσεις των παραπάνω συστημάτων θα βρούμε δύο ζεύγη αριθμών που τις επαληθεύουν τα οποία θα παριστάνουν σημεία στο επίπεδο έτσι ώστε να σχεδιαστούν οι ευθείες.
\begin{rlist}
\item Στην πρώτη εξίσωση επιλέγουμε $ x=0 $ οπότε έχουμε
\[ 3x-2y=4\Rightarrow 3\cdot0-2y=4\Rightarrow y=-2 \]
Αποκτάμε έτσι το σημείο $ A(0,-2) $. Ένα δεύτερο σημείο θα βρεθεί παίρνοντας π.χ. $ y=0 $ οπότε με πράξεις προκύπτει
\[ 3x-2y=4\Rightarrow 3χ-2\cdot0=4\Rightarrow3x=4\Rightarrow y=\frac{4}{3} \]
Προκύπτει έτσι το σημείο $ B\left( \frac{4}{3},0\right) $. Με παρόμοιο τρόπο υπολογίζουμε δύο σημεία και της 2\tss{ης} ευθείας. Έχουμε από τη 2\tss{η} εξίσωση για $ y=0 $\\
\wrapr{-7mm}{12}{4.5cm}{-7mm}{\begin{tikzpicture}
\begin{axis}[aks_on,belh ar,xlabel={\footnotesize $x$},
ylabel={\footnotesize $y$},xmin=-.3,xmax=3.5,ymin=-.3,ymax=3.5,x=1cm,y=1cm]
\addplot[grafikh parastash,\xrwmath,domain=-.2:3.3]{1.5*x-2};
\addplot[grafikh parastash,\xrwmath,domain=-.2:3.2]{3-x};
\draw[dashed] (axis cs:2,0)--(axis cs:2,1)--(axis cs:0,1);
\node (A) at (axis cs:2,1){};
\end{axis}
\tkzDrawPoint(A)
\tkzLabelPoint[right](A){$M(2,1)$}
\node at (0,0) {$O$};
\node at (2.4,2) {\footnotesize $\varepsilon_2$};
\node at (1.2,2) {\footnotesize $\varepsilon_1$};
\end{tikzpicture}}{
\[ x+y=3\Rightarrow x+0=3\Rightarrow x=3 \] και παίρνουμε έτσι το σημείο $ \varGamma(3,0) $. Επίσης για $ x=0 $
\[ x+y=3\Rightarrow 0+y=3\Rightarrow y=3 \] άρα το δεύτερο σημείο της θα είναι το $ \varDelta(0,3) $. Σχεδιάζοντας τις δύο ευθείες προκύπτει το διπλανό σχήμα. Από το σχήμα αυτό παρατηρούμε ότι οι δύο ευθείες έχουν ένα κοινό σημείο $ M $. Από το σημείο αυτό αν σχεδιάσουμε κάθετες γραμμές πάνω στους άξονες $ x'x $ και $ y'y $ προκύπτουν οι συντεταγμένες του κοινού σημείου οι οποίες είναι $ (x,y)=(2,1) $. Οι συντεταγμένες αυτές είναι η λύση του συστήματος.\\}
\item Με παρόμοιο τρόπο όπως και στο προηγούμενο παράδειγμα δύο σημεία για κάθε ευθεία είναι τα $ A(1,0) $, $ B(0,1) $ και $ \varGamma(2,-1) $, $ \varDelta(2{,}5,0) $ αντίστοιχα. Σχεδιάζοντας τις δύο ευθείες στο σύστημα συντεταγμένων παρατηρούμε ότι είναι παράλληλες άρα δεν έχουν κοινά σημεία οπότε το σύστημα είναι αδύνατο.
\item Σχεδιάζοντας τις ευθείες και του τρίτου συστήματος με τον τρόπο που είδαμε στα προηγούμενα παραδείγματα παρατηρούμε ότι ταυτίζονται. Αυτό σημαίνει ότι έχουν άπειρα κοινά σημεία και κατά συνέπεια το σύστημα είναι αόριστο.
\begin{center}
\begin{tabular}{cc}
\begin{tikzpicture}
\begin{axis}[aks_on,belh ar,xlabel={\footnotesize $x$},
ylabel={\footnotesize $y$},xmin=-.3,xmax=3.5,ymin=-1,ymax=2.5,x=1cm,y=1cm]
\addplot[grafikh parastash,\xrwmath,domain=-.2:3.3]{2*x-1};
\addplot[grafikh parastash,\xrwmath,domain=-.2:3.2]{2*x-3.5};
\end{axis}
\node at (0,.8) {$O$};
\node at (2.7,3) {\footnotesize $\varepsilon_2$};
\node at (1.1,2.7) {\footnotesize $\varepsilon_1$};
\node at (3,0.4) {\footnotesize {Καμία λύση}};
\end{tikzpicture}	& \begin{tikzpicture}
\begin{axis}[aks_on,belh ar,xlabel={\footnotesize $x$},
ylabel={\footnotesize $y$},xmin=-.3,xmax=3.5,ymin=-1,ymax=2.5,x=1cm,y=1cm]
\addplot[grafikh parastash,\xrwmath,domain=-.2:3.3]{x/3-1/3};
\addplot[grafikh parastash,\xrwmath,domain=-.2:3.2,dashed]{x/3-1/3-.05};
\end{axis}
\node at (0,0.75) {$O$};
\node at (2.7,1.2) {\footnotesize $\varepsilon_2$};
\node at (1.2,1.2) {\footnotesize $\varepsilon_1$};
\node at (2,2.5) {\footnotesize {Άπειρες λύσεις}};
\end{tikzpicture} \\ 
\end{tabular} 
\end{center}
\end{rlist}
\begin{Methodos}[Επίλυση σύνθετου συστήματος]{3cm}
Αν μας ζητείται να λύσουμε ένα σύστημα του οποίου οι εξισώσεις δεν είναι στην απλή γραμμική μορφή όπως φαίνεται στον \textbf{Ορισμό 3}, τότε
\begin{bhma}
\item \textbf{Πράξεις}\\
Εκτελούμε πράξεις και στα δύο μέλη κάθε εξίσωσης και διαχωρίζουμε τους γνωστούς από τους άγνωστους όρους, ώστε να τις φέρουμε σε γραμμική μορφή.
\item \textbf{Λύση γραμμικού συστήματος}\\
Λύνουμε το γραμμικό πλέον σύστημα με οποιαδήποτε μέθοδο μας συμφέρει, επιλέγοντας μια από τις \textbf{Μεθόδους 1,2,3} και \textbf{4}.
\end{bhma}
\end{Methodos}
\Paradeigma{Σύνθετο σύστημα}
\textbf{Να λυθεί το παρακάτω σύστημα με οποιαδήποτε μέθοδο.}
{\boldmath\[ \ccases{
\;\dfrac{x+2}{3}+\dfrac{1-y}{2}=2\\
\;\dfrac{2x-1}{5}+\dfrac{y}{3}=-\dfrac{2}{15}} \]}
\lysh\\
Η μορφή στην οποία βρίσκεται κάθε εξίσωση του συστήματος δεν είναι η απλή γραμμική. Αυτό σημαίνει ότι δεν μπορεί να εφαρμοστεί ακόμα κάποια από τις μεθόδους επίλυσης. Κάνοντας πράξεις θα απλοποιήσουμε τη μορφή του συστήματος.
\begin{gather*}
\ccases{
\;\dfrac{x+2}{3}+\dfrac{1-y}{2}=2\\
\;\dfrac{2x-1}{5}+\dfrac{y}{3}=-\dfrac{2}{15}}\Rightarrow\ccases{
\;6\dfrac{x+2}{3}+6\dfrac{1-y}{2}=2\cdot 6\\[2mm]
\;15\dfrac{2x-1}{5}+15\dfrac{y}{3}=-15\dfrac{2}{15}}\Rightarrow\\
\ccases{
\;2(x+2)+3(1-y)=12\\
\;3(2x-1)+5y=-2}\Rightarrow\ccases{2x+4+3-3y=12\\6x-3+5y=-2}\Rightarrow\ccases{2x-3y=5\\6x+5y=1}
\end{gather*}
Το τελευταίο σύστημα έχει τη ζητούμενη μορφή οπότε μπορούμε να το λύσουμε με μια από τις παραπάνω μεθόδους. Με τη μέθοδο των οριζουσών έχουμε :
\[ D=\begin{vmatrix}
2& -3\\6& 5
\end{vmatrix}=28\;,\;D_x=\begin{vmatrix}
5& -3\\1& 5
\end{vmatrix}=28\;,\;D_y=\begin{vmatrix}
2& 5\\6& 1
\end{vmatrix}=-28 \]
Οπότε οι τιμές των δύο μεταβλητών είναι $ x=\frac{D_x}{D}=\frac{28}{28}=1 $ και $ y=\frac{D_y}{D}=\frac{-28}{28}=-1 $ που μας δίνουν τη λύση $ (x,y)=(1,-1) $.
\begin{Methodos}[Επίλυση συστήματος {$\mathbold{3\times 3}$}]{4cm}
Για να λυθεί ένα $ 3\times 3 $ γραμμικό σύστημα με μεταβλητές έστω $ x,y,z $, θα χρησιμοποιήσουμε τη μέθοδο της αντικατάστασης ώστε να μεταβούμε σε ένα $ 2\times 2 $ γραμμικό σύστημα.
\begin{bhma}
\item \textbf{Επιλογή εξίσωσης}\\
Επιλέγουμε μια από τις τρεις εξισώσεις για να λύσουμε ως προς οποιονδήποτε άγνωστο.
\item \textbf{Αντικατάσταση}\\
Αντικαθιστούμε τη μεταβλητή αυτή στις υπόλοιπες δύο εξισώσεις του συστήματος με αποτέλεσμα να μετατραπούν σε γραμμικές εξισώσεις με δύο άγνωστους.
\item \textbf{Επίλυση συστήματος {\boldmath{$ 2\times 2 $}}}\\
Απλοποιούμε τις δύο νέες εξισώσεις ώστε να τις φέρουμε στην απλή γραμμική μορφή και λύνουμε το $ 2\times 2 $ σύστημα με οποιαδήποτε μέθοδο.
\item \textbf{Εύρεση τρίτου αγνώστου}\\
Όταν βρεθούν οι τιμές των δύο μεταβλητών του $ 2\times 2 $ συστήματος τις αντικαθιστούμε στη σχέση που προέκυψε στο \textbf{1\tss{ο} Βήμα} και υπολογίζουμε και τον τρίτο άγνωστο.
\item \textbf{Λύση συστήματος}\\
Σχηματίζουμε τη λύση του αρχικού συστήματος η οποία θα είναι μια διατεταγμένη τριάδα αριθμών της μορφής $ (x,y,z)=(x_0,y_0,z_0) $.
\end{bhma}
\end{Methodos}\mbox{}\\
\Paradeigma{Λύση γραμμικού συστήματοσ {$\mathbold{3\times 3}$}}
\textbf{Να λυθεί το παρακάτω γραμμικό σύστημα}
{\boldmath\[ \systeme{2x-y+3z=8,3x+4y-z=1,x+y-4z=-3} \]}
\lysh\\
Επιλέγουμε την 3\tss{η} εξίσωση προκειμένου να λύσουμε ως προς τη μεταβλητή $ x $ οπότε θα έχουμε
\begin{equation} \systeme{2x-y+3z=8,3x+4y-z=1,x+y-4z=-3}\Rightarrow x=4z-y-3 
\end{equation}
Αντικαθιστώντας τη μεταβλητή $ x $ από τη σχέση (8) στις δύο πρώτες εξισώσεις του συστήματος προκύπτει
\[ \ccases{2(4z-y-3)-y+3z=8\\3(4z-y-3)+4y-z=1}\Rightarrow\ccases{8z-2y-6-y+3z=8\\12z-3y-9+4y-z=1}\Rightarrow
\systeme{-3y+11z=14,y+11z=10} \]
Η μορφή του συστήματος μας επιτρέπει να χρησιμοποιήσουμε ένα τέχνασμα για να φτάσουμε γρήγορα στη λύση. Αφαιρώντας κατά μέλη έχουμε
\begin{equation}
\begin{tabular}{rr}
$  \displaystyle\systeme{-3y+11z=14,y+11z=10} $  &  \\ 
\hhline{-~}  $ -4y=4 $ & $ \Rightarrow y=-1 $
\end{tabular}
\end{equation}
Από την πρώτη εξίσωση και τη σχέση (9) υπολογίζουμε τον άγνωστο $ z $
\begin{equation} -3y+11z=14\Rightarrow-3\cdot(-1)+11z=14\Rightarrow3+11z=14\Rightarrow11z=11\Rightarrow z=1 \end{equation}
Τις τιμές των μεταβλητών $ y,z $ από τις σχέσεις (9) και (10) τις αντικαθιστούμε στην ισότητα (8) και υπολογίζουμε τη μεταβλητή $ x $.
\[ x=4z-y-3=4\cdot1-(-1)-3=4+1-3=2 \]
Επομένως η λύση του συστήματος θα είναι $ (x,y,z)=(2,-1,1) $.
\begin{Methodos}[Επίλυση προβλημάτων]{4cm}
Συχνά καλούμαστε να λύσουμε προβλήματα τα οποία μας ζητούν την εύρεση δύο άγνωστων αριθμών, οι οποίοι σχετίζονται μεταξύ τους. Τότε χρειάζεται η κατασκευή και επίλυση ενός συστήματος ώστε να βρεθούν συγχρόνως και οι δύο άγνωστοι. Για να γίνει αυτό
\begin{bhma}
\item \textbf{Συμβολισμός αγνώστων}\\
Αφού εντοπίσουμε τους ζητούμενους άγνωστους αριθμούς του προβλήματος, τους συμβολίζουμε χρησιμοποιώντας δύο μεταβλητές.
\item \textbf{Κατασκευή συστήματος}\\
Με τη βοήθεια των δεδομένων του προβλήματος, αναγνωρίζουμε τις σχέσεις μεταξύ των δύο αγνώστων και κατασκευάζουμε τις εξισώσεις.
\item \textbf{Επίλυση συστήματος}\\
Με τις εξισώσεις αυτές σχηματίζουμε το γραμμικό σύστημα το οποίο και λύνουμε.
\item \textbf{Λύση συστήματος - Εξέταση περιορισμών}\\
Αφού βρεθεί η λύση του συστήματος, επαληθεύουμε τη λύση αυτή εξετάζοντας τυχόν περιορισμούς του προβλήματος.
\end{bhma}
\end{Methodos}
\Paradeigma{Επίλυση προβλήματοσ}
\textbf{Θέλουμε να μοιράσουμε 210 βιβλία σε 40 πακέτα των 4 και 6 βιβλίων. Πόσα μικρά πακέτα των 4 βιβλίων και πόσα μεγάλα πακέτα των 6 θα χρειαστούμε;}\\\\
\lysh\\
Από την εκφώνηση του προβλήματος παρατηρούμε ότι αυτό που ζητάει το πρόβλημα είναι ο αριθμός των μικρών πακέτων, δηλαδή των πακέτων με τα 4 βιβλία, και ο αριθμός των μεγάλων πακέτων, αυτών με τα 6 βιβλία. Έτσι θα πρέπει να κατασκευάσουμε 2 εξισώσεις με 2 άγνωστους αριθμούς και συνδυάζοντας τες να βρούμε μοναδική λύση γι αυτούς.
Συμβολίζουμε τους άγνωστους αριθμούς με μεταβλητές : \begin{gather*}
x\;:\;\textrm{Το πλήθος των μικρών πακέτων με τα 4 βιβλία}\\
y\;:\;\textrm{Το πλήθος των μεγάλων πακέτων με τα 6 βιβλία}
\end{gather*}
Κατασκευάζουμε τις 2 εξισώσεις, χρησιμοποιώντας τα δεδομένα του προβλήματος.
\begin{enumerate}[label=\bf\textit{\arabic*\textsuperscript{o}\;στοιχείο},leftmargin=0cm,itemindent=1.8cm]
\item \mbox{}\\Όλα τα πακέτα μαζί θα πρέπει να είναι 40. Άρα η πρόταση αυτή θα γραφτεί συμβολικά \begin{equation}\label{par:prob1}
x+y=40
\end{equation} 
\item \mbox{}\\Όλα τα βιβλία είναι 210. Αναλυτικά λοιπόν θα έχουμε : 
\begin{itemize}
\item Ένα μικρό πακέτο έχει 4 βιβλία οπότε $ x $ μικρά πακέτα θα έχουν $ 4\cdot x $ βιβλία.
\item Ένα μεγάλο πακέτο έχει 6 βιβλία οπότε $ x $ μικρά πακέτα θα έχουν $ 6\cdot y $ βιβλία.
\end{itemize}
Επομένως η ζητούμενη ισότητα θα είναι
\begin{equation}\label{par:prob2}
4x+6y=210
\end{equation} 
\end{enumerate}
Συνδυάζοντας λοιπόν τις εξισώσεις \eqref{par:prob1} και \eqref{par:prob2} προκύπτει το σύστημα \begin{equation}\label{par:prob3}
\systeme[xy]{4x+6y=210,x+y=40}
\end{equation}
Λύνοντας το σύστημα \eqref{par:prob3} θα φτάσουμε στο ζητούμενο. Με τη μέθοδο της αντικατάστασης λύνουμε τη δεύτερη εξίσωση ως προς $ x $
\begin{equation}\label{par:prob4}
\systeme[xy]{4x+6y=210,x+y=40}\Rightarrow x=40-y \end{equation}
Αντικαθιστώντας το αποτέλεσμα στην πρώτη έχουμε
\[ 4x+6y=210\Rightarrow 4(40-y)+6y=210\Rightarrow 160 -4y+6y=210\Rightarrow2y=50\Rightarrow y=25 \]
Έτσι, ο αριθμός των μεγάλων πακέτων είναι $ 25 $.Θέτοντας την τιμή αυτή στη σχέση \eqref{par:prob4} ο αριθμός των μικρών πακέτων θα είναι
\[ x=40-y=40-25=15 \]
Έχουμε λοιπόν $ (x,y)=(15,25) $ άρα $ 15 $ μικρά και $ 25 $ μεγάλα πακέτα.
\begin{Methodos}[Παραμετρικά συστήματα]{4cm}
Η επίλυση-διερεύνηση ενός παραμετρικού συστήματος γίνεται ευκολότερα με τη μέθοδο των οριζουσών.
\begin{bhma}
\item \textbf{Υπολογισμός ορίζουσας}\\
Υπολογίζουμε την ορίζουσα $ D $ του συστήματος η οποία θα αποτελεί μια παράσταση που θα περιέχει την παράμετρο.
\item \textbf{Περιπτώσεις}\\
Η ορίζουσα, ως αλγεβρική παράσταση, παίρνει διάφορες τιμές οπότε στο βήμα αυτό εξετάζουμε τις παρακάτω περιπτώσεις.
\begin{itemize}
\item Αν $ D\neq0 $ τότε το σύστημα θα έχει μοναδική λύση, η οποία υπολογίζεται σύμφωνα με τη \textbf{Μέθοδο 3}.Η λύση γράφεται με τη βοήθεια της παραμέτρου.
\item Αν $ D=0 $ τότε τοποθετούμε στο αρχικό σύστημα τις τιμές της παραμέτρου που μηδενίζουν την ορίζουσα και με οποιαδήποτε μέθοδο καταλήγουμε είτε σε αόριστο είτε σε αδύνατο σύστημα.
\end{itemize}
\end{bhma}
\end{Methodos}
\Paradeigma{Επίλυση παραμετρικού συστήματοσ}
\textbf{Να βρεθούν οι λύσεις του παρακάτω συστήματος για κάθε τιμή της παραμέτρου {\boldmath$ {\lambda\in\mathbb{R}} $}.}
{\boldmath\[ \systeme[xy]{\lambda x-y=1,(\lambda\-2)x+\lambda y=2} \]}
\lysh\\
Ξεκινάμε υπολογίζοντας την ορίζουσα των συντελεστών του συστήματος η οποία θα είναι
\[ D=\begin{vmatrix}
\lambda & -1\\\lambda-2 & \lambda
\end{vmatrix}=\lambda^2-(\lambda-2)\cdot(-1)=\lambda^2+\lambda-2 \]
Εξετάζουμε τώρα τις εξής περιπτώσεις
\begin{rlist}
\item Έστω $ D\neq0\Rightarrow \lambda^2+\lambda-2\neq0\Rightarrow \lambda\neq1\,\textrm{ και }\,\lambda\neq-2 $. Τότε το σύστημα θα έχει μοναδική λύση την οποία υπολογίζουμε με τη βοήθεια των τύπων. Έχουμε
\[ D_x=\begin{vmatrix}
1 & -1\\2 & \lambda
\end{vmatrix}=\lambda+2\;,\;D_y=\begin{vmatrix}
\lambda & 1\\\lambda-2 & 2
\end{vmatrix}=\lambda+2 \]
Επομένως οι τιμές των μεταβλητών του συστήματος θα γραφτούν με τη βοήθεια της παραμέτρου ως εξής 
\begin{gather*} x=\frac{D_x}{D}=\frac{\lambda+2}{\lambda^2+\lambda-2}=\frac{\lambda+2}{(\lambda-1)(\lambda+2)}=\frac{1}{\lambda-1}\\
y=\frac{D_y}{D}=\frac{\lambda+2}{\lambda^2+\lambda-2}=\frac{\lambda+2}{(\lambda-1)(\lambda+2)}=\frac{1}{\lambda-1}
\end{gather*}
Η λύση λοιπόν του συστήματος θα δίνεται από τον τύπο $ (x,y)=\left(\frac{1}{\lambda-1},\frac{1}{\lambda-1} \right)  $.
\item Αν $ D=0\Rightarrow\lambda^2+\lambda-2=0 $ τότε $ \lambda=1 $ ή $ \lambda=-2 $. Εδώ διακρίνουμε τις εξής υποπεριπτώσεις :
\begin{itemize}
\item Αν $ \lambda=1 $ τότε το αρχικό σύστημα θα πάρει τη μορφή
\[ \systeme{x-y=1,-x+y=2}\Rightarrow\systeme{x-y=1,x-y=-2} \]
Παρατηρούμε ότι οι εξισώσεις του συστήματος έχουν ίσα πρώτα μέλη ενώ τα δεύτερα τους μέλη είναι άνισα. Οπότε το σύστημα θα είναι αδύνατο άρα δεν έχει καμία λύση.
\item Αν $ \lambda=-2 $ τότε θα προκύψει το σύστημα
\[ \systeme{-2x-y=1,-4x-2y=2}\Rightarrow\systeme{-2x-y=1,-2x-y=1} \]
στο οποίο παρατηρούμε ότι οι εξισώσεις ταυτίζονται άρα το σύστημα είναι αόριστο οπότε θα έχει άπειρες λύσεις. Για την παραμετρική μορφή των λύσεων θα έχουμε :
\[ -2x-y=1\Rightarrow y=-2x-1 \]
και θέτοντας όπου $ x=\kappa $ προκύπτουν οι λύσεις : $ (x,y)=(\kappa,-2\kappa-1) $.
\end{itemize}
\end{rlist}
\newpage
\noindent
\Alyta
\bcc{Γραμμικη Εξισωση}
\begin{multicols}{2}
\Askhsh \textbf{Σημείο ευθείας}\\
Να εξεταστεί αν το σημείο $ A(2,1) $ ανήκει σε καθεμία από τις παρακάτω ευθείες.
\begin{multicols}{2}
\begin{rlist}[leftmargin=5mm]
\item $ x-3y=4 $
\item $ 2x+3y=7 $
\item $ 4x+2y=5 $
\item $ 8x-7y=9 $
\end{rlist}
\end{multicols}
\Askhsh \textbf{Σημεία τομής με άξονες}\\
Να βρεθούν τα σημεία τομής των παρακάτω ευθειών με τους άξονες $ x'x $ και $ y'y $.
\begin{multicols}{2}
\begin{rlist}[leftmargin=5mm]
\item $ x-2y=4 $
\item $ 4x-y=8 $
\item $ 2x-3y=-6 $
\item $ 7x-4y=11 $
\end{rlist}
\end{multicols}
\Askhsh \textbf{Σημεία τομής με άξονες}\\
Να βρεθούν τα σημεία τομής των παρακάτω ευθειών με τους άξονες $ x'x $ και $ y'y $.
\begin{multicols}{2}
\begin{rlist}[leftmargin=5mm]
\item $ x=3 $
\item $ y=5 $
\item $ -x=-7 $
\item $ 2y=4 $
\end{rlist}
\end{multicols}
\Askhsh \textbf{Εύρεση παραμέτρων}\\
Να βρεθούν οι τιμές της παραμέτρου $ \lambda\in\mathbb{R} $ ώστε η ευθεία $ \lambda x+(\lambda-1)y=4 $ να διέρχεται από το σημείο $ A(-2,3) $.\\\\
\Askhsh \textbf{Εύρεση παραμέτρων}\\
Να βρεθούν οι τιμές της παραμέτρου $ \lambda\in\mathbb{R} $ ώστε η ευθεία $ (\lambda^2-1)x+(1-\lambda)y=2 $ να διέρχεται από το σημείο $ A(1,3) $.
\end{multicols}
\bcc{Μεθοδος Αντικαταστασης}
\begin{multicols}{2}
\Askhsh \textbf{Επίλυση συστήματος}\\
Να λυθούν τα παρακάτω γραμμικά συστήματα με τη μέθοδο της αντικατάστασης.
\begin{multicols}{2}
\begin{rlist}[leftmargin=5mm]
\item $ \systeme{x-y=1,x=4} $
\item $ \systeme{2x+4y=8,y=3} $
\item $ \systeme{x=4,x-y=9} $
\item $ \systeme{x-y=2,x+y=8} $
\end{rlist}
\end{multicols}
\Askhsh \textbf{Επίλυση συστήματος}\\
Να λυθούν τα παρακάτω γραμμικά συστήματα με τη μέθοδο της αντικατάστασης.
\begin{multicols}{2}
\begin{rlist}[leftmargin=5mm]
\item $ \systeme{3x+2y=5,2x-y=1} $
\item $ \systeme{x+4y=-2,3x-7y=13} $
\item $ \systeme{4x-3y=-1,5x-2y=4} $
\item $ \systeme{7x+2y=29,3x-y=18} $
\end{rlist}
\end{multicols}

\Askhsh \textbf{Επίλυση συστήματος}\\
Να λυθούν τα παρακάτω γραμμικά συστήματα με τη μέθοδο της αντικατάστασης.
\begin{multicols}{2}
\begin{rlist}[leftmargin=5mm]
\item $ \systeme{x-2y=4,2x-4y=8} $
\item $ \systeme{3x-4y=1,-6x+8y=-2} $
\item $ \systeme{4x+2y=6,6x+3y=9} $
\end{rlist}
\end{multicols}

\Askhsh \textbf{Επίλυση συστήματος}\\
Να λυθούν τα παρακάτω γραμμικά συστήματα με τη μέθοδο της αντικατάστασης.
\begin{multicols}{2}
\begin{rlist}[leftmargin=5mm]
\item $ \systeme{-x+y=2,2x-2y=3} $
\item $ \systeme{x=2y-1,4x-8y=5} $
\item $ \systeme{2x+y=1,y=7-2x} $
\end{rlist}
\end{multicols}

\end{multicols}
\bcc{Μεθοδος Αντιθετων Συντελεστων}
\begin{multicols}{2}
\Askhsh \textbf{Επίλυση συστήματος}\\
Να λυθούν τα παρακάτω γραμμικά συστήματα με τη μέθοδο των αντίθετων συντελεστών.
\begin{multicols}{2}
\begin{rlist}[leftmargin=5mm]
\item $ \systeme{x-y=3,x+y=7} $
\item $ \systeme{2x-3y=1,4x-5y=1} $
\item $ \systeme{x+y=10,3x+y=16} $
\item $ \systeme{-x-y=4,7x+4y=-19} $
\end{rlist}
\end{multicols}
\Askhsh \textbf{Επίλυση συστήματος}\\
Να λυθούν τα παρακάτω γραμμικά συστήματα με τη μέθοδο των αντίθετων συντελεστών.
\begin{multicols}{2}
\begin{rlist}[leftmargin=5mm]
\item $ \systeme{4x-5y=-1,3x+7y=10} $
\item $ \systeme{4x-y=7,x+2y=4} $
\item $ \systeme{11x-8y=27,5x+9y=-13} $
\item $ \systeme{8x+6y=28,7x-5y=4} $
\end{rlist}
\end{multicols}
\end{multicols}
\bcc{Μεθοδος Οριζουσων}
\begin{multicols}{2}
\Askhsh \textbf{Επίλυση συστήματος}\\
Να λυθούν τα παρακάτω γραμμικά συστήματα με τη μέθοδο των οριζουσών.
\begin{multicols}{2}
\begin{rlist}[leftmargin=5mm]
\item $ \systeme{2x+y=5,x-4y=-2} $
\item $ \systeme{3x+5y=16,4x-y=6} $
\item $ \systeme{x+5y=12,7x+3y=17} $
\item $ \systeme{6x-y=20,4x+9y=-6} $
\end{rlist}
\end{multicols}
\Askhsh \textbf{Επίλυση συστήματος}\\
Να λυθούν τα παρακάτω γραμμικά συστήματα με τη μέθοδο των οριζουσών.
\begin{multicols}{2}
\begin{rlist}[leftmargin=5mm]
\item $ \systeme{2x+y=5,4x+2y=3} $
\item $ \systeme{4x+2y=8,6x+3y=12} $
\item $ \systeme{x-y=3,2x-2y=5} $
\item $ \systeme{3x+5y=2,6x+10y=4} $
\end{rlist}
\end{multicols}
\end{multicols}
\bcc{Γραφικη Επιλυση}
\begin{multicols}{2}
\Askhsh \textbf{Γραφική επίλυση συστήματος}\\
Να λυθούν γραφικά τα παρακάτω γραμμικά συστήματα.
\begin{multicols}{2}
\begin{rlist}[leftmargin=5mm]
\item $ \systeme{x-y=3,3x+y=13} $
\item $ \systeme{2x+y=4,x+4y=8} $
\item $ \systeme{3x-y=2,6x-2y=4} $
\item $ \systeme{x-2y=-3,-2x+4y=5} $
\end{rlist}
\end{multicols}
\Askhsh \textbf{Κοινά σημεία ευθειών}\\
Να βρεθούν, αν υπάρχουν, τα κοινά σημεία των παρακάτω ευθειών.
\begin{rlist}
\item $ x+3y=6 $ και $ 2x+y=8 $
\item $ 3x+4y=5 $ και $ -x+5y=3 $
\item $ 2x-y=10 $ και $ 4x-2y=7 $
\item $ 3x-y=2 $ και $ 6x-2y=4 $
\end{rlist}
\end{multicols}
\bcc{Συνθετα Συστηματα}
\begin{multicols}{2}
\Askhsh \textbf{Επίλυση συστήματος}\\
Να λυθούν τα παρακάτω γραμμικά συστήματα με οποιαδήποτε μέθοδο επίλυσης.
\begin{rlist}[leftmargin=5mm]
\item $ \ccases{2(x-1)+3(y+2)=11\\x+3-(4-y)=2} $
\item $ \ccases{3(x+y)-2y=1+x\\x-4y+2=3x+4} $
\end{rlist}
\Askhsh \textbf{Επίλυση συστήματος}\\
Να λυθούν τα παρακάτω γραμμικά συστήματα με οποιαδήποτε μέθοδο επίλυσης.
\begin{rlist}[leftmargin=5mm]
\item $ \ccases{4(x-3)+3(y+2)=1\\3x-5=2(3-y)+2} $
\item $ \ccases{5(x-y)+3(2x+y)=16\\15-x-y=3x+2y-2} $
\end{rlist}
\end{multicols}
\Askhsh \textbf{Επίλυση συστήματος}\\
Να λυθούν τα παρακάτω γραμμικά συστήματα με οποιαδήποτε μέθοδο επίλυσης.
\begin{multicols}{3}
\begin{rlist}[leftmargin=3mm]
\item $ \ccases{
2(x-1)-(y-2)=9\\
-(1-x)+3y=0} $
\item $ \ccases{
2(x-1)+3(y+2)=13\\
x-(2y-1)=2} $
\item $\ccases{
\;2(x-2)+3(y+1)=1\\
\;4x-(2-y)=2}$
\end{rlist}
\end{multicols}
\Askhsh \textbf{Επίλυση συστήματος}\\
Να λυθούν τα παρακάτω γραμμικά συστήματα με οποιαδήποτε μέθοδο.
\begin{multicols}{2}
\begin{enumerate}[label=\roman*.,itemsep=3mm]
\item $\ccases{
\;(2x-1)(y+1)-(x+4)(2y-3)=1\\
\;(1-x)(3y+1)+(x+2)(3y+4)=2}$
\item $\ccases{
\;\dfrac{x+1}{3}-\dfrac{3(y-2)}{4}=1\\[3mm]
\;\dfrac{x}{2}-\dfrac{2-y}{2}=x+y}$
\item $\ccases{
\;\dfrac{x-1}{2}+\dfrac{x-y}{3}=1-2x\\[3mm]
\;\dfrac{3y-x}{4}-\dfrac{3(y-2x)}{2}=\dfrac{1}{8}}$
\item $\ccases{
\;\dfrac{3x^2-x+1}{3}-\dfrac{2x^2-y}{2}=-2\\[3mm]
\;\dfrac{5y^2-x}{5}-\dfrac{y(3y-2)}{3}=\dfrac{1}{8}}$
\end{enumerate}\end{multicols}
\bcc{Γραμμικα Συστηματα {\boldmath$ 3\times3$}}
\Askhsh \textbf{Επίλυση συστήματος}\\
Να επιλυθούν τα παρακάτω $ 3\times3 $ γραμμικά συστήματα.
\begin{multicols}{3}
\begin{enumerate}[label=\roman*.,itemsep=0mm]
\item $\ccases{
3x-2y+z=6\\
x-3y-z=3\\
2x+y-4z=-3}$
\item $\ccases{
x-2y+z=4\\
x-y-z=2\\
2x+3y-3z=0}$
\item $\ccases{
x-2y+3z=6\\
2x-4y+6z=12\\
x+y-z=0}$
\end{enumerate}\end{multicols}
\begin{multicols}{2}
\Askhsh \textbf{Σύνθετη άσκηση}\\
Οι ορίζουσες $ D,D_x,D_y $ ενός $ 2\times2 $ γραμμικού συστήματος με μεταβλητές $ x,y $ ικανοποιούν τις παρακάτω εξισώσεις :
\[ \ccases{
\;D-2D_{x}-2D_{y}=-6\\
\;4D-3D_{x}-2D_{y}=-1\\
\;2D+3D_{x}-D_{y}=-4} \]
Να βρεθεί η λύση $ (x,y) $ του $ 2\times2 $ γραμμικού συστήματος.\\\\
\Askhsh \textbf{Εύρεση παραμέτρων}\\
Να βρεθεί η εξίσωση της παραβολής $ y=ax^2+\beta x+\gamma $ η οποία διέρχεται από τα σημεία
\begin{enumerate}[label=\roman*.,itemsep=0mm]
\item $ A(-2,1) $ , $ B(3,0) $ και $ \varGamma(1,-2) $
\item $ A(-1,1) $ , $ B(1,3) $ και $ \varGamma(0,-2) $
\item $ A(-4,3) $ , $ B(1,2) $ και $ \varGamma(0,1) $
\item $ A(-2,4) $ , $ B(3,9) $ και $ \varGamma(1,1) $
\end{enumerate}
\end{multicols}
\bcc{Προβληματα}
\begin{multicols}{2}
\Askhsh \textbf{Πρόβλημα}\\
Ένα ξενοδοχείο έχει 45 δωμάτια, άλλα δίκλινα και άλλα τρίκλινα. Συνολικά τα κρεβάτια είναι 110. Πόσα είναι τα δίκλινα και πόσα τα τρίκλινα δωμάτια;\\\\
\Askhsh \textbf{Πρόβλημα}\\
Ένας μαθητής έχει στο πορτοφόλι του 15 χαρτονομίσματα. Κάποια είναι των 5\officialeuro\; και κάποια των 10\officialeuro. Με τα χρήματα αυτά αγοράζει ένα κινητό τηλέφωνο αξίας 112\officialeuro\; και παίρνει ρέστα 8\officialeuro. Πόσα χαρτονομίσματα είναι των 5\officialeuro\; και πόσα των 10\officialeuro;\\\\
\Askhsh \textbf{Πρόβλημα}\\
Μια εταιρία κινητής τηλεφωνίας έχει τις εξής χρεώσεις : 0{,}07\officialeuro/sms και 0{,}09\officialeuro/1' ομιλίας. Ένας συνδρομητής, με μια κάρτα των 10\officialeuro\;ξόδεψε συνολικά 120 λεπτά και μηνύματα. Πόσα ήταν τα λεπτά ομιλίας και πόσα τα μηνύματα;\\\\
\Askhsh \textbf{Πρόβλημα}\\
Ένας πατέρας είναι 32 χρόνια μεγαλύτερος από το γιο του. Σε 8 χρόνια ο πατέρας θα έχει τα 3πλάσια χρόνια από το γιο του. Ποια είναι η ηλικία του πατέρα και του γιού;\\\\
\Askhsh \textbf{Πρόβλημα}\\
Σε ένα κουτί υπάρχουν κόκκινες και πράσινες μπάλες. Αν προσθέσουμε στο κουτί 3 κόκκινες μπάλες, οι πράσινες θα είναι διπλάσιες από τις κόκκινες ενώ αν προσθέσουμε 4 πράσινες τότε, κόκκινες και πράσινες θα είναι ίσες. Πόσες μπάλες από το κάθε χρώμα υπάρχουν;\\\\
\Askhsh \textbf{Πρόβλημα}\\
Σε μια φάρμα ζουν 80 σε πλήθος κότες και αγελάδες. Αν όλα τα ζώα έχουν συνολικά 260 πόδια να βρεθούν πόσες κότες και πόσες αγελάδες ζουν στη φάρμα.\\\\
\Askhsh \textbf{Πρόβλημα}\\
Σε ένα ορθογώνιο, το μήκος είναι διπλάσιο του πλάτους ενώ η περίμετρος είναι ίση με το μήκος αυξημένο κατά 12 μέτρα. Να βρεθούν οι πλευρές του ορθογωνίου.\\\\
\Askhsh \textbf{Πρόβλημα}\\
Η περίμετρος του τριγώνου του παρακάτω σχήματος είναι 28 εκατοστά. Να βρεθούν πραγματικοί οι αριθμοί $ x, y\in\mathbb{R} $ ώστε το τρίγωνο να είναι ισοσκελές.
\vspace{-5mm}
\begin{center}
\begin{tikzpicture}
\tkzDefPoint(3,0){C}
\tkzDefPoint(0,0){B}
\tkzDefPoint(1.5,1){A}
\tkzDrawPolygon[pl](A,B,C)
\tkzText(.3,.6){{\scriptsize $ 2x-1 $}}
\tkzText(3.1,.6){{\scriptsize $ 3x+2y-4 $}}
\tkzLabelPoint[above](A){$A$}
\tkzLabelPoint[left](B){$B$}
\tkzLabelPoint[right](C){$\varGamma$}
\tkzDrawPoints(A,B,C)
\end{tikzpicture}
\end{center}
\end{multicols}
\bcc{Παραμετρικα Συστηματα}
\begin{multicols}{2}
\Askhsh \textbf{Παραμετρικά συστήματα}\\
Δίνεται το παρακάτω παραμετρικό σύστημα με παράμετρο $ \lambda\in\mathbb{R} $.
\[ \ccases{
\lambda x+(\lambda+2)y=\lambda\\
x+ \lambda y=\lambda-1} \]
\begin{rlist}
\item Να βρεθούν οι ορίζουσες $ D,D_x,D_y $ του συστήματος με τη βοήθεια της παραμέτρου $ \lambda $.
\item Να εξεταστεί για ποιες τιμές της παραμέτρου το σύστημα έχει μοναδική λύση.
\item Για ποια τιμή της παραμέτρου το σύστημα είναι αόριστο και για ποια αδύνατο;
\end{rlist}
\Askhsh \textbf{Παραμετρικά συστήματα}\\
Να βρεθούν οι λύσεις των παρακάτω συστημάτων για κάθε τιμή της παραμέτρου $ \lambda\in\mathbb{R} $.
\begin{enumerate}[label=\roman*.,itemsep=0mm]
\item $\ccases{
2\lambda x+(\lambda+3)y=2\\
x+\lambda y=-1}$
\item $\ccases{
x+\lambda y=2-\lambda\\
\lambda x+y=\lambda}$
\end{enumerate}
\Askhsh \textbf{Παραμετρικά συστήματα}\\
Να βρεθούν οι λύσεις των παρακάτω συστημάτων για κάθε τιμή της παραμέτρου $ \lambda\in\mathbb{R} $.
\begin{enumerate}[label=\roman*.,itemsep=0mm]
\item $\ccases{
(\lambda^2+1)x-y=2\\
2\lambda x+y=4}$
\item $\ccases{
(\lambda+2)x-3y=\lambda+2\\
\lambda x+(\lambda-2)y=1}$
\item $\ccases{
\lambda^2x+4y=2\lambda\\
(\lambda-1) x+y=\lambda-1}$
\end{enumerate}
\Askhsh \textbf{Παραμετρικά συστήματα}\\
Να βρεθούν οι λύσεις των παρακάτω συστημάτων για κάθε τιμή της παραμέτρου $ \lambda\in\mathbb{R} $.
\begin{enumerate}[label=\roman*.,itemsep=0mm]
\item $\ccases{
\lambda x+(\lambda-3)y=-1\\
2x+(\lambda-3)y=1}$
\item $\ccases{
(\lambda+1) x-3y=-1\\
x+(\lambda-3)y=1}$
\end{enumerate}
\end{multicols}
\section{Μη γραμμικά Συστήματα}
\orismoi
\Orismos{Μη γραμμικό σύστημα}
Ένα σύστημα εξισώσεων θα ονομάζεται μη γραμμικό όταν τουλάχιστον μια εξίσωσή του δεν αποτελεί γραμμική εξίσωση.\\\\
\begin{Methodos}[Αντικατάσταση]{5cm}
Μια γενική μέθοδος επίλυσης μη γραμμικών συστημάτων είναι η μέθοδος της αντικατάστασης που συναντήσαμε και στα γραμμικά συστήματα. Έχει ως εξής
\begin{bhma}
\item \textbf{Επιλογή εξίσωσης}\\
Επιλέγουμε εκείνη την εξίσωση του συστήματος η οποία είναι στην πιο απλή μορφή και είναι εύκολο να λυθεί ως προς κάποιον άγνωστο. Λύνουμε ως προς αυτόν τον άγνωστο.
\item \textbf{Αντικατάσταση}\\
Τον άγνωστο αυτό τον αντικαθιστούμε στην άλλη εξίσωση του συστήματος και υπολογίζουμε και τη δεύτερη μεταβλητή. Αν οι εξισώσεις του συστήματος είναι περισσότερες από δύο τότε αντικαθιστούμε τον άγνωστο στις υπόλοιπες εξισώσεις του συστήματος και καταλήγουμε σε ένα σύστημα μιας τάξης μικρότερης. Επαναλαμβάνουμε τη διαδικασία όσες φορές χρειαστεί.
\end{bhma}
\end{Methodos}
\Paradeigma{Λύση μη γραμμικού συστήματοσ}
\textbf{Να λυθεί το παρακάτω μη γραμμικό σύστημα}
{\boldmath\[ \systeme[xy]{x^2+y^2=1,2x-y=1} \]}
\lysh\\
Παρατηρούμε η 2\tss{η} εξίσωση του συστήματος είναι γραμμική και κατά συνέπεια είναι εύκολο να λυθεί ως προς τη μεταβλητή $ y $. Προκύπτει
\begin{equation}
\systeme[xy]{x^2+y^2=1,2x-y=1}\Rightarrow y=2x-1
\end{equation}
Με αντικατάσταση της σχέσης (1) στην πρώτη εξίσωση παίρνουμε
\begin{gather*} x^2+y^2=1\Rightarrow x^2+\left(2x-1\right)^2=1\Rightarrow x^2+4x^2-4x+1=1\Rightarrow 5x^2-4x=0\\
\Rightarrow x(5x-4)=0\Rightarrow\ccases{x=0\\5x-4=0\Rightarrow x=\frac{4}{5}}
\end{gather*}
Η παραπάνω εξίσωση μας έδωσε δύο λύσεις ως τιμές της μεταβλητής $ x $. Οπότε για κάθε τιμή αυτή θα βρεθεί και η αντίστοιχη τιμή της μεταβλητής $ y $ και κατά συνέπεια θα προκύψουν δύο λύσεις του συστήματος. Αντικαθιστώντας στη σχέση (1) έχουμε
\begin{rlist}
\item Για $ x=0 $ έχουμε $ y=2x-1=2\cdot0-1=-1 $.
\item Για $ x=\frac{4}{5} $ έχουμε $ y=2x-1=2\cdot\frac{4}{5}-1=\frac{3}{5} $.
\end{rlist}
Οι λύσεις λοιπόν του συστήματος θα είναι $ (x,y)=(0,-1) $ και $ (x,y)=\left(\frac{4}{5},\frac{3}{5}\right)  $.
\begin{Methodos}[Ανάθεση]{7cm}
Σε ορισμένα μη γραμμικά συστήματα οι μεταβλητές των εξισώσεων βρίσκονται στην ίδια μορφή ή και μέσα σε ίδιες αλγεβρικές παραστάσεις. Σ' αυτές τις περιπτώσεις το μη γραμμικό σύστημα μπορεί να μετατραπεί σε γραμμικό ως εξής :
\begin{bhma}
\item \textbf{Ανάθεση}\\
Αναγνωρίζουμε την κοινή σύνθετη αλγεβρική παράσταση στην οποία βρίσκεται μέσα κάθε μεταβλητή σε όλες τις εξισώσεις και την θέτουμε ίση με μια νέα μεταβλητή.
\item \textbf{Αντικατάσταση}\\
Αντικαθιστούμε σε όλες τις εξισώσεις τις παραστάσεις αυτές με τις νέες μεταβλητές που ορίσαμε οπότε και καταλήγουμε σε ένα γραμμικό σύστημα με νέους άγνωστους.
\item \textbf{Λύση γραμμικού συστήματος}\\
Λύνουμε το γραμμικό σύστημα με οποιαδήποτε μέθοδο προτιμάμε.
\item \textbf{Αρχικοί άγνωστοι}\\
Αφού λυθεί το γραμμικό σύστημα και βρεθούν οι νέες μεταβλητές, αντικαθιστούμε τις τιμές στις σχέσεις του \textbf{1\tss{ου} Βήματος} οπότε προκύπτουν εξισώσεις με τις αρχικές μεταβλητές του συστήματος, τις οποίες είτε λύνουμε αν περιέχουν μόνο μια μεταβλητή είτε συνδυάζουμε αν περιέχουν περισσότερες για να προκύψει σύστημα, γραμμικό ή μη γραμμικό. Στην περίπτωση συστήματος εργαζόμαστε χρησιμοποιώντας τις προηγούμενες μεθόδους.
\end{bhma}
\end{Methodos}
\Paradeigma{Λύση συστήματοσ με ανάθεση}
\textbf{Να λυθεί το παρακάτω σύστημα}
{\boldmath\[ \systeme{\dfrac{3}{x}+\dfrac{4}{y}=-1,\dfrac{2}{x}-\dfrac{5}{y}=7} \]}
\lysh\\
Παρατηρούμε ότι και στις δύο εξισώσεις οι μεταβλητές βρίσκονται μέσα στις ίδιες αλγεβρικές παραστάσεις. Αυτές είναι οι $ \frac{1}{x} $ και $ \frac{1}{y} $. Θέτουμε λοιπόν
\begin{equation}
\lambda=\frac{1}{x}\;\textrm{ και }\,\kappa=\frac{1}{y}
\end{equation}
Το αρχικό σύστημα μετατρέπεται σε γραμμικό ως εξής
\[ \systeme{\dfrac{3}{x}+\dfrac{4}{y}=-1,\dfrac{2}{x}-\dfrac{5}{y}=7}\Rightarrow \systeme{3\dfrac{1}{x}+4\dfrac{1}{y}=-1,2\dfrac{1}{x}-5\dfrac{1}{y}=7}\Rightarrow\systeme[\lambda\kappa]{3\lambda+4\kappa=1,2\lambda-5\kappa=7} \]
Με τις γνωστές μεθόδους υπολογίζουμε τη λύση του γραμμικού συστήματος η οποία είναι $ (\lambda,\kappa)=(1,-1) $. Με τις τιμές αυτές των νέων μεταβλητών θα υπολογίσουμε τις τιμές των αρχικών μεταβλητών από τη σχέση (2). Έχουμε λοιπόν :
\begin{gather*}
\lambda=\frac{1}{x}\Rightarrow 1=\frac{1}{x}\Rightarrow x=1\;\;\textrm{και}\;\;
\kappa=\frac{1}{y}\Rightarrow -1=\frac{1}{y}\Rightarrow y=-1
\end{gather*}
Η λύση λοιπόν του αρχικού συστήματος θα είναι $ (x,y)=(1,-1) $.\\\\
\Paradeigma{Λύση συστήματοσ με ανάθεση}
\textbf{Να λυθεί το παρακάτω σύστημα}
{\boldmath\[ \ccases{\dfrac{2}{x-3y}-\dfrac{3}{2x+y}=-2\\[3mm]-\dfrac{4}{x-3y}+\dfrac{9}{2x+y}=5} \]}
Παρατηρούμε ότι οι παραστάσεις $ \frac{1}{x-3y} $ και $ \frac{1}{2x+y} $ εμφανίζονται και στις δύο εξισώσεις. Άρα θέτουμε
\begin{equation}
z=\frac{1}{x-3y}\;\textrm{ και }\;t=\frac{1}{2x+y}
\end{equation}
Το αρχικό σύστημα θα μετατραπεί ως εξής
\[ \ccases{\dfrac{2}{x-3y}-\dfrac{3}{2x+y}=-2\\[3mm]-\dfrac{4}{x-3y}+\dfrac{9}{2x+y}=5}\Rightarrow\systeme[zt]{2z-3t=-2,-4z+9t=5} \]
Λύνοντας το γραμμικό σύστημα που προέκυψε προκύπτει η λύση για τις μεταβλητές $ z,t $ η οποία θα είναι $ (z,t)=\left(-\frac{1}{2},\frac{1}{3} \right)  $. Με αντικατάσταση των τιμών αυτών στις σχέσεις (3) οι σχέσεις αυτές μας δίνουν ένα νέο γραμμικό σύστημα με τις αρχικές μεταβλητές.
\[ \ccases{z=\frac{1}{x-3y}\\t=\frac{1}{2x+y}}\Rightarrow \ccases{\frac{1}{x-3y}=-\frac{1}{2}\\\frac{1}{2x+y}=\frac{1}{3}}\Rightarrow \systeme{x-3y=-2,2x+y=3} \]
Με τις γνωστές μεθόδους επίλυσης γραμμικών συστημάτων προκύπτει η λύση του παραπάνω συστήματος η οποία είναι και λύση του αρχικού συστήματος $ (x,y)=(1,1) $.
\newpage
\noindent
\Alyta
\bcc{ΜΗ ΓΡΑΜΜΙΚΑ ΣΥΣΤΗΜΑΤΑ}
\begin{multicols}{2}
\Askhsh \textbf{Μη γραμμικά συστήματα}\\
Να λυθούν τα παρακάτω μη γραμμικά συστήματα συστήματα
\begin{multicols}{2}
\begin{enumerate}[label=\roman*.,itemsep=1mm]
\item $\ccases{
x^2+y^2=1\\
x+y=0}$
\item $\ccases{
x^2-y^2=1\\
2x-y=0}$
\item $\ccases{
x^2+y^2=4\\
xy=1}$
\item $\ccases{
x^2-y^2=16\\
\dfrac{1}{x}+\dfrac{1}{y}=1}$
\end{enumerate}\end{multicols}
\Askhsh \textbf{Μη γραμμικά συστήματα}\\
Να λυθούν τα παρακάτω μη γραμμικά συστήματα συστήματα
\begin{multicols}{2}
\begin{enumerate}[label=\roman*.,itemsep=1mm]
\item $\ccases{
x^2+2y=4\\
x-y=-5}$
\item $\ccases{
x+y^2=2y-1\\
x^2+y=1}$
\item $\ccases{
x^4-y^4=1\\
x^2-y^2=1}$
\item $\ccases{
x^3+y^3=1\\
x+y=1}$
\end{enumerate}\end{multicols}
\end{multicols}
\Askhsh \textbf{Μη γραμμικά συστήματα}\\
Να λυθούν τα παρακάτω μη γραμμικά συστήματα συστήματα.
\begin{multicols}{2}
\begin{enumerate}[label=\roman*.,itemsep=1mm]
\item $\ccases{
\;(x-y)^2+(x+y)^2=13\\
\;(x-y)-2(x+y)=-4}$
\item $\ccases{
\;2x-3y=2\\
\;|x-y|=1}$
\item $\ccases{
\;2\sqrt{x}+\sqrt{y}=5\\
\;x+4y=6}$
\item $\ccases{
\;\dfrac{1}{x-y^2}+\dfrac{1}{y-x^2}=2\\
\;x^2+y^2-x-y=4}$
\end{enumerate}\end{multicols}
\Askhsh \textbf{Μη γραμμικά συστήματα}\\
Να λυθούν τα παρακάτω μη γραμμικά συστήματα συστήματα
\begin{multicols}{3}
\begin{enumerate}[label=\roman*.,itemsep=3mm]
\item $\ccases{
x^2+y^2+z^2=1\\
x+y+z=0\\
x-y+z=0}$
\item $\ccases{
xyz=9\\
x-y=2\\
xy-z=0}$
\item $\ccases{
\dfrac{1}{x}+\dfrac{1}{y}+\dfrac{1}{z}=1\\
x+y+z=1\\
x-z=1}$
\end{enumerate}\end{multicols}
\Askhsh \textbf{Μη γραμμικά συστήματα}\\
Να λυθούν τα παρακάτω μη γραμμικά συστήματα συστήματα
\begin{multicols}{3}
\begin{enumerate}[label=\roman*.,itemsep=3mm]
\item $\ccases{
x^3+y^3=3z^3\\
x^2y+xy^2=2z^3\\
x+2y+z=-1}$
\item $\ccases{
x^2+y^2+z^2=98\\
\sqrt{x}-\sqrt{y}+\sqrt{z}=0\\
x+2y+z=11}$
\item $\ccases{
xy=2\\
yz=3\\
xz=8}$
\item $\ccases{
y(x^2-y^2)=z\\
-y(x^2+y^2)=z\\
(1-x)^2=-(y-z-1)^2}$
\item $\ccases{
x^3+y^3=4\\
y^3+z^3=2\\
x^3+z^3=4}$
\item $\ccases{
x+y^2=2y-1\\
x^2+y=1}$
\end{enumerate}\end{multicols}
\Askhsh \textbf{Επίλυση με ανάθεση}\\
Να λυθούν τα παρακάτω μη γραμμικά συστήματα
\begin{multicols}{2}
\begin{enumerate}[label=\roman*.,itemsep=3mm]
\item $\ccases{
\;|x|-|y|=3\\
\;2|x|+3|y|=11}$
\item $\ccases{
\;|x-1|+2|y+2|=7\\
\;3|x-1|-4|y+2|=1}$
\item $\ccases{
\;x^2+2y^3=0\\
\;3x^2+5y^3=11}$
\item $\ccases{
\;2(x^2+3x-3)+3(y^2-5y+7)=5\\
\;-(x^2+3x-3)+2(y^2-5y+7)=1}$
\item $\ccases{
\;\sqrt{x}-3\sqrt{y}=-1\\
\;2\sqrt{x}+9\sqrt{y}=13}$
\item $\ccases{
\;\dfrac{1}{x}+\dfrac{1}{y}=7\\
\;\dfrac{2}{x}-\dfrac{1}{2y}=4}$
\item $\ccases{
\;2\textrm{ημ}x+\textrm{συν}y=2\\
\;3\textrm{ημ}x-4\textrm{συν}y=\dfrac{5}{2}} $
\end{enumerate}\end{multicols}
\Askhsh \textbf{Κοινά σημεία καμπυλών}\\
Σε καθένα από τα παρακάτω ερωτήματα να βρεθούν τα κοινά σημεία των καμπυλών.
\begin{enumerate}[label=\roman*.,itemsep=0mm,leftmargin=4mm]
\item του κύκλου $ x^2+y^2=1 $ και της ευθείας $ 2x+3y=1 $
\item του κύκλου $ x^2+y^2=4 $ και της παραβολής $ 3x-y^2=0 $
\item της υπερβολής $ xy=4 $ και της ευθείας $ 3x-y=-1 $
\item των κύκλων $ x^2+y^2=12 $ και $ x^2+y^2=9 $
\end{enumerate}
\bcc{ΠΡΟΒΛΗΜΑΤΑ}
\begin{multicols*}{2}
\Askhsh \textbf{Πρόβλημα}\\
Ένα ορθογώνιο παραλληλόγραμμο έχει περίμετρο $ 24m $ και εμβαδόν $ 32m^2 $. Να βρεθεί το μήκος και το πλάτος του ορθογωνίου.\\\\
\Askhsh \textbf{Πρόβλημα}\\
Δύο τετράγωνα με πλευρές $ x-2 $ και $ 2y+3 $ αντίστοιχα έχουν συνολικό εμβαδόν 90τ.μέτρα. Αν ξέρουμε ότι η περίμετρος του 2\tss{ου} είναι 3πλάσια από την περίμετρο του 1\tss{ου} τότε να βρεθούν οι αριθμοί $ x, y $.\\\\
\Askhsh \textbf{Πρόβλημα}\\
Η διαγώνιος μιας τηλεόρασης είναι $ 42'' $ και γνωρίζουμε επίσης ότι οι πλευρές έχουν αναλογία 16:9 (HD Video Standard). Να βρεθούν οι διαστάσεις της τηλεόρασης.\\
{\footnotesize \textit{Υπόδειξη} : Το μέγεθος μιας τηλεόρασης δίνεται από το μήκος της διαγωνίου της οθόνης δοσμένο σε ίντσες. Μια ίντσα $ 1''=2{,}54cm $}\\\\
\Askhsh \textbf{Πρόβλημα}\\
Αν $ A\varDelta=12 $ είναι το ύψος του ορθογωνίου τριγώνου στην υποτείνουσα και $ B\varDelta=x, \varGamma\varDelta=y $ να βρεθούν οι θετικοί πραγματικοί οι αριθμοί $ x, y\in\mathbb{R}^+ $ ώστε να ισχύει $ A\varDelta^2=B\varDelta\cdot\varGamma\varDelta $.
\begin{center}
\begin{tikzpicture}
\tkzDefPoint(3,0){B}
\tkzDefPoint(0,0){A}
\tkzDefPoint(0,1){C}
\tkzDefPoint(0.33,.9){D}
\tkzMarkRightAngle(B,A,C)
\tkzMarkRightAngle[size=.2](A,D,B)
\tkzDrawPolygon(A,B,C)
\tkzDrawSegment(A,D)
\tkzLabelPoint[left](A){{\scriptsize A}}
\tkzLabelPoint[right](B){{\scriptsize B}}
\tkzLabelPoint[above](C){{\scriptsize $ \varGamma $}}
\tkzLabelPoint[above right](D){{\scriptsize $ \varDelta $}}
\tkzText(-.2,.5){{\scriptsize $ 15 $}}
\tkzText(1.5,-.2){{\scriptsize $ 20 $}}
\end{tikzpicture}
\end{center}
\Askhsh \textbf{Πρόβλημα}\\
Δύο αυτοκίνητα Α και Β κινούνται με μέση συνολική ταχύτητα $ 150km/h $. Αν γνωρίζουμε ότι κάθε αυτοκίνητο διένυσε απόσταση $ 56km/h $ και ταξίδευαν συνολικά για 1μιση ώρα τότε να βρεθεί ο χρόνος που ταξίδευε το κάθε αμάξι.\\\\
\end{multicols*}
\end{document}








%\documentclass[twoside,10pt]{book}
\usepackage[amsbb,mtpfrak,zswash,mtpcal]{mtpro2}
\usepackage[no-math,cm-default]{fontspec}
\usepackage{xunicode}
\usepackage{xltxtra}
\usepackage{xgreek}
\defaultfontfeatures{Mapping=tex-text,Scale=MatchLowercase}
\setmainfont[Mapping=tex-text,Numbers=Lining,Scale=1.0,BoldFont={Minion Pro Bold}]{Minion Pro}
\defaultfontfeatures{Ligatures=TeX}
\font\kefalaio="Minion Pro Bold" at 36pt
\font\ArKef="Minion Pro Bold Italic" at 72pt
\font\OnKef="Times New Roman" at 20pt
\font\OnPar="Minion Pro Bold" at 18pt
\newfontfamily\scfont{GFS Artemisia}
\usepackage[inner=2.00cm, outer=1.50cm, top=3.00cm, bottom=2.00cm,paperwidth=17cm,paperheight=24cm]{geometry}
\usepackage{amsmath}
\usepackage[amsbb,mtpfrak,zswash,mtpcal]{mtpro2}
\usepackage{makeidx}
\usepackage{longtable,xcolor,varwidth}
\usepackage{float}
\usepackage{subfig}
\def\xrwma{cyan!70!black}
\def\xrwmath{cyan}
\usepackage{etoolbox}
\makeatletter
\newif\ifLT@nocaption
\preto\longtable{\LT@nocaptiontrue}
\appto\endlongtable{%
\ifLT@nocaption
\addtocounter{table}{\m@ne}%
\fi}
\preto\LT@caption{%
\noalign{\global\LT@nocaptionfalse}}
\makeatother
\makeindex
\usepackage{tikz,pgfplots}
\usepackage{tkz-euclide,tkz-fct}
\usetikzlibrary{fadings}
\usepackage{wrap-rl}
\usetkzobj{all}
\usepackage{calc}
\usepackage{cleveref}
\usepackage[colorlinks=false, pdfborder={0 0 0}]{hyperref}
\usepackage[framemethod=TikZ]{mdframed}
\definecolor{steelblue}{cmyk}{.7,.278,0,.294}
\definecolor{doc}{cmyk}{1,0.455,0,0.569}
\definecolor{olivedrab}{cmyk}{0.25,0,0.75,0.44}
\usepackage{capt-of}
\usepackage{titletoc}
\usepackage[explicit]{titlesec}
\usepackage{graphicx}
\usepackage{multicol}
\usepackage{multirow}
\usepackage{enumitem}
\usepackage{tabularx}
\usepackage[decimalsymbol=comma]{siunitx}
\tikzset{>=latex}
\makeatletter
\pretocmd{\@part}{\gdef\parttitle{#1}}{}{}
\pretocmd{\@spart}{\gdef\parttitle{#1}}{}{}
\makeatother
\usepackage[titletoc]{appendix}
\usepackage{fancyhdr}
\pagestyle{fancy}
\fancyheadoffset{0cm}
\renewcommand{\headrulewidth}{\iftopfloat{0pt}{.5pt}}
\renewcommand{\chaptermark}[1]{\markboth{#1}{}}
\renewcommand{\sectionmark}[1]{\markright{\it\thesection\ #1}}
\fancyhf{}
\fancyhead[LE]{\thepage\ $\cdot$\ \scfont\scshape\nouppercase{\leftmark}}
\fancyhead[RO]{\nouppercase{\rightmark} $\cdot$\ \thepage}
\fancypagestyle{plain}{%
\fancyhead{} %
\renewcommand{\headrulewidth}{0pt}}

\newcounter{thewrhma}[chapter]
\renewcommand{\thethewrhma}{\thechapter.\arabic{thewrhma}} 
\newcommand{\Thewrhma}[1]{\refstepcounter{thewrhma}{\textbf{\textcolor{\xrwmath}{{\large Θεώρημα\hspace{2mm}\thethewrhma\;}:\;}\hspace{1mm}}} \MakeUppercase{\textbf{#1}}\\}{}

\newcounter{porisma}[chapter]
\renewcommand{\theporisma}{\thechapter.\arabic{porisma}}\newcommand{\Porisma}[1]{\refstepcounter{porisma}\textcolor{black}{\textbf{ΠΟΡΙΣΜΑ\hspace{2mm}\theporisma\hspace{1mm} \MakeUppercase{#1}}}\\}{}

\newcounter{protasi}[chapter]
\renewcommand{\theprotasi}{\thechapter.\arabic{protasi}}\newcommand{\Protasi}[1]{\refstepcounter{protasi}\textcolor{black}{\textbf{ΠΡΟΤΑΣΗ\hspace{2mm}\theprotasi\hspace{1mm} \MakeUppercase{#1}}}\\}{}


\newcounter{orismos}[chapter]
\renewcommand{\theorismos}{\arabic{orismos}}   
\newcommand{\Orismos}[1]{\refstepcounter{orismos}{\textbf{\textbf{\textcolor{\xrwma}{{\large Ορισμός\hspace{2mm}\theorismos\;}:\;}}}}\hspace{1mm} \MakeUppercase{\textbf{#1}\\}}{}
\usepackage{venndiagram,mathimatika}
%-------- ΣΤΥΛ ΚΕΦΑΛΑΙΟΥ ---------
\newcommand*\chapterlabel{}
\newcommand{\fancychapter}{%
\titleformat{\chapter}
{
\normalfont\Huge}
{\gdef\chapterlabel{\thechapter\ }}{0pt}
{\begin{tikzpicture}[remember picture,overlay]
\node[yshift=-7cm] at (current page.north west)
{\begin{tikzpicture}[remember picture, overlay]
%\node[inner sep=0pt] at ($(current page.north) +			(0cm,-1.38in)$) {\includegraphics[width=17cm]{Kefalaio}};
\node[anchor=west,xshift=.1\paperwidth,yshift=.14\paperheight,rectangle]
{{\color{white}\fontsize{30}{20}\textbf{\textcolor{black}{\contour{white}{ΚΕΦΑΛΑΙΟ}}}}};
\node[anchor=west,xshift=.09\paperwidth,yshift=.08\paperheight,rectangle] {\fontsize{24}{20} {\color{black}{{\textcolor{black}{\contour{white}{\sc##1}}}}}};
%\fill[fill=black] (12.2,2) rectangle (14.8,4.7);
\node[anchor=west,xshift=.74\paperwidth,yshift=.11\paperheight,rectangle]
{{\color{white}\fontsize{80}{20}\textbf{\textit{\textcolor{white}{\contour{black}{\thechapter}}}}}};
\end{tikzpicture}
};
\end{tikzpicture}
}
\titlespacing*{\chapter}{0pt}{20pt}{30pt}
}
%------------------------------------------------


\usepackage[outline]{contour}
\newcommand{\regularchapter}{%
\titleformat{\chapter}[display]
{\normalfont\huge\bfseries}{\chaptertitlename\ \thechapter}{20pt}{\Huge##1}
\titlespacing*{\chapter}
{0pt}{-20pt}{40pt}
}

\apptocmd{\mainmatter}{\fancychapter}{}{}
\apptocmd{\backmatter}{\regularchapter}{}{}
\apptocmd{\frontmatter}{\regularchapter}{}{}

\titlespacing*{\section}
{0pt}{30pt}{0pt}
\usepackage{booktabs}
\usepackage{hhline}
\DeclareRobustCommand{\perthousand}{%
\ifmmode
\text{\textperthousand}%
\else
\textperthousand
\fi}


\contentsmargin{0cm}
\titlecontents{part}[-1pc]
{\addvspace{10pt}%
\bf\Large ΜΕΡΟΣ\quad }%
{}
{}
{\;\dotfill}%
%------------------------------------------
\titlecontents{chapter}[0pc]
{\addvspace{30pt}%
\begin{tikzpicture}[remember picture, overlay]%
\draw[fill=black,draw=black] (-.3,.5) rectangle (3.7,1.1); %
\pgftext[left,x=0cm,y=0.75cm]{\color{white}\sc\Large\bfseries Κεφάλαιο\ \thecontentslabel};%
\end{tikzpicture}\large\sc}%
{}
{}
{\hspace*{-2.3em}\hfill\normalsize Σελίδα \thecontentspage}%
\titlecontents{section}[2.4pc]
{\addvspace{1pt}}
{\contentslabel[\thecontentslabel]{2pc}}
{}
{\;\dotfill\;\small \thecontentspage}
[]
\titlecontents*{subsection}[4pc]
{\addvspace{-1pt}\small}
{}
{}
{\ --- \small\thecontentspage}
[ \textbullet\ ][]

\makeatletter
\renewcommand{\tableofcontents}{%
\chapter*{%
\vspace*{-20\p@}%
\begin{tikzpicture}[remember picture, overlay]%
\pgftext[right,x=12cm,y=0.2cm]{\Huge\sc\bfseries \contentsname};%
\draw[fill=black,draw=black] (9.5,-.75) rectangle (12.5,1);%
\clip (9.5,-.75) rectangle (15,1);
\pgftext[right,x=12cm,y=0.2cm]{\color{white}\Huge\bfseries \contentsname};%
\end{tikzpicture}}%
\@starttoc{toc}}
\makeatother

\usepackage[contents={},scale=1,opacity=1,color=black,angle=0]{background}

\newcommand\blfootnote[1]{%
\begingroup
\renewcommand\thefootnote{}\footnote{#1}%
\addtocounter{footnote}{-1}%
\endgroup
}
\usepackage{epstopdf}
\epstopdfsetup{update}
\usepackage{textcomp}

\titleformat{\section}
{\normalfont\Large\bf}%
{}{0em}%
{{\color{black}\titlerule[0pt]}\vskip-.2\baselineskip{\parbox[t]{\dimexpr\textwidth-2\fboxsep\relax}{\raggedright\strut\itshape{\LARGE{\thesection~#1}}\strut}}}[\vskip 0\baselineskip{\color{black}\titlerule[1pt]}]
\titlespacing*{\section}{0pt}{0pt}{30pt}

\newcommand{\methodologia}{\begin{center}
{\large \textbf{ΜΕΘΟΔΟΛΟΓΙΑ}}\\\vspace{-2mm}
\begin{tikzpicture}
\shade[left color=white, right color=black,] (-3cm,0) rectangle (0,.2mm);
\shade[left color=black, right color=white,] (0,0) rectangle (3cm,.2mm);   
\end{tikzpicture}
\end{center}}

\newcommand{\orismoi}{\begin{center}
\vspace{-3mm}{\large \textbf{\textcolor{\xrwma}{ΟΡΙΣΜΟΙ}}}\\\vspace{-2mm}
\begin{tikzpicture}
\shade[left color=white, right color=cyan!80!black,] (-3cm,0) rectangle (0,.2mm);
\shade[left color=cyan!80!black, right color=white,] (0,0) rectangle (3cm,.2mm);   
\end{tikzpicture}
\end{center}}
\newcommand{\thewrhmata}{\begin{center}
{\large \textbf{\textcolor{\xrwmath}{ΘΕΩΡΗΜΑΤΑ - ΠΟΡΙΣΜΑΤΑ - ΠΡΟΤΑΣΕΙΣ\\ΚΡΙΤΗΡΙΑ - ΙΔΙΟΤΗΤΕΣ}}}\\\vspace{-2mm}
\begin{tikzpicture}
\shade[left color=white, right color=\xrwmath,] (-5cm,0) rectangle (0,.2mm);
\shade[left color=\xrwmath, right color=white,] (0,0) rectangle (5cm,.2mm);   
\end{tikzpicture}
\end{center}}
\usepackage[labelfont={footnotesize,it,bf},font={footnotesize}]{caption}

%-------- ΠΙΝΑΚΕΣ ---------
\usepackage{booktabs}
%----------------------
%----- ΥΠΟΛΟΓΙΣΤΗΣ ----------
%\usepackage{calculator}
%----------------------------

%----- ΟΡΙΖΟΝΤΙΑ ΛΙΣΤΑ ------
\usepackage{xparse}
\newcounter{answers}
\renewcommand\theanswers{\arabic{answers}}
\ExplSyntaxOn
\NewDocumentCommand{\results}{m}
{
\seq_set_split:Nnn \l_results_a_seq {,}{#1}
\par\nobreak\noindent\setcounter{answers}{0}
\seq_map_inline:Nn \l_results_a_seq
{
\makebox[.18\linewidth][l]{\stepcounter{answers}\theanswers.~##1}\hfill
}
\par
}
\seq_new:N \l_results_a_seq
\ExplSyntaxOff
%----------------------------
%------ ΜΗΚΟΣ ΓΡΑΜΜΗΣ ΚΛΑΣΜΑΤΟΣ ---------
\DeclareRobustCommand{\frac}[3][0pt]{%
{\begingroup\hspace{#1}#2\hspace{#1}\endgroup\over\hspace{#1}#3\hspace{#1}}}
%----------------------------------------
\usepackage{microtype}
\usepackage{float}

\usepackage{caption}

%---- ΟΡΙΖΟΝΤΙΟ - ΚΑΤΑΚΟΡΥΦΟ - ΠΛΑΓΙΟ ΑΓΚΙΣΤΡΟ ------
\newcommand{\orag}[3]{\node at (#1)
{$ \overcbrace{\rule{#2mm}{0mm}}^{{\scriptsize #3}} $};}

\newcommand{\kag}[3]{\node at (#1)
{$ \undercbrace{\rule{#2mm}{0mm}}_{{\scriptsize #3}} $};}

\newcommand{\Pag}[4]{\node[rotate=#1] at (#2)
{$ \overcbrace{\rule{#3mm}{0mm}}^{{\rotatebox{-#1}{\scriptsize$#4$}}}$};}
%-----------------------------------------
\tikzstyle{pl}=[line width=0.3mm]
\tikzstyle{plm}=[line width=0.4mm]
%------- ΣΤΥΛ ΠΑΡΑΔΕΙΓΜΑΤΟΣ -------
\newcounter{paradeigma}[section]
\renewcommand{\theparadeigma}{\bf\thechapter.\arabic{paradeigma}}   
\newcommand{\Paradeigma}[1]{\refstepcounter{paradeigma}\textcolor{cyan}{\textbf{{\large Παράδειγμα\hspace{2mm}\theparadeigma\;:\;}\hspace{1mm}}} \MakeUppercase{\textbf{#1}}\\}{}
%-----------------------------------

%------- ΣΤΥΛ ΛΥΣΗΣ ------------------
\newcommand{\lysh}{{\textbf{ΛΥΣΗ}}}
%------------------------------------

%------ ΛΥΜΕΝΑ ΠΑΡΑΔΕΙΓΜΑΤΑ ΤΙΤΛΟΣ ---------
\newcommand{\Lymena}{\begin{center}
\begin{tikzpicture}
\path[left color=cyan!70!black,right color=cyan!80!black,middle color=cyan!80!white] (-7cm,-.6cm) rectangle (6.5cm,.6cm);
\node at (-.25cm,0) {\Large \textcolor{white}{\textbf{ΛΥΜΕΝΑ ΠΑΡΑΔΕΙΓΜΑΤΑ}}};  
\end{tikzpicture}
\end{center}}
%--------------------------------------

%--------- ΑΛΥΤΕΣ ΑΣΚΗΣΕΙΣ ΤΙΤΛΟΣ ----------
\newcommand{\Alyta}{\begin{center}
\begin{tikzpicture}
\path[left color=cyan!70!black,right color=cyan!80!black,middle color=cyan!80!white] (-7cm,-.6cm) rectangle (6.5cm,.6cm);
\node at (-.25cm,0) {\Large \textcolor{white}{\textbf{ΑΣΚΗΣΕΙΣ - ΠΡΟΒΛΗΜΑΤΑ}}};  
\end{tikzpicture}
\end{center}}
%--------------------------------------------
\usetikzlibrary{shadows,calc}
\usepackage{tcolorbox}
\tcbuselibrary{skins,theorems,breakable}
%---------- ΜΕΘΟΔΟΣ --------------
\newcounter{Methodos}[chapter]
\renewcommand{\theMethodos}{\thechapter.\arabic{Methodos}}
\newenvironment{Methodos}[2][\linewidth]
{\refstepcounter{Methodos}
\begin{tcolorbox}[breakable,
enhanced standard,
boxrule=0.7pt,titlerule=-.2pt,drop fuzzy shadow southeast=black!50,
width=\linewidth,
title style={color=white},
overlay unbroken and first={
\path[left color=cyan!70!black,right color=cyan,draw=black]
([yshift=-\pgflinewidth]frame.north west) to ([yshift=-5pt]title.south west)[rounded corners=2pt] -- ([xshift=-#2-15pt,yshift=-5pt]title.south east) to[rounded corners=2pt] ([xshift=-#2,yshift=-\pgflinewidth]frame.north east) -- cycle;
},
fonttitle=\bfseries,
before=\par\medskip\noindent,
after=\par\medskip,
toptitle=3pt,
top=11pt,topsep at break=-5pt,
colback=white,title={\large Μέθοδος \theMethodos} : {\textcolor{black}{\MakeUppercase{#1}}}]}
{\end{tcolorbox}}
%------------------------------------------
%---------- ΛΙΣΤΕΣ ----------------------
\newlist{bhma}{enumerate}{3}
\setlist[bhma]{label=\bf\textit{\arabic*\textsuperscript{o}\;Βήμα :},leftmargin=0cm,itemindent=1.5cm,ref=\bf{\arabic*\textsuperscript{o}\;Βήμα}}
\newlist{rlist}{enumerate}{3}
\setlist[rlist]{itemsep=0mm,label=\roman*.}


%----ΣΤΥΛ ΑΣΚΗΣΗΣ ----------
\newcounter{askhsh}[chapter]
\renewcommand{\theaskhsh}{\bf{{\large{\thechapter}}.\arabic{askhsh}}}   
\newcommand{\Askhsh}{\refstepcounter{askhsh}\textcolor{\xrwma}{{\theaskhsh}\hspace{1mm}}}{}
%---------------------------

\newlist{brlist}{enumerate}{3}
\setlist[brlist]{itemsep=0mm,label=\bf\roman*.}
\newlist{tropos}{enumerate}{3}
\setlist[tropos]{label=\bf\textit{\arabic*\textsuperscript{oς}\;Τρόπος :},leftmargin=0cm,itemindent=2.3cm,ref=\bf{\arabic*\textsuperscript{oς}\;Τρόπος}}
% Αν μπει το bhma μεσα σε tropo τότε
%\begin{bhma}[leftmargin=.7cm]
\newcommand{\tss}[1]{\textsuperscript{#1}}
\newcommand{\tssL}[1]{\MakeLowercase{\textsuperscript{#1}}}
%------------------------------------------
\setlength{\parindent}{0pt}
\setlist[itemize]{itemsep=0mm}
\tkzSetUpPoint[size=7,fill=white]
\newcommand{\twocolkentro}[1]{
\twocolumn[
\begin{@twocolumnfalse}
#1
\end{@twocolumnfalse}]}
\newcommand{\bcc}[1]{
\begin{center}
{\color{\xrwma}{\hrulefill}\raisebox{-2.5mm}{\rule{.4pt}{5mm}}}\hspace{1em}\raisebox{-.65ex}{\begin{varwidth}{\dimexpr0.7\textwidth-2em\relax}\centering{\textbf{\textcolor{\xrwma}{#1}}}\end{varwidth}}\hspace*{1em}{\color{\xrwma}{\raisebox{-2.5mm}{\rule{.4pt}{5mm}}\hrulefill}}
\end{center}}



\begin{document}
\mainmatter
\pagestyle{fancy}
\chapter{Ιδιότητες Συναρτήσεων}
\section{Μονοτονία - Ακρότατα}
\orismoi
\Orismos{Μονοτονία}
Μια συνάρτηση αύξουσα ή φθίνουσα, χαρακτηρίζεται ως \textbf{μονότονη}, ενώ μια γνησίως αύξουσα ή γνησίως φθίνουσα συνάρτηση ως \textbf{γνησίως μονότονη}. Οι χαρακτηρισμοί αυτοί αφορούν τη \textbf{μονοτονία} μιας συνάρτησης, μια ιδιότητα των συναρτήσεων η οποία δείχνει την αύξηση ή τη μείωση των τιμών μιας συνάρτησης σε ένα διάστημα του πεδίου ορισμού.
\begin{enumerate}[itemsep=0mm,label=\bf\arabic*.]
\item \textbf{Γνησίως αύξουσα}\\ Μια συνάρτηση $ f $ ορισμένη σε ένα διάστημα $ \varDelta $ ονομάζεται γνησίως αύξουσα στο $ \varDelta $ εαν για κάθε ζεύγος αριθμών $ x_1,x_2\in\varDelta $ με $ x_1<x_2 $ ισχύει \[ f(x_1)<f(x_2) \]
\item \textbf{Γνησίως φθίνουσα}\\ Μια συνάρτηση $ f $ ορισμένη σε ένα διάστημα $ \varDelta $ ονομάζεται γνησίως φθίνουσα στο $ \varDelta $ εαν για κάθε ζεύγος αριθμών $ x_1,x_2\in\varDelta $ με $ x_1<x_2 $ ισχύει \[ f(x_1)>f(x_2) \]
\begin{center}
\begin{tabular}{p{5cm}p{5cm}}
\begin{tikzpicture}
\draw[dashed] (3.3,1.4) node[anchor=north]{\scriptsize $x_2$} -- 
(3.3,2.58)--(1,2.58) node[left]{\scriptsize $f(x_2)$};
\draw[dashed] (2,1.4) node[anchor=north]{\scriptsize $x_1$}-- 
(2,2.08)--(1,2.08)node[left]{\scriptsize $f(x_1)$};
\begin{axis}[x=1cm,y=1cm,aks_on,xmin=-1,xmax=3,
ymin=-1.4,ymax=2,ticks=none,xlabel={\footnotesize $ x $},
ylabel={\footnotesize $ y $},belh ar]
\addplot[grafikh parastash,\xrwma,domain=-.8:3]{ln(x+1)};
\end{axis}
\tkzDrawPoint[size=7,fill=black](2,2.09)
\tkzDrawPoint[size=7,fill=black](3.3,2.59)
\node[fill=white,inner sep=.1mm] at (2.7,0.6) {\scriptsize $ x_1<x_2\Rightarrow f(x_1)<f(x_2)$};
\end{tikzpicture}\captionof{figure}{Γνησίως αύξουσα}	& \begin{tikzpicture}
\draw[dashed] (2.6,1.4) node[anchor=north]{\scriptsize $x_2$} -- 
(2.6,2.02)--(1,2.02) node[left]{\scriptsize $f(x_2)$};
\draw[dashed] (1.5,1.4) node[anchor=north]{\scriptsize $x_1$}-- 
(1.5,2.7)--(1,2.7)node[left]{\scriptsize $f(x_1)$};
\begin{axis}[x=1cm,y=1cm,aks_on,xmin=-1,xmax=3,
ymin=-1.4,ymax=2,ticks=none,xlabel={\footnotesize $ x $},
ylabel={\footnotesize $ y $},belh ar,clip=false]
\addplot[grafikh parastash,\xrwma,domain=-.6:3]{-0.2*(x+.5)^2+1.5};
\end{axis}
\tkzDrawPoint[size=7,fill=black](2.6,2.02)
\tkzDrawPoint[size=7,fill=black](1.5,2.7)
\node[fill=white,inner sep=.1mm] at (1.95,0.6) {\scriptsize $ x_1<x_2\Rightarrow f(x_1)>f(x_2)$};
\end{tikzpicture}\captionof{figure}{Γνησίως φθίνουσα} \\ 
\end{tabular} 
\end{center}
\end{enumerate}
\Orismos{Ολικά Ακρότατα}
Ακρότατα ονομάζονται οι μέγιστες ή ελάχιστες τιμές μιας συνάρτησης $ f:D_f\rightarrow\mathbb{R} $ τις οποίες παίρνει σε ένα διάστημα ή σε ολόκληρο το πεδίο ορισμού της.
\begin{enumerate}[itemsep=0mm,label=\bf\arabic*.]
\item \textbf{Ολικό μέγιστο}\\
Μια συνάρτηση $ f:D_f\rightarrow\mathbb{R} $ παρουσιάζει ολικό μέγιστο σε ένα σημείο $ x_0\in D_f $ του πεδίου ορισμού της όταν η τιμή $ f(x_0) $ είναι μεγαλύτερη από κάθε άλλη $ f(x) $ για κάθε σημείο $ x_0 $ του πεδίου ορισμού. \[ f(x)\leq f(x_0)\;\;,\;\;\textrm{για κάθε } x\in D_f \]
\item \textbf{Ολικό ελάχιστο}\\
Μια συνάρτηση $ f:D_f\rightarrow\mathbb{R} $ παρουσιάζει ολικό ελάχιστο σε ένα σημείο $ x_0\in D_f $ του πεδίου ορισμού της όταν η τιμή $ f(x_0) $ είναι μικρότερη από κάθε άλλη $ f(x) $ για κάθε σημείο $ x_0 $ του πεδίου ορισμού. \[ f(x)\geq f(x_0)\;\;,\;\;\textrm{για κάθε } x\in D_f \]
\begin{center}
\begin{tabular}{p{5cm}p{5cm}}
\begin{tikzpicture}
\begin{axis}[x=1cm,y=1cm,aks_on,xmin=-.7,xmax=3.2,
ymin=-1,ymax=2,ticks=none,xlabel={\footnotesize $ x $},
ylabel={\footnotesize $ y $},belh ar,clip=false]
\addplot[grafikh parastash,domain=-.3:2.3]{-x^2+2*x};
\end{axis}
\tkzDrawPoint[size=7,fill=black](1.7,2)
\node at (1.95,0.4) {\scriptsize $ f(x)\leq f(x_0)$};
\draw[dashed] (1.7,1) node[anchor=north]{\scriptsize $x_0$} -- 
(1.7,2)--(0.7,2) node[left]{\scriptsize $f(x_0)$};
\node at (0.5,0.8) {\footnotesize $ O $};
\end{tikzpicture}\captionof{figure}{Ολικό μέγιστο}	& \begin{tikzpicture}
\begin{axis}[x=1cm,y=1cm,aks_on,xmin=-.7,xmax=3,
ymin=-.7,ymax=2.3,ticks=none,xlabel={\footnotesize $ x $},
ylabel={\footnotesize $ y $},belh ar,clip=false]
\addplot[grafikh parastash,domain=-.3:2.3]{x^2-2*x+1.5};
\end{axis}
\tkzDrawPoint[size=7,fill=black](1.7,1.2)
\node at (2.1,0.2) {\scriptsize $ f(x)\leq f(x_0)$};
\draw[dashed] (1.7,0.7) node[anchor=north]{\scriptsize $x_0$} -- 
(1.7,1.2)--(0.7,1.2) node[left]{\scriptsize $f(x_0)$};
\node[fill=white,inner sep=.5pt] at (0.5,0.5) {\footnotesize $ O $};
\end{tikzpicture}\captionof{figure}{Ολικό ελάχιστο} \\ 
\end{tabular} 
\end{center}
\end{enumerate}
\Orismos{Άρτια - Περιττή συνάρτηση}
\vspace{-5mm}
\begin{enumerate}[itemsep=0mm,label=\bf\arabic*.]
\item \textbf{Άρτια συνάρτηση}\\ Άρτια ονομάζεται μια συνάρτηση $ f:D_f\rightarrow\mathbb{R} $ για την οποία ισχύουν οι παρακάτω συνθήκες :
\begin{enumerate}[itemsep=0mm,label=\roman*.]
\item $ \forall x\in D_f\Rightarrow -x\in D_f $
\item $ f(-x)=f(x)\;,\;\forall x\in D_f$
\end{enumerate}
\item \textbf{Περιττή συνάρτηση}\\ Περιττή ονομάζεται μια συνάρτηση $ f:D_f\rightarrow\mathbb{R} $ για την οποία ισχύουν οι παρακάτω συνθήκες :
\begin{enumerate}[itemsep=0mm,label=\roman*.]
\item $ \forall x\in D_f\Rightarrow -x\in D_f $
\item $ f(-x)=-f(x)\;,\;\forall x\in D_f$
\end{enumerate}
\end{enumerate}
\begin{center}
\begin{tabular}{p{4.5cm}p{4.5cm}}
\begin{tikzpicture}
\begin{axis}[x=2cm,y=3cm,aks_on,xmin=-1,xmax=1,ymin=-.1,ymax=0.9,ticks=none,xlabel={\footnotesize $ x $},ylabel={\footnotesize $ y $},belh ar]
\addplot[grafikh parastash,domain=-.85:.85]{(x^2)};
\draw[dashed](axis cs:.7,0)node[below]{{\footnotesize $ x $}}--(axis cs:.7,.49)--(axis cs:-.7,.49)--(axis cs:-.7,0)node[below]{{\footnotesize $ -x $}};
\end{axis}
\node[fill=white,inner sep=.1mm] at (2,2.5){\scriptsize $f(-x)=f(x)$};
\end{tikzpicture}\captionof{figure}{Άρτια συνάρτηση}	& \begin{tikzpicture}
\node at (3.4,0) {\scriptsize $f(-x)=-f(x)$};
\begin{axis}[x=2cm,y=1.8cm,aks_on,xmin=-1,xmax=1,ymin=-.9,ymax=.9,ticks=none,xlabel={\footnotesize $ x $},ylabel={\footnotesize $ y $},belh ar]
\addplot[grafikh parastash,domain=-.9:.9]{(x^3)};
\draw[dashed](axis cs:.7,0)node[below]{{\footnotesize $ x $}}--(axis cs:.7,.343)--(axis cs:0,.343)node[left]{{\footnotesize $ f(x) $}};
\draw[dashed](axis cs:-.7,0)node[above]{{\footnotesize $ -x $}}--(axis cs:-.7,-.343)--(axis cs:0,-.343)node[right]{{\footnotesize $ f(-x) $}};
\end{axis}
\end{tikzpicture}\captionof{figure}{Περιττή συνάρτηση} \\ 
\end{tabular} 
\end{center}
\begin{itemize}[itemsep=0mm]
\item Η γραφική παράσταση μιας άρτιας συνάρτησης είναι συμμετρική ως προς τον κατακόρυφο άξονα.
\item H γραφική παράσταση μιας περιττής συνάρτησης είναι συμμετρική ως προς την αρχή των αξόνων.
\item Η αρχή των αξόνων για μια περιττή συνάρτηση ονομάζεται \textbf{κέντρο συμμετρίας} της.
\end{itemize}
\section{Μετατόπιση γραφικής παράστασης}
\thewrhmata
\Thewrhma{Κατακόρυφη μετατόπιση}
Η γραφική παράσταση $ C_f $ μιας συνάρτησης $ f $ μετατοπίζεται κατακόρυφα κατά $ c $ μονάδες προς τα πάνω ή προς τα κάτω, εαν αυξήσουμε ή μειώσουμε αντίστοιχα τις τεταγμένες $ f(x) $ των σημείων της κατά $ c $ μονάδες.
\[ g(x)=f(x)\pm c\;\;,\;\;c>0 \]
Η γραφική παράσταση $ C_g $ της νέας συνάρτησης $ g(x) $ προκύπτει από κατακόρυφη μετατόπιση της $ C_f $ κατά $ c $ μονάδες.
\begin{center}
\begin{tabular}{p{5cm}cp{5cm}}
\begin{tikzpicture}
\begin{axis}[aks_on,belh ar,xlabel={\footnotesize$x$},ylabel={\footnotesize$y$}
,xmin=-2,xmax=2.,ymin=-1,ymax=3,x=1cm,y=1cm]
\addplot[clip=false,domain=-1.8:1.8,grafikh parastash]{x^2-.7};
\addplot[domain=-1.8:1.8,pl,samples=200]{x^2};
\addplot[domain=-1.5:1.5,grafikh parastash]{x^2+.7};
\draw[pl,-latex] (axis cs:.5,.25) -- (axis cs:.5,.95);
\draw[pl,-latex] (axis cs:.5,.25) -- (axis cs:.5,-.45);
\draw[pl,-latex] (axis cs:-.5,.25) -- (axis cs:-.5,-.45);
\draw[pl,-latex] (axis cs:-.5,.25) -- (axis cs:-.5,.95);
\node at (axis cs:-.25,.5) {\footnotesize$+c$};
\node at (axis cs:-.25,-.25) {\footnotesize$-c$};
\node at (axis cs:.25,.5) {\footnotesize$+c$};
\node at (axis cs:.25,-.25) {\footnotesize$-c$};
\end{axis}
\end{tikzpicture} & & \begin{tikzpicture}
\begin{axis}[clip=false,aks_on,belh ar,xlabel={\footnotesize$x$},ylabel={\footnotesize$y$}
,xmin=-2,xmax=4.2,ymin=-1,ymax=3,x=1cm,y=1cm]
\addplot[domain=-1.8:1.8,grafikh parastash]{x^2-.7};
\addplot[domain=-.8:2.8,pl,samples=200]{(x-1)^2-.7};
\addplot[domain=.2:3.8,grafikh parastash]{(x-2)^2-.7};
\draw[pl,-latex] (axis cs:2,.3) -- (axis cs:3,.3);
\draw[pl,-latex] (axis cs:2.5,1.55) -- (axis cs:3.5,1.55);
\draw[pl,-latex] (axis cs:0,.3) -- (axis cs:-1,.3);
\draw[pl,-latex] (axis cs:-.5,1.55) -- (axis cs:-1.5,1.55);
\node at (axis cs:-.5,.5) {\footnotesize$+c$};
\node at (axis cs:3,1.7) {\footnotesize$-c$};
\node at (axis cs:2.5,.5) {\footnotesize$-c$};
\node at (axis cs:-1,1.7) {\footnotesize$+c$};
\end{axis}
\end{tikzpicture} \\ 
\end{tabular} 
\end{center}
\Thewrhma{Οριζόντια μετατόπιση}
Η γραφική παράσταση $ C_f $ μιας συνάρτησης $ f $ μετατοπίζεται οριζόντια κατά $ c $ μονάδες προς τα αριστερά ή προς τα δεξιά, εαν αυξήσουμε ή μειώσουμε αντίστοιχα τις τετμημένες $ x $ των σημείων της κατά $ c $ μονάδες.
\[ g(x)=f(x\pm c)\;\;,\;\;c>0  \]
Η γραφική παράσταση $ C_g $ της νέας συνάρτησης $ g(x) $ προκύπτει από οριζόντια μετατόπιση της $ C_f $ κατά $ c $ μονάδες.
\end{document}








%\chapter{Τριγωνομετρία}
\section{Τριγωνομετρικοί αριθμοί}
\orismoi
\Orismos{Τριγωνομετρικοί αριθμοί}
Έστω $ AB\varGamma $ ένα ορθογώνιο τρίγωνο, με $ A=90\degree $ τότε οι τριγωνομετρικοί αριθμοί των οξειών γωνιών του τριγώνου ορίζονται ως εξής :\\
\begin{minipage}{\linewidth}\mbox{}\\
\vspace{-1cm}
\begin{WrapText1}{5}{3.3cm}
\vspace{-2mm}
\begin{tikzpicture}[scale=.8]
\tkzDefPoint(0,0){A}
\tkzDefPoint(3,0){B}
\tkzDefPoint(0,4){C}
\tkzMarkAngle[fill=\xrwma!50,size=.5](C,B,A)
\tkzMarkRightAngle[size=.3](B,A,C)
\tkzDrawPolygon[pl](A,B,C)
\tkzText(2.2,.2){$ \omega $}
\tkzLabelPoint[left](A){$ A $}
\tkzLabelPoint[right](B){$ B $}
\tkzLabelPoint[left](C){$ \varGamma $}
\tkzDrawPoints[size=7,fill=white](A,B,C)
\end{tikzpicture}\captionof{figure}{Τριγωνομετρικοί αριθμοί}
\end{WrapText1}
\begin{enumerate}[itemsep=0mm,label=\bf\arabic*.]
\item \textbf{Ημίτονο}\\
Ημίτονο μιας οξέιας γωνίας ενός ορθογωνίου τριγώνου ονομάζεται ο λόγος της απέναντι κάθετης πλευράς προς την υποτείνουσα.
\[ \textrm{Ημίτονο}=\frac{\textrm{Απέναντι Κάθετη}}{\textrm{Υποτείνουσα}}\;\;,\;\;\hm{\omega}=\frac{A\varGamma}{B\varGamma} \]
\item \textbf{Συνημίτονο}\\
Συνημίτονο μιας οξέιας γωνίας ενός ορθογωνίου τριγώνου ονομάζεται ο λόγος της προσκείμενης κάθετης πλευράς προς την υποτείνουσα.
\end{enumerate}
\[ \textrm{Συνημίτονο}=\frac{\textrm{Προσκείμενη Κάθετη}}{\textrm{Υποτείνουσα}}\;\;,\;\;\syn{\omega}=\frac{AB}{B\varGamma} \]
\begin{enumerate}[itemsep=0mm,label=\bf\arabic*.,start=3]
\item \textbf{Εφαπτομένη}\\
Εφαπτομένη μιας οξέιας γωνίας ενός ορθογωνίου τριγώνου ονομάζεται ο λόγος της απέναντι κάθετης πλευράς προς την προσκείμενη κάθετη.
\[ \textrm{Εφαπτομένη}=\frac{\textrm{Απέναντι Κάθετη}}{\textrm{Προσκείμενη Κάθετη}}\;\;,\;\;\ef{\omega}=\frac{A\varGamma}{AB} \]
\item \textbf{Συνεφαπτομένη}\\
Συνεφαπτομένη μιας οξέιας γωνίας ενός ορθογωνίου τριγώνου ονομάζεται ο λόγος της προσκείμενης κάθετης πλευράς προς την απέναντι κάθετη.
\[ \textrm{Συνεφαπτομένη}=\frac{\textrm{Προσκείμενη Κάθετη}}{\textrm{Απέναντι Κάθετη}}\;\;,\;\;\syf{\omega}=\frac{AB}{A\varGamma} \]
\end{enumerate}
\end{minipage}\mbox{}\\\\
Στον ακόλουθο πίνακα βλέπουμε τους τριγωνομετρικούς αριθμούς των βασικότερων γωνιών.
\begin{center}
\begin{tabular}{c||>{\centering\arraybackslash}m{.8cm}>{\centering\arraybackslash}m{.8cm}>{\centering\arraybackslash}m{.8cm}>{\centering\arraybackslash}m{.8cm}>{\centering\arraybackslash}m{.8cm}>{\centering\arraybackslash}m{.8cm}>{\centering\arraybackslash}m{.8cm}>{\centering\arraybackslash}m{.8cm}>{\centering\arraybackslash}m{.8cm}}
\hline  \multicolumn{10}{c}{\textbf{Βασικές Γωνίες}} \rule[-2ex]{0pt}{5.5ex}  \\ 
\hhline{==========} \rule[-2ex]{0pt}{5.5ex} \textbf{Μοίρες} & $ 0\degree $ & $ 30\degree $ & $ 45\degree $ & $ 60\degree $ & $ 90\degree $ & $ 120\degree $ & $ 135\degree $ & $ 150\degree $ & $ 180\degree $ \\ 
\rule[-2ex]{0pt}{4ex} \textbf{Ακτίνια} & $ 0 $ & $ \frac{\pi}{6} $ & $ \frac{\pi}{4} $ & $ \frac{\pi}{3} $ & $ \frac{\pi}{2} $ & $ \frac{2\pi}{3} $ & $ \frac{3\pi}{4} $ & $ \frac{5\pi}{6} $ & $ \pi $ \\ 
\hline \rule[-2ex]{0pt}{5.5ex} \textbf{Σχήμα} & \begin{tikzpicture}
\fill[fill=\xrwma!50] (0,0) -- (.3,0) arc (0:0:.3) -- cycle;
\draw (-.35,0) -- (.35,0);
\draw (0,-.35) -- (0,.35);
\draw (0,0) circle (.3);
\coordinate (A) at (0:.3);
\draw (0,0) -- (A);
\end{tikzpicture} & \begin{tikzpicture}
\fill[fill=\xrwma!50] (0,0) -- (.3,0) arc (0:30:.3) -- cycle;
\draw (-.35,0) -- (.35,0);
\draw (0,-.35) -- (0,.35);
\draw (0,0) circle (.3);
\coordinate (A) at (30:.3);
\draw (0,0) -- (A);
\end{tikzpicture} & \begin{tikzpicture}
\fill[fill=\xrwma!50] (0,0) -- (.3,0) arc (0:45:.3) -- cycle;
\draw (-.35,0) -- (.35,0);
\draw (0,-.35) -- (0,.35);
\draw (0,0) circle (.3);
\coordinate (A) at (45:.3);
\draw (0,0) -- (A);
\end{tikzpicture} & \begin{tikzpicture}
\fill[fill=\xrwma!50] (0,0) -- (.3,0) arc (0:60:.3) -- cycle;
\draw (-.35,0) -- (.35,0);
\draw (0,-.35) -- (0,.35);
\draw (0,0) circle (.3);
\coordinate (A) at (60:.3);
\draw (0,0) -- (A);
\end{tikzpicture} & \begin{tikzpicture}
\fill[fill=\xrwma!50] (0,0) -- (.3,0) arc (0:90:.3) -- cycle;
\draw (-.35,0) -- (.35,0);
\draw (0,-.35) -- (0,.35);
\draw (0,0) circle (.3);
\coordinate (A) at (90:.3);
\draw (0,0) -- (A);
\end{tikzpicture} & \begin{tikzpicture}
\fill[fill=\xrwma!50] (0,0) -- (.3,0) arc (0:120:.3) -- cycle;
\draw (-.35,0) -- (.35,0);
\draw (0,-.35) -- (0,.35);
\draw (0,0) circle (.3);
\coordinate (A) at (120:.3);
\draw (0,0) -- (A);
\end{tikzpicture} & \begin{tikzpicture}
\fill[fill=\xrwma!50] (0,0) -- (.3,0) arc (0:135:.3) -- cycle;
\draw (-.35,0) -- (.35,0);
\draw (0,-.35) -- (0,.35);
\draw (0,0) circle (.3);
\coordinate (A) at (135:.3);
\draw (0,0) -- (A);
\end{tikzpicture} & \begin{tikzpicture}
\fill[fill=\xrwma!50] (0,0) -- (.3,0) arc (0:150:.3) -- cycle;
\draw (-.35,0) -- (.35,0);
\draw (0,-.35) -- (0,.35);
\draw (0,0) circle (.3);
\coordinate (A) at (150:.3);
\draw (0,0) -- (A);
\end{tikzpicture} & \begin{tikzpicture}
\fill[fill=\xrwma!50] (0,0) -- (.3,0) arc (0:180:.3) -- cycle;
\draw (-.35,0) -- (.35,0);
\draw (0,-.35) -- (0,.35);
\draw (0,0) circle (.3);
\coordinate (A) at (180:.3);
\draw (0,0) -- (A);
\end{tikzpicture} \\ 
\hline \rule[-2ex]{0pt}{5ex} $ \hm{\omega} $ & $ 0 $ & $ \frac{1}{2} $ & $ \frac{\sqrt{2}}{2} $ & $ \frac{\sqrt{3}}{2} $ & $ 1 $ & $ \frac{\sqrt{3}}{2} $ & $ \frac{\sqrt{2}}{2} $ & $ \frac{1}{2} $ & $ 0 $ \\ 
\rule[-2ex]{0pt}{4ex} $ \syn{\omega} $ & $ 1 $ & $ \frac{\sqrt{3}}{2} $ & $ \frac{\sqrt{2}}{2} $ & $ \frac{1}{2} $ & $ 0 $ & $ -\frac{1}{2} $ & $ -\frac{\sqrt{2}}{2} $ & $ -\frac{\sqrt{3}}{2} $ & $ -1 $ \\ 
\rule[-2ex]{0pt}{4ex} $ \ef{\omega} $ & $ 0 $ & $ \frac{\sqrt{3}}{3} $ & $ 1 $ & $ \sqrt{3} $ & \begin{minipage}{.8cm}
\begin{center}
{\scriptsize Δεν\\\vspace{-1mm}ορίζεται}
\end{center}
\end{minipage} & $ -\sqrt{3} $ & $ -1 $ & $ -\frac{\sqrt{3}}{3} $ & $ 0 $ \\
\rule[-2ex]{0pt}{4ex} $ \syf{\omega} $ & \begin{minipage}{.8cm}
\begin{center}
{\scriptsize Δεν\\\vspace{-1mm}ορίζεται}
\end{center}
\end{minipage} & $ \sqrt{3} $ & $ 1 $ & $ \frac{\sqrt{3}}{3} $ & $ 0 $ & $ -\frac{\sqrt{3}}{3} $ & $ -1 $ & $ -\sqrt{3} $ & \begin{minipage}{.8cm}
\begin{center}
{\scriptsize Δεν\\\vspace{-1mm}ορίζεται}
\end{center}
\end{minipage} \\ 
\hline 
\end{tabular}\captionof{table}{Τριγωνομετρικοί αριθμοί βασικών γωνιών}
\end{center}\mbox{}\\
\Orismos{τριγωνομετρικοσ κυκλοσ}
Τριγωνομετρικός κύκλος ονομάζεται ο κύκλος με ακτίνα  και κέντρο την αρχή των αξόνων ενός ορθογωνίου συστήματος συντεταγμένων, στους άξονες του οποίου παίρνουν τιμές οι τριγωνομετρικοί αριθμοί των γωνιών.
\begin{center}
\begin{tabular}{p{6.85cm}p{6.85cm}}
\begin{tikzpicture}[>=latex,scale=2]
\fill[fill=black!50] (0,0) -- (.2,0) arc (0:60:.2) -- cycle;
%axis
\draw[->] (-1.2,0) -- coordinate (x axis mid) (1.5,0) node[right,fill=white] {{\footnotesize $ x $}};
\foreach \x in {-1,-0.8,-0.6,-0.4,-0.2,0,0.2,0.4,0.6,0.8,1}
\draw (\x,.5pt) -- (\x,-.5pt)
node[anchor=north] {{\tiny \x}};
\foreach \y in {-1,-0.8,-0.6,-0.4,-0.2,0,0.2,0.4,0.6,0.8,1}
\draw (.5pt,\y) -- (-.5pt,\y)
node[anchor=east] {{\tiny \y}};
\draw[->] (0,-1.2) -- (0,1.5) node[above,fill=white] {{\footnotesize $ y $}};
\draw[-] (1,-1.2) -- (1,1.8);
\draw[-] (-1.2,1) -- (1.2,1);
\draw[-,thick] (0,1) -- (1.732/3,1);
\draw[-,thick] (1,0) -- (1,1.732);
\draw[-,dashed] (-.7,-1.732*0.7) -- (1,1.732);
\draw circle (1);
\coordinate (A) at (60:1);
\tkzDefPoint(0,0){O}
\tkzDefPoint(cos(pi/3),0){B}
\tkzDefPoint(0,sin(pi/3)){C}
\tkzDefPoint(1,tan(pi/3)){D}
\tkzDefPoint(cot(pi/3),1){E}
\tkzDefPoint(1,0){F}
\tkzDefPoint(0,1){G}
\tkzDrawSegment(O,A)
\tkzDrawSegments[thin,dashed](A,B A,C)
\tkzText(0,1.75){{\scriptsize Άξονας Ημιτόνων}}
\tkzText(1.6,-.12){{\scriptsize Άξονας}}
\tkzText(1.6,-.23){{\scriptsize Συνημιτόνων}}
\tkzText(-1,1.2){{\scriptsize Άξονας}}
\tkzText(-.75,1.1){{\scriptsize Συνεφαπτομένων}}
\tkzText(1.23,-.9){{\scriptsize Άξονας}}
\tkzText(1.4,-1){{\scriptsize Εφαπτομένων}}
\tkzText(-.5,-1.1){{\scriptsize $ \delta $}}
\tkzDrawSegment[thick](O,B)
\tkzDrawSegment[thick](O,C)
\tkzDrawPoints[size=7,fill=white](O,A,B,C,D,E,F,G)
\tkzText(-.4,.43){{{\scriptsize \textrm{ημ}$ \omega $}}$\LEFTRIGHT\{.{ \rule{0pt}{18mm} } $}
\tkzText(.25,-.25){$ \undercbrace{\rule{9mm}{0mm}}_{{\scriptsize \textrm{συν}\omega}} $}
\tkzText(1.2,.87){$\LEFTRIGHT.\}{ \rule{0pt}{35mm} } ${{\scriptsize \textrm{εφ}$ \omega $}}}
\tkzText(.3,1.12){$ \overcbrace{\rule{11mm}{0mm}}^{{\scriptsize \textrm{σφ}\omega}} $}
\tkzText(.25,.15){$ \omega $}
\tkzLabelPoint[below left](O){{\tiny $ O $}}
\tkzLabelPoint[above=1mm,right](A){{\tiny $ M $}}
\tkzLabelPoint[above right](B){{\tiny $ M_1 $}}
\tkzLabelPoint[above=1mm, left](C){{\tiny $ M_2 $}}
\tkzLabelPoint[left](D){{\tiny $ K $}}
\tkzLabelPoint[above](E){{\tiny $ \varLambda $}}
\tkzLabelPoint[below right](F){{\tiny $ A $}}
\tkzLabelPoint[above left](G){{\tiny $ B $}}
\draw [->] (.984*.9,.173*.9) arc (10:45:.9);
\draw [->] (.984*.9,-.173*.9) arc (-10:-45:.9);
\tkzText(.72,.35){$ + $}
\tkzText(.72,-.35){$ - $}
\tkzText(.35,.45){$ \rho $}
\tkzText(-1,.9){{\scriptsize $ \varepsilon_2 $}}
\tkzText(.9,-1){{\scriptsize $ \varepsilon_1 $}}
\end{tikzpicture}\captionof{figure}{Τριγωνομετρικός κύκλος} & \begin{tikzpicture}[>=latex,scale=2]
\fill[fill=black!50] (0,0) -- (.2,0) arc (0:45:.2) -- cycle;
%axis
\draw[->] (-1.2,0) -- (1.5,0) node[right,fill=white] {{\footnotesize $ x $}};
\draw[->] (0,-1.2) -- (0,1.5) node[above,fill=white] {{\footnotesize $ y $}};

\foreach \gwnia/\xtext in {
30/\frac{\pi}{6},
45/\frac{\pi}{4},
60/\frac{\pi}{3},
90/\frac{\pi}{2},
120/\frac{2\pi}{3},
135/\frac{3\pi}{4},
150/\frac{5\pi}{6},
180/\pi,
210/\frac{7\pi}{6},
240/\frac{4\pi}{3},
270/\frac{3\pi}{2},
300/\frac{5\pi}{3},
330/\frac{11\pi}{6},
360/2\pi}
\draw (\gwnia:0.85cm) node {{\scriptsize $\xtext$}};
\foreach \gwnia/\xtext in {
90/\frac{\pi}{2},
180/\pi,
270/\frac{3\pi}{2},
360/2\pi}
\draw (\gwnia:0.85cm) node[fill=white] {{\scriptsize $\xtext$}};
\tkzDefPoint(0,0){O}
\coordinate (A) at (45:1);
\tkzDrawSegment(O,A)
\draw circle (1);
\foreach \gwnia in {0,30,45,60,90,120,135,150,180,210,240,270,300,330}{
\coordinate (P) at (\gwnia:1);
\draw (\gwnia:1.22cm) node[fill=white] {{\scriptsize $\gwnia^\circ$}};
\draw[draw=black,fill=white] (P) circle (.7pt);};
\tkzText(.25,.1){$ \omega $}
\end{tikzpicture}\captionof{figure}{Βασικές γωνίες}
\end{tabular}
\end{center}
\begin{itemize}[itemsep=0mm]
\item Κάθε γωνία $ \omega $ έχει πλευρές, τον θετικό ημιάξονα $ Ox $ και την ακτίνα $ \rho $ του κύκλου, μετρώντας τη γωνία αυτή αριστερόστροφα, φορά που ορίζεται ως \textbf{θετική}.
\item Ο οριζόντιος άξονας $ x'x $ είναι ο άξονας συνημιτόνων ενώ ο κατακόρυφος $ y'y $ ο άξονας ημιτόνων.
\item Κάθε σημείο $ M $ του κύκλου έχει συντεταγμένες $ M(\syn{\omega},\hm{\omega}) $.
\item Η τετμημέμη του σημείου είναι ίση με το συνημίτονο της γωνίας, ενώ η τεταγμένη ίση με το ημίτονο της.
\[ x=\syn{\omega}\;\;,\;\;y=\hm{\omega} \]
\item Η εφαπτόμενη ευθεία στον κύκλο στο σημείο $ A(1,0) $ είναι ο \textbf{άξονας των εφαπτομένων}. Η εφαπτομένη της γωνίας $ \omega $ είναι η τεταγμένη του σημείου τομής της ευθείας $ \varepsilon_1 $ με το φορέα $ \delta $ της ακτίνας.
\[ y_{\!_K}=\ef{\omega} \]
\item Η εφαπτόμενη ευθεία στον κύκλο στο σημείο $ B(0,1) $ είναι ο \textbf{άξονας των συνεφαπτομένων}. Η συνεφαπτομένη της γωνίας $ \omega $ είναι η τετμημένη του σημείου τομής της ευθείας $ \varepsilon_2 $ με το φορέα $ \delta $ της ακτίνας.
\[ x_{\!_K}=\syf{\omega} \]	
\end{itemize}
\Orismos{τριγ. αρ. γωνιασ σε συστημα συντεταγμενων}
Έστω $ Oxy $ ένα ορθογώνιο σύστημα συντεταγμένων και $ M(x,y) $ ένα σημείο του. Ενώνοντας το σημείο $ M $ με την αρχή των αξόνων, το ευθύγραμμο τμήμα που προκύπτει δημιουργεί μια γωνία $ \omega $ με το θετικό οριζόντιο ημιάξονα $ Ox $.
Το μήκος του ευθύγραμμου τμήματος $ OM $ είναι :
\[ OM=\rho=\sqrt{x^2+y^2} \]
Οι τριγωνομετρικοί αριθμοί της γωνίας αυτής ορίζονται με τη βοήθεια των συντεταγμένων του σημείου και είναι :\\
\begin{minipage}{\linewidth}\mbox{}\\
\vspace{-1cm}
\begin{WrapText1}{14}{4.7cm}
\vspace{.5cm}
\begin{tikzpicture}[y=.8cm,x=.9cm]
\draw[draw=black,fill=black!10] (0,0) -- (.5,0) arc (0:40:.5) -- cycle;
\draw[-latex]  (-.4,0)  -- coordinate (x axis mid) (4,0) node[right,fill=white] {{\footnotesize $ x $}};
\draw[-latex] (0,-.4) -- (0,3.5) node[above,fill=white] {{\footnotesize $ y $}};
\draw (3,.1) -- (3,-.1) node[anchor=north] {\scriptsize $ x $};
\draw (.1,2.5) -- (-.1,2.5) node[anchor=east] {\scriptsize $ y $};
\draw[dashed] (3,0) -- (3,2.5) -- (0,2.5);
\tkzDefPoint(0,0){O}
\tkzDefPoint(3,2.5){M}
\tkzDefPoint(3,0){A}
\tkzDefPoint(0,2.5){B}
\tkzDrawSegment(O,M)
\tkzDrawPoint[size=7,fill=white](M)
\tkzDrawPoint[size=7,fill=white](A)
\tkzDrawPoint[size=7,fill=white](B)
\tkzLabelPoint[below left](O){$ O $}
\tkzLabelPoint[above](M){$ M(x,y) $}
\tkzLabelPoint[above right](A){{\footnotesize $ A(x,0) $}}
\tkzLabelPoint[above right](B){{\footnotesize $ B(0,y) $}}
\tkzText(1.5,-.4){$ \undercbrace{\rule{25mm}{0mm}}_{{\scriptsize x}} $}
\tkzText(-.3,1.25){{{\scriptsize $ y $}}$\LEFTRIGHT\{.{ \rule{0pt}{20mm} } $}
\tkzText[fill=white,inner sep=.2mm](2.7,1){{\footnotesize $ \rho=\sqrt{x^2+y^2} $}}
\tkzText(.7,.2){{\footnotesize $ \omega $}}
\end{tikzpicture}\captionof{figure}{Τριγωνομετρικοί αριθμοί σε σύστημα συντεταγμένων}\end{WrapText1}
\begin{enumerate}[itemsep=0mm,label=\bf\arabic*.]
\item \textbf{Ημίτονο}\\
Ημίτονο της γωνίας $ \omega $ ονομάζεται ο λόγος της τεταγμένης του σημείου προς την απόσταση του από την αρχή των αξόνων.
\[ \hm{\omega}=\frac{AM}{OM}=\frac{y}{\rho} \]
\item \textbf{Συνημίτονο}\\
Συνημίτονο της γωνίας $ \omega $ ονομάζεται ο λόγος της τετμημένης του σημείου προς την απόσταση του από την αρχή των αξόνων.
\[ \syn{\omega}=\frac{BM}{OM}=\frac{x}{\rho} \]
\end{enumerate}

\begin{enumerate}[itemsep=0mm,label=\bf\arabic*.,start=3]
\item \textbf{Εφαπτομένη}\\
Εφαπτομένη της γωνίας $ \omega $ τριγώνου ονομάζεται ο λόγος της τεταγμένης του σημείου προς την τετμημένη του.
\[ \ef{\omega}=\frac{AM}{BM}=\frac{y}{x}\;\;,\;\;x\neq0 \]
\item \textbf{Συνεφαπτομένη}\\
Συνεφαπτομένη της γωνίας $ \omega $ ονομάζεται ο λόγος της τετμημένης του σημείου προς την τεταγμένη του.
\[ \syf{\omega}=\frac{BM}{AM}=\frac{x}{y}\;\;.\;\;y\neq0 \]
\end{enumerate}\end{minipage}\mbox{}\\\\\\
\thewrhmata
\Thewrhma{Άκρα τριγωνομετρικών αριθμών}
To ημίτονο και το συνημίτονο οποιασδήποτε γωνίας $ \omega $ παίρνει τιμές από $-1$ μέχρι $ 1 $. Οι παρακάτω σχέσεις είναι ισοδύναμες :
\begin{multicols}{2}
\begin{rlist}
\item  $ -1\leq\hm{\omega}\leq1\;\;,\;\;-1\leq\syn{\omega}\leq1 $
\item $ |\hm{\omega}|\leq1\ ,\ |\syn{\omega}|\leq1 $
\end{rlist}
\end{multicols}
\Thewrhma{βασικεσ τριγωνομετρικεσ ταυτοτητεσ}
Για οποιαδήποτε γωνία $ \omega $ ισχύουν οι παρακάτω βασικές τριγωνομετρικές ταυτότητες :
\begin{multicols}{3}
\begin{enumerate}[itemsep=0mm]
\item $ \hm^2{\omega}+\syn^2{\omega}=1 $
\item $ \ef{\omega}={\dfrac{\hm{\omega}}{\syn{\omega}}} $
\item $ \syf{\omega}={\dfrac{\syn{\omega}}{\hm{\omega}}} $
\item $ \ef{\omega}\cdot\syf{\omega}=1 $
\item $ \syn^2{\omega}=\dfrac{1}{1+\ef^2{\omega}} $
\item $ \hm^2{\omega}=\dfrac{\ef^2{\omega}}{1+\ef^2{\omega}} $
\end{enumerate}
\end{multicols}
\Thewrhma{αναγωγη στο 1\textsuperscript{\MakeLowercase{o}} τεταρτημοριο}\label{th:an_tet}
Οι τριγωνομετρικοί αριθμοί γωνιών που καταλήγουν στο 2\tss{ο}, 3\tss{ο} ή 4\tss{ο} ανάγωνται σε τριγωνομετρικούς αριθμούς γωνιών του 1\textsuperscript{ου} τεταρτημορίου σύμφωνα με τους παρακάτω τύπους.
\begin{enumerate}[itemsep=0mm,label=\bf\arabic*.]
\item \textbf{Παραπληρωματικές γωνίες (2\textsuperscript{ο} τεταρτημόριο)}\\
Γωνίες που καταλήγουν στο 2\tss{ο} τεταρτημόριο μπορούν να γραφτούν ως παραπληρωματικές γωνιών του 1\tss{ου} τεταρτημορίου. Εαν $ \omega $ είναι μια γωνία του 1\textsuperscript{ου} τεταρτημορίου τότε η παραπληρωματική της θα είναι της μορφής $ 180\degree-\omega $. Οι σχέσεις μεταξύ των τριγωνομετρικών τους αριθμών φαίνονται παρακάτω :\\
\begin{minipage}{\linewidth}\mbox{}\\
\vspace{-1cm}
\begin{WrapText1}{7}{6cm}
\begin{tikzpicture}[>=latex,scale=2]
\clip (-1.5,-.3) rectangle (1.4,1.4);
\draw[fill=\xrwma!10] (0,0) -- (.2,0) arc (0:40:.2) -- cycle;
\draw[fill=\xrwma!30] (0,0) -- (.15,0) arc (0:140:.15) -- cycle;
%axis
\draw[->] (-1.2,0) -- (1.2,0) node[right,fill=white] {{\footnotesize $ x $}};
\draw[->] (0,-1.2) -- (0,1.2) node[above,fill=white] {{\footnotesize $ y $}};
\tkzDefPoint(0,0){O}
\tkzDefPoint(cos(2*pi/9),0){D}
\tkzDefPoint(-cos(2*pi/9),0){E}
\tkzDefPoint(0,sin(2*pi/9)){F}
\coordinate (A) at (40:1);
\coordinate (B) at (140:1);
\tkzDrawSegments(O,A O,B)
\draw circle (1);
\tkzText(.3,.1){{\footnotesize $ \omega $}}

\tkzText(0,.3){{\footnotesize $ 180^{\mathrm{o}}-\omega $}}
\draw[dashed] (A) -- (D) node[anchor=north]{{\footnotesize $ x $}};
\draw[dashed] (B) -- (E)node[anchor=north]{{\footnotesize $ -x $}};
\draw[dashed] (A) -- (B);
\tkzDrawPoints[size=7,fill=white](A,B,D,E,F)
\tkzLabelPoint[above left](F){{\footnotesize $ y $}}
\tkzLabelPoint[above right](A){{\footnotesize $ M(x,y) $}}
\tkzLabelPoint[above left](B){{\footnotesize $ N(-x,y) $}}
\tkzLabelPoint[below left](O){$ O $}
\end{tikzpicture}
\end{WrapText1}
\begin{itemize}[itemsep=0mm]
\item $ \hm{\left( 180\degree-\omega\right) }=\hm{\omega} $
\item $ \syn{\left( 180\degree-\omega\right) }=-\syn{\omega} $
\item $ \ef{\left( 180\degree-\omega\right) }=-\ef{\omega} $
\item $ \syf{\left( 180\degree-\omega\right) }=-\syf{\omega} $
\end{itemize}
Οι παραπληρωματικές γωνίες έχουν ίσα ημίτονα και αντίθετους όλους τους υπόλοιπους τριγωνομετρικούς αριθμούς. Τα σημεία $ M,N $ του τριγωνομετρικού κύκλου, των γωνιών $ \omega $ και $ 180\degree-\omega $ αντίστοιχα, είναι συμμετρικα ως προς άξονα $ y'y $ και κατά συνέπεια έχουν αντίθετες τετμημένες.
\end{minipage}
\item \textbf{Γωνίες με διαφορά $ \mathbold{180\degree} $ (3\tss{ο} Τεταρτημόριο)}\\
Γωνίες που καταλήγουν στο 3\tss{ο} τεταρτημόριο μπορούν να γραφτούν ως γωνίες με διαφορά $ 180\degree $ γωνιών του 1\tss{ου} τεταρτημορίου. Εάν $ \omega $ είναι μια γωνία του 1\textsuperscript{ου} τεταρτημορίου, η γωνία η οποία διαφέρει από την $ \omega $ κατά $ 180\degree $ θα είναι της μορφής $ 180\degree+\omega $. Οι σχέσεις που συνδέουν τους τριγωνομετρικούς αριθμούς των δύο γωνιών θα είναι :\\
\begin{minipage}{\linewidth}\mbox{}\\
\vspace{-1cm}
\begin{WrapText2}{12}{4cm}
\begin{tikzpicture}[>=latex,scale=1.5]
\draw[fill=\xrwma!10] (0,0) -- (.2,0) arc (0:40:.2) -- cycle;
\draw[fill=\xrwma!30] (0,0) -- (.15,0) arc (0:220:.15) -- cycle;
%axis
\draw[->] (-1.2,0) -- (1.2,0) node[right,fill=white] {{\footnotesize $ x $}};
\draw[->] (0,-1.2) -- (0,1.2) node[above,fill=white] {{\footnotesize $ y $}};
\tkzDefPoint(0,0){O}
\tkzDefPoint(cos(2*pi/9),0){D}
\tkzDefPoint(-cos(2*pi/9),0){E}
\tkzDefPoint(0,-sin(2*pi/9)){F}
\tkzDefPoint(0,sin(2*pi/9)){C}
\coordinate (A) at (40:1);
\coordinate (B) at (220:1);
\tkzDrawSegments(O,A O,B)
\draw circle (1);
\tkzText(.3,.1){{\footnotesize $ \omega $}}

\tkzText(-.2,.27){{\footnotesize $ 180^{\mathrm{o}}+\omega $}}
\draw[dashed] (A) -- (D) node[anchor=north]{{\footnotesize $ x $}};
\draw[dashed] (B) -- (E)node[anchor=south]{{\footnotesize $ -x $}};
\draw[dashed] (A) -- (C);
\draw[dashed] (B) -- (F);
\tkzDrawPoints[size=7,fill=white](A,B,C,D,E,F)
\tkzLabelPoint[left](C){{\footnotesize $ y $}}
\tkzLabelPoint[right](F){{\footnotesize $ -y $}}
\tkzLabelPoint[above,fill=white,inner sep=.2mm,yshift=1mm](A){{\footnotesize $ M(x,y) $}}
\tkzLabelPoint[below,fill=white,inner sep=.2mm,yshift=-1mm](B){{\footnotesize $ N(-x,-y) $}}
\tkzLabelPoint[below right](O){$ O $}
\end{tikzpicture}
\end{WrapText2}
\begin{multicols}{2}
\begin{itemize}[itemsep=0mm,leftmargin=2mm]
\item $ \hm{\left( 180\degree+\omega\right) }=-\hm{\omega} $
\item $ \syn{\left( 180\degree+\omega\right) }=-\syn{\omega} $
\item $ \ef{\left( 180\degree+\omega\right) }=\ef{\omega} $
\item $ \syf{\left( 180\degree+\omega\right) }=\syf{\omega} $
\end{itemize}
\end{multicols}
Οι γωνίες με διαφορά $ 180\degree $ έχουν αντίθετα ημίτονα και συνημίτονα ενώ έχουν ίσες εφαπτομένες και συνεφαπτομένες. Τα σημεία $ M,N $ του τριγωνομετρικού κύκλου, των γωνιών $ \omega $ και $ 180\degree+\omega $ αντίστοιχα, είναι συμμετρικά ως προς την αρχή των αξόνων και κατά συνέπεια έχουν αντίθετες συντεταγμένες.
\end{minipage}
\item \textbf{Αντίθετες γωνίες - Γωνίες με άθροισμα {\boldmath{$ 360\degree $}} (4\textsuperscript{ο} Τεταρτημόριο)}\\
Γωνίες που καταλήγουν στο 4\tss{ο} τεταρτημόριο μπορούν να γραφτούν ως αντίθετες γωνιών του 1\tss{ου} τεταρτημορίου. Η αντίθετη γωνία, μιας γωνίας $ \omega $ του 1\textsuperscript{ου} τεταρτημορίου, ορίζεται να είναι η γωνία η οποία έχει ίσο μέτρο με τη γωνία $ \omega $, με φορά αντίθετη απ' αυτήν και θα έχει τη μορφή $ -\omega $. Επιπλέον η γωνία η οποία έχει με τη γωνία $ \omega $, άθροισμα $ 360\degree $ καταλήγει στο ίδιο σημείο και θα είναι $ 360\degree-\omega $.\\
\begin{minipage}{\linewidth}\mbox{}\\
\vspace{-1cm}
\begin{WrapText1}{9}{4.3cm}
\begin{tikzpicture}[>=latex,scale=1.5]
\draw[fill=\xrwma!10] (0,0) -- (.2,0) arc (0:40:.2) -- cycle;
\draw[fill=\xrwma!30] (0,0) -- (.15,0) arc (0:320:.15) -- cycle;
\draw[fill=\xrwma!50] (0,0) -- (.25,0) arc (0:-40:.25) -- cycle;
%axis
\draw[->] (-1.2,0) -- (1.2,0) node[right,fill=white] {{\footnotesize $ x $}};
\draw[->] (0,-1.2) -- (0,1.2) node[above,fill=white] {{\footnotesize $ y $}};
\tkzDefPoint(0,0){O}
\tkzDefPoint(cos(2*pi/9),0){D}
\tkzDefPoint(0,-sin(2*pi/9)){F}
\tkzDefPoint(0,sin(2*pi/9)){C}
\coordinate (A) at (40:1);
\coordinate (B) at (320:1);
\tkzDrawSegments(O,A O,B)
\draw circle (1);
\tkzText(.3,.1){{\footnotesize $ \omega $}}
\tkzText(.35,-.1){{\footnotesize $ -\omega $}}
\tkzText(-.2,.27){{\footnotesize $ 360^{\mathrm{o}}-\omega $}}
\draw[dashed] (A) -- (B);
\draw[dashed] (B) -- (F);
\draw[dashed] (A) -- (C);
\tkzDrawPoints[size=7,fill=white](A,B,C,D,F)
\tkzLabelPoint[left](C){{\footnotesize $ y $}}
\tkzLabelPoint[left](F){{\footnotesize $ -y $}}
\tkzLabelPoint[above right](A){{\footnotesize $ M(x,y) $}}
\tkzLabelPoint[below right](B){{\footnotesize $ N(x,-y) $}}
\tkzLabelPoint[below left](O){$ O $}
\end{tikzpicture}
\end{WrapText1}
\begin{itemize}[itemsep=0mm]
\item $ \hm{\left( -\omega\right) }=\hm{\left( 360\degree-\omega\right) }=-\hm{\omega} $
\item $ \syn{\left( -\omega\right) }=\syn{\left( 360\degree-\omega\right) }=\syn{\omega} $
\item $ \ef{\left( -\omega\right) }=\ef{\left( 360\degree-\omega\right) }=-\ef{\omega} $
\item $ \syf{\left( -\omega\right) }=\syf{\left( 360\degree-\omega\right) }=-\syf{\omega} $
\end{itemize}
Οι γωνίες με άθροισμα $ 360\degree $ καθώς και οι αντίθετες έχουν ίσα συνημίτονα και αντίθετους όλους τους υπόλοιπους τριγωνομετρικούς αριθμούς. Τα σημεία $ M,N $ του τριγωνομετρικού κύκλου, των γωνιών $ \omega $ και $ 360\degree-\omega $ αντίστοιχα, είναι συμμετρικα ως προς τον άξονα $ x'x $ και κατά συνέπεια έχουν αντίθετες τεταγμένες. Τα σημεία του κύκλου των γωνιών $ 360\degree-\omega $ και $ -\omega $ καθώς και οι ακτίνες τους ταυτίζονται.
\end{minipage}
\item \textbf{Συμπληρωματικές γωνίες}\\
Η συμπληρωματική γωνία μιας οξείας γωνίας $ \omega $ θα είναι της μορφής $ 90\degree-\omega $ η οποία ανήκει και αυτή στο 1\textsuperscript{ο} τεταρτημόριο. Οι τριγωνομετρικοί αριθμοί τους συνδέονται από τις παρακάτω σχέσεις :\\
\begin{minipage}{\linewidth}\mbox{}\\
\vspace{-1cm}
\begin{WrapText2}{13}{5cm}
\begin{tikzpicture}[>=latex,scale=2.5]
\clip (-.35,-.3) rectangle (1.4,1.4);
\draw[fill=\xrwma!10] (0,0) -- (.2,0) arc (0:30:.2) -- cycle;
\draw[fill=\xrwma!30] (0,0) -- (.15,0) arc (0:60:.15) -- cycle;
%axis
\draw[->] (-1.2,0) -- (1.2,0) node[right,fill=white] {{\footnotesize $ x $}};
\draw[->] (0,-1.2) -- (0,1.2) node[above,fill=white] {{\footnotesize $ y $}};
\tkzDefPoint(0,0){O}
\tkzDefPoint(cos(pi/6),0){D}
\tkzDefPoint(0,sin(pi/6)){C}
\tkzDefPoint(cos(pi/3),0){E}
\tkzDefPoint(0,sin(pi/3)){F}
\coordinate (A) at (30:1);
\coordinate (B) at (60:1);
\tkzDrawSegments(O,A O,B)
\draw circle (1);
\tkzText(.3,.07){{\footnotesize $ \omega $}}
\tkzText(-.1,.27){{\footnotesize $ 90^{\mathrm{o}}-\omega $}}
\draw[dashed] (A) -- (B);
\draw[dashed] (B) -- (F);
\draw[dashed] (B) -- (E);
\draw[dashed] (A) -- (C);
\draw[dashed] (A) -- (D);
\draw (-.3,-.3) -- (.8,.8);
\draw[-latex] (-.1,.23) -- (0.12,0.02);
\tkzDrawPoints[size=7,fill=white](A,B,C,D,E,F)
\tkzLabelPoint[left](C){{\footnotesize $ y_{\!_M} $}}
\tkzLabelPoint[below](D){{\footnotesize $ x_{\!_M} $}}
\tkzLabelPoint[left](F){{\footnotesize $ y_{\!_N} $}}
\tkzLabelPoint[below](E){{\footnotesize $ x_{\!_N} $}}
\tkzLabelPoint[above right](A){{\footnotesize $ M(x,y) $}}
\tkzLabelPoint[above right](B){{\footnotesize $ N(y,x) $}}
\tkzLabelPoint[below left](O){$ O $}
\tkzText(1,.75){{\footnotesize $ y=x $}}
\end{tikzpicture}
\end{WrapText2}
\begin{multicols}{2}
\begin{itemize}[itemsep=0mm,leftmargin=3mm]
\item $ \hm{\left( 90\degree-\omega\right) }=\syn{\omega} $
\item $ \syn{\left( 90\degree-\omega\right) }=\hm{\omega} $
\item $ \ef{\left( 90\degree-\omega\right) }=\syf{\omega} $
\item $ \syf{\left( 90\degree-\omega\right) }=\ef{\omega} $
\end{itemize}
\end{multicols}
Για δύο συμπληρωματικές γωνίες έχουμε οτι το ημίτονο της μιας είναι ίσο με το συνημίτονο της άλλης και η εφαπτομένη της μιας είναι ίση με τη συνεφαπτομένη της άλλης. Τα σημεία $ M,N $ του τριγωνομετρικού κύκλου, των γωνιών $ \omega $ και $ 90\degree-\omega $ αντίστοιχα, είναι συμμετρικα ως προς την ευθεία $ y=x $ οπότε έχουν συμμετρικές συντεταγμένες.
\end{minipage}
\item \textbf{Γωνίες με διαφορά $ \mathbold{90\degree} $}\\
\begin{minipage}{\linewidth}\mbox{}\\
\vspace{-5mm}
\begin{WrapText1}{9}{4.7cm}
\begin{tikzpicture}[>=latex,scale=1.8]
\clip (-1.25,-.3) rectangle (1.5,1.4);
\draw[fill=\xrwma!30] (0,0) -- (.2,0) arc (0:30:.2) -- cycle;
\draw[fill=\xrwma!50] (0,0) -- (.15,0) arc (0:120:.15) -- cycle;
%axis
\draw[->] (-1.2,0) -- (1.2,0) node[right,fill=white] {{\footnotesize $ x $}};
\draw[->] (0,-1.2) -- (0,1.2) node[above,fill=white] {{\footnotesize $ y $}};
\tkzDefPoint(0,0){O}
\tkzDefPoint(cos(pi/6),0){D}
\tkzDefPoint(0,sin(pi/6)){C}
\tkzDefPoint(cos(2*pi/3),0){E}
\tkzDefPoint(0,sin(2*pi/3)){F}
\coordinate (A) at (30:1);
\coordinate (B) at (120:1);
\tkzDrawSegments(O,A O,B)
\draw circle (1);
\tkzText(.3,.07){{\footnotesize $ \omega $}}
\tkzText(0.17,.27){{\footnotesize $ 90^{\mathrm{o}}+\omega $}}
\draw[dashed] (B) -- (F);
\draw[dashed] (B) -- (E);
\draw[dashed] (A) -- (C);
\draw[dashed] (A) -- (D);
\draw[->] (0.16,0.22) -- (0.04,0.06);
\tkzDrawPoints[size=7,fill=white](A,B,C,D,E,F)
\tkzLabelPoint[left](C){{\footnotesize $ y_{\!_M} $}}
\tkzLabelPoint[below](D){{\footnotesize $ x_{\!_M} $}}
\tkzLabelPoint[right](F){{\footnotesize $ y_{\!_N} $}}
\tkzLabelPoint[below](E){{\footnotesize $ x_{\!_N} $}}
\tkzLabelPoint[above right,xshift=-2mm,fill=white,inner sep=.2mm,yshift=2mm](A){{\footnotesize $ M(x,y) $}}
\tkzLabelPoint[above left](B){{\footnotesize $ N(-y,x) $}}
\tkzLabelPoint[below left](O){$ O $}
\tkzMarkRightAngle[size=.08](A,O,B)
\end{tikzpicture}
\end{WrapText1}
Γωνίες οι οποίες διαφέρουν κατά $ 90\degree $ έχουν τη μορφή $ \omega $ και $ 90\degree+\omega $. Οι τριγωνομετρικοί αριθμοί της γωνίας $ 90\degree+\omega $ δίνονται από τις παρακάτω σχέσεις :
\begin{multicols}{2}
\begin{itemize}[itemsep=0mm,leftmargin=2mm]
\item $ \hm{\left( 90\degree+\omega\right) }=\syn{\omega} $
\item $ \syn{\left( 90\degree+\omega\right) }=-\hm{\omega} $
\item $ \ef{\left( 90\degree+\omega\right) }=-\syf{\omega} $
\item $ \syf{\left( 90\degree+\omega\right) }=-\ef{\omega} $
\end{itemize}
\end{multicols}
Για δύο γωνίες με διαφορά $ 90\degree $ ισχύει οτι το ημίτονο της μιας είναι ίσο με το συνημίτονο της άλλης, ενώ συνημίτονο, εφαπτομένη και συνεφαπτομένη της πρώτης γωνίας είναι αντίθετα με τα ημίτονο, συνεφαπτομένη και εφαπτομένη αντίστοιχα, της δεύτερης.
\end{minipage}
\item \textbf{Γωνίες με διαφορά {\boldmath{$ 270\degree $}}}\\
Η γωνία η οποία διαφέρει κατά $ 270\degree $ από μια γωνία $ \omega $ θα είναι της μορφής $ 270\degree+\omega $. Για τον υπολογισμό των τριγωνομετρικών αριθμών της χρησιμοποιούμε τους παρακάτω μετασχηματισμούς :\\
\wrapr{-10mm}{9}{4.3cm}{-1mm}{\begin{tikzpicture}[>=latex,scale=1.5]
\draw[fill=\xrwma!10] (0,0) -- (.2,0) arc (0:30:.2) -- cycle;
\draw[fill=\xrwma!30] (0,0) -- (.15,0) arc (0:300:.15) -- cycle;
%axis
\draw[-latex] (-1.2,0) -- (1.2,0) node[right,fill=white] {{\footnotesize $ x $}};
\draw[-latex] (0,-1.2) -- (0,1.2) node[above,fill=white] {{\footnotesize $ y $}};
\tkzDefPoint(0,0){O}
\tkzDefPoint(cos(pi/6),0){D}
\tkzDefPoint(cos(10*pi/6),0){E}
\tkzDefPoint(0,sin(10*pi/6)){F}
\tkzDefPoint(0,sin(pi/6)){C}
\coordinate (A) at (30:1);
\coordinate (B) at (300:1);
\tkzDrawSegments(O,A O,B)
\draw circle (1);
\tkzText(.33,.1){{\footnotesize $ \omega $}}

\tkzText(-.22,.27){{\footnotesize $ 270^{\mathrm{o}}+\omega $}}
\draw[dashed] (A) -- (D) node[anchor=north]{{\footnotesize $ x_{_{\!M}} $}};
\draw[dashed] (B) -- (E) node[yshift=-2.4mm,xshift=-1mm]{{\footnotesize $ x_{_{\!N}} $}};
\draw[dashed] (A) -- (C);
\draw[dashed] (B) -- (F);
\tkzDrawPoints[size=7,fill=white](A,B,C,D,E,F)
\tkzLabelPoint[left](C){{\footnotesize $ y_{_{\!M}} $}}
\tkzLabelPoint[left](F){{\footnotesize $ y_{_{\!N}} $}}
\tkzLabelPoint[above right](A){{\footnotesize $ M(x,y) $}}
\tkzLabelPoint[below right](B){{\footnotesize $ N(y,-x) $}}
\tkzLabelPoint[below left,xshift=-.3mm](O){$ O $}
\end{tikzpicture}}{
\begin{multicols}{2}
\begin{itemize}[itemsep=0mm,leftmargin=2mm]
\item $ \hm{\left( 270\degree+\omega\right) }=-\syn{\omega} $
\item $ \syn{\left( 270\degree+\omega\right) }=\hm{\omega} $
\item $ \ef{\left( 270\degree+\omega\right) }=-\syf{\omega} $
\item $ \syf{\left( 270\degree+\omega\right) }=-\ef{\omega} $
\end{itemize}
\end{multicols}
Για δύο γωνίες με διαφορά $ 270\degree $ ισχύει οτι το συνημίτονο της μιας είναι ίσο με το ημίτονο της άλλης, ενώ το ημίτονο, η εφαπτομένη και η συνεφαπτομένη της πρώτης είναι αντίθετα με το συνημίτονο, τη συνεφαπτομένη και την εφαπτομένη της δεύτερης αντίστοιχα.}
\item \textbf{Γωνίες με άθροισμα {\boldmath{$ 270\degree $}}}\\
Η γωνία η οποία έχει άθροισμα $ 270\degree $ με μια γωνία $ \omega $ θα γράφεται ως $ 270\degree-\omega $. Οι τριγωνομετρικοί αριθμοί αυτής δίνονται από τους παρακάτω τύπους :\\
\wrapr{-10mm}{9}{4.3cm}{-3mm}{\begin{tikzpicture}[>=latex,scale=1.5]
\draw[fill=\xrwma!50] (0,0) -- (.2,0) arc (0:30:.2) -- cycle;
\draw[fill=\xrwma!30] (0,0) -- (.15,0) arc (0:240:.15) -- cycle;
%axis
\draw[-latex] (-1.2,0) -- (1.2,0) node[right,fill=white] {{\footnotesize $ x $}};
\draw[-latex] (0,-1.2) -- (0,1.2) node[above,fill=white] {{\footnotesize $ y $}};
\tkzDefPoint(0,0){O}
\tkzDefPoint(cos(pi/6),0){D}
\tkzDefPoint(cos(8*pi/6),0){E}
\tkzDefPoint(0,sin(8*pi/6)){F}
\tkzDefPoint(0,sin(pi/6)){C}
\coordinate (A) at (30:1);
\coordinate (B) at (240:1);
\tkzDrawSegments(O,A O,B)
\draw circle (1);
\tkzText(.33,.1){{\footnotesize $ \omega $}}

\tkzText(-.22,.27){{\footnotesize $ 270^{\mathrm{o}}-\omega $}}
\draw[dashed] (A) -- (D) node[anchor=north]{{\footnotesize $ x_{_{\!M}} $}};
\draw[dashed] (B) -- (E) node[yshift=-2.4mm,xshift=-2mm]{{\footnotesize $ x_{_{\!N}} $}};
\draw[dashed] (A) -- (C);
\draw[dashed] (B) -- (F);
\tkzDrawPoints[size=7,fill=white](A,B,C,D,E,F)
\tkzLabelPoint[left](C){{\footnotesize $ y_{_{\!M}} $}}
\tkzLabelPoint[right](F){{\footnotesize $ y_{_{\!N}} $}}
\tkzLabelPoint[above right](A){{\footnotesize $ M(x,y) $}}
\tkzLabelPoint[below,xshift=-3mm](B){{\footnotesize $ N(-y,-x) $}}
\tkzLabelPoint[below right,xshift=-.5mm](O){$ O $}
\end{tikzpicture}}{
\begin{multicols}{2}
\begin{itemize}[itemsep=0mm,leftmargin=3mm]
\item $ \hm{\left( 270\degree-\omega\right) }=-\syn{\omega} $
\item $ \syn{\left( 270\degree-\omega\right) }=-\hm{\omega} $
\item $ \ef{\left( 270\degree-\omega\right) }=\syf{\omega} $
\item $ \syf{\left( 270\degree-\omega\right) }=\ef{\omega} $
\end{itemize}
\end{multicols}
Για δύο γωνίες με άθροισμα $ 270\degree $ ισχύει οτι το ημίτονο και συνημίτονο της μιας είναι αντίθετα με το συνημίτονο και ημίτονο της άλλης αντοίστοιχα, ενώ η εφαπτομένη και η συνεφαπτομένη της πρώτης είναι ίση με τη συνεφαπτομένη και την εφαπτομένη της δεύτερης αντίστοιχα.}
\item \textbf{Γωνίες με διαφορά $ \mathbold{\kappa\cdot360\degree} $}\\
Εαν στρέψουμε μια γωνία $ \omega $ κατά γωνία της μορφής $ \kappa\cdot360\degree $ με $ \kappa\in\mathbb{Z} $ δηλαδή ακέραια πολλαπλάσια ενός κύκλου προκύπτει γωνία του τύπου $ \kappa\cdot360\degree+\omega $. Γωνίες αυτής της μορφής διαφέρουν κατά πολλαπλάσια ενός κύκλου. Οι τριγωνομετρικοί αριθμοί των δύο γωνιών συνδέονται με τις παρακάτω σχέσεις :\\
\begin{minipage}{\linewidth}\mbox{}\\
\vspace{-1cm}
\begin{WrapText2}{12}{4.7cm}
\newcommand\bigangle[2][]{% 
\draw[->,domain=0:#2,variable=\t,samples=200,>=latex,#1]
plot ({(\t+#2)*cos(\t)/(#2*10)},
{(\t+#2)*sin(\t)/(#2*10)})	;}
\begin{tikzpicture}[>=latex,scale=1.5]
\draw[fill=\xrwma!30] (0,0) -- (.2,0) arc (0:40:.2) -- cycle;
%axis
\draw[->] (-1.2,0) -- (1.2,0) node[right,fill=white] {{\footnotesize $ x $}};
\draw[->] (0,-1.2) -- (0,1.2) node[above,fill=white] {{\footnotesize $ y $}};
\tkzDefPoint(0,0){O}
\tkzDefPoint(cos(2*pi/9),0){D}
\tkzDefPoint(0,sin(2*pi/9)){F}
\coordinate (A) at (40:1);
\coordinate (B) at (400:1);
\tkzDrawSegment(O,A)
\draw circle (1);
\tkzText(.3,.1){\footnotesize$ \omega $}
\tkzText(-.25,.27){{\footnotesize $ 360^{\mathrm{o}}+\omega $}}
\draw[dashed] (A) -- (D) node[anchor=north]{{\footnotesize $ x $}};
\draw[dashed] (A) -- (F);
\tkzDrawPoints[size=7,fill=white](A,D,F)
\tkzLabelPoint[left](F){{\footnotesize $ y $}}
\tkzLabelPoint[above right](A){\footnotesize$ M(x,y) $}
\tkzLabelPoint[below left](O){$ O $}
\bigangle{400}
\end{tikzpicture}
\end{WrapText2}
\begin{multicols}{2}
\begin{itemize}[itemsep=0mm,leftmargin=0mm]
\item $ \hm{\left( \kappa\cdot360\degree+\omega\right)}=\hm{\omega} $
\item $ \syn{\left(
\kappa\cdot360\degree+\omega\right)}=\syn{\omega}$
\item $ \ef{\left( \kappa\cdot360\degree+\omega\right) }=\ef{\omega} $
\item $ \syf{\left( \kappa\cdot360\degree+\omega\right) }=\syf{\omega} $
\end{itemize}
\end{multicols}
Οι γωνίες με διαφορά $ \kappa\cdot360\degree $ έχουν ίσους όλους τους τριγωνομετρικούς τους αριθμούς καθώς ταυτίζονται τα σημεία των γωνιών πάνω στον τριγωνομετρικό κύκλο και οι ακτίνες των γωνιών.
\end{minipage}
\end{enumerate}\mbox{}\\\\\\
Στον ακόλουθο συγκεντρωτικό πίνακα βλέπουμε όλες τις σχέσεις μεταξύ δύο γωνιών $ \varphi $ και $ \omega $ καθώς και μεταξύ των τριγωνομετρικών αριθμών τους, με τις οποίες γίνεται η αναγωγή στο 1\tss{ο} τεταρτημόριο.
\begin{center}
\begin{longtable}{c|c|cccc}
\hline 
\rule[-2ex]{0pt}{5ex} \bmath{Σχέση γωνίας $ \varphi $ με την $ \omega $} & \bmath{Συμβολισμός $ \varphi= $}  & \bmath{$ \hm{\varphi} $} & \bmath{$ \syn{\varphi} $} & \bmath{$ \ef{\varphi} $} & \bmath{$ \syf{\varphi} $} \\ 
\hhline{======} 
\rule[-2ex]{0pt}{5ex} Αντίθετη & $ -\omega $ & $ -\hm{\omega} $ & $ \syn{\omega} $ & $ -\ef{\omega} $ & $ -\syf{\omega} $ \\  
\rule[-2ex]{0pt}{5ex} Παραπληρωματική & $ 180\degree-\omega $ & $ \hm{\omega} $ & $ -\syn{\omega} $ & $ -\ef{\omega} $ & $ -\syf{\omega} $ \\  
\rule[-2ex]{0pt}{5ex} Με διαφορά $180\degree$ & $ 180\degree+\omega $ & $ -\hm{\omega} $ & $ -\syn{\omega} $ & $ \ef{\omega} $ & $ \syf{\omega} $ \\  
\rule[-2ex]{0pt}{5ex} Συμπληρωματική & $ 90\degree-\omega $ & $ \syn{\omega} $ & $ \hm{\omega} $ & $ \syf{\omega} $ & $ \ef{\omega} $ \\ 
\rule[-2ex]{0pt}{5ex} Με διαφορά $ 90\degree $  & $ 90\degree+\omega $ & $ \syn{\omega} $ & $ -\hm{\omega} $ & $ -\syf{\omega} $ & $ -\ef{\omega} $ \\  
\rule[-2ex]{0pt}{5ex} Με άθροισμα $ 270\degree $ & $ 270\degree-\omega $ & $ -\syn{\omega} $ & $ -\hm{\omega} $ & $ \syf{\omega} $ & $ \ef{\omega} $ \\  
\rule[-2ex]{0pt}{5ex} Με διαφορά $ 270\degree $ & $ 270\degree+\omega $ & $ -\syn{\omega} $ & $ \hm{\omega} $ & $ -\syf{\omega} $ & $ -\ef{\omega} $ \\ 
\rule[-2ex]{0pt}{5ex} Με άθροισμα $ 360\degree $ & $ 360\degree-\omega $ & $ -\hm{\omega} $ & $ \syn{\omega} $ & $ -\ef{\omega} $ & $ -\syf{\omega} $ \\ 
\rule[-2ex]{0pt}{5ex} Με διαφορά $ \kappa\cdot 360\degree $ & $ \kappa\cdot360\degree+\omega $ & $ \hm{\omega} $ & $ \syn{\omega} $ & $ \ef{\omega} $ & $ \syf{\omega} $ \\ 
\hline 
\end{longtable} 
\end{center}
\section{Τριγωνομετρικές εξισώσεις}
\orismoi
\Orismos{Τριγωνομετρική εξίσωση}
Τριγωνομετρική ονομάζεται κάθε εξίσωση στην οποία η μεταβλητή περιέχεται σε έναν τουλάχιστον τριγωνομετρικό αριθμό. Οι βασικές τριγωνομετρικές εξισώσεις είναι της μορφής:
\[ \hm{x}=a\ \ ,\ \ \syn{x}=a\ \ ,\ \ \ef{x}=a\ \ ,\ \ \syf{x}=a \]
\thewrhmata
\Thewrhma{Τριγωνομετρικές εξισώσεις}
Οι λύσεις των βασικών τριγωνομετρικών εξισώσεων δίνονται από τους παρακάτω τύπους :
\begin{enumerate}[itemsep=0mm,label=\bf\arabic*.]
\item \textbf{Η εξίσωση {\boldmath$ \hm{x}=a $}}\\
Σε κάθε εξίσωση της μορφής $ \hm{x}=a $ διακρίνουμε τις εξής περιπτώσεις :
\begin{rlist}
\item Αν $ a\in[-1,1] $ τότε θα υπάρχει γωνία $ \theta\in[0,2\pi) $ ώστε η εξίσωση να έχει τα παρακάτω σύνολα λύσεων :
\[ x=2\kappa\pi+\theta\ \textrm{ή} \ x=2\kappa\pi+(\pi-\theta)\ \ ,\ \ \kappa\in\mathbb{Z}\]
\item Αν $ a\in(-\infty,-1)\cup(1,+\infty) $ τότε η εξίσωση είναι αδύνατη.
\end{rlist}
\item \textbf{Η εξίσωση {\boldmath$ \syn{x}=a $}}\\
Σε κάθε εξίσωση της μορφής $ \syn{x}=a $ διακρίνουμε τις εξής περιπτώσεις :
\begin{rlist}
\item Αν $ a\in[-1,1] $ τότε θα υπάρχει γωνία $ \theta\in[0,2\pi) $ ώστε η εξίσωση να έχει τα παρακάτω σύνολα λύσεων :
\[ x=2\kappa\pi+\theta\ \textrm{ή} \ x=2\kappa\pi-\theta\ \ ,\ \ \kappa\in\mathbb{Z}\]
\item Αν $ a\in(-\infty,-1)\cup(1,+\infty) $ τότε η εξίσωση είναι αδύνατη.
\end{rlist}
\item \textbf{Η εξίσωση {\boldmath$ \ef{x}=a $}}\\
Σε κάθε εξίσωση της μορφής $ \ef{x}=a $ για κάθε τιμή του πραγματικού αριθμού $ a $ θα υπάρχει γωνία $ \theta\in\left(-\frac{\pi}{2},\frac{\pi}{2} \right)  $ ώστε οι λύσεις να δίνονται από τον τύπο :
\[ x=\kappa\pi+\theta\ \ ,\ \ \kappa\in\mathbb{Z} \]
\item \textbf{Η εξίσωση {\boldmath$ \syf{x}=a $}}\\
Σε κάθε εξίσωση της μορφής $ \syf{x}=a $ για κάθε τιμή του πραγματικού αριθμού $ a $ θα υπάρχει γωνία $ \theta\in\left(0,\pi\right) $ ώστε οι λύσεις να δίνονται από τον τύπο :
\[ x=\kappa\pi+\theta\ \ ,\ \ \kappa\in\mathbb{Z} \]
\end{enumerate}
%\chapter{Πολυώνυμα}
\section{Η έννοια του πολυωνύμου}
\orismoi
\Orismos{Μεταβλητή}
Μεταβλητή ονομάζεται το σύμβολο το οποίο χρησιμοποιούμε για εκφράσουμε έναν άγνωστο αριθμό. Η μεταβλητή μπορεί να βρίσκεται μέσα σε μια εξίσωση και γενικά σε μια αλγεβική παράσταση.
Συμβολίζεται με ένα γράμμα όπως $ a,\beta,x,y,\ldots $ κ.τ.λ.\\\\
\Orismos{ΜΟΝΏΝΥΜΟ}
Μονώνυμο ονομάζεται η ακέραια αλγεβρική παράσταση η οποία έχει μεταξύ των μεταβλητών μόνο την πράξη του πολλαπλασιασμού.
\[ \textrm{{\scriptsize Συντελεστής} }\longrightarrow a\cdot \undercbrace{x^{\nu_1}y^{\nu_2}\cdot \ldots\cdot z^{\nu_\kappa}}_{\textrm{κύριο μέρος}}\;\;,\;\;\nu_1,\nu_2,\ldots,\nu_\kappa\in\mathbb{N} \]
\begin{itemize}[itemsep=0mm]
\item Το γινόμενο των μεταβλητών ενός μονωνύμου ονομάζεται \textbf{κύριο μέρος}.
\item  Ο σταθερός αριθμός με τον οποίο πολλαπλασιάζουμε το κύριο μέρος ενός μονωνύμου ονομάζεται \textbf{συντελεστής}.
\item Τα μονώνυμα μιας μεταβλητής είναι της μορφής $ ax^\nu $, όπου $ a\in\mathbb{R} $ και $ \nu\in\mathbb{N} $.
\end{itemize}
\Orismos{ΠΟΛΥΏΝΥΜΟ}	Πολυώνυμο ονομάζεται η ακέραια αλγεβρική παράσταση η οποία είναι άθροισμα
ανόμοιων μονωνύμων.
\begin{itemize}[itemsep=0mm]
\item Κάθε μονώνυμο μέσα σ' ένα πολυώνυμο ονομάζεται \textbf{όρος} του πολυωνύμου.
\item Το πολυώνυμο με 3 όρους ονομάζεται \textbf{τριώνυμο}.
\item Οι αριθμοί ονομάζονται \textbf{σταθερά πολυώνυμα} ενώ το 0 \textbf{μηδενικό πολυώνυμο}.
\item  Κάθε πολυώνυμο συμβολίζεται με ένα κεφαλαίο γράμμα όπως : $ P, Q, A, B\ldots $ τοποθετώντας δίπλα από το όνομα μια παρένθεση η οποία περιέχει τις μεταβλητές του δηλαδή : $ P(x), Q(x,y), A(z,w), B(x_1,x_2,\ldots,x_\nu) $.
\item \textbf{Βαθμός} ενός πολυωνύμου ορίζεται ως ο μεγαλύτερος εκθέτης της κάθε μεταβλητής. Ο όρος που περιέχει τη μεταβλητή με το μεγαλύτερο εκθέτη ονομάζεται \textbf{μεγιστοβάθμιος}.
\item Τα πολυώνυμα μιας μεταβλητής τα γράφουμε κατά φθίνουσες δυνάμεις της μεταβλητής δηλαδή από τη μεγαλύτερη στη μικρότερη. Έχουν τη μορφή :
\end{itemize}
\[ P(x)=a_\nu x^\nu+a_{\nu-1}x^{\nu-1}+\ldots+a_1x+a_0 \]
\Orismos{ΤΙΜΉ ΠΟΛΥΩΝΎΜΟΥ}
Τιμή ενός πολυωνύμου $ P(x)=a_\nu x^\nu+a_{\nu-1}x^{\nu-1}+\ldots+a_1x+a_0 $ ονομάζεται ο πραγματικός αριθμός που προκύπτει ύστερα από πράξεις αν αντικαταστίσουμε τη μεταβλητή του πολυωνύμου με έναν αριθμό $ x_0 $. Συμβολίζεται με $ P(x_0) $ και είναι ίση με :
\[ P(x_0)=a_\nu x_0^\nu+a_{\nu-1}x_0^{\nu-1}+\ldots+a_1x_0+a_0 \]
\Orismos{ΡΊΖΑ ΠΟΛΥΩΝΎΜΟΥ}
Ρίζα ενός πολυωνύμου $ P(x)=a_\nu x^\nu+a_{\nu-1}x^{\nu-1}+\ldots+a_1x+a_0 $ ονομάζεται κάθε πραγματικός αριθμός $ \rho\in\mathbb{R} $ ο οποίος μηδενίζει το πολυώνυμο.
\[ P(\rho)=0 \]
\thewrhmata
\Thewrhma{Βαθμός πολυωνύμου}
Έστω δύο πολυώνυμα $ A(x)=a_\nu x^\nu+a_{\nu-1}x^{\nu-1}+\ldots+a_1x+a_0 $ και $ B(x)=\beta_\mu x^\mu+\beta_{\mu-1}x^{\mu-1}+\ldots+\beta_1x+\beta_0 $ βαθμών $ \nu $ και $ \mu $ αντίστοιχα με $ \nu\geq\mu $. Τότε ισχύουν οι παρακάτω προτάσεις :
\begin{rlist}
\item Ο βαθμός του αθροίσματος ή της διαφοράς $ A(x)\pm B(x) $ είναι μικρότερος ίσος του μέγιστου των βαθμών των πολυωνύμων $ A(x) $ και $ B(x) $ : $ \textrm{βαθμός}(A(x)+B(x))\leq\max\{\nu,\mu\} $.
\item Ο βαθμός του γινομένου $ A(x)\cdot B(x) $ ισούται με το άθροισμα των βαθμών των πολυωνύμων $ A(x) $ και $ B(x) $ : $ \textrm{βαθμός}(A(x)\cdot B(x))=\nu+\mu $.
\item Ο βαθμός του πηλίκου $ \pi(x) $ της διαίρεσης $ A(x):B(x) $ ισούται με τη διαφορά των βαθμών των πολυωνύμων $ A(x) $ και $ B(x) $ : $ \textrm{βαθμός}(A(x): B(x))=\nu-\mu $.
\item Ο βαθμός της δύναμης $ [A(x)]^\kappa $ του πολυωνύμου $ A(x) $ ισούται με το γινόμενο του εκθέτη $ \kappa $ με το βαθμό του $ A(x) $ : $ \textrm{βαθμός}([A(x)]^\kappa)=\nu\cdot\kappa $.
\end{rlist}
\Thewrhma{Ίσα πολυώνυμα}
Δύο πολυώνυμα $ A(x)=a_\nu x^\nu+a_{\nu-1}x^{\nu-1}+\ldots+a_1x+a_0 $ και $ B(x)=\beta_\mu x^\mu+\beta_{\mu-1}x^{\mu-1}+\ldots+\beta_1x+\beta_0 $ βαθμών $ \nu $ και $ \mu $ αντίστοιχα με $ \nu\geq\mu $ θα είναι μεταξύ τους ίσα αν και μόνο αν οι συντελεστές των ομοβάθμιων όρων τους είναι ίσοι.
\begin{gather*}
A(x)=B(x)\Leftrightarrow a_i=\beta_i\ ,\ \textrm{για κάθε }i=0,1,2,\ldots,\mu\\
\textrm{και }a_i=0\ ,\ \textrm{για κάθε }i=\mu+1,\mu+2,\ldots,\nu
\end{gather*}
Ένα πολυώνυμο $ A(x)=a_\nu x^\nu+a_{\nu-1}x^{\nu-1}+\ldots+a_1x+a_0 $ ισούται με το μηδενικό πολυώνυμο αν και μόνο αν όλοι του οι συντελεστές είναι μηδενικοί.
\[ A(x)=0\Leftrightarrow a_i=0 \ ,\ \textrm{για κάθε }i=0,1,2,\ldots,\nu\]
\newpage
\noindent
\Lymena
\begin{Methodos}[Τιμή πολυωνύμου]{5cm}
Αν $ A(x) $ είναι ένα πολυώνυμο μιας μεταβλητής τότε προκειμένου να υπολογίσουμε την τιμή του για δοσμένη τιμή της μεταβλητής του
\begin{bhma}
\item \textbf{Αντικατάσταση τιμών}\\
Αντικαθιστούμε την τιμή της μεταβλητής $ x $ που μας δίνεται στο πολυώνυμο, οπότε μετατρέπεται από αλγεβρική σε αριθμιτική παράσταση.
\item \textbf{Πράξεις}\\
Εκτελούμε τις πράξεις μέσα στην αριθμιτική παράσταση που προέκυψε με τη γνωστή σειρά και υπολογίζουμε το αποτέλεσμα.
\end{bhma}
\end{Methodos}
\Paradeigma{Υπολογισμόσ τιμήσ}
\textbf{Να υπολογιστεί η τιμή του παρακάτω πολυωνύμου}
{\boldmath  $ A(x)=x^3+4x^2+3x-7 $}
\textbf{εαν θέσουμε όπου {\boldmath$ x=-2 $}}.\\\\
\lysh\\
Αν θέσουμε όπου $ x=-2$ τότε προκύπτει η παρακάτω αριθμητική παράσταση :
\begin{align*}
A(-2)=(-2)^3+4(-2)^2+3(-2)-7=-8+4\cdot 4+3(-2)-7=-8+16-6-7=-5
\end{align*}
Η τιμή λοιπόν του πολυωνύμου για τις δοσμένες τιμές των μεταβλητών του θα είναι ίση με $ -41 $.\\\\
\Paradeigma{Υπολογισμόσ τιμήσ}
\textbf{Να υπολογιστεί η τιμή του παρακάτω πολυωνύμου}
{\boldmath $ P(x)=5x^3-3x^2+2x-4 $}
\textbf{εαν μας δίνεται οτι {\boldmath$ x=1$}}.\\\\
\lysh\\
Το πολυώνυμο που μας δίνεται είναι μιας μεταβλητής. Θέτοντας λοιπόν όπου $ x=1 $ η τιμή του θα συμβολιστεί με $ P(1) $. Θα έχουμε λοιπόν
\begin{align*} P(x)=5x^3-3x^2+2x-4\xRightarrow{x=1}P(1)&=5\cdot 1^3-3\cdot1^2+2\cdot1-4\\
&=5\cdot 1-3\cdot1+2\cdot1-4\\
&=5-3+2-4=0 
\end{align*}
Προέκυψε λοιπόν η τιμή του πολυωνύμου $ P(1)=0 $ όποτε ο αριθμός $ 1 $ είναι ρίζα του πολυωνύμου.
\begin{Methodos}[Αλλαγή μεταβλητήσ]{5cm}
Όπως και στην προηγούμενη μέθοδο αντικαταστήσαμε στη θέση των μεταβλητών σταθερούς αριθμούς με τον ίδιο τρόπο μπορούμε να θέσουμε στη θέση των αρχικών μεταβλητών, νέες μεταβλητές.
\begin{bhma}
\item \textbf{Αντικατάσταση}\\
Αντικαθιστούμε στη θέση των αρχικών μεταβλητών τις νέες μεταβλητές που μας δίνονται.
\item \textbf{Απλοποίηση}\\
Προκύπτει τότε μια νέα αλγεβρική παράσταση την οποία απλοποιούμε εκτελώνας όλες τις δυνατές πράξεις.
\end{bhma}
\end{Methodos}
\Paradeigma{Αλλαγή μεταβλητήσ}
\textbf{Δίνεται το πολυώνυμο {\boldmath$ P(x)=2x^2-3x+5 $}. Να βρεθούν τα πολυώνυμα}
{\boldmath
\begin{multicols}{3}
\begin{brlist}
\item $ P(t) $
\item $ P(2x) $
\item $ P(-3s) $
\end{brlist}\end{multicols}}
\lysh
\begin{rlist}
\item Αντικαθιστώντας τη μεταβλητή $ t $ στη θέση της μεταβλητής $ x $ του πολυωνύμου $ P $ παρατηρούμε οτι γίνεται μόνο αλλαγή του συμβολισμού της πράγμα που σημαίνει οτι η δομή του πολυωνύμου δεν θα αλλάξει. Έχουμε λοιπόν
\[ P(x)=2x^2-3x+5\xRightarrow{x\rightarrow t}P(t)=2t^2-3t+5 \]
\item Θέτοντας στη θέση της μεταβλητής $ x $ το μονώνυμο $ 2x $ στο πολυώνυμο $ P $ θα προκύψει
\begin{align*}
 P(x)=2x^2-3x+5\xRightarrow{x\rightarrow 2x}P(2x)&=2(2x)^2-3\cdot(2x)+5\\
&=2\cdot 4x^2-6x+5=8x^2-6x+5
\end{align*}
\item Θέτοντας όπου $ x $ το μονώνυμο $ -3s $ έχουμε
\begin{align*}
 P(x)=2x^2-3x+5\xRightarrow{x\rightarrow -3s}P(-3s)&=2(-3s)^2-3\cdot(-3s)+5\\
&=2\cdot 9s^2+9s+5=18s^2+9s+5
\end{align*}
\end{rlist}
\begin{Methodos}[Ισότητα πολυωνύμων]{4cm}
Γνωρίζουμε ότι δύο πολυώνυμα είναι ίσα αν και μόνο αν οι συντελεστές των ομοβάθμιων όρων του είναι ίσοι. Έτσι για τον υπολογισμό των συντελεστών των πολυωνύμων :
\begin{bhma}
\item \textbf{Ίσοι συντελεστές}\\
Εξισώνουμε τους συντελεστές των ομοβάθμιων όρων τους. Αν κάποιο πολυώνυμο έχει μεγαλύτερο βαθμό τότε οι συντελεστές των παραπανίσιων όρων ισούνται με το $ 0 $.
\item \textbf{Εύρεση συντελεστών}\\
Λύνουμε τις εξισώσεις ή τα συστήματα εξισώσεων που θα προκύψουν οπότε προσδιορίζουμε τους ζητούμενους συντελεστές.
\end{bhma}
\end{Methodos}
\Paradeigma{Ισότητα πολυωνύμων}
\textbf{Δίνονται τα πολυώνυμα {\boldmath$ A(x)=x^3+\beta x^2-4x+\delta $} και {\boldmath$ B(x)=a x^3-3x^2+\gamma x-7 $}. Να υπολογίσετε τους πραγματικούς αριθμούς {\boldmath$ a,\beta,\gamma,\delta $} ώστε τα πολυώνυμα {\boldmath$ A(x),B(x) $} να είναι μεταξύ τους ίσα.}\\\\
\lysh\\
Για να ισχύει η ισότητα $ A(x)=B(x) $ θα πρέπει να έχουμε
\[ A(x)=B(x)\Leftrightarrow x^3+\beta x^2-4x+\delta=a x^3-3x^2+\gamma x-7\Leftrightarrow a=1\ ,\ \beta=-3\ ,\ \gamma=-4\ , \ \delta=-7 \]
\Paradeigma{Ισότητα πολυωνύμων}
\textbf{Δίνονται τα παρακάτω πολυώνυμα {\[ \boldmath A(x)=(a-2)x^4+3x^3-2x^2+(\beta+\gamma)x+2\ \textrm{και}\  B(x)=(2\beta-\gamma)x^3-\delta x^2+2 \]}Να υπολογίσετε τους πραγματικούς αριθμούς {\boldmath$ a,\beta,\gamma,\delta $} ώστε τα πολυώνυμα {\boldmath$ A(x),B(x) $} να είναι μεταξύ τους ίσα.}\\\\
\lysh\\
Προκειμένου να είναι τα δύο πολυώνυμα ίσα θα πρέπει να να έχουν ίσους συντελεστές οπότε προκύπτουν οι παρακάτω ισότητες :
\begin{gather*}
A(x)=B(x)\Leftrightarrow (a-2)x^4+3x^3-2x^2+(\beta+\gamma)x+2=(2\beta-\gamma)x^3-\delta x^2+2\Leftrightarrow\\
a-2=0\quad ,\quad \systeme[\beta\gamma]{\ 2\beta-\gamma=3, \ \beta+\gamma=0}\quad ,\quad -\delta=-2
\end{gather*}
Έπειτα από τη λύση των εξισώσεων και του γραμμικού συστήματος παίρνουμε τους αριθμούς : $ a=2\ ,\ \beta=2\ ,\ \gamma=-1 $ και $ \delta=2 $.
\begin{Methodos}[Πρόσθεση - Αφαίρεση πολυωνύμων]{3cm}
Για να προσθέσουμε ή να αφαιρέσουμε δύο ή περισσότερα πολυώνυμα μεταξύ τους εκτελούμε τις πράξεις μεταξύ των συντελεστών των όμοιων μονωνύμων τους κάνοντας αναγωγή ομοίων όρων.
\end{Methodos}
\Paradeigma{Πρόσθεση πολυωνύμων}
\textbf{Δίνονται τα πολυώνυμα {\boldmath$ A(x)=x^3-5x^2+2x+1 $} και {\boldmath$ B(x)=3x^3-x^2+5x+4$}. Να βρεθούν τα πολυώνυμα}
{\boldmath
\begin{multicols}{2}
\begin{brlist}
\item $ A(x)+B(x) $
\item $ B(x)-A(x) $
\end{brlist}
\end{multicols}}
\lysh\\
Όπως και στην πρόσθεση έτσι και στην αφαίρεση των πολυωνύμων θα χρειαστεί να ξεχωρίσουμε τους όμοιους μεταξύ τους όρους.
\begin{rlist}
\item Έχουμε λοιπόν
\begin{gather*}
A(x)+B(x)=\left( x^3-5x^2+2x+1\right) +\left( 3x^3-x^2+5x+4\right)=\\
x^3+3x^3-5x^2-x^2+2x+5x+1+4=4x^3-6x^2+7x+5
\end{gather*}
\item Για τη διαφορά των δύο πολυωνύμων θα χρειαστεί να αλλάξουμε τα πρόσημα του δεύτερου πολυωνύμου.
\begin{gather*}
B(x)-A(x)=\left( 3x^3-x^2+5x+4\right)-\left( x^3-5x^2+2x+1\right)=\\
3x^3-x^2+5x+4-x^3+5x^2-2x-1=2x^3+4x^2+3x+3
\end{gather*}
\end{rlist}
\begin{Methodos}[Πολλαπλασιασμόσ πολυωνύμων]{3cm}
Για τον πολλαπλασιασμό πολυωνύμων κάνουμε χρήση της επιμεριστικής ιδιότητας.
\begin{bhma}
\item \textbf{Πολλαπλασιασμός}\\
Για να πολλαπλασιάσουμε δύο πολυώνυμα μεταξύ τους πολλαπλασιάζουμε κάνοντας χρήση της επιμεριστικής ιδιότητας κάθε όρο του πρώτου με κάθεναν από τους όρους του δεύτερο πολυωνύμου.
\item \textbf{Αναγωγή ομοίων όρων}\\
Αφού βρεθεί το ανάπτυγμα του γινομένου προσθέτουμε αν υπάρχουν τους όμοιους όρους που θα προκύψουν μεταξύ τους ώστε να απλοποιηθεί η παράσταση.
\end{bhma}
\end{Methodos}
\Paradeigma{Πολλαπλασιασμόσ πολυωνύμων}
\textbf{Να υπολογιστεί το γινόμενο {\boldmath$ A(x)\cdot B(x) $} των πολυωνύμων {\boldmath$ A(x)=x^2-4x+3 $} και {\boldmath$ B(x)=3x+5 $}.}\\\\
\lysh\\
Το γινόμενο των πολυωνύμων θα έχει ως εξής :
\begin{gather*}
A(x)\cdot B(x)=\left( x^2-4x+3\right)\cdot(3x+5)=x^2\cdot 3x+x^2\cdot 5-4x\cdot 3x-4x\cdot 5+3\cdot 3x+3\cdot 5\\=
3x^3+5x^2-12x^2-20x+9x+15=3x^3-7x^2-11x+15
\end{gather*}
\begin{Methodos}[Βαθμός πολυωνύμου]{4cm}
Ο βαθμός ενός πολυωνύμου καθορίζεται από το μέγιστο εκθέτη της μεταβλητής του. Για να βρεθεί ο βαθμός ενός πολυωνύμου
\end{Methodos}
\section{Διαίρεση πολυωνύμων}
\orismoi
\Orismos{ευκλειδεια διαιρεση πολυωνυμων}
Ευκλείδεια διαίρεση ονομάζεται η διαδικασία με την οποία για κάθε ζεύγος πολυωνύμων $ \varDelta(x),\delta(x) $ (Διαιρετέος και διαιρέτης αντίστοιχα) προκύπτουν μοναδικά πολυώνυμα $ \pi(x),\upsilon(x) $ (πηλίκο και υπόλοιπο) για τα οποία ισχύει :
\[ \varDelta(x)=\delta(x)\cdot\pi(x)+\upsilon(x) \]
\begin{itemize}[itemsep=0mm]
\item Η παραπάνω ισότητα ονομάζεται \textbf{ταυτότητα της ευκλείδειας διαίρεσης}.
\item Εαν $ \upsilon(x)=0 $ τότε η διαίρεση ονομάζεται \textbf{τέλεια} ενώ η ταυτότητα της διαίρεσης είναι
\[ \varDelta(x)=\delta(x)\cdot\pi(x) \]
\item Στην τέλεια διαίρεση τα πολυώνυμα $ \delta(x),\pi(x) $ ονομάζονται \textbf{παράγοντες} ή \textbf{διαιρέτες}.
\end{itemize}
\thewrhmata
\Thewrhma{Διαίρεση με {\MakeLowercase{$ \mathbold{x-\rho} $}}}
Το υπόλοιπο της διαίρεσης ενός πολυωνύμου $ P(x) $ με διαρέτη ένα πολυώνυμο 1\tss{ου} βαθμού της μορφής $ x-\rho $ ισούται με την τιμή του πολυωνύμου $ P(x) $ για $ x=\rho $.
\[ \upsilon=P(\rho) \]
\Thewrhma{Ρίζα πολυωνύμου}
Ένα πολυώνυμο $ P(x) $ έχει παράγοντα ένα πολυώνυμο της μορφής $ x-\rho $ αν και μόνο αν ο πραγματικός αριθμός $ \rho $ είναι ρίζα του πολυωνύμου $ P(x) $.
\[ x-\rho\ \textrm{ παράγοντας }\ \Leftrightarrow P(\rho)=0 \]
\section{Διαίρεση με σχήμα Horner}
\section{Πολυωνυμικές εξισώσεις - Ανισώσεις}
\orismoi
\Orismos{Πολυωνυμικη εξισωση}
Πολυωνυμική εξίσωση ν-οστού βαθμού ονομάζεται κάθε πολυωνυμική εξίσωση της οποίας η αλγεβρική παράσταση είναι πολυώνυμο ν-οστού βαθμού.
\[ a_\nu x^\nu+a_{\nu-1}x^{\nu-1}+\ldots+a_1x+a_0=0 \]
όπου $ a_\kappa\in\mathbb{R}\;\;,\;\;\kappa=0,1,2,\ldots,\nu $. \textbf{Ρίζα} μιας πολυωνυμικής εξίσωσης ονομάζεται η ρίζα του πολυωνύμου της εξίσωσης.\\\\
\thewrhmata
\Thewrhma{Θεώρημα ακέραιων ριζών}
Αν ένας μη μηδενικός ακέραιος αριθμός $ \rho\neq0 $ είναι ρίζα μιας πολυωνυμικής εξίσωσης $ a_\nu x^\nu+a_{\nu-1}x^{\nu-1}+\ldots+a_1x+a_0=0 $ με ακέραιους συντελεστές $ a_\nu ,a_{\nu-1},\ldots,a_1,a_0\in\mathbb{Z} $ τότε ο αριθμός αυτός θα είναι διαιρέτης του σταθερού όρου $ a_0 $ του πολυωνύμου.
\section{Μη πολυωνυμικές εξισώσεις}
\orismoi
\Orismos{Κλασματικη εξίσωση}
Κλασματική ονομάζεται μια εξίσωση η οποία περιέχει τουλάχιστον μια ρητή αλγεβρική παράσταση. Γενικά έχει τη μορφή :
\[ \dfrac{P(x)}{Q(x)}+R(x)= 0\]
όπου $ P(x),Q(x),R(x) $ πολυώνυμα με $ Q(x)\neq0 $.\\\\
\Orismos{Άρρητη εξίσωση}
Άρρητη ονομάζεται κάθε εξίσωση που περιέχει τουλάχιστον μια άρρητη αλγεβρική παράσταση. Θα είναι
\[ \sqrt[\nu]{P(x)}+Q(x)=0 \]
όπου $ P(x),Q(x) $ πολυώνυμα με $ P(x)\geq0 $.
%\chapter{Εκθετική - Λογαριθμική συνάρτηση}
\section{Εκθετική συνάρτηση}
\orismoi
\Orismos{Εκθετική συνάρτηση}
Εκθετική ονομάζεται κάθε συνάρτηση $ f $ της οποίας ο τύπος αποτελεί δύναμη με θετική βάση, διάφορη της μονάδας και εκθέτη που περιέχει την ανεξάρτητη μετβλητή. Η απλή εκθετική συνάρτηση θα είναι της μορφής :
\[ f(x)=a^x\;\;,\;\;0<a\neq1 \]
\thewrhmata
\Thewrhma{Ιδιότητεσ εκθετικών συναρτήσεων}
Οι ιδιότητες των εκθετικών συναρτήσεων της μορφής $ f(x)=a^x $, με $ 0<a\neq1 $, είναι οι εξής. Σε ορισμένες ιδιότητες διακρίνουμε δύο περιπτώσεις για τη βάση $ a $ της συνάρτησης.
\begin{rlist}
\item Η συνάρτηση $ f $ έχει πεδίο ορισμού το σύνολο $ \mathbb{R} $.
\item Το σύνολο τιμών της είναι το σύνολο $ (0,+\infty) $ των θετικών πραγματικών αριθμών.
\item Η συνάρτηση δεν έχει ακρότατες τιμές.
\begin{enumerate}[itemsep=0mm,label=\bf\arabic*.,leftmargin=0cm]
\item[\textbf{A.}] \textbf{Για {\boldmath$ a>1 $}}
\begin{itemize}
\item Αν η βάση $ a $ της εκθετικής συνάρτησης είναι μεγαλύτερη της μονάδας τότε η συνάρτηση $ f(x)=a^x $ είναι γνησίως αυξουσα στο $ \mathbb{R} $.
\item Η συνάρτηση δεν έχει ρίζες στο $ \mathbb{R} $.
\item Η γραφική παράστασή της έχει οριζόντια ασύμπτωτη τον άξονα $ x'x $ στη μεριά του $ -\infty $ ενώ τέμνει τον κατακόρυφο άξονα $ y'y $ στο σημείο $ A(0,1) $.
\item Για κάθε ζεύγος αριθμών $ x_1,x_2\in\mathbb{R} $ ισχύει 
\begin{gather*}
\textrm{Αν }x_1<x_2\Leftrightarrow a^{x_1}<a^{x_2} \\
\textrm{Αν }x_1=x_2\Leftrightarrow a^{x_1}=a^{x_2}
\end{gather*}
\end{itemize}
\begin{tabular}{p{6cm}p{6cm}}
\begin{tikzpicture}
\begin{axis}[x=.7cm,y=.7cm,aks_on,xmin=-3,xmax=3,
ymin=-.5,ymax=4,ticks=none,xlabel={\footnotesize $ x $},
ylabel={\footnotesize $ y $},belh ar]
\begin{scope}
\clip (axis cs:-3,0) rectangle (axis cs:3,3.7);
\addplot[grafikh parastash,domain=-2.7:2.7]{1.8^x};
\end{scope}
\node at (axis cs:-.3,-0.3) {\footnotesize$O$};
\end{axis}
\tkzDefPoint(-.5,1){B}
\tkzDefPoint(2.1,1.05){A}
\tkzDrawPoint[fill=black](A)
\tkzLabelPoint[above left,yshift=-1mm](A){$ (0,1) $}
\node at (3,0.7) {\footnotesize$a>1$};
\node at (3,2.5) {\footnotesize$C_f$};
\end{tikzpicture}\captionof{figure}{Εκθετική συνάρτηση με $ a>1 $}	& \begin{tikzpicture}
\begin{axis}[x=.7cm,y=.7cm,aks_on,xmin=-3,xmax=3,
ymin=-.5,ymax=4,ticks=none,xlabel={\footnotesize $ x $},
ylabel={\footnotesize $ y $},belh ar]
\begin{scope}
\clip (axis cs:-3,0) rectangle (axis cs:3,3.7);
\addplot[grafikh parastash,domain=-2.7:2.7]{0.55^x};
\end{scope}
\node at (axis cs:-.3,-0.3) {\footnotesize$O$};
\end{axis}
\tkzDefPoint(-.8,1){B}
\tkzDefPoint(2.1,1.05){A}
\tkzDrawPoint[fill=black](A)
\tkzLabelPoint[above right,yshift=-1mm](A){$ (0,1) $}
\node at (1.2,0.7) {\footnotesize$0<a<1$};
\node at (1.2,2.5) {\footnotesize$C_f$};
\end{tikzpicture}\captionof{figure}{Εκθετική συνάρτηση με $ 0<a<1 $} \\ 
\end{tabular} 
\end{enumerate}
\begin{enumerate}[itemsep=0mm,label=\bf\arabic*.,leftmargin=0cm,start=2]
\item[\textbf{B.}] \textbf{Για {\boldmath$ 0<a<1 $}}
\begin{itemize}
\item Αν η βάση $ a $ της εκθετικής συνάρτησης είναι μικρότερη της μονάδας τότε η συνάρτηση $ f(x)=a^x $ είναι γνησίως φθίνουσα στο $ \mathbb{R} $.
\item Η συνάρτηση δεν έχει ρίζες στο $ \mathbb{R} $.
\item Η γραφική παράστασή της έχει οριζόντια ασύμπτωτη τον άξονα $ x'x $ στη μεριά του $ +\infty $ ενώ τέμνει τον κατακόρυφο άξονα $ y'y $ στο σημείο $ A(0,1) $.
\item Για κάθε ζεύγος αριθμών $ x_1,x_2\in\mathbb{R} $ ισχύει 
\begin{gather*}
\textrm{Αν }x_1<x_2\Leftrightarrow a^{x_1}>a^{x_2} \\
\textrm{Αν }x_1=x_2\Leftrightarrow a^{x_1}=a^{x_2}
\end{gather*}
\end{itemize}
\end{enumerate}
\item Οι γραφικές παραστάσεις των εκθετικών συναρτήσεων με αντίστροφες βάσεις $ f(x)=a^x $ και $ g(x)=\left(\frac{1}{a}\right)^x  $, με $ 0<a\neq1 $, είναι συμμετρικές ως προς τον άξονα $ y'y $.
\end{rlist}
\begin{center}
\begin{tikzpicture}
\begin{axis}[x=.7cm,y=.7cm,aks_on,xmin=-3,xmax=3,
ymin=-.5,ymax=4,ticks=none,xlabel={\footnotesize $ x $},
ylabel={\footnotesize $ y $},belh ar]
\begin{scope}
\clip (axis cs:-3,0) rectangle (axis cs:3,3.7);
\addplot[grafikh parastash,domain=-2.7:2.7]{1.8^x};
\addplot[grafikh parastash,domain=-2.7:2.7]{0.55^x};
\end{scope}
\node at (axis cs:-.3,-0.3) {\footnotesize$O$};
\end{axis}
\tkzDrawPoint[fill=black](2.1,1.05)
\node at (3,2.5) {\footnotesize$C_f$};
\node at (1.2,2.5) {\footnotesize$C_g$};
\node at (.8,.9) {\footnotesize$f(x)=a^x$};
\node at (3.7,.9) {\footnotesize$g(x)=\left(\frac{1}{a}\right)^x$};
\end{tikzpicture}\captionof{figure}{Εκθετικές συναρτήσεις με αντίστροφες βάσεις}
\end{center}
\section{Λογάριθμος}
\orismoi
\Orismos{Λογάριθμοσ}
Λογάριθμος με βάση ένα θετικό αριθμό $ a\neq1 $ ενός θετικού αριθμού $ \beta $ ονομάζεται ο εκθέτης στον οποίο θα υψωθεί ο αριθμός $ a $ ώστε να δώσει τον αριθμό $ \beta $. Συμβολίζεται :
\[ \log_{a}{\beta} \]
με $ 0<a\neq1\;\textrm{και}\; \beta>0 $.
\begin{itemize}
\item Ο αριθμός $ a $ ονομάζεται \textbf{βάση του λογαρίθμου}.
\item Ο αριθμός $ \beta $ έχει το ρόλο του αποτελέσματος της δύναμης με βάση $ a $, ενώ ολόκληρος ο λογάριθμος, το ρόλο του εκθέτη.
\item Αν ο λογάριθμος (εκθέτης) με βάση $ a $ του $ \beta $ είναι ίσος με $ x $ τότε θα ισχύει :
\[ \log_{a}{\beta}=x\Leftrightarrow a^x=\beta \]
\item Εαν η βάση ενός λογαρίθμου είναι ο αριθμός $ 10 $ τότε ο λογάριθμος ονομάζεται \textbf{δεκαδικός λογάριθμος} και συμβολίζεται : $ \log{x} $.
\item Εαν η βάση του λογαρίθμου είναι ο αριθμός $ e $ τότε ο λογάριθμος ονομάζεται \textbf{φυσικός λογάριθμος} και συμβολίζεται : $ \ln{x} $.
\end{itemize}
\thewrhmata
\Thewrhma{Ιδιότητεσ Λογαρίθμων}
Για οπουσδήποτε θετικούς πραγματικούς αριθμούς $ x,y\in\mathbb{R}^+ $ έχουμε τις ακόλουθες ιδιότητες που αφορούν το λογάριθμο τους με βάση έναν θετικό πραγματικό αριθμό $ a $.
\begin{center}
\begin{longtable}{cc}
\hline \rule[-2ex]{0pt}{5.5ex} \textbf{Ιδιότητα} & \textbf{Συνθήκη} \\
\hhline{==}\rule[-2ex]{0pt}{5.5ex} Λογάριθμος γινομένου & $ \log_{a}(x\cdot y)=\log_{a}x+\log_{a}y $ \\
\rule[-2ex]{0pt}{5.5ex}  Λογάριθμος πηλίκου & $ \log_{a}\left( \dfrac{x}{y}\right) =\log_{a}x-\log_{a}y $ \\
\rule[-2ex]{0pt}{5.5ex}  Λογάριθμος δύναμης & $ \log_{a}x^\kappa=\kappa\cdot\log_{a}x\;\;,\;\;\kappa\in\mathbb{Z} $ \\
\rule[-2ex]{0pt}{5.5ex}  Λογάριθμος ρίζας & $ \log_{a}\!\sqrt[\nu]{x}=\dfrac{1}{\nu}\log_{a}x\;\;,\;\;\nu\in\mathbb{N} $ \\
\rule[-2ex]{0pt}{5.5ex}  Λογάριθμος ως εκθέτης & $ a^{\log_{a}x}=x $ \\
\rule[-2ex]{0pt}{5.5ex}  Λογάριθμος δύναμης με κοινή βάση & $ \log_{a}a^x=x $ \\
\rule[-2ex]{0pt}{5.5ex}  Αλλαγή βάσης & $ \log_{a}x=\dfrac{\log_{\beta}{x}}{\log_{\beta}{a}} $ \\
\hline
\end{longtable}\captionof{table}{Ιδιότητες λογαρίθμων}
\end{center}
Επίσης για κάθε λογάριθμο με οποιαδήποτε βάση $ a\in\mathbb{R}^+ $ και $ a\neq1 $ έχουμε :
\begin{multicols}{2}
\begin{rlist}
\item $ \log_{a}1=0 $
\item $ \log_{a}a=1 $
\end{rlist}
\end{multicols}
\section{Λογαριθμική συνάρτηση}
\orismoi
\Orismos{Λογαριθμική συνάρτηση}
Λογαριθμική ονομάζεται κάθε συνάρτηση $ f $ της οποίας η τιμή της $ f(x) $ δίνεται με τη βοήθεια ενός λογαρίθμου, για κάθε στοιχείο του πεδίου ορισμού $ x\in D_f $. Θα είναι :
\[ f(x)=\log_ax\;\;,\;\;0<a\neq1 \]
\begin{itemize}
\item Αν η βάση $ a $ του λογαρίθμου γίνει ίση με τον αριθμό $ 10 $ ή $ e $ τότε αποκτάμε τη συνάρτηση $ f(x)=\log{x} $ ή $ f(x)=\ln{x} $ αντίστοιχα.
\end{itemize}
\thewrhmata
\Thewrhma{Ιδιότητεσ λογαριθμικών συναρτήσεων}
Για κάθε λογαριθμική συνάρτηση της μορφής $ f(x)=\log_{a}{x} $ ισχύουν οι ακόλουθες ιδιότητες.
\begin{rlist}
\item Η συνάρτηση $ f $ έχει πεδίο ορισμού το σύνολο $ (0,+\infty) $ των θετικών πραγματικών αριθμών.
\item Το σύνολο τιμών της είναι το σύνολο $ \mathbb{R} $ των πραγματικών αριθμών.
\item Η συνάρτηση δεν έχει μέγιστη και ελάχιστη τιμή.
\begin{enumerate}[itemsep=0mm,label=\bf\arabic*.,leftmargin=0cm]
\item \textbf{Για {\boldmath$ a>1 $}}
\begin{itemize}
\item Αν η βάση $ a $ του λογαρίθμου είναι μεγαλύτερη της μονάδας τότε η συνάρτηση $ f(x)=\log_{a}x $ είναι γνησίως αυξουσα στο $ (0,+\infty) $.
\item Η συνάρτηση έχει ρίζα τον αριθμό $ x=1 $.
\item Η γραφική παράστασή της έχει κατακόρυφη ασύμπτωτη τον άξονα $ y'y $ στη μεριά του $ -\infty $ ενώ τέμνει τον οριζόντιο άξονα $ x'x $ στο σημείο $ A(1,0) $.
\item Για κάθε ζεύγος αριθμών $ x_1,x_2\in\mathbb{R} $ ισχύει \begin{gather*}
\textrm{Αν }x_1<x_2\Leftrightarrow \log_{a}{x_1}<\log_{a}{x_2} \\
\textrm{Αν }x_1=x_2\Leftrightarrow \log_{a}{x_1}=\log_{a}{x_2}
\end{gather*}
\item Για $ x>1 $ ισχύει $ \log_{a}x>0 $ ενώ για $ 0<x<1 $ έχουμε $ \log_{a}x<0 $.
\end{itemize}
\end{enumerate}
\begin{tabular}{p{6cm}p{6.2cm}}
\begin{tikzpicture}
\begin{axis}[x=.7cm,y=.7cm,aks_on,xmin=-.5,xmax=5,
ymin=-3,ymax=3.4,ticks=none,xlabel={\footnotesize $ x $},
ylabel={\footnotesize $ y $},belh ar]
\begin{scope}
\clip (axis cs:-3,-3) rectangle (axis cs:4.7,3);
\addplot[grafikh parastash,domain=-2.7:4.7]{log2(x)};
\end{scope}
\node at (axis cs:-.3,-0.3) {\footnotesize$O$};
\end{axis}
\node at (2,0.7) {\footnotesize$a>1$};
\tkzDefPoint(-.5,1){B}
\tkzDefPoint(1.05,2.1){A}
\tkzDrawPoint[fill=\xrwma](A)
\tkzLabelPoint[below right](A){$ A(0,1) $}
\node at (.8,.4) {\footnotesize$C_f$};
\end{tikzpicture}\captionof{figure}{Λογαριθμική συνάρτηση με $ a>1 $}	& \begin{tikzpicture}
\begin{axis}[x=.7cm,y=.7cm,aks_on,xmin=-.5,xmax=5,
ymin=-3,ymax=3.4,ticks=none,xlabel={\footnotesize $ x $},
ylabel={\footnotesize $ y $},belh ar]
\begin{scope}
\clip (axis cs:-3,-3) rectangle (axis cs:4.7,3);
\addplot[grafikh parastash,domain=-2.7:4.7]{ln(x)/ln(.5)};
\end{scope}
\node at (axis cs:-.3,-0.3) {\footnotesize$O$};
\end{axis}
\node at (2,3.3) {\footnotesize$0<a<1$};
\tkzDefPoint(-.5,1){B}
\tkzDefPoint(1.05,2.1){A}
\tkzDrawPoint[fill=\xrwma](A)
\tkzLabelPoint[above right](A){$ A(0,1) $}
\node at (.8,4) {\footnotesize$C_g$};
\end{tikzpicture}\captionof{figure}{Λογαριθμική συνάρτηση με $ 0<a<1 $} \\ 
\end{tabular} 
\begin{enumerate}[itemsep=0mm,label=\bf\arabic*.,leftmargin=0cm,start=2]
\item \textbf{Για {\boldmath$ 0<a<1 $}}
\begin{itemize}
\item Αν η βάση $ a $ του λογαρίθμου είναι μεγαλύτερη της μονάδας τότε η συνάρτηση $ f(x)=\log_{a}x $ είναι γνησίως φθίνουσα στο $ (0,+\infty) $.
\item Η συνάρτηση έχει ρίζα τον αριθμό $ x=1 $.
\item Η γραφική παράστασή της έχει κατακόρυφη ασύμπτωτη τον άξονα $ y'y $ στη μεριά του $ +\infty $ ενώ τέμνει τον οριζόντιο άξονα $ x'x $ στο σημείο $ A(1,0) $.
\item Για κάθε ζεύγος αριθμών $ x_1,x_2\in\mathbb{R} $ ισχύει 
\begin{gather*}
\textrm{Αν }x_1<x_2\Leftrightarrow \log_{a}{x_1}>\log_{a}{x_2} \\
\textrm{Αν }x_1=x_2\Leftrightarrow \log_{a}{x_1}=\log_{a}{x_2}
\end{gather*}
\item Για $ x>1 $ ισχύει $ \log_{a}x<0 $ ενώ για $ 0<x<1 $ έχουμε $ \log_{a}x>0 $.
\end{itemize}
\end{enumerate}
\item Οι γραφικές παραστάσεις των λογαριθμικών συναρτήσεων με αντίστροφες βάσεις $ f(x)=\log_a{x} $ και $ g(x)=\log_{\frac{1}{a}}{x}  $, με $ 0<a\neq1 $, είναι συμμετρικές ως προς τον άξονα $ x'x $.
\end{rlist}
\begin{center}
\begin{tikzpicture}
\begin{axis}[x=.7cm,y=.7cm,aks_on,xmin=-.5,xmax=5,
ymin=-3,ymax=3.4,ticks=none,xlabel={\footnotesize $ x $},
ylabel={\footnotesize $ y $},belh ar]
\begin{scope}
\clip (axis cs:-3,-3) rectangle (axis cs:4.7,3);
\addplot[grafikh parastash,domain=-2.7:4.7]{log2(x)};
\addplot[grafikh parastash,domain=-2.7:4.7]{ln(x)/ln(.5)};
\end{scope}
\node at (axis cs:-.3,-0.3) {\footnotesize$O$};
\end{axis}
\tkzDrawPoint[fill=black](1.05,2.1)
\node at (.8,.4) {\footnotesize$C_f$};
\node at (.8,4) {\footnotesize$C_g$};
\node at (3.2,2.9) {\footnotesize$f(x)=\log_{a}x$};
\node at (3.2,1.3) {\footnotesize$g(x)=\log_{\frac{1}{a}}x$};
\end{tikzpicture}\captionof{figure}{Λογαριθμικές συναρτήσεις με αντίστροφες βάσεις}
\end{center}
\end{document}