\chapter[Ιδιοτητες Συναρτησεων]{Ιδιότητες Συναρτήσεων}
\section{Μονοτονία - Ακρότατα}
\orismoi
\Orismos{Μονοτονία}
Μια συνάρτηση αύξουσα ή φθίνουσα, χαρακτηρίζεται ως \textbf{μονότονη}, ενώ μια γνησίως αύξουσα ή γνησίως φθίνουσα συνάρτηση ως \textbf{γνησίως μονότονη}. Οι χαρακτηρισμοί αυτοί αφορούν τη \textbf{μονοτονία} μιας συνάρτησης, μια ιδιότητα των συναρτήσεων η οποία δείχνει την αύξηση ή τη μείωση των τιμών μιας συνάρτησης σε ένα διάστημα του πεδίου ορισμού.
\begin{enumerate}[itemsep=0mm,label=\bf\arabic*.]
\item \textbf{Γνησίως αύξουσα}\\ Μια συνάρτηση $ f $ ορισμένη σε ένα διάστημα $ \varDelta $ ονομάζεται γνησίως αύξουσα στο $ \varDelta $ εαν για κάθε ζεύγος αριθμών $ x_1,x_2\in\varDelta $ με $ x_1<x_2 $ ισχύει \[ f(x_1)<f(x_2) \]
\item \textbf{Γνησίως φθίνουσα}\\ Μια συνάρτηση $ f $ ορισμένη σε ένα διάστημα $ \varDelta $ ονομάζεται γνησίως φθίνουσα στο $ \varDelta $ εαν για κάθε ζεύγος αριθμών $ x_1,x_2\in\varDelta $ με $ x_1<x_2 $ ισχύει \[ f(x_1)>f(x_2) \]
\begin{center}
\begin{tabular}{p{5cm}p{5cm}}
\begin{tikzpicture}
\draw[dashed] (3.3,1.4) node[anchor=north]{\scriptsize $x_2$} -- 
(3.3,2.58)--(1,2.58) node[left]{\scriptsize $f(x_2)$};
\draw[dashed] (2,1.4) node[anchor=north]{\scriptsize $x_1$}-- 
(2,2.08)--(1,2.08)node[left]{\scriptsize $f(x_1)$};
\begin{axis}[x=1cm,y=1cm,aks_on,xmin=-1,xmax=3,
ymin=-1.4,ymax=2,ticks=none,xlabel={\footnotesize $ x $},
ylabel={\footnotesize $ y $},belh ar]
\addplot[grafikh parastash,\xrwma,domain=-.8:3]{ln(x+1)};
\end{axis}
\tkzDrawPoint[size=7,fill=black](2,2.09)
\tkzDrawPoint[size=7,fill=black](3.3,2.59)
\node[fill=white,inner sep=.1mm] at (2.7,0.6) {\scriptsize $ x_1<x_2\Rightarrow f(x_1)<f(x_2)$};
\end{tikzpicture}\captionof{figure}{Γνησίως αύξουσα}	& \begin{tikzpicture}
\draw[dashed] (2.6,1.4) node[anchor=north]{\scriptsize $x_2$} -- 
(2.6,2.02)--(1,2.02) node[left]{\scriptsize $f(x_2)$};
\draw[dashed] (1.5,1.4) node[anchor=north]{\scriptsize $x_1$}-- 
(1.5,2.7)--(1,2.7)node[left]{\scriptsize $f(x_1)$};
\begin{axis}[x=1cm,y=1cm,aks_on,xmin=-1,xmax=3,
ymin=-1.4,ymax=2,ticks=none,xlabel={\footnotesize $ x $},
ylabel={\footnotesize $ y $},belh ar,clip=false]
\addplot[grafikh parastash,\xrwma,domain=-.6:3]{-0.2*(x+.5)^2+1.5};
\end{axis}
\tkzDrawPoint[size=7,fill=black](2.6,2.02)
\tkzDrawPoint[size=7,fill=black](1.5,2.7)
\node[fill=white,inner sep=.1mm] at (1.95,0.6) {\scriptsize $ x_1<x_2\Rightarrow f(x_1)>f(x_2)$};
\end{tikzpicture}\captionof{figure}{Γνησίως φθίνουσα} \\ 
\end{tabular} 
\end{center}
\end{enumerate}
\Orismos{Ολικά Ακρότατα}
Ακρότατα ονομάζονται οι μέγιστες ή ελάχιστες τιμές μιας συνάρτησης $ f:D_f\rightarrow\mathbb{R} $ τις οποίες παίρνει σε ένα διάστημα ή σε ολόκληρο το πεδίο ορισμού της.
\begin{enumerate}[itemsep=0mm,label=\bf\arabic*.]
\item \textbf{Ολικό μέγιστο}\\
Μια συνάρτηση $ f:D_f\rightarrow\mathbb{R} $ παρουσιάζει ολικό μέγιστο σε ένα σημείο $ x_0\in D_f $ του πεδίου ορισμού της όταν η τιμή $ f(x_0) $ είναι μεγαλύτερη από κάθε άλλη $ f(x) $ για κάθε σημείο $ x_0 $ του πεδίου ορισμού. \[ f(x)\leq f(x_0)\;\;,\;\;\textrm{για κάθε } x\in D_f \]
\item \textbf{Ολικό ελάχιστο}\\
Μια συνάρτηση $ f:D_f\rightarrow\mathbb{R} $ παρουσιάζει ολικό ελάχιστο σε ένα σημείο $ x_0\in D_f $ του πεδίου ορισμού της όταν η τιμή $ f(x_0) $ είναι μικρότερη από κάθε άλλη $ f(x) $ για κάθε σημείο $ x_0 $ του πεδίου ορισμού. \[ f(x)\geq f(x_0)\;\;,\;\;\textrm{για κάθε } x\in D_f \]
\begin{center}
\begin{tabular}{p{5cm}p{5cm}}
\begin{tikzpicture}
\begin{axis}[x=1cm,y=1cm,aks_on,xmin=-.7,xmax=3.2,
ymin=-1,ymax=2,ticks=none,xlabel={\footnotesize $ x $},
ylabel={\footnotesize $ y $},belh ar,clip=false]
\addplot[grafikh parastash,domain=-.3:2.3]{-x^2+2*x};
\end{axis}
\tkzDrawPoint[size=7,fill=black](1.7,2)
\node at (1.95,0.4) {\scriptsize $ f(x)\leq f(x_0)$};
\draw[dashed] (1.7,1) node[anchor=north]{\scriptsize $x_0$} -- 
(1.7,2)--(0.7,2) node[left]{\scriptsize $f(x_0)$};
\node at (0.5,0.8) {\footnotesize $ O $};
\end{tikzpicture}\captionof{figure}{Ολικό μέγιστο}	& \begin{tikzpicture}
\begin{axis}[x=1cm,y=1cm,aks_on,xmin=-.7,xmax=3,
ymin=-.7,ymax=2.3,ticks=none,xlabel={\footnotesize $ x $},
ylabel={\footnotesize $ y $},belh ar,clip=false]
\addplot[grafikh parastash,domain=-.3:2.3]{x^2-2*x+1.5};
\end{axis}
\tkzDrawPoint[size=7,fill=black](1.7,1.2)
\node at (2.1,0.2) {\scriptsize $ f(x)\leq f(x_0)$};
\draw[dashed] (1.7,0.7) node[anchor=north]{\scriptsize $x_0$} -- 
(1.7,1.2)--(0.7,1.2) node[left]{\scriptsize $f(x_0)$};
\node[fill=white,inner sep=.5pt] at (0.5,0.5) {\footnotesize $ O $};
\end{tikzpicture}\captionof{figure}{Ολικό ελάχιστο} \\ 
\end{tabular} 
\end{center}
\end{enumerate}
\Orismos{Άρτια - Περιττή συνάρτηση}
\vspace{-5mm}
\begin{enumerate}[itemsep=0mm,label=\bf\arabic*.]
\item \textbf{Άρτια συνάρτηση}\\ Άρτια ονομάζεται μια συνάρτηση $ f:D_f\rightarrow\mathbb{R} $ για την οποία ισχύουν οι παρακάτω συνθήκες :
\begin{enumerate}[itemsep=0mm,label=\roman*.]
\item $ \forall x\in D_f\Rightarrow -x\in D_f $
\item $ f(-x)=f(x)\;,\;\forall x\in D_f$
\end{enumerate}
\item \textbf{Περιττή συνάρτηση}\\ Περιττή ονομάζεται μια συνάρτηση $ f:D_f\rightarrow\mathbb{R} $ για την οποία ισχύουν οι παρακάτω συνθήκες :
\begin{enumerate}[itemsep=0mm,label=\roman*.]
\item $ \forall x\in D_f\Rightarrow -x\in D_f $
\item $ f(-x)=-f(x)\;,\;\forall x\in D_f$
\end{enumerate}
\end{enumerate}
\begin{center}
\begin{tabular}{p{4.5cm}p{4.5cm}}
\begin{tikzpicture}
\begin{axis}[x=2cm,y=3cm,aks_on,xmin=-1,xmax=1,ymin=-.1,ymax=0.9,ticks=none,xlabel={\footnotesize $ x $},ylabel={\footnotesize $ y $},belh ar]
\addplot[grafikh parastash,domain=-.85:.85]{(x^2)};
\draw[dashed](axis cs:.7,0)node[below]{{\footnotesize $ x $}}--(axis cs:.7,.49)--(axis cs:-.7,.49)--(axis cs:-.7,0)node[below]{{\footnotesize $ -x $}};
\end{axis}
\node[fill=white,inner sep=.1mm] at (2,2.5){\scriptsize $f(-x)=f(x)$};
\end{tikzpicture}\captionof{figure}{Άρτια συνάρτηση}	& \begin{tikzpicture}
\node at (3.4,0) {\scriptsize $f(-x)=-f(x)$};
\begin{axis}[x=2cm,y=1.8cm,aks_on,xmin=-1,xmax=1,ymin=-.9,ymax=.9,ticks=none,xlabel={\footnotesize $ x $},ylabel={\footnotesize $ y $},belh ar]
\addplot[grafikh parastash,domain=-.9:.9]{(x^3)};
\draw[dashed](axis cs:.7,0)node[below]{{\footnotesize $ x $}}--(axis cs:.7,.343)--(axis cs:0,.343)node[left]{{\footnotesize $ f(x) $}};
\draw[dashed](axis cs:-.7,0)node[above]{{\footnotesize $ -x $}}--(axis cs:-.7,-.343)--(axis cs:0,-.343)node[right]{{\footnotesize $ f(-x) $}};
\end{axis}
\end{tikzpicture}\captionof{figure}{Περιττή συνάρτηση} \\ 
\end{tabular} 
\end{center}
\begin{itemize}[itemsep=0mm]
\item Η γραφική παράσταση μιας άρτιας συνάρτησης είναι συμμετρική ως προς τον κατακόρυφο άξονα.
\item H γραφική παράσταση μιας περιττής συνάρτησης είναι συμμετρική ως προς την αρχή των αξόνων.
\item Η αρχή των αξόνων για μια περιττή συνάρτηση ονομάζεται \textbf{κέντρο συμμετρίας} της.
\end{itemize}
\section{Μετατόπιση γραφικής παράστασης}
\thewrhmata
\Thewrhma{Κατακόρυφη μετατόπιση}
Η γραφική παράσταση $ C_f $ μιας συνάρτησης $ f $ μετατοπίζεται κατακόρυφα κατά $ c $ μονάδες προς τα πάνω ή προς τα κάτω, εάν αυξήσουμε ή μειώσουμε αντίστοιχα τις τεταγμένες $ f(x) $ των σημείων της κατά $ c $ μονάδες.
\[ g(x)=f(x)\pm c\;\;,\;\;c>0 \]
Η γραφική παράσταση $ C_g $ της νέας συνάρτησης $ g(x) $ προκύπτει από κατακόρυφη μετατόπιση της $ C_f $ κατά $ c $ μονάδες.
\begin{center}
\begin{tabular}{p{5cm}cp{5cm}}
\begin{tikzpicture}
\begin{axis}[aks_on,belh ar,xlabel={\footnotesize$x$},ylabel={\footnotesize$y$}
,xmin=-2,xmax=2.,ymin=-1,ymax=3,x=1cm,y=1cm]
\addplot[clip=false,domain=-1.8:1.8,grafikh parastash]{x^2-.7};
\addplot[domain=-1.8:1.8,pl,samples=200]{x^2};
\addplot[domain=-1.5:1.5,grafikh parastash]{x^2+.7};
\draw[pl,-latex] (axis cs:.5,.25) -- (axis cs:.5,.95);
\draw[pl,-latex] (axis cs:.5,.25) -- (axis cs:.5,-.45);
\draw[pl,-latex] (axis cs:-.5,.25) -- (axis cs:-.5,-.45);
\draw[pl,-latex] (axis cs:-.5,.25) -- (axis cs:-.5,.95);
\node at (axis cs:-.25,.5) {\footnotesize$+c$};
\node at (axis cs:-.25,-.25) {\footnotesize$-c$};
\node at (axis cs:.25,.5) {\footnotesize$+c$};
\node at (axis cs:.25,-.25) {\footnotesize$-c$};
\end{axis}
\end{tikzpicture} & & \begin{tikzpicture}
\begin{axis}[clip=false,aks_on,belh ar,xlabel={\footnotesize$x$},ylabel={\footnotesize$y$}
,xmin=-2,xmax=4.2,ymin=-1,ymax=3,x=1cm,y=1cm]
\addplot[domain=-1.8:1.8,grafikh parastash]{x^2-.7};
\addplot[domain=-.8:2.8,pl,samples=200]{(x-1)^2-.7};
\addplot[domain=.2:3.8,grafikh parastash]{(x-2)^2-.7};
\draw[pl,-latex] (axis cs:2,.3) -- (axis cs:3,.3);
\draw[pl,-latex] (axis cs:2.5,1.55) -- (axis cs:3.5,1.55);
\draw[pl,-latex] (axis cs:0,.3) -- (axis cs:-1,.3);
\draw[pl,-latex] (axis cs:-.5,1.55) -- (axis cs:-1.5,1.55);
\node at (axis cs:-.5,.5) {\footnotesize$+c$};
\node at (axis cs:3,1.7) {\footnotesize$-c$};
\node at (axis cs:2.5,.5) {\footnotesize$-c$};
\node at (axis cs:-1,1.7) {\footnotesize$+c$};
\end{axis}
\end{tikzpicture} \\ 
\end{tabular} 
\end{center}
\Thewrhma{Οριζόντια μετατόπιση}
Η γραφική παράσταση $ C_f $ μιας συνάρτησης $ f $ μετατοπίζεται οριζόντια κατά $ c $ μονάδες προς τα αριστερά ή προς τα δεξιά, εάν αυξήσουμε ή μειώσουμε αντίστοιχα τις τετμημένες $ x $ των σημείων της κατά $ c $ μονάδες.
\[ g(x)=f(x\pm c)\;\;,\;\;c>0  \]
Η γραφική παράσταση $ C_g $ της νέας συνάρτησης $ g(x) $ προκύπτει από οριζόντια μετατόπιση της $ C_f $ κατά $ c $ μονάδες.