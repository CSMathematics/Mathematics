%# Database File : DTX-Geometria-KwnT-ExisKyklou-GenExKykl-SolSE1-1
%@ Database source: Mathematics
\begin{alist}
\item Για την εξίσωση $ x^2+y^2+2x+4y-4=0 $ έχουμε $ A=2,B=4 $ και $ \varGamma=-4 $. Είναι λοιπόν
\[ A^2+B^2-4\varGamma=2^2+4^2-4\cdot(-4)=4+16+16=36>0 \]
Η εξίσωση παριστάνει κύκλο με ακτίνα
\[ \rho=\frac{\sqrt{A^2+B^2-4\varGamma}}{2}=\frac{\sqrt{36}}{2}=\frac{6}{2}=3 \]
και κέντρο
\[ K\left(-\frac{A}{2},-\frac{B}{2}\right)\equiv K\left(-\frac{2}{2},-\frac{4}{2}\right)\equiv K(-1,-2) \]
\item Έχουμε ότι $ A=6,B=-10 $ και $ \varGamma=34 $.
\[ A^2+B^2+4\varGamma=6^2+(-10)^2-4\cdot 34=36+100-136=0 \]
Η εξίσωση λοιπόν παριστάνει σημείο με συντεταγμένες
\[ K\left(-\frac{A}{2},-\frac{B}{2}\right)\equiv K\left(-\frac{6}{2},-\frac{-10}{2}\right)\equiv K(-3,5) \]
\item Είναι $ A=-2,B=4 $ και $ \varGamma=9 $.
\[ A^2+B^2-4\varGamma=(-2)^2+4^2-4\cdot 9=4+16-36=-16<0 \]
Άρα η εξίσωση δεν παριστάνει κανένα σημείο.
\item Από την εξίσωση έχουμε ότι $ A=1,B=2 $ και $ \varGamma=\frac{1}{4} $ οπότε
\[ A^2+B^2-4\cdot\varGamma=1^2+2^2-4\cdot\frac{1}{4} \]
\end{alist}
%# End of file DTX-Geometria-KwnT-ExisKyklou-GenExKykl-SolSE1-1