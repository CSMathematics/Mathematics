%# Database File : DTX-Geometria-KwnT-Elleipsi-StoixEll-SolSE2-1
%@ Database source: Mathematics
\begin{alist}
\item Οι εστίες βρίσκονται πάνω στον άξονα $ x'x $ άρα η εξίσωση της έλλειψης έχει τη μορφή
\[ \frac{x^2}{a^2}+\frac{y^2}{\beta^2}=1 \]
Από τις συντεταγμένες των εστιών προκύπτει $\gamma=3$. Επιπλέον έχουμε
\[ \varepsilon=0{,}6\Rightarrow\frac{\gamma}{a}=\frac{6}{10}\Rightarrow \frac{3}{a}=\frac{3}{5}\Rightarrow a=5 \]
Άρα παίρνουμε
\[ \beta^2=a^2-\gamma^2=5^2-3^2=25-9=16 \]
Η εξίσωση της έλλειψης θα είναι
\[ \frac{x^2}{25}+\frac{y^2}{16}=1 \]
\item Παρατηρούμε ότι ο μεγάλος άξονας της έλειψης είναι ο οριζόντιος επομένως $ 2a=8\Rightarrow a=4 $ και $ 2\beta=6\Rightarrow\beta=3 $. Άρα η εξίσωση της έλλειψης θα είναι 
\[ \frac{x^2}{a^2}+\frac{y^2}{\beta^2}=1\Rightarrow\frac{x^2}{16}+\frac{y^2}{9}=1 \]
\item Από την εκκεντρότητα της έλλειψης παίρνουμε
\[ \varepsilon=0{,}8\Rightarrow \frac{\gamma}{a}=0{,}8\Rightarrow \gamma=0{,}8a \]
Επιπλέον ο μεγάλος άξονας είναι ο οριζόντιος. Θα είναι
\[ A'A=10\Rightarrow 2a=10\Rightarrow a=5 \]
Άρα $ \gamma=0{,}8a=0{,}8\cdot 5=4 $. Έχουμε λοιπόν
\begin{align}
\beta^2=a^2-\gamma^2&\Rightarrow \beta^2=5^2-4^4\\&\Rightarrow \beta^2=9
\end{align}
Επομένως η εξίσωση της έλλειψης θα είναι
\[ \frac{x^2}{a^2}+\frac{y^2}{\beta^2}=1\Rightarrow \frac{x^2}{25}+\frac{y^2}{9}=1 \]
\item Έχουμε $ \varepsilon=\frac{\sqrt{3}}{2}\Rightarrow \frac{\gamma}{a}=\frac{\sqrt{3}}{2}\Rightarrow \gamma =\frac{\sqrt{3}}{2} a $. Επίσης Από το μήκος του μικρού άξονα προκύπτει
\[ B'B=4\Rightarrow 2\beta=4\Rightarrow \beta=2 \]
Άρα
\[ \beta^2=a^2-\gamma^2\Rightarrow 2^2=a^2-\left(\frac{\sqrt{3}}{2}a\right)^2\Rightarrow \frac{1}{4}a^2=4\Rightarrow a^2=16 \]
Η εξίσωση της έλλειψης θα είναι
\[ \frac{x^2}{a^2}+\frac{y^2}{\beta^2}=1\Rightarrow\frac{x^2}{16}+\frac{y^2}{4}=1 \]
\item Απο τις συντεταγμένες των κορυφών συμπαιρένουμε ότι ο μεγάλος άξονας είναι ο οριζόντιος οπότε η εξίσωση της έλλειψης έχει τη μορφή
\[ \frac{x^2}{a^2}+\frac{y^2}{\beta^2}=1 \]
Επιπλέον από τα σημεία αυτά έχουμε άμεσα ότι $a=7$ και $\beta=4$ άρα
\[ \frac{x^2}{7^2}+\frac{y^2}{4^2}=1\Rightarrow \frac{x^2}{49}+\frac{y^2}{16}=1 \]
\end{alist}
%# End of file DTX-Geometria-KwnT-Elleipsi-StoixEll-SolSE2-1