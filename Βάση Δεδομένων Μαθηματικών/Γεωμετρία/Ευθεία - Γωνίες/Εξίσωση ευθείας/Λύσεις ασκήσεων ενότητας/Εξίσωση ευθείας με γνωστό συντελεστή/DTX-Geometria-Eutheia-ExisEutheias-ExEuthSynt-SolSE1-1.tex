%# Database File : DTX-Geometria-Eutheia-ExisEutheias-ExEuthSynt-SolSE1-1
%@ Database source: Mathematics
\begin{alist}
\item Η εξίσωση της ευθείας είναι
\begin{align*}
y-y_A&=\lambda(x-x_A)\Rightarrow\\
y-(-4)&=2(x-3)\Rightarrow\\
y+4&=2x-6\Rightarrow\\
y&=2x-10
\end{align*}
\item Αφού ισχύει $ \varepsilon\parallel\varepsilon_1\Rightarrow \lambda_{\varepsilon}=\lambda_{\varepsilon_1}\Rightarrow \lambda_{\varepsilon}=3 $. Οπότε η εξίσωση της ευθείας είναι:
\begin{align*}
y-y_A&=\lambda(x-x_A)\Rightarrow\\
y-(-4)&=3(x-3)\Rightarrow\\
y+4&=3x-9\Rightarrow\\
y&=3x-13
\end{align*}
\item Έχουμε ότι $ \varepsilon\perp\varepsilon_1\Rightarrow \lambda_{\varepsilon}\cdot\lambda_{\varepsilon_1}=-1\Rightarrow\lambda_{\varepsilon}\cdot\frac{1}{4}=-1\Rightarrow \lambda_{\varepsilon}=-4 $. Άρα
\begin{align*}
y-y_A&=\lambda(x-x_A)\Rightarrow\\
y-(-4)&=-4(x-3)\Rightarrow\\
y+4&=-4x+12\Rightarrow\\
y&=-4x+8
\end{align*}
\item Αφού η ευθεία είναι οριζόντια τότε $ \lambda=0 $ άρα έχει εξίσωση
\[ y-y_A=\lambda(x-x_A)\Rightarrow y-(-4)=0(x-3)\Rightarrow y=-4 \]
\end{alist}
%# End of file DTX-Geometria-Eutheia-ExisEutheias-ExEuthSynt-SolSE1-1