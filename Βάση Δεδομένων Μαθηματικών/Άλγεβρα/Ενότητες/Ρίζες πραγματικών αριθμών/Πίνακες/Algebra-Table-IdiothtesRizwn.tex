%# Στοιχεία αρχείου Algebra-Table-IdiothtesRizwn--------
\begin{longtable}{cc}
\hline \rule[-2ex]{0pt}{5.5ex} \textbf{Ιδιότητα} & \textbf{Συνθήκη} \\
\hhline{==}\rule[-2ex]{0pt}{5.5ex} Τετράγωνο ρίζας & $ \left(\!\sqrt{x}\;\right)^2=x\;\;,\;\; x\geq0  $ \\
\rule[-2ex]{0pt}{5.5ex} Ν-οστή δύναμη ν-οστής ρίζας & $ \left(\!\sqrt[\nu]{x}\;\right)^\nu=x\;\;,\;\; x\geq0  $ \\
\rule[-2ex]{0pt}{5.5ex} Ρίζα τετραγώνου & $ \sqrt{x^2}=|x|\;\;,\;\; x\in\mathbb{R} $\\
\rule[-2ex]{0pt}{5.5ex} Ν-οστή ρίζα ν-οστής δύναμης & $ \sqrt[\nu]{x^\nu}=\begin{cases}
|x|&  x\in\mathbb{R}\textrm{ αν }\nu\textrm{ άρτιος}\\x&  x\geq0\textrm{ και } \nu\in\mathbb{N}\end{cases} $\\
\hhline{~-} \multirow{3}{*}{Ρίζα γινομένου} & $ \sqrt{x\cdot y}=\!\sqrt{x}\cdot\!\sqrt{y}\;\;,\;\; x,y\geq0 $ \rule[-2ex]{0pt}{5.5ex}\\
\rule[-2ex]{0pt}{5.5ex} & $ \sqrt[\nu]{x\cdot y}=\!\sqrt[\nu]{x}\cdot\!\sqrt[\nu]{y}\;\;,\;\; x,y\geq0 $ \\
\hhline{~-} \multirow{3}{*}{Ρίζα πηλίκου} & $ \sqrt{\dfrac{x}{y}}\;=\dfrac{\sqrt{x}}{\sqrt{y}}\;\;,\;\; x\geq0\textrm{ και }y>0 $ \rule[-2ex]{0pt}{6.5ex}\\
\rule[-2ex]{0pt}{7.5ex} & $ \sqrt[\nu]{\dfrac{x}{y}}\;=\dfrac{\sqrt[\nu]{x}}{\sqrt[\nu]{y}}\;\;,\;\; x\geq0\textrm{ και }y>0 $ \\
\hhline{~-}\rule[-2ex]{0pt}{5.5ex} Μ-οστή ρίζα ν-οστής ρίζας  & $ \sqrt[\mu]{\!\sqrt[\nu]{x}}=\!\sqrt[\nu\cdot\mu]{x}\;\;,\;\; x\geq0 $ \\
\rule[-2ex]{0pt}{5.5ex} Απλοποίηση ρίζας & $ \sqrt[\nu]{x^\nu\cdot y}=x\!\sqrt[\nu]{y}\;\;,\;\; x,y\geq0  $ \\
\rule[-2ex]{0pt}{5.5ex} Απλοποίηση τάξης και δύναμης & $ \sqrt[\mu\cdot\rho]{x^{\nu\cdot\rho}}=\!\sqrt[\mu]{x^{\nu}}\;\;,\;\; x\geq0 $ \\
\hline
\end{longtable}
%#--------------------------------------------------