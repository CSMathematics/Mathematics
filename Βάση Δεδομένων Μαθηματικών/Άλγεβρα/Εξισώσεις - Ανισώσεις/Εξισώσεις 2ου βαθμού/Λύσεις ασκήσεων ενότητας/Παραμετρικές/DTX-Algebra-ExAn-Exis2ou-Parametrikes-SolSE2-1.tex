%# Database File : DTX-Algebra-ExAn-Exis2ou-Parametrikes-SolSE2-1
%@ Database source: Mathematics
Η εξίσωση είναι της μορφής $ ax^2+\beta x+\gamma=0 $ με $a=\lambda-1,\beta=\lambda-2$ και $ \gamma=-1 $. Δια κρίνουμε τις εξής περιπτώσεις:
\begin{alist}
\item Αν $\lambda-1\neq 0\Rightarrow \lambda\neq 1$ τότε η εξίσωση είναι 2ου βαθμού. Θα έχουμε λοιπόν
\[\Delta=\beta^2-4a\gamma=(\lambda-2)^2-4\cdot(\lambda-1)\cdot(-1)=\lambda^2-4\lambda+3+4\lambda-4=\lambda^2\geq 0\]
\begin{itemize}
\item Αν $ \Delta>0\Rightarrow \lambda^2>0\Rightarrow \lambda\neq 0 $ τότε η εξίσωση έχει 2 άνισες λύσεις
\[ x_{1,2}=\frac{-\beta\pm\sqrt{\Delta}}{2a}=\frac{-(\lambda-2)\pm\sqrt{\lambda^2}}{2(\lambda-1)}=\frac{-\lambda+2\pm\lambda}{2(\lambda-1)} \]
άρα θα είναι
\[ x_1=\frac{-\lambda+2+\lambda}{2(\lambda-1)}=\frac{2}{2(\lambda-1)}=\frac{1}{\lambda-1}\ \ ,\ \ x_2=\frac{-\lambda+2-\lambda}{2(\lambda-1)}=\frac{-2\lambda+2}{2(\lambda-1)}=\frac{-2(\lambda-1)}{2(\lambda-1)}=-1 \]
\item Αν $ \Delta=0\Rightarrow \lambda^2=0\Rightarrow \lambda=0 $ τότε η εξίσωση έχει μια διπλή λύση
\[ x=-\frac{\beta}{2a}=-\frac{\lambda-2}{2(\lambda-1)}=-\frac{0-2}{2(0-1)}=-1 \]
\end{itemize}
\item 
\end{alist}
%# End of file DTX-Algebra-ExAn-Exis2ou-Parametrikes-SolSE2-1