%# Database File : Ana-ThBolzano-RizaExis-SolSE2----
Θα σχηματίσουμε από τη ζητούμενη ισότητα την αντίστοιχη εξίσωση θέτοντας όπου $ x_0 $ τη μεταβλητή $ x $. Προκύπτει λοιπόν η εξίσωση
\[ e^{x}=\hm{(\pi x)}-2x\Rightarrow e^{x}-\hm{(\pi x)}+2x=0 \]
Θεωρούμε τη συνάρτηση $ f:\mathbb{R}\to\mathbb{R} $ με τύπο $ f(x)=e^{x}-\hm{(\pi x)}+2x $. Θα ισχύει ότι
\begin{rlist}
\item η $ f $ είναι συνεχής στο διάστημα $ [-1,0] $ και
\item \begin{itemize}
\item $ f(-1)=e^{-1}-\hm{(-\pi)}+2(-1)=\frac{1}{e}-2<0 $
\item $ f(0)=e^0-\hm{0}+2\cdot 0=1>0 $
\end{itemize}
οπότε προκύπτει $ f(-1)\cdot f(0)=\frac{1}{e}-2<0 $
\end{rlist}
Σύμφωνα λοιπόν με το θεώρημα Bolzano η $ f $ θα έχει μια τουλάχιστον ρίζα $ x_0\in(-1,0) $, ή ισοδύναμα η εξίσωση θα έχει μια τουλάχιστον λύση $ x_0 $ στο $ (-1,0) $ άρα τελικά υπάρχει $ x_0\in(-1,0) $ τέτοιο ώστε
\[ e^{x_0}=\hm{(\pi x_0)}-2x_0 \]
%# End of file Ana-ThBolzano-RizaExis-SolSE2