%# Database File : Ana-ThBolzano-RizaExis-SolSE3----
\wrapr{-5mm}{7}{5cm}{-5mm}{\parat
\begin{tcolorbox}[title=\Parathrhsh,hbox,lifted shadow={1mm}{-2mm}{3mm}{0.3mm}%
{black!50!white}]
\begin{varwidth}{4cm}
{\small Οι δύο εξισώσεις είναι ισοδύναμες στο $ (0,1) $ γιατί στο διάστημα αυτό δεν ανήκει το $ x=1 $ του περιορισμού.}
\end{varwidth}
\end{tcolorbox}}{
Για την αρχική εξίσωση απαιτούμε να ισχύει $ x-1\neq0\Rightarrow x\neq1 $. Όμως για κάθε $ x\in(0,1) $ η αρχική μετατρέπεται στην ισοδύναμη εξίσωση:
\begin{equation}\label{par:ex}
e^x=(x-1)\left( x^2-3\right)
\end{equation}
Στη συνέχεια, η τελευταία θα γραφτεί:
\[ e^x-(x-1)\left( x^2-3\right)=0 \]
Ορίζουμε έτσι τη συνάρτηση $ f(x)=e^x-(x-1)\left( x^2-3\right) $ με πεδίο ορισμού το $ \mathbb{R} $. Το θεώρημα Bolzano εφαρμόζεται στο διάστημα $ [0,1] $ και έτσι έχουμε ότι}
\begin{rlist}
\item Η $ f $ είναι συνεχής στο διάστημα $ [0,1] $ και επιπλέον
\item \begin{itemize}
\item $ f(0)=e^0-(0-1)\left( 0^2-3\right)=-2<0 $
\item $ f(1)=e^1-(1-1)\left( 1^2-3\right)=e>0 $
\end{itemize}
οπότε παίρνουμε $ f(0)\cdot f(1)=-2e<0 $.
\end{rlist}
Έτσι σύμφωνα με το θεώρημα Bolzano η εξίσωση \eqref{par:ex} και κατά συνέπεια η αρχική εξίσωση θα έχει μια τουλάχιστον λύση $ x_0 $ στο ανοικτό διάστημα $ (0,1) $.
%# End of file Ana-ThBolzano-RizaExis-SolSE3