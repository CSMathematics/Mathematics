%# Database File : Ana-ThBolzano-RizaExis-SolSE1----
Μεταφέροντας όλους τους όρους της εξίσωσης στο πρώτο μέλος, αυτή θα πάρει τη μορφή:
\[ x^2-\syn{(x\pi)}-e^x=0 \]
Ορίζουμε τη συνάρτηση $ f(x)=x^2-\syn{(x\pi)}-e^x $ με πεδίο ορισμού το $ \mathbb{R} $. Γι αυτήν θα έχουμε ότι
\begin{rlist}
\item είναι συνεχής στο κλειστό διάστημα $ [-2,0] $ και
\item \begin{itemize}
\item $ f(-2)=(-2)^2-\syn{(-2\pi)}-e^{-2}=4-1-e^{-2}=3-\frac{1}{e^2}>0 $
\item $ f(0)=0^2-\syn{0}-e^0=-1-1=-2<0 $
\end{itemize}
οπότε παίρνουμε $ f(-2)\cdot f(0)=-2\left(3-\frac{1}{e^2} \right)<0 $.
\end{rlist}
Έτσι σύμφωνα με το θεώρημα του Bolzano η συνάρτηση $ f $ θα έχει μια τουλάχιστον ρίζα $ x_0\in(-2,0) $, ή ισοδύναμα η αρχική εξίσωση θα έχει μια τουλάχιστον λύση $ x_0 $ στο ανοικτό διάστημα $ (-2,0) $.
%# End of file Ana-ThBolzano-RizaExis-SolSE1