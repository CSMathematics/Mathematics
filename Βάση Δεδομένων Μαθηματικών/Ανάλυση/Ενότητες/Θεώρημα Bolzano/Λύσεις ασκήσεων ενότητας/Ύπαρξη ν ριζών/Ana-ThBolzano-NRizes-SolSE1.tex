%# Database File : Ana-ThBolzano-NRizes-SolSE1----
\wrapr{-5mm}{7}{5cm}{-8mm}{\tcbset{
enhanced,colback=red!5!white,boxrule=0.1pt,
colframe=\xrwma,fonttitle=\bfseries}
\begin{tcolorbox}[title=\Parathrhsh,hbox,lifted shadow={1mm}{-2mm}{3mm}{0.3mm}%
{black!50!white}]
\begin{varwidth}{4cm}
{\small Οι τιμές $ f(0) $ και $ f(2) $ στα άκρα του αρχικού διαστήματος είναι ομόσημες. Έτσι η επιλογή του ενδιάμεσου σημείου είναι τέτοια ώστε η τιμή του να είναι ετερόσημη με τις προηγούμενες.}
\end{varwidth}
\end{tcolorbox}}{Ως ενδιάμεσο σημείο επιλέγουμε το $ x=1 $ έτσι ώστε να χωρίσουμε το αρχικό διάστημα σε δύο υποδιαστήματα $ [0,1],[1,2] $. Για τη συνάρτηση $ f $ έχουμε ότι:
\begin{rlist}
\item είναι συνεχής στα διαστήματα $ [0,1] $ και $ [1,2] $ ενώ
\item \begin{itemize}
\item $ f(0)=e^0-\hm{0}-3\cdot0=1>0 $
\item $ f(1)=e^1-\hm{\pi}-3\cdot1=e-3<0 $
\item $ f(2)=e^2-\hm{2\pi}-3\cdot2=e^2-6>0 $
\end{itemize}
οπότε προκύπτει ότι $ f(0)\cdot f(1)=e-3<0 $ και $ f(1)\cdot f(2)=(e-3)\left( e^2-6\right)<0 $
\end{rlist}
Σύμφωνα λοιπόν με το θεώρημα του Bolzano υπάρχει τουλάχιστον ένα $ x_1\in(0,1) $ και ένα $ x_2\in(1,2) $ έτσι ώστε $ f(x_1)=f(x_2)=0 $ άρα η $ f $ έχει τουλάχιστον δύο ρίζες στο $ (0,2) $.}
%# End of file Ana-ThBolzano-NRizes-SolSE1