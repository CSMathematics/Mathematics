%# Database File : Ana-ThBolzano-MonadRiz-SolSE1----
Η αρχική εξίσωση γράφεται ισοδύναμα στη μορφή
\[ e^x+x-2=0 \]
και έτσι ορίζουμε τη συνάρτηση $ f:\mathbb{R}\to\mathbb{R} $ με $ f(x)=e^x+x-2 $. Για τη συνάρτηση αυτή θα έχουμε ότι
\begin{rlist}
\item είναι συνεχής στο διάστημα $ [0,1] $ και επιπλέον
\item \begin{itemize}
\item $ f(0)=e^0+0-2=-1<0 $
\item $ f(1)=e^1+1-2=e-1>0 $
\end{itemize}
άρα θα ισχύει $ f(0)\cdot f(1)=1-e<0 $.
\end{rlist}
Έτσι η εξίσωση θα έχει τουλάχιστον μια λύση $ x_0\in(0,1) $. Για να αποδείξουμε τη μοναδικότητα αυτής της λύσης εξετάζουμε τη συνάρτηση ως προς τη μονοτονία της. Έχουμε λοιπόν για κάθε $ x_1,x_2\in\mathbb{R} $ με $ x_1<x_2 $ ότι:
\begin{gather*}
x_1<x_2\Rightarrow e^{x_1}<e^{x_2}\Rightarrow\\ e^{x_1}+x_1<e^{x_2}+x_2\Rightarrow\\ e^{x_1}+x_1-2<e^{x_2}+x_2-2\Rightarrow\\ f(x_1)<f(x_2)
\end{gather*}
Επομένως η συνάρτηση $ f $ είναι γνησίως αύξουσα στο $ \mathbb{R} $ άρα η λύση $ x_0\in(0,1) $ είναι μοναδική.
%# End of file Ana-ThBolzano-MonadRiz-SolSE1