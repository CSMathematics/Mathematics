%# Database File : Ana-ThBolzano-RizKlDiast-SolSE1----
Η συνάρτηση $ f $ έχει πεδίο ορισμού το σύνολο $ \mathbb{R} $. Γι αυτήν επίσης θα έχουμε ότι:
\begin{rlist}
\item είναι συνεχής στο διάστημα $ [-a,a] $ και επιπλέον
\item $ f(-a)=\hm{(-a)} $\ \ και\ \ 
$ f(a)=\hm{a} $.\\
Γνωρίζουμε όμως ότι οι αντίθετες γωνίες $ -a $ και $ a $ έχουν αντίθετα ημίτονα άρα θα ισχύει $ \hm{(-a)}=-\hm{a} $ και έτσι παίρνουμε:
\[ f(-a)\cdot f(a)=\hm{(-a)}\cdot\hm{a}=-\hm^2{a}\leq0 \]
\end{rlist}
Εξετάζουμε τώρα τις παρακάτω περιπτώσεις:
\begin{itemize}
\item Αν $ f(-a)\cdot f(a)<0 $ τότε σύμφωνα με το θεώρημα Bolzano θα υπάρχει τουλάχιστον ένας αριθμός $ x_0 $ στο ανοικτό διάστημα $ (-a,a) $ τέτοιος ώστε
\[ f(x_0)=\hm{x_0}=0 \]
\item Αν $ f(-a)\cdot f(a)=0 $ τότε θα ισχύει $ f(-a)=0 $ ή $ f(a)=0 $ άρα το $ a $ θα είναι ρίζα της $ f $.
\end{itemize}
Από τις δύο παραπάνω περιπτώσεις καταλήγουμε στο συμπέρασμα ότι η ρίζα της συνάρτησης θα ανήκει στο κλειστό διάστημα $ [-a,a] $.
%# End of file Ana-ThBolzano-RizKlDiast-SolSE1