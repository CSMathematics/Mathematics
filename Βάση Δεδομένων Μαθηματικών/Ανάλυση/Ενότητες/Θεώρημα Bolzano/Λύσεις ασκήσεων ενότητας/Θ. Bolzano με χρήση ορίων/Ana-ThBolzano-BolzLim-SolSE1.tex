%# Database File : Ana-ThBolzano-BolzLim-SolSE1----
Παρατηρούμε ότι η συνάρτηση δεν ορίζεται στο $ 0 $ το οποίο είναι το κάτω άκρο του διαστήματος. Έτσι υπολογίζουμε το όριο της $ f $ στο $ 0 $ και έχουμε ότι:
\[ \lim_{x\to 0}{f(x)}=\lim_{x\to 0}{(\ln{x}+x)}=-\infty<0 \]
Άρα θα υπάρχει ένας πραγματικός αριθμός $ x_1 $ κοντά στο $ 0 $ έτσι ώστε $ f(x_1)<0 $. Στη συνέχεια εφαρμόζουμε το Θ. Bolzano για τη συνάρτηση $ f $ στο διάστημα $ [x_1,1] $ και ισχύει ότι:
\begin{rlist}
\item η $ f $ είναι συνεχής στο $ [x_1,1] $ ενώ
\item \begin{itemize}
\item $ f(x_1)<0 $
\item $ f(1)=\ln1+1=1>0 $
\end{itemize}
άρα παίρνουμε $ f(x_1)\cdot f(1)<0 $.
\end{rlist}
Έτσι, από το θεώρημα Bolzano, θα υπάρχει τουλάχιστον ένα $ x_0\in(x_1,1)\subseteq(0,1) $ τέτοιο ώστε $ f(x_0)=0 $.
%# End of file Ana-ThBolzano-BolzLim-SolSE1