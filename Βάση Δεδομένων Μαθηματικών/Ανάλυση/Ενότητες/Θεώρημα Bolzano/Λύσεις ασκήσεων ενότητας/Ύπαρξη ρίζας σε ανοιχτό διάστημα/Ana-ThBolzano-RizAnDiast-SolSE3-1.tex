%# Database File : Ana-ThBolzano-RizAnDiast-SolSE3----
Για να υπάρχει τουλάχιστον ένα κοινό σημείο $ A(x_0,f(x_0)) $ των δύο γραφικών παραστάσεων αρκεί ισοδύναμα να υπάρχει τουλάχιστον ένα $ x_0\in(2,4) $ τέτοιο ώστε $ f(x_0)=g(x_0) $. Απαιτούμε λοιπόν να ισχύει $ f(x)=g(x) $ και ορίζουμε τη συνάρτηση
\[ h(x)=f(x)-g(x)=x^2-3x-\ln{(x-1)}\ ,\ x\in(1,+\infty) \]
Για τη συνάρτηση $ h $ έχουμε ότι:
\begin{rlist}
\item είναι συνεχής στο διάστημα $ [2,4] $ και επίσης
\item \begin{itemize}
\item $ h(2)=2^2-3\cdot2-\ln1=-2<0 $
\item $ h(4)=4^2-3\cdot4-\ln{3}=4-\ln{3}>0 $
\end{itemize}
οπότε προκύπτει ότι $ h(2)\cdot h(4)=-2(4-\ln3)<0 $.
\end{rlist}
Έτσι σύμφωνα με το θεώρημα Bolzano υπάρχει τουλάχιστον ένα $ x_0\in(2,4) $ τέτοιο ώστε $ h(x_0)=0 $ ή ισοδύναμα $ f(x_0)=g(x_0) $.
%# End of file Ana-ThBolzano-RizAnDiast-SolSE3