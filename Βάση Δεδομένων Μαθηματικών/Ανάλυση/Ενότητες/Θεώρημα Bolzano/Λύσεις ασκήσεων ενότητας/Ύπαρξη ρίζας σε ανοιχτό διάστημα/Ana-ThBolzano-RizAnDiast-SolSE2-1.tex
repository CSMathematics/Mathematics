%# Database File : Ana-ThBolzano-RizAnDiast-SolSE2----
\wrapr{-5mm}{7}{5cm}{-5mm}{\parat
\begin{tcolorbox}[title=\Parathrhsh,hbox,    %%<<---- here
lifted shadow={1mm}{-2mm}{3mm}{0.3mm}%
{black!50!white}]
\begin{varwidth}{4cm}
{\small Παρόλο που δε γνωρίζουμε τις τιμές $ f(-1),f(1) $, το γινόμενό τους είναι μια γνήσια αρνητική παράσταση.}
\end{varwidth}
\end{tcolorbox}}{
Εξετάζουμε όπως προηγουμένως αν πληρούνται οι υποθέσεις του θεωρήματος Bolzano. Η συνάρτηση $ f $ είναι:
\begin{rlist}
\item συνεχής στο διάστημα $ [-1,1] $ και επίσης
\item \begin{itemize}
\item $ f(-1)=a(-1)^3-1=-a-1 $
\item $ f(1)=a\cdot1^3+1=a+1  $
\end{itemize}
οπότε θα ισχύει $ f(-1)\cdot f(1)=(-a-1)(a+1)=-(a+1)^2<0 $ αφού σύμφωνα με την υπόθεση $ a\neq-1 $.
\end{rlist}
Έτσι θα υπάρχει τουλάχιστον ένα $ x_0\in(-1,1) $ τέτοιο ώστε να ισχύει $ f(x_0)=0 $.}
%# End of file Ana-ThBolzano-RizAnDiast-SolSE2