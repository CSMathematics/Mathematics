%# Database File : DTX-Analysh-Synart-AntistrSyn-Def1
%@ Database source: Mathematics
Έστω μια συνάρτηση $ f:A\to\mathbb{R} $ με σύνολο τιμών $ f(A) $. Η συνάρτηση με την οποία κάθε $ y\in f(A) $ αντιστοιχεί σε ένα \textbf{μοναδικό} $ x\in A $ για το οποίο ισχύει $ f(x)=y $, λέγεται αντίστροφη συνάρτηση της $ f $.
\begin{center}
\begin{tikzpicture}[scale=.6]
\draw(0,0) ellipse (1cm and 1.5cm);
\draw(4,0) ellipse (1cm and 1.5cm);
\draw[fill=\xrwma!50] (4.1,0) ellipse (.6cm and 1.1cm);
\draw[latex-] (0,.2) arc (140:40:2.6);
\tkzDefPoint(0,.2){A}
\tkzDefPoint(4,.2){B}
\tkzDrawPoints(A,B)
\tkzLabelPoint[left](A){{\footnotesize $ x $}}
\tkzLabelPoint[right](B){{\footnotesize $ y $}}
\tkzText(0,1.8){$ A $}
\tkzText(4,1.8){$ B $}
\tkzText(2,1.45){$ f^{-1} $}
\draw[-latex] (3.5,0) -- (2.7,-1) node[anchor=north east] {\footnotesize $ f\left( A \right)  $};
\end{tikzpicture}
\end{center}
\begin{itemize}[itemsep=0mm]
\item Συμβολίζεται με $ f^{-1} $ και είναι $ f^{-1}:f(A)\to A $.
\item Το πεδίο ορισμού της $ f^{-1} $ είναι το σύνολο τιμών $ f(A) $ της $ f $, ενώ το σύνολο τιμών της $ f^{-1} $ είναι το πεδίο ορισμού $ A $ της $ f $.
\item Ισχύει ότι $ x=f^{-1}(y) $ για κάθε $ y\in f(A) $.
\end{itemize}
%# End of file DTX-Analysh-Synart-AntistrSyn-Def1