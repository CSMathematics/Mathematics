%# Database File : Analysh-Or-Prakseis_Synarthsewn
%@ Database source: Mathematics
Δίνονται δύο συναρτήσεις $ f,g $ με πεδία ορισμού $ A,B $ αντίστοιχα. 
\begin{enumerate}
\item Η συνάρτηση $ f+g $ του αθροίσματος των δύο συναρτήσεων ορίζεται ως η συνάρτηση με τύπο $ (f+g)(x)=f(x)+g(x) $ και πεδίο ορισμού $ D_{f+g}=A\cap B $.
\item Η συνάρτηση $ f-g $ της διαφοράς των δύο συναρτήσεων ορίζεται ως η συνάρτηση με τύπο $ (f-g)(x)=f(x)-g(x) $ και πεδίο ορισμού $ D_{f-g}=A\cap B $.
\item Η συνάρτηση $ f\cdot g $ του γινομένου των δύο συναρτήσεων ορίζεται ως η συνάρτηση με τύπο $ (f\cdot g)(x)=f(x)\cdot g(x) $ και πεδίο ορισμού $ D_{f\cdot g}=A\cap B $.
\item Η συνάρτηση $ \frac{f}{g} $ του πηλίκου των δύο συναρτήσεων ορίζεται ως η συνάρτηση με τύπο $ \left(\frac{f}{g}\right)(x)=\dfrac{f(x)}{g(x)} $ και πεδίο ορισμού $ D_{\frac{f}{g}}=\{x\in A\cap B:g(x)\neq 0\} $.
\end{enumerate}
Αν $ A\cap B=\varnothing $ τότε οι παραπάνω συναρτήσεις δεν ορίζονται.
%# End of file Analysh-Or-Prakseis_Synarthsewn