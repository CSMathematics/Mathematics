%# Database File : Analysh-Or-Synthesh_Synarthsewn
%@ Database source: Mathematics
Η σύνθεση μιας συνάρτησης $ f $ με μια συνάρτηση $ g $ με πεδία ορισμού $ A,B $ αντίστοιχα, ονομάζεται η συνάρτηση $ g\circ f $ με τύπο και πεδίο ορισμού
\[ (g\circ f)(x)=g(f(x))\ \ ,\ \ D_{g\circ f}=\{x\in\mathbb{R}| x\in A\ \textrm{ και }\ f(x)\in B\} \]
\begin{center}
\begin{tikzpicture}[scale=.6]
\draw(0,0) ellipse (1cm and 1.5cm);
\draw(4,0) ellipse (1cm and 1.5cm);
\begin{scope}
\draw[clip](4,0) ellipse (1cm and 1.5cm);
\draw[fill=\xrwma!30] (5,0) ellipse (1cm and 1.5cm);
\end{scope}
\draw (5,0) ellipse (1cm and 1.5cm);
\draw (9,0) ellipse (1cm and 1.5cm);
\draw (-.1,0)[fill=\xrwma!30] ellipse (.6cm and 1.1cm);
\draw[-latex] (0,.2) arc (140:40:2.95);
\draw[-latex] (4.5,.2) arc (140:40:2.95);
\draw[-latex] (0,.2) arc (140:40:5.9);
\tkzDefPoint(0,.2){A}
\tkzDefPoint(4.5,.15){B}
\tkzDefPoint(9,.15){C}
\tkzDrawPoints(A,B,C)
\tkzLabelPoint[left](A){{\footnotesize $ x $}}
\tkzLabelPoint[below](B){{\footnotesize $ f(x) $}}
\tkzLabelPoint[below](C){{\footnotesize $ g(f(x)) $}}
\tkzText(0,1.8){\footnotesize $ A $}
\tkzText(3.8,1.8){\footnotesize $ f(A) $}
\tkzText(5.2,1.8){\footnotesize $ B $}
\tkzText(2.4,1.55){\footnotesize $ f $}
\tkzText(6.4,1.55){\footnotesize $ g $}
\tkzText(4.5,2.55){\footnotesize $ g\circ f $}
\tkzText(9,1.8){\footnotesize $ g(B) $}
\draw[-latex](4.5,-0.8)--(3.2,-1.9)
node[anchor=east]{\footnotesize $ f\left( A \right)\cap B  $};
\draw[-latex] (0,-.5) -- (-.8,-1.7) node[anchor=east,xshift=1mm] {\footnotesize $ D_{g\circ f}  $};
\end{tikzpicture}
\end{center}
\begin{itemize}[itemsep=0mm]
\item Διαβάζεται «σύνθεση της $ f $ με τη $ g $» ή «$ g $ σύνθεση $ f $».
\item Για να ορίζεται η συνάρτηση $ g\circ f $ θα πρέπει να ισχύει $ f(A)\cap B\neq\varnothing $.
\item Αντίστοιχα ορίζεται και η σύνθεση $ f\circ g $ με πεδίο ορισμού το $ D_{f\circ g}=\{x\in\mathbb{R}|x\in B\ \ \textrm{και}\ \ g(x)\in A\} $ και τύπο $ (f\circ g)(x)=f(g(x)) $.
\end{itemize}
%# End of file Analysh-Or-Synthesh_Synarthsewn