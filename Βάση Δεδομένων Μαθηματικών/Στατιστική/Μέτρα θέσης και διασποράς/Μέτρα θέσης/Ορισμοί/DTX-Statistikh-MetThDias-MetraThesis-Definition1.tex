%# Database File : DTX-Statistikh-MetThDias-MetraThesis-Definition1
%@ Database source: Mathematics
Η μέση τιμή ορίζεται ως το πηλίκο του αθροίσματος των παρατηρήσεων ενός δείγματος προς το πλήθος τους. Συμβολίζεται $ \bar{x} $ και είναι :
\[ \bar{x}=\frac{t_1+t_2+\ldots+t_\nu}{\nu}=\frac{1}{\nu}\sum_{i=1}^{\nu}{t_i} \]
Εναλλακτικοί τύποι για τη μέση τιμή είναι οι ακόλουθοι οι οποίοι χρησιμοποιούνται σε κατανομές συχνοτήτων. Αν κάποια μεταβλητή $ X $ έχει τιμές $ x_1,x_2\ldots,x_\kappa $ με συχνότητες $ \nu_1,\nu_2\ldots,\nu_\kappa $ και σχετικές συχνότητες $ f_1,f_2,\ldots,f_\kappa $ τότε θα έχουμε :
\[ \bar{x}=\frac{1}{\nu}\sum_{i=1}^{\kappa}{x_i\nu_i}\ \textrm{ και }\ \bar{x}=\sum_{i=1}^{\kappa}{x_if_i} \]
%# End of file DTX-Statistikh-MetThDias-MetraThesis-Definition1