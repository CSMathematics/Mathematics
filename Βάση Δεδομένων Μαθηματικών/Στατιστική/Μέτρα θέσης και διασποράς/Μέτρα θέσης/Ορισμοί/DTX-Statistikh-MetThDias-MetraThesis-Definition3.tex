%# Database File : DTX-Statistikh-MetThDias-MetraThesis-Definition3
%@ Database source: Mathematics
Διάμεσος ονομάζεται η κεντρική παρατήρηση $ \nu $ σε πλήθους παρατηρήσεων όταν αυτές έχουν τοποθετηθεί σε αύξουσα σειρά. Συμβολίζεται με $ \delta $. Ξεχωρίζουμε τις εξής περιπτώσεις :
\begin{rlist}
\item Αν το πλήθος των $ \nu $ παρατηρήσεων είναι περιττό τότε η διάμεσος ισούται με τη μεσαία παρατήρηση.
\[ \delta=t_{_{\frac{\nu}{2}}} \]
\item Αν το πλήθος των $ \nu $ παρατηρήσεων είναι άρτιο τότε η διάμεσος ισούται με το ημιάθροισμα των δύο μεσαίων παρατηρήσεων.
\[ \delta=\frac{t_{_{\frac{\nu}{2}}}+t_{_{\frac{\nu}{2}+1}}}{2} \]
\end{rlist}
Η διάμεσος σε κατανομή συχνοτήτων ισούται με την τιμή $ x_i $ για την οποία η σχετική αθροιστική συχνότητα $ F_i\% $ είτε ισούται είτε ξεπερνάει για πρώτη φορά το $ 50\% $. Δηλαδή
\[ \delta=x_i\ \textrm{ για την οποία }\ F_{i-1}\%<50\%\leq F_i\% \]
%# End of file DTX-Statistikh-MetThDias-MetraThesis-Definition3