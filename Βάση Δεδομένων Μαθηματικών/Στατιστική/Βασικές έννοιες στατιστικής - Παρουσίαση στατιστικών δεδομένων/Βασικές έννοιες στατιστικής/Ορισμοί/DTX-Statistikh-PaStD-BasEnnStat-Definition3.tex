%# Database File : DTX-Statistikh-PaStD-BasEnnStat-Definition3
%@ Database source: Mathematics
Μεταβλητή ονομάζεται το χαρακτηριστικό ως προς το οποίο εξετάζονται τα στοιχεία ενός πληθυσμού.
\begin{itemize}
\item Συμβολίζεται με οποιοδήποτε κεφαλαίο γράμμα : $ X,Y,A,B,\ldots $
\item Οι πιθανές τιμές οι οποίες μπορεί να πάρει μια μεταβλητή ονομάζονται \textbf{τιμές της μεταβλητής}. Συμβολίζονται με το ίδιο μικρό γράμμα του ονόματος της μεταβλητής π.χ. $ x_i,y_i\ldots $ όπου ο δείκτης $ i $ φανερώνει τον αύξοντα αριθμό της τιμής.
\item Τα στατιστικά δεδομένα που συλλέγονται από ένα πληθυσμό ή δείγμα που εξετάζεται ως προς κάποια μεταβλητή ονομάζονται \textbf{παρατηρήσεις}. Συμβολίζονται συνήθως με $ t_i $ όπου ο δείκτης $ i $ φανερώνει τον αύξοντα αριθμό της παρατήρησης.
\end{itemize} 
Οι μεταβλητές διακρίνονται στις εξής κατηγορίες :
\begin{enumerate}[label=\bf\arabic*.]
\item \textbf{Ποιοτικές}\\
Ποιοτική ονομάζεται κάθε μεταβλητή της οποίας οι τιμές δεν είναι αριθμητικές.
\item \textbf{Ποσοτικές}\\
Ποσοτική ονομάζεται κάθε μεταβλητή της οποίας οι τιμές είναι αριθμοί. Οι ποσοτικές μεταβλητές χωρίζονται σε διακριτές και συνεχείς.
\begin{rlist}
\item \textbf{Διακριτές} ονομάζονται οι ποσοτικές μεταβλητές που παίρνουν μεμονωμένες τιμές από το σύνολο των πραγματικών αριθμών ή ένα διάστημα αυτού.
\item \textbf{Συνεχείς} ονομάζονται οι ποσοτικές μεταβλητές που παίρνουν όλες τις τιμές στο σύνολο ή σε ένα διάστημα πραγματικών αριθμών.
\end{rlist}
\end{enumerate}
%# End of file DTX-Statistikh-PaStD-BasEnnStat-Definition3