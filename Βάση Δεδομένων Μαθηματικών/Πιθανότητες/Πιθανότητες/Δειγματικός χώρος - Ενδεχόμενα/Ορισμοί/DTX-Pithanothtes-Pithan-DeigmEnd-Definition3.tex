%# Database File : DTX-Pithanothtes-Pithan-DeigmEnd-Definition3
%@ Database source: Mathematics
Δίνεται o δειγματικός χώρος $ \varOmega=\{\omega_1,\omega_2,\ldots,\omega_{\nu}\} $ ενός πειράματος τύχης. Ενδεχόμενο ονομάζεται οποιοδήποτε σύνολο $A$ το οποίο περιέχει ένα ή περισσότερα στοιχεία του δειγματικού χώρου.
\begin{itemize}[itemsep=0mm]
\item Κάθε ενδεχόμενο είναι υποσύνολο του δειγματικού του χώρου.
\item Τα ενδεχόμενα που έχουν ένα στοιχείο ονομάζονται \textbf{απλά} ενδεχόμενα, ενώ αν περιέχουν περισσότερα στοιχεία ονομάζονται \textbf{σύνθετα}.
\item Εάν το αποτέλεσμα ενός πειράματος είναι στοιχείο ενός ενδεχομένου τότε λέμε ότι το ενδεχόμενο \textbf{πραγματοποιείται}.
\item Τα στοιχεία ενός ενδεχομένου ονομάζονται \textbf{ευνοϊκές περιπτώσεις}.
\item Ο δειγματικός χώρος $ \varOmega $ ονομάζεται \textbf{βέβαιο} ενδεχόμενο, ενώ το κενό σύνολο ονομάζεται \textbf{αδύνατο} ενδεχόμενο.
\item Εάν δύο ενδεχόμενα $ A,B $ δεν έχουν κοινά στοιχεία τότε ονομάζονται \textbf{ασυμβίβαστα} ή ξένα μεταξύ τους δηλαδή : \[ A,B \textrm{ ασυμβίβαστα }\Leftrightarrow A\cap B=\varnothing \]
\end{itemize}
%# End of file DTX-Pithanothtes-Pithan-DeigmEnd-Definition3