%# Database File : DTX-Pithanothtes-Pithan-Pithanothta-Definition2
%@ Database source: Mathematics
Η πιθανότητα ενός ενδεχομένου $ A=\{a_1,a_2,\ldots,a_\kappa\} $ ενός δειγματικού χώρου $ \varOmega=\{\omega_1,\omega_2,\ldots,\omega_\nu\} $ ορίζεται ώς το άθροισμα των πιθανοτήτων $ P(a_i)\ ,\ i=1,2,\ldots,\nu $ των απλών ενδεχομένων του.
\[ P(A)=P(a_1)+P(a_2)+\ldots+P(a_\kappa) \]
\begin{itemize}[itemsep=0mm]
\item Για κάθε στοιχείο $ \omega_i\ ,\ i=1,2,\ldots,\nu $ του δειγματικού χώρου $ \varOmega $ ονομάζουμε τον αριθμό $ P(\omega_i) $ πιθανότητα του ενδεχομένου $ \{\omega_i\} $.
\item Ο παραπάνω ορισμός ονομάζεται \textbf{αξιωματικός ορισμός} της πιθανότητας και εφαρμόζεται όταν το ενδεχόμενο $ A $ δεν αποτελείται από ισοπίθανα απλά ενδεχόμενα $ \{a_i\}\ ,\ i=1,2,\ldots,\kappa $.
\end{itemize}
%# End of file DTX-Pithanothtes-Pithan-Pithanothta-Definition2