%# Database File : DTX-Pithanothtes-Pithan-Pithanothta-Definition1
%@ Database source: Mathematics
Πιθανότητα ενός ενδεχομένου $ A=\{a_1,a_2,\ldots,a_\kappa\} $ ενός δειγματικού χώρου $ \varOmega $ ονομάζεται ο λόγος του πλήθους των ευνοϊκών περιπτώσεων του $ A $ προς το πλήθος όλων των δυνατών περιπτώσεων.
\[ P(A)=\frac{N(A)}{N(\varOmega)} \]
\begin{itemize}[itemsep=0mm]
\item Ο παραπάνω ορισμός ονομάζεται \textbf{κλασικός ορισμός} της πιθανότητας και εφαρμόζεται όταν το ενδεχόμενο $ A $ αποτελείται από ισοπίθανα απλά ενδεχόμενα $ \{a_i\}\ ,\ i=1,2,\ldots,\kappa $.
\item Το πλήθος των στοιχείων ενός ενδεχομένου $ A $ συμβολίζεται με $ N(A) $.
\end{itemize}
%# End of file DTX-Pithanothtes-Pithan-Pithanothta-Definition1