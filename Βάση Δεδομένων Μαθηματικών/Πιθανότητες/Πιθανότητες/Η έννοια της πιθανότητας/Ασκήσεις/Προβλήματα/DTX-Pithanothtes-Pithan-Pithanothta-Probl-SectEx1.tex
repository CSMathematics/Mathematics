%# Database File : DTX-Pithanothtes-Pithan-Pithanothta-Probl-SectEx1
%@ Database source: Mathematics
Ο καθηγητής των μαθηματικών σε ένα λύκειο πρόκειται να επιλέξει μαθητές από όλες τις τάξεις για να εκπροσωπήσουν το σχολείο στη διεθνή Μαθηματική Ολυμπιάδα του \the\year{}. Θα πρέπει να λάβει υπ όψιν του το αν ο μαθητής είναι αγόρι (α) ή κορίτσι (κ), την τάξη στην οποία πηγαίνει (Α΄, Β΄ ή Γ΄) και το αν έχει συμμετάσχει ξανά (ναι : (ν) ή όχι (ο) ) σε οποιονδήποτε διαγωνισμό μαθηματικών.
\begin{rlist}
\item Να βρεθεί ο δειγματικός χώρος του πειράματος.
\item Αν ο καθηγητής επιλέξει τυχαία έναν μαθητή να βρεθεί το ενδεχόμενο
\begin{enumerate}[itemsep=0mm]
\item[Α :] Ο μαθητής να είναι αγόρι.
\item[Β :] Ο μαθητής να ανήκει σε κάποια ομάδα προσανατολισμού και να μην έχει συμμετάσχει ξανά σε διαγωνισμό
\item[Γ :] Ο μαθητής να είναι κορίτσι και να έχει συμμετάσχει ξανά σε μαθηματικό διαγωνισμό.
\end{enumerate}
\item Να υπολογιστούν οι πιθανότητες των παραπάνω ενδεχομένων Α, Β, Γ.
\end{rlist}
%# End of file DTX-Pithanothtes-Pithan-Pithanothta-Probl-SectEx1