\documentclass[11pt]{article}
\usepackage[english,greek]{babel}
\usepackage[utf8]{inputenc}
\usepackage{nimbusserif}
\usepackage[T1]{fontenc}
\usepackage[left=2.00cm, right=2.00cm, top=3.00cm, bottom=2.00cm,a4paper]{geometry}
\usepackage{amsmath}
\let\myBbbk\Bbbk
\let\Bbbk\relax
\usepackage[amsbb,subscriptcorrection,zswash,mtpcal,mtphrb,mtpfrak]{mtpro2}
\usepackage{graphicx,multicol,multirow,enumitem,tabularx,mathimatika,gensymb,venndiagram,hhline,longtable,tkz-euclide,fontawesome5,eurosym,tcolorbox,tabularray}
\tcbuselibrary{skins,theorems,breakable}
\newlist{rlist}{enumerate}{3}
\setlist[rlist]{itemsep=0mm,label=\roman*.}
\newlist{alist}{enumerate}{3}
\setlist[alist]{itemsep=0mm,label=\alph*.}
\newlist{balist}{enumerate}{3}
\setlist[balist]{itemsep=0mm,label=\bf\alph*.}
\newlist{Alist}{enumerate}{3}
\setlist[Alist]{itemsep=0mm,label=\Alph*.}
\newlist{bAlist}{enumerate}{3}
\setlist[bAlist]{itemsep=0mm,label=\bf\Alph*.}
\renewcommand{\textstigma}{\textsigma\texttau}
\newlist{thema}{enumerate}{3}
\setlist[thema]{label=\bf\large{ΘΕΜΑ \textcolor{black}{\Alph*}},itemsep=0mm,leftmargin=0cm,itemindent=18mm}
\newlist{erwthma}{enumerate}{3}
\setlist[erwthma]{label=\bf{\large{\textcolor{black}{\Alph{themai}.\arabic*}}},itemsep=0mm,leftmargin=0.8cm}

\newcommand{\lysh}{\textcolor{black}{\textbf{\faCheck\ \ ΛΥΣΗ}}}
\renewcommand{\textstigma}{\textsigma\texttau}
%----------- ΟΡΙΣΜΟΣ------------------
\newcounter{orismos}[section]
\renewcommand{\theorismos}{\thesection.\arabic{orismos}}   
\newcommand{\Orismos}{\refstepcounter{orismos}{\textbf{\textcolor{black}{\kerkissans{Ορισμός\hspace{2mm}\theorismos}}\;:\;}{}}}

\newenvironment{orismos}[1]
{\begin{tcolorbox}[title=\Orismos {\textcolor{black}{\kerkissans{#1}}},breakable,bottomtitle=-1.5mm,
enhanced standard,titlerule=-.2pt,toprule=0pt, rightrule=0pt, bottomrule=0pt,
colback=white,left=2mm,top=1mm,bottom=0mm,
boxrule=0pt,
colframe=white,borderline west={1.5mm}{0pt}{black},leftrule=2mm,sharp corners,coltitle=black]}
{\end{tcolorbox}}

\newcommand{\kerkissans}[1]{{\fontfamily{maksf}\selectfont \textbf{#1}}}
\renewcommand{\textdexiakeraia}{}

\usepackage[
backend=biber,
style=alphabetic,
sorting=ynt
]{biblatex}

\DeclareTblrTemplate{caption}{nocaptemplate}{}
\DeclareTblrTemplate{capcont}{nocaptemplate}{}
\DeclareTblrTemplate{contfoot}{nocaptemplate}{}
\NewTblrTheme{mytabletheme}{
  \SetTblrTemplate{caption}{nocaptemplate}{}
  \SetTblrTemplate{capcont}{nocaptemplate}{}
  \SetTblrTemplate{contfoot}{nocaptemplate}{}
}

\NewTblrEnviron{mytblr}
\SetTblrStyle{firsthead}{font=\bfseries}
\SetTblrStyle{firstfoot}{fg=red2}
\SetTblrOuter[mytblr]{theme=mytabletheme}
\SetTblrInner[mytblr]{
rowspec={t{7mm}},columns = {c},
  width = 0.85\linewidth,
  row{odd} = {bg=cyan!20!white!80,fg=black,ht=8mm},
 row{even} = {bg=cyan!70!gray!50,fg=black,ht=8mm},
hlines={white},vlines={white},
row{1} = {bg=cyan!70!black, fg=white, font=\Large\bfseries\fontfamily{maksf}},rowhead = 1,
  hline{2} = {.7mm}, % midrule  
}

\begin{document}
%\[ \sqrt{x}=a\;\;,\;\;\textrm{ όπου }x\geq0\textrm{ και }a\geq0 \]
%\begin{itemize}[itemsep=0mm]
%\item Το $ x $ ονομάζεται \textbf{υπόριζο}.
%\item Δεν ορίζεται ρίζα αρνητικού αριθμού.
%\end{itemize}
%\[ \sqrt[\nu]{x}=a\;\;,\;\;\textrm{ όπου }x\geq0\textrm{ και }a\geq0 \]
%
%\[ a^{\frac{\mu}{\nu}}=\!\sqrt[\nu]{a^\mu} \]
%\begin{itemize}
%\item $a>0$ αν $\mu\in\mathbb{Z}$ και $\nu\in\mathbb{N}^*$
%\item $a\geq 0$ αν $\mu,\nu\in\mathbb{N}^*$
%\end{itemize}
%\begin{itemize}[label*=\faEdit]
%\item $ \left(\sqrt{x}\;\right)^2=x\;\;,\;\; x\geq0  $
%\item  $ \sqrt{x^2}=|x|\;\;,\;\; x\in\mathbb{R} $
%\item $ \sqrt{x\cdot y}=\!\sqrt{x}\cdot\!\sqrt{y}\;\;,\;\; x,y\geq0 $
%\item $ \sqrt{\dfrac{x}{y}}\;=\dfrac{\sqrt{x}}{\sqrt{y}}\;\;,\;\; x\geq0\textrm{ και }y>0 $
%\item $ \sqrt{x\pm y}\neq\!\sqrt{x}\pm\!\sqrt{y}\;\;,\;\; x,y\geq0 $
%\end{itemize}
%----------
%\begin{itemize}[label*=\faEdit]
%\item $ \left(\sqrt[\nu]{x}\;\right)^\nu=x\;\;,\;\; x\geq0  $ 
%\item $ \sqrt[\nu]{x^\nu}=\begin{cases}
%|x|&  ,x\in\mathbb{R}\textrm{ αν }\nu\textrm{ άρτιος}\\x&  ,x\geq0\textrm{ και } \nu\in\mathbb{N}\end{cases} $
%\item $ \sqrt[\nu]{x\cdot y}=\!\sqrt[\nu]{x}\cdot\!\sqrt[\nu]{y}\;\;,\;\; x,y\geq0 $ 
%\item $ \sqrt[\nu]{\dfrac{x}{y}}\;=\dfrac{\sqrt[\nu]{x}}{\sqrt[\nu]{y}}\;\;,\;\; x\geq0\textrm{ και }y>0 $
%\item $ \sqrt[\nu]{x\pm y}\neq\!\sqrt[\nu]{x}\pm\!\sqrt[\nu]{y}\;\;,\;\; x,y\geq0 $
% \item $ \sqrt[\mu]{\!\sqrt[\nu]{x}}=\!\sqrt[\nu\cdot\mu]{x}\;\;,\;\; x\geq0 $ 
%\item $ \sqrt[\nu]{x^\nu\cdot y}=x\!\sqrt[\nu]{y}\;\;,\;\; x,y\geq0  $ 
% \item $ \sqrt[\mu\cdot\rho]{x^{\nu\cdot\rho}}=\!\sqrt[\mu]{x^{\nu}}\;\;,\;\; x\geq0 $ 
%\item $ \sqrt[\nu]{x_1\cdot x_2\cdot\ldots\cdot x_\nu}=\!\sqrt[\nu]{x_1}\cdot\!\sqrt[\nu]{x_2}\cdot\ldots\cdot\!\sqrt[\nu]{x_\nu} $\\όπου $ x_1,x_2,\ldots x_\nu\geq0 $ και $ \nu\in\mathbb{N} $.
%\item $ \sqrt[\mu_1]{\!\sqrt[\mu_2]{\mbox{}^{\ddots}\sqrt[\mu_{\nu}]{x}}}\;\;\;=\sqrt[\mu_1\cdot\mu_2\cdot\ldots\cdot\mu_\nu]{x}$\\με $ x\geq0 $ και $ \mu_1,\mu_2,\ldots,\mu_\nu\in\mathbb{N} $.
%\end{itemize}
%\newpage
%$a\cdot a\cdot a\cdot\ldots\cdot a=a^{\nu}$ όπου $a\in\mathbb{R}$ και $\nu\in\mathbb{N}$
%\begin{itemize}[itemsep=0mm]
%\item Ο αριθμός $a$ λέγεται \textbf{βάση} της δύναμης.
%\item Ο αριθμός $\nu$ λέγεται \textbf{εκθέτης} της δύναμης.
%\item Η δύναμη $a^2$ λέγεται και \textbf{στο τετράγωνο}.
%\item Η δύναμη $a^3$ λέγεται και \textbf{στον κύβο}.
%\end{itemize}
%\begin{itemize}[label*={\footnotesize \faPlay},itemsep=0mm]
%\item $ a^1=a\;$
%\item $\;a^0=1\;,\;\textrm{όπου }a\neq0$
%\item $a^{-\nu}=\dfrac{1}{a^\nu}\;,\;\textrm{όπου }a\neq0$
%\item $ a^\nu\cdot a^\mu=a^{\nu+\mu} $
%\item $ a^\nu: a^\mu=a^{\nu-\mu} $
%\item $ \left(a\cdot\beta\right)^\nu=a^\nu\cdot\beta^\nu $
%\item $ \left(\dfrac{a}{\beta}\right)^\nu=\dfrac{a^\nu}{\beta^\nu}\;\;,\;\;\beta\neq0 $
%\item $ \left( a^\nu\right)^\mu=a^{\nu\cdot\mu} $
%\item $ \left( \dfrac{a}{\beta}\right)^{-\nu}=\left(\dfrac{\beta}{a}\right)^\nu\;\;,\;\;a,\beta\neq0 $ 
%\item $
%a^{\nu_1}\cdot a^{\nu_2}\cdot\ldots\cdot a^{\nu_\kappa}=a^{\nu_1+\nu_2+\ldots+\nu_\kappa}$
%\item $ 
%\left( a_1\cdot a_2\cdot\ldots\cdot a_\kappa\right)^\nu=a_1^\nu\cdot a_2^\nu\cdot\ldots\cdot a_\kappa^\nu $
%\end{itemize}
\faCheck
\end{document}