\documentclass[11pt,landscape]{article}
\usepackage[english,greek]{babel}
\usepackage[utf8]{inputenc}
\usepackage{nimbusserif}
\usepackage[T1]{fontenc}
\usepackage[left=2.00cm, right=2.00cm, top=3.00cm, bottom=2.00cm,a3paper]{geometry}
\usepackage{amsmath}
\let\myBbbk\Bbbk
\let\Bbbk\relax
\usepackage[amsbb,subscriptcorrection,zswash,mtpcal,mtphrb,mtpfrak]{mtpro2}
\usepackage{graphicx,multicol,multirow,enumitem,tabularx,mathimatika,gensymb,venndiagram,hhline,longtable,tkz-euclide,fontawesome5,eurosym,tcolorbox,tabularray}
\tcbuselibrary{skins,theorems,breakable}
\newlist{rlist}{enumerate}{3}
\setlist[rlist]{itemsep=0mm,label=\roman*.}
\newlist{alist}{enumerate}{3}
\setlist[alist]{itemsep=0mm,label=\alph*.}
\newlist{balist}{enumerate}{3}
\setlist[balist]{itemsep=0mm,label=\bf\alph*.}
\newlist{Alist}{enumerate}{3}
\setlist[Alist]{itemsep=0mm,label=\Alph*.}
\newlist{bAlist}{enumerate}{3}
\setlist[bAlist]{itemsep=0mm,label=\bf\Alph*.}
\renewcommand{\textstigma}{\textsigma\texttau}
\newlist{thema}{enumerate}{3}
\setlist[thema]{label=\bf\large{ΘΕΜΑ \textcolor{white}{\Alph*}},itemsep=0mm,leftmargin=0cm,itemindent=18mm}
\newlist{erwthma}{enumerate}{3}
\setlist[erwthma]{label=\bf{\large{\textcolor{white}{\Alph{themai}.\arabic*}}},itemsep=0mm,leftmargin=0.8cm}

\newcommand{\lysh}{\textcolor{white}{\textbf{\faCheck\ \ ΛΥΣΗ}}}
\renewcommand{\textstigma}{\textsigma\texttau}
%----------- ΟΡΙΣΜΟΣ------------------
\newcounter{orismos}[section]
\renewcommand{\theorismos}{\thesection.\arabic{orismos}}   
\newcommand{\Orismos}{\refstepcounter{orismos}{\textbf{\textcolor{white}{\kerkissans{Ορισμός\hspace{2mm}\theorismos}}\;:\;}{}}}

\newenvironment{orismos}[1]
{\begin{tcolorbox}[title=\Orismos {\textcolor{white}{\kerkissans{#1}}},breakable,bottomtitle=-1.5mm,
enhanced standard,titlerule=-.2pt,toprule=0pt, rightrule=0pt, bottomrule=0pt,
colback=white,left=2mm,top=1mm,bottom=0mm,
boxrule=0pt,
colframe=white,borderline west={1.5mm}{0pt}{white},leftrule=2mm,sharp corners,coltitle=white]}
{\end{tcolorbox}}

\newcommand{\kerkissans}[1]{{\fontfamily{maksf}\selectfont \textbf{#1}}}
\renewcommand{\textdexiakeraia}{}

\usepackage[
backend=biber,
style=alphabetic,
sorting=ynt
]{biblatex}

\DeclareTblrTemplate{caption}{nocaptemplate}{}
\DeclareTblrTemplate{capcont}{nocaptemplate}{}
\DeclareTblrTemplate{contfoot}{nocaptemplate}{}
\NewTblrTheme{mytabletheme}{
  \SetTblrTemplate{caption}{nocaptemplate}{}
  \SetTblrTemplate{capcont}{nocaptemplate}{}
  \SetTblrTemplate{contfoot}{nocaptemplate}{}
}

\NewTblrEnviron{mytblr}
\SetTblrStyle{firsthead}{font=\bfseries}
\SetTblrStyle{firstfoot}{fg=red2}
\SetTblrOuter[mytblr]{theme=mytabletheme}
\SetTblrInner[mytblr]{
rowspec={t{7mm}},columns = {c},
  width = 0.85\linewidth,
  row{odd} = {bg=cyan!20!white!80,fg=black,ht=8mm},
 row{even} = {bg=cyan!70!gray!50,fg=black,ht=8mm},
hlines={white},vlines={white},
row{1} = {bg=cyan!70!black, fg=white, font=\Large\bfseries\fontfamily{maksf}},rowhead = 1,
  hline{2} = {.7mm}, % midrule  
}

\begin{document}
\begin{mytblr}{column{2} = {c,5.5cm},column{3} = {c,4.3cm},column{4} = {c,5.8cm},column{5} = {c,6.7cm},column{1} = {font=\Large\bfseries\fontfamily{maksf},m}}
& {Παραλληλόγραμμο} & {Ορθογώνιο} & {Ρόμβος} & {Τετράγωνο} \\
Σχήμα  & \begin{tikzpicture}
\tkzDefPoint(-3,-.5){D}
\tkzDefPoint(-2,1){A}
\tkzDefPoint(.5,1){B}
\tkzDefPoint(-.5,-.5){C}
\tkzDefPoint(-2,1){E}
\draw[pl] (-3,-0.5) -- (-2,1) -- (0.5,1) -- (-0.5,-0.5) -- cycle;
\tkzLabelPoint[above](A){$A$}
\tkzLabelPoint[above](B){$B$}
\tkzLabelPoint[below](C){$\varGamma$}
\tkzLabelPoint[below](D){$\varDelta$}
\draw[pl] (A)--(C);
\draw[pl] (B)--(D);
\tkzInterLL(A,C)(B,D)\tkzGetPoint{O}
\tkzLabelPoint[above,xshift=.3mm](O){$O$}
\tkzDrawPoints(A,B,C,D,O)
\end{tikzpicture} & \begin{tikzpicture}[scale=1]
\tkzDefPoint(0,0){D}
\tkzDefPoint(0,1.8){A}
\tkzDefPoint(3,1.8){B}
\tkzDefPoint(3,0){C}
\tkzDefPoint(1.5,.9){O}
\draw[pl] (0,0) -- (0,1.8) -- (3,1.8) -- (3,0) -- cycle;
\draw[pl] (A)--(C);
\draw[pl] (B)--(D);
\tkzMarkRightAngle(C,D,A)
\tkzMarkRightAngle(B,C,D)
\tkzMarkRightAngle(D,A,B)
\tkzMarkRightAngle(A,B,C)
\tkzLabelPoint[above left](A){$A$}
\tkzLabelPoint[above right](B){$B$}
\tkzLabelPoint[right](C){$\varGamma$}
\tkzLabelPoint[left](D){$\varDelta$}
\tkzLabelPoint[above](O){$O$}
\tkzDrawPoints(A,B,C,D,O)
\end{tikzpicture} & \begin{tikzpicture}[scale=.7]
\tkzDefPoint(0,1.5){D}
\tkzDefPoint(3,3){A}
\tkzDefPoint(6,1.5){B}
\tkzDefPoint(3,0){C}
\tkzDefPoint(3,1.5){O}
\tkzMarkRightAngle[size=.4](B,O,A)
\tkzMarkAngle[size=.7](B,D,A)
\tkzMarkAngle[size=.7](C,D,B)
\tkzMarkAngle[size=.7](A,B,D)
\tkzMarkAngle[size=.7](D,B,C)
\tkzMarkAngle[size=.5](D,A,C)
\tkzMarkAngle[size=.5](C,A,B)
\tkzMarkAngle[size=.5](B,C,A)
\tkzMarkAngle[size=.5](A,C,D)
\draw[pl] (A)--(B)--(C)--(D) -- cycle;
\draw[pl] (A)--(C);
\draw[pl] (B)--(D);
\tkzLabelPoint[above](A){$A$}
\tkzLabelPoint[right](B){$B$}
\tkzLabelPoint[below](C){$\varGamma$}
\tkzLabelPoint[left](D){$\varDelta$}
\tkzLabelPoint[above left](O){$O$}
\tkzDrawPoints(A,B,C,D,O)
\end{tikzpicture} & \begin{tikzpicture}[scale=.7]
\tkzDefPoint(0,-1.5){D}
\tkzDefPoint(0,1.5){A}
\tkzDefPoint(3,1.5){B}
\tkzDefPoint(3,-1.5){C}
\tkzDefPoint(1.5,0){O}
\tkzMarkRightAngle[scale=1.5](C,D,A)
\tkzMarkRightAngle[scale=1.5](B,C,D)
\tkzMarkRightAngle[scale=1.5](D,A,B)
\tkzMarkRightAngle[scale=1.5](A,B,C)
\draw[pl] (A) -- (B) -- (C) -- (D) -- cycle;
\draw[pl] (A)--(C);
\draw[pl] (B)--(D);
\tkzLabelPoint[above](A){$A$}
\tkzLabelPoint[above](B){$B$}
\tkzLabelPoint[below](C){$\varGamma$}
\tkzLabelPoint[below](D){$\varDelta$}
\tkzLabelPoint[above](O){$O$}
\tkzDrawPoints(A,B,C,D,O)
\end{tikzpicture} \\
\textbf{Ορισμός}  &\large Παραλληλόγραμμο ονομάζεται το τετράπλευρο το οποίο έχει τις απέναντι πλευρές του ανά δύο παράλληλες. &\large Ορθογώνιο ονομάζεται το παραλληλόγραμμο το οποίο έχει μία γωνία ορθή &\large Ρόμβος ονομάζεται το παραλληλόγραμμο το οποίο έχει δύο διαδοχικές πλευρές ίσες. &\large Τετράγωνο ονομάζεται το παραλληλόγραμμο το οποίο είναι και ορθογώνιο και ρόμβος. \\
\textbf{Ιδιότητες} & 
\large\begin{alist}[leftmargin=5mm]
\item Οι απέναντι πλευρές είναι ίσες.
\item Οι απέναντι γωνίες είναι ίσες.
\item Δύο διαδοχικές γωνίες είναι παραπληρωματικές.
\item Οι διαγώνιοι διχοτομούνται.
\end{alist} & 
\large\begin{alist}[leftmargin=5mm]
\item Οι διαγώνιοι είναι ίσες.
\item Όλες οι γωνίες είναι ίσες.
\item Έχει όλες τις ιδιότητες ενός παραλληλογράμμου.
\end{alist} & 
\large\begin{alist}[leftmargin=5mm]
\item Οι διαδοχικές πλευρές είναι ίσες.
\item Οι διαγώνιοι τέμνονται κάθετα.
\item Οι διαγώνιοι διχοτομούν τις γωνίες του.
\item Έχει όλες τις ιδιότητες ενός παραλληλογράμμου.
\end{alist} & 
\large\begin{alist}[leftmargin=5mm]
\item Όλες οι πλευρές είναι ίσες.
\item Όλες οι γωνίες είναι ίσες.
\item Οι απέναντι πλευρές είναι παράλληλες.
\item Οι διαγώνιοι είναι ίσες, διχοτομούν τις γωνίες του και τέμνονται κάθετα.
\end{alist} \\
\textbf{Κριτήρια}  & 
\large \begin{alist}[leftmargin=5mm]
\item Οι απέναντι πλευρές είναι παράλληλες.
\item Οι απέναντι πλευρές είναι ίσες.
\item Δύο απέναντι πλευρές είναι παράλληλες και ίσες.
\item Οι απέναντι γωνίες είναι ίσες.
\item Οι διαγώνιοι διχοτομούνται.
\end{alist} & 
\large\begin{alist}[leftmargin=5mm]
\item Είναι παραλληλόγραμμο και έχει μια ορθή γωνία.
\item Είναι παραλληλόγραμμο και οι διαγώνιοι είναι ίσες.
\item Έχει 3 ορθές γωνίες.
\item Έχει όλες τις γωνίες ίσες.
\end{alist} & 
\large\begin{alist}[leftmargin=5mm]
\item Όλες οι πλευρές του είναι ίσες.
\item Είναι παραλληλόγραμμο και έχει δύο διαδοχικές πλευρές ίσες.
\item Είναι παραλληλόγραμμο και έχει διαγώνιους κάθετες.
\item Είναι παραλληλόγραμμο και μια διαγώνιος διχοτομεί μια γωνία.
\end{alist} & Παραλληλόγραμμο και
\large\begin{alist}[leftmargin=5mm]
\item Έχει μια ορθή γωνία και δύο διαδοχικές πλευρές ίσες.
\item Έχει μια ορθή γωνία και διαγώνιους κάθετες.
\item Έχει μια ορθή γωνία και μια διαγώνιος διχοτομεί μια γωνία.
\item Έχει διαγώνιους ίσες και κάθετες.
\item Έχει διαγώνιους ίσες και δύο διαδοχικές πλευρές ίσες.
\item Έχει διαγώνιους ίσες και μια απ' αυτές διχοτομεί μια γωνία.
\end{alist}
\end{mytblr}
%\newpage
%\begin{mytblr}{column{2}={3.1cm},rows={m}}
% \SetCell[c=4]{c}{\boldmath$ AB\varGamma\varDelta $} Παραλληλόγραμμο και\\ 
% && \SetCell[c=2]{c} \kerkissans{Ιδιότητες Ορθογωνίου}\\ 
% &&  Μια ορθή γωνία & Διαγώνιοι ίσες \rule[-2ex]{0pt}{5.5ex}\\ 
%\SetCell[r=3]{c}\rotatebox{90}{\kerkissans{Ιδιότητες ρόμβου}} & Διαδοχικές πλευρές ίσες & 1ο Κριτήριο & 4ο Κριτήριο\\ 
% & Διαγώνιοι κάθετες & 2ο Κριτήριο & 5ο Κριτήριο \\ 
% & Διαγώνιος διχοτομεί μια γωνία & 3ο Κριτήριο & 6ο Κριτήριο \\  
%\end{mytblr} 
%\noindent
%Τραπέζιο ονομάζεται το τετράπλευρο \\το οποίο έχει δύο απέναντι πλευρές \\του παράλληλες.
%\begin{center}
%\begin{tabular}{p{3.9cm}cp{4cm}}
%\begin{tikzpicture}
%\tkzDefPoint(0,-1.5){D}
%\tkzDefPoint(0.5,.5){A}
%\tkzDefPoint(2.5,.5){B}
%\tkzDefPoint(3.5,-1.5){C}
%\tkzDefPoint(.25,-.5){M}
%\tkzDefPoint(3,-.5){N}
%\tkzDefPoint(0.9,0.5){E}
%\tkzDefPoint(0.9,-1.5){Z}
%\tkzMarkRightAngle(C,Z,E)
%\draw (0.9,0.5) -- (0.9,-1.5);
%\draw[pl] (0,-1.5) -- (0.5,0.5) -- (2.5,0.5) -- (3.5,-1.5) -- cycle;
%\draw[plm,white](M)--(N);
%\tkzLabelPoint[above left](A){$A$}
%\tkzLabelPoint[above right](B){$B$}
%\tkzLabelPoint[below right](C){$\varGamma$}
%\tkzLabelPoint[below left](D){$\varDelta$}
%\tkzLabelPoint[left](M){$M$}
%\tkzLabelPoint[right](N){$N$}
%\tkzLabelPoint[above](E){$E$}
%\tkzLabelPoint[below](Z){$Z$}
%\tkzDrawPoints(A,B,C,D,M,N)
%\node at (1.5,0.7) {\footnotesize$\beta$};
%\node at (1.7,-1.8) {\footnotesize$B$};
%\node at (.7,-.2) {\footnotesize$ \upsilon $};
%\node at (1.75,-.35) {\footnotesize$ \delta $};
%\end{tikzpicture} & \hspace{.5cm} & \begin{tikzpicture}
%\tkzDefPoint(0,-1.5){D}
%\tkzDefPoint(0.75,.5){A}
%\tkzDefPoint(2.75,.5){B}
%\tkzDefPoint(3.5,-1.5){C}
%\tkzDefPoint(.25,-.5){M}
%\tkzDefPoint(3,-.5){N}
%\tkzDefPoint(0.9,0.5){E}
%\tkzDefPoint(0.9,-1.5){Z}
%\tkzDrawSegment[pl](A,B)
%\tkzDrawSegment[pl](C,D)
%\tkzDrawSegment[plm,white](A,D)
%\tkzDrawSegment[plm,white](B,C)
%\tkzMarkSegment[mark=|](A,D)
%\tkzMarkSegment[mark=|](B,C)
%\tkzLabelPoint[above](A){$A$}
%\tkzLabelPoint[above](B){$B$}
%\tkzLabelPoint[below](C){$\varGamma$}
%\tkzLabelPoint[below](D){$\varDelta$}
%\tkzDrawPoints(A,B,C,D)
%\node at (1.7,0.7) {\footnotesize$\beta$};
%\node at (1.7,-1.8) {\footnotesize$B$};
%\end{tikzpicture} \\ 
%\end{tabular} 
%\end{center}
%\begin{itemize}[itemsep=0mm,label=\faEdit]
%\item $AB$ και $\varGamma\varDelta$ : \textbf{βάσεις}.
%\item $MN$ : \textbf{διάμεσος} του τραπεζίου.
%\item $MN\parallel ΑΒ\parallel\varGamma\varDelta$ και $MN=\dfrac{AB+\varGamma\varDelta}{2}$
%\item $EZ$ : \textbf{ύψος} του τραπεζίου.
%\item Το τραπέζιο το οποίο έχει τις μη \\παράλληλες πλευρές του ίσες \\ονομάζεται \textbf{ισοσκελές τραπέζιο}.
%\end{itemize}
%{\footnotesize \faPlay} {Ιδιότητες ισοσκελούς τραπεζίου}
%\begin{itemize}[itemsep=0mm,label=\faEdit]
%\item $ \hat{A}=\hat{B} $ και $ \hat{\varGamma}=\hat{\varDelta} $.
%\item $ A\varGamma=B\varDelta $.
%\end{itemize}
%\vspace{-7mm}
%\begin{center}
%\begin{tikzpicture}
%\tkzDefPoint(0,-1.5){D}
%\tkzDefPoint(0.75,.5){A}
%\tkzDefPoint(2.75,.5){B}
%\tkzDefPoint(3.5,-1.5){C}
%\tkzDefPoint(.25,-.5){M}
%\tkzDefPoint(3,-.5){N}
%\tkzDefPoint(0.9,0.5){E}
%\tkzDefPoint(0.9,-1.5){Z}
%\tkzMarkAngle[size=.4,mark=|,fill=white](D,A,B)
%\tkzMarkAngle[size=.4,mark=|,fill=white](A,B,C)
%\tkzMarkAngle[size=.4,mark=||,fill=white](C,D,A)
%\tkzMarkAngle[size=.4,mark=||,fill=white](B,C,D)
%\tkzDrawSegment[pl](A,B)
%\tkzDrawSegment[pl](C,D)
%\tkzDrawSegment[plm](A,D)
%\tkzDrawSegment[plm](B,C)
%\draw[pl,white] (A)--(C);
%\draw[pl,white] (B)--(D);
%\tkzMarkSegments[mark=|](A,D B,C)
%\tkzMarkSegments[mark=|](A,C B,D)
%\tkzLabelPoint[above](A){$A$}
%\tkzLabelPoint[above](B){$B$}
%\tkzLabelPoint[below](C){$\varGamma$}
%\tkzLabelPoint[below](D){$\varDelta$}
%\tkzDrawPoints(A,B,C,D)
%\node at (1.7,0.7) {\footnotesize$\beta$};
%\node at (1.7,-1.8) {\footnotesize$B$};
%\end{tikzpicture}
%\end{center}
%{\footnotesize \faPlay} {Κριτήρια για ισοσκελές τραπέζιο}\\
%$ AB\varGamma\varDelta $ με $ AB\parallel\varGamma\varDelta $ ισοσκελές αν
%\begin{itemize}[itemsep=0mm,label=\faEdit]
%\item Οι προσκείμενες γωνίες μιας βάσης\\είναι ίσες.
%\item Οι διαγώνιοι είναι ίσες.
%\end{itemize}
\end{document}