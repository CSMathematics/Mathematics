\documentclass[11pt]{article}
\usepackage[english,greek]{babel}
\usepackage[utf8]{inputenc}
\usepackage{nimbusserif}
\usepackage[T1]{fontenc}
\usepackage[left=2.00cm, right=2.00cm, top=3.00cm, bottom=2.00cm,a4paper]{geometry}
\usepackage{amsmath}
\let\myBbbk\Bbbk
\let\Bbbk\relax
\usepackage[amsbb,subscriptcorrection,zswash,mtpcal,mtphrb,mtpfrak]{mtpro2}
\usepackage{graphicx,multicol,multirow,enumitem,tabularx,mathimatika,gensymb,venndiagram,hhline,longtable,tkz-euclide,fontawesome5,eurosym,tcolorbox,tabularray}
\tcbuselibrary{skins,theorems,breakable}
\newlist{rlist}{enumerate}{3}
\setlist[rlist]{itemsep=0mm,label=\roman*.}
\newlist{alist}{enumerate}{3}
\setlist[alist]{itemsep=0mm,label=\alph*.}
\newlist{balist}{enumerate}{3}
\setlist[balist]{itemsep=0mm,label=\bf\alph*.}
\newlist{Alist}{enumerate}{3}
\setlist[Alist]{itemsep=0mm,label=\Alph*.}
\newlist{bAlist}{enumerate}{3}
\setlist[bAlist]{itemsep=0mm,label=\bf\Alph*.}
\renewcommand{\textstigma}{\textsigma\texttau}
\newlist{thema}{enumerate}{3}
\setlist[thema]{label=\bf\large{ΘΕΜΑ \textcolor{black}{\Alph*}},itemsep=0mm,leftmargin=0cm,itemindent=18mm}
\newlist{erwthma}{enumerate}{3}
\setlist[erwthma]{label=\bf{\large{\textcolor{black}{\Alph{themai}.\arabic*}}},itemsep=0mm,leftmargin=0.8cm}

\newcommand{\lysh}{\textcolor{black}{\textbf{\faCheck\ \ ΛΥΣΗ}}}
\renewcommand{\textstigma}{\textsigma\texttau}
%----------- ΟΡΙΣΜΟΣ------------------
\newcounter{orismos}[section]
\renewcommand{\theorismos}{\thesection.\arabic{orismos}}   
\newcommand{\Orismos}{\refstepcounter{orismos}{\textbf{\textcolor{black}{\kerkissans{Ορισμός\hspace{2mm}\theorismos}}\;:\;}{}}}

\newenvironment{orismos}[1]
{\begin{tcolorbox}[title=\Orismos {\textcolor{black}{\kerkissans{#1}}},breakable,bottomtitle=-1.5mm,
enhanced standard,titlerule=-.2pt,toprule=0pt, rightrule=0pt, bottomrule=0pt,
colback=white,left=2mm,top=1mm,bottom=0mm,
boxrule=0pt,
colframe=white,borderline west={1.5mm}{0pt}{black},leftrule=2mm,sharp corners,coltitle=black]}
{\end{tcolorbox}}

\newcommand{\kerkissans}[1]{{\fontfamily{maksf}\selectfont \textbf{#1}}}
\renewcommand{\textdexiakeraia}{}

\usepackage[
backend=biber,
style=alphabetic,
sorting=ynt
]{biblatex}

\DeclareTblrTemplate{caption}{nocaptemplate}{}
\DeclareTblrTemplate{capcont}{nocaptemplate}{}
\DeclareTblrTemplate{contfoot}{nocaptemplate}{}
\NewTblrTheme{mytabletheme}{
  \SetTblrTemplate{caption}{nocaptemplate}{}
  \SetTblrTemplate{capcont}{nocaptemplate}{}
  \SetTblrTemplate{contfoot}{nocaptemplate}{}
}

\NewTblrEnviron{mytblr}
\SetTblrStyle{firsthead}{font=\bfseries}
\SetTblrStyle{firstfoot}{fg=red2}
\SetTblrOuter[mytblr]{theme=mytabletheme}
\SetTblrInner[mytblr]{
rowspec={t{7mm}},columns = {c},
  width = 0.85\linewidth,
  row{odd} = {bg=green!20!white!80,fg=black,ht=8mm},
 row{even} = {bg=green!70!gray!50,fg=black,ht=8mm,m},
hlines={white},vlines={white},
row{1} = {bg=green!70!black, fg=white, font=\large\bfseries\fontfamily{maksf}},rowhead = 1,
  hline{2} = {.7mm}, % midrule  
}

\begin{document}
%\begin{mytblr}{}
%& \textbf{Αριθμητική Πρόοδος} & \textbf{Γεωμετρική Πρόοδος} \\ 
% \textbf{Όροι} & $ a_{\nu+1}=a_\nu+\omega $ & $ a_{\nu+1}=\lambda\cdot a_\nu $ \\ 
% \textbf{Διαφορά / Λόγος} & $ \omega=a_{\nu+1}-a_\nu $ & $ \lambda=\dfrac{a_{\nu+1}}{a_\nu} $ \\ 
%\textbf{Μέσος} & $ \beta=\dfrac{a+\gamma}{2}\Leftrightarrow 2\beta=a+\gamma $ & $ \beta^2=a\cdot\gamma $ \\ 
%\textbf{Γενικός Όρος} & $ a_\nu=a_1+(\nu-1)\omega $ & $ a_\nu=a_1\cdot\lambda^{\nu-1} $ \\
%\textbf{Άθροισμα} & {$S_\nu=\dfrac{\nu}{2}(a_1+a_\nu)$ ή \\[2mm]$S_\nu=\dfrac{\nu}{2}\left[2a_1+(\nu-1)\omega\right] $} & $S_\nu=a_1\dfrac{\lambda^\nu-1}{\lambda-1}$
%\end{mytblr}
%\begin{minipage}{7cm}
%\begin{itemize}[label={\footnotesize \faPlay}]
%\item Αριθμητική πρόοδος.
%Κάθε όρος της προκύπτει από τον προηγούμενο, προσθέτοντας κάθε φορά τον ίδιο σταθερό αριθμό.
%\[ a_{\nu+1}=a_\nu+\omega \]
%\item Ο αριθμός $ \omega=a_{\nu+1}-a_\nu $ ονομάζεται \textbf{διαφορά} της αριθμητικής προόδου και είναι σταθερός.\\\\
%\item Τρεις πραγματικοί αριθμοί $ a,\beta,\gamma $ είναι διαδοχικοί όροι αριθμητικής προόδου αν και μόνο αν ισχύει \[ 2\beta=a+\gamma\;\;\textrm{ ή }\;\;\beta=\frac{a+\gamma}{2} \]
%\item Ο $\beta$ λέγεται \textbf{αριθμητικός μέσος} των $a,\gamma$.
%\item Γενικός όρος της $ a_\nu $
%\[ a_\nu=a_1+(\nu-1)\omega \]
%
%\item Άθροισμα των $ \nu $ πρώτων όρων
%\[ S_\nu=\frac{\nu}{2}(a_1+a_\nu)\;\;,\;\;S_\nu=\frac{\nu}{2}\left[2a_1+(\nu-1)\omega\right]  \]
%\end{itemize}
%\end{minipage}
\begin{minipage}{7cm}
\begin{itemize}[label={\footnotesize \faPlay}]
\item Γεωμετρική πρόοδος.
Κάθε όρος της προκύπτει από τον προηγούμενο, πολλαπλασιζοντας κάθε φορά τον ίδιο σταθερό αριθμό.
\[ a_{\nu+1}=\lambda a_\nu \]
\item Ο αριθμός $ \lambda=\dfrac{a_{\nu+1}}{a_\nu} $ ονομάζεται \textbf{λόγος} της γεωμετρικής προόδου και είναι σταθερός.\\\\
\item Τρεις πραγματικοί αριθμοί $ a,\beta,\gamma $ είναι διαδοχικοί όροι αριθμητικής προόδου αν και μόνο αν ισχύει \[ 2\beta=a+\gamma\;\;\textrm{ ή }\;\;\beta=\frac{a+\gamma}{2} \]
\item Ο $\beta$ λέγεται \textbf{αριθμητικός μέσος} των $a,\gamma$.
\item Γενικός όρος της $ a_\nu $
\[ a_\nu=a_1+(\nu-1)\omega \]

\item Άθροισμα των $ \nu $ πρώτων όρων
\[ S_\nu=\frac{\nu}{2}(a_1+a_\nu)\;\;,\;\;S_\nu=\frac{\nu}{2}\left[2a_1+(\nu-1)\omega\right]  \]
\end{itemize}
\end{minipage}
\end{document}