\documentclass[twoside,nofonts,internet,math,spyros]{frontisthrio-diag}
\usepackage[amsbb,subscriptcorrection,zswash,mtpcal,mtphrb,mtpfrak]{mtpro2}
\usepackage[no-math,cm-default]{fontspec}
\usepackage{amsmath}
\usepackage{xunicode}
\usepackage{xgreek}
\let\hbar\relax
\defaultfontfeatures{Mapping=tex-text,Scale=MatchLowercase}
\setmainfont[Mapping=tex-text,Numbers=Lining,Scale=1.0,BoldFont={Minion Pro Bold}]{Minion Pro}
\newfontfamily\scfont{GFS Artemisia}
\font\icon = "Webdings"
\usepackage{fontawesome}
\newfontfamily{\FA}{fontawesome.otf}
\xroma{red!70!black}
%------TIKZ - ΣΧΗΜΑΤΑ - ΓΡΑΦΙΚΕΣ ΠΑΡΑΣΤΑΣΕΙΣ ----
\usepackage{tikz,pgfplots}
\usepackage{tkz-euclide}
\usetkzobj{all}
\usepackage[framemethod=TikZ]{mdframed}
\usetikzlibrary{decorations.pathreplacing}
\tkzSetUpPoint[size=7,fill=white]
%-----------------------
\usepackage{calc,tcolorbox}
\tcbuselibrary{skins,theorems,breakable}
\usepackage{hhline}
\usepackage[explicit]{titlesec}
\usepackage{graphicx}
\usepackage{multicol}
\usepackage{multirow}
\usepackage{tabularx}
\usetikzlibrary{backgrounds}
\usepackage{sectsty}
\sectionfont{\centering}
\usepackage{enumitem}
\usepackage{adjustbox}
\usepackage{mathimatika,gensymb,eurosym,wrap-rl}
\usepackage{systeme,regexpatch}
%-------- ΜΑΘΗΜΑΤΙΚΑ ΕΡΓΑΛΕΙΑ ---------
\usepackage{mathtools}
%----------------------
%-------- ΠΙΝΑΚΕΣ ---------
\usepackage{booktabs}
%----------------------
%----- ΥΠΟΛΟΓΙΣΤΗΣ ----------
\usepackage{calculator}
%----------------------------
%------ ΔΙΑΓΩΝΙΟ ΣΕ ΠΙΝΑΚΑ -------
\usepackage{array}
\newcommand\diag[5]{%
\multicolumn{1}{|m{#2}|}{\hskip-\tabcolsep
$\vcenter{\begin{tikzpicture}[baseline=0,anchor=south west,outer sep=0]
\path[use as bounding box] (0,0) rectangle (#2+2\tabcolsep,\baselineskip);
\node[minimum width={#2+2\tabcolsep-\pgflinewidth},
minimum  height=\baselineskip+#3-\pgflinewidth] (box) {};
\draw[line cap=round] (box.north west) -- (box.south east);
\node[anchor=south west,align=left,inner sep=#1] at (box.south west) {#4};
\node[anchor=north east,align=right,inner sep=#1] at (box.north east) {#5};
\end{tikzpicture}}\rule{0pt}{.71\baselineskip+#3-\pgflinewidth}$\hskip-\tabcolsep}}
%---------------------------------
%---- ΟΡΙΖΟΝΤΙΟ - ΚΑΤΑΚΟΡΥΦΟ - ΠΛΑΓΙΟ ΑΓΚΙΣΤΡΟ ------
\newcommand{\orag}[3]{\node at (#1)
{$ \overcbrace{\rule{#2mm}{0mm}}^{{\scriptsize #3}} $};}
\newcommand{\kag}[3]{\node at (#1)
{$ \undercbrace{\rule{#2mm}{0mm}}_{{\scriptsize #3}} $};}
\newcommand{\Pag}[4]{\node[rotate=#1] at (#2)
{$ \overcbrace{\rule{#3mm}{0mm}}^{{\rotatebox{-#1}{\scriptsize$#4$}}}$};}
%-----------------------------------------
%------------------------------------------
\newcommand{\tss}[1]{\textsuperscript{#1}}
\newcommand{\tssL}[1]{\MakeLowercase{\textsuperscript{#1}}}
%---------- ΛΙΣΤΕΣ ----------------------
\newlist{bhma}{enumerate}{3}
\setlist[bhma]{label=\bf\textit{\arabic*\textsuperscript{o}\;Βήμα :},leftmargin=0cm,itemindent=1.8cm,ref=\bf{\arabic*\textsuperscript{o}\;Βήμα}}
\newlist{rlist}{enumerate}{3}
\setlist[rlist]{itemsep=0mm,label=\roman*.}
\newlist{brlist}{enumerate}{3}
\setlist[brlist]{itemsep=0mm,label=\bf\roman*.}
\newlist{tropos}{enumerate}{3}
\setlist[tropos]{label=\bf\textit{\arabic*\textsuperscript{oς}\;Τρόπος :},leftmargin=0cm,itemindent=2.3cm,ref=\bf{\arabic*\textsuperscript{oς}\;Τρόπος}}
% Αν μπει το bhma μεσα σε tropo τότε
%\begin{bhma}[leftmargin=.7cm]
\tkzSetUpPoint[size=7,fill=white]
\tikzstyle{pl}=[line width=0.3mm]
\tikzstyle{plm}=[line width=0.4mm]
\usepackage{etoolbox}
\makeatletter
\renewrobustcmd{\anw@true}{\let\ifanw@\iffalse}
\renewrobustcmd{\anw@false}{\let\ifanw@\iffalse}\anw@false
\newrobustcmd{\noanw@true}{\let\ifnoanw@\iffalse}
\newrobustcmd{\noanw@false}{\let\ifnoanw@\iffalse}\noanw@false
\renewrobustcmd{\anw@print}{\ifanw@\ifnoanw@\else\numer@lsign\fi\fi}
\makeatother

\usepackage{path}
\pathal

\begin{document}
\titlos{Μαθηματικά Β΄ Γυμνασίου}{Διαγωνισμός Υποτροφίας}{2018 - 2019}
\vspace{-5mm}
\begin{thema}
\item\mbox{}\\\vspace{-5mm}
\begin{erwthma}
\item Να απαντήσετε στις ακόλουθες ερωτήσεις.
\begin{alist}
\item Τι ονομάζεται εξίσωση;
\item Να διατυπώσετε το Πυθαγόρειο θεώρημα.
\item Ποια σχέση συνδέει μια εγγεγραμμένη γωνία με το αντίστοιχο τόξο της;
\item Τι ονομάζουμε ημίτονο μιας οξείας $ \omega $ γωνίας ενός ορθογωνίου τριγώνου;
\end{alist}\monades{4$ \times $0,5=2}
\item \swstolathos
\begin{alist}
\item 
\item 
\item 
\item 
\item 
\end{alist}\monades{5$ \times $0,2=1}
\item Να επιλέξετε τη σωστή απάντηση σε καθεμία από τις παρακάτω προτάσεις.
\begin{alist}
\item 
\begin{multicols}{4}
\begin{rlist}
\item 
\item 
\item 
\item 
\end{rlist}
\end{multicols}
\item 
\begin{multicols}{3}
\begin{rlist}
\item 
\item 
\item 
\end{rlist}
\end{multicols}
\item 
\begin{multicols}{4}
\begin{rlist}
\item 
\item 
\item 
\item 
\end{rlist}
\end{multicols}
\item 
\begin{multicols}{4}
\begin{rlist}
\item 
\item 
\item 
\item 
\end{rlist}
\end{multicols}
\item 
\begin{multicols}{4}
\begin{rlist}
\item 
\item 
\item 
\item 
\end{rlist}
\end{multicols}
\end{alist}\monades{5$ \times $0,2=1}
\item Να συμπληρώσετε τα κενά στις παρακάτω προτάσεις.
\begin{alist}
\item 
\end{alist}
\end{erwthma}\monades{5$ \times $0,2=1}
\item \mbox{}\\\vspace{-5mm}
\begin{erwthma}
\item Να λυθεί η παρακάτω εξίσωση
\[ \frac{x-3}{2}-\frac{4-3x}{5}-1=3(x-2)+\frac{4}{5} \]
\monades{2}
\item 
Να λυθούν οι ακόλουθες εξισώσεις.
\begin{multicols}{2}
\begin{alist}
\item $ 3(1+2x)-4x-7=2x-4 $
\item $ \frac{x-2}{5}-x=-2+\frac{3-4x}{5} $
\end{alist}
\end{multicols}
\monades{2}
\item Να λυθεί η παρακάτω εξίσωση
\[ \lambda x+3-2(1-x)=4-2\lambda +x \]
\begin{multicols}{2}
\begin{alist}
\item όταν $ \lambda=2 $
\item όταν $ \lambda=-1 $
\end{alist}
\end{multicols}\monades{2$ \times $0,5=1}
\end{erwthma}
\item \mbox{}\\\vspace{-5mm}
\begin{erwthma}
\item 
Να υπολογίσετε τις παρακάτω παραστάσεις.
\begin{multicols}{2}
\begin{alist}
\item $ \sqrt{\dfrac{16}{25}}+\sqrt{\dfrac{9}{100}} $
\item $ \sqrt{22+\sqrt{4+\sqrt{25}}} $
\end{alist}
\end{multicols}
\monades{0,5+0,5=1}
\item 
\begin{alist}
\item Να υπολογίσετε την πλευρά $ AB $ του ορθογωνίου τριγώνου $ AB\varGamma $.
\begin{center}
\begin{tikzpicture}
\tkzDefPoint(0,0){A}
\tkzDefPoint(4,0){B}
\tkzDefPoint(0,2){C}
\tkzMarkRightAngle[size=.25](B,A,C)
\draw[pl] (A)--(B)--(C)--cycle;
\tkzLabelPoint[left](A){$ A $}
\tkzLabelPoint[right](B){$ B $}
\tkzLabelPoint[left](C){$ \varGamma $}
\tkzDrawPoints(A,B,C)
\node at (-.3,1){$ 8 $};
\node at (2.5,1){$ 17 $};
\end{tikzpicture}
\end{center}\monades{1}
\item Να αποδείξετε ότι το τρίγωνο $ AB\varGamma $ είναι ορθογώνιο.
\begin{center}
\begin{tikzpicture}
\tkzDefPoint(0,0){A}
\tkzDefPoint(4,0){B}
\tkzDefPoint(0,2){C}
\tkzMarkRightAngle[size=.25](B,A,C)
\draw[pl] (A)--(B)--(C)--cycle;
\tkzLabelPoint[left](A){$ A $}
\tkzLabelPoint[right](B){$ B $}
\tkzLabelPoint[left](C){$ \varGamma $}
\tkzDrawPoints(A,B,C)
\node at (-.3,1){$ 5 $};
\node at (2.5,1){$ 13 $};
\node at (2,-.3){$ 12 $};
\end{tikzpicture}
\end{center}\monades{1}
\end{alist}
\item Δίνεται το τρίγωνο $ AB\varGamma $ του παρακάτω σχήματος και $ A\varDelta $ το ύψος του. Αν γνωρίζουμε ότι $ A\varGamma=12 $, $ \hat{\varGamma}=30\degree $ και $ \hat{B}=45\degree $ τότε να υπολογίσετε τις πλευρές
\begin{center}
\begin{tikzpicture}
\tkzDefPoint(0,0){D}
\tkzDefPoint(4,0){C}
\tkzDefPoint(0,2){A}
\tkzDefPoint(-2,0){B}
\tkzMarkAngle[size=.5](A,C,B)
\tkzLabelAngle[pos=1](A,C,B){$ 30\degree $}
\tkzMarkAngle[size=.5](C,B,A)
\tkzLabelAngle[pos=1](C,B,A){$ 45\degree $}
\tkzMarkRightAngle[size=.25](C,D,A)
\draw[pl] (A)--(B)--(C)--cycle;
\draw[pl] (A)--(D);
\tkzLabelPoint[above](A){$ A $}
\tkzLabelPoint[left](B){$ B $}
\tkzLabelPoint[right](C){$ \varGamma $}
\tkzLabelPoint[below](D){$ \varDelta $}
\tkzDrawPoints(A,B,C)
\node at (2.5,1){$ 12 $};
\end{tikzpicture}
\end{center}
\begin{multicols}{3}
\begin{alist}
\item $ A\varDelta $
\item $ B\varDelta $
\item $ B\varGamma $
\end{alist}
\end{multicols}
\monades{0,7+0,6+0,7=2}
\end{erwthma}
\item\mbox{}\\\vspace{-1.2cm}
\begin{erwthma}
\wrapr{-5mm}{7}{4cm}{-5mm}{
\begin{tikzpicture}
\tkzDefPoint(120:2){A}
\tkzDefPoint(220:2){B}
\tkzDefPoint(0:2){C}
\draw[pl] (0,0) circle (2);
\draw[pl](A)--(B)--(C)--cycle;
\node at (60:2.3){$120\degree$};
\node at (290:2.3){$140\degree$};
\tkzLabelPoint[above,xshift=-1mm](A){$A$}
\tkzLabelPoint[left,yshift=-1mm](B){$B$}
\tkzLabelPoint[right](C){$\varGamma$}
\tkzDrawPoints(A,B,C)
\end{tikzpicture}
}{\item Έστω τρίγωνο $ AB\varGamma $ εγγεγραμμένο σε κύκλο όπως φαίνεται στο διπλανό σχήμα. Αν γνωρίζουμε ότι $ \widearc{A\varGamma}=140\degree $ και $ \widearc{B\varGamma}=120\degree $ τότε να υπολογίσετε τις γωνίες του τριγώνου.}
\item \begin{tikzpicture}
\tkzDefPoint(120:2){A}
\tkzDefPoint(180:2){B}
\tkzDefPoint(0:2){C}
\draw[pl] (0,0) circle (2);
\draw[pl](A)--(B)--(C)--cycle;
\tkzLabelPoint[above,xshift=-1mm](A){$A$}
\tkzLabelPoint[left](B){$B$}
\tkzLabelPoint[right](C){$\varGamma$}
\tkzDrawPoints(A,B,C)
\end{tikzpicture}
\end{erwthma}
\end{thema}
\vfill
\begin{flushright}
Διάρκεια $ 2 $ ώρες και $ 30 $ λεπτά.\\
\textit{Καλή επιτυχία!}
\end{flushright}
\end{document}
