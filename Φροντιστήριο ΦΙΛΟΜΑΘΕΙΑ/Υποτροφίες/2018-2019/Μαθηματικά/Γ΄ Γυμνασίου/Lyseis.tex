\documentclass[twoside,nofonts,internet,math,spyros]{frontisthrio-diag}
\usepackage[amsbb,subscriptcorrection,zswash,mtpcal,mtphrb,mtpfrak]{mtpro2}
\usepackage[no-math,cm-default]{fontspec}
\usepackage{amsmath}
\usepackage{xunicode}
\usepackage{xgreek}
\let\hbar\relax
\defaultfontfeatures{Mapping=tex-text,Scale=MatchLowercase}
\setmainfont[Mapping=tex-text,Numbers=Lining,Scale=1.0,BoldFont={Minion Pro Bold}]{Minion Pro}
\newfontfamily\scfont{GFS Artemisia}
\font\icon = "Webdings"
\usepackage{fontawesome}
\newfontfamily{\FA}{fontawesome.otf}
\xroma{red!70!black}
%------TIKZ - ΣΧΗΜΑΤΑ - ΓΡΑΦΙΚΕΣ ΠΑΡΑΣΤΑΣΕΙΣ ----
\usepackage{tikz,pgfplots}
\usepackage{tkz-euclide}
\usetkzobj{all}
\usepackage[framemethod=TikZ]{mdframed}
\usetikzlibrary{decorations.pathreplacing}
\tkzSetUpPoint[size=7,fill=white]
%-----------------------
\usepackage{calc,tcolorbox}
\tcbuselibrary{skins,theorems,breakable}
\usepackage{hhline}
\usepackage[explicit]{titlesec}
\usepackage{graphicx}
\usepackage{multicol}
\usepackage{multirow}
\usepackage{tabularx}
\usetikzlibrary{backgrounds}
\usepackage{sectsty}
\sectionfont{\centering}
\usepackage{enumitem}
\usepackage{adjustbox}
\usepackage{mathimatika,gensymb,eurosym,wrap-rl}
\usepackage{systeme,regexpatch}
%-------- ΜΑΘΗΜΑΤΙΚΑ ΕΡΓΑΛΕΙΑ ---------
\usepackage{mathtools}
%----------------------
%-------- ΠΙΝΑΚΕΣ ---------
\usepackage{booktabs}
%----------------------
%----- ΥΠΟΛΟΓΙΣΤΗΣ ----------
\usepackage{calculator}
%----------------------------
%------ ΔΙΑΓΩΝΙΟ ΣΕ ΠΙΝΑΚΑ -------
\usepackage{array}
\newcommand\diag[5]{%
\multicolumn{1}{|m{#2}|}{\hskip-\tabcolsep
$\vcenter{\begin{tikzpicture}[baseline=0,anchor=south west,outer sep=0]
\path[use as bounding box] (0,0) rectangle (#2+2\tabcolsep,\baselineskip);
\node[minimum width={#2+2\tabcolsep-\pgflinewidth},
minimum  height=\baselineskip+#3-\pgflinewidth] (box) {};
\draw[line cap=round] (box.north west) -- (box.south east);
\node[anchor=south west,align=left,inner sep=#1] at (box.south west) {#4};
\node[anchor=north east,align=right,inner sep=#1] at (box.north east) {#5};
\end{tikzpicture}}\rule{0pt}{.71\baselineskip+#3-\pgflinewidth}$\hskip-\tabcolsep}}
%---------------------------------
%---- ΟΡΙΖΟΝΤΙΟ - ΚΑΤΑΚΟΡΥΦΟ - ΠΛΑΓΙΟ ΑΓΚΙΣΤΡΟ ------
\newcommand{\orag}[3]{\node at (#1)
{$ \overcbrace{\rule{#2mm}{0mm}}^{{\scriptsize #3}} $};}
\newcommand{\kag}[3]{\node at (#1)
{$ \undercbrace{\rule{#2mm}{0mm}}_{{\scriptsize #3}} $};}
\newcommand{\Pag}[4]{\node[rotate=#1] at (#2)
{$ \overcbrace{\rule{#3mm}{0mm}}^{{\rotatebox{-#1}{\scriptsize$#4$}}}$};}
%-----------------------------------------
%------------------------------------------
\newcommand{\tss}[1]{\textsuperscript{#1}}
\newcommand{\tssL}[1]{\MakeLowercase{\textsuperscript{#1}}}
%---------- ΛΙΣΤΕΣ ----------------------
\newlist{bhma}{enumerate}{3}
\setlist[bhma]{label=\bf\textit{\arabic*\textsuperscript{o}\;Βήμα :},leftmargin=0cm,itemindent=1.8cm,ref=\bf{\arabic*\textsuperscript{o}\;Βήμα}}
\newlist{rlist}{enumerate}{3}
\setlist[rlist]{itemsep=0mm,label=\roman*.}
\newlist{brlist}{enumerate}{3}
\setlist[brlist]{itemsep=0mm,label=\bf\roman*.}
\newlist{tropos}{enumerate}{3}
\setlist[tropos]{label=\bf\textit{\arabic*\textsuperscript{oς}\;Τρόπος :},leftmargin=0cm,itemindent=2.3cm,ref=\bf{\arabic*\textsuperscript{oς}\;Τρόπος}}
% Αν μπει το bhma μεσα σε tropo τότε
%\begin{bhma}[leftmargin=.7cm]
\tkzSetUpPoint[size=7,fill=white]
\tikzstyle{pl}=[line width=0.3mm]
\tikzstyle{plm}=[line width=0.4mm]
\usepackage{etoolbox}
\makeatletter
\renewrobustcmd{\anw@true}{\let\ifanw@\iffalse}
\renewrobustcmd{\anw@false}{\let\ifanw@\iffalse}\anw@false
\newrobustcmd{\noanw@true}{\let\ifnoanw@\iffalse}
\newrobustcmd{\noanw@false}{\let\ifnoanw@\iffalse}\noanw@false
\renewrobustcmd{\anw@print}{\ifanw@\ifnoanw@\else\numer@lsign\fi\fi}
\makeatother

\usepackage{path}
\pathal

\begin{document}
\begin{center}
{\Large \textbf{Λύσεις διαγωνίσματος}}
\end{center}
\begin{thema}
\item\mbox{}\\\vspace{-5mm}
\begin{erwthma}
\item 
\begin{alist}
\item Ταυτότητα ονομάζεται μια ισότητα που περιέχει μεταβλητές και αληθεύει για κάθε τιμή των μεταβλητών.
\item $ \varDelta=\beta^2-4a\gamma $\ \ ,\ \ $ x_{1,2}=\dfrac{-\beta\pm\sqrt{\varDelta}}{2a} $
\item Η διάμεσος, η διχοτόμος και το ύψος.
\item Δύο τρίγωνα είναι ίσα αν έχουν μια πλευρά ίση και τις προσκείμενες στην πλευρά γωνίες ίσες, μια προς μια.
\end{alist}
\item 
\begin{alist}
\item Σωστό
\item Λάθος. Πρέπει τα τρίγωνα να είναι ίσα για να συμβεί αυτό.
\item Σωστό
\item Λάθος. Μπορεί να έχει μια λύση ή να είναι αδύνατη.
\item Λάθος. Μόνο η διάμεσος που καταλήγει στη βάση είναι και διχοτόμος και ύψος.
\end{alist}
\item 
\begin{multicols}{2}
\begin{alist}
\item ii. $ 4x^2-12x+9 $
\item i. έχει δύο λύσεις
\item iii. $ (2x+5)(2x-5) $
\item iii. τις περιεχόμενες γωνίες ίσες
\item i. $ \ef{x}=\frac{8}{15} $
\end{alist}
\end{multicols}
\item 
\begin{alist}
\item γινόμενο
\item δύο γωνίες
\item αντικατάστασης\ \ ,\ \ αντίθετων συντελεστών
\item ισαπέχει
\item $ \varDelta<0 $
\end{alist}
\end{erwthma}
\item \mbox{}\\\vspace{-5mm}
\begin{erwthma}
\item Κάνοντας τις πράξεις στο 1\tss{ο} μέλος θα καταλήξουμε στο 2\tss{ο}.
\begin{align*}
&\left( x+y\right)^2-\left( x-y\right)^2=\\
&=x^2+2xy+y^2-\left(x^2-2xy+y^2 \right)=\\
&=x^2+2xy+y^2-x^2+2xy-y^2=4xy
\end{align*}
\item Η παράσταση $ A $ έχει την ίδια μορφή με το πρώτο μέλος της ταυτότητας που αποδείξαμε, με $ x=20 $ και $ y=\frac{1}{80} $ άρα δίχως πράξεις γράφουμε το δεύτερο μέλος της βάζοντας όπου $ x $ και $ y $ τους αριθμούς αυτούς.
\[ A=\left( 20+\frac{1}{80}\right)^2-\left( 20-\frac{1}{80}\right)^2=4\cdot 20\cdot \frac{1}{80} \]
\item 
\begin{alist}
\item Βγάζοντας κοινό παράγοντα από όλους τους όρους της παράστασης έχουμε
\[ A=12x^3y^4-16x^4y^2z+18x^3y^3z^2=2x^3y^2\left( 6y^2-8xz+9yz^2\right)  \]
\item Η παράσταση θα παραγοντοποιηθεί με ομαδοποίηση άρα 
\begin{align*}
B&=x^3-4x^2+5x-20=\\
&=x^2(x-4)+5(x-4)=(x-4)\left( x^2+5\right) 
\end{align*}
\item Η παράσταση είναι διαφορά τετραγώνων με $ a=x-2 $ και $ \beta=3 $ άρα
\[ \varGamma=(x-2)^2-9=(x-2)^2-3^2=(x-2+3)(x-2-3)=(x+1)(x-5) \]
\item Η παράσταση αποτελεί ανάπτυγμα ταυτότητας άρα 
\[ \varDelta=4x^2-4x+1=(2x)^2-2\cdot 2x\cdot 1+1^2=(2x-1)^2 \]
\end{alist}

\end{erwthma}
\item \mbox{}\\\vspace{-5mm}
\begin{erwthma}
\item 

\begin{alist}
\item Για την εξίσωση $ x^2-8x+15=0 $ έχουμε $ a=1,\beta=-8 $ και $ \gamma=15 $ άρα
\[ \varDelta=\beta^2-4a\gamma=(-8)^2-4\cdot 1\cdot 15=64-60=4>0 \]
Άρα η εξίσωση έχει δύο λύσεις που είναι
\[ x_{1,2}=\frac{-\beta\pm\sqrt{\varDelta}}{2a}=\frac{-(-8)\pm\sqrt{4}}{2\cdot 1}=\frac{8\pm 2}{2} \]
άρα έχουμε 
\[ x_1=\frac{8+2}{2}=\frac{10}{2}=5\ \ \textrm{και}\ \ x_2=\frac{8-2}{2}=\frac{6}{2}=3\]
\item Για την εξίσωση $ (x-1)^2 +3x-5=7x-12 $ αρχικά αναπτύσσουμε την ταυτότητα, έπειτα μεταφέρουμε όλους τους όρους στο πρώτο μέλος και ύστερα από αναγωγή ομοίων όρων προκύπτει απλή εξίσωση 2\tss{ου} βαθμού. Είναι λοιπόν
\begin{gather*}
(x-1)^2 +3x-5=7x-12\Rightarrow\\
x^2-2x+1+3x-5=7x-12\Rightarrow\\
x^2-2x+1+3x-5-7x+12=0\Rightarrow\\
x^2-6x+8=0
\end{gather*}
Άρα σύμφωνα με τα γνωστά
\[ \varDelta=\beta^2-4a\gamma=(-6)^2-4\cdot 1\cdot 8=36-32=4>0 \]
Η εξίσωση λοιπόν έχει δύο λύσεις που είναι
\[ x_{1,2}=\frac{-\beta\pm\sqrt{\varDelta}}{2a}=\frac{-(-6)\pm\sqrt{4}}{2\cdot 1}=\frac{6\pm 2}{2} \]
άρα έχουμε 
\[ x_1=\frac{6+2}{2}=\frac{8}{2}=4\ \ \textrm{και}\ \ x_2=\frac{6-2}{2}=\frac{4}{2}=2\]
\end{alist}

\item Με τη μέθοδο των αντίθετων συντελεστών θα έχουμε
\begin{center}
\begin{tabular}{lrr}
$ \displaystyle\sysdelim\{|\systeme{3x+2y=7,x-4y=-7}\synt{1}{(-3)}\Rightarrow $ & $ \sysdelim\{.\systeme{3x+2y=7,-3x+12y=21} $  &  \\ 
\hhline{~-~}& $ 14y=28 $ & $ \Rightarrow y=2 $
\end{tabular}
\end{center}
Αντικαθιστώντας την τιμή του $ y $ στη δεύτερη εξίσωση έχουμε
\[ x-4\cdot 2=-7\Rightarrow x-8=-7\Rightarrow x=8-7\Rightarrow x=1 \]
Άρα η λύση του συστήματος θα είναι $ (x,y)=(1,2) $.
\item 
\begin{rlist}
\item Από τη σχέση $ \hm^2{x}+\syn^2{x}=1 $ παίρνουμε ότι
\begin{gather*}
\hm^2{x}+\syn^2{x}=1\Rightarrow\\
\left( \frac{5}{13}\right)^2+\syn^2{x}=1\Rightarrow\\
\frac{25}{169}+\syn^2{x}=1\Rightarrow\\
\syn^2{x}=1-\frac{25}{169}\Rightarrow\\
\syn^2{x}=\frac{144}{169}\Rightarrow\\
\syn{x}=\pm\frac{12}{13}
\end{gather*}
Γνωρίζουμε όμως ότι η γωνία $ x $ είναι αμβλεία και έτσι έχει αρνητικό συνημίτονο. Επομένως θα είναι $ \syn{x}=-\frac{12}{13} $.
\item Από τη σχέση $ \ef{x}=\frac{\hm{x}}{\syn{x}} $ έχουμε
\[ \ef{x}=\frac{\hm{x}}{\syn{x}}=\ef{x}=\frac[2mm]{\frac{5}{13}}{-\frac{12}{13}}=-\frac{5\cdot 13}{12\cdot 13}=-\frac{5}{12} \]
\end{rlist}
\item Αναπτύσσοντας τις ταυτότητες θα έχουμε
\begin{align*}
&(\hm{x}+\syn{x})^2+(\hm{x}-\syn{x})^2=\\
&=\hm^2{x}+2\hm{x}\syn{x}+\syn^2{x}+\hm^2{x}-2\hm{x}\syn{x}+\syn^2{x}=\\
&=2\hm^2{x}+2\syn^2{x}=\\
&=2\left( \hm^2{x}+\syn^2{x}\right) =2
\end{align*}
\end{erwthma}
\item\mbox{}\\
\begin{erwthma}
\item\mbox{}\\\vspace{-11mm}
\begin{alist}[leftmargin=5mm]
\item\mbox{}\\ \wrapr{-9.9mm}{7}{3.4cm}{-11mm}{\begin{tikzpicture}[scale=0.8]
\tkzDefPoint(0,0){B}
\tkzDefPoint(3,0){C}
\tkzDefPoint(1.5,4){A}
\tkzDefPoint(2.25,2){D}
\tkzDefPoint(.75,2){E}
\tkzInterLL(B,D)(C,E)\tkzGetPoint{M}
\draw[pl] (A)--(B)--(C)--cycle;
\draw[pl,\xrwma] (B)--(D);
\draw[pl,\xrwma] (C)--(E);
\draw[pl,\xrwma] (A)--(M);
\tkzLabelPoint[above](A){$ A $}
\tkzLabelPoint[left](B){$ B $}
\tkzLabelPoint[right](C){$ \varGamma $}
\tkzLabelPoint[right](D){$ \varDelta $}
\tkzLabelPoint[left](E){$ E $}
\tkzLabelPoint[below,yshift=-1mm](M){$ M $}
\tkzDrawPoints(A,B,C,D,E,M)
\end{tikzpicture}
}{Συγκρίνουμε τα τρίγωνα $ AB\varDelta $ και $ A\varGamma E $. Έχουμε ότι 
\begin{rlist}
\item $ AB=A\varGamma\label{1} $ διότι το τρίγωνο $ AB\varGamma $ είναι ισοσκελές. 
\item Επίσης
\[ AB=A\varGamma\Rightarrow\frac{AB}{2}=\frac{A\varGamma}{2}\Rightarrow A\varDelta=AE\label{2} \]
\item Τέλος η γωνία $ A $ είναι κοινή γωνία των δύο τριγώνων \label{3}.
\end{rlist}}\mbox{}\\\\
Από τις σχέσεις \ref{1},\ref{2} και \ref{3} παίρνουμε ότι τα τρίγωνα είναι ίσα σύμφωνα με το 1\tss{ο} κριτήριο ισότητας τριγώνων. Κατά συνέπεια θα ισχύει $ B\varDelta=\varGamma E $.
\item Συγκρίνουμε τα τρίγωνα $ ABM $ και $ A\varGamma M $. Έχουμε ότι 
\begin{rlist}
\item $ AB=A\varGamma\label{1} $ διότι το τρίγωνο $ AB\varGamma $ είναι ισοσκελές. 
\item $ AM $ κοινή πλευρά
\item και $ BM=\varGamma M $ όπως γνωρίζουμε από την υπόθεση.
\end{rlist}
Άρα από τις σχέσεις \ref{1},\ref{2} και \ref{3} παίρνουμε ότι τα τρίγωνα είναι ίσα σύμφωνα με το 3\tss{ο} κριτήριο ισότητας τριγώνων. Έτσι θα είναι
\[ B\hat{A}M=\varGamma\hat{A}M\Rightarrow AM\ \ \textrm{διχοτόμος της γωνίας }\hat{A} \]
\end{alist}
\wrapr{-5mm}{7}{4cm}{-5mm}{
\begin{tikzpicture}
\tkzDefPoint(0,0){A}
\tkzDefPoint(0,2){B}
\tkzDefPoint(3,0){C}
\tkzDrawAltitude[draw=\xrwma](B,C)(A)\tkzGetPoint{D}
\tkzMarkRightAngle[size=.25](C,A,B)
\tkzMarkRightAngle[size=.25](A,D,C)
\draw[pl] (A)--(B)--(C)--cycle;
\tkzLabelPoint[left](A){$ A $}
\tkzLabelPoint[left](B){$ B $}
\tkzLabelPoint[right](C){$ \varGamma $}
\tkzLabelPoint[above right](D){$ \varDelta $}
\tkzDrawPoints(A,B,C,D)
\end{tikzpicture}
}{
\item 
\begin{alist}
\item Τα τρίγωνα $ AB\varGamma $ και $ A\varGamma\varDelta $ είναι όμοια διοτι
\begin{rlist}
\item $ \hat{A}=\hat{\varDelta}=90\degree $ διότι είναι ορθογώνια και
\item η γωνία $ \hat{\varGamma} $ είναι κοινή γωνία
\end{rlist}
\item Αφού, όπως δείξαμε προηγουμένως, τα τρίγωνα είναι όμοια τότε οι πλευρές τους θα είναι ανάλογες δηλαδή
\[ \frac{AB}{A\varDelta}=\frac{A\varGamma}{\varGamma\varDelta}=\frac{B\varGamma}{A\varGamma} \]
Από τα δύο πρώτα κλάσματα έχουμε
\begin{gather*}
\frac{AB}{A\varDelta}=\frac{A\varGamma}{\varGamma\varDelta}\Rightarrow\frac{15}{12}=\frac{20}{\varGamma\varDelta}\Rightarrow\\
15\varGamma\varDelta=240\Rightarrow \varGamma\varDelta=\frac{240}{15}=16
\end{gather*}
Ομοίως από το 1\tss{ο} και το 3\tss{ο} κλάσμα παίρνουμε
\begin{gather*}
\frac{AB}{A\varDelta}=\frac{B\varGamma}{A\varGamma}\Rightarrow\frac{15}{12}=\frac{B\varGamma}{20}\Rightarrow\\
12 B\varGamma=300\Rightarrow B\varGamma=\frac{300}{12}=25
\end{gather*}
Άρα $ B\varDelta=B\varGamma-\varGamma\varDelta=25-16=9 $.
\end{alist}}
\end{erwthma}
\end{thema}
\end{document}
