\documentclass[twoside,nofonts,internet,math,spyros]{frontisthrio-diag}
\usepackage[amsbb,subscriptcorrection,zswash,mtpcal,mtphrb,mtpfrak]{mtpro2}
\usepackage[no-math,cm-default]{fontspec}
\usepackage{amsmath}
\usepackage{xunicode}
\usepackage{xgreek}
\let\hbar\relax
\defaultfontfeatures{Mapping=tex-text,Scale=MatchLowercase}
\setmainfont[Mapping=tex-text,Numbers=Lining,Scale=1.0,BoldFont={Minion Pro Bold}]{Minion Pro}
\newfontfamily\scfont{GFS Artemisia}
\font\icon = "Webdings"
\usepackage{fontawesome}
\newfontfamily{\FA}{fontawesome.otf}
\xroma{red!70!black}
%------TIKZ - ΣΧΗΜΑΤΑ - ΓΡΑΦΙΚΕΣ ΠΑΡΑΣΤΑΣΕΙΣ ----
\usepackage{tikz,pgfplots}
\usepackage{tkz-euclide}
\usetkzobj{all}
\usepackage[framemethod=TikZ]{mdframed}
\usetikzlibrary{decorations.pathreplacing}
\tkzSetUpPoint[size=7,fill=white]
%-----------------------
\usepackage{calc,tcolorbox}
\tcbuselibrary{skins,theorems,breakable}
\usepackage{hhline}
\usepackage[explicit]{titlesec}
\usepackage{graphicx}
\usepackage{multicol}
\usepackage{multirow}
\usepackage{tabularx}
\usetikzlibrary{backgrounds}
\usepackage{sectsty}
\sectionfont{\centering}
\usepackage{enumitem}
\usepackage{adjustbox}
\usepackage{mathimatika,gensymb,eurosym,wrap-rl}
\usepackage{systeme,regexpatch}
%-------- ΜΑΘΗΜΑΤΙΚΑ ΕΡΓΑΛΕΙΑ ---------
\usepackage{mathtools}
%----------------------
%-------- ΠΙΝΑΚΕΣ ---------
\usepackage{booktabs}
%----------------------
%----- ΥΠΟΛΟΓΙΣΤΗΣ ----------
\usepackage{calculator}
%----------------------------
%------ ΔΙΑΓΩΝΙΟ ΣΕ ΠΙΝΑΚΑ -------
\usepackage{array}
\newcommand\diag[5]{%
\multicolumn{1}{|m{#2}|}{\hskip-\tabcolsep
$\vcenter{\begin{tikzpicture}[baseline=0,anchor=south west,outer sep=0]
\path[use as bounding box] (0,0) rectangle (#2+2\tabcolsep,\baselineskip);
\node[minimum width={#2+2\tabcolsep-\pgflinewidth},
minimum  height=\baselineskip+#3-\pgflinewidth] (box) {};
\draw[line cap=round] (box.north west) -- (box.south east);
\node[anchor=south west,align=left,inner sep=#1] at (box.south west) {#4};
\node[anchor=north east,align=right,inner sep=#1] at (box.north east) {#5};
\end{tikzpicture}}\rule{0pt}{.71\baselineskip+#3-\pgflinewidth}$\hskip-\tabcolsep}}
%---------------------------------
%---- ΟΡΙΖΟΝΤΙΟ - ΚΑΤΑΚΟΡΥΦΟ - ΠΛΑΓΙΟ ΑΓΚΙΣΤΡΟ ------
\newcommand{\orag}[3]{\node at (#1)
{$ \overcbrace{\rule{#2mm}{0mm}}^{{\scriptsize #3}} $};}
\newcommand{\kag}[3]{\node at (#1)
{$ \undercbrace{\rule{#2mm}{0mm}}_{{\scriptsize #3}} $};}
\newcommand{\Pag}[4]{\node[rotate=#1] at (#2)
{$ \overcbrace{\rule{#3mm}{0mm}}^{{\rotatebox{-#1}{\scriptsize$#4$}}}$};}
%-----------------------------------------
%------------------------------------------
\newcommand{\tss}[1]{\textsuperscript{#1}}
\newcommand{\tssL}[1]{\MakeLowercase{\textsuperscript{#1}}}
%---------- ΛΙΣΤΕΣ ----------------------
\newlist{bhma}{enumerate}{3}
\setlist[bhma]{label=\bf\textit{\arabic*\textsuperscript{o}\;Βήμα :},leftmargin=0cm,itemindent=1.8cm,ref=\bf{\arabic*\textsuperscript{o}\;Βήμα}}
\newlist{rlist}{enumerate}{3}
\setlist[rlist]{itemsep=0mm,label=\roman*.}
\newlist{brlist}{enumerate}{3}
\setlist[brlist]{itemsep=0mm,label=\bf\roman*.}
\newlist{tropos}{enumerate}{3}
\setlist[tropos]{label=\bf\textit{\arabic*\textsuperscript{oς}\;Τρόπος :},leftmargin=0cm,itemindent=2.3cm,ref=\bf{\arabic*\textsuperscript{oς}\;Τρόπος}}
% Αν μπει το bhma μεσα σε tropo τότε
%\begin{bhma}[leftmargin=.7cm]
\tkzSetUpPoint[size=7,fill=white]
\tikzstyle{pl}=[line width=0.3mm]
\tikzstyle{plm}=[line width=0.4mm]
\usepackage{etoolbox}
\makeatletter
\renewrobustcmd{\anw@true}{\let\ifanw@\iffalse}
\renewrobustcmd{\anw@false}{\let\ifanw@\iffalse}\anw@false
\newrobustcmd{\noanw@true}{\let\ifnoanw@\iffalse}
\newrobustcmd{\noanw@false}{\let\ifnoanw@\iffalse}\noanw@false
\renewrobustcmd{\anw@print}{\ifanw@\ifnoanw@\else\numer@lsign\fi\fi}
\makeatother

\usepackage{path}
\pathal

\begin{document}
\titlos{Μαθηματικά Γ΄ Γυμνασίου}{Διαγωνισμός Υποτροφίας}{2018 - 2019}
\vspace{-5mm}
\begin{thema}
\item\mbox{}\\\vspace{-5mm}
\begin{erwthma}
\item Να απαντήσετε στις ακόλουθες ερωτήσεις.
\begin{alist}
\item Τι ονομάζεται ταυτότητα;
\item Γράψτε το τύπο από τον οποίο δίνεται η διακρίνουσα $ \varDelta $ της εξίσωσης $ ax^2+\beta x+\gamma=0,\ a\neq0 $ καθώς και τον τύπο από τον οποίο δίνονται οι λύσεις της εξίσωσης όταν $ \varDelta>0 $.
\item Ποια είναι τα δευτερεύοντα στοιχεία ενός τριγώνου;
\item Να διατυπώσετε το 2\tss{ο} κριτήριο ισότητας τριγώνων.
\end{alist}\monades{4$ \times $0,5=2}
\item \swstolathos
\begin{alist}
\item Υπάρχει γωνία $ x $ με $ \hm{x}=\frac{3}{5} $ και $ \syn{x}=\frac{4}{5} $.
\item Σε δύο τρίγωνα απέναντι από ίσες γωνίες βρίσκονται ίσες πλευρές.
\item Δύο όμοια τρίγωνα έχουν πλευρές ανάλογες.
\item Μια εξίσωση 2\tss{ου} βαθμού έχει πάντα δύο λύσεις.
\item Σε κάθε ισοσκελές τρίγωνο μια διάμεσος είναι και διχοτόμος και ύψος.
\end{alist}\monades{5$ \times $0,2=1}
\item Να επιλέξετε τη σωστή απάντηση σε καθεμία από τις παρακάτω προτάσεις.
\begin{alist}
\item Το ανάπτυγμα της ταυτότητας $ (2x-3)^2 $ είναι:
\begin{multicols}{4}
\begin{rlist}
\item $ 4x^2-9 $
\item $ 4x^2-12x+9 $
\item $ 4x^2+12x+9 $
\item $ 4x^2+9 $
\end{rlist}
\end{multicols}
\item Η εξίσωση $ 2x^2-5x+3=0 $
\begin{multicols}{3}
\begin{rlist}
\item έχει δύο λύσεις.
\item έχει μια λύση.
\item είναι αδύνατη.
\end{rlist}
\end{multicols}
\item Η παράσταση $ 4x^2-25 $ γράφεται σε παραγοντοποιημένη μορφή ως:
\begin{multicols}{4}
\begin{rlist}
\item $ (4x-5)(4x+5) $
\item $ (4x^2-5)(4x^2+5) $
\item $ (2x+5)(2x-5) $
\item $ (2x-5)^2 $
\end{rlist}
\end{multicols}
\item Για να είναι δύο τρίγωνα ίσα πρέπει να έχουν δύο πλευρές ίσες μια προς μια και 
\begin{multicols}{2}
\begin{rlist}
\item μια γωνία ίση.
\item τις προσκείμενες γωνίες ίσες.
\item τις περιεχόμενες γωνίες ίσες.
\item δύο γωνίες ίσες.
\end{rlist}
\end{multicols}
\item Αν για μια οξεία γωνία $ x $ ισχύει $ \hm{x}=\frac{8}{17} $ και $ \syn{x}=\frac{15}{17} $ τότε
\begin{multicols}{4}
\begin{rlist}
\item $ \ef{x}=\frac{8}{15} $
\item $ \ef{x}=\frac{15}{8} $
\item $ \ef{x}=\frac{17}{15} $
\item $ \ef{x}=\frac{17}{8} $
\end{rlist}
\end{multicols}
\end{alist}\monades{5$ \times $0,2=1}
\item Να συμπληρώσετε τα κενά στις παρακάτω προτάσεις.
\begin{alist}
\item Παραγοντοποίηση ονομάζεται η διαδικασία με την οποία μια παράσταση που είναι άθροισμα μετατρέπεται σε \ldots\ldots\ldots\ldots\ldots\ldots\ldots
\item Δύο τρίγωνα είναι όμοια όταν έχουν \ldots\ldots\ldots\ldots\ldots\ldots\ldots\ldots\ldots\ldots\ldots\ ίσες μια προς μια.
\item Οι αλγεβρικές μέθοδοι για να λυθεί ένα γραμμικό σύστημα εξισώσεων είναι η μέθοδος της \ldots\ldots\ldots\ldots\ \ldots\ldots\ldots\ldots\ldots\ και η μέθοδος των \ldots\ldots\ldots\ldots\ldots\ldots\ldots\ldots\ldots\ldots\ldots\ldots\ldots\ldots\ldots
\item Κάθε σημείο της μεσοκαθέτου ενός ευθύγραμμου τμήματος \ldots\ldots\ldots\ldots\ldots\ldots\ldots\ldots\ από τα άκρα του.
\item Μια εξίσωση δευτέρου βαθμού είναι αδύνατη όταν ισχύει \ldots\ldots\ldots\ldots\ldots\ldots
\end{alist}
\end{erwthma}\monades{5$ \times $0,2=1}
\item \mbox{}\\\vspace{-5mm}
\begin{erwthma}
\item Να αποδείξετε ότι για οποιουσδήποτε αριθμούς $ x,y $ ισχύει η παρακάτω ταυτότητα.
\[ \left( x+y\right)^2-\left( x-y\right)^2=4xy \]\monades{2}
\item Να υπολογίσετε \textbf{χρησιμοποιώντας το προηγούμενο ερώτημα} την τιμή της παρακάτω παράστασης.
\[ A=\left( 20+\frac{1}{80}\right)^2-\left( 20-\frac{1}{80}\right)^2 \]\monades{1}
\item Να παραγοντοποιηθούν οι παρακάτω παραστάσεις.
\begin{multicols}{2}
\begin{alist}
\item $ A=12x^3y^4-16x^4y^2z+18x^3y^3z^2 $
\item $ B=x^3-4x^2+5x-20 $
\item $ \varGamma=(x-2)^2-9 $
\item $ \varDelta=4x^2-4x+1 $
\end{alist}
\end{multicols}\monades{4$ \times $0,5=2}
\end{erwthma}
\item \mbox{}\\\vspace{-5mm}
\begin{erwthma}
\item Να λυθούν οι παρακάτω εξισώσεις.
\begin{multicols}{2}
\begin{alist}
\item $ x^2-8x+15=0 $
\item $ (x-1)^2 +3x-5=7x-12 $
\end{alist}
\end{multicols}\monades{0,5+1=1,5}
\item Να λυθεί το ακόλουθο γραμμικό σύστημα: $ \systeme{3x+2y=7,x-4y=-7} $.
\monades{1}
\item Αν για μια \textbf{αμβλεία} γωνία ισχύει $ \hm{x}=\frac{5}{13} $ τότε να δείξετε ότι:
\begin{multicols}{2}
\begin{rlist}
\item $ \syn{x}=-\frac{12}{13} $
\item $ \ef{x}=-\frac{5}{12} $
\end{rlist}
\end{multicols}
\monades{1+0,5=1,5}
\item Να αποδείξετε ότι
\[ (\hm{x}+\syn{x})^2+(\hm{x}-\syn{x})^2=2 \]\monades{1}
\end{erwthma}
\item\mbox{}\\
\begin{erwthma}
\wrapr{-17mm}{7}{3.4cm}{-5mm}{\begin{tikzpicture}[scale=0.8]
\tkzDefPoint(0,0){B}
\tkzDefPoint(3,0){C}
\tkzDefPoint(1.5,4){A}
\tkzDefPoint(2.25,2){D}
\tkzDefPoint(.75,2){E}
\tkzInterLL(B,D)(C,E)\tkzGetPoint{M}
\draw[pl] (A)--(B)--(C)--cycle;
\draw[pl,\xrwma] (B)--(D);
\draw[pl,\xrwma] (C)--(E);
\draw[pl,\xrwma] (A)--(M);
\tkzLabelPoint[above](A){$ A $}
\tkzLabelPoint[left](B){$ B $}
\tkzLabelPoint[right](C){$ \varGamma $}
\tkzLabelPoint[right](D){$ \varDelta $}
\tkzLabelPoint[left](E){$ E $}
\tkzLabelPoint[below,yshift=-1mm](M){$ M $}
\tkzDrawPoints(A,B,C,D,E,M)
\end{tikzpicture}
}{
\item Δίνεται ισοσκελές τρίγωνο $ AB\varGamma $ με $ AB=A\varGamma $. Αν $ B\varDelta $ και $ \varGamma E $ είναι διάμεσοι τότε 
\begin{alist}[leftmargin=5mm]
\item να δείξετε ότι $ B\varDelta=\varGamma E $.\monades{1,5}
\item αν $ M $ είναι το σημείο τομής των διαμέσων και $ BM=\varGamma M $ να δείξετε ότι η $ AM $ διχοτομεί τη γωνία $ \hat{A} $.\monades{1}
\end{alist}}
\wrapr{-5mm}{7}{4cm}{-5mm}{
\begin{tikzpicture}
\tkzDefPoint(0,0){A}
\tkzDefPoint(0,2){B}
\tkzDefPoint(3,0){C}
\tkzDrawAltitude[draw=\xrwma](B,C)(A)\tkzGetPoint{D}
\tkzMarkRightAngle[size=.25](C,A,B)
\tkzMarkRightAngle[size=.25](A,D,C)
\draw[pl] (A)--(B)--(C)--cycle;
\tkzLabelPoint[left](A){$ A $}
\tkzLabelPoint[left](B){$ B $}
\tkzLabelPoint[right](C){$ \varGamma $}
\tkzLabelPoint[above right](D){$ \varDelta $}
\tkzDrawPoints(A,B,C,D)
\end{tikzpicture}
}{
\item Δίνεται ορθογώνιο τρίγωνο $ AB\varGamma $ με $ \hat{A}=90\degree $ και $ A\varDelta $ το ύψος του τριγώνου. Επίσης αν γνωρίζουμε ότι $ AB=15, A\varDelta=12 $ και $ A\varGamma=20 $ τότε
\begin{alist}
\item να δείξετε ότι τα τρίγωνα $ AB\varGamma $ και $ A\varGamma\varDelta $ είναι όμοια.\monades{1}
\item να δείξετε ότι $ \varGamma\varDelta=16 $ και $ B\varDelta=9 $.\monades{1,5}
\end{alist}}
\end{erwthma}
\end{thema}
\vfill
\begin{flushright}
Διάρκεια $ 2 $ ώρες και $ 30 $ λεπτά.\\
\textit{Καλή επιτυχία!}
\end{flushright}
\end{document}
