\documentclass[twoside,nofonts,internet,fil,maria]{frontisthrio-diag}
\usepackage[amsbb,subscriptcorrection,zswash,mtpcal,mtphrb,mtpfrak]{mtpro2}
\usepackage[no-math,cm-default]{fontspec}
\usepackage{amsmath}
\usepackage{xunicode}
\usepackage{xgreek}
\let\hbar\relax
\defaultfontfeatures{Mapping=tex-text,Scale=MatchLowercase}
\setmainfont[Mapping=tex-text,Numbers=Lining,Scale=1.0,BoldFont={Minion Pro Bold}]{Minion Pro}
\newfontfamily\scfont{GFS Artemisia}
\font\icon = "Webdings"
\usepackage{fontawesome}
\newfontfamily{\FA}{fontawesome.otf}
\xroma{red!70!black}
%------TIKZ - ΣΧΗΜΑΤΑ - ΓΡΑΦΙΚΕΣ ΠΑΡΑΣΤΑΣΕΙΣ ----
\usepackage{tikz,pgfplots}
\usepackage{tkz-euclide}
\usetkzobj{all}
\usepackage[framemethod=TikZ]{mdframed}
\usetikzlibrary{decorations.pathreplacing}
\tkzSetUpPoint[size=7,fill=white]
%-----------------------
\usepackage{calc,tcolorbox}
\tcbuselibrary{skins,theorems,breakable}
\usepackage{hhline}
\usepackage[explicit]{titlesec}
\usepackage{graphicx}
\usepackage{multicol}
\usepackage{multirow}
\usepackage{tabularx}
\usetikzlibrary{backgrounds}
\usepackage{sectsty}
\sectionfont{\centering}
\usepackage{enumitem}
\usepackage{adjustbox}
\usepackage{mathimatika,gensymb,eurosym,wrap-rl}
\usepackage{systeme,regexpatch}
%-------- ΜΑΘΗΜΑΤΙΚΑ ΕΡΓΑΛΕΙΑ ---------
\usepackage{mathtools}
%----------------------
%-------- ΠΙΝΑΚΕΣ ---------
\usepackage{booktabs}
%----------------------
%----- ΥΠΟΛΟΓΙΣΤΗΣ ----------
\usepackage{calculator}
%----------------------------
%------ ΔΙΑΓΩΝΙΟ ΣΕ ΠΙΝΑΚΑ -------
\usepackage{array}
\newcommand\diag[5]{%
\multicolumn{1}{|m{#2}|}{\hskip-\tabcolsep
$\vcenter{\begin{tikzpicture}[baseline=0,anchor=south west,outer sep=0]
\path[use as bounding box] (0,0) rectangle (#2+2\tabcolsep,\baselineskip);
\node[minimum width={#2+2\tabcolsep-\pgflinewidth},
minimum  height=\baselineskip+#3-\pgflinewidth] (box) {};
\draw[line cap=round] (box.north west) -- (box.south east);
\node[anchor=south west,align=left,inner sep=#1] at (box.south west) {#4};
\node[anchor=north east,align=right,inner sep=#1] at (box.north east) {#5};
\end{tikzpicture}}\rule{0pt}{.71\baselineskip+#3-\pgflinewidth}$\hskip-\tabcolsep}}
%---------------------------------
%---- ΟΡΙΖΟΝΤΙΟ - ΚΑΤΑΚΟΡΥΦΟ - ΠΛΑΓΙΟ ΑΓΚΙΣΤΡΟ ------
\newcommand{\orag}[3]{\node at (#1)
{$ \overcbrace{\rule{#2mm}{0mm}}^{{\scriptsize #3}} $};}
\newcommand{\kag}[3]{\node at (#1)
{$ \undercbrace{\rule{#2mm}{0mm}}_{{\scriptsize #3}} $};}
\newcommand{\Pag}[4]{\node[rotate=#1] at (#2)
{$ \overcbrace{\rule{#3mm}{0mm}}^{{\rotatebox{-#1}{\scriptsize$#4$}}}$};}
%-----------------------------------------
%------------------------------------------
\newcommand{\tss}[1]{\textsuperscript{#1}}
\newcommand{\tssL}[1]{\MakeLowercase{\textsuperscript{#1}}}
%---------- ΛΙΣΤΕΣ ----------------------
\newlist{bhma}{enumerate}{3}
\setlist[bhma]{label=\bf\textit{\arabic*\textsuperscript{o}\;Βήμα :},leftmargin=0cm,itemindent=1.8cm,ref=\bf{\arabic*\textsuperscript{o}\;Βήμα}}
\newlist{rlist}{enumerate}{3}
\setlist[rlist]{itemsep=0mm,label=\roman*.}
\newlist{brlist}{enumerate}{3}
\setlist[brlist]{itemsep=0mm,label=\bf\roman*.}
\newlist{tropos}{enumerate}{3}
\setlist[tropos]{label=\bf\textit{\arabic*\textsuperscript{oς}\;Τρόπος :},leftmargin=0cm,itemindent=2.3cm,ref=\bf{\arabic*\textsuperscript{oς}\;Τρόπος}}
% Αν μπει το bhma μεσα σε tropo τότε
%\begin{bhma}[leftmargin=.7cm]
\tkzSetUpPoint[size=7,fill=white]
\tikzstyle{pl}=[line width=0.3mm]
\tikzstyle{plm}=[line width=0.4mm]
\usepackage{etoolbox}
\makeatletter
\renewrobustcmd{\anw@true}{\let\ifanw@\iffalse}
\renewrobustcmd{\anw@false}{\let\ifanw@\iffalse}\anw@false
\newrobustcmd{\noanw@true}{\let\ifnoanw@\iffalse}
\newrobustcmd{\noanw@false}{\let\ifnoanw@\iffalse}\noanw@false
\renewrobustcmd{\anw@print}{\ifanw@\ifnoanw@\else\numer@lsign\fi\fi}
\makeatother

\usepackage{path}
\pathal

\begin{document}
\titlos{Νεοελληνική γλώσσα Β΄ Γυμνασίου}{Διαγωνισμός Υποτροφίας}{2018 - 2019}
\textcolor{\xrwma}{\textbf{Α΄ ΚΕΙΜΕΝΟ}}
\begin{center}
ΟΙ ΣΧΕΣΕΙΣ ΣΤΑ ΧΡΟΝΙΑ ΤΟΥ FACEBOOK
\end{center}
… Αλλά τι είναι ένας φίλος σήμερα; Μπορούμε όντως να τον βρούμε στο διαδίκτυο;
Σύμφωνα με την ψυχαναλύτρια Daniele Brun, 4 είναι τα συστατικά στοιχεία της
φιλίας : η εμπιστοσύνη, η ένταξη σε μια περιορισμένη κοινότητα φίλων , κοινά
ενδιαφέροντα και, εφόσον μιλάμε για διαδικτυακή φιλία, ένα κοινό βίωμα που
μπορεί να συντηρηθεί μέσω του διαδικτύου. Άλλοι προσθέτουν την ειλικρίνεια, την
ευθύτητα, την ανιδιοτέλεια και την έλλειψη ζήλιας.

Τα κριτήρια της φιλίας διαφοροποιούνται ανάλογα με το επάγγελμα, το
μορφωτικό επίπεδο και το φύλο. Για τους εργάτες, ο φίλος είναι μέλος της
οικογένειας, κάποιος στον οποίο μπορείς να υπολογίζεις σε μια δύσκολη στιγμή.
Στην κατηγορία των υπαλλήλων προέχει η εχεμύθεια, τα κοινά ενδιαφέροντα και η
διάρκεια της σχέσης. Για τα στελέχη επιχειρήσεων με με πτυχία ανώτατης
εκπαίδευσης το σημαντικό στοιχείο μιας φιλίας είναι η κοινωνική «εγγύτητα», η
απλότητα , αλλά και η ανεπιτήδευτη συμπεριφορά.
Σύμφωνα με την καθηγήτρια Daniele Brun , ο φίλος μας αντανακλά μια θετική
εικόνα για τον εαυτό μας . Αυτό συμβαίνει , διότι , μέσα από την ελευθερία λόγου
της φιλίας, μπορούμε να κρίνουμε και να κρινόμαστε , να εκφράζουμε τις σκέψεις
μας για τους πάντες και τα πάντα. Να γιατί είναι τόσο δύσκολο να ζούμε χωρίς
φίλους.

Όμως, το βασικό ερώτημα παραμένει : ένας φίλος στο Facebook κι ένας φίλος
στην πραγματική ζωή είναι το ίδιο πράγμα; Μπορούμε να λέμε ότι έχουμε φίλο
κάποιον χωρίς να τον συναντάμε; Έρευνα του Πανεπιστημίου Αθηνών έδειξε ότι οι
άνθρωποι κατανοούν και βιώνουν τη φιλία στο διαδίκτυο με εντελώς διαφορετικό
τρόπο απ’ ό,τι στην πραγματική ζωή. Έτσι, δεν περιμένουν και δεν προσφέρουν
στους διαδικτυακούς φίλους ό,τι προσφέρουν στους φίλους της πραγματικής ζωής.
Η φιλία εμπεριέχει την εμπιστοσύνη, την αμοιβαία αποδοχή, την επιθυμία για
συναναστροφή, άρα και τη φυσική εγγύτητα. Όπως λέει χαρακτηριστικά και η
Daniele Brun, στη φιλία υπάρχει πάντα το στοιχείο της φυσικής παρουσίας, της
σωματικής «συνάντησης» που δεν μπορεί να παραβλεφθεί .Επίσης, στην
πραγματική φιλία εκδηλώνονται και αρνητικά συναισθήματα ή συμπεριφορές,
όπως ζήλια, έπαρση, εμπάθεια που μπορεί να μεταφερθούν και στη διαδικτυακή
κοινότητα, με τη διαφορά ότι εκεί οδηγούν ευκολότερα σε ρήξεις . Στην πραγματική
ζωή, ωστόσο, καλείσαι να διαχειριστείς αυτά τα συναισθήματα, καθώς ο άλλος
βρίσκεται μπροστά σου, ενώ στο διαδίκτυο πατώντας delete τελειώνεις μια κι έξω
μαζί του.

Άλλες μελέτες έχουν δείξει ότι ακόμα και οι πιο εξοικειωμένοι με το διαδίκτυο,
όπως οι έφηβοι, δίνουν το ίδιο μεγάλη σημασία στη φυσική παρουσία. Ξέρουν ότι
κάποια στιγμή θα πρέπει να εγκαταλείψουν την οθόνη του υπολογιστή που κρύβει
πολύ καλά τα αντιπαθητικά σπυράκια τους και να συναντηθούν με τον άλλο στον
κόσμο των τριών διαστάσεων , για να του μεταδώσουν ένα πιο σαφές μήνυμα.

Αντίθετα απ’ ό,τι θα περίμενε κανείς, οι έφηβοι μοιράζονται άλλο ένα κοινό σημείο
με τους μεγαλύτερούς τους. Ενώ έχουν πολλούς διαδικτυακούς φίλους, στην
πραγματικότητα μόνον 3-4 θεωρούν σημαντικούς και θέλουν να εμβαθύνουν τη
σχέση τους μαζί τους .
\begin{flushright}
\begin{minipage}{5.5cm}
\begin{flushleft}
\textit{Νίκος Κυριακίδης\\
(12/ 2011 από το περιοδ. «Ε» της Ελευθεροτυπίας , απόσπασμα σε διασκευή.)}
\end{flushleft}
\end{minipage}
\end{flushright}
\textcolor{\xrwma}{\textbf{Β΄ ΠΑΡΑΤΗΡΗΣΕΙΣ}}
\begin{enumerate}
\item Να γράψετε περίληψη του αποσπάσματος ( περίπου 150 λέξεις ).\monades{2}
\item Ποια χαρακτηριστικά στοιχεία διαφοροποιούν τη φιλία της πραγματικής ζωής από τη φιλία του διαδικτύου σύμφωνα με την άποψη του συγγραφέα;\\ \monades{2}
\item Να δώσετε  έναν πλαγιότιτλο στη δεύτερη παράγραφο.\monades{1}
\item Να γράψετε μια συνώνυμη για κάθε μία από τις παρακάτω λέξεις του κειμένου: 
 \begin{center}
ζήλιας,σημαντικό ,αντανακλά, έπαρση, σαφές.
\end{center} \monades{5}
\item Να αναφέρετε ποια είναι η δομή και ο τρόπος (ή οι τρόποι) ανάπτυξης της
2ης και 3ης παραγράφου του κειμένου.\monades{2}
\item Σε ένα άρθρο που θα δημοσιευτεί στην εφημερίδα του σχολείου σας να αναφέρετε τα κριτήρια με τα οποία πρέπει να επιλέγουμε τους φίλους μας. \monades{8}
\end{enumerate}



\end{document}
