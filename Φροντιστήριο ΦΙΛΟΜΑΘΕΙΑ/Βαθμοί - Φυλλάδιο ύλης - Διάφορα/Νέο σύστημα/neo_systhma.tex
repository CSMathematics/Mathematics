\documentclass[twoside,nofonts,internet,math,spyros]{Aithsh-gnwsh}
\usepackage[amsbb,subscriptcorrection,zswash,mtpcal,mtphrb,mtpfrak]{mtpro2}
\usepackage[no-math,cm-default]{fontspec}
\usepackage{amsmath}
\usepackage{xunicode}
\usepackage{xgreek}
\let\hbar\relax
\defaultfontfeatures{Mapping=tex-text,Scale=MatchLowercase}
\setmainfont[Mapping=tex-text,Numbers=Lining,Scale=1.0,BoldFont={Minion Pro Bold}]{Minion Pro}
\newfontfamily\scfont{GFS Artemisia}
\font\icon = "Webdings"
\usepackage{fontawesome}
\newfontfamily{\FA}{fontawesome.otf}
%------TIKZ - ΣΧΗΜΑΤΑ - ΓΡΑΦΙΚΕΣ ΠΑΡΑΣΤΑΣΕΙΣ ----
\usepackage{tikz,pgfplots}
\usepackage{tkz-euclide}
\usetkzobj{all}
\usepackage[framemethod=TikZ]{mdframed}
\usetikzlibrary{decorations.pathreplacing}
\tkzSetUpPoint[size=7,fill=white]
%-----------------------
\usepackage{calc,tcolorbox}
\tcbuselibrary{skins,theorems,breakable}
\usepackage{hhline}
\usepackage[explicit]{titlesec}
\usepackage{graphicx}
\usepackage{multicol}
\usepackage{multirow}
\usepackage{tabularx}
\usetikzlibrary{backgrounds}
\usepackage{sectsty}
\sectionfont{\centering}
\usepackage{enumitem}
\usepackage{adjustbox}
\usepackage{mathimatika,gensymb,eurosym,wrap-rl}
\usepackage{systeme,regexpatch}
%-------- ΜΑΘΗΜΑΤΙΚΑ ΕΡΓΑΛΕΙΑ ---------
\usepackage{mathtools}
%----------------------
%-------- ΠΙΝΑΚΕΣ ---------
\usepackage{booktabs}
%----------------------
%----- ΥΠΟΛΟΓΙΣΤΗΣ ----------
\usepackage{calculator}
%----------------------------
%------ ΔΙΑΓΩΝΙΟ ΣΕ ΠΙΝΑΚΑ -------
\usepackage{array}
\newcommand\diag[5]{%
\multicolumn{1}{|m{#2}|}{\hskip-\tabcolsep
$\vcenter{\begin{tikzpicture}[baseline=0,anchor=south west,outer sep=0]
\path[use as bounding box] (0,0) rectangle (#2+2\tabcolsep,\baselineskip);
\node[minimum width={#2+2\tabcolsep-\pgflinewidth},
minimum  height=\baselineskip+#3-\pgflinewidth] (box) {};
\draw[line cap=round] (box.north west) -- (box.south east);
\node[anchor=south west,align=left,inner sep=#1] at (box.south west) {#4};
\node[anchor=north east,align=right,inner sep=#1] at (box.north east) {#5};
\end{tikzpicture}}\rule{0pt}{.71\baselineskip+#3-\pgflinewidth}$\hskip-\tabcolsep}}
%---------------------------------
%---- ΟΡΙΖΟΝΤΙΟ - ΚΑΤΑΚΟΡΥΦΟ - ΠΛΑΓΙΟ ΑΓΚΙΣΤΡΟ ------
\newcommand{\orag}[3]{\node at (#1)
{$ \overcbrace{\rule{#2mm}{0mm}}^{{\scriptsize #3}} $};}
\newcommand{\kag}[3]{\node at (#1)
{$ \undercbrace{\rule{#2mm}{0mm}}_{{\scriptsize #3}} $};}
\newcommand{\Pag}[4]{\node[rotate=#1] at (#2)
{$ \overcbrace{\rule{#3mm}{0mm}}^{{\rotatebox{-#1}{\scriptsize$#4$}}}$};}
%-----------------------------------------
%------------------------------------------
\newcommand{\tss}[1]{\textsuperscript{#1}}
\newcommand{\tssL}[1]{\MakeLowercase{\textsuperscript{#1}}}
%---------- ΛΙΣΤΕΣ ----------------------
\newlist{bhma}{enumerate}{3}
\setlist[bhma]{label=\bf\textit{\arabic*\textsuperscript{o}\;Βήμα :},leftmargin=0cm,itemindent=1.8cm,ref=\bf{\arabic*\textsuperscript{o}\;Βήμα}}
\newlist{rlist}{enumerate}{3}
\setlist[rlist]{itemsep=0mm,label=\roman*.}
\newlist{brlist}{enumerate}{3}
\setlist[brlist]{itemsep=0mm,label=\bf\roman*.}
\newlist{tropos}{enumerate}{3}
\setlist[tropos]{label=\bf\textit{\arabic*\textsuperscript{oς}\;Τρόπος :},leftmargin=0cm,itemindent=2.3cm,ref=\bf{\arabic*\textsuperscript{oς}\;Τρόπος}}
% Αν μπει το bhma μεσα σε tropo τότε
%\begin{bhma}[leftmargin=.7cm]
\tkzSetUpPoint[size=7,fill=white]
\tikzstyle{pl}=[line width=0.3mm]
\tikzstyle{plm}=[line width=0.4mm]
\usepackage{etoolbox}
\makeatletter
\renewrobustcmd{\anw@true}{\let\ifanw@\iffalse}
\renewrobustcmd{\anw@false}{\let\ifanw@\iffalse}\anw@false
\newrobustcmd{\noanw@true}{\let\ifnoanw@\iffalse}
\newrobustcmd{\noanw@false}{\let\ifnoanw@\iffalse}\noanw@false
\renewrobustcmd{\anw@print}{\ifanw@\ifnoanw@\else\numer@lsign\fi\fi}
\makeatother


\begin{document}
\titlos
\vspace{-1.7cm}
\begin{center}
Διεύθυνση σπουδών: Αραπατζής Αναστάσιος, Φρόνιμος Σπύρος, Πρεντουλή Μαρία-Σπυριδούλα\\
\today
\end{center}
\begin{center}
{\Large \textbf{Νέο σύστημα εκπαίδευσης}}
\end{center}
Το νέο σύστημα εκπαίδευσης θα εφαρμοστεί τη σχολική χρονιά 2019-2020 και αφορά αρχικά τους μαθητές που τελειώνουν τη φετινή Β΄ Λυκείου. Σε περίπτωση που ψηφιστεί το Δεκέμβριο το νομοσχέδιο αυτό, τότε θα ισχύουν οι εξής αλλαγές:\\
{\large \textbf{Αλλαγές στα μαθήματα}}
\begin{enumerate}
\item Οι 3 ομάδες προσανατολισμού στην Γ΄ Λυκείου γίνονται 4:
\begin{multicols}{2}
\begin{rlist}
\item Ανθρωπιστικών σπουδών
\item Θετικών σπουδών
\item Σπουδών Υγείας
\item Οικονομικών σπουδών και Πληροφορικής
\end{rlist}
\end{multicols}
\item Τα μαθήματα σε κάθε ομάδα προσανατολισμού είναι 3 με 6 διδακτικές ώρες το καθένα.
\item Τα μαθήματα γενικής παιδείας γίνονται 3.
\item Αντικαθίσταται στην ομάδα Ανθρωπιστικών σπουδών, το μάθημα των Λατινικών από το μάθημα της Κοινωνιολογίας.
\item Εμπλουτίζεται το μάθημα των Οικονομικών.
\item Καταργούνται οι συντελεστές βαρύτητας στα μαθήματα των ομάδων προσανατολισμού.
\end{enumerate}
{\large \textbf{Αλλαγές στις εξετάσεις}}
\begin{enumerate}[resume]
\item Οι ενδοσχολικές εξετάσεις θα γίνονται σε επίπεδο νομών και δήμων. Θα δίνονται μόνο τα 3 μαθήματα προσανατολισμού μαζί με το μάθημα της νεοελληνικής γλώσσας και γραμματείας.
\item Η πρόσβαση στην Γ΄ Βάθμια εκπαίδευση γίνεται με 2 τρόπους:
\begin{rlist}
\item Με ελεύθερη πρόσβαση μόνο με την απόκτηση του απολυτηρίου.
\item Με πανελλήνιες εξετάσεις (ισχύον σύστημα).
\end{rlist}
\item Ο βαθμός πρόσβασης στην Γ΄ Βάθμια εκπαίδευση για τους μαθητές που θα δώσουν πανελλήνιες εξετάσεις σχηματίζεται κατά 10\% από το βαθμό του απολυτηρίου και κατά 90\% από τα αποτελέσματα των εξετάσεων.
\item Ο βαθμός του απολυτηρίου προκύπτει κατά 40\% από το βαθμό στις ενδοσχολικές εξετάσεις και κατά 60\% από το βαθμό των 2 τετραμήνων.
\item Η πρώτη \textbf{υποχρεωτική} υποβολή μηχανογραφικού γίνεται στο τέλος της Β΄ Λυκείου (πιθανόν την περίοδο του Ιουλίου). Στη δήλωση αυτή κάθε μαθητής μπορεί να δηλώσει έως και 10 σχολές. Η μη υποβολή μηχανογραφικού την περίοδο αυτή, αποκλείει το μαθητή από την Γ΄ βάθμια εκπαίδευση.
\item Όσες σχολές συγκεντρώσουν αριθμό δηλώσεων ίσο ή μικρότερο από τον αριθμό των εισακτέων που έχουν αναγγείλει, θα χαρακτηρίζονται ως ΤΕΠ: Τμήματα ελεύθερης πρόσβασης. Οι υπόλοιπες σχολές χαρακτηρίζονται ως ΤΠΠΕ: Τμήματα Πρόσβασης με Πανελλήνιες Εξετάσεις. Αυτό θα γίνεται γνωστό λίγες μέρες αργότερα.
\item Το Φεβρουάριο κάθε μαθητής, γνωρίζοντας αν το μηχανογραφικό του περιέχει τουλάχιστον ένα ΤΕΠ, υποχρεούται να δηλώσει αν θα δώσει πανελλήνιες ή αν θα περάσει μόνο με το βαθμό του απολυτηρίου σε \textbf{ένα} ΤΕΠ της επιλογής του. Μετά την δήλωση αυτή δεν γίνεται δεκτή καμία αλλαγή.
\item Κάθε μαθητής μπορεί να δηλώσει σχολές μόνο από το πεδίο στο οποίο αντιστοιχεί η ομάδα προσανατολισμού του.
\end{enumerate}
\begin{center}
{\large \textbf{Μαθήματα γενικής παιδείας}}
\end{center}
\begin{enumerate}
\item Νεοελληνική γλώσσα και Γραμματεία - 6 ώρες.
\item Θρησκευτικά - 1 ώρα.
\item Φυσική αγωγή - 2 ώρες.
\end{enumerate}
\setlength\arrayrulewidth{1pt}\arrayrulecolor{white}
\begin{center}
{\large \textbf{Μαθήματα ομάδων προσανατολισμού}}\mbox{}\\\vspace{3mm}
\begin{tabular}{l|c|c|c|c}
\rowcolor{cyan!50!gray}\rule[-2ex]{0pt}{5ex}  & Ανθρωπιστικών σπουδών      & Θετικών σπουδών & Σπουδών Υγείας & Οικονομικών σπουδών και Πλ. \\ 
\hhline{=====}\rowcolor[HTML]{ECF4FF}\rule[-2ex]{0pt}{5ex}
1 & Αρχαία ελληνική γραμματεία & Μαθηματικά      & Φυσική         & Μαθηματικά                           \\
\rule[-2ex]{0pt}{5ex} 2 & Ιστορία                    & Φυσική          & Χημεία         & Α.Ε.Π.Π.                             \\
\rowcolor[HTML]{ECF4FF}\rule[-2ex]{0pt}{5ex} 3 & Κοινωνιολογία              & Χημεία          & Βιολογία       & Α.Ο.Θ.                              
\end{tabular}
\end{center}
\vspace{0.5cm}
Ακολουθεί ένα διάγραμμα στο οποίο φαίνεται η διαδικασία για την εισαγωγή στην Γ΄ βάθμια εκπαίδευση.
\vspace{.5cm}

\end{document}



