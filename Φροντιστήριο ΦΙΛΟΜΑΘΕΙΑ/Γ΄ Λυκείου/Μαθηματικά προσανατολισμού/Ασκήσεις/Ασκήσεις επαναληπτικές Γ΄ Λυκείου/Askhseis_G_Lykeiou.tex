\documentclass[10pt,a4paper]{article}
\usepackage[utf8]{inputenc}
\usepackage{nimbusserif}
\usepackage[T1]{fontenc}
\usepackage[english,greek]{babel}
\usepackage{amsmath,mathimatika,gensymb,multicol,enumitem}
\let\myBbbk\Bbbk
\let\Bbbk\relax
\usepackage[amsbb,subscriptcorrection,zswash,mtpcal,mtphrb,mtpfrak]{mtpro2}
\usepackage[left=2cm,right=2cm,top=2cm,bottom=2cm]{geometry}
\newcommand{\en}[1]{\selectlanguage{english}{#1}}
\setlist[enumerate]{itemsep=0mm}
\begin{document}
\begin{enumerate}
\item Δίνονται οι ακόλουθες συναρτήσεις καθώς και ένα σημείο $ x_0 $ του πεδίου ορισμού τους:
\begin{multicols}{2}
\begin{enumerate}[itemsep=0mm,label=\roman*.]
\item $ f(x)=x^2-4x+3,\ x_0=0 $
\item $ f(x)=x^3-3x^2+9x-5,\ x_0=1 $
\item $ f(x)=\frac{x-1}{x+2},\ x_0=-1 $
\item $ f(x)=\frac{x}{x^2+1},\ x_0=1 $
\item $ f(x)=\hm{x}-x,\ x\in[0,2\pi],\ x_0=\pi$
\item $ f(x)=x\ln{x},\ x_0=e^2 $
\item $ f(x)=\sqrt{x^2-2x},\ x_0=3 $
\item $ f(x)=e^xx^2,\ x_0=1 $
\item $ f(x)=\ln{\left(4-x^2\right)},\ x_0=\sqrt{3} $
\item $ f(x)=x-\frac{4}{x^2},\ x_0=2 $
\end{enumerate}
\end{multicols}
Για καθεμία απ' αυτές
\begin{enumerate}[itemsep=0mm,label=\alph*.]
\item να βρεθεί το πεδίο ορισμού
\item να μελετήσετε τη μονοτονία και τα ακρότατα
\item να βρεθεί το σύνολο τιμών
\item η εφαπτομένη στο σημείο $ A(x_0,f(x_0)) $
\end{enumerate}
\item Για καθεμία από τις παρακάτω συναρτήσεις
\begin{multicols}{2}
\begin{enumerate}[itemsep=0mm,label=\roman*.]
\item $ f(x)=\ln{(x-1)} $
\item $ f(x)=\sqrt{3-x} $
\item $ f(x)=e^{x-4} $
\item $ f(x)=\frac{e^x}{e^x+1} $
\end{enumerate}
\end{multicols}
\begin{enumerate}[itemsep=0mm,label=\alph*.]
\item να βρεθεί η αντίστροφη συνάρτηση $ f^{-1} $
\item να βρεθούν τα σημεία τομής της $ C_{f^{-1}} $ με τους άξονες $ x'x $ και $ y'y $.
\end{enumerate}
\item Δίνονται οι παρακάτω συναρτήσεις $ f,g $:
\begin{multicols}{2}
\begin{enumerate}[itemsep=0mm,label=\roman*.]
\item $ f(x)=\ln{x}\ ,\ g(x)=\sqrt{x-1} $
\item $ f(x)=\frac{1}{x-1}\ ,\ g(x)=\sqrt{2-x} $
\item $ f(x)=e^x\ ,\ g(x)=\frac{1}{1-x} $
\item $ f(x)=\frac{x}{x-1}\ ,\ g(x)=\frac{2}{x-3} $
\end{enumerate}
\end{multicols}
Για καθεμία απ' αυτές
\begin{enumerate}[itemsep=0mm,label=\alph*.]
\item να βρεθεί το πεδίο ορισμού
\item να βρεθεί η σύνθετη συνάρτηση $ g\circ f $
\item να βρεθεί η σύνθετη συνάρτηση $ f\circ g $
\end{enumerate}
\item Δίνονται οι παρακάτω παραγωγίσιμες συναρτήσεις:
\begin{multicols}{2}
\begin{enumerate}[itemsep=0mm,label=\roman*.]
\item $ f(x)=\ccases{x^2+\beta x+a & x\geq1\\\frac{2x+a}{x+1} & x<1} $
\item $  $
\end{enumerate}
\end{multicols}
Για καθεμία απ' αυτές
\begin{enumerate}[itemsep=0mm,label=\alph*.]
\item Να βρεθούν οι παράμετροι $ a,\beta $.
\item Να μελετηθούν ως προς τη μονοτονία και τα ακρότατα.
\end{enumerate}
\item Να υπολογιστούν τα παρακάτω όρια
\begin{multicols}{3}
\begin{enumerate}[itemsep=0mm,label=\roman*.]
\item $ \lim\limits_{x\to 2}\dfrac{x^3-3x-2}{x^2-4} $
\item $ \lim\limits_{x\to 0}\dfrac{\hm^2{x}-x^2}{x^2} $
\item $ \lim\limits_{x\to -1}\dfrac{x^2-3x+4}{x^3-2x^2+x} $
\item $ \lim\limits_{x\to 1}\dfrac{\sqrt{4x+5}-3}{1-x} $
\item $ \lim\limits_{x\to +\infty}{\dfrac{(x-2)^2-3x}{4-2x^2}} $
\item $ \lim\limits_{x\to 3}\dfrac{\hm{(\pi x)}-1}{x^2-9} $
\item $ \lim\limits_{x\to 2}\left(\dfrac{1}{x-2}-\dfrac{2}{x^2-4}\right) $
\end{enumerate}
\end{multicols}
\item Δίνεται η συνάρτηση $ f(x)=x^2-4x+3 $. Να βρεθεί η εξίσωση της εφαπτομένης της $ C_f $ η οποία
\begin{enumerate}[itemsep=0mm,label=\roman*.]
\item έχει συντελεστή διεύθυνσης $ \lambda=2 $.
\item είναι παράλληλη με την ευθεία $ \varepsilon:y=3x-2 $.
\item είναι κάθετη στην ευθεία $\zeta: 6x-2y+5=0 $.
\item σχηματίζει γωνία $ \omega=45\degree $ με τον άξονα $ x'x $.
\end{enumerate}
\item Για καθεμία από τις ακόλουθες συναρτήσεις να βρεθεί η εξίσωση τη εφαπτομένης της $ C_f $ η οποία διέρχεται από το εξωτερικό σημείο $ A $ που δίνεται.
\begin{multicols}{2}
\begin{enumerate}[itemsep=0mm,label=\roman*.]
\item $ f(x)=x^2+3x+2\ ,\ A(-2,-4) $
\item $ f(x)=\dfrac{x-1}{x+2}\ ,\ A(-5,-2) $
\end{enumerate}
\end{multicols}
\item Να βρεθεί για καθεμία από τις παρακάτω συναρτήσεις η πρώτη παράγωγος.
\begin{multicols}{3}
\begin{enumerate}[label=\roman*.]
\item $ f(x)=x^2-3x+2 $
\item $ f(x)=x^3-3x^2+5x-4 $
\item $ f(x)=2x^4-x^2+\sqrt{2} $
\item $ f(x)=3x^3-2t+\hm{\theta} $
\item $ f(y)=x^3-3y^2+y-x $
\item $ f(x)=x^2+\ln{2} $
\item $ f(x)=e^x+\sqrt{x}-\ln{x} $
\item $ f(x)=\hm{x}+\syn{x} $
\item $ f(x)=x\ln{x} $
\item $ f(x)=x^2e^x $
\item $ f(x)=x\hm{x} $
\item $ f(x)=e^x\syn{x} $
\item $ f(x)=\dfrac{x}{x-1} $
\item $ f(x)=\dfrac{x^2}{x^2-1} $
\item $ f(x)=\dfrac{\ln{x}}{x} $
\item $ f(x)=\dfrac{\hm{x}}{x^2} $
\item $ f(x)=\dfrac{e^x}{\sqrt{x}} $
\end{enumerate}
\end{multicols}
\item Να βρεθεί για καθεμία από τις παρακάτω συναρτήσεις η πρώτη παράγωγος.
\begin{multicols}{3}
\begin{enumerate}[label=\roman*.]
\item $ f(x)=(x^2-x)^3 $
\item $ f(x)=\hm^4{x} $
\item $ f(x)=\ln^2{x} $
\item $ f(x)=\left(\sqrt{x}-\frac{1}{x}\right)^5 $
\end{enumerate}
\end{multicols}
\end{enumerate}
\end{document}
