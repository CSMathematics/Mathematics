\documentclass[11pt,a4paper,twocolumn]{article}
\usepackage[english,greek]{babel}
\usepackage[utf8]{inputenc}
\usepackage{nimbusserif}
\usepackage[T1]{fontenc}
\usepackage[left=1.50cm, right=1.50cm, top=2.00cm, bottom=2.00cm]{geometry}
\usepackage{amsmath}
\let\myBbbk\Bbbk
\let\Bbbk\relax
\usepackage[amsbb,subscriptcorrection,zswash,mtpcal,mtphrb,mtpfrak]{mtpro2}
\usepackage{graphicx,multicol,multirow,enumitem,tabularx,mathimatika,gensymb,venndiagram,hhline,longtable,tkz-euclide,fontawesome5,eurosym,tcolorbox,tabularray}
\usepackage[explicit]{titlesec}
\tcbuselibrary{skins,theorems,breakable}
\newlist{rlist}{enumerate}{3}
\setlist[rlist]{itemsep=0mm,label=\roman*.}
\newlist{alist}{enumerate}{3}
\setlist[alist]{itemsep=0mm,label=\alph*.,leftmargin=4mm}
\newlist{balist}{enumerate}{3}
\setlist[balist]{itemsep=0mm,label=\bf\alph*.}
\newlist{Alist}{enumerate}{3}
\setlist[Alist]{itemsep=0mm,label=\Alph*.}
\newlist{bAlist}{enumerate}{3}
\setlist[bAlist]{itemsep=0mm,label=\bf\Alph*.}
\newlist{askhseis}{enumerate}{3}
\setlist[askhseis]{label={\Large\thesection}.\arabic*.}
\renewcommand{\textstigma}{\textsigma\texttau}
\newlist{thema}{enumerate}{3}
\setlist[thema]{label=\bf\large{ΘΕΜΑ \textcolor{black}{\Alph*}},itemsep=0mm,leftmargin=0cm,itemindent=18mm}
\newlist{erwthma}{enumerate}{3}
\setlist[erwthma]{label=\bf{\large{\textcolor{black}{\Alph{themai}.\arabic*}}},itemsep=0mm,leftmargin=0.8cm}



\newcommand{\kerkissans}[1]{{\fontfamily{maksf}\selectfont \textbf{#1}}}
\renewcommand{\textdexiakeraia}{}

\usepackage[
backend=biber,
style=alphabetic,
sorting=ynt
]{biblatex}

\DeclareTblrTemplate{caption}{nocaptemplate}{}
\DeclareTblrTemplate{capcont}{nocaptemplate}{}
\DeclareTblrTemplate{contfoot}{nocaptemplate}{}
\NewTblrTheme{mytabletheme}{
  \SetTblrTemplate{caption}{nocaptemplate}{}
  \SetTblrTemplate{capcont}{nocaptemplate}{}
  \SetTblrTemplate{contfoot}{nocaptemplate}{}
}

\NewTblrEnviron{mytblr}
\SetTblrStyle{firsthead}{font=\bfseries}
\SetTblrStyle{firstfoot}{fg=red2}
\SetTblrOuter[mytblr]{theme=mytabletheme}
\SetTblrInner[mytblr]{
rowspec={t{7mm}},columns = {c},
  width = 0.85\linewidth,
  row{odd} = {bg=red9,fg=black,ht=8mm},
 row{even} = {bg=red7,fg=black,ht=8mm},
hlines={white},vlines={white},
row{1} = {bg=red4, fg=white, font=\bfseries\fontfamily{maksf}},rowhead = 1,
  hline{2} = {.7mm}, % midrule  
}


\titleformat{\section}{\Large}{\kerkissans{\thesection}}{10pt}{\Large\kerkissans{#1}}

\setlength{\columnsep}{5mm}


\begin{document}
\twocolumn[{
\centering
\kerkissans{{\huge Η έννοια της συνάρτησης}\\\vspace{3mm} {\Large ΑΣΚΗΣΕΙΣ}\\\today}\vspace{5mm}}]
\section{Τιμές συνάρτησης}
\begin{askhseis}
\item Δίνεται η συνάρτηση $f(x)=x^2-4x+5$ με $x\in\mathbb{R}$. Υπολογίστε τις τιμές $f(2),f(-1),f(0)$ και $f(f(4))$.
\item Δίνεται η συνάρτηση $f(x)=\dfrac{x^2}{x-1}$ με $x\neq 1$. Υπολογίστε τις τιμές $f(2),f(0),f(3)$ και $f(f(2))$.
\item Για τη συνάρτηση $f(x)=\sqrt{5-x}$ με $x\leq 5$ να βρείτε τις ακόλουθες τιμές: $f(1), f(-4), f(3)$ και $f(7)$.
\item Δίνεται η συνάρτηση $f:\mathbb{R}\to\mathbb{R}$ με τύπο
\[ f(x)=\begin{cdcases}
3x-4 & ,x\geq1\\ x^2+2x & ,x<1
\end{cdcases} \]
Να βρεθούν οι τιμές $f(3),f(0),f(1)$ και $f(f(-3))$.
\item Έστω η συνάρτηση $f:\mathbb{R}\to\mathbb{R}$ με τύπο $f(x)=x^2-2x$. Να βρείτε τις τιμές του $x\in D_f$ έτσι ώστε $f(x)=8$.
\item Δίνεται η συνάρτηση $f:\mathbb{R}\to\mathbb{R}$ με τύπο
\[ f(x)=\begin{cdcases}
2x+1 & ,x>0\\ x^2-2x & ,x\leq 0
\end{cdcases} \]
Να βρεθούν οι τιμές του $x\in D_f$ τέτοιες ώστε να ισχύει $f(x)=3$.
\item Δίνεται η συνάρτηση $f:\mathbb{R}\to\mathbb{R}$ με τύπο
\[ f(x)=x^2-3x+2 \]
\begin{alist}
\item Υπολογίστε τις τιμές $f(-1),f(3)$ και $f(0)$.
\item Να βρεθούν οι τιμές του $x$ έτσι ώστε $f(x)=0$.
\item Να βρεθούν οι τιμές του $x$ έτσι ώστε $f(x)\geq 6$.
\end{alist}
\item Δίνεται η συνάρτηση $f(x)=\sqrt{2x+4}-\sqrt{7-x}$.
\begin{alist}
\item Να βρεθεί το πεδίο ορισμού της.
\item Να υπολογίσετε τις τιμές $f(-2)$ και $f(f(0))$.
\item Για ποιες τιμές του $x$ ισχύει $f(x)=0$?
\end{alist}
\end{askhseis}
\section{Πεδίο ορισμού}
\begin{askhseis}
\item Να βρεθεί το πεδίο ορισμού των παρακάτω συναρτήσεων.
\begin{multicols}{2}
\begin{alist}
\item $f(x)=\dfrac{2}{x}$
\item $f(x)=\dfrac{x}{x-3}$
\item $f(x)=\dfrac{1}{4-2x}$
\item $f(x)=\dfrac{2x+1}{3x+12}$
\end{alist}
\end{multicols}
\item Να βρεθεί το πεδίο ορισμού των παρακάτω συναρτήσεων.
\begin{multicols}{2}
\begin{alist}
\item $f(x)=\dfrac{1}{|x|}$
\item $f(x)=\dfrac{4-x}{|x|-2}$
\item $f(x)=\dfrac{x}{|x+1|-2}$
\item $f(x)=\dfrac{3}{|4-x|-3}$
\end{alist}
\end{multicols}
\item Να βρεθεί το πεδίο ορισμού των παρακάτω συναρτήσεων.
\begin{multicols}{2}
\begin{alist}
\item $f(x)=\dfrac{1}{x^2}$
\item $f(x)=\dfrac{x}{x^2-4}$
\item $f(x)=\dfrac{2-x}{18-2x^2}$
\item $f(x)=\dfrac{x^2}{x^2+3x}$
\end{alist}
\end{multicols}
\item Να βρεθεί το πεδίο ορισμού των παρακάτω συναρτήσεων.
\begin{multicols}{2}
\begin{alist}
\item $f(x)=\dfrac{1}{x^2+x}$
\item $f(x)=\dfrac{x^2+3}{x^2-3x-10}$
\item $f(x)=\dfrac{x+2}{x^2+4x-12}$
\item $f(x)=\dfrac{4}{x^2+x-6}$
\end{alist}
\end{multicols}
\item Να βρεθεί το πεδίο ορισμού των παρακάτω συναρτήσεων.
\begin{alist}
\item $f(x)=\dfrac{x}{x^3-1}$
\item $f(x)=\dfrac{2-3x}{x^3-7x+6}$
\item $f(x)=\dfrac{\hm{x}}{x^3-2x^2-5x+6}$
\item $f(x)=\dfrac{e^x}{x^3+x^2-12}$
\end{alist}
\item Βρείτε το πεδίο ορισμού των ακόλουθων συναρτήσεων.
\begin{multicols}{2}
\begin{alist}
\item $f(x)=\sqrt{x-1}$
\item $f(x)=\sqrt{4-x}$
\item $f(x)=\sqrt{2x-3}$
\item $f(x)=\sqrt{9-4x}$
\end{alist}
\end{multicols}
\item Βρείτε το πεδίο ορισμού των ακόλουθων συναρτήσεων.
\begin{multicols}{2}
\begin{alist}
\item $f(x)=\sqrt{|x|-2}$
\item $f(x)=\sqrt{3-|x|}$
\item $f(x)=\sqrt{|x+3|-5}$
\item $f(x)=\sqrt{5-|2x-1|}$
\item $f(x)=\sqrt{|3x|+2}$
\item $f(x)=\sqrt{|x|+x}$
\end{alist}
\end{multicols}
\item Βρείτε το πεδίο ορισμού των ακόλουθων συναρτήσεων.
\begin{multicols}{2}
\begin{alist}
\item $f(x)=\sqrt{x^2-1}$
\item $f(x)=\sqrt{4-x^2}$
\item $f(x)=\sqrt{x^2+3x}$
\item $f(x)=\sqrt{-x^2+2x}$
\item $f(x)=\sqrt{x^2-2x-8}$
\item $f(x)=\sqrt{-x^2+5x-4}$
\end{alist}
\end{multicols}
\item Να βρείτε το πεδίο ορισμού των ακόλουθων συναρτήσεων.
\begin{alist}
\item $f(x)=\sqrt{x^3+x-2}$
\item $f(x)=\sqrt{8-x^3}$
\item $f(x)=\sqrt{x^3-5x^2+2x+8}$
\item $f(x)=\sqrt{x^3-x}$
\end{alist}
\item Να βρείτε το πεδίο ορισμού των ακόλουθων συναρτήσεων.
\begin{multicols}{2}
\begin{alist}
\item $f(x)=\sqrt{\dfrac{1}{x-3}}$
\item $f(x)=\sqrt{\dfrac{x+2}{3-x}}$
\item $f(x)=\sqrt{\dfrac{x^2-1}{4-x}}$
\item $f(x)=\sqrt{\dfrac{2x+3}{x^2-2x}}$
\end{alist}
\end{multicols}
\item Να βρείτε το πεδίο ορισμού των ακόλουθων συναρτήσεων.
\begin{multicols}{2}
\begin{alist}
\item $f(x)=\sqrt{e^x-1}$
\item $f(x)=\sqrt{2^x-4}$
\item $f(x)=\sqrt{27-3^x}$
\item $f(x)=\sqrt{\left(\frac{1}{2}\right)^x-8}$
\end{alist}
\end{multicols}
\item Να βρείτε το πεδίο ορισμού των παρακάτω συναρτήσεων.
\begin{multicols}{2}
\begin{alist}
\item $f(x)=\ln{(x-1)}$
\item $f(x)=\ln{(3-x)}$
\item $f(x)=\ln{(2x+5)}$
\item $f(x)=\ln{(8-4x)}$
\end{alist}
\end{multicols}
\item Να βρείτε το πεδίο ορισμού των παρακάτω συναρτήσεων.
\begin{alist}
\item $f(x)=\ln{(x^2-1)}$
\item $f(x)=\ln{(4-x^2)}$
\item $f(x)=\ln{(x^2-2x-3)}$
\item $f(x)=\ln{(-x^2+4x+12)}$
\item $f(x)=\ln{(x^2+4x)}$
\item $f(x)=\ln{(-x^2+2x)}$
\end{alist}
\item Να βρείτε το πεδίο ορισμού των ακόλουθων συναρτήσεων.
\begin{alist}
\item $f(x)=\sqrt{\ln{x}}$
\item $f(x)=\sqrt{\ln{(x-1)-1}}$
\item $f(x)=\sqrt{\log{(2-x)}}$
\item $f(x)=\sqrt{-2\log{(x+1)}}$
\end{alist}
\item Να βρείτε το πεδίο ορισμού των παρακάτω συναρτήσεων.
\begin{alist}
\item $f(x)=\begin{cdcases}
x+2 & ,x>2\\ 3x^2 & ,x\leq 2
\end{cdcases}$
\item $f(x)=\begin{cdcases}
\sqrt{x-1} & ,1\leq x<3\\ 2x+4 & ,x\geq 3
\end{cdcases}$
\item $f(x)=\begin{cdcases}
\hm{\pi x} & ,x<2\\ e^x+1 & ,2<x\leq 4
\end{cdcases}$
\end{alist}
\end{askhseis}
\section{Παράμετροι}
\begin{askhseis}
\item Δίνεται η συνάρτηση $f(x)=x^2+(\lambda-1)x+3$ με $x\in\mathbb{R}$ και $\lambda\in\mathbb{R}$, για την οποία ισχύει $f(2)=3$. Να δείξετε ότι $\lambda=-1$.
\item Έστω η συνάρτηση $f(x)=x^2-2x-4$ με $x\in\mathbb{R}$ για την οποία ισχύει $f(\lambda)=\lambda$ με $\lambda>0$.
\begin{alist}
\item Να δείξετε ότι $\lambda=4$.
\item Να υπολογίσετε τις τιμές $f(3),f(-1)$ και $f(4)$.
\end{alist}
\item Δίνεται η συνάρτηση $f(x)=x^3+ax^2+3x-4$ με $x\in\mathbb{R}$ και $a\in\mathbb{R}$ για την οποία ισχύει $f(3)=5$.
\begin{alist}
\item Να δείξετε ότι $a=-3$.
\item Να λύσετε την εξίσωση $f(x)=-2$.
\item Να λύσετε την ανίσωση $f(x)<5$.
\end{alist}
\item Δίνεται η συνάρτηση $f(x)=\sqrt{x+a}-3$ με $a\in\mathbb{R}$ για την οποία ισχύει $f(4)=1$.
\begin{alist}
\item Να δείξετε ότι $a=5$.
\item Βρείτε το πεδίο ορισμού της $f$.
\item Να λυθεί η εξίσωση $f(x)=-x$.
\end{alist}
\end{askhseis}
\section{Προβλήματα}
\begin{askhseis}
\item Ένα ορθογώνιο παραλληλόγραμμο έχει πλευρές $x,y$ και σταθερή περίμετρο $40\ cm$.
\begin{alist}
\item Να εκφράσετε το πλάτος $y$ του ορθογωνίου ως συνάρτηση του μήκους $x$.
\item Να ορίσετε τη συνάρτηση $E$ που δίνει το εμβαδόν του ορθογωνίου. Ποιο είναι το πεδίο ορισμού της? 
\end{alist}
\item Ένα ορθογώνιο τρίγωνο έχει κάθετες πλευρές $x,y$ και σταθερό εμβαδόν $24\ cm^2$.
\begin{alist}
\item Να εκφράσετε την πλευρά $y$ ως συνάρτηση της πλευράς $x$ του τριγώνου.
\item Να κατασκευάσετε τη συνάρτηση $d(x)$ με την οποία δίνεται η υποτείνουσα του τριγώνου. 
\end{alist}
\item Σε ένα κατάστημα ηλεκτρονικών ειδών, ένας υπάλληλος παίρνει βασικό μισθό $900$\euro\ το μήνα. Για κάθε προϊόν που θα καταφέρει να πουλήσει σε έναν πελάτη, παίρνει ένα επιπλέον χρηματικό ποσό της τάξης του $7\%$ επί της τιμής του προϊόντος. Να ορίσετε τη συνάρτηση $f$ με την οποία δίνεται ο συνολικός μισθός του υπαλλήλου. Ποιο είναι το πεδίο ορισμού της?
\item 
\end{askhseis}
\section{Θεωρία}
\begin{askhseis}
\item Τι ονομάζεται πραγματική συνάρτηση με πεδίο ορισμού $A$?
\item Πως ορίζεται το σύνολο τιμών μιας πραγματικής συνάρτησης $f$?
\item Να χαρακτηρίσετε τις παρακάτω προτάσεις γράφοντας τη λέξη \textbf{Σωστό}, αν η πρόταση είναι σωστή, ή \textbf{Λάθος}, αν η πρόταση είναι λανθασμένη.
\begin{alist}
\item Η συνάρτηση $f(x)=\frac{1}{x}$ ορίζεται στο $\mathbb{R}$.
\item Μια συνάρτηση $f$ με πεδίο ορισμού το σύνολο $A=\{-2,-1,0,1\}$ μπορεί να έχει σύνολο τιμών το $B=\{-4,1,2,5,7\}$.
\item Μια συνάρτηση $f$ με πεδίο ορισμού το σύνολο $A=\{0,1,3,4,5\}$ μπορεί να έχει σύνολο τιμών το $B=\{1,2,3,4\}$.
\item Αν μια συνάρτηση $f:\mathbb{R}\to\mathbb{R}$ έχει σύνολο τιμών $f(D_f)=\{3\}$, τότε είναι σταθερή.
\item Έστω δύο ποσά $x,y$. Αν συνδέονται με τη σχέση
\[ y=\begin{cdcases}
3x+4 & ,x\geq 0\\ -2x+5 & ,x\leq 0
\end{cdcases} \]
τότε η σχέση αυτή ορίζει μια συνάρτηση με πεδίο ορισμού το $\mathbb{R}$.
\end{alist}
\end{askhseis}
\end{document}
