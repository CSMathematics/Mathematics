\documentclass[11pt,a4paper,twocolumn]{article}
\usepackage[english,greek]{babel}
\usepackage[utf8]{inputenc}
\usepackage{nimbusserif}
\usepackage[T1]{fontenc}
\usepackage[left=1.50cm, right=1.50cm, top=2.00cm, bottom=2.00cm]{geometry}
\usepackage{amsmath}
\let\myBbbk\Bbbk
\let\Bbbk\relax
\usepackage[amsbb,subscriptcorrection,zswash,mtpcal,mtphrb,mtpfrak]{mtpro2}
\usepackage{graphicx,multicol,multirow,enumitem,tabularx,mathimatika,gensymb,venndiagram,hhline,longtable,tkz-euclide,fontawesome5,eurosym,tcolorbox,tabularray}
\usepackage[explicit]{titlesec}
\tcbuselibrary{skins,theorems,breakable}
\newlist{rlist}{enumerate}{3}
\setlist[rlist]{itemsep=0mm,label=\roman*.}
\newlist{alist}{enumerate}{3}
\setlist[alist]{itemsep=0mm,label=\alph*.}
\newlist{balist}{enumerate}{3}
\setlist[balist]{itemsep=0mm,label=\bf\alph*.}
\newlist{Alist}{enumerate}{3}
\setlist[Alist]{itemsep=0mm,label=\Alph*.}
\newlist{bAlist}{enumerate}{3}
\setlist[bAlist]{itemsep=0mm,label=\bf\Alph*.}
\newlist{askhseis}{enumerate}{3}
\setlist[askhseis]{label={\Large\thesection}.\arabic*.}
\renewcommand{\textstigma}{\textsigma\texttau}
\newlist{thema}{enumerate}{3}
\setlist[thema]{label=\bf\large{ΘΕΜΑ \textcolor{black}{\Alph*}},itemsep=0mm,leftmargin=0cm,itemindent=18mm}
\newlist{erwthma}{enumerate}{3}
\setlist[erwthma]{label=\bf{\large{\textcolor{black}{\Alph{themai}.\arabic*}}},itemsep=0mm,leftmargin=0.8cm}

\newcommand{\kerkissans}[1]{{\fontfamily{maksf}\selectfont \textbf{#1}}}
\renewcommand{\textdexiakeraia}{}

\usepackage[
backend=biber,
style=alphabetic,
sorting=ynt
]{biblatex}

\DeclareTblrTemplate{caption}{nocaptemplate}{}
\DeclareTblrTemplate{capcont}{nocaptemplate}{}
\DeclareTblrTemplate{contfoot}{nocaptemplate}{}
\NewTblrTheme{mytabletheme}{
\SetTblrTemplate{caption}{nocaptemplate}{}
\SetTblrTemplate{capcont}{nocaptemplate}{}
\SetTblrTemplate{contfoot}{nocaptemplate}{}
}

\NewTblrEnviron{mytblr}
\SetTblrStyle{firsthead}{font=\bfseries}
\SetTblrStyle{firstfoot}{fg=red2}
\SetTblrOuter[mytblr]{theme=mytabletheme}
\SetTblrInner[mytblr]{
rowspec={t{7mm}},columns = {c},
width = 0.85\linewidth,
row{odd} = {bg=red9,fg=black,ht=8mm},
row{even} = {bg=red7,fg=black,ht=8mm},
hlines={white},vlines={white},
row{1} = {bg=red4, fg=white, font=\bfseries\fontfamily{maksf}},rowhead = 1,
hline{2} = {.7mm}, % midrule  
}
\newcounter{askhsh}
\setcounter{askhsh}{1}
\newcommand{\askhsh}{{\large\theaskhsh.}\ \addtocounter{askhsh}{1}}

\titleformat{\section}{\Large}{\kerkissans{\thesection}}{10pt}{\Large\kerkissans{#1}}

\setlength{\columnsep}{5mm}
\titleformat{\paragraph}
{\large}%
{}{0em}%
{\textcolor{red!80!black}{\faSquare\ \ \kerkissans{\bmath{#1}}}}
\setlength{\parindent}{0pt}

\newcommand{\eng}[1]{\selectlanguage{english}#1\selectlanguage{greek}}

\begin{document}
\twocolumn[{
\centering
\kerkissans{{\huge Παράγωγος συνάρτηση}\\\vspace{3mm} {\Large ΑΣΚΗΣΕΙΣ}}\vspace{5mm}}]
\paragraph{Απλές συναρτήσεις}
\askhsh Να βρείτε την παράγωγο των παρακάτω συναρτήσεων
\begin{multicols}{2}
\begin{alist}
\item $ f(x)=x^2 $
\item $ f(x)=x^3 $
\item $ f(x)=x^7 $
\item $ f(x)=\sqrt{x} $
\item $ f(x)=\dfrac{1}{x} $
\item $ f(x)=\hm{x} $
\item $ f(x)=\syn{x} $
\item $ f(x)=\ef{x} $
\item $ f(x)=e^x$
\item $ f(x)=\ln{x}$
\item $ f(x)=3^x$
\item $ f(x)=\left(\dfrac{1}{2}\right)^x$
\end{alist}
\end{multicols}
\askhsh Να βρείτε την παράγωγο των παρακάτω συναρτήσεων
\begin{multicols}{2}
\begin{alist}
\item $ f(x)=2x^3 $
\item $ f(x)=4\sqrt{x} $
\item $ f(x)=\dfrac{5}{x} $
\item $ f(x)=-2\hm{x} $
\item $ f(x)=-3\ef{x} $
\item $ f(x)=\sqrt{3}\syf{x} $
\item $ f(x)=5e^x$
\item $ f(x)=\sqrt{2}\ln{x}$
\item $ f(x)=3\cdot 2^x$
\end{alist}
\end{multicols}
\askhsh Να βρείτε την παράγωγο των παρακάτω συναρτήσεων
\begin{multicols}{2}
\begin{alist}
\item $ f(x)=\sqrt[3]{x} $
\item $ f(x)=\sqrt[4]{x^3} $
\item $ f(x)=\sqrt[5]{x^2} $
\item $ f(x)=\sqrt[4]{x^7} $
\item $ f(x)=x\cdot\sqrt{x} $
\item $ f(x)=x^2\cdot\sqrt[4]{x^3} $
\end{alist}
\end{multicols}
\askhsh Να βρείτε την πρώτη παράγωγο των παρακάτω συναρτήσεων
\begin{alist}
\begin{multicols}{2}
\item $ f(x)=x-1 $
\item $ f(x)=2x+3 $
\item $ f(x)=4-x $
\item $ f(x)=8-5x $
\item $ f(x)=\dfrac{x}{2}+3 $
\item $ f(x)=5-\dfrac{3x}{4} $
\end{multicols}
\end{alist}
\askhsh Να βρείτε την παράγωγο των παρακάτω συναρτήσεων
\begin{alist}
\item $ f(x)=x^2+4x+3 $
\item $ f(x)=x^2-5x $
\item $ f(x)=3x^2-2x+1 $
\item $ f(x)=x^3+4x^2-2x+5 $
\item $ f(x)=2x^3+5x^2-7 $
\item $ f(x)=x^4-5x^3+x^2+3 $
\end{alist}
\askhsh Να βρείτε την παράγωγο των παρακάτω συναρτήσεων
\begin{alist}
\item $ f(x)=\dfrac{x^2}{2}-x-2 $
\item $ f(x)=\dfrac{3x^2}{4}-\dfrac{2x}{3}+1 $
\item $ f(x)=\dfrac{x^3}{3}-\dfrac{x^2}{2}+4x $
\item $ f(x)=\dfrac{x^4}{4}+\dfrac{5x^3}{3}+4x-7 $
\end{alist}
\askhsh Να βρείτε την παράγωγο των παρακάτω συναρτήσεων
\begin{alist}
\item $ f(x)=\sqrt{x}+\dfrac{1}{x} $
\item $ f(x)=\hm{x}+\syn{x}+\sqrt{3} $
\item $ f(x)=x^3+\sqrt{x}+\syn{x} $
\item $ f(x)=\hm{x}-x^4+2 $
\item $ f(x)=\dfrac{1}{x}-\pi+\hm{\dfrac{\pi}{5}} $
\item $ f(x)=\syn{x}-\ef{x}+\dfrac{1}{2} $
\item $ f(x)=3e^x-2\hm{x}-4x^3$
\item $ f(x)=3\ln{x}-5\cdot 3^x+\ln{2}$
\item $ f(x)=3e^x-\dfrac{4}{x}$
\end{alist}
\askhsh Να υπολογίσετε την παράγωγο των παρακάτω συναρτήσεων
\begin{multicols}{2}
\begin{alist}
\item $ f(x)=x\cdot\syn{x} $
\item $ f(x)=x^2\cdot\hm{x} $
\item $ f(x)=4x^3\cdot\ln{x} $
\item $ f(x)=\hm{x}\cdot\syn{x} $
\item $ f(x)=xe^x$
\item $ f(x)=x^2\ln{x}$
\item $ f(x)=e^x\hm{x}$
\item $ f(x)=\left(x^2+2x\right)e^x$
\end{alist}
\end{multicols}
\askhsh Να βρείτε την παράγωγο των παρακάτω συναρτήσεων
\begin{multicols}{2}
\begin{alist}
\item $ f(x)=\dfrac{x}{x+1} $
\item $ f(x)=\dfrac{3x}{x-2} $
\item $ f(x)=\dfrac{x+4}{3-x} $
\item $ f(x)=\dfrac{2x-1}{x} $
\item $ f(x)=\dfrac{4-3x}{2-x} $
\item $ f(x)=\dfrac{3x+5}{2x-4} $
\end{alist}
\end{multicols}
\askhsh Να βρείτε την παράγωγο των παρακάτω συναρτήσεων
\begin{multicols}{2}
\begin{alist}
\item $ f(x)=\dfrac{x^2}{x-2} $
\item $ f(x)=\dfrac{x+3}{x^2} $
\item $ f(x)=\dfrac{3x-4}{x^2-1} $
\item $ f(x)=\dfrac{x^2}{x^2+1} $
\item $ f(x)=\dfrac{4-x}{x^3} $
\item $ f(x)=\dfrac{x^3}{x+1} $
\end{alist}
\end{multicols}
\askhsh Να βρείτε την παράγωγο των παρακάτω συναρτήσεων
\begin{multicols}{2}
\begin{alist}
\item $ f(x)=\dfrac{x^2-2x}{x+4} $
\item $ f(x)=\dfrac{x^2+3x+2}{x^2-4x} $
\item $ f(x)=\dfrac{9-x^2}{x^2+1} $
\item $ f(x)=\dfrac{x^3}{x^2+x-2} $
\item $ f(x)=\dfrac{1-2x}{x^3-1} $
\item $ f(x)=\dfrac{x}{x^3+8} $
\end{alist}
\end{multicols}
\askhsh Να βρείτε την παράγωγο των παρακάτω συναρτήσεων
\begin{multicols}{2}
\begin{alist}
\item $ f(x)=\dfrac{\hm{x}}{x} $
\item $ f(x)=\dfrac{\syn{x}}{x+1} $
\item $ f(x)=\dfrac{x}{\syn{x}} $
\item $ f(x)=\dfrac{\hm{x}}{\sqrt{x}} $
\item $ f(x)=\dfrac{\ln{x}}{x}$
\item $ f(x)=\dfrac{x^2}{e^x}$
\item $ f(x)=\dfrac{\ln{x}-x}{x^2}$
\item $ f(x)=\dfrac{x}{\ln{x}}$
\item $ f(x)=\dfrac{e^x}{\syn{x}}$
\end{alist}
\end{multicols}
\askhsh Να βρείτε την παράγωγο των παρακάτω συναρτήσεων
\begin{multicols}{2}
\begin{alist}
\item $ f(x)=\dfrac{\sqrt{x}}{2-x} $
\item $ f(x)=\dfrac{x}{\sqrt{x}-1} $
\item $ f(x)=\dfrac{1+\sqrt{x}}{x} $
\item $ f(x)=\dfrac{x^2-1}{\sqrt{x}} $
\end{alist}
\end{multicols}
\paragraph{Σύνθετες συναρτήσεις}
\askhsh Να βρείτε την παράγωγο των παρακάτω συναρτήσεων
\begin{multicols}{2}
\begin{alist}
\item $ f(x)=(x+3)^4 $
\item $ f(x)=(2x-5)^3 $
\item $ f(x)=(3-x)^7 $
\item $ f(x)=(1-4x)^5 $
\item $ f(x)=\left(\dfrac{x}{2}-1\right)^4 $
\item $ f(x)=\left(2-\dfrac{3x}{4}\right)^3 $
\end{alist}
\end{multicols}
\askhsh Να βρείτε την παράγωγο των παρακάτω συναρτήσεων
\begin{alist}
\item $ f(x)=\hm^3{x} $
\item $ f(x)=\syn^4{x} $
\item $ f(x)=\ln^5{x} $
\item $ f(x)=\ef^2{x} $
\item $ f(x)=(\hm{x}-\syn{x})^3 $
\item $ f(x)=(\ln{x}+x)^2 $
\end{alist}
\askhsh Να βρείτε την παράγωγο των παρακάτω συναρτήσεων
\begin{multicols}{2}
\begin{alist}
\item $ f(x)=\sqrt{2x+4} $
\item $ f(x)=\sqrt{3-x} $
\item $ f(x)=\sqrt{x^2-4} $
\item $ f(x)=\sqrt{3x+x^2} $
\item $ f(x)=\sqrt{x^3} $
\item $ f(x)=\sqrt{1-x^3} $
\end{alist}
\end{multicols}

\askhsh Να βρείτε την παράγωγο των παρακάτω συναρτήσεων
\begin{alist}
\item $ f(x)=\sqrt{\hm{x}}\ ,\ x\in\left(0,\frac{\pi}{2}\right) $
\item $ f(x)=\sqrt{\syn{x}}\ ,\ x\in\left(0,\frac{\pi}{2}\right) $
\item $ f(x)=\sqrt{\ln{x}+x}\ ,\ x\in\left(1,\frac{\pi}{2}\right) $
\item $ f(x)=\sqrt{e^x} $
\item $ f(x)=\sqrt{x^2+\frac{1}{x}} $
\item $ f(x)=\sqrt{2^x} $
\end{alist}
\askhsh Να βρείτε την παράγωγο των παρακάτω συναρτήσεων
\begin{multicols}{2}
\begin{alist}
\item $ f(x)=\dfrac{1}{x^2} $
\item $ f(x)=\dfrac{1}{2x-4} $
\item $ f(x)=\dfrac{1}{\hm{x}} $
\item $ f(x)=\dfrac{1}{\ef{x}} $
\item $ f(x)=\dfrac{2}{\sqrt{x}} $
\item $ f(x)=\dfrac{3}{\syn{x}} $
\end{alist}
\end{multicols}
\askhsh Να βρείτε την παράγωγο των παρακάτω συναρτήσεων.
\begin{multicols}{2}
\begin{alist}
\item $ f(x)=\hm{(2x)} $
\item $ f(x)=\syn{(3x+2)} $
\item $ f(x)=\hm{(x^2+x)} $
\item $ f(x)=\syn{(\sqrt{x})} $
\item $ f(x)=\hm{\frac{1}{x}} $
\item $ f(x)=\syn{(x^3)} $
\end{alist}
\end{multicols}
\askhsh Να βρείτε την παράγωγο των παρακάτω συναρτήσεων.
\begin{multicols}{2}
\begin{alist}
\item $ f(x)=\ln{(x^3)} $
\item $ f(x)=e^{3x+4} $
\item $ f(x)=\ln{(x^2-4x)} $
\item $ f(x)=2^{\hm{x}} $
\item $ f(x)=e^{\frac{1}{x}} $
\item $ f(x)=\ln{(\hm{x})} $
\item $ f(x)=e^{\syn{x}}$
\item $ f(x)=\ln{\left(e^x+x\right)}$
\end{alist}
\end{multicols}
\askhsh Να βρείτε την παράγωγο των παρακάτω συναρτήσεων.
\begin{alist}
\item $ f(x)=\sqrt{x^2-2x}+\sqrt{9-x^2} $
\item $ f(x)=\dfrac{1}{x^2-4}-\dfrac{3}{x^2-1} $
\item $ f(x)=\hm^3{x}-\syn^2{x} $
\item $ f(x)=(x^3+2x)^4+(1-2x)^5 $
\item $ f(x)=\ef{\left(3x+\pi\right)}+\syf{\left(x+\frac{\pi}{3}\right)} $
\item $ f(x)=\syn^2{x}+\syn{x^2} $
\item $ f(x)=\ln{\left(x^2+1\right)+\ln{\left(4-x^2\right)}}$
\item $ f(x)=e^{2x+1}+2^{2x}$
\end{alist}
\askhsh Να βρείτε την παράγωγο των παρακάτω συναρτήσεων.
\begin{alist}
\item $ f(x)=x\cdot\hm{(2x)} $
\item $ f(x)=\sqrt{x-2}\cdot\syn{x} $
\item $ f(x)=x^2\cdot\sqrt{2x-8} $
\item $ f(x)=3x\cdot\syn^2{x} $
\item $ f(x)=\sqrt{x}\cdot\ef{(2x)} $
\item $ f(x)=\hm^2{x}\cdot\hm{2x} $
\item $ f(x)=x^2\cdot\ln{\left(x^3\right)}$
\item $ f(x)=e^{3x}\cdot\syn{(2x)}$
\item $ f(x)=e^{x^2}\cdot\ln{\frac{x}{2}}$
\end{alist}
\askhsh Να βρείτε την παράγωγο των παρακάτω συναρτήσεων.
\begin{multicols}{2}
\begin{alist}
\item $ f(x)=\dfrac{\sqrt{x-1}}{x} $
\item $ f(x)=\dfrac{\hm{(3x)}}{x-4} $
\item $ f(x)=\dfrac{x+2}{\syn{(2x)}} $
\item $ f(x)=\dfrac{(2x-1)^4}{\sqrt{x-2}} $
\end{alist}
\end{multicols}
\askhsh Να βρείτε την παράγωγο των παρακάτω συναρτήσεων.
\begin{multicols}{2}
\begin{alist}
\item $ f(x)=\sqrt{\dfrac{2}{x+1}} $
\item $ f(x)=\sqrt{\dfrac{x+3}{x-2}} $
\item $ f(x)=\sqrt{\dfrac{e^x}{x}} $
\item $ f(x)=\sqrt{\dfrac{1}{\ln{x}}} $
\end{alist}
\end{multicols}

\askhsh Να βρείτε την παράγωγο των παρακάτω συναρτήσεων.
\begin{multicols}{2}
\begin{alist}
\item $ f(x)=\sqrt[3]{x-1} $
\item $ f(x)=\sqrt[3]{(x-2)^5}$
\item $ f(x)=\sqrt[5]{(x+3)^4}$
\item $ f(x)=\sqrt[3]{(1-2x)^4}$
\item $ f(x)=\sqrt[4]{\hm^3{x}}$
\item $ f(x)=\sqrt[3]{\ln^4{x}}$
\end{alist}
\end{multicols}

\paragraph{Συνδυαστικές ασκήσεις}
\askhsh Δίνεται η συνάρτηση $f:\mathbb{R}\to\mathbb{R}$ με τύπο
\[ f(x)=x^2-ax+3 \]
όπου $a\in\mathbb{R}$, της οποίας η γραφική παράσταση διέρχεται από το σημείο $A(-1,6)$.
\begin{alist}
\item Να δείξετε ότι $a=2$.
\item Να λύσετε την εξίσωση $f'(x)=0$.
\item Υπολογίστε το όριο
\[\lim_{x\to1}{\frac{f'(x)}{x^2-1}}\]
\end{alist}
\askhsh Δίνεται η συνάρτηση $f(x)=a\sqrt{x}+\beta$ για την οποία ισχύει $f(1)=5$ και $f'(4)=\frac{1}{2}$.
\begin{alist}
\item Να δείξετε ότι $a=2$ και $\beta=3$.
\item Να υπολογίσετε το όριο
\[\lim_{x\to1}{\frac{f(x)-5}{x-1}}\]
\end{alist}
\end{document}