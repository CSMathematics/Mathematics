\documentclass[11pt,a4paper,modern]{FFExercises}
\usepackage[english,greek]{babel}
\usepackage[utf8]{inputenc}
\usepackage{nimbusserif}
\usepackage[T1]{fontenc}
\usepackage{amsmath}
\let\myBbbk\Bbbk
\let\Bbbk\relax
\usepackage[amsbb,subscriptcorrection,zswash,mtpcal,mtphrb,mtpfrak]{mtpro2}
\usepackage{graphicx,multicol,multirow,enumitem,tabularx,mathimatika,gensymb,venndiagram,hhline,longtable,tkz-euclide,fontawesome5,eurosym,tcolorbox,tabularray,tikzpagenodes,relsize}
\definecolor{xrwma}{HTML}{3572c2}
\usetikzlibrary{calc}
\usetikzlibrary{positioning}
\tcbuselibrary{skins,theorems,breakable}
\renewcommand{\textstigma}{\textsigma\texttau}
\renewcommand{\textdexiakeraia}{}

\ekthetesdeiktes
\begin{document}

\titlos{Μαθηματικά προσανατολισμού}{Γ Λυκείου}{Συνέχεια συνάρτησης}
\paragraph{Ορισμός συνέχειας}
\askhsh Δίνεται συνάρτηση $f:\mathbb{R}\to\mathbb{R}$ με τύπο
\[ f(x)=\begin{cdcases}
\frac{\hm{x}}{x} & ,x<0\\ \frac{x+1}{\sqrt{x}-2} & ,x\geq 0
\end{cdcases} \]
\begin{alist}
\item Να εξετάσετε αν η $f$ είναι συνεχής στο $0$.
\item Να υπολογίσετε τα όρια
\begin{multicols}{2}
\begin{rlist}
\item $\lim\limits_{x\to-\infty}{f(x)}$
\item $\lim\limits_{x\to+\infty}{f(x)}$
\end{rlist}
\end{multicols}
\end{alist}
\paragraph{Εύρεση παραμέτρου}
\askhsh Δίνεται η συνάρτηση $f:\mathbb{R}\to\mathbb{R}$ με τύπο
\[ f(x)=\begin{cdcases}
\frac{x^2-4}{x+2} & ,x\neq -2\\ 2\kappa+3 & ,x\neq -2
\end{cdcases} \]
Να βρεθεί η τιμή του $\kappa\in\mathbb{R}$ ώστε η $f$ να είναι συνεχής.\\\\
\askhsh
Δίνεται συνάρτηση $f:\mathbb{R}\to\mathbb{R}$ με τύπο
\[ f(x)=\begin{cdcases}
x^2+ax+2 & ,x<1\\ 2ax+3 & ,x\geq 1
\end{cdcases} \]
όπου $a\in\mathbb{R}$. Να βρεθεί η τιμή της παραμέτρου $a$ έτσι ώστε η $f$ να είναι συνεχής.\\\\
\askhsh Να βρεθεί η τιμή της παραμέτρου $\lambda\in\mathbb{R}$ έτσι ώστε η συνάρτηση
\[ f(x)=\begin{cdcases}
(\lambda+2)e^x+4x+3 & ,x\leq 0\\ x^2+\hm{x}+4-\lambda & ,x>0
\end{cdcases} \]
να είναι συνεχής.\\\\


\end{document}
