\documentclass[11pt,a4paper,twocolumn]{article}
\usepackage[english,greek]{babel}
\usepackage[utf8]{inputenc}
\usepackage{nimbusserif}
\usepackage[T1]{fontenc}
\usepackage[left=1.50cm, right=1.50cm, top=2.00cm, bottom=2.00cm]{geometry}
\usepackage{amsmath}
\let\myBbbk\Bbbk
\let\Bbbk\relax
\usepackage[amsbb,subscriptcorrection,zswash,mtpcal,mtphrb,mtpfrak]{mtpro2}
\usepackage{graphicx,multicol,multirow,enumitem,tabularx,mathimatika,gensymb,venndiagram,hhline,longtable,tkz-euclide,fontawesome5,eurosym,tcolorbox,tabularray,mathtools}
\usepackage[explicit]{titlesec}
\tcbuselibrary{skins,theorems,breakable}
\newlist{rlist}{enumerate}{3}
\setlist[rlist]{itemsep=0mm,label=\roman*.}
\newlist{alist}{enumerate}{3}
\setlist[alist]{itemsep=0mm,label=\alph*.}
\newlist{balist}{enumerate}{3}
\setlist[balist]{itemsep=0mm,label=\bf\alph*.}
\newlist{Alist}{enumerate}{3}
\setlist[Alist]{itemsep=0mm,label=\Alph*.}
\newlist{bAlist}{enumerate}{3}
\setlist[bAlist]{itemsep=0mm,label=\bf\Alph*.}
\newlist{askhseis}{enumerate}{3}
\setlist[askhseis]{label={\Large\thesection}.\arabic*.}
\renewcommand{\textstigma}{\textsigma\texttau}
\newlist{thema}{enumerate}{3}
\setlist[thema]{label=\bf\large{ΘΕΜΑ \textcolor{black}{\Alph*}},itemsep=0mm,leftmargin=0cm,itemindent=18mm}
\newlist{erwthma}{enumerate}{3}
\setlist[erwthma]{label=\bf{\large{\textcolor{black}{\Alph{themai}.\arabic*}}},itemsep=0mm,leftmargin=0.8cm}

\newcommand{\kerkissans}[1]{{\fontfamily{maksf}\selectfont \textbf{#1}}}
\renewcommand{\textdexiakeraia}{}

\usepackage[
backend=biber,
style=alphabetic,
sorting=ynt
]{biblatex}

\DeclareTblrTemplate{caption}{nocaptemplate}{}
\DeclareTblrTemplate{capcont}{nocaptemplate}{}
\DeclareTblrTemplate{contfoot}{nocaptemplate}{}
\NewTblrTheme{mytabletheme}{
  \SetTblrTemplate{caption}{nocaptemplate}{}
  \SetTblrTemplate{capcont}{nocaptemplate}{}
  \SetTblrTemplate{contfoot}{nocaptemplate}{}
}

\NewTblrEnviron{mytblr}
\SetTblrStyle{firsthead}{font=\bfseries}
\SetTblrStyle{firstfoot}{fg=red2}
\SetTblrOuter[mytblr]{theme=mytabletheme}
\SetTblrInner[mytblr]{
rowspec={t{7mm}},columns = {c},
  width = 0.85\linewidth,
  row{odd} = {bg=red9,fg=black,ht=8mm},
 row{even} = {bg=red7,fg=black,ht=8mm},
hlines={white},vlines={white},
row{1} = {bg=red4, fg=white, font=\bfseries\fontfamily{maksf}},rowhead = 1,
  hline{2} = {.7mm}, % midrule  
}
\newcounter{askhsh}
\setcounter{askhsh}{1}
\newcommand{\askhsh}{{\large\theaskhsh.}\ \addtocounter{askhsh}{1}}

\titleformat{\section}{\Large}{\kerkissans{\thesection}}{10pt}{\Large\kerkissans{#1}}

\setlength{\columnsep}{5mm}
\titleformat{\paragraph}
{\large}%
{}{0em}%
{\textcolor{red!80!black}{\faSquare\ \ \kerkissans{\bmath{#1}}}}
\setlength{\parindent}{0pt}



\begin{document}
\twocolumn[{
\centering
\kerkissans{{\huge Ισότητα - Πράξεις - Σύνθεση συναρτήσεων}\\\vspace{3mm} {\Large ΑΣΚΗΣΕΙΣ}}\vspace{5mm}}]
\paragraph{Ισότητα συναρτήσεων}
\askhsh Σε καθένα από τα παρακάτω ερωτήματα, να εξετάσετε αν οι συναρτήσεις $f$ και $g$ είναι ίσες. Αν δεν είναι ίσες, να βρεθεί το σύνολο στο οποίο ισχύει $f(x)=g(x)$.
\begin{alist}
\item $f(x)=\ln{x^2}$ και $g(x)=2\ln{x}$
\item $f(x)=\sqrt[3]{x^2}$ και $g(x)=x^{\frac{2}{3}}$
\item $f(x)=\ln{(x-1)}+\ln{(x+1)}$ και\\ $g(x)=\ln{\left(x^2-1\right)}$
\item $f(x)=\sqrt{x+3}\cdot\sqrt{x-4}$ και\\ $g(x)=\sqrt{x^2-x-12}$
\item $\ln{(x+2)}-\ln{(2-x)}$ και $g(x)=\ln{\dfrac{2+x}{2-x}}$
\item $f(x)=\sqrt{3-x}\cdot\sqrt{x+1}$ και\\ $g(x)=\sqrt{-x^2+2x+3}$
\end{alist}
\askhsh Σε καθένα από τα παρακάτω ερωτήματα, να εξετάσετε αν οι συναρτήσεις $f$ και $g$ είναι ίσες. Αν δεν είναι ίσες, να βρεθεί το σύνολο στο οποίο ισχύει $f(x)=g(x)$.
\begin{alist}
\item $\dfrac{x^2-3x-10}{x-5}$ και $g(x)=x+2$
\item $\dfrac{4^x+1}{2^x}$ και $g(x)=2^x+2^{-x}$
\item $f(x)=\sqrt[3]{x^5}$ και $g(x)=x^{\frac{5}{3}}$
\item $f(x)=\ln{x^3}$ και $g(x)=3\ln{x}$
\item $f(x)=\dfrac{\sqrt{x^2-2x+1}}{x-1}$ και $g(x)=1$
\item $f(x)=|2x-1|+3$ και\\
$g(x)=\begin{cdcases}
2x+2 & ,x\in\left[\frac{1}{2},+\infty\right)\\
4-2x & ,x\in\left(-\infty,\frac{1}{2}\right)
\end{cdcases}$
\end{alist}
\askhsh Σε καθένα από τα παρακάτω ερωτήματα, να εξετάσετε αν οι συναρτήσεις $f$ και $g$ είναι ίσες. Αν δεν είναι ίσες, να βρεθεί το σύνολο στο οποίο ισχύει $f(x)=g(x)$.
\begin{alist}
\item $f(x)=\ef{x}\cdot\syn{x}$ και $g(x)=\hm{x}$
\item $f(x)=\dfrac{1}{x-1}+\dfrac{2}{x-3}$ και\\$g(x)=\dfrac{3x-5}{x^2-4x+3}$
\item $f(x)=\ln{\dfrac{x+3}{x-3}}$ και\\$g(x)=\ln{(x+3)}-\ln{(x-3)}$
\item $f(x)=\dfrac{1}{\sqrt{x}-1}-\dfrac{1}{\sqrt{x}+1}$ και\\
$g(x)=\dfrac{2}{x-1}$
\item $f(x)=\dfrac{2x-6}{x^2-7x+12}$ και $g(x)=\dfrac{2}{x-4}$
\end{alist}
\paragraph{Πράξεις συναρτήσεων}
\askhsh Σε καθένα από τα παρακάτω ερωτήματα δίνονται συναρτήσεις $f,g$. Να ορίσετε τις συναρτήσεις $f+g,f-g,f\cdot g$ και $\frac{f}{g}$.
\begin{alist}
\item $f(x)=x^2-4$ και $g(x)=x^2+x-2$
\item $f(x)=\dfrac{1}{x}$ και $g(x)=\dfrac{2}{x^2}$
\item $f(x)=\dfrac{x+1}{x-2}$ και $g(x)=\dfrac{3-x}{x-2}$
\item $f(x)=2^x$ και $g(x)=4^x$
\end{alist}
\askhsh Σε καθένα από τα παρακάτω ερωτήματα δίνονται συναρτήσεις $f,g$. Να ορίσετε τις συναρτήσεις $f+g,f-g,f\cdot g$ και $\frac{f}{g}$.
\begin{alist}
\item $f(x)=\dfrac{x}{x+1}$ και $g(x)=\dfrac{2}{x-1}$
\item $f(x)=\dfrac{1-2x}{x^2-4}$ και $g(x)=\dfrac{3x+4}{x+2}$
\end{alist}
\paragraph{Σύνθεση συναρτήσεων}
\askhsh Σε καθένα από τα ακόλουθα ερωτήματα, να ορίσετε τις συνθέσεις $g\circ f$ και $f\circ g$ των δοσμένων συναρτήσεων $f$ και $g$.
\begin{alist}
\item $f(x)=\sqrt{x}$ και $g(x)=x^2$
\item $f(x)=\ln{x}$ και $g(x)=e^x$
\item $f(x)=\sqrt[3]{x}$ και $g(x)=x^3$
\item $f(x)=\dfrac{1}{x}$ και $g(x)=\dfrac{1}{x-1}$
\item $f(x)=\dfrac{1}{x-2}$ και $g(x)=\sqrt{x}$
\item $f(x)=\dfrac{1}{x}$ και $g(x)=\ln{x}$
\end{alist}

\askhsh Σε καθένα από τα ακόλουθα ερωτήματα, να ορίσετε τις συνθέσεις $g\circ f$ και $f\circ g$ των δοσμένων συναρτήσεων $f$ και $g$.
\begin{alist}
\item $f(x)=\sqrt{x+2}$ και $g(x)=x^2-3$
\item $f(x)=\sqrt{4-x}$ και $g(x)=\sqrt{x}$
\item $f(x)=\ln{x}$ και $g(x)=\sqrt{x-1}$
\item $f(x)=\dfrac{x}{x-1}$ και $g(x)=e^x$
\item $f(x)=e^x-1$ και $g(x)=\ln{(x-2)}$
\item $f(x)=\dfrac{1}{e^x-1}$ και $g(x)=\ln{x}$
\end{alist}

\askhsh Δίνονται οι συναρτήσεις $f(x)=\sqrt{x-2}$ και $g(x)=\dfrac{1}{x-1}$.
\begin{alist}
\item Να ορίσετε τη συνάρτηση $g\circ f$.
\item Να εξετάσετε αν οι συναρτήσεις $g\circ f$ και $h(x)=\dfrac{\sqrt{x-2}+1}{x-3}$ είναι ίσες.
\item Να βρεθούν τα σημεία τομής της $C_{g\circ f}$ με την ευθεία $y=1$.
\end{alist}

\end{document}