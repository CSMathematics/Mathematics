\documentclass[11pt,a4paper,twocolumn]{article}
\usepackage[english,greek]{babel}
\usepackage[utf8]{inputenc}
\usepackage{nimbusserif}
\usepackage[T1]{fontenc}
\usepackage[left=1.50cm, right=1.50cm, top=2.00cm, bottom=2.00cm]{geometry}
\usepackage{amsmath}
\let\myBbbk\Bbbk
\let\Bbbk\relax
\usepackage[amsbb,subscriptcorrection,zswash,mtpcal,mtphrb,mtpfrak]{mtpro2}
\usepackage{graphicx,multicol,multirow,enumitem,tabularx,mathimatika,gensymb,venndiagram,hhline,longtable,tkz-euclide,fontawesome5,eurosym,tcolorbox,tabularray,mathtools}
\usepackage[explicit]{titlesec}
\tcbuselibrary{skins,theorems,breakable}
\newlist{rlist}{enumerate}{3}
\setlist[rlist]{itemsep=0mm,label=\roman*.}
\newlist{alist}{enumerate}{3}
\setlist[alist]{itemsep=0mm,label=\alph*.}
\newlist{balist}{enumerate}{3}
\setlist[balist]{itemsep=0mm,label=\bf\alph*.}
\newlist{Alist}{enumerate}{3}
\setlist[Alist]{itemsep=0mm,label=\Alph*.}
\newlist{bAlist}{enumerate}{3}
\setlist[bAlist]{itemsep=0mm,label=\bf\Alph*.}
\newlist{askhseis}{enumerate}{3}
\setlist[askhseis]{label={\Large\thesection}.\arabic*.}
\renewcommand{\textstigma}{\textsigma\texttau}
\newlist{thema}{enumerate}{3}
\setlist[thema]{label=\bf\large{ΘΕΜΑ \textcolor{black}{\Alph*}},itemsep=0mm,leftmargin=0cm,itemindent=18mm}
\newlist{erwthma}{enumerate}{3}
\setlist[erwthma]{label=\bf{\large{\textcolor{black}{\Alph{themai}.\arabic*}}},itemsep=0mm,leftmargin=0.8cm}

\newcommand{\kerkissans}[1]{{\fontfamily{maksf}\selectfont \textbf{#1}}}
\renewcommand{\textdexiakeraia}{}

\usepackage[
backend=biber,
style=alphabetic,
sorting=ynt
]{biblatex}

\DeclareTblrTemplate{caption}{nocaptemplate}{}
\DeclareTblrTemplate{capcont}{nocaptemplate}{}
\DeclareTblrTemplate{contfoot}{nocaptemplate}{}
\NewTblrTheme{mytabletheme}{
  \SetTblrTemplate{caption}{nocaptemplate}{}
  \SetTblrTemplate{capcont}{nocaptemplate}{}
  \SetTblrTemplate{contfoot}{nocaptemplate}{}
}

\NewTblrEnviron{mytblr}
\SetTblrStyle{firsthead}{font=\bfseries}
\SetTblrStyle{firstfoot}{fg=red2}
\SetTblrOuter[mytblr]{theme=mytabletheme}
\SetTblrInner[mytblr]{
rowspec={t{7mm}},columns = {c},
  width = 0.85\linewidth,
  row{odd} = {bg=red9,fg=black,ht=8mm},
 row{even} = {bg=red7,fg=black,ht=8mm},
hlines={white},vlines={white},
row{1} = {bg=red4, fg=white, font=\bfseries\fontfamily{maksf}},rowhead = 1,
  hline{2} = {.7mm}, % midrule  
}
\newcounter{askhsh}
\setcounter{askhsh}{1}
\newcommand{\askhsh}{\large\theaskhsh.\ \addtocounter{askhsh}{1}}

\titleformat{\section}{\Large}{\kerkissans{\thesection}}{10pt}{\Large\kerkissans{#1}}

\setlength{\columnsep}{5mm}
\titleformat{\paragraph}
{\normalfont}%
{}{0em}%
{\textcolor{red!80!black}{\faSquare\ \ \kerkissans{\bmath{#1}}}}
\setlength{\parindent}{0pt}

\begin{document}
\twocolumn[{
\centering
\kerkissans{{\huge Ορισμένο ολοκλήρωμα}\\\vspace{3mm} {\Large ΑΣΚΗΣΕΙΣ}}\vspace{5mm}}]
\section{Ιδιότητες ολοκληρωμάτων}
\askhsh
\section{Υπολογισμός απλών ολοκληρωμάτων}
%\paragraph{Απλά ολοκληρώματα}
\askhsh Υπολογίστε τα παρακάτω ολοκληρώματα.
\begin{multicols}{2}
\begin{alist}
\item $\displaystyle\int_{0}^{1}{x^2}\d x$
\item $\displaystyle\int_{1}^{4}{\left(3x-2\right)}\d x$
\item $\displaystyle\int_{1}^{3}{\left(x^2+2x\right)}\d x$
\item $\displaystyle\int_{-1}^{1}{x(x-1)}\d x$
\item $\displaystyle\int_{-2}^{2}{\!\!(x+1)(x-2)}\d x$
\item $\displaystyle\int_{-1}^{3}{x(2x-1)^2}\d x$
\end{alist}
\end{multicols}
\askhsh Υπολογίστε τα παρακάτω ολοκληρώματα.
\begin{multicols}{2}
\begin{alist}
\item $\displaystyle\int_{1}^{9}{\dfrac{1}{2\sqrt{x}}}\d x$
\item $\displaystyle\int_{1}^{4}{\sqrt{x}}\d x$
\item $\displaystyle\int_{1}^{8}{\sqrt[3]{x}}\d x$
\item $\displaystyle\int_{1}^{4}{x\sqrt{x}}\d x$
\item $\displaystyle\int_{0}^{1}{\sqrt[4]{x^3}}\d x$
\item $\displaystyle\int_{1}^{2}{x^2\sqrt[3]{x^2}}\d x$
\end{alist}
\end{multicols}
\askhsh Υπολογίστε τα παρακάτω ολοκληρώματα.
\begin{multicols}{2}
\begin{alist}
\item $\displaystyle\int_{1}^{2}{\dfrac{1}{x^2}}\d x$
\item $\displaystyle\int_{1}^{4}{\dfrac{1}{x^3}}\d x$
\item $\displaystyle\int_{1}^{e}{\dfrac{1}{x}}\d x$
\item $\displaystyle\int_{1}^{2}{\dfrac{3}{x^7}}\d x$
\item $\displaystyle\int_{1}^{3}{\dfrac{-2}{x^4}}\d x$
\item $\displaystyle\int_{1}^{8}{\dfrac{1}{\sqrt[3]{x}}}\d x$
\end{alist}
\end{multicols}
\askhsh Υπολογίστε τα παρακάτω ολοκληρώματα.
\begin{multicols}{2}
\begin{alist}
\item $\displaystyle\int_{0}^{\pi}{\hm{x}}\d x$
\item $\displaystyle\int_{0}^{\frac{\pi}{2}}{\syn{x}}\d x$
\item $\displaystyle\int_{\frac{\pi}{6}}^{\frac{\pi}{4}}{\dfrac{1}{\syn^2{x}}}\d x$
\item $\displaystyle\int_{\frac{\pi}{4}}^{\frac{\pi}{2}}{\dfrac{1}{\hm^2{x}}}\d x$
\item $\displaystyle\int_{0}^{\frac{\pi}{3}}{\sqrt{1-\hm^2{x}}}\d x$
\item $\displaystyle\int_{0}^{2\pi}{\!\!\!(2\hm{x}-3\syn{x})}\d x$
\end{alist}
\end{multicols}
\askhsh Υπολογίστε τα παρακάτω ολοκληρώματα.
\begin{multicols}{2}
\begin{alist}
\item $\displaystyle\int_{0}^{1}{e^x}\d x$
\item $\displaystyle\int_{0}^{\ln{3}}{4e^x}\d x$
\item $\displaystyle\int_{0}^{1}{2^x}\d x$
\item $\displaystyle\int_{-\log{2}}^{\log{2}}{10^x}\d x$
\item $\displaystyle\int_{-1}^{1}{\left(\frac{1}{2}\right)^x}\d x$
\item $\displaystyle\int_{-1}^{1}{\ln{5}\cdot5^x}\d x$
\end{alist}
\end{multicols}

\askhsh Υπολογίστε τα παρακάτω ολοκληρώματα.
\begin{multicols}{2}
\begin{alist}
\item $\displaystyle\int_{0}^{1}{e^x}\d x$
\item $\displaystyle\int_{0}^{\ln{3}}{4e^x}\d x$
\item $\displaystyle\int_{0}^{1}{2^x}\d x$
\item $\displaystyle\int_{-\log{2}}^{\log{2}}{10^x}\d x$
\item $\displaystyle\int_{-1}^{1}{\left(\frac{1}{2}\right)^x}\d x$
\item $\displaystyle\int_{-1}^{1}{\ln{5}\cdot5^x}\d x$
\end{alist}
\end{multicols}

\paragraph{Γινόμενο - πηλίκο}
\askhsh Να υπολογίσετε υα παρακάτω ολοκληρώματα
\begin{alist}
\item $\displaystyle\int_{-1}^{1}{(xe^x+e^x)\d x}$
\item $\displaystyle\int_{1}^{e}{(\ln{x}+1)\d x}$
\item $\displaystyle\int_{0}^{\frac{\pi}{2}}{(x\syn{x}+\hm{x})\d x}$
\item $\displaystyle\int_{-2}^{2}{()\d x}$
\end{alist}
\paragraph{Σύνθετες συναρτήσεις}
\askhsh
\section{Παραγοντική ολοκλήρωση}
\paragraph{Η συνάρτηση $P(x)\cdot e^{ax}$}
\askhsh Να υπολογίσετε τα παρακάτω ολοκληρώματα
\begin{multicols}{2}
\begin{alist}
\item $\displaystyle\int_{0}^{1}{xe^x\d x}$
\item $\displaystyle\int_{-1}^{2}{x^2e^x\d x} $
\item $\displaystyle\int_{0}^{2}{xe^{3x}\d x}$
\item $\displaystyle\int_{-2}^{2}{x^3e^{2x}\d x}$
\item $\displaystyle\int_{1}^{4}{\left(x^2+1\right)e^x \d x}$
\item $\displaystyle\int_{0}^{1}{\left(x^2-2x\right)e^{3x}\d x}$
\end{alist}
\end{multicols}
\paragraph{Η συνάρτηση $P(x)\cdot \hm{(ax)}$ ή $P(x)\cdot\syn{(ax)}$}
\askhsh Να υπολογίσετε τα παρακάτω ολοκληρώματα
\begin{multicols}{2}
\begin{alist}
\item $\displaystyle\int_{0}^{\pi}{x\hm{x}\d x}$
\item $\displaystyle\int_{-1}^{1}{x^2\syn{x}\d x} $
\item $\displaystyle\int_{0}^{\frac{\pi}{2}}{x\hm{3x}\d x}$
\item $\displaystyle\int_{-\frac{\pi}{3}}^{\frac{\pi}{3}}{4x^3\syn{2x}\d x}$
\item $\displaystyle\int_{1}^{4}{\left(x^2-x\right)\hm{x} \d x}$
\item $\displaystyle\int_{0}^{1}{\left(x^2+3\right)\syn{2x}\d x}$
\end{alist}
\end{multicols}
\paragraph{Η συνάρτηση $P(x)\cdot \ln{ax}$}
\paragraph{Η συνάρτηση $e^{ax}\cdot \hm{(\beta x)}$ ή $e^{ax}\cdot \syn{(\beta x)}$}
\section{Μέθοδος αντικατάστασης}
\section{Εμβαδόν επίπεδου χωρίου}
\paragraph{Χωρίο μεταξύ $C_f,x'x,x=a,x=\beta$}
\askhsh Δίνεται η συνάρτηση $f(x)=x^2-3x-4$.
\begin{alist}
\item Να λύσετε την εξίσωση $f(x)=0$.
\item Να βρείτε το εμβαδόν του χωρίου που βρίσκεται μεταξύ της γραφικής παράστασης της $f$, τον άξονα $x'x$ και τις ευθείες $x=-2$ και $x=5$.
\end{alist}
\askhsh Δίνεται η συνάρτηση $f(x)=(x-2)e^x$
\paragraph{Χωρίο μεταξύ $C_f,x'x$}
\paragraph{Χωρίο μεταξύ $C_f,x'x,x=a$}
\end{document}
