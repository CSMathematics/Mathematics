\documentclass[internet]{diag-pan-xelatex}
\usepackage[amsbb]{mtpro2}
\usepackage[no-math,cm-default]{fontspec}
\usepackage{xunicode}
\usepackage{xltxtra}
\usepackage{xgreek}
\defaultfontfeatures{Mapping=tex-text,Scale=MatchLowercase}
\setmainfont[Mapping=tex-text,Numbers=Lining,Scale=1.0,BoldFont={Minion Pro Bold}]{Minion Pro}
\newfontfamily\scfont{GFS Artemisia}
\font\icon = "Webdings"
\usepackage{mtpro2}
\usepackage{amsmath}
\xroma{cyan}
\usepackage{tikz}
\usepackage{tkz-euclide,tkz-fct}
\usepackage{wrapfig}
\usetkzobj{all}
\usepackage{calc}
\usepackage{systeme,regexpatch}
\usepackage[framemethod=TikZ]{mdframed}
\usepackage{adjustbox}
\makeatletter
% change the definition of \sysdelim not to store `\left` and `\right`
\def\sysdelim#1#2{\def\SYS@delim@left{#1}\def\SYS@delim@right{#2}}
\sysdelim\{. % reinitialize

% patch the internal command to use
% \LEFTRIGHT<left delim><right delim>{<system>}
% instead of \left<left delim<system>\right<right delim>
\regexpatchcmd\SYS@systeme@iii
  {\cB.\c{SYS@delim@left}(.*)\c{SYS@delim@right}\cE.}
  {\c{SYS@MT@LEFTRIGHT}\cB\{\1\cE\}}
  {}{}
\def\SYS@MT@LEFTRIGHT{%
  \expandafter\expandafter\expandafter\LEFTRIGHT
  \expandafter\SYS@delim@left\SYS@delim@right}
\makeatother
\newcommand{\synt}[2]{{\scriptsize \begin{matrix}
\times#1\\\\ \times#2
\end{matrix}}}

\usepackage{graphicx}
\usepackage{multicol}
\usepackage{multirow}
\usepackage{enumitem}
\usepackage{tabularx}
\usepackage[decimalsymbol=comma]{siunitx}

\begin{document}
\titlos{Μαθηματικά Κατεύθυνσης Γ΄ Λυκείου}{ΕΠΑΝΑΛΗΠΤΙΚΟ}
\begin{thema}
\item\mbox{}\\\vspace{-5mm} 
\begin{erwthma}
\item Έστω μια συνάρτηση $ f $ ορισμένη σ' ένα διάστημα $ \varDelta $ και $ x_0 $ εσωτερικό σημείο του $ \varDelta $. Αν η $ f $ παρουσιάζει τοπικό ακρότατο στο $ x_0 $ και είναι παραγωγίσιμη σ αυτό το σημείο τότε να δείξεις οτι $ f'(x_0)=0 $.\\
\monades{10}
\item Να δώσεις τη γεωμερτρική ερμηνεία του θεωρήματος Rolle.\monades{5}
\item Να χαρακτηρίσεις τις παρακάτω προτάσεις ως σωστές (Σ) ή λανθασμένες (Λ).
\begin{enumerate}[label=\bf\textcolor{black}{{\large \alph*.}}]
\item Αν $ z\in\mathbb{C}^*$, τότε οι εικόνες των $ z,-iz,\dfrac{z}{i^2} $ ανήκουν στον ίδιο κύκλο με κέντρο $ O(0,0) $.
\item $ \displaystyle{\lim_{x\rightarrow-\infty}{\int_{0}^{x}{e^t\mathrm{d}t}}=1} $
\item Μια συνάρτηση $ f:A\rightarrow\mathbb{R} $ είναι 1-1 όταν για οποιαδήποτε $ x_1,x_2\in A $ ισχύει η συνεπαγωγή \[ x_1=x_2\Rightarrow f(x_1)=f(x_2) \]
\item Αν οι συναρτήσεις $ f,g $ έχουν όριο στο $ x_0 $ και ισχύει $ f(x)<g(x) $ κοντά στο $ x_0 $ τότε \[ \lim_{x\rightarrow x_0}f(x)<\lim_{x\rightarrow x_0}g(x) \]
\item Αν $ f,g $ συναρτήσεις ορισμένες στο $ \mathbb{R}^* $ με $ f'(x)=g'(x) $ για $ x\in\mathbb{R} $ τότε υπάρχει σταθερά $ c $ τέτοια ώστε \[ f(x)=g(x)+c \]
\monades{10}
\end{enumerate}
\end{erwthma}
\item\mbox{}\\
\hspace{-2cm}Αν για τους μιγαδικούς $ z=a+i $ και $ u=\dfrac{\beta-i}{\beta+i} $, με $ a,\beta\in\mathbb{R} $, ισχύει \[ z+2u-\overline{z}=0 \textrm{ και } z+\overline{z}=2ui \]
τότε να αποδείξεις ότι : \begin{erwthma}
\item $ a=\beta=1 $ και $ z+u=1 $ \monades{7}
\item Η εικόνα του $ u $ κινείται στον μοναδιαίο κύκλο.\monades{4}
\item Η εικόνα του $ w=\PARENS{\dfrac{z}{u}}^3 $ ανήκει στη διχοτόμο της γωνίας του 1ου και 3ου τεταρτημορίου στο μιγαδικό επίπεδο.\monades{7}
\item $ \left|\dfrac{w^3}{z-w-2u}\right|=16 $\monades{7}
\end{erwthma}
\item\mbox{}\\
Έστω η παραγωγίσιμη συνάρτηση $ f:\mathbb{R}\rightarrow\mathbb{R} $ για την οποία ισχύει : \[ f^3(x)+f(x)=2e^x -2x\]
Να δείξεις ότι :
\begin{erwthma}
\item $ f(x)\geq1 $ για κάθε $ x\in\mathbb{R} $.\monades{9}
\item Η εξίσωση $ x+1-xe^{f(x)}=0 $ έχει μια τουλάχιστον ρίζα $ x_0 $ στο διάστημα $ (0,1) $.\monades{7}
\item Η συνάρτηση $ g(x)=\LEFTRIGHT\{.{\begin{aligned}
&\dfrac{f(x)-1}{x^2} & x\neq0\\
&\;\;\;\;\;\;\dfrac{1}{4} & x=0
\end{aligned}} \;\;\;$ είναι συνεχής.\monades{9}
\end{erwthma}
\item\mbox{}\\Δίνεται η συνάρτηση $ f:\mathbb{R}\rightarrow\mathbb{R} $ για την οποία ισχύουν :
\begin{multicols}{2}
\begin{enumerate}[itemsep=0mm,label=\roman*.]
\item Είναι δύο φορές παραγωγίσιμη στο $ \mathbb{R} $.\\
\item $ f(x)>0 $ για κάθε $ x\in\mathbb{R} $.\\
\item $ \displaystyle{\int_{0}^{a}f(t)\mathrm{d}t=4} $\\
\item  $ g(x)=\displaystyle{\int_{0}^{x}f(t)\mathrm{d}t\cdot\int_{x}^{a}f(t)\mathrm{d}t\;,\;x\in\mathbb{R}} $.
\end{enumerate}
\end{multicols}
Να δείξεις οτι :
\begin{erwthma}
\item $ g(x)=-\PARENS{\int_{0}^{x}f(t)\mathrm{d}t-2}^2\;,\;x\in\mathbb{R}. $\monades{7}
\item Υπάρχει ένα τουλάχιστον $ \xi\in(0,a) $ , ώστε $ \displaystyle\int_{0}^{\xi}f(t)\mathrm{d}t=2 $ και $ \displaystyle\int_{a}^{\xi}f(t)\mathrm{d}t=-2 $\monades{8}
\item Η συνάρτηση $ g $ παρουσιάζει μέγιστο στο σημείο $ \xi $.\monades{5}
\item $ \displaystyle\lim_{x\rightarrow\xi}\dfrac{4-g(x)}{\left(x-\xi\right)^2}=f^2\left(\xi\right) $\monades{5}
\end{erwthma}
\end{thema}
\end{document}