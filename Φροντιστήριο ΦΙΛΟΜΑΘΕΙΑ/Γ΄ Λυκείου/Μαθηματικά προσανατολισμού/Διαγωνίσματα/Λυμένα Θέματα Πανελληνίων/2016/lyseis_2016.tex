\documentclass[internet]{diag-pan-xelatex}
\usepackage[amsbb]{mtpro2}
\usepackage[no-math,cm-default]{fontspec}
\usepackage{xunicode}
\usepackage{xgreek}
\usepackage{fontawesome}
\newfontfamily{\FA}{fontawesome.otf}
\defaultfontfeatures{Mapping=tex-text,Scale=MatchLowercase}
\setmainfont[Mapping=tex-text,Numbers=Lining,Scale=1.0,BoldFont={Minion Pro Bold}]{Minion Pro}
\newfontfamily\scfont{GFS Artemisia}
\font\mymath = "MyMathSymbols" at 16pt
\font\icon = "Webdings"
\usepackage{mtpro2}
\xroma{green!50!yellow!58!black}
\usepackage{tikz}
\usepackage{tkz-euclide,tkz-fct,pgfplots,tkz-tab}
\usepackage{wrapfig}
\usetkzobj{all}
\usepackage{calc}
\usepackage{systeme,regexpatch,amsmath}
\usepackage[framemethod=TikZ]{mdframed}
\usepackage{adjustbox,mathimatika}
\tkzSetUpPoint[size=7,fill=white]
\newlist{rlist}{enumerate}{3}
\setlist[rlist]{itemsep=0mm,label=\roman*.}
\usepackage{graphicx}
\usepackage{multicol}
\usepackage{multirow}
\usepackage{enumitem}
\usepackage{tabularx}
\usepackage{pgfplots,mathimatika}
\usepackage{etoolbox}
\makeatletter
\renewrobustcmd{\anw@true}{\let\ifanw@\iffalse}
\renewrobustcmd{\anw@false}{\let\ifanw@\iffalse}\anw@false
\newrobustcmd{\noanw@true}{\let\ifnoanw@\iffalse}
\newrobustcmd{\noanw@false}{\let\ifnoanw@\iffalse}\noanw@false
\renewrobustcmd{\anw@print}{\ifanw@\ifnoanw@\else\numer@lsign\fi\fi}
\makeatother





\begin{document}
\titlos{ΜΑΘΗΜΑΤΙΚΑ ΚΑΤΕΥΘΥΝΣΗΣ Γ΄ ΛΥΚΕΙΟΥ}{ΠΑΝΕΛΛΗΝΙΕΣ 2016 - ΑΠΑΝΤΗΣΕΙΣ}
\begin{thema}
\item \mbox{}\\\vspace{-5mm}
\begin{erwthma}
\item Σχολικό βιβλίο σελ. 262
\item Σχολικό βιβλίο σελ. 142
\item Σχολικό βιβλίο σελ. 246
\item \begin{multicols}{5}
\begin{rlist}
\item Λ
\item Σ
\item Λ
\item Σ
\item Σ
\end{rlist}
\end{multicols}
\end{erwthma}
\item \mbox{}\\\vspace{-5mm}
\begin{erwthma}
\item Υπολογίζουμε την παράγωγο της συνάρτησης $ f $ η οποία θα είναι 
\[ f'(x)=\frac{\left(x^2\right)'\left( x^2+1\right)-x^2\left( x^2+1\right)' }{\left( x^2+1\right)^2}=\frac{2x\left( x^2+1\right)-2x^3}{\left( x^2+1\right)^2}=\frac{2x}{\left( x^2+1\right)^2} \]
Εξετάζουμε που μηδενίζεται η παράγωγος οπότε θα έχουμε \[ f'(x)=0\Rightarrow \frac{2x}{\left( x^2+1\right)^2}=0\Rightarrow 2x=0\Rightarrow x=0 \]
Από τον πίνακα προσήμων της παραγώγου και μονοτονίας της συνάρτησης $ f $ παρατηρούμε ότι
\begin{center}
\begin{tikzpicture}
\tkzTabInit[espcl=2.5,color,colorC = \xrwma!40,
colorL = \xrwma!20,
colorV = \xrwma!40]
{$x$ / .8 ,$f'(x)$ /.8 ,$f(x)$/.8}%
{$-\infty$ , $0$ , $+\infty$ }%
\tkzTabLine{ , -, z ,+, }
\tkzTabVar{ +/ , -/ ,+/ }
\end{tikzpicture}
\end{center}
η συνάρτηση είναι γνησίως φθίνουσα στο διάστημα $ (-\infty,0] $ και γνησίως αύξουσα στο διάστημα $ [0,+\infty) $ ενώ παρουσιάζει ολικό ελάχιστο στη θέση $ x=0 $ το οποίο είναι το $ f(0)=0 $.\\
\item Υπολογίζουμε τη δεύτερη παράγωγο της συνάρτησης $ f $ οπότε θα έχουμε
\[ f''(x)=\frac{\left(2x\right)'\left( x^2+1\right)^2-2x\left[ \left( x^2+1\right)^2\right] ' }{\left( x^2+1\right)^4}=\frac{2\left( x^2+1\right)^2-2x\cdot2\left( x^2+1\right)\cdot2x }{\left( x^2+1\right)^4}=\frac{2\left(1-3x^2\right) }{\left( x^2+1\right)^2}\]
Θέτοντας τη δεύτερη παράγωγο της $ f $ ίση με το $ 0 $ θα προκύψει
\[ f''(x)=0\Rightarrow\frac{2\left(1-3x^2\right) }{\left( x^2+1\right)^2}=0\Rightarrow 2\left(1-3x^2\right) =0\Rightarrow x=\pm\frac{\sqrt{3}}{3} \]
Στον παρακάτω πίνακα φαίνονται τα πρόσημα της δεύτερης παραγώγου και η κυρτότητα της συνάρτησης $ f $. Βλέπουμε λοιπόν ότι
\begin{center}
\begin{tikzpicture}
\tkzTabInit[espcl=2,color,colorC = \xrwma!40,
colorL = \xrwma!20,
colorV = \xrwma!40]
{$x$ / .8 ,$f''(x)$ /.8 ,$f(x)$/.8}%
{$-\infty$ , $\frac{\sqrt{3}}{3}$, $ \frac{\sqrt{3}}{3} $ , $+\infty$ }%
\tkzTabLine{ , -, z, + ,z ,-, }
\tkzTabLine{ , \koilh ,\textrm{Σ.Κ.}, \kyrth,\textrm{Σ.Κ.},\koilh }
\end{tikzpicture}
\end{center}
η συνάρτηση είναι κυρτή στο διάστημα $ \left[-\frac{\sqrt{3}}{3},\frac{\sqrt{3}}{3}\right]  $ ενώ είναι κοίλη στα διαστήματα $ \left(-\infty,-\frac{\sqrt{3}}{3} \right]  $ και $ \left[\frac{\sqrt{3}}{3},+\infty \right)  $. Επίσης $ f\left( \frac{\sqrt{3}}{3}\right)=f\left( -\frac{\sqrt{3}}{3}\right)=\frac{1}{4} $ άρα σημεία καμπής της γραφικής παράστασης της συνάρτησης είναι τα $ A\left( \frac{\sqrt{3}}{3},\frac{1}{4} \right)  $ και $ B\left( -\frac{\sqrt{3}}{3},\frac{1}{4} \right)  $.\\
\item Επειδή το πεδίο ορισμού είναι το $ \mathbb{R} $ τότε δεν εξετάζουμε την ύπαρξη κατακόρυφης ασύμπτωτης. Για τις οριζόντιες ασύμπτωτες θα έχουμε
\begin{itemize}
\item $ \displaystyle{\lim_{x\to+\infty}{f(x)}=\lim_{x\to+\infty}{\frac{x^2}{x^2+1}}=\lim_{x\to+\infty}{\frac{x^2}{x^2}}=1} $.
\item Ομοίως στο $ -\infty $ θα ισχύει $ \displaystyle{\lim_{x\to+\infty}{f(x)}=1} $.
\end{itemize}
Έτσι η οριζόντια ευθεία $ y=1 $ είναι η οριζόντια ασύμπτωτη της $ C_f $ και στο $ -\infty $ και στο $ +\infty $. Επίσης αφού η $ C_f $ έχει οριζόντιες ασύμπτωτες τότε δεν έχει πλάγιες.\\
\item Σύμφωνα με τα προηγούμενα ερωτήματα για τη χάραξη της γραφικής παράστασης της συνάρτησης $ f $ λαμβάνουμε υπόψιν μας τη μονοτονία και την κυρτότητά της αφού αρχικά σχεδιάσουμε τα σημεία καμπής, το σημείο στο οποίο παρουσιάζει ολοκό ελάχιστο και την οριζόντια ασύμπτωτη $ y=1 $. Η γραφική παράσταση φαίνεται στο παρακάτω σχήμα.
\begin{center}
\begin{tikzpicture}
\begin{axis}[aks_on,belh ar,xlabel={\footnotesize$x$},ylabel={\footnotesize$y$},ymin=-.2,ymax=1.7,xmin=-5,xmax=5,x=1cm,y=1.5cm]
\addplot[domain=-4.5:4.5,grafikh parastash] {x^2/(x^2+1)};
\addplot[-] {1};
\coordinate (A) at (axis cs:.577,.25){};
\coordinate (B) at (axis cs:-.577,.25){};
\draw[dashed] (axis cs:-.577,0)--(B)--(A)--(axis cs:.577,0);
\end{axis}
\tkzDrawPoints[fill=\xrwma,color=\xrwma](A,B)
\tkzLabelPoint[right](A){$A$}
\tkzLabelPoint[left](B){$B$}
\node at (2,2){\footnotesize$y=1$};
\node at (2.5,1){\footnotesize$f(x)=\frac{x^2}{x^2+1}$};
\end{tikzpicture}
\end{center}
\end{erwthma}
\item \mbox{}\\\vspace{-5mm}
\begin{erwthma}
\item\label{g1} Θέτουμε $ g(x)=e^{x^2}-x^2-1 $ με $ x\in\mathbb{R} $. Παρατηρούμε ότι μια προφανής ρίζα της συνάρτησης $ g $ είναι το $ 0 $ αφού $ g(0)=e^{0^2}-0^2-1=0 $. Θα εξετάσουμε αν είναι και μοναδική. Έχουμε 
\[ g'(x)=\left( e^{x^2}-x^2-1\right)'=2xe^{x^2}-2x=2x\left( e^{x^2}-1\right) \]
Έστω λοιπόν $ g'(x)=0 $ οπότε προκύπτει
\[ g'(x)=0\Rightarrow 2x\left( e^{x^2}-1\right)=0\Rightarrow\ccases{x=0\\e^{x^2}-1=0\Rightarrow x=0} \]
άρα η $ g' $ μηδενίζεται για $ x=0 $. Από τον παρακάτω πίνακα προσήμων της $ g' $ και μονοτονίας της συνάρτησης $ g $ φαίνεται ότι
\begin{center}
\begin{tikzpicture}
\tkzTabInit[espcl=2.5,color,colorC = \xrwma!40,
colorL = \xrwma!20,
colorV = \xrwma!40]
{$x$ / .8 ,$g'(x)$ /.8 ,$g(x)$/.8}%
{$-\infty$ , $0$ , $+\infty$ }%
\tkzTabLine{ , -, z ,+, }
\tkzTabVar{ +/ , -/ ,+/ }
\end{tikzpicture}
\end{center}
η συνάρτηση $ g $ παρουσιάζει ελάχιστη τιμή την $ g(0)=0 $ ενώ εκατέρωθεν του $ 0 $ είναι γνησίως μονότονη. Αυτό σημαίνει ότι $ g(x)\geq0 $ με το $ = $ να ισχύει μόνο στο $ 0 $. Άρα η $ x=0 $ είναι μοναδική ρίζα της συνάρτησης.\\
\item Από την ισότητα $ f^2(x)=\left( e^{x^2}-x^2-1\right)^2  $ παίρνουμε $ |f(x)|=\left| e^{x^2}-x^2-1\right| $. Γνωρίζουμε ότι η συνάρτηση $ f $ δε μηδενίζεται και ως συνεχής θα διατηρεί το πρόσημό της. Ξεχωρίζουμε τις εξής περιπτώσεις
\begin{enumerate}[label=\roman*.]
\item Αν $ x\in(-\infty,0) $ και
\begin{itemize}
\item $ f(x)>0 $ τότε από την τελευταία ισότητα θα προκύψει : $ |f(x)|=\left| e^{x^2}-x^2-1\right|\Rightarrow f(x)=e^{x^2}-x^2-1 $
\item $ f(x)<0 $ τότε ο τύπος της συνάρτησης θα είναι : $ |f(x)|=\left| e^{x^2}-x^2-1\right|\Rightarrow f(x)=-\left( e^{x^2}-x^2-1\right)=-e^{x^2}+x^2+1 $.
\end{itemize}
\item Ομοίως αν $ x\in(0,+\infty) $ και
\begin{itemize}
\item $ f(x)>0 $ τότε ομοίως με προηγουμένως θα έχουμε : $ |f(x)|=\left| e^{x^2}-x^2-1\right|\Rightarrow f(x)=e^{x^2}-x^2-1 $
\item $ f(x)<0 $ τότε θα ισχύει : $ |f(x)|=\left| e^{x^2}-x^2-1\right|\Rightarrow f(x)=-\left( e^{x^2}-x^2-1\right)=-e^{x^2}+x^2+1 $.
\end{itemize}
\end{enumerate}
Συνδυάζοντας τις παραπάνω περιπτώσεις προκύπτουν οι ακόλουθοι τέσσερις τύποι για τη συνάρτηση $ f $
\begin{multicols}{2}
\begin{enumerate}
\item $ f(x)=e^{x^2}-x^2-1\ ,\ x\in\mathbb{R} $
\item $ f(x)=\ccases{e^{x^2}-x^2-1 &,\  x\geq0\\-e^{x^2}+x^2+1 &,\  x<0} $
\item $ f(x)=-e^{x^2}+x^2+1\ ,\ x\in\mathbb{R} $
\item $ f(x)=\ccases{e^{x^2}-x^2-1 &,\  x<0\\-e^{x^2}+x^2+1 &,\  x\geq0} $
\end{enumerate}
\end{multicols}
\item Από το ερώτημα \ref{g1} γνωρίζουμε ότι $ f'(x)=2x\left( e^{x^2}-1\right) $ άρα για τη δεύτερη παράγωγο της συνάρτησης $ f(x)=e^{x^2}-x^2-1 $ θα έχουμε
\[ f''(x)=\left[2x\left( e^{x^2}-1\right)\right]'=2\left( e^{x^2}-1\right)+2x\cdot2x e^{x^2}=2\left( e^{x^2}-1\right)+4x^2e^{x^2} \]
Παρατηρούμε ότι ισχύει $ e^{x^2}\geq1 $ για κάθε $ x\in\mathbb{R} $ με την ισότητα να ισχύει μόνο για $ x=0 $. Επίσης $ 4x^2e^{x^2}\geq0 $ για κάθε $ x\in\mathbb{R} $. Επομένως η δεύτερη παράγωγος μηδενίζεται σε μεμονωμένα σημεία ενώ στο υπόλοιπο πδίο ορισμού διατηρεί το πρόσημό της με $ f''(x)>0 $. Άρα η $ f $ είναι κυρτή σε όλο το $ \mathbb{R} $.\\
\item Θεωρούμε τη συνάρτηση $ h(x)=f(x+3)-f(x) $. Η συνάρτηση $ h $ είναι παραγωγίσιμη με $ h'(x)=f'(x+3)-f'(x) $. Γνωρίζουμε επίσης ότι η συνάρτηση $ f $ είναι κυρτή οπότε η $ f' $ θα είναι αύξουσα. Έτσι θα ισχύει :
\[ x<x+3\xRightarrow{f\nearrow} f'(x)<f'(x+3)\Rightarrow f'(x+3)-f'(x)>0\Rightarrow h'(x)>0 \]
Αυτό σημαίνει ότι η συνάρτηση $ h $ είναι συνάρτηση $ 1-1 $ άρα για την αρχική εξίσωση θα ισχύει
\[ f(|\hm{x}+3|)-f(|\hm{x}|)=f(x+3)-f(x)\Rightarrow h(|\hm{x}|)=h(x)\xRightarrow{h\ 1-1}|\hm{x}|=x\Rightarrow x=0 \]
αφού είναι γνωστό ότι για κάθε $ x\geq0 $ ισχύει $ |\hm{x}|\leq x $ με την ισότητα να ισχύει μόνο για $ x=0 $.
\end{erwthma}
\newpage
\noindent
\item \mbox{}\\\vspace{-5mm}
\begin{erwthma}
\item\label{d1} Από τη δοσμένη σχέση $ \int_{0}^{\pi}{\left(f(x)+f''(x)\right)\cdot\hm{x}dx }=\pi $ μεδιάσπαση και παραγοντική ολοκλήρωση προκύπτει ότι
\begin{gather*}
\int_{0}^{\pi}{\left(f(x)+f''(x)\right)\cdot\hm{x}dx }=\pi
\Rightarrow\int_{0}^{\pi}{f(x)\cdot\hm{x}dx}+\int_{0}^{\pi}{f''(x)\cdot\hm{x}dx}=\pi\Rightarrow\\ \int_{0}^{\pi}{f(x)\cdot(-\syn{x})'dx}+\int_{0}^{\pi}{f''(x)\cdot\hm{x}dx}=\pi\Rightarrow\\
-\left[f(x)\cdot\syn{x}\right]_{0}^{\pi}+\int_{0}^{\pi}{f'(x)\cdot\syn{x}dx}+\left[f(x)\cdot\hm{x} \right]_{0}^{\pi}-\int_{0}^{\pi}{f'(x)\cdot\syn{x}dx}=\pi\Rightarrow\\
-\left[f(x)\cdot\syn{x}\right]_{0}^{\pi}+\left[f(x)\cdot\hm{x} \right]_{0}^{\pi}=\pi \Rightarrow\\ -f(\pi)\cdot\syn{\pi}+f(0)\cdot\syn{0}+f(\pi)\cdot\hm{\pi}-f(0)\cdot\hm{0}=\pi\Rightarrow f(\pi)+f(0)=\pi
\end{gather*}
Θέτουμε $ g(x)=\frac{f(x)}{\hm{x}}\Rightarrow f(x)=g(x)\cdot\hm{x} $. Επειδή η συνάρτηση $ f $ είναι συνεχής τότε υπολογίζοντας το όριο της στο $ 0 $ θα ισχύει
\[ \lim_{x\to0}f(x)=\lim_{x\to0}\left( g(x)\cdot\hm{x}\right)=1\cdot0=0 \] οπότε προκύπτει $ f(0)=0 $ και έτσι $ f(\pi)=\pi $. Επιπλεον από το όριο που μας δίνει η υπόθεση κατασκευάζοντας το $ f'(0) $ θα έχουμε 
\begin{gather*}
\lim_{x\to0}{\frac{f(x)}{\hm{x}}}=1\Rightarrow \lim_{x\to0}{\frac{f(x)\cdot x}{x\cdot\hm{x}}}=1\Rightarrow \lim_{x\to0}{\frac{f(x)-f(0)}{x-0}\cdot\frac{x}{\hm{x}}}=1\\\Rightarrow f'(0)\lim_{x\to0}{\frac{x}{\hm{x}}}=1\Rightarrow f'(0)=1
\end{gather*}
\item α) Έστω ότι η συνάρτηση $ f $ παρουσιάζει ακρότατο σε ένα τυχαίο σημείο $ x_0\in\mathbb{R} $. Από το θεώρημα του Fermat λοιπόν προκύπτει ότι $ f'(x_0)=0 $ και παραγωγίζοντας τη συναρτησιακή σχέση της υπόθεσης προκύπτει
\[ \left(e^{f(x)}+x=f(f(x))+e^x\right)'=e^{f(x)}\cdot f'(x)+1=f'(f(x))\cdot f'(x)+e^x  \]
Θέτοντας στην τελευταία σχέση όπου $ x=x_0 $ έχουμε 
\[ e^{f(x_0)}\cdot f'(x_0)+1=f'(f(x_0))\cdot f'(x_0)+e^{x_0}\Rightarrow e^{x_0}=1\Rightarrow x_0=0 \]
Έτσι παίρνουμε $ f'(0)=0 $ που είναι άτοπο διότι από το ερώτημα \ref{d1} λεχουμε $ f'(0)=1 $. Επομένως η $ f $ δεν έχει ακρότατα στο $ \mathbb{R} $.\\\\
β) Ισχύει ότι $ f'(x)\neq0 $ για κάθε $ x\in\mathbb{R} $ άρα η παράγωγος της συνάρτησης $ f $ ως συνεχής διατηρεί το πρόσημό της στο $ \mathbb{R} $. Επίσης $ f'(0)=1>0 $ άρα $ f'(x)>0 $ για κάθε $ x\in\mathbb{R} $.
\item Από την υπόθεση γνωρίζουμε ότι $\displaystyle f(\mathbb{R})=\mathbb{R}=\left(-\infty,+\infty \right)\Rightarrow\left(\lim_{x\to-\infty}{f(x)},\lim_{x\to+\infty}{f(x)} \right)=(-\infty,+\infty)  $. Επίσης έχουμε ότι $ f'(x)>0 $ άρα η συνάρτηση $ f $ είναι γνησίως αύξουσα στο $ \mathbb{R} $. Αυτό σημαίνει ότι $ \displaystyle{\lim_{x\to+\infty}{f(x)}=+\infty} $. Για κάθε $ x>0 $ θα ισχύει
\[ x>0\xRightarrow{f\nearrow}f(x)>f(0)\Rightarrow f(x)>0 \]
Θα ισχύει λοιπόν 
\[ \left|\frac{\hm{x}+\syn{x}}{f(x)} \right|\leq \left|\frac{\hm{x}}{f(x)} \right|+\left|\frac{\syn{x}}{f(x)} \right|\leq \frac{1}{f(x)}+\frac{1}{f(x)}=\frac{2}{f(x)}\Rightarrow -\frac{2}{f(x)}\leq \frac{\hm{x}+\syn{x}}{f(x)}\leq \frac{2}{f(x)} \]
Έχουμε έτσι $ \displaystyle{\lim_{x\to+\infty}{\left(-\frac{2}{f(x)}\right) }=\lim_{x\to+\infty}{\frac{2}{f(x)}}=0} $. Έτσι από το κριτήριο παρεμβολής θα πάρουμε 
\[ \lim_{x\to+\infty}{\frac{\hm{x}+\syn{x}}{f(x)}}=0 \]
\item Για κάθε $ x\in\left[0,e^\pi \right] $ θα ισχύει 
\[ 1\leq x\leq e^\pi\Rightarrow \ln{1}\leq \ln{x}\leq \ln{e^\pi}\Rightarrow f(0)\leq f(\ln{x})\leq f(\pi)\Rightarrow 0\leq \frac{f(\ln{x})}{x}\leq \frac{\pi}{x} \]
Ολοκληρώνοντας κάθε μέλος της τελευταίας ανισότητας από το $ 1 $ ως το $ e^\pi $ παίρνουμε 
\[ \int_{1}^{e^\pi}0dx\leq \int_{1}^{e^\pi}\frac{f(\ln{x})}{x}dx\leq \int_{1}^{e^\pi}\frac{\pi}{x}dx \]
Οι ισότητες στην τελευταία σχέση ισχύουν μόνο για $ x=1 $ και $ x=e $ οπότε παίρνουμε \[ 0< \int_{1}^{e^\pi}\frac{f(\ln{x})}{x}dx< [\pi\cdot\ln{x}]_{0}^{e^\pi}\Rightarrow 0< \int_{1}^{e^\pi}\frac{f(\ln{x})}{x}dx< \pi^2\]
\end{erwthma}
\end{thema}
\end{document}
