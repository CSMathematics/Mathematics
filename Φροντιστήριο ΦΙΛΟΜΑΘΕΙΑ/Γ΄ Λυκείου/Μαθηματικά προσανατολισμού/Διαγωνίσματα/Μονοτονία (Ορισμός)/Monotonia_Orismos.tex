\documentclass[twoside,nofonts,ektypwsh]{frontisthrio-diag}
\usepackage[amsbb,subscriptcorrection,zswash,mtpcal,mtphrb,mtpfrak]{mtpro2}
\usepackage[no-math,cm-default]{fontspec}
\usepackage{amsmath}
\usepackage{xunicode}
\usepackage{xgreek}
\let\hbar\relax
\defaultfontfeatures{Mapping=tex-text,Scale=MatchLowercase}
\setmainfont[Mapping=tex-text,Numbers=Lining,Scale=1.0,BoldFont={Nimbus Roman Bold}]{Nimbus Roman}
%\newfontfamily\scfont{GFS Artemisia}
\usepackage{fontawesome5}
%\newfontfamily{\FA}{fontawesome.otf}
\xroma{red!70!black}
%------TIKZ - ΣΧΗΜΑΤΑ - ΓΡΑΦΙΚΕΣ ΠΑΡΑΣΤΑΣΕΙΣ ----
\usepackage{tikz,pgfplots}
\usepackage{tkz-euclide}
\usepackage[framemethod=TikZ]{mdframed}
\usetikzlibrary{decorations.pathreplacing}
\tkzSetUpPoint[size=7,fill=white]
%-----------------------
\usepackage{calc,tcolorbox}
\tcbuselibrary{skins,theorems,breakable}
\usepackage{hhline}
\usepackage[explicit]{titlesec}
\usepackage{graphicx}
\usepackage{multicol}
\usepackage{multirow}
\usepackage{tabularx}
\usetikzlibrary{backgrounds}
\usepackage{sectsty}
\sectionfont{\centering}
\usepackage{enumitem}
\usepackage{adjustbox}
\usepackage{mathimatika,gensymb,eurosym,wrap-rl}
\usepackage{systeme,regexpatch}
%-------- ΜΑΘΗΜΑΤΙΚΑ ΕΡΓΑΛΕΙΑ ---------
\usepackage{mathtools}
%----------------------
%-------- ΠΙΝΑΚΕΣ ---------
\usepackage{booktabs}
%----------------------
%----- ΥΠΟΛΟΓΙΣΤΗΣ ----------
\usepackage{calculator}
%----------------------------



\tkzSetUpPoint[size=7,fill=white]
\tikzstyle{pl}=[line width=0.3mm]
\tikzstyle{plm}=[line width=0.4mm]
\usepackage{etoolbox}
\makeatletter
\renewrobustcmd{\anw@true}{\let\ifanw@\iffalse}
\renewrobustcmd{\anw@false}{\let\ifanw@\iffalse}\anw@false
\newrobustcmd{\noanw@true}{\let\ifnoanw@\iffalse}
\newrobustcmd{\noanw@false}{\let\ifnoanw@\iffalse}\noanw@false
\renewrobustcmd{\anw@print}{\ifanw@\ifnoanw@\else\numer@lsign\fi\fi}
\makeatother

\def\TyposDiagvnismatos{B}
\begin{document}
\titlos{Γ΄ Λυκείου - Μαθηματικά προσανατολισμού}{Μονοτονία ακρότατα}
\begin{thema}
\item \mbox{}\\\vspace{-5mm}
\begin{erwthma}
\item Έστω μια συνάρτηση $ f:A\to\mathbb{R} $ και $ \varDelta $ ένα διάστημα του πεδίου ορισμού της. Πότε η συνάρτηση $ f $ λέγεται γνησίως αύξουσα στο διάστημα $ \varDelta $;
\monades{8}
\item Να δώσετε τον ορισμό του ολικού ελάχιστου μιας συνάρτησης $ f $ με πεδίο ορισμού ένα σύνολο $ A $.\\\monades{7}
\item \swstolathospan
\begin{alist}
\item Αν μια συνάρτηση $ f:A\to\mathbb{R} $ είναι γνησίως μονότονη σε ένα διάστημα $ \varDelta\subseteq A $ του πεδίου ορισμού της, τότε έχει το πολύ μια ρίζα στο $ \varDelta $.
\item Από τη σχέση $ f(x)\geq 3 $ για κάθε $ x\in D_f $ συμπεραίνουμε ότι το $ 3 $ είναι ολικό ελάχιστο της $ f $.
\item Η συνάρτηση $ f(x)=\frac{1}{x} $ είναι γνησίως φθίνουσα στο $ \mathbb{R}^* $.
\item Η συνάρτηση $ f(x)=3x+2 $ δεν έχει ακρότατα.
\item Αν για μια συνάρτηση $ f $ ισχύει $ f(2)<f(3) $ με $ 2,3\in\varDelta $ τότε είναι η $ f $ είναι γνησίως αύξουσα στο διάστημα $ \varDelta $.
\end{alist}\monades{10}
\end{erwthma}
\item \mbox{}\\
Δίνονται οι ακόλουθες συναρτήσεις $ f:A\to\mathbb{R} $ και $ g:B\to\mathbb{R} $ με τύπους $ f(x)=\frac{1}{x-2}-x^3 $ και $ g(x)=\ln{(x-1)} $.
\begin{erwthma}
\item Να βρείτε τα πεδία ορισμού $ A,B $ των συναρτήσεων $ f,g $ αντίστοιχα.\monades{5}
\item Να αποδείξετε ότι η συνάρτηση $ f $ είναι γνησίως φθίνουσα στα διαστήματα $ (-\infty,2) $, $ (2,+\infty) $, ενώ η $ g $ είναι γνησίως αύξουσα στο $ (1,+\infty) $.\monades{10}
\item Να δείξετε ότι η συνάρτηση $ g-f $ είναι γνησίως αύξουσα.\\\monades{10}
\end{erwthma}
\item\mbox{}\\\vspace{-7mm}
\begin{erwthma}
\item Να λύσετε την ανίσωση
\[ 2e^{2-x}-\ln{(x-1)\leq x^3-8} \]\monades{8}\\
Δίνεται η συνάρτηση $ f(x)=\frac{2}{x}-\ln{x} $.
\item Να μελετήσετε τη συνάρτηση $ f $ ως προς τη μονοτονία της.\monades{7}
\item Να λύσετε την ανίσωση
\[ \frac{2}{x^2+1}-\frac{2}{2x^2+7}>\ln{\frac{x^2+1}{2x^2+7}} \]\monades{10}
\end{erwthma}
\item \mbox{}\\Δίνεται η συνάρτηση $ f(x)=ax^3+\beta x^2+\gamma x+\delta $ με $ a,\beta,\gamma,\delta\in\mathbb{R} $. Γνωρίζουμε ότι η $ f $ είναι περιττή και ότι η $ C_f $ διέρχεται από τα σημεία $ A(-1,-4) $ και $ B(2,26) $.
\begin{erwthma}
\item Να δείξετε ότι $ a=3,\beta=0,\gamma=1,\delta=0 $.\monades{7}
\item Να μελετήσετε την $ f $ ως προς τη μονοτονία.\monades{8}
\item Να λύσετε την ανίσωση
\[ (x^2-3)^3-(2x-1)^3<\frac{-x^2+2x+2}{3} \]\monades{8}
\end{erwthma}
\end{thema}
\diarkeia{3}
\kaliepityxia
\end{document}
