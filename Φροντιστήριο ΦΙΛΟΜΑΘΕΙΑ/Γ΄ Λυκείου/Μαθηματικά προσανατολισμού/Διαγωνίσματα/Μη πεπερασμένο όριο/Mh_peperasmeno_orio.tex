\documentclass[ektypwsh]{frontisthrio-diag}
\usepackage[amsbb,subscriptcorrection,zswash,mtpcal,mtphrb,mtpfrak]{mtpro2}
\usepackage[no-math,cm-default]{fontspec}
\usepackage{amsmath}
\usepackage{xunicode}
\usepackage{xgreek}
\let\hbar\relax
\defaultfontfeatures{Mapping=tex-text,Scale=MatchLowercase}
\setmainfont[Mapping=tex-text,Numbers=Lining,Scale=1.0,BoldFont={Minion Pro Bold}]{Minion Pro}
\newfontfamily\scfont{GFS Artemisia}
\font\icon = "Webdings"
\usepackage{fontawesome5}
\newfontfamily{\FA}{fontawesome.otf}
\xroma{cyan!70!black}
%------TIKZ - ΣΧΗΜΑΤΑ - ΓΡΑΦΙΚΕΣ ΠΑΡΑΣΤΑΣΕΙΣ ----
\usepackage{tikz,pgfplots}
\usepackage{tkz-euclide}
\usetkzobj{all}
\usepackage[framemethod=TikZ]{mdframed}
\usetikzlibrary{decorations.pathreplacing}
\tkzSetUpPoint[size=7,fill=white]
%-----------------------
\usepackage{calc,tcolorbox}
\tcbuselibrary{skins,theorems,breakable}
\usepackage{hhline}
\usepackage[explicit]{titlesec}
\usepackage{graphicx}
\usepackage{multicol}
\usepackage{multirow}
\usepackage{tabularx}
\usetikzlibrary{backgrounds}
\usepackage{sectsty}
\sectionfont{\centering}
\usepackage{enumitem}
\usepackage{adjustbox}
\usepackage{mathimatika,gensymb,eurosym,wrap-rl}
\usepackage{systeme,regexpatch}
%-------- ΜΑΘΗΜΑΤΙΚΑ ΕΡΓΑΛΕΙΑ ---------
\usepackage{mathtools}
%----------------------
%-------- ΠΙΝΑΚΕΣ ---------
\usepackage{booktabs}
%----------------------
%----- ΥΠΟΛΟΓΙΣΤΗΣ ----------
\usepackage{calculator}
%----------------------------

%------------------------------------------
\newcommand{\tss}[1]{\textsuperscript{#1}}
\newcommand{\tssL}[1]{\MakeLowercase{\textsuperscript{#1}}}
%---------- ΛΙΣΤΕΣ ----------------------
\newlist{bhma}{enumerate}{3}
\setlist[bhma]{label=\bf\textit{\arabic*\textsuperscript{o}\;Βήμα :},leftmargin=0cm,itemindent=1.8cm,ref=\bf{\arabic*\textsuperscript{o}\;Βήμα}}

\newlist{tropos}{enumerate}{3}
\setlist[tropos]{label=\bf\textit{\arabic*\textsuperscript{oς}\;Τρόπος :},leftmargin=0cm,itemindent=2.3cm,ref=\bf{\arabic*\textsuperscript{oς}\;Τρόπος}}
% Αν μπει το bhma μεσα σε tropo τότε
%\begin{bhma}[leftmargin=.7cm]
\tkzSetUpPoint[size=7,fill=white]
\tikzstyle{pl}=[line width=0.3mm]
\tikzstyle{plm}=[line width=0.4mm]
\usepackage{etoolbox}
\makeatletter
\renewrobustcmd{\anw@true}{\let\ifanw@\iffalse}
\renewrobustcmd{\anw@false}{\let\ifanw@\iffalse}\anw@false
\newrobustcmd{\noanw@true}{\let\ifnoanw@\iffalse}
\newrobustcmd{\noanw@false}{\let\ifnoanw@\iffalse}\noanw@false
\renewrobustcmd{\anw@print}{\ifanw@\ifnoanw@\else\numer@lsign\fi\fi}
\makeatother


\begin{document}
\titlos{Γ΄ Λυκείου - Μαθηματικά Προσανατολισμού}{Μη πεπερασμένο όριο}{Β}
\begin{thema}
\item\mbox{}\\\vspace{-7mm}
\begin{erwthma}
\item Δίνονται οι συναρτήσεις $ f,g $ για τις οποίες ισχύει $ f(x)\leq g(x) $ κοντά στο $ x_0 $ και $ \lim\limits_{x\to x_0}{g(x)}=-\infty $. Να δείξετε ότι $ \lim\limits_{x\to x_0}{f(x)}=-\infty $.\monades{10}
\item Να δείξετε ότι το όριο $ \lim\limits_{x\to 0}{\frac{1}{x^{2\nu+1}}} $ δεν υπάρχει.\monades{5}
\item \swstolathospan
\begin{alist}
\item Αν $ \lim\limits_{x\to x_0}{|f(x)|}=+\infty $ τότε είναι $ \lim\limits_{x\to x_0}{f(x)}=\pm\infty $.
\item Αν $ \lim\limits_{x\to x_0}{f(x)}=0 $ τότε $ \lim\limits_{x\to x_0}{\frac{1}{f(x)}}=+\infty $.
\item Αν $ f:\mathbb{R}\to\mathbb{R} $ συνάρτηση με $ \lim\limits_{x\to 0}{f(x)}=+\infty $ τότε το όριο $ \lim\limits_{x\to 0}{\frac{f(x)}{x}} $ δεν υπάρχει.
\item Αν $ \lim\limits_{x\to x_0^+}{f(x)}=+\infty $ και $ \lim\limits_{x\to x_0^-}{f(x)}=-\infty $ τότε το όριο $ \lim\limits_{x\to x_0}{\frac{1}{f(x)}} $ δεν υπάρχει.
\item Αν $ \lim\limits_{x\to x_0}{f(x)}=0 $ και $ f(x)<0 $ κοντά στο $ x_0 $ τότε $ \lim\limits_{x\to x_0}\frac{1}{f(x)}=+\infty $.
\end{alist}
\end{erwthma}\monades{10}
\item Να υπολογίσετε τα παρακάτω όρια.
\begin{multicols}{2}
\begin{erwthma}
\item $ \lim\limits_{x\to 1}{\frac{x^2-3x}{x^2-2x+1}} $
\item $ \lim\limits_{x\to -2}{\frac{3-2x}{x^2-x-6}} $
\item $ \lim\limits_{x\to 3}{\frac{\sqrt{4+\hm{x}}-2}{x^3}} $
\end{erwthma}\monades{8+8+9=25}
\end{multicols}
\item Δίνεται συνάρτηση $ f $ με τύπο \[ f(x)=\frac{x\sqrt{x}-4\sqrt{x}+ax+\beta}{x^2-4x} \]
με $ a,\beta\in\mathbb{R} $, για την οποία ισχύει $ \lim\limits_{x\to 4}{f(x)}=\frac{1}{4} $.
\begin{erwthma}
\item Να βρεθεί το πεδίο ορισμού της $ f $.\monades{5}
\item Να δείξετε ότι $ a=-1 $ και $ \beta=4 $.\monades{8}
\item Να αποδείξετε ότι $ f(x)=\frac{\sqrt{x}-1}{x} $.\monades{5}
\item Να βρείτε το όριο $ \lim\limits_{x\to 0}{f(x)} $.\monades{7}
\end{erwthma}
\item Δίνεται συνάρτηση $ f:\mathbb{R}\to \mathbb{R} $ για την οποία ισχύει
\[ \lim_{x\to 0}{\left( f(x)\cdot\frac{1-\syn{2x}}{x^3-3x^2}\right)} =+\infty \]
\begin{erwthma}
\item Να δείξετε ότι $ \lim\limits_{x\to 0}{f(x)}=-\infty $.\monades{10}
\item Να υπολογίσετε το όριο $ \lim\limits_{x\to 0}{\frac{\sqrt{f^2(x)-13f(x)+7}}{f(x)-21}} $.\monades{10}
\item Αν $ g:\mathbb{R}\to\mathbb{R} $ είναι μια τυχαία συνάρτηση, να βρείτε το όριο $ \lim\limits_{x\to 0}{\left(f^2(x)+g^2(x)\right)} $.\monades{5}
\end{erwthma}
\end{thema}
\kaliepityxia
\diarkeia{3}
\end{document}



