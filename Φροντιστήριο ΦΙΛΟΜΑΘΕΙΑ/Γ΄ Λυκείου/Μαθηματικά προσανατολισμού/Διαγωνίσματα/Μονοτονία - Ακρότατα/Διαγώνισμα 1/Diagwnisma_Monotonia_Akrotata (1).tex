\documentclass[twoside,nofonts,internet,math,spyros]{frontisthrio-diag}
\usepackage[amsbb,subscriptcorrection,zswash,mtpcal,mtphrb,mtpfrak]{mtpro2}
\usepackage[no-math,cm-default]{fontspec}
\usepackage{amsmath}
\usepackage{xunicode}
\usepackage{xgreek}
\let\hbar\relax
\defaultfontfeatures{Mapping=tex-text,Scale=MatchLowercase}
\setmainfont[Mapping=tex-text,Numbers=Lining,Scale=1.0,BoldFont={Minion Pro Bold}]{Minion Pro}
\newfontfamily\scfont{GFS Artemisia}
\font\icon = "Webdings"
\usepackage{fontawesome}
\newfontfamily{\FA}{fontawesome.otf}
\xroma{red!70!black}
%------TIKZ - ΣΧΗΜΑΤΑ - ΓΡΑΦΙΚΕΣ ΠΑΡΑΣΤΑΣΕΙΣ ----
\usepackage{tikz,pgfplots}
\usepackage{tkz-euclide}
\usetkzobj{all}
\usepackage[framemethod=TikZ]{mdframed}
\usetikzlibrary{decorations.pathreplacing}
\tkzSetUpPoint[size=7,fill=white]
%-----------------------
\usepackage{calc,tcolorbox}
\tcbuselibrary{skins,theorems,breakable}
\usepackage{hhline}
\usepackage[explicit]{titlesec}
\usepackage{graphicx}
\usepackage{multicol}
\usepackage{multirow}
\usepackage{tabularx}
\usetikzlibrary{backgrounds}
\usepackage{sectsty}
\sectionfont{\centering}
\usepackage{enumitem}
\usepackage{adjustbox}
\usepackage{mathimatika,gensymb,eurosym,wrap-rl}
\usepackage{systeme,regexpatch}
%-------- ΜΑΘΗΜΑΤΙΚΑ ΕΡΓΑΛΕΙΑ ---------
\usepackage{mathtools}
%----------------------
%-------- ΠΙΝΑΚΕΣ ---------
\usepackage{booktabs}
%----------------------
%----- ΥΠΟΛΟΓΙΣΤΗΣ ----------
\usepackage{calculator}
%----------------------------
%------ ΔΙΑΓΩΝΙΟ ΣΕ ΠΙΝΑΚΑ -------
\usepackage{array}
\newcommand\diag[5]{%
\multicolumn{1}{|m{#2}|}{\hskip-\tabcolsep
$\vcenter{\begin{tikzpicture}[baseline=0,anchor=south west,outer sep=0]
\path[use as bounding box] (0,0) rectangle (#2+2\tabcolsep,\baselineskip);
\node[minimum width={#2+2\tabcolsep-\pgflinewidth},
minimum  height=\baselineskip+#3-\pgflinewidth] (box) {};
\draw[line cap=round] (box.north west) -- (box.south east);
\node[anchor=south west,align=left,inner sep=#1] at (box.south west) {#4};
\node[anchor=north east,align=right,inner sep=#1] at (box.north east) {#5};
\end{tikzpicture}}\rule{0pt}{.71\baselineskip+#3-\pgflinewidth}$\hskip-\tabcolsep}}
%---------------------------------
%---- ΟΡΙΖΟΝΤΙΟ - ΚΑΤΑΚΟΡΥΦΟ - ΠΛΑΓΙΟ ΑΓΚΙΣΤΡΟ ------
\newcommand{\orag}[3]{\node at (#1)
{$ \overcbrace{\rule{#2mm}{0mm}}^{{\scriptsize #3}} $};}
\newcommand{\kag}[3]{\node at (#1)
{$ \undercbrace{\rule{#2mm}{0mm}}_{{\scriptsize #3}} $};}
\newcommand{\Pag}[4]{\node[rotate=#1] at (#2)
{$ \overcbrace{\rule{#3mm}{0mm}}^{{\rotatebox{-#1}{\scriptsize$#4$}}}$};}
%-----------------------------------------
%------------------------------------------
\newcommand{\tss}[1]{\textsuperscript{#1}}
\newcommand{\tssL}[1]{\MakeLowercase{\textsuperscript{#1}}}
%---------- ΛΙΣΤΕΣ ----------------------
\newlist{bhma}{enumerate}{3}
\setlist[bhma]{label=\bf\textit{\arabic*\textsuperscript{o}\;Βήμα :},leftmargin=0cm,itemindent=1.8cm,ref=\bf{\arabic*\textsuperscript{o}\;Βήμα}}
\newlist{rlist}{enumerate}{3}
\setlist[rlist]{itemsep=0mm,label=\roman*.}
\newlist{brlist}{enumerate}{3}
\setlist[brlist]{itemsep=0mm,label=\bf\roman*.}
\newlist{tropos}{enumerate}{3}
\setlist[tropos]{label=\bf\textit{\arabic*\textsuperscript{oς}\;Τρόπος :},leftmargin=0cm,itemindent=2.3cm,ref=\bf{\arabic*\textsuperscript{oς}\;Τρόπος}}
% Αν μπει το bhma μεσα σε tropo τότε
%\begin{bhma}[leftmargin=.7cm]
\tkzSetUpPoint[size=7,fill=white]
\tikzstyle{pl}=[line width=0.3mm]
\tikzstyle{plm}=[line width=0.4mm]
\usepackage{etoolbox}
\makeatletter
\renewrobustcmd{\anw@true}{\let\ifanw@\iffalse}
\renewrobustcmd{\anw@false}{\let\ifanw@\iffalse}\anw@false
\newrobustcmd{\noanw@true}{\let\ifnoanw@\iffalse}
\newrobustcmd{\noanw@false}{\let\ifnoanw@\iffalse}\noanw@false
\renewrobustcmd{\anw@print}{\ifanw@\ifnoanw@\else\numer@lsign\fi\fi}
\makeatother

\usepackage{path}
\pathal

\begin{document}
\titlos{ΜΑΘΗΜΑΤΙΚΑ ΠΡΟΣΑΝΑΤΟΛΙΣΜΟΥ Γ΄ ΛΥΚΕΙΟΥ}{Διαγώνισμα}{Διαφορικός Λογισμός - Μονοτονία - Ακρότατα}
\begin{thema}
\item \mbox{}\\\vspace{-5mm}
\begin{erwthma}
\item Δίνεται μια συνεχής συνάρτηση $ f:\Delta\to\mathbb{R} $. Να δείξετε ότι αν $ f'(x)>0 $ για κάθε εσωτερικό σημείο του $ \Delta $ τότε η $ f $ είναι γνησίως αύξουσα σε όλο το $ \Delta $.
\monades{10}
\item Να διατυπώσετε το θεώρημα του Fermat.\monades{5}
\item \swstolathospan
\begin{alist}
\item Αν μια συνάρτηση είναι συνεχής σε ένα διάστημα $ \varDelta $, παραγωγίσιμη σε κάθε εσωτερικό σημείο του $ \varDelta $ και είναι γνησίως φθίνουσα στο $ \varDelta $ τότε ισχύει $ f'(x)<0 $ για κάθε $ x\in\varDelta $.
\item Αν μια συνάρτηση $ f $ είναι ορισμένη και παραγωγίσιμη σε ένα διάστημα $ [a,\beta] $ και παρουσιάζει τοπικό μέγιστο στο $ x_0\in[a,\beta] $ τότε ισχύει $ f'(x_0)=0 $.
\item Για μια συνάρτηση $ f:[a,\beta]\to\mathbb{R} $ πιθανές θέσεις ακρότατων είναι τα άκρα του διαστήματος και τα κρίσιμα σημεία της. 
\item Αν ισχύει $ f'(x)\geq 0 $ για κάθε $ x\in\mathbb{R} $ τότε η $ f $ είναι γνησίως αύξουσα στο $ \mathbb{R} $.
\item Αν ισχύει $ f'(x)>0 $ για κάθε $ x\in\mathbb{R}^* $ τότε η $ f $ είναι γνησίως αύξουσα στο $ \mathbb{R}^* $.
\end{alist}\monades{10}
\end{erwthma}
\item\mbox{}\\
Δίνεται η συνάρτηση 
\[ f(x)=x^3+ax-1-\hm{2x} \]
για την οποία ισχύει ότι $ f'(0)=0 $.
\begin{erwthma}
\item Να αποδείξετε ότι $ a=2 $.\monades{4}
\item Να εξετάσετε αν η $ f $ έχει ακρότατο στο $ 0 $.\monades{8}
\item Να βρείτε το σύνολο τιμών της $ f $.\monades{6}
\item Να αποδείξετε ότι η εξίσωση \[ \frac{2017+\hm{2x}}{x^2+2}=x \]
έχει μοναδική ρίζα στο $ \mathbb{R} $.\monades{7}
\end{erwthma}
\item \mbox{}\\
Δίνεται παραγωγίσιμη συνάρτηση $ f:\mathbb{R}\to\mathbb{R} $, για την οποία ισχύει $ f(0)=2\ln{3} $ και:
\[ f'(x)=(4x-8)e^{-f(x)}\ \ ,\ \ \textrm{για κάθε }x\in\mathbb{R} \]
\begin{erwthma}
\item Να δείξετε ότι ο τύπος της $ f $ είναι $ f(x)=\ln{\left(2x^2-8x+9\right) } $.\monades{5}
\item Να μελετήσετε την $ f $ ως προς τη μονοτονία και τα ακρότατα.\monades{7}
\item Να βρείτε το σύνολο τιμών της $ f $.\monades{6}
\item Να βρείτε τους αριθμούς $ x $ και $ y $ για τους οποίους ισχύει:
\[ f\left( ye^x-x^2-yx\right)+f(y)=0 \]\monades{7}
\end{erwthma}
\item \mbox{}\\
Από ένα φύλλο λαμαρίνας σχήματος τετραγώνου πλευράς $ 6 $ μέτρων κατασκευάζεται μια δεξαμενή σχήματος ορθογωνίου παραλληλεπιπέδου, ανοιχτή από πάνω. Από τις γωνίες του φύλλου λαμαρίνας κόβονται τέσσερα ίσα τετράγωνα πλευράς $ x $ μέτρων, με $ 0<x<3 $, και στη συνέχεια οι πλευρές διπλώνονται προς τα πάνω όπως φαίνεται στο παρακάτω σχήμα.
\begin{center}
\begin{tikzpicture}
\draw[dashed] (-2,2)  -- (-2,-0.5) -- (0.5,-0.5) -- (0.5,2) -- cycle;
\draw[plm,\xrwma] (-1.5,2) -- (-1.5,1.5) -- (-2,1.5) -- (-2,0) -- (-1.5,0) -- (-1.5,-0.5) -- (0,-0.5) -- (0,0) -- (0.5,0) -- (0.5,1.5) -- (0,1.5) -- (0,2) -- cycle;
\draw[|-|] (0.8,2) -- (0.8,-0.5);
\node at (-1.75,2.25) {$x$};
\node at (1.15,0.75) {$6m$};
\draw[fill=\xrwma!20] (3.5,0.75) -- (6,0.75) -- (5,0) -- (2.5,0) -- (3.5,0.75);
\draw[fill=\xrwma!20] (3.5,0.75) -- (2.5,-0.5) -- (2.5,0) -- (3.5,1.25);
\draw[fill=\xrwma!20] (2.5,-0.5) node (v1) {} -- (5,-0.5) node (v6) {} -- (5,0) node (v3) {} -- (2.5,0) node (v7) {} -- (2.5,-0.5);
\draw[fill=\xrwma!20] (3.5,0.75) node (v2) {} -- (5.75,0.75) node (v5) {} -- (6,1.25) node (v4) {} -- (3.5,1.25) node (v8) {} -- (3.5,0.75) node (v9) {};
\draw[fill=\xrwma!20] (5,0) -- (6,1.25) -- (6,0.75) node (v10) {} -- (5,-0.5) -- (5,0) node (v11) {};
\node at (2.25,-0.25) {$ x $};
\end{tikzpicture}
\end{center}
\begin{erwthma}
\item Να αποδείξετε ότι ο όγκος της δεξαμενής ως συνάρτηση του $ x $ είναι:
\[ f(x)=4x(3-x)^2\ \ ,\ \ 0<x<3 \]\monades{5}
\item Να βρείτε για ποια τιμή του $ x $ η δεξαμενή έχει το μέγιστο όγκο.\monades{7}
\item Να βρείτε το όριο $ {\displaystyle{\lim_{x\to 0}{\frac{f(x+2)-8}{x}}}} $.\monades{6}
\item Αν $ x_1,x_2\in\left(0,\frac{\pi}{2} \right) $, με $ x_1<x_2 $, να αποδείξετε ότι:
\[ \frac{\syn{x_1}}{\syn{x_2}}>\left(\frac{3-\syn{x_2}}{3-\syn{x_1}} \right)^2  \]\monades{7}
\end{erwthma}
\end{thema}
\kaliepityxia
\end{document}
