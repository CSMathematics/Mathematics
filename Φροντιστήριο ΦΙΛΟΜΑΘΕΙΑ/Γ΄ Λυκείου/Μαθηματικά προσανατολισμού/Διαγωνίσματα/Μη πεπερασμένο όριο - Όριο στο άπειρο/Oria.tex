\documentclass[ektypwsh]{frontisthrio-diag}
\usepackage[T1]{fontenc}
\usepackage[english,greek]{babel}
\usepackage{amsmath}
\usepackage{libertine} % txfontsb,libertinus,libertine,kerkis,nimbusserif
\let\Bbbk\relax
\usepackage[amsbb,subscriptcorrection,zswash,mtpcal,mtphrb,mtpfrak]{mtpro2}
\newcommand{\kerkissans}[1]{{\fontfamily{maksf}\selectfont #1}}
\usetikzlibrary{decorations.pathreplacing,backgrounds}
\tkzSetUpPoint[size=2.8,fill=white]
%-----------------------
\usepackage[explicit]{titlesec}
\usepackage{sectsty}
\sectionfont{\centering}
\usepackage{graphicx,tabularx,fontawesome5,pgfplots,calc,tcolorbox,hhline,adjustbox,mathimatika,gensymb,eurosym,wrap-rl,systeme,regexpatch,mathtools,booktabs,calculator}
\let\vary\relax
\usepackage{exsheets}
%----------------------------
\tcbuselibrary{skins,theorems,breakable}
\tikzstyle{pl}=[line width=0.3mm]
\tikzstyle{plm}=[line width=0.4mm]
\usepackage{etoolbox}
\makeatletter
\renewrobustcmd{\anw@true}{\let\ifanw@\iffalse}
\renewrobustcmd{\anw@false}{\let\ifanw@\iffalse}\anw@false
\newrobustcmd{\noanw@true}{\let\ifnoanw@\iffalse}
\newrobustcmd{\noanw@false}{\let\ifnoanw@\iffalse}\noanw@false
\renewrobustcmd{\anw@print}{\ifanw@\ifnoanw@\else\numer@lsign\fi\fi}
\makeatother
\renewcommand{\textstigma}{\textsigma\texttau}
\def\TyposDiagvnismatos{B}
\ekthetesdeiktes
\begin{document}
\titlos{Τάξη - Μάθημα}{Κεφάλαιο}
\begin{thema}
\item\mbox{}\\\vspace{-7mm}
\begin{erwthma}
\item \swstolathospan
\begin{alist}
\item Αν $ \lim\limits_{x\to x_0}{f(x)}=0 $ τότε 
\[ \lim_{x\to x_0}{\frac{1}{f(x)}}=\pm \infty \]
\item Αν $ \lim\limits_{x\to x_0}{|f(x)|}=+\infty $ τότε $ \lim\limits_{x\to x_0}{f(x)}=+\infty $ ή $ \lim\limits_{x\to x_0}{f(x)}=-\infty $.
\item Αν $ \lim\limits_{x\to x_0^-}{f(x)}=+\infty $ και $ \lim\limits_{x\to x_0^+}{f(x)}=-\infty $ τότε το όριο $ \lim\limits_{x\to x_0}{\frac{1}{f(x)}} $ δεν υπάρχει.
\item Αν $ f:\mathbb{R}\to\mathbb{R} $ συνάρτηση με $ \lim\limits_{x\to 0}{f(x)}=+\infty $, τότε το όριο $ \lim\limits_{x\to 0}{\frac{x}{f(x)}} $ δεν υπάρχει.
\item Αν ισχύει $ \lim\limits_{x\to +\infty}{\frac{P(x)}{Q(x)}}=0 $ τότε $ \lim\limits_{x\to +\infty}{\frac{Q(x)}{P(x)}}=+\infty $.
\item Ισχύει ότι $ \lim\limits_{x\to 0}{\frac{1}{x^{2\nu}}}=+\infty $.
\item Αν $ \lim\limits_{x\to+\infty}{f(x)}=+\infty $ τότε $ \lim\limits_{x\to +\infty}{\ln{\frac{1}{f(x)}}}=-\infty $.
\item Το όριο $ \lim\limits_{x\to 0}{\frac{1}{x^{2\nu+1}}} $ δεν υπάρχει.
\item Αν $ \lim\limits_{x\to x_0}{f(x)}=0 $ τότε $ \lim\limits_{x\to x_0}{\left[f(x)\cdot\syn{\frac{1}{f(x)}}\right]}=0 $.
\end{alist}
\item Να συμπληρώσετε τα κενά.
\begin{alist}
\item $ \lim\limits_{x\to -\infty}{x^{2\nu}}=\ldots\ldots\ldots $, όπου $ \nu\in\mathbb{N}^* $.
\item $ \lim\limits_{x\to +\infty}{\frac{1}{x^{\nu}}}=\ldots\ldots\ldots $, όπου $ \nu\in\mathbb{N}^* $.
\item Αν $ a>1 $, τότε $ \lim\limits_{x\to -\infty}{a^x}=\ldots\ldots\ldots $ και $ \lim\limits_{x\to +\infty}{a^x}=\ldots\ldots\ldots $
\item $ \lim\limits_{x\to 0^+}{\log{x}}=\ldots\ldots\ldots $
\end{alist}
\item Από τις παρακάτω παραστάσεις να επιλέξετε αυτές που αποτελούν απροσδιοριστία.
\begin{alist}
\begin{multicols}{3}
\item $ (+\infty)-(-\infty) $
\item $ (+\infty)+(+\infty) $
\item $ (-\infty)-(+\infty) $
\item $ (-\infty)+(+\infty) $
\item $ (-\infty)-(-\infty) $
\item $ (-\infty)+(-\infty) $
\item $ 0\cdot(\pm\infty) $
\item $ \frac{\infty}{0} $
\item $ \frac{-\infty}{+\infty} $
\end{multicols}
\end{alist}
\end{erwthma}
\item Να υπολογιστούν τα παρακάτω όρια
\begin{multicols}{2}
\begin{erwthma}
\item $ \displaystyle\lim\limits_{x\to 3}\frac{x^2-3x}{x^3-6x^2+9x} $
\item $ \displaystyle\lim\limits_{x\to -2}{\frac{2x-1}{x^2-x-6}} $
\item $ \displaystyle\lim\limits_{x\to 1}{\left(\frac{1}{(x-1)^2}-\frac{2}{x-1}\right)} $
\item $ \lim\limits_{x\to -\infty}{\frac{3^x-5^x}{5^{x+2}+3^{x+1}}} $
\item $ \lim\limits_{x\to+\infty}{\left(\sqrt{4x^2-3x+1}-x\right)} $
\item $ \lim\limits_{x\to +\infty}{\left(\ln(x^2-x)-\ln(2x^3-x^2+4)\right)} $
\end{erwthma}
\end{multicols}
\item Δίνεται η συνάρτηση $ f:\mathbb{R}\to\mathbb{R} $ με τύπο
\[ f(x)=\begin{cdcases}
\sqrt{9x^2+x+6}-\beta x & ,x\leq 1\\\frac{ax^2+\beta x+5}{x-x^2} & ,x>1
\end{cdcases} \]
για την οποία υπάρχει το όριο $ \lim\limits_{x\to1}{f(x)} $.
\begin{erwthma}
\item Να δείξετε ότι $ a=-2,\beta=-3 $.
\item Να βρείτε τα όρια $ \lim\limits_{x\to +\infty}{f(x)} $ και $ \lim\limits_{x\to-\infty}{f(x)} $.
\item Να βρείτε το όριο $ \lim\limits_{x\to +\infty}{\left[(f(x)-2)\hm{x}\right]} $.
\end{erwthma}
\item Δίνεται συνάρτηση $ f:\mathbb{R}\to\mathbb{R} $ για την οποία ισχύει
\[ \frac{\hm{x}+2x^2+10x}{x+2}\leq f(x)\leq \frac{2x^2+8x+7}{x+1} \]
για κάθε $ x>0 $.
\begin{erwthma}
\item Να δείξετε ότι
\[ \lim_{x\to +\infty}{\frac{f(x)}{x}}=2\ \ \textrm{και}\ \ \lim_{x\to +\infty}{(f(x)-2x)}=6 \]
\item Να βρείτε το όριο
\[ \lim_{x\to +\infty}{\frac{f(x)+3x+x^2\cdot\hm{\frac{1}{x}}}{xf(x)-2x^2-4x+3}} \]
\end{erwthma}
\end{thema}
\end{document}
