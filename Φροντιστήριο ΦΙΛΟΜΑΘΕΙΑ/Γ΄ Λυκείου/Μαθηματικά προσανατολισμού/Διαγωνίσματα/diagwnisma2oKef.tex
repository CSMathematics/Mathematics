\documentclass[11pt,a4paper]{article}
\usepackage[english,greek]{babel}
\usepackage[utf8]{inputenc}
\usepackage{nimbusserif}
\usepackage[T1]{fontenc}
\usepackage[left=2.00cm, right=2.00cm, top=3.00cm, bottom=2.00cm]{geometry}
\usepackage{amsmath}
\let\myBbbk\Bbbk
\let\Bbbk\relax
\usepackage[amsbb,subscriptcorrection,zswash,mtpcal,mtphrb,mtpfrak]{mtpro2}
\usepackage{graphicx,multicol,multirow,enumitem,tabularx,mathimatika,gensymb,venndiagram,hhline,longtable,tkz-euclide,fontawesome5,eurosym,tcolorbox,wrap-rl}
\tcbuselibrary{skins,theorems,breakable}
\newlist{rlist}{enumerate}{3}
\setlist[rlist]{itemsep=0mm,label=\roman*.}
\newlist{alist}{enumerate}{3}
\setlist[alist]{itemsep=0mm,label=\alph*.}
\newlist{balist}{enumerate}{3}
\setlist[balist]{itemsep=0mm,label=\bf\alph*.}
\newlist{Alist}{enumerate}{3}
\setlist[Alist]{itemsep=0mm,label=\Alph*.}
\newlist{bAlist}{enumerate}{3}
\setlist[bAlist]{itemsep=0mm,label=\bf\Alph*.}
\renewcommand{\textstigma}{\textsigma\texttau}
\newlist{thema}{enumerate}{3}
\setlist[thema]{label=\bf\large{ΘΕΜΑ \textcolor{black}{\Alph*}},itemsep=0mm,leftmargin=0cm,itemindent=18mm}
\newlist{erwthma}{enumerate}{3}
\setlist[erwthma]{label=\bf{\large{\textcolor{black}{\Alph{themai}.\arabic*}}},itemsep=0mm,leftmargin=0.8cm}

\newcommand{\lysh}{\textcolor{black}{\textbf{\faCheck\ \ ΛΥΣΗ}}}
\renewcommand{\textstigma}{\textsigma\texttau}
%----------- ΟΡΙΣΜΟΣ------------------
\newcounter{orismos}[section]
\renewcommand{\theorismos}{\thesection.\arabic{orismos}}   
\newcommand{\Orismos}{\refstepcounter{orismos}{\textbf{\textcolor{black}{\kerkissans{Ορισμός\hspace{2mm}\theorismos}}\;:\;}{}}}

\newenvironment{orismos}[1]
{\begin{tcolorbox}[title=\Orismos {\textcolor{black}{\kerkissans{#1}}},breakable,bottomtitle=-1.5mm,
enhanced standard,titlerule=-.2pt,toprule=0pt, rightrule=0pt, bottomrule=0pt,
colback=white,left=2mm,top=1mm,bottom=0mm,
boxrule=0pt,
colframe=white,borderline west={1.5mm}{0pt}{black},leftrule=2mm,sharp corners,coltitle=black]}
{\end{tcolorbox}}

\newcommand{\kerkissans}[1]{{\fontfamily{maksf}\selectfont \textbf{#1}}}
\renewcommand{\textdexiakeraia}{}

\usepackage[
backend=biber,
style=alphabetic,
sorting=ynt
]{biblatex}

\begin{document}
\begin{thema}
\item\mbox{}\\\vspace{-5mm}
\begin{erwthma}
\item Πότε μια συνάρτηση $f$ με πεδίο ορισμού $A$ παρουσιάζει στο $x_0\in A$, τοπικό ελάχιστο?
\item Πότε η ευθεία $y=\lambda x+\beta$ λέγεται ασύμπτωτη της γραφικής παράστασης μιας συνάρτησης $f$, στο $+\infty$?
\item Αν οι συναρτήσεις $f,g$ είναι παραγωγίσιμες στο $x_0$, να αποδείξετε ότι και η $f+g$ είναι παραγωγίσιμη στο $x_0$ και ισχύει
\[ (f+g)'(x_0)=f'(x_0)+g'(x_0) \]
\item Να χαρακτηρίσετε καθεμία από τις παρακάτω προτάσεις ως \textbf{Σωστή} ή \textbf{Λανθασμένη}.
\begin{alist}
\item Αν μια παραγωγίσιμη συνάρτηση $f$ είναι γνησίως αύξουσα σε ένα διάστημα $\varDelta$, τότε ισχύει $f'(x)>0$ για κάθε $x\in\varDelta$.
\item Κρίσιμα σημεία μιας συνάρτησης $f$ ονομάζονται τα σημεία στα οποία μηδενίζεται η παράγωγος της $f$, ή τα σημεία στα οποία η $f$ δεν είναι παραγωγίσιμη.
\item Η συνάρτηση $f(x)=x^4$ είναι κυρτή στο $\mathbb{R}$.
\item Αν για μια παραγωγίσιμη συνάρτηση $f$, ισχύει $f'(x_0)$ σε κάποιο σημείο $x_0\in(a,\beta)$, τότε η $f$ έχει τοπικό ακρότατο στο σημείο αυτό.
\item Ισχύει ότι $\left(\sqrt{x}\right)'=\frac{2}{\sqrt{x}}$.
\end{alist}
\end{erwthma}
\item Δίνεται η συνάρτηση $f(x)=\ln{\left(x^2+1\right)}$.
\begin{erwthma}
\item Να μελετήσετε τη συνάρτηση $f$ ως προς τη μονοτονία και τα ακρότατα.
\item Να μελετήσετε την $f$ ως προς την κυρτότητα και τα σημεία καμπής.
\item Να βρεθεί η εξίσωση της εφαπτομένης της $C_{f'}$ στο σημείο της $M(2,f'(2))$.
\item Να υπολογίσετε το όριο
\[ \lim_{x\to+\infty}{\frac{f(x)}{x^2}} \]
\end{erwthma}
\item Δίνεται η συνάρτηση $f:\mathbb{R}\to\mathbb{R}$ με τύπο
\[ f(x)=\begin{cdcases}
xe^x & ,x\geq 0\\x^2+ax & ,x<0
\end{cdcases} \]
της οποίας η γραφική παράσταση διέρχεται από το σημείο $M(-2,2)$.
\begin{erwthma}
\item Να αποδείξετε ότι $a=1$.
\item Να αποδείξετε ότι η $f$ είναι κυρτή στο $\mathbb{R}$.
\item Να βρεθεί η εξίσωση της εφαπτομένης της $C_f$ στο σημείο της $M(1,f(1))$.
\item Να αποδείξετε ότι ισχύει 
\[ x\left(e^{x-1}-2\right)\geq-1 \]
για κάθε $x\geq 0$.
\end{erwthma}
\item (Τράπεζα Θεμάτων - 29644)\mbox{}\\
\wrapr{-5mm}{12}{6cm}{-5mm}{
\begin{tikzpicture}
\draw[line width=.1mm,xshift=5mm,yshift=2.5mm,opacity=0.3] (0,0) grid (5,5);
\draw[line width=.05mm,xshift=5mm,yshift=2.5mm,opacity=0.1,xstep=0.2,ystep=0.25] (0,0) grid (5,5);
\begin{axis}[aks_on,belh ar,xlabel={\footnotesize $ x $},xtick={-3,-2,-1,0,1,2},ytick={-6,-5,...,4},
ylabel={\footnotesize $ y $},xmin=-3.5,xmax=2.5,ymin=-6.5,ymax=4,x=1cm,y=.5cm]
\addplot[domain=-1:2,draw=black,line width=.3mm] {-x^2+3};
\addplot[domain=-3:-1,draw=black,line width=.3mm] {(x+1)^3+2};
\node at (axis cs:-.2,-.52){$O$};
\node at (axis cs:1.3,2.2){$C_f$};
\node at(axis cs:-2.5,.4){\footnotesize$A$};
\node at(axis cs:2,.4){\footnotesize$B$};
\fill (axis cs:-2.26,0) circle (.7mm);
\fill (axis cs:1.732,0) circle (.7mm);
\end{axis}
\end{tikzpicture}
}{
Στο διπλανό σχήμα δίνεται η γραφική παράσταση μιας συνεχούς συνάρτησης $f$ στο  διάστημα $[-3,2]$ η οποία παρουσιάζει μέγιστο στο $0$ το $3$ και τέμνει τον άξονα $x'x$ στα σημεία $A$ και $B$. Έστω  η συνάρτηση $g$ με $g(x)=f(x)+x$, $x\in[-3,2]$.
\begin{erwthma}
\item Να αποδείξετε ότι: 
\begin{alist}
\item Η συνάρτηση $g$ είναι συνεχής στο $[-3,2]$.
\item Η εξίσωση $g(x)=0$ έχει μία τουλάχιστον ρίζα.	
\end{alist}
\item Αν η συνάρτηση $f$ είναι παραγωγίσιμη στο $(-1,2)$, να αποδείξετε ότι η εφαπτομένη ευθεία της γραφικής παράστασης της συνάρτησης $g$, στο σημείο που η $f$ παρουσιάζει μέγιστο, έχει εξίσωση $y=x+3$.
\end{erwthma}}
\end{thema}
\end{document}
