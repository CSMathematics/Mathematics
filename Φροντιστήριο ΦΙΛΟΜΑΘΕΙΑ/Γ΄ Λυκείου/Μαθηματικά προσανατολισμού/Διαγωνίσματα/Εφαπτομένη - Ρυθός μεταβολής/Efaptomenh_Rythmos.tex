\documentclass[ektypwsh]{frontisthrio-diag}
\usepackage[T1]{fontenc}
\usepackage[english,greek]{babel}
\usepackage{amsmath}
\usepackage{nimbusserif} % txfontsb,libertinus,libertine,kerkis,nimbusserif
\let\Bbbk\relax
\usepackage[amsbb,subscriptcorrection,zswash,mtpcal,mtphrb,mtpfrak]{mtpro2}
\newcommand{\kerkissans}[1]{{\fontfamily{maksf}\selectfont #1}}
\usetikzlibrary{decorations.pathreplacing,backgrounds}
\tkzSetUpPoint[size=2.8,fill=white]
%-----------------------
\usepackage[explicit]{titlesec}
\usepackage{sectsty}
\sectionfont{\centering}
\usepackage{graphicx,tabularx,fontawesome5,pgfplots,calc,tcolorbox,hhline,adjustbox,mathimatika,gensymb,eurosym,wrap-rl,systeme,regexpatch,mathtools,booktabs,calculator}
\let\vary\relax
\usepackage{exsheets}
%----------------------------
\tcbuselibrary{skins,theorems,breakable}
\tikzstyle{pl}=[line width=0.3mm]
\tikzstyle{plm}=[line width=0.4mm]
\usepackage{etoolbox}
\makeatletter
\renewrobustcmd{\anw@true}{\let\ifanw@\iffalse}
\renewrobustcmd{\anw@false}{\let\ifanw@\iffalse}\anw@false
\newrobustcmd{\noanw@true}{\let\ifnoanw@\iffalse}
\newrobustcmd{\noanw@false}{\let\ifnoanw@\iffalse}\noanw@false
\renewrobustcmd{\anw@print}{\ifanw@\ifnoanw@\else\numer@lsign\fi\fi}
\makeatother

\def\TyposDiagvnismatos{B}

\begin{document}
\titlos{Γ Λυκείου - Μαθηματικά}{Εφαπτομένη - Ρυθμός μεταβολής}
\begin{thema}
\item\mbox{}\\\vspace{-7mm}
\begin{erwthma}
\item Να δώσετε τον ορισμό της εφαπτόμενης ευθείας $ \varepsilon $ της γραφικής παράστασης μιας συνάρτησης  $ f $ στο σημείο της $ M(x_0,f(x_0)) $.
\item Να δώσετε τον ορισμό του ρυθμού μεταβολής ενός ποσού $ y $ ως προς ένα ποσό $ x $ σε ένα σημείο $ x_0 $.
\item Για καθεμία από τις ακόλουθες προτάσεις να γράψετε την αντίστοιχη σχέση που προκύπτει από αυτήν.
\begin{alist}
\item Έστω $ x(t) $ μια παραγωγίσιμη συνάρτηση που δίνει τη θέση ενός κινητού τη χρονική στιγμή $ t $.
\begin{rlist}
\item Με τι ισούται η ταχύτητα του κινητού?
\item Με τι ισούται η επιτάχυνση του κινητού?
\item Πότε το σώμα είναι στιγμιαία ακίνητο?
\item Πότε το σώμα κινείται προς την αρνητική φορά?
\item Πότε η ταχύτητα του σώματος αυξάνεται?
\end{rlist}
\item Έστω $ f,g $ δύο συναρτήσεις και $ x_0 $ ένα κοινό σημείο του πεδίου ορισμού τους. Ποιες συνθήκες ισχύουν ώστε οι $ C_f $ και $ C_g $ να δέχονται κοινή εφαπτομένη στο κοινό τους σημείο $ M(x_0,y_0) $?
\end{alist}
\item \swstolathospan
\begin{alist}
\item Αν $ C_f $ είναι η γραφική παράσταση μιας συνάρτησης $ f $ τότε σε κάθε σημείο $ M(x_0,f(x_0)) $ υπάρχει μοναδική εφαπτόμενη ευθεία.
\item Αν $ V'(t)=2m^3/min $ είναι ο ρυθμός μεταβολής του όγκου του νερού μιας δεξαμενής τότε η δεξαμενή αδειάζει με ρυθμό $ 2m^3 $ ανά λεπτό.
\item Αν $ a(t_0)=0 $ είναι η στιγμιαία επιτάχυνση ενός κινητού τη χρονική στιγμή $ t_0 $ τότε το σώμα κινείται με σταθερή ταχύτητα.
\item Η συνάρτηση $ f(x)=|x| $ έχει εφαπτομένη στο σημείο $ M(0,f(0)) $.
\item Το μέσο κόστος παραγωγής $ K_{\mu}(x) $ μιας επιχείρησης που παράγει $ x $ προϊόντα το μήνα, δίνεται από τη σχέση \[ K_{\mu}(x)=\frac{K(x)}{x} \]
όπου $ K(x) $ είναι η συνάρτηση του κόστους.
\end{alist}
\end{erwthma}
\item Δίνεται η συνάρτηση $ f(x)=x^2+\beta x+\gamma $. Η ευθεία $ y=-2x-14 $ εφάπτεται στη γραφική παράσταση της $ f $ στο σημείο $ M(-3,f(-3)) $.
\begin{erwthma}
\item Να αποδείξετε ότι $ \beta=4 $ και $ \gamma=-5 $.
\item Να βρείτε τις εφαπτομένες της $ C_f $ οι οποίες
\begin{alist}
\item είναι κάθετες στην ευθεία $ \zeta: 2x+8y-10=0 $,
\item διέρχονται από το σημείο $ P(-2,-10) $.
\end{alist}
\item Να βρείτε τις κοινές εφαπτομένες των γραφικών παραστάσεων των συναρτήσεων $ f(x) $ και 
\[ g(x)=-x^2+4x-7 \]
\end{erwthma}
\item Μια δεξαμενή αδειάζει προκειμένου να καθαριστεί. Αν η ποσότητα του νερού σε λίτρα δίνεται από τη συνάρτηση $ \Pi(t)=800(8-t)^3 $ με $ t\in[0,8] $, και $ t $ είναι ο χρόνος σε λεπτά από τη στιγμή που άρχισε το άδειασμα της δεξαμενής, να βρείτε το ρυθμό μεταβολής της ποσότητας του νερού όταν
\begin{erwthma}
\item $ t=2\min $,
\item η δεξαμενή έχει $ 6400 $ λίτρα νερό,
\item η δεξαμενή έχει αδειάσει κατά το ήμισυ.
\end{erwthma}
\item Ένα κινητό $ M $ ξεκινά από την αρχή των αξόνων $ O $ και κινείται κατά μήκος της γραφικής παράστασης της συνάρτησης $ f(x)=x^2+2x $ έτσι ώστε η τετμημένη του να αυξάνεται με ρυθμό $ 2\mu/s $. Η προβολή του σημείου $ M $ πάνω στον άξονα $ x'x $ είναι το σημείο $ A $.
\begin{erwthma}
\item Να βρείτε το ρυθμό μεταβολής του εμβαδού του τριγώνου $ OAM $, όταν το σημείο $ M $ έχει τετμημένη ίση με $ \dfrac{1}{2} $.
\item Σε ποιο σημείο της καμπύλης, ο ρυθμός μεταβολής της τεταγμένης του $ M $ ισούται με το ρυθμό μεταβολής της τετμημένης;
\item Έστω $ \varepsilon $ η εφαπτομένη της $ C_f $ στο σημείο $ M $ και $ \Sigma $ το σημείο τομής της με τον άξονα $ y'y $. Να βρείτε το ρυθμό μεταβολής της τεταγμένης του $ \Sigma $, όταν η τετμημένη του $ M $ ισούται με $ -5 $.
\end{erwthma}
\end{thema}
\diarkeia{3}
\kaliepityxia
\end{document}
