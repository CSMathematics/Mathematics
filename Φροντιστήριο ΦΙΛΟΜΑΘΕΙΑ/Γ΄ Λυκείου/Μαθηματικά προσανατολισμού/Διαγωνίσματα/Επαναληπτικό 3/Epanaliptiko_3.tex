\documentclass[ektypwsh]{diag-pan-xelatex}
\usepackage[amsbb]{mtpro2}
\usepackage[no-math,cm-default]{fontspec}
\usepackage{xunicode}
\usepackage{xgreek}
\defaultfontfeatures{Mapping=tex-text,Scale=MatchLowercase}
\setmainfont[Mapping=tex-text,Numbers=Lining,Scale=1.0,BoldFont={Minion Pro Bold}]{Minion Pro}
\newfontfamily\scfont{GFS Artemisia}
\font\icon = "Webdings"
\usepackage{mtpro2}
\usepackage[left=2.00cm, right=2.00cm, top=2.00cm, bottom=3.00cm]{geometry}
\xroma{cyan}
\usepackage{tikz}
\usepackage{tkz-euclide,tkz-fct}
\usepackage{wrapfig}
\usetkzobj{all}
\usepackage{calc}
\usepackage{systeme,regexpatch}
\usepackage[framemethod=TikZ]{mdframed}
\usepackage{adjustbox}
\makeatletter
\def\sysdelim#1#2{\def\SYS@delim@left{#1}\def\SYS@delim@right{#2}}
\sysdelim\{. % reinitialize
% patch the internal command to use
% \LEFTRIGHT<left delim><right delim>{<system>}
% instead of \left<left delim<system>\right<right delim>
\regexpatchcmd\SYS@systeme@iii
  {\cB.\c{SYS@delim@left}(.*)\c{SYS@delim@right}\cE.}
  {\c{SYS@MT@LEFTRIGHT}\cB\{\1\cE\}}
  {}{}
\def\SYS@MT@LEFTRIGHT{%
  \expandafter\expandafter\expandafter\LEFTRIGHT
  \expandafter\SYS@delim@left\SYS@delim@right}
\makeatother
\newcommand{\synt}[2]{{\scriptsize \begin{matrix}
\times#1\\\\ \times#2
\end{matrix}}}

\usepackage{graphicx}
\usepackage{multicol}
\usepackage{multirow}
\usepackage{enumitem}
\usepackage{tabularx}
\usepackage[decimalsymbol=comma]{siunitx}
\begin{document}
\titlos{ΜΑΘΗΜΑΤΙΚΑ ΚΑΤΕΥΘΥΝΣΗΣ Γ΄ ΛΥΚΕΙΟΥ}{ΕΠΑΝΑΛΗΠΤΙΚΟ}
\begin{thema}
\item \mbox{}\\\vspace{-5mm}
\begin{erwthma}
\item 
\monades{12}
\end{erwthma}
\item 
\item Δίνεται συνάρτηση $ f:R \to R $, παραγωγίσιμη, με $ f(0)=1 $ και $ f(x)-a=f'(x)-ax $ για κάθε $ x\in R $ και $ a>0 $.
\begin{erwthma}
\item Να δείξετε ότι $ f(x)=e^x-ax $ και στη συνέχεια να βρείτε την ελάχιστη τιμή της συνάρτησης $ f $ .\monades{6}
\item Να δείξετε ότι η μεγαλύτερη τιμή του $ a>0 $ για την οποία ισχύει $ e^x \geq ax $, για κάθε $ x \in R $, είναι η $ a=e $.\monades{7}
\item Να δείξετε ότι $ 2f(x)<f(x+2015)+f(x-2015) $ για κάθε $ x \in R $.\monades{7}
\item Αν για την τιμή του $ a $ του ερωτήματος 3, ισχύει: $ f(x)+b^x \geq x+2 $
για κάθε $ x \in R $, να δείξετε ότι $ \displaystyle{b=e^e} $.\monades{5}
\end{erwthma}

\item 
\end{thema}
\end{document}

