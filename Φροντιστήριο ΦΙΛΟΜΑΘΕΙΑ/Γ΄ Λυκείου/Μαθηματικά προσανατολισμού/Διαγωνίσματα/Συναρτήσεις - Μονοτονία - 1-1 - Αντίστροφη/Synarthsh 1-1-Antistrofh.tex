\documentclass[11pt,a4paper]{article}
\usepackage[english,greek]{babel}
\usepackage[utf8]{inputenc}
\usepackage{nimbusserif}
\usepackage[T1]{fontenc}
\usepackage[left=2.00cm, right=2.00cm, top=3.00cm, bottom=2.00cm]{geometry}
\usepackage{amsmath}
\let\myBbbk\Bbbk
\let\Bbbk\relax
\usepackage[amsbb,subscriptcorrection,zswash,mtpcal,mtphrb,mtpfrak]{mtpro2}
\usepackage{graphicx,multicol,multirow,enumitem,tabularx,mathimatika,gensymb,venndiagram,hhline,longtable,tkz-euclide,fontawesome5,eurosym,tcolorbox}
\tcbuselibrary{skins,theorems,breakable}
\newlist{rlist}{enumerate}{3}
\setlist[rlist]{itemsep=0mm,label=\roman*.}
\newlist{alist}{enumerate}{3}
\setlist[alist]{itemsep=0mm,label=\alph*.}
\newlist{balist}{enumerate}{3}
\setlist[balist]{itemsep=0mm,label=\bf\alph*.}
\newlist{Alist}{enumerate}{3}
\setlist[Alist]{itemsep=0mm,label=\Alph*.}
\newlist{bAlist}{enumerate}{3}
\setlist[bAlist]{itemsep=0mm,label=\bf\Alph*.}
\renewcommand{\textstigma}{\textsigma\texttau}
\newlist{thema}{enumerate}{3}
\setlist[thema]{label=\bf\large{ΘΕΜΑ \textcolor{black}{\Alph*}},itemsep=0mm,leftmargin=0cm,itemindent=18mm}
\newlist{erwthma}{enumerate}{3}
\setlist[erwthma]{label=\bf{\large{{\Alph{themai}.\arabic*}}},itemsep=0mm,leftmargin=0.8cm}

\newcommand{\lysh}{\textcolor{black}{\textbf{\faCheck\ \ ΛΥΣΗ}}}
\renewcommand{\textstigma}{\textsigma\texttau}
%----------- ΟΡΙΣΜΟΣ------------------
\newcounter{orismos}[section]
\renewcommand{\theorismos}{\thesection.\arabic{orismos}}   
\newcommand{\Orismos}{\refstepcounter{orismos}{\textbf{\textcolor{black}{\kerkissans{Ορισμός\hspace{2mm}\theorismos}}\;:\;}{}}}

\newenvironment{orismos}[1]
{\begin{tcolorbox}[title=\Orismos {\textcolor{black}{\kerkissans{#1}}},breakable,bottomtitle=-1.5mm,
enhanced standard,titlerule=-.2pt,toprule=0pt, rightrule=0pt, bottomrule=0pt,
colback=white,left=2mm,top=1mm,bottom=0mm,
boxrule=0pt,
colframe=white,borderline west={1.5mm}{0pt}{black},leftrule=2mm,sharp corners,coltitle=black]}
{\end{tcolorbox}}

\newcommand{\kerkissans}[1]{{\fontfamily{maksf}\selectfont \textbf{#1}}}
\renewcommand{\textdexiakeraia}{}

\usepackage[
backend=biber,
style=alphabetic,
sorting=ynt
]{biblatex}
\newcommand{\monades}[1]{
\hspace*{\fill}
\textbf{\textit{{Μονάδες #1}}}}
\newcommand{\kaliepityxia}{\vfill
\begin{flushright}
Καλή Επιτυχία!
\end{flushright}}
\newcommand{\swstolathos}{Να χαρακτηρίσετε τις παρακάτω προτάσεις ως σωστές (Σ) ή λανθασμένες (Λ).}
\newcommand{\swstolathosd}{Να χαρακτηρίσετε τις παρακάτω προτάσεις ως σωστές (Σ) ή λανθασμένες (Λ). Στη συνέχεια να διορθώσετε τις προτάσεις που είναι λανθασμένες γράφοντας τη σωστή απάντηση.}
\newcommand{\swstolathospan}{Να χαρακτηρίσετε καθεμία από τις παρακάτω προτάσεις ως \textbf{Σωστή} ή \textbf{Λανθασμένη}.}
\newcommand{\diarkeia}[1]{\begin{flushright}
Διάρκεια εξετάσεων : #1 ώρες.
\end{flushright}}


\newcommand{\titlos}[2]{
\begin{center}
\includegraphics[width=0.4\linewidth]{/home/spyros/texmf/tex/latex/local/frontisthrio/Logotypo-Filomatheia_1}\\
\vspace{-1mm}
{\faIcon{map-marker-alt}} : Ιακώβου Πολυλά 24 - \ Πεζόδρομος\,\,|\,\,{\faIcon{phone-alt}} : 26610 20144\,\,|\,\, {\faIcon{mobile-alt}} : 6932327283 - 6955058444\\
\vspace{-1mm}
\rule{14.7cm}{.1mm}\\
\vspace{2mm}
{\textbf{\today}}\\
\vspace{3mm}
\end{center}

\begin{center}
{\Large\MakeUppercase{\textbf{#1}}}
\vspace{-4mm}
\section*{\huge \textcolor{black}{#2}}
\end{center}
\vspace{5mm}}


\begin{document}
\titlos{Γ΄ Λυκείου - Μαθηματικά προσανατολισμού}{Συνάρτηση $ 1-1 $ - Αντίστροφη}
\begin{thema}
\item \mbox{}\\\vspace{-5mm}
\begin{erwthma}
\item Να δώσετε τον ορισμό της $ 1-1 $ συνάρτησης.
\monades{5}
\item Να δώσετε τον ορισμό της αντίστροφης συνάρτησης $ f^{-1} $ μιας συνάρτησης $ f $.\monades{6}
\item Να αποδείξετε ότι οι γραφικές παραστάσεις $ C_f $ και $ C_{f^{-1}} $ των συναρτήσεων $ f $ και $ f^{-1} $ είναι συμμετρικές ως προς την ευθεία $ y=x $.\monades{6}
\item \swstolathospan
\begin{alist}
\item Αν μια συνάρτηση $ f:A\to\mathbb{R} $ είναι γνησίως μονότονη σε κάθε διάστημα του πεδίου ορισμού της τότε είναι και $ 1-1 $.
\item Έστω $ f $ μια αντιστρέψιμη συνάρτηση. Το σύνολο τιμών της $ f $ είναι το πεδίο ορισμού της $ f^{-1} $.
\item Έστω μια συνάρτηση $ f $ η οποία είναι $ 1-1 $. Οι συναρτήσεις $ (f\circ f^{-1})(x) $ και $ (f^{-1}\circ f)(x) $ είναι μεταξύ τους ίσες.
\item Αν $ f $ είναι μια γνησίως αύξουσα συνάρτηση τότε οι γραφικές παραστάσεις $ C_f $ και $ C_{f^{-1}} $ των συναρτήσεων $ f $ και $ f^{-1} $ αντίστοιχα τέμνονται πάνω στην ευθεία $ y=x $.
\item Οι εξισώσεις $ f(x)=f^{-1}(x) $ και $ f(x)=x $ είναι πάντα ισοδύναμες.
\end{alist}\monades{8}
\end{erwthma}
\item \mbox{}\\
Δίνεται η συνάρτηση $ f:(1,+\infty)\to\mathbb{R} $ με τύπο $ f(x)=\frac{\ln{x}-1}{\ln{x}+1} $.
\begin{erwthma}
\item Να δείξετε ότι η συνάρτηση $ f $ είναι $ 1-1 $.\monades{6}
\item Να ορίσετε τη συνάρτηση $ f\circ g $ όπου $ g(x)=e^x $ και να δείξετε ότι η συνάρτηση $ h=f\circ g $ είναι $ 1-1 $.\monades{7}
\item Να ορίσετε τη συνάρτηση $ h^{-1} $.\monades{7}
\item Να λυθεί η εξίσωση
\[ h^{-1}\left(h(|x|+2)-\frac{1}{2}\right)=1 \]\monades{5}
\end{erwthma}
\item Δίνεται γνησίως μονότονη συνάρτηση $ f:\mathbb{R}\to\mathbb{R} $ της οποίας η γραφική παράσταση διέρχεται από τα σημεία $ A(-1,3) $ και $ B(2,1) $.
\begin{erwthma}
\item Να δείξετε βρείτε το είδος της μονοτονίας της συνάρτησης $ f $ και να δείξετε ότι αντιστρέφεται.\\\monades{5}
\item Να λυθεί η ανίσωση
\[ f\left(-3+f^{-1}(x^2)\right)<3 \]\monades{7}
\item Να λυθεί η εξίσωση
\[ f^{-1}\left(f(\ln{x})-2\right)=2 \]\monades{6}
\item Αν ισχύει η ισότητα
\[ \ln{a}-f^{-1}(a+2)=-a+2 \]
να δείξετε ότι $ a=1 $.\monades{7}
\end{erwthma}
\item Δίνεται γνησίως μονότονη συνάρτηση $ f:\mathbb{R}\to\mathbb{R} $ με σύνολο τιμών το $ \mathbb{R} $ για την οποία ισχύει:
\[ f\left(e^x+2\right)+f(x+3)=x \]
για κάθε $ x\in\mathbb{R} $.
\begin{erwthma}
\item Να δείξετε ότι η $ f $ είναι γνησίως αύξουσα.\monades{7}
\item Να δείξετε ότι η $ C_f $ τέμνει τον άξονα $ x'x $ στο σημείο $ A(3,0) $.\monades{5}
\item Να λύσετε την ανίσωση \[ f\left(6-f^{-1}\left(x^2-4\right)\right)>0 \]\monades{6}
\item Να λύσετε την ανίσωση
\[ f^{-1}(x)-f(3-x)<3 \]\monades{7}
\end{erwthma}
\end{thema}
\diarkeia{3}
\kaliepityxia
\end{document}
