\documentclass[twoside,nofonts,ektypwsh,math,spyros]{frontisthrio-diag}
\usepackage[amsbb,subscriptcorrection,zswash,mtpcal,mtphrb,mtpfrak]{mtpro2}
\usepackage[no-math,cm-default]{fontspec}
\usepackage{amsmath}
\usepackage{xunicode}
\usepackage{xgreek}
\let\hbar\relax
\defaultfontfeatures{Mapping=tex-text,Scale=MatchLowercase}
\setmainfont[Mapping=tex-text,Numbers=Lining,Scale=1.0,BoldFont={Nimbus Roman Bold}]{Nimbus Roman}
\usepackage{fontawesome5}
%\newfontfamily{\FA}{fontawesome.otf}
\xroma{red!70!black}
%------TIKZ - ΣΧΗΜΑΤΑ - ΓΡΑΦΙΚΕΣ ΠΑΡΑΣΤΑΣΕΙΣ ----
\usepackage{tikz,pgfplots}
\usepackage{tkz-euclide}
\usepackage[framemethod=TikZ]{mdframed}
\usetikzlibrary{decorations.pathreplacing}
\tkzSetUpPoint[size=7,fill=white]
%-----------------------
\usepackage{calc,tcolorbox}
\tcbuselibrary{skins,theorems,breakable}
\usepackage{hhline}
\usepackage[explicit]{titlesec}
\usepackage{graphicx}
\usepackage{multicol}
\usepackage{multirow}
\usepackage{tabularx}
\usetikzlibrary{backgrounds}
\usepackage{sectsty}
\sectionfont{\centering}
\usepackage{enumitem}
\usepackage{adjustbox}
\usepackage{mathimatika,gensymb,eurosym,wrap-rl}
\usepackage{systeme,regexpatch}
%-------- ΜΑΘΗΜΑΤΙΚΑ ΕΡΓΑΛΕΙΑ ---------
\usepackage{mathtools}
%----------------------
%-------- ΠΙΝΑΚΕΣ ---------
\usepackage{booktabs}
%----------------------
%----- ΥΠΟΛΟΓΙΣΤΗΣ ----------
\usepackage{calculator}
%----------------------------
%------ ΔΙΑΓΩΝΙΟ ΣΕ ΠΙΝΑΚΑ -------
\usepackage{array}
\newcommand\diag[5]{%
\multicolumn{1}{|m{#2}|}{\hskip-\tabcolsep
$\vcenter{\begin{tikzpicture}[baseline=0,anchor=south west,outer sep=0]
\path[use as bounding box] (0,0) rectangle (#2+2\tabcolsep,\baselineskip);
\node[minimum width={#2+2\tabcolsep-\pgflinewidth},
minimum  height=\baselineskip+#3-\pgflinewidth] (box) {};
\draw[line cap=round] (box.north west) -- (box.south east);
\node[anchor=south west,align=left,inner sep=#1] at (box.south west) {#4};
\node[anchor=north east,align=right,inner sep=#1] at (box.north east) {#5};
\end{tikzpicture}}\rule{0pt}{.71\baselineskip+#3-\pgflinewidth}$\hskip-\tabcolsep}}
%---------------------------------

%---------- ΛΙΣΤΕΣ ----------------------
% Αν μπει το bhma μεσα σε tropo τότε
%\begin{bhma}[leftmargin=.7cm]
\tkzSetUpPoint[size=7,fill=white]
\tikzstyle{pl}=[line width=0.3mm]
\tikzstyle{plm}=[line width=0.4mm]
\usepackage{etoolbox}
\makeatletter
\renewrobustcmd{\anw@true}{\let\ifanw@\iffalse}
\renewrobustcmd{\anw@false}{\let\ifanw@\iffalse}\anw@false
\newrobustcmd{\noanw@true}{\let\ifnoanw@\iffalse}
\newrobustcmd{\noanw@false}{\let\ifnoanw@\iffalse}\noanw@false
\renewrobustcmd{\anw@print}{\ifanw@\ifnoanw@\else\numer@lsign\fi\fi}
\makeatother

\usepackage{path}


\begin{document}
\titlos{Γ΄ Λυκείου - Μαθηματικά Προσανατολισμού}{Η έννοια της συνάρτησης}{Β}
\begin{thema}
\item\mbox{}\\\vspace{-7mm}
\begin{erwthma}
\item Να δώσετε τον ορισμό της πραγματικής συνάρτησης πραγματικής μεταβλητής. \monades{8}
\item Να δώσετε τον ορισμό της γραφικής παράστασης μιας συνάρτησης $ f $.\monades{7}
\item \swstolathospan
\begin{rlist}
\item Το σημείο $ A(3,2) $ ανήκει στη γραφική παράσταση της συνάρτησης $ f(x)=3x-4 $.
\item Η συνάρτηση $ f(x)=x^2,\ D_f=[-3,3) $ είναι άρτια.
\item Η γραφική παράσταση της συνάρτησης $ f(x)=\sqrt{x-1} $ τέμνει τον άξονα $ y'y $.
\item Η γραφική παράσταση μιας περιττής συνάρτησης είναι συμμετρική ως προς την αρχή των αξόνων.
\item Η συνάρτηση $ f(x)=t^2 $ είναι σταθερή συνάρτηση.
\end{rlist}\monades{10}
\end{erwthma}
\item\mbox{}\\ Δίνεται η συνάρτηση $ f $ με τύπο $ f(x)=\frac{x^2-x-2}{\sqrt{x-1}} $.
\begin{erwthma}
\item Να βρεθεί το πεδίο ορισμού της συνάρτησης $ f $.\monades{10}
\item Να βρεθούν οι τιμές $ f(2),f(-3),f(f(5)) $.\monades{6}
\item Να βρεθούν τα σημεία τομής της $ C_f $ με τους άξονες $ x'x $ και $ y'y $.\monades{9}
\end{erwthma}
\item\mbox{}\\ Δίνεται η συνάρτηση $ f $ με τύπο
\[ f(x)=\ccases{ax^2+\beta &, x>1\\
4a(x-2)-3\beta &, x\leq 1} \]
όπου $ a,\beta\in\mathbb{R} $, της οποίας η γραφική παράσταση διέρχεται από το σημείο $ A(3,5) $ ενώ τέμνει την ευθεία $ y=0 $ στο σημείο $ B $ με τετμημένη $ x=-1 $.
\begin{erwthma}
\item Να δείξετε ότι $ a=1 $ και $ \beta=-4 $.\monades{9}
\item Να βρεθούν τα διαστήματα στα οποία η $ C_f $ βρίσκεται κάτω από τον άξονα $ x'x $.\monades{8}
\item Να σχεδιάσετε τη γραφική παράσταση της συνάρτησης $ f $.\monades{8}
\end{erwthma}
\item\mbox{}\\ Δίνονται οι συναρτήσεις $ f,g $ με τύπους
\[ f(x)=x^2-9\ \ \textrm{και}\ \ g(x)=(3-\lambda)x+2\lambda-12 \]
\begin{erwthma}
\item Να βρεθούν τα διαστήματα στα οποία η $ C_f $ δεν βρίσκεται πάνω από τον άξονα $ x'x $.\monades{8}
\item Να βρείτε για ποιες τιμές του $ \lambda $ οι $ C_f $ και $ C_g $ έχουν κοινά σημεία.\monades{9}
\item Για $ \lambda=2 $ να βρεθεί το εμβαδόν που σχηματίζει η $ C_g $ με τους άξονες $ x'x $ και $ y'y $.\monades{8}
\end{erwthma}
\end{thema}
\kaliepityxia
\end{document}
