\documentclass[twoside,nofonts,ektypwsh]{frontisthrio-diag}
\usepackage[amsbb,subscriptcorrection,zswash,mtpcal,mtphrb,mtpfrak]{mtpro2}
\usepackage[no-math,cm-default]{fontspec}
\usepackage{amsmath}
\usepackage{xunicode}
\usepackage{xgreek}
\let\hbar\relax
\defaultfontfeatures{Mapping=tex-text,Scale=MatchLowercase}
\setmainfont[Mapping=tex-text,Numbers=Lining,Scale=1.0,BoldFont={Minion Pro Bold}]{Minion Pro}
\newfontfamily\scfont{GFS Artemisia}
\font\icon = "Webdings"
\usepackage{fontawesome5}
\newfontfamily{\FA}{fontawesome.otf}
\xroma{cyan!70!black}
%------TIKZ - ΣΧΗΜΑΤΑ - ΓΡΑΦΙΚΕΣ ΠΑΡΑΣΤΑΣΕΙΣ ----
\usepackage{tikz,pgfplots}
\usepackage{tkz-euclide}
\usetkzobj{all}
\usepackage[framemethod=TikZ]{mdframed}
\usetikzlibrary{decorations.pathreplacing}
\tkzSetUpPoint[size=7,fill=white]
%-----------------------
\usepackage{calc,tcolorbox}
\tcbuselibrary{skins,theorems,breakable}
\usepackage{hhline}
\usepackage[explicit]{titlesec}
\usepackage{graphicx}
\usepackage{multicol}
\usepackage{multirow}
\usepackage{tabularx}
\usetikzlibrary{backgrounds}
\usepackage{sectsty}
\sectionfont{\centering}
\usepackage{enumitem}
\usepackage{adjustbox}
\usepackage{mathimatika,gensymb,eurosym,wrap-rl}
\usepackage{systeme,regexpatch}
%-------- ΜΑΘΗΜΑΤΙΚΑ ΕΡΓΑΛΕΙΑ ---------
\usepackage{mathtools}
%----------------------
%-------- ΠΙΝΑΚΕΣ ---------
\usepackage{booktabs}
%----------------------
%----- ΥΠΟΛΟΓΙΣΤΗΣ ----------
\usepackage{calculator}
%----------------------------
%------------------------------------------
\newcommand{\tss}[1]{\textsuperscript{#1}}
\newcommand{\tssL}[1]{\MakeLowercase{\textsuperscript{#1}}}
%---------- ΛΙΣΤΕΣ ----------------------
\newlist{bhma}{enumerate}{3}
\setlist[bhma]{label=\bf\textit{\arabic*\textsuperscript{o}\;Βήμα :},leftmargin=0cm,itemindent=1.8cm,ref=\bf{\arabic*\textsuperscript{o}\;Βήμα}}
\newlist{tropos}{enumerate}{3}
\setlist[tropos]{label=\bf\textit{\arabic*\textsuperscript{oς}\;Τρόπος :},leftmargin=0cm,itemindent=2.3cm,ref=\bf{\arabic*\textsuperscript{oς}\;Τρόπος}}
% Αν μπει το bhma μεσα σε tropo τότε
%\begin{bhma}[leftmargin=.7cm]
\tkzSetUpPoint[size=7,fill=white]
\tikzstyle{pl}=[line width=0.3mm]
\tikzstyle{plm}=[line width=0.4mm]
\usepackage{etoolbox}
\makeatletter
\renewrobustcmd{\anw@true}{\let\ifanw@\iffalse}
\renewrobustcmd{\anw@false}{\let\ifanw@\iffalse}\anw@false
\newrobustcmd{\noanw@true}{\let\ifnoanw@\iffalse}
\newrobustcmd{\noanw@false}{\let\ifnoanw@\iffalse}\noanw@false
\renewrobustcmd{\anw@print}{\ifanw@\ifnoanw@\else\numer@lsign\fi\fi}
\makeatother

\usepackage{path}
\pathALa

\begin{document}
\titlos{Γ΄ Λυκείου - Μαθηματικά Προσανατολισμού}{Ισότητα - Πράξεις - Σύνθεση συναρτήσεων}{Β}
\begin{thema}
\item\mbox{}\\\vspace{-7mm}
\begin{erwthma}
\item Πότε δύο συναρτήσεις $ f,g $ με πεδία ορισμού $ D_f,D_g $ αντίστοιχα, λέγονται ίσες;\monades{5}
\item Να δώσετε τον ορισμό της πρόσθεσης $ f+g $ και του πηλίκου $ \frac{f}{g} $ δύο συναρτήσεων $ f,g $.\monades{5}
\item Να ορίσετε τη σύνθεση $ g\circ f $ δύο συναρτήσεων $ f,g $.\monades{5}
\item \swstolathospan
\begin{alist}
\item Η συνάρτηση $ f(x)=\sqrt[3]{x^2} $ ισούται με τη συνάρτηση $ g(x)=x^{\frac{2}{3}} $.
\item Η συνάρτηση $ f(x)=\sqrt[4]{x^3} $ ισούται με τη συνάρτηση $ g(x)=x^{\frac{3}{4}} $.
\item Οι συναρτήσεις $ f\circ g $ και $ g\circ f $ είναι πάντα ίσες.
\item Για κάθε $ x\in(0,+\infty) $ οι συναρτήσεις $ f(x)=2\ln{x} $ και $ g(x)=\ln{x^2} $ είναι ίσες.
\item Η συνάρτηση $ h\circ g $ έχει πεδίο ορισμού το σύνολο $ D_{h\circ g}=\{x\in D_g\ \textrm{και}\ g(x)\in D_h\} $.
\end{alist}\monades{10}
\end{erwthma}
\item\mbox{}\\
Δίνονται οι συναρτήσεις $ f,g $ με τύπους $ f(x)=\frac{1}{x-3} $ και $ g(x)=\sqrt{x+2} $ αντίστοιχα.
\begin{erwthma}
\item Να ορίσετε τη συνάρτηση $ f\circ g $.\monades{9}
\item Να ορίσετε τη συνάρτηση $ f+g $.\monades{6}
\item Να ορίσετε τη συνάρτηση $ f\circ f $ και να εξετάσετε αν είναι ίση με τη συνάρτηση $ h(x)=\frac{3-x}{3x-10} $.\\\monades{10}
\end{erwthma}
\item\mbox{}\\
Δίνονται οι συναρτήσεις $ f,g $ με τύπους $ f(x)=2\ln{\left(x-4\right)} $ και $ g(x)=\ln{\left(x^2-8x+16 \right) } $ αντίστοιχα.
\begin{erwthma}
\item Να εξετάσετε αν οι δύο συναρτήσεις είναι ίσες.\monades{8}
\item Να ορίσετε τη συνάρτηση $ \frac{g}{f} $ και στη συνέχεια να σχεδιάσετε τη γραφική της παράσταση.\monades{8}
\item Να βρεθούν υα σημεία τομής της γραφικής παράστασης της συνάρτησης $ f+g $ με τους άξονες $ x'x $ και $ y'y $.\monades{9}
\end{erwthma}
\item\mbox{}\\
Δίνονται οι συναρτήσεις $ f,g $ με 
\[ f(x)=(a^2-a)x^2-(2\beta+1)x+\beta\ \textrm{και}\ g(x)=2ax^2+(\beta^2-a+1)x-3\beta-11 \]
των οποίων οι γραφικές παραστάσεις τέμνονται στην ευθεία $ x=-1 $.
\begin{erwthma}
\item Να δείξετε ότι $ a=2$ και $\beta=-3 $.\monades{10}
\item Να εξηγήσετε γιατί δεν υπάρχει συνάρτηση $ h:\mathbb{R}\to\mathbb{R} $ ώστε να ισχύει
\[ (h\circ f)(x)=g(x) \]\monades{15}
\end{erwthma}
\end{thema}
\diarkeia{3}
\kaliepityxia
\end{document}
