\documentclass[twoside,nofonts,ektypwsh]{frontisthrio-diag}
\usepackage[amsbb,subscriptcorrection,zswash,mtpcal,mtphrb,mtpfrak]{mtpro2}
\usepackage[no-math,cm-default]{fontspec}
\usepackage{amsmath}
\usepackage{xunicode}
\usepackage{xgreek}
\let\hbar\relax
\defaultfontfeatures{Mapping=tex-text,Scale=MatchLowercase}
\setmainfont[Mapping=tex-text,Numbers=Lining,Scale=1.0,BoldFont={Minion Pro Bold}]{Minion Pro}
\newfontfamily\scfont{GFS Artemisia}
\font\icon = "Webdings"
\usepackage{fontawesome5}
\newfontfamily{\FA}{fontawesome.otf}
\xroma{red!70!black}
%------TIKZ - ΣΧΗΜΑΤΑ - ΓΡΑΦΙΚΕΣ ΠΑΡΑΣΤΑΣΕΙΣ ----
\usepackage{tikz,pgfplots}
\usepackage{tkz-euclide}
\usetkzobj{all}
\usepackage[framemethod=TikZ]{mdframed}
\usetikzlibrary{decorations.pathreplacing}
\tkzSetUpPoint[size=7,fill=white]
%-----------------------
\usepackage{calc,tcolorbox}
\tcbuselibrary{skins,theorems,breakable}
\usepackage{hhline}
\usepackage[explicit]{titlesec}
\usepackage{graphicx}
\usepackage{multicol}
\usepackage{multirow}
\usepackage{tabularx}
\usetikzlibrary{backgrounds}
\usepackage{sectsty}
\sectionfont{\centering}
\usepackage{enumitem}
\usepackage{adjustbox}
\usepackage{mathimatika,gensymb,eurosym,wrap-rl}
\usepackage{systeme,regexpatch}
%-------- ΜΑΘΗΜΑΤΙΚΑ ΕΡΓΑΛΕΙΑ ---------
\usepackage{mathtools}
%----------------------
%-------- ΠΙΝΑΚΕΣ ---------
\usepackage{booktabs}
%----------------------
%----- ΥΠΟΛΟΓΙΣΤΗΣ ----------
\usepackage{calculator}
%----------------------------


%------------------------------------------
\newcommand{\tss}[1]{\textsuperscript{#1}}
\newcommand{\tssL}[1]{\MakeLowercase{\textsuperscript{#1}}}
%---------- ΛΙΣΤΕΣ ----------------------
\newlist{bhma}{enumerate}{3}
\setlist[bhma]{label=\bf\textit{\arabic*\textsuperscript{o}\;Βήμα :},leftmargin=0cm,itemindent=1.8cm,ref=\bf{\arabic*\textsuperscript{o}\;Βήμα}}
\newlist{tropos}{enumerate}{3}
\setlist[tropos]{label=\bf\textit{\arabic*\textsuperscript{oς}\;Τρόπος :},leftmargin=0cm,itemindent=2.3cm,ref=\bf{\arabic*\textsuperscript{oς}\;Τρόπος}}
% Αν μπει το bhma μεσα σε tropo τότε
%\begin{bhma}[leftmargin=.7cm]
\tkzSetUpPoint[size=7,fill=white]
\tikzstyle{pl}=[line width=0.3mm]
\tikzstyle{plm}=[line width=0.4mm]
\usepackage{etoolbox}
\makeatletter
\renewrobustcmd{\anw@true}{\let\ifanw@\iffalse}
\renewrobustcmd{\anw@false}{\let\ifanw@\iffalse}\anw@false
\newrobustcmd{\noanw@true}{\let\ifnoanw@\iffalse}
\newrobustcmd{\noanw@false}{\let\ifnoanw@\iffalse}\noanw@false
\renewrobustcmd{\anw@print}{\ifanw@\ifnoanw@\else\numer@lsign\fi\fi}
\makeatother

\usepackage{path}
\ekthetesdeiktes
\begin{document}
\titlos{Γ΄ Λυκείου}{Συναρτήσεις}{Μαθηματικά Προσανατολισμού}
\begin{thema}
\item \mbox{}\\\vspace{-5mm}
\begin{erwthma}
\item Να δώσετε τον ορισμό μιας γνησίως αύξουσας συνάρτησης.\monades{5}
\item Να δώσετε τον ορισμό μιας συνάρτησης $ 1-1 $.\monades{5}
\item \swstolathospan
\begin{alist}
\item Κάθε συνάρτηση $ 1-1 $ είναι και γνησίως μονότονη.
\item Δύο συναρτήσεις $ f:A\to\mathbb{R} $ και $ g:B\to\mathbb{R} $ είναι ίσες αν ισχύει $ f(x)=g(x) $ για κάθε $ x\in A\cap B $.
\item Εάν για μια συνάρτηση $ f:A\to\mathbb{R} $ ισχύει η σχέση $ f(x)\leq 3 $ για κάθε $ x\in A $ τότε το $ 3 $ είναι το μέγιστο της $ f $.
\item Για οποιεσδήποτε συναρτήσεις $ f,g $ ισχύει $ f\circ g=g\circ f $.
\item Εάν $ f:A\to\mathbb{R} $ είναι μια συνάρτηση $ 1-1 $ τότε το πεδίο ορισμού της $ f^{-1} $ είναι το $ f(A) $.
\end{alist}\monades{10}
\item Πότε λέμε ότι μια συνάρτηση $ f $ παρουσιάζει μέγιστο σε ένα σημείο $ x_0 $ του πεδίου ορισμού της;\\
\monades{5}
\end{erwthma}
\item \mbox{}\\
Δίνεται η συνάρτηση $ f(x)=\ln{x}+x^2+e^x $ με $ x\in(0,+\infty) $.
\begin{erwthma}
\item Να δείξετε ότι η συνάρτηση είναι γνησίως αύξουσα στο διάστημα $ (0,+\infty) $.\monades{8}
\item Να λύσετε την εξίσωση \[ \ln{x}+x^2=1+e-e^x \]\monades{7}
\item Να λύσετε την ανίσωση  \[ \ln{\dfrac{x^2+4}{4x}}+\left( x^2+4\right)^2-16x^2\leq e^{4x}-e^{x^2+4} \] \monades{10}
\end{erwthma}
\item \mbox{}\\
Δίνεται η συνάρτηση $ f(x)=\dfrac{e^x}{e^x+1},\ x\in\mathbb{R} $.
\begin{erwthma}
\item Να αποδείξετε ότι η συνάρτηση είναι αντιστρέψιμη.\monades{7}
\item Να ορίσετε την αντίστροφη συνάρτηση της $ f $.\monades{9}
\item Να λύσετε την ανίσωση
\[ f^{-1}\left(\frac{7}{6}-f(\ln{x}) \right)\geq0  \]\monades{9}
\end{erwthma}
\end{thema}
\kaliepityxia
\end{document}
