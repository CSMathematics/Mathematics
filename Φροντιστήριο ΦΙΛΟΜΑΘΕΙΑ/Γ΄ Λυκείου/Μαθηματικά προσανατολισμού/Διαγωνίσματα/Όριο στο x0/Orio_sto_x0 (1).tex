\documentclass[twoside,nofonts,ektypwsh,math,spyros]{frontisthrio-diag}
\usepackage[amsbb,subscriptcorrection,zswash,mtpcal,mtphrb,mtpfrak]{mtpro2}
\usepackage[no-math,cm-default]{fontspec}
\usepackage{amsmath}
\usepackage{xunicode}
\usepackage{xgreek}
\let\hbar\relax
\defaultfontfeatures{Mapping=tex-text,Scale=MatchLowercase}
\setmainfont[Mapping=tex-text,Numbers=Lining,Scale=1.0,BoldFont={Minion Pro Bold}]{Minion Pro}
\newfontfamily\scfont{GFS Artemisia}
\font\icon = "Webdings"
\usepackage{fontawesome}
\newfontfamily{\FA}{fontawesome.otf}
\xroma{red!70!black}
%------TIKZ - ΣΧΗΜΑΤΑ - ΓΡΑΦΙΚΕΣ ΠΑΡΑΣΤΑΣΕΙΣ ----
\usepackage{tikz,pgfplots}
\usepackage{tkz-euclide}
\usetkzobj{all}
\usepackage[framemethod=TikZ]{mdframed}
\usetikzlibrary{decorations.pathreplacing}
\tkzSetUpPoint[size=7,fill=white]
%-----------------------
\usepackage{calc,tcolorbox}
\tcbuselibrary{skins,theorems,breakable}
\usepackage{hhline}
\usepackage[explicit]{titlesec}
\usepackage{graphicx}
\usepackage{multicol}
\usepackage{multirow}
\usepackage{tabularx}
\usetikzlibrary{backgrounds}
\usepackage{sectsty}
\sectionfont{\centering}
\usepackage{enumitem}
\usepackage{adjustbox}
\usepackage{mathimatika,gensymb,eurosym,wrap-rl}
\usepackage{systeme,regexpatch}
%-------- ΜΑΘΗΜΑΤΙΚΑ ΕΡΓΑΛΕΙΑ ---------
\usepackage{mathtools}
%----------------------
%-------- ΠΙΝΑΚΕΣ ---------
\usepackage{booktabs}
%----------------------
%----- ΥΠΟΛΟΓΙΣΤΗΣ ----------
\usepackage{calculator}
%----------------------------
%------ ΔΙΑΓΩΝΙΟ ΣΕ ΠΙΝΑΚΑ -------
\usepackage{array}
\newcommand\diag[5]{%
\multicolumn{1}{|m{#2}|}{\hskip-\tabcolsep
$\vcenter{\begin{tikzpicture}[baseline=0,anchor=south west,outer sep=0]
\path[use as bounding box] (0,0) rectangle (#2+2\tabcolsep,\baselineskip);
\node[minimum width={#2+2\tabcolsep-\pgflinewidth},
minimum  height=\baselineskip+#3-\pgflinewidth] (box) {};
\draw[line cap=round] (box.north west) -- (box.south east);
\node[anchor=south west,align=left,inner sep=#1] at (box.south west) {#4};
\node[anchor=north east,align=right,inner sep=#1] at (box.north east) {#5};
\end{tikzpicture}}\rule{0pt}{.71\baselineskip+#3-\pgflinewidth}$\hskip-\tabcolsep}}
%---------------------------------
%---- ΟΡΙΖΟΝΤΙΟ - ΚΑΤΑΚΟΡΥΦΟ - ΠΛΑΓΙΟ ΑΓΚΙΣΤΡΟ ------
\newcommand{\orag}[3]{\node at (#1)
{$ \overcbrace{\rule{#2mm}{0mm}}^{{\scriptsize #3}} $};}
\newcommand{\kag}[3]{\node at (#1)
{$ \undercbrace{\rule{#2mm}{0mm}}_{{\scriptsize #3}} $};}
\newcommand{\Pag}[4]{\node[rotate=#1] at (#2)
{$ \overcbrace{\rule{#3mm}{0mm}}^{{\rotatebox{-#1}{\scriptsize$#4$}}}$};}
%-----------------------------------------
%------------------------------------------
\newcommand{\tss}[1]{\textsuperscript{#1}}
\newcommand{\tssL}[1]{\MakeLowercase{\textsuperscript{#1}}}
%---------- ΛΙΣΤΕΣ ----------------------
\newlist{bhma}{enumerate}{3}
\setlist[bhma]{label=\bf\textit{\arabic*\textsuperscript{o}\;Βήμα :},leftmargin=0cm,itemindent=1.8cm,ref=\bf{\arabic*\textsuperscript{o}\;Βήμα}}
\newlist{rlist}{enumerate}{3}
\setlist[rlist]{itemsep=0mm,label=\roman*.}
\newlist{brlist}{enumerate}{3}
\setlist[brlist]{itemsep=0mm,label=\bf\roman*.}
\newlist{tropos}{enumerate}{3}
\setlist[tropos]{label=\bf\textit{\arabic*\textsuperscript{oς}\;Τρόπος :},leftmargin=0cm,itemindent=2.3cm,ref=\bf{\arabic*\textsuperscript{oς}\;Τρόπος}}
% Αν μπει το bhma μεσα σε tropo τότε
%\begin{bhma}[leftmargin=.7cm]
\tkzSetUpPoint[size=7,fill=white]
\tikzstyle{pl}=[line width=0.3mm]
\tikzstyle{plm}=[line width=0.4mm]
\usepackage{etoolbox}
\makeatletter
\renewrobustcmd{\anw@true}{\let\ifanw@\iffalse}
\renewrobustcmd{\anw@false}{\let\ifanw@\iffalse}\anw@false
\newrobustcmd{\noanw@true}{\let\ifnoanw@\iffalse}
\newrobustcmd{\noanw@false}{\let\ifnoanw@\iffalse}\noanw@false
\renewrobustcmd{\anw@print}{\ifanw@\ifnoanw@\else\numer@lsign\fi\fi}
\makeatother
\ekthetesdeiktes
\usepackage{path}
\pathal

\begin{document}
\titlos{Μαθηματικά Προσανατολισμού Γ΄ Λυκείου}{Όριο συνάρτησης σε σημείο}{}
\begin{thema}
\item\mbox{}\\
\vspace{-5mm}
\begin{erwthma}
\item Να αποδείξετε ότι για ένα πολυώνυμο $ P(x)=a_\nu x^\nu+a_{\nu-1}x^{\nu-1}+\ldots+a_1 x+a_0 $ ισχύει ότι
\[ \lim_{x\to x_0}{P(x)}=P(x_0) \]
για κάποιο $ x_0\in\mathbb{R} $.\monades{10}
\item Να διατυπώσετε το κριτήριο παρεμβολής για τον υπολογισμό του ορίου μιας συνάρτησης $ f $ σε ένα σημείο $ x_0 $ του πεδίου ορισμού της.\monades{5}
\item \swstolathospan
\begin{alist}
\item Για οποιεσδήποτε συναρτήσεις $ f,g $ ισχύει πάντα ότι $ {\displaystyle\lim_{x\to x_0}{(f(x)+g(x))}=\lim_{x\to x_0}{f(x)}+\lim_{x\to x_0}{g(x)}} $ όπου $ x_0\in D_f\cap D_g $.
\item Αν για μια συνάρτηση $ f:A\to\mathbb{R} $ υπάρχει το όριο $ {\displaystyle\lim_{x\to x_0}{f(x)}} $ σε κάποιο σημείο $ x_0\in A $ τότε $ {\displaystyle\lim_{x\to x_0^-}{f(x)}=\displaystyle\lim_{x\to x_0^+}{f(x)}} $.
\item Ισχύει $ {\displaystyle\lim_{x\to0}{\frac{\syn{x}-1}{x}}=0} $.
\item Το όριο $ {\displaystyle\lim_{x\to 0}{\frac{|x|}{x}}} $ υπάρχει και ισούται με τη μονάδα.
\item Ισχύει ότι $ {\displaystyle\lim_{x\to x_0}{\frac{P(x)}{Q(x)}}=\frac{P(x_0)}{Q(x_0)}} $.
\end{alist}\monades{10}
\end{erwthma}
\item \mbox{}\\
Να υπολογίσετε τα παρακάτω όρια
\begin{multicols}{3}
\begin{erwthma}
\item $ {\displaystyle\lim_{x\to -2}\dfrac{x^3+x+10}{x^2+2x}} $
\item $ {\displaystyle\lim_{x\to 3}\dfrac{\sqrt{x^2-2x}-\sqrt{3}}{9-x^2}} $
\item $ {\displaystyle\lim_{x\to 2}\dfrac{|x^2-4|+|3x-1|-5}{2-|4-x|}} $
\end{erwthma}
\end{multicols}\monades{8+8+9=25}
\item\mbox{}\\
Δίνεται συνάρτηση $ f:\mathbb{R}\to\mathbb{R} $ για την οποία ισχύει
\[ x^2+x\leq f(x)\leq 12\sqrt{x+3}-22 \]
για κάθε $ x\in[-3,+\infty) $. Να υπολογίσετε τα παρακάτω όρια.
\begin{multicols}{3}
\begin{erwthma}
\item $ {\displaystyle{\lim_{x\to 1}{f(x)}}} $
\item $ {\displaystyle{\lim_{x\to 1}{\frac{f(x)-f(1)}{x-1}}}} $
\item $ {\displaystyle{\lim_{x\to 1}{\frac{f(x)-\hm{(x-1)}-2}{x^2-1}}}} $
\end{erwthma}
\end{multicols}\monades{7+9+9=25}
\item \mbox{}\\
Δίνεται συνάρτηση $ f:\mathbb{R}\to\mathbb{R} $ για την οποία ισχύει η παρακάτω σχέση:
\[ f^2(x)+x^2\leq 4f(x)+5 \]
για κάθε $ x\in\mathbb{R} $.
\begin{erwthma}
\item Να υπολογίσετε το όριο $ {\displaystyle{\lim_{x\to 3}{f(x)}}} $.\monades{10}
\item Αν ισχύει ότι 
\[ \lim_{x\to 3}{\frac{af^2(x)+\beta-f(x)}{f(x)^2-4}}=1 \]
τότε να υπολογίσετε τις παραμέτρους $ a $ και $ \beta $.\monades{15}
\end{erwthma}
\end{thema}
\kaliepityxia
\end{document}
