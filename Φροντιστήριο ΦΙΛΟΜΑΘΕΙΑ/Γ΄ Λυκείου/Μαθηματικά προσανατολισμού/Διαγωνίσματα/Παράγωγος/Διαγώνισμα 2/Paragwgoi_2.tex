\documentclass[ektypwsh]{frontisthrio-diag}
\usepackage[amsbb,subscriptcorrection,zswash,mtpcal,mtphrb,mtpfrak]{mtpro2}
\usepackage[no-math]{fontspec}
\usepackage{amsmath}
\usepackage{xunicode}
\usepackage{xgreek}
\let\hbar\relax
\defaultfontfeatures{Mapping=tex-text,Scale=MatchLowercase}
\setmainfont[Mapping=tex-text,Numbers=Lining,Scale=1.0,BoldFont={Minion Pro Bold}]{Minion Pro}
\newfontfamily\scfont{GFS Artemisia}
\font\icon = "Webdings"
\usepackage{fontawesome5}
\newfontfamily{\FA}{fontawesome.otf}
\xroma{red!70!black}
%------TIKZ - ΣΧΗΜΑΤΑ - ΓΡΑΦΙΚΕΣ ΠΑΡΑΣΤΑΣΕΙΣ ----
\usepackage{tikz,pgfplots}
\usepackage{tkz-euclide}
\usetkzobj{all}
\usepackage[framemethod=TikZ]{mdframed}
\usetikzlibrary{decorations.pathreplacing}
\tkzSetUpPoint[size=7,fill=white]
%-----------------------
\usepackage{calc,tcolorbox}
\tcbuselibrary{skins,theorems,breakable}
\usepackage{hhline}
\usepackage[explicit]{titlesec}
\usepackage{graphicx}
\usepackage{multicol}
\usepackage{multirow}
\usepackage{tabularx}
\usetikzlibrary{backgrounds}
\usepackage{sectsty}
\sectionfont{\centering}
\usepackage{enumitem}
\usepackage{adjustbox}
\usepackage{mathimatika,gensymb,eurosym,wrap-rl}
\usepackage{systeme,regexpatch}
%-------- ΜΑΘΗΜΑΤΙΚΑ ΕΡΓΑΛΕΙΑ ---------
\usepackage{mathtools}
%----------------------
%-------- ΠΙΝΑΚΕΣ ---------
\usepackage{booktabs}
%----------------------
%----- ΥΠΟΛΟΓΙΣΤΗΣ ----------
\usepackage{calculator}
%----------------------------

%------------------------------------------
\newcommand{\tss}[1]{\textsuperscript{#1}}
\newcommand{\tssL}[1]{\MakeLowercase{\textsuperscript{#1}}}
%---------- ΛΙΣΤΕΣ ----------------------
\newlist{bhma}{enumerate}{3}
\setlist[bhma]{label=\bf\textit{\arabic*\textsuperscript{o}\;Βήμα :},leftmargin=0cm,itemindent=1.8cm,ref=\bf{\arabic*\textsuperscript{o}\;Βήμα}}
\newlist{tropos}{enumerate}{3}
\setlist[tropos]{label=\bf\textit{\arabic*\textsuperscript{oς}\;Τρόπος :},leftmargin=0cm,itemindent=2.3cm,ref=\bf{\arabic*\textsuperscript{oς}\;Τρόπος}}
% Αν μπει το bhma μεσα σε tropo τότε
%\begin{bhma}[leftmargin=.7cm]
\tkzSetUpPoint[size=7,fill=white]
\tikzstyle{pl}=[line width=0.3mm]
\tikzstyle{plm}=[line width=0.4mm]
\usepackage{etoolbox}
\makeatletter
\renewrobustcmd{\anw@true}{\let\ifanw@\iffalse}
\renewrobustcmd{\anw@false}{\let\ifanw@\iffalse}\anw@false
\newrobustcmd{\noanw@true}{\let\ifnoanw@\iffalse}
\newrobustcmd{\noanw@false}{\let\ifnoanw@\iffalse}\noanw@false
\renewrobustcmd{\anw@print}{\ifanw@\ifnoanw@\else\numer@lsign\fi\fi}
\makeatother

\usepackage{path}

\begin{document}
\titlos{Μαθηματικά προσανατολισμού Γ΄ Λυκείου}{Παράγωγοι}{B}
\begin{thema}
\item\mbox{}\\
\vspace{-7mm}
\begin{erwthma}
\item Να αποδείξετε ότι αν μια συνάρτηση $ f $ είναι παραγωγίσιμη σε ένα σημείο $ x_0 $ του πεδίου ορισμού της, τότε είναι και συνεχής στο σημείο αυτό.\monades{5}
\item Να δώσετε τον ορισμό της παραγωγίσιμης συνάρτησης σε ένα σημείο του πεδίου ορισμού της.\\\monades{5}
\item Να δείξετε ότι $ (\sqrt{x})'=\frac{1}{2\sqrt{x}} $.\monades{5}
\item \swstolathospan
\begin{alist}
\item Αν μια συνάρτηση $ f $ είναι συνεχής σε ένα σημείο $ x_0 $ του πεδίου ορισμού της τότε είναι και παραγωγίσιμη σ' αυτό.
\item Αν μια συνάρτηση $ f $ δεν είναι συνεχής σε ένα σημείο $ x_0 $ του πεδίου ορισμού της τότε δεν είναι παραγωγίσιμη σ' αυτό.
\item Το πεδίο ορισμού της παραγώγου μιας συνάρτησης $ f $ είναι υποσύνολο του πεδίου ορισμού της $ f $.
\item Η παράγωγος του γινομένου δύο συναρτήσεων $ f,g $ ισούται με $ (f(x)\cdot g(x))'=f'(x)\cdot g'(x) $.
\item Ισχύει ότι $ \left(x^x\right)'=x\cdot x^{x-1} $.
\end{alist}
\end{erwthma}\monades{10}
\item\mbox{}\\
Να υπολογίσετε τις παραγώγους των παρακάτω συναρτήσεων.
\begin{multicols}{2}
\begin{erwthma}
\item $ f(x)=x^2\cdot\hm{x} $
\item $ f(x)=\dfrac{e^x}{\syn{x}} $
\item $ f(x)=\ln{(\syn{x})},\ x\in\left(0,\frac{\pi}{2} \right)  $
\item $ f(x)=3^x\cdot\ln{x} $
\item $ f(x)=(x^2-1)^x $
\item $ f(x)=\sqrt[3]{x^5} $
\item $ f(x)=\sqrt[3]{(2-x)^4}  $
\item $ f(x)=|x-3|+2x $
\end{erwthma}
\end{multicols}\monades{25}
\item\mbox{}\\
Δίνεται η συνάρτηση $ f:(0,+\infty)\to\mathbb{R} $ με τύπο
\[ f(x)=\frac{\ln{x}+1}{x} \]
\begin{erwthma}
\item Να βρεθεί η παράγωγος της $ f $.\monades{8}
\item Να βρεθεί η εξίσωση της εφαπτομένης της γραφικής παράστασης της $f$ η οποία έχει κλίση $\lambda=0$.\\\monades{9}
\item Να βρεθεί η εξίσωση της εφαπτομένης της γραφικής παράστασης της $f$ στο σημείο $A(e,f(e))$.\\\monades{8}
\end{erwthma} 
\item\mbox{}\\
Δίνεται οι συναρτήσεις $f(x)=a\syn{x}$ και $g(x)=\beta x^3-x$, με $a,\beta\in\mathbb{R}$, για τις οποίες ισχύει $f'\left(\frac{\pi}{6}\right)$ και $g''(2)=24$.
\begin{erwthma}
\item Να δείξετε ότι $a=4$ και $\beta=2$.\monades{8}
\item Να υπολογίσετε το όριο 
\[\lim_{x\to 0}{\frac{f'(x)}{g(x)}} \]\monades{8}
\item Να δείξετε ότι η εφαπτομένη της γραφικής παράστασης της $g$ στο σημείο $A(1,g(1))$ σχηματίζει με τους άξονες τρίγωνο $OAB$ το οποίο έχει εμβαδόν $(OAB)=1.6$.\monades{9}
\end{erwthma}
\end{thema}
\end{document}
