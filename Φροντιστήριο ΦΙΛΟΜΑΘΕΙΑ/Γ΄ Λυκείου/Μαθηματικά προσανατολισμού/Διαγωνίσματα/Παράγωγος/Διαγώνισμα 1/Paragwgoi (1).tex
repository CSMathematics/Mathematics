\documentclass[twoside,nofonts,internet,math,spyros]{frontisthrio-diag}
\usepackage[amsbb,subscriptcorrection,zswash,mtpcal,mtphrb,mtpfrak]{mtpro2}
\usepackage[no-math,cm-default]{fontspec}
\usepackage{amsmath}
\usepackage{xunicode}
\usepackage{xgreek}
\let\hbar\relax
\defaultfontfeatures{Mapping=tex-text,Scale=MatchLowercase}
\setmainfont[Mapping=tex-text,Numbers=Lining,Scale=1.0,BoldFont={Minion Pro Bold}]{Minion Pro}
\newfontfamily\scfont{GFS Artemisia}
\font\icon = "Webdings"
\usepackage{fontawesome}
\newfontfamily{\FA}{fontawesome.otf}
\xroma{red!70!black}
%------TIKZ - ΣΧΗΜΑΤΑ - ΓΡΑΦΙΚΕΣ ΠΑΡΑΣΤΑΣΕΙΣ ----
\usepackage{tikz,pgfplots}
\usepackage{tkz-euclide}
\usetkzobj{all}
\usepackage[framemethod=TikZ]{mdframed}
\usetikzlibrary{decorations.pathreplacing}
\tkzSetUpPoint[size=7,fill=white]
%-----------------------
\usepackage{calc,tcolorbox}
\tcbuselibrary{skins,theorems,breakable}
\usepackage{hhline}
\usepackage[explicit]{titlesec}
\usepackage{graphicx}
\usepackage{multicol}
\usepackage{multirow}
\usepackage{tabularx}
\usetikzlibrary{backgrounds}
\usepackage{sectsty}
\sectionfont{\centering}
\usepackage{enumitem}
\usepackage{adjustbox}
\usepackage{mathimatika,gensymb,eurosym,wrap-rl}
\usepackage{systeme,regexpatch}
%-------- ΜΑΘΗΜΑΤΙΚΑ ΕΡΓΑΛΕΙΑ ---------
\usepackage{mathtools}
%----------------------
%-------- ΠΙΝΑΚΕΣ ---------
\usepackage{booktabs}
%----------------------
%----- ΥΠΟΛΟΓΙΣΤΗΣ ----------
\usepackage{calculator}
%----------------------------
%------ ΔΙΑΓΩΝΙΟ ΣΕ ΠΙΝΑΚΑ -------
\usepackage{array}
\newcommand\diag[5]{%
\multicolumn{1}{|m{#2}|}{\hskip-\tabcolsep
$\vcenter{\begin{tikzpicture}[baseline=0,anchor=south west,outer sep=0]
\path[use as bounding box] (0,0) rectangle (#2+2\tabcolsep,\baselineskip);
\node[minimum width={#2+2\tabcolsep-\pgflinewidth},
minimum  height=\baselineskip+#3-\pgflinewidth] (box) {};
\draw[line cap=round] (box.north west) -- (box.south east);
\node[anchor=south west,align=left,inner sep=#1] at (box.south west) {#4};
\node[anchor=north east,align=right,inner sep=#1] at (box.north east) {#5};
\end{tikzpicture}}\rule{0pt}{.71\baselineskip+#3-\pgflinewidth}$\hskip-\tabcolsep}}
%---------------------------------
%---- ΟΡΙΖΟΝΤΙΟ - ΚΑΤΑΚΟΡΥΦΟ - ΠΛΑΓΙΟ ΑΓΚΙΣΤΡΟ ------
\newcommand{\orag}[3]{\node at (#1)
{$ \overcbrace{\rule{#2mm}{0mm}}^{{\scriptsize #3}} $};}
\newcommand{\kag}[3]{\node at (#1)
{$ \undercbrace{\rule{#2mm}{0mm}}_{{\scriptsize #3}} $};}
\newcommand{\Pag}[4]{\node[rotate=#1] at (#2)
{$ \overcbrace{\rule{#3mm}{0mm}}^{{\rotatebox{-#1}{\scriptsize$#4$}}}$};}
%-----------------------------------------
%------------------------------------------
\newcommand{\tss}[1]{\textsuperscript{#1}}
\newcommand{\tssL}[1]{\MakeLowercase{\textsuperscript{#1}}}
%---------- ΛΙΣΤΕΣ ----------------------
\newlist{bhma}{enumerate}{3}
\setlist[bhma]{label=\bf\textit{\arabic*\textsuperscript{o}\;Βήμα :},leftmargin=0cm,itemindent=1.8cm,ref=\bf{\arabic*\textsuperscript{o}\;Βήμα}}
\newlist{rlist}{enumerate}{3}
\setlist[rlist]{itemsep=0mm,label=\roman*.}
\newlist{brlist}{enumerate}{3}
\setlist[brlist]{itemsep=0mm,label=\bf\roman*.}
\newlist{tropos}{enumerate}{3}
\setlist[tropos]{label=\bf\textit{\arabic*\textsuperscript{oς}\;Τρόπος :},leftmargin=0cm,itemindent=2.3cm,ref=\bf{\arabic*\textsuperscript{oς}\;Τρόπος}}
% Αν μπει το bhma μεσα σε tropo τότε
%\begin{bhma}[leftmargin=.7cm]
\tkzSetUpPoint[size=7,fill=white]
\tikzstyle{pl}=[line width=0.3mm]
\tikzstyle{plm}=[line width=0.4mm]
\usepackage{etoolbox}
\makeatletter
\renewrobustcmd{\anw@true}{\let\ifanw@\iffalse}
\renewrobustcmd{\anw@false}{\let\ifanw@\iffalse}\anw@false
\newrobustcmd{\noanw@true}{\let\ifnoanw@\iffalse}
\newrobustcmd{\noanw@false}{\let\ifnoanw@\iffalse}\noanw@false
\renewrobustcmd{\anw@print}{\ifanw@\ifnoanw@\else\numer@lsign\fi\fi}
\makeatother

\usepackage{path}
\pathal

\begin{document}
\titlos{Μαθηματικά προσανατολισμού Γ΄ Λυκείου}{Διαγώνισμα}{Παράγωγοι}
\begin{thema}
\item\mbox{}\\
\vspace{-7mm}
\begin{erwthma}
\item Να αποδείξετε ότι αν μια συνάρτηση $ f $ είναι παραγωγίσιμη σε ένα σημείο $ x_0 $ του πεδίου ορισμού της, τότε είναι και συνεχής στο σημείο αυτό.\monades{10}
\item Να δώσετε τον ορισμό της παραγωγίσιμης συνάρτησης σε ένα σημείο του πεδίου ορισμού της.\\\monades{5}
\item \swstolathospan
\begin{alist}
\item Αν μια συνάρτηση $ f $ είναι συνεχής σε ένα σημείο $ x_0 $ του πεδίου ορισμού της τότε είναι και παραγωγίσιμη σ' αυτό.
\item Αν μια συνάρτηση $ f $ δεν είναι συνεχής σε ένα σημείο $ x_0 $ του πεδίου ορισμού της τότε δεν είναι παραγωγίσιμη σ' αυτό.
\item Το πεδίο ορισμού της παραγώγου μιας συνάρτησης $ f $ είναι υποσύνολο του πεδίου ορισμού της $ f $.
\item Η παράγωγος του γινομένου δύο συναρτήσεων $ f,g $ ισούται με $ (f(x)\cdot g(x))'=f'(x)\cdot g'(x) $.
\item Ισχύει ότι $ \left(x^x\right)'=x\cdot x^{x-1} $.
\end{alist}
\end{erwthma}\monades{10}
\item\mbox{}\\
Να υπολογίσετε τις παραγώγους των παρακάτω συναρτήσεων.
\begin{multicols}{2}
\begin{erwthma}
\item $ f(x)=x^2\cdot\hm{x} $
\item $ f(x)=\dfrac{e^x\cdot x}{\syn{x}} $
\item $ f(x)=\ln{(\syn{x})},\ x\in\left(0,\frac{\pi}{2} \right)  $
\item $ f(x)=3^x\cdot\ln{x} $
\item $ f(x)=(x^2-1)^x $
\item $ f(x)=e^x\cdot x^3\cdot\ln{x} $
\item $ f(x)=\sqrt[3]{x^5} $
\item $ f(x)=\sqrt[3]{(2-x)^4}  $
\item $ f(x)=|x-3|+2x $
\item $ f(x)=\hm^3{(x^2-x)} $
\end{erwthma}
\end{multicols}\monades{25}
\item\mbox{}\\
Δίνεται η συνάρτηση $ f:(0,+\infty)\to\mathbb{R} $ με τύπο
\[ f(x)=\frac{\ln{x}+1}{x} \]
\begin{erwthma}
\item Να βρεθεί η παράγωγος της $ f $.\monades{8}
\item Να λυθεί η ανίσωση $ f'(x)<0 $.\monades{7}
\item Να δειχθεί ότι η γραφική παράσταση της $ f'' $ τέμνει τον οριζόντιο άξονα σε ένα σημείο ακριβώς.\\\monades{10}
\end{erwthma} 
\end{thema}
\end{document}
