\documentclass[ektypwsh]{diag-pan-xelatex}
\usepackage[amsbb]{mtpro2}
\usepackage[no-math,cm-default]{fontspec}
\usepackage{xunicode}
\usepackage{xltxtra}
\usepackage{xgreek}
\defaultfontfeatures{Mapping=tex-text,Scale=MatchLowercase}
\setmainfont[Mapping=tex-text,Numbers=Lining,Scale=1.0,BoldFont={Minion Pro Bold}]{Minion Pro}
\newfontfamily\scfont{GFS Artemisia}
\font\icon = "Webdings"
\usepackage{mtpro2}
\usepackage{amsmath}
\xroma{cyan}
\usepackage{tikz}
\usepackage{tkz-euclide,tkz-fct}
\usepackage{wrapfig}
\usetkzobj{all}
\usepackage{calc}
\usepackage{systeme,regexpatch}
\usepackage[framemethod=TikZ]{mdframed}
\usepackage{adjustbox}
\makeatletter
% change the definition of \sysdelim not to store `\left` and `\right`
\def\sysdelim#1#2{\def\SYS@delim@left{#1}\def\SYS@delim@right{#2}}
\sysdelim\{. % reinitialize

% patch the internal command to use
% \LEFTRIGHT<left delim><right delim>{<system>}
% instead of \left<left delim<system>\right<right delim>
\regexpatchcmd\SYS@systeme@iii
  {\cB.\c{SYS@delim@left}(.*)\c{SYS@delim@right}\cE.}
  {\c{SYS@MT@LEFTRIGHT}\cB\{\1\cE\}}
  {}{}
\def\SYS@MT@LEFTRIGHT{%
  \expandafter\expandafter\expandafter\LEFTRIGHT
  \expandafter\SYS@delim@left\SYS@delim@right}
\makeatother
\newcommand{\synt}[2]{{\scriptsize \begin{matrix}
\times#1\\\\ \times#2
\end{matrix}}}

\usepackage{graphicx}
\usepackage{multicol}
\usepackage{multirow}
\usepackage{enumitem}
\usepackage{tabularx}
\usepackage[decimalsymbol=comma]{siunitx}
\newlist{rlist}{enumerate}{3}
\setlist[rlist]{itemsep=0mm,label=\roman*.}
%---------------------------------
\makeatletter
\renewcommand*{\@alph}[1]{%
  \ifcase#1\or α\or β\or γ\or
    δ\or ε\or ζ\or η\or θ\or ι\or κ\or
    λ\or μ\or ν\or ξ\or ο\or π\or ρ\or σ\or
    τ\or υ\or φ\or χ\or ψ\or
    ω\else\@ctrerr\fi
}
\renewcommand*{\@Alph}[1]{%
  \ifcase#1\or Α\or Β\or Γ\or
    Δ\or Ε\or Ζ\or Η\or Θ\or Ι\or Κ\or
    Λ\or Μ\or Ν\or Ξ\or Ο\or Π\or Ρ\or Σ\or
    Τ\or Υ\or Φ\or Χ\or Ψ\or
    Ω\else\@ctrerr\fi
}
\makeatother
%--------------------------------

\begin{document}
\titlos{ΜΑΘΗΜΑΤΙΚΑ ΚΑΤΕΥΘΥΝΣΗΣ Γ΄ ΛΥΚΕΙΟΥ}{Θ. ROLLE - Θ.Μ.Τ. - ΣΥΝΕΠΕΙΕΣ}
\begin{thema}
\item\mbox{}\\\vspace{-5mm}
\begin{erwthma}
\item Να αποδειχθεί οτι αν για μια συνάρτηση $ f:A\rightarrow\mathbb{R} $ ισχύει $ f'(x)\neq0 $ για κάθε $ x\in A $ τότε η αυτή είναι $ 1-1 $.
\monades{10}
\item Να διατυπώσετε τη γεωμετρική ερμηνεία του Θ.Μ.Τ. για μια συνάρτηση $ f $ ορισμένη σε ένα κλειστό διάστημα $ [a,\beta] $.\monades{5}
\item Να χαρακτηρίσετε τις παρακάτω προτάσεις ως σωστές (Σ) ή λανθασμένες (Λ).
\begin{rlist}
\item Αν για μια παραγωγίσιμη συνάρτηση $ f:[a,\beta]\rightarrow\mathbb{R} $ έχουμε οτι $ f(a)=f(\beta) $ τότε θα υπάρχει $ x_0\in(a,\beta) $ ώστε $ f'(x_0)=0 $.
\item Έστω μια συνάρτηση $ f:[a,\beta]\rightarrow\mathbb{R} $ συνεχής και παραγωγίσιμη σ' αυτό. Τότε θα υπάρχει τουλάχιστον ένα $ \xi\in(a,\beta) $ ώστε να ισχύει $ f(\beta)-f(a)=f'(\xi)(\beta-a) $.
\item Δίνεται η παραγωγίσιμη συνάρτηση $ f[0,+\infty)\rightarrow\mathbb{R} $ με γνησίως φθίνουσα παράγωγο. Αν $ f(0)=0 $ τότε θα ισχύει $ f(x)>xf'(x) $ για κάθε $ x>0 $.
\item Αν $ f,g $ είναι δύο παραγωγίσιμες συναρτήσεις με $ f'(x)=g'(x) $ τότε θα ισχύει $ f'(x)=g'(x) $.
\item Μεταξύ δύο διαδοχικών ριζών μιας παραγωγίσιμης συνάρτησης $ f:A\rightarrow\mathbb{R} $ υπάρχει τουλάχιστον ένα σημείο στο οποίο η εφαπτόμενη ευθεία να είναι παράλληλη με τον οριζόντιο άξονα.\\\monades{10}
\end{rlist}
\end{erwthma}
\item\mbox{}\\
Δίνεται η συνάρτηση $ f(x)=\ln\left( \frac{1}{x}\right)  $ και $ a,\beta $ δύο αριθμοί του πεδίου ορισμού της με $ 1<a<\beta $. Να δειχθεί οτι:
\begin{erwthma}
\item H παράγωγος της συνάρτησης $ f $ είναι γνησίως αύξουσα στο διάστημα $ [a,\beta] $.\monades{10}
\item Ισχύει η παρακάτω σχέση :
\[ \frac{1}{\beta}<\frac{\ln\left(\frac{a}{\beta}\right) }{a-\beta}<\frac{1}{a} \]\monades{15}
\end{erwthma}
\item \mbox{}\\
\vspace{-5mm}
\begin{erwthma}
\item Δίνεται η δύο φορές παραγωγίσιμη συνάρτηση $ f:(0,+\infty)\rightarrow\mathbb{R} $ με $ f(1)=0 $ και $ f'(1)=1 $ για την οποία ισχύει η σχέση \[ x^2f''(x)+=6x-2f(x)+4xf'(x) \]
Να βρεθεί ο τύπος της συνάρτησης $ f $.\monades{8}
\item Δίνεται η παραγωγίσιμη συνάρτηση $ f:(0,+\infty)\rightarrow\mathbb{R} $ και $ a,\beta $ δύο σημεία του πεδίου ορισμού της με $ a<b $. Αν η $ f $ έχει ρίζα το $ \beta $ να δειχθεί οτι υπάρχει τουλάχιστον ένα $ \xi\in(a,\beta) $ ώστε να ισχύει : \[ (x-a)f'(\xi)=f(x) \]\monades{8}
\item Έστω μια συνάρτηση $ f:\mathbb{R}\rightarrow\mathbb{R} $ παραγωγίσιμη για την οποία ισχύει \monades{9}
\end{erwthma}
\item \mbox{}\\
Δίνεται μια συνάρτηση $ f:(0,+\infty) $ συνεχής και παραγωγίσιμη στο πεδίο ορισμού της για την οποία ισχύουν οι σχέσεις $ f(\sqrt{2})=0 $, $ f'(\sqrt{2})=-\sqrt{2} $ και $ f''(x)=e^{f(x)} $ για κάθε $ x>0 $.
\begin{erwthma}
\item Να δειχθεί οτι  $ \left[ f'(x)\right]^2=2f''(x)  $\monades{8}
\item Να δειχθεί οτι $ f'(x)\neq0 $ για κάθε $ x\in\mathbb{R} $.\monades{8}
\item Να δειχθεί οτι $ f(x)=\ln{\left( \dfrac{2}{x^2}\right) } $ για κάθε $ x>0 $.\monades{9}
\end{erwthma}
\end{thema}

\end{document}
