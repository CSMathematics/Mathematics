\documentclass[11pt,a4paper]{article}
\usepackage[utf8]{inputenc}
\usepackage{nimbusserif}
\usepackage[T1]{fontenc}
\usepackage[english,greek]{babel}
\usepackage{amsmath} 
\let\myBbbk\Bbbk 
\let\Bbbk\relax 
\usepackage[amsbb,subscriptcorrection,zswash,mtpcal,mtphrb,mtpfrak]{mtpro2}
\usepackage[left=2.00cm, right=2.00cm, top=2.00cm, bottom=2.00cm]{geometry}
%------TIKZ - ΣΧΗΜΑΤΑ - ΓΡΑΦΙΚΕΣ ΠΑΡΑΣΤΑΣΕΙΣ ---- 
\usepackage{tikz,pgfplots,tkz-tab} 
\usepackage{tkz-euclide} 
\usepackage[framemethod=TikZ]{mdframed} 
\usetikzlibrary{decorations.pathreplacing} 
\tkzSetUpPoint[size=2.9,fill=white]
%----------------------- 
\usepackage{calc,tcolorbox} 
\tcbuselibrary{skins,theorems,breakable} 
\usepackage{hhline} 
\usepackage[explicit]{titlesec} 
\usepackage{graphicx} 
\usepackage{multicol} 
\usepackage{multirow} 
\usepackage{tabularx} 
\usetikzlibrary{backgrounds} 
\usepackage{sectsty} 
\sectionfont{\centering} 
\usepackage{enumitem} 
\usepackage{adjustbox} 
\usepackage{mathimatika,gensymb,eurosym,wrap-rl} 
\usepackage{systeme,regexpatch} 
%-------- ΜΑΘΗΜΑΤΙΚΑ ΕΡΓΑΛΕΙΑ --------- 
\usepackage{mathtools} 
%---------------------- 
%-------- ΠΙΝΑΚΕΣ --------- 
\usepackage{booktabs} 
%---------------------- 
%----- ΥΠΟΛΟΓΙΣΤΗΣ ---------- 
\usepackage{calculator} 
%---------------------------- 
%------------------------------------------ 
\newcommand{\tss}[1]{\textsuperscript{#1}} 
\newcommand{\tssL}[1]{\MakeLowercase{\textsuperscript{#1}}} 
\tikzstyle{pl}=[line width=0.3mm] 
\tikzstyle{plm}=[line width=0.4mm] 
\usepackage{etoolbox} 
\makeatletter 
\renewrobustcmd{\anw@true}{\let\ifanw@\iffalse} 
\renewrobustcmd{\anw@false}{\let\ifanw@\iffalse}\anw@false 
\newrobustcmd{\noanw@true}{\let\ifnoanw@\iffalse} 
\newrobustcmd{\noanw@false}{\let\ifnoanw@\iffalse}\noanw@false 
\renewrobustcmd{\anw@print}{\ifanw@\ifnoanw@\else\numer@lsign\fi\fi} 
\makeatother
\newlist{alist}{enumerate}{3}
\setlist[alist]{itemsep=0mm,label=\alph*.}
\newlist{rlist}{enumerate}{3}
\setlist[rlist]{itemsep=0mm,label=\roman*.}
\newlist{balist}{enumerate}{3}
\setlist[balist]{itemsep=0mm,label=\bf\alph*.}
\newlist{Alist}{enumerate}{3}
\setlist[Alist]{itemsep=0mm,label=\Alph*.}
\newlist{bAlist}{enumerate}{3}
\setlist[bAlist]{itemsep=0mm,label=\bf\Alph*.}
\renewcommand{\textstigma}{\textsigma\texttau}
\makeatletter
\xpatchcmd{\tkzTabLine}
{\node at (Z\thetkz@cnt@impair\thetkz@cnt@lg){$0$};} % search
{\node[fill=white,inner sep=.5mm] at (Z\thetkz@cnt@impair\thetkz@cnt@lg){$0$};} % replace
{}{}
\makeatother
\newcommand{\en}[1]{\selectlanguage{english}{#1}\selectlanguage{greek}}
\newcommand{\roloi}[4][]{
\draw[line width=.5mm,#1] (0,0) circle(2);
\foreach \n in {1,2,...,12}{
\tkzDefPoint(30*\n-90:2){A_\n}
%\tkzDrawPoint(A_\n)
\node at (-30*\n+90:1.65){\n};}
\draw[plm,,#1] (0,0)--(90-30*#2-0.5*#3:1);
\draw[pl,#1] (0,0)--(90-6*#3-0.1*#4:1.5);
\draw[#1](0,0)--(90-6*#4:1.2);
\tkzDrawPoint[fill=#1,color=#1](0,0)
\foreach \s in {1,2,...,12}{
\draw[#1](90-30*\s:1.85)--(90-30*\s:2);}
\foreach \t in {1,2,...,60}{
\draw[#1](90-6*\t:1.93)--(90-6*\t:2);}}


\begin{document}
\begin{enumerate}


\item
%@ Κωδικός: Ana-Paragogos-ParagGin-AA1
%@ Ενότητα: Παράγωγος συνάρτηση
%@ Είδος: Παράγωγος γινομένου
%@ Δυσκολία: 1
Για καθεμία από τις παρακάτω συναρτήσεις να βρεθεί η πρώτη παράγωγος.
\begin{multicols}{2}
\begin{alist}
\item $ f(x)=x^2\cdot e^x $
\item $ f(x)=x\ln{x} $
\item $ f(x)=e^x\cdot\hm{x} $
\item $ f(x)=\sqrt{x}\cdot\syn{x} $
\item $ f(x)=x^3\cdot 2^x $
\item $ f(x)=x\cdot\ef{x} $
\item $ f(x)=\left(x^2-3x\right)\cdot e^x $
\item $ f(x)=\left(x^3-x^2\right)\cdot\hm{x} $
\end{alist}
\end{multicols}


\item
%@ Κωδικός: Ana-Paragogos-ParagGin-AA2
%@ Ενότητα: Παράγωγος συνάρτηση
%@ Είδος: Παράγωγος γινομένου
%@ Δυσκολία: 1
Για καθεμία από τις παρακάτω συναρτήσεις να βρεθεί η πρώτη παράγωγος.
\begin{multicols}{2}
\begin{alist}
\item $ f(x)=\hm{x}\cdot\syn{x} $
\item $ f(x)=\sqrt{x}\cdot\ln{x} $
\item $ f(x)=e^x\left(x-2\right) $
\item $ f(x)=\syf{x}\cdot\hm{x} $
\end{alist}
\end{multicols}


\item
%@ Κωδικός: Ana-Paragogos-ParagPilik-AA1
%@ Ενότητα: Παράγωγος συνάρτηση
%@ Είδος: Παράγωγος πηλίκου
%@ Δυσκολία: 1
Για καθεμία από τις παρακάτω συναρτήσεις, να βρεθεί η πρώτη παράγωγος.
\begin{multicols}{2}
\begin{alist}
\item $ f(x)=\dfrac{x}{x-2} $
\item $ f(x)=\dfrac{x^2}{x+1} $
\item $ f(x)=\dfrac{\hm{x}}{e^x} $
\item $ f(x)=\dfrac{\ln{x}}{x} $
\item $ f(x)=\dfrac{x^2+2x}{x^3+4} $
\item $ f(x)=\dfrac{\ef{x}}{\sqrt{x}} $
\end{alist}
\end{multicols}


\item
%@ Κωδικός: Ana-Paragogos-ParagPilik-AA2
%@ Ενότητα: Παράγωγος συνάρτηση
%@ Είδος: Παράγωγος πηλίκου
%@ Δυσκολία: 1
Για καθεμία από τις παρακάτω συναρτήσεις να βρεθεί η πρώτη παράγωγος.
\begin{multicols}{2}
\begin{alist}
\item $ f(x)=\dfrac{x^2-3x}{4-x} $
\item $ f(x)=\dfrac{\hm{x}}{\ln{x}} $
\item $ f(x)=\dfrac{\ln{x}+1}{x} $
\item $ f(x)=\dfrac{x^2}{e^x} $
\end{alist}
\end{multicols}


\item
%@ Κωδικός: Ana-Paragogos-ParagPilik-AA3
%@ Ενότητα: Παράγωγος συνάρτηση
%@ Είδος: Παράγωγος πηλίκου
%@ Δυσκολία: 1
Για καθεμία από τις παρακάτω συναρτήσεις να βρεθεί η πρώτη παράγωγος.
\begin{multicols}{2}
\begin{alist}
\item $ f(x)=\dfrac{x\cdot e^x}{x+1} $
\item $ f(x)=\dfrac{e^x\cdot\syn{x}}{x-2} $
\item $ f(x)=\dfrac{x}{e^x\cdot\hm{x}} $
\item $ f(x)=\dfrac{2^x}{x\cdot\syn{x}} $
\end{alist}
\end{multicols}


\item
%@ Κωδικός: Ana-Paragogos-ParagSynth-AA1
%@ Ενότητα: Παράγωγος συνάρτηση
%@ Είδος: Παράγωγος σύνθετης συνάρτησης (Πίνακας)
%@ Δυσκολία: 1
Για καθεμία από τις παρακάτω συναρτήσεις να βρεθεί η πρώτη παράγωγος.
\begin{multicols}{2}
\begin{alist}
\item $ f(x)=(x^2+4x)^3 $
\item $ f(x)=(x-\hm{x})^5 $
\item $ f(x)=\syn^4{x} $
\item $ f(x)=\ef^3{x} $
\item $ f(x)=(\ln{x}-x)^2 $
\item $ f(x)=\ln^5{x} $
\end{alist}
\end{multicols}


\item
%@ Κωδικός: Ana-Paragogos-ParagSynth-AA2
%@ Ενότητα: Παράγωγος συνάρτηση
%@ Είδος: Παράγωγος σύνθετης συνάρτησης (Πίνακας)
%@ Δυσκολία: 1
Για καθεμία από τις παρακάτω συναρτήσεις να βρεθεί η πρώτη παράγωγος.
\begin{multicols}{2}
\begin{alist}
\item $ f(x)=\sqrt{x^2+2x} $
\item $ f(x)=\sqrt{3-x} $
\item $ f(x)=\sqrt{\hm{x}}\ ,\ x\in(0,\pi) $
\item $ f(x)=\sqrt{\ln{x}-1} $
\item $ f(x)=\sqrt{e^x-1} $
\end{alist}
\end{multicols}


\item
%@ Κωδικός: Ana-Paragogos-ParagSynth-AA3
%@ Ενότητα: Παράγωγος συνάρτηση
%@ Είδος: Παράγωγος σύνθετης συνάρτησης (Πίνακας)
%@ Δυσκολία: 1
Για καθεμία από τις παρακάτω συναρτήσεις να βρεθεί η πρώτη παράγωγος.
\begin{multicols}{2}
\begin{alist}
\item $ f(x)=\hm{(x+2)} $
\item $ f(x)=\syn{(\ln{x})} $
\item $ f(x)=\hm{(\sqrt{x})} $
\item $ f(x)=\syn{(x^3-4x)} $
\item $ f(x)=\hm{\frac{1}{x}} $
\item $ f(x)=\syn{e^x} $
\end{alist}
\end{multicols}


\item
%@ Κωδικός: Ana-Paragogos-ParagSynth-AA4
%@ Ενότητα: Παράγωγος συνάρτηση
%@ Είδος: Παράγωγος σύνθετης συνάρτησης (Πίνακας)
%@ Δυσκολία: 1
Για καθεμία από τις παρακάτω συναρτήσεις να βρεθεί η πρώτη παράγωγος.
\begin{multicols}{2}
\begin{alist}
\item $ f(x)=\dfrac{1}{x^2} $
\item $ f(x)=\dfrac{2}{x^3-2x} $
\item $ f(x)=\dfrac{1}{\hm{x}} $
\item $ f(x)=\dfrac{3}{\sqrt{x}} $
\item $ f(x)=\dfrac{4}{e^x} $
\item $ f(x)=\dfrac{\pi}{\ln{x}} $
\end{alist}
\end{multicols}


\item
%@ Κωδικός: Ana-Paragogos-ParagSynth-AA5
%@ Ενότητα: Παράγωγος συνάρτηση
%@ Είδος: Παράγωγος σύνθετης συνάρτησης (Πίνακας)
%@ Δυσκολία: 1
Για καθεμία από τις παρακάτω συναρτήσεις να βρεθεί η πρώτη παράγωγος.
\begin{multicols}{2}
\begin{alist}
\item $ f(x)=\ef{x^3} $
\item $ f(x)=\syf{(\sqrt{x})} $
\item $ f(x)=\ef{e^x} $
\item $ f(x)=\syf{\frac{1}{x}} $
\item $ f(x)=\ef{(\ln{x})} $
\item $ f(x)=\syf{2^x} $
\end{alist}
\end{multicols}


\item
%@ Κωδικός: Ana-Paragogos-ParagSynth-AA6
%@ Ενότητα: Παράγωγος συνάρτηση
%@ Είδος: Παράγωγος σύνθετης συνάρτησης (Πίνακας)
%@ Δυσκολία: 1
Για καθεμία από τις παρακάτω συναρτήσεις να βρεθεί η πρώτη παράγωγος.
\begin{multicols}{2}
\begin{alist}
\item $ f(x)=e^{x^2+2x} $
\item $ f(x)=3^{\sqrt{x}} $
\item $ f(x)=2^{\ln{x}} $
\item $ f(x)=4^{\hm{x}} $
\item $ f(x)=e^{\frac{1}{x}} $
\item $ f(x)=2^{x+\sqrt{x}} $
\end{alist}
\end{multicols}


\item
%@ Κωδικός: Ana-Paragogos-ParagSynth-AA7
%@ Ενότητα: Παράγωγος συνάρτηση
%@ Είδος: Παράγωγος σύνθετης συνάρτησης (Πίνακας)
%@ Δυσκολία: 1
Για καθεμία από τις παρακάτω συναρτήσεις να βρεθεί η πρώτη παράγωγος.
\begin{multicols}{2}
\begin{alist}
\item $ f(x)=\ln{x^4} $
\item $ f(x)=\ln{(x^2-3x)} $
\item $ f(x)=\ln{(\hm{x})}\ ,\ x\in(0,\pi) $
\item $ f(x)=\ln{\sqrt{x}} $
\item $ f(x)=\ln{(2^x+x^2)} $
\item $ f(x)=\ln{(\ln{x})} $
\end{alist}
\end{multicols}


\item
%@ Κωδικός: Ana-Paragogos-ParPollTyp-AA1
%@ Ενότητα: Παράγωγος συνάρτηση
%@ Είδος: Παράγωγος συνάρτησης πολλαπλού τύπου - απόλυτες τιμές
%@ Δυσκολία: 1
Να βρεθεί η πρώτη παράγωγος των παρακάτω συναρτήσεων.
\begin{multicols}{2}
\begin{alist}
\item $ f(x)=2x-1-|x-3| $
\item $ f(x)=|4-x|-3+4x $
\item $ f(x)=x^2+|2x-1| $
\item $ f(x)=x^2+x-|x-2| $
\end{alist}
\end{multicols}


\item
%@ Κωδικός: Ana-Paragogos-ParPollTyp-AA2
%@ Ενότητα: Παράγωγος συνάρτηση
%@ Είδος: Παράγωγος συνάρτησης πολλαπλού τύπου - απόλυτες τιμές
%@ Δυσκολία: 1
Να βρεθεί η πρώτη παράγωγος των παρακάτω συναρτήσεων.
\begin{multicols}{2}
\begin{alist}
\item $ f(x)=\begin{cdcases}
x^2+x+1 & ,x\geq 0\\ \hm{x}+x & ,x<0
\end{cdcases} $
\item $ f(x)=\begin{cdcases}
x^2-3x & ,x<3\\ \sqrt{x+1}-2 & ,x\geq 3
\end{cdcases} $
\end{alist}
\end{multicols}


\item
%@ Κωδικός: Ana-Paragogos-ParSynEkth-AA1
%@ Ενότητα: Παράγωγος συνάρτηση
%@ Είδος: Παράγωγος σύνθετης εκθετικής
%@ Δυσκολία: 1
Για καθεμία από τις παρακάτω συναρτήσεις να βρεθεί η πρώτη παράγωγος.
\begin{multicols}{2}
\begin{alist}
\item $ f(x)=x^x $
\item $ f(x)=(x+3)^x $
\item $ f(x)=\left( \sqrt{x}\right) ^{x-2} $
\item $ f(x)=\hm^x{x}\ ,\ x\in(0,\pi) $
\item $ f(x)=x^{\ln{x}} $
\item $ f(x)=(x^2-3x+2)^{x} $
\end{alist}
\end{multicols}

\end{enumerate}
\end{document}