\documentclass[twoside,nofonts,ektypwsh,math,spyros]{frontisthrio}
\usepackage[amsbb,subscriptcorrection,zswash,mtpcal,mtphrb,mtpfrak]{mtpro2}
\usepackage[no-math,cm-default]{fontspec}
\usepackage{amsmath}
\usepackage{xunicode}
\usepackage{xgreek}
\let\hbar\relax
\defaultfontfeatures{Mapping=tex-text,Scale=MatchLowercase}
\setmainfont[Mapping=tex-text,Numbers=Lining,Scale=1.0,BoldFont={Minion Pro Bold}]{Minion Pro}
\newfontfamily\scfont{GFS Artemisia}
\font\icon = "Webdings"
\usepackage{fontawesome}
\newfontfamily{\FA}{fontawesome.otf}
\xroma{red!70!black}
%------TIKZ - ΣΧΗΜΑΤΑ - ΓΡΑΦΙΚΕΣ ΠΑΡΑΣΤΑΣΕΙΣ ----
\usepackage{tikz,pgfplots}
\usepackage{tkz-euclide}
\usetkzobj{all}
\usepackage[framemethod=TikZ]{mdframed}
\usetikzlibrary{decorations.pathreplacing}
\tkzSetUpPoint[size=7,fill=white]
%-----------------------
\usepackage{calc,tcolorbox}
\tcbuselibrary{skins,theorems,breakable}
\usepackage{hhline}
\usepackage[explicit]{titlesec}
\usepackage{graphicx}
\usepackage{multicol}
\usepackage{multirow}
\usepackage{tabularx}
\usetikzlibrary{backgrounds}
\usepackage{sectsty}
\sectionfont{\centering}
\usepackage{enumitem}
\usepackage{adjustbox}
\usepackage{mathimatika,gensymb,eurosym,wrap-rl}
\usepackage{systeme,regexpatch}
%-------- ΜΑΘΗΜΑΤΙΚΑ ΕΡΓΑΛΕΙΑ ---------
\usepackage{mathtools}
%----------------------
%-------- ΠΙΝΑΚΕΣ ---------
\usepackage{booktabs}
%----------------------
%----- ΥΠΟΛΟΓΙΣΤΗΣ ----------
\usepackage{calculator}
%----------------------------
%------ ΔΙΑΓΩΝΙΟ ΣΕ ΠΙΝΑΚΑ -------
\usepackage{array}
\newcommand\diag[5]{%
\multicolumn{1}{|m{#2}|}{\hskip-\tabcolsep
$\vcenter{\begin{tikzpicture}[baseline=0,anchor=south west,outer sep=0]
\path[use as bounding box] (0,0) rectangle (#2+2\tabcolsep,\baselineskip);
\node[minimum width={#2+2\tabcolsep-\pgflinewidth},
minimum  height=\baselineskip+#3-\pgflinewidth] (box) {};
\draw[line cap=round] (box.north west) -- (box.south east);
\node[anchor=south west,align=left,inner sep=#1] at (box.south west) {#4};
\node[anchor=north east,align=right,inner sep=#1] at (box.north east) {#5};
\end{tikzpicture}}\rule{0pt}{.71\baselineskip+#3-\pgflinewidth}$\hskip-\tabcolsep}}
%---------------------------------
%---- ΟΡΙΖΟΝΤΙΟ - ΚΑΤΑΚΟΡΥΦΟ - ΠΛΑΓΙΟ ΑΓΚΙΣΤΡΟ ------
\newcommand{\orag}[3]{\node at (#1)
{$ \overcbrace{\rule{#2mm}{0mm}}^{{\scriptsize #3}} $};}
\newcommand{\kag}[3]{\node at (#1)
{$ \undercbrace{\rule{#2mm}{0mm}}_{{\scriptsize #3}} $};}
\newcommand{\Pag}[4]{\node[rotate=#1] at (#2)
{$ \overcbrace{\rule{#3mm}{0mm}}^{{\rotatebox{-#1}{\scriptsize$#4$}}}$};}
%-----------------------------------------
%------------------------------------------
\newcommand{\tss}[1]{\textsuperscript{#1}}
\newcommand{\tssL}[1]{\MakeLowercase{\textsuperscript{#1}}}
%---------- ΛΙΣΤΕΣ ----------------------
\newlist{bhma}{enumerate}{3}
\setlist[bhma]{label=\bf\textit{\arabic*\textsuperscript{o}\;Βήμα :},leftmargin=0cm,itemindent=1.8cm,ref=\bf{\arabic*\textsuperscript{o}\;Βήμα}}
\newlist{rlist}{enumerate}{3}
\setlist[rlist]{itemsep=0mm,label=\roman*.}
\newlist{brlist}{enumerate}{3}
\setlist[brlist]{itemsep=0mm,label=\bf\roman*.}
\newlist{tropos}{enumerate}{3}
\setlist[tropos]{label=\bf\textit{\arabic*\textsuperscript{oς}\;Τρόπος :},leftmargin=0cm,itemindent=2.3cm,ref=\bf{\arabic*\textsuperscript{oς}\;Τρόπος}}
% Αν μπει το bhma μεσα σε tropo τότε
%\begin{bhma}[leftmargin=.7cm]
\tkzSetUpPoint[size=7,fill=white]
\tikzstyle{pl}=[line width=0.3mm]
\tikzstyle{plm}=[line width=0.4mm]
\usepackage{etoolbox}
\makeatletter
\renewrobustcmd{\anw@true}{\let\ifanw@\iffalse}
\renewrobustcmd{\anw@false}{\let\ifanw@\iffalse}\anw@false
\newrobustcmd{\noanw@true}{\let\ifnoanw@\iffalse}
\newrobustcmd{\noanw@false}{\let\ifnoanw@\iffalse}\noanw@false
\renewrobustcmd{\anw@print}{\ifanw@\ifnoanw@\else\numer@lsign\fi\fi}
\makeatother
\ekthetesdeiktes

\begin{document}
\titlos{Γ΄ Λυκείου - Μαθηματικά Προσανατολισμού}{Όρια - Συνέχεια}{Όριο στο άπειρο}
\orismoi
\Orismos{Όριο στο άπειρο}
Έστω μια συνάρτηση $ f $ η οποία ορίζεται σε ένα διάστημα της μορφής $ (a,+\infty) $. Το όριο της $ f $ όταν το $ x $ τείνει στο $ +\infty $ θα είναι είτε ένας πραγματικός αριθμός είτε ένα από τα $ \pm\infty $.
\[ \displaystyle{\lim_{x\rightarrow+\infty}f(x)=\ccases{\mathcal{l}\in\mathbb{R}\ \ \textrm{ή}\\\pm\infty}} \]
Ομοίως αν η συνάρτηση ορίζεται σε ένα διάστημα της μορφής $ (-\infty,a) $ έχουμε το όριο της όταν το $ x\to-\infty $: $ {\displaystyle{\lim_{x\rightarrow-\infty}{f(x)}}} $.\\\\
\thewrhmata
\Thewrhma{Όρια βασικών συναρτήσεων στο άπειρο}
\vspace{-5mm}
\begin{enumerate}[label=\bf\arabic*.]
\item \textbf{Όριο πολυωνυμικών συναρτήσεων}\\
Έστω $ P(x)=a_\nu x^\nu+a_{\nu-1}x^{\nu-1}+\ldots+a_1x+a_0 $ ένα πολυώνυμο $ \nu- $στού βαθμού. Το όριο του πολυωνύμου όταν $ x\to\pm\infty $ ισούται με το όριο του μεγιστοβάθμιου όρου.
\[ \lim_{x\rightarrow\pm\infty}{P(x)}=\lim_{x\rightarrow\pm\infty}{a_\nu x^\nu} \]
\item \textbf{Όριο ρητής συνάρτησης}\\
Έστω $ f(x)=\dfrac{P(x)}{Q(x)} $ μια ρητή συνάρτηση με $ P(x)=a_\nu x^\nu+a_{\nu-1}x^{\nu-1}+\ldots+a_1x+a_0 $ και $ Q(x)=\beta_\mu x^\mu+\beta_{\mu-1}x^{\mu-1}+\ldots+\beta_1x+\beta_0 $ βαθμών $ \nu $ και $ \mu $ αντίστοιχα. Το όριο της συνάρτησης όταν $ x\to\pm\infty $ ισούται με 
\[ \displaystyle{\lim_{x\rightarrow\pm\infty}\frac{P(x)}{Q(x)}=\lim_{x\rightarrow\pm\infty}\frac{a_\nu x^\nu}{\beta_\mu x^\mu}} \]
\begin{rlist}
\item Αν $ \nu>\mu $ τότε $ \displaystyle\lim_{x\rightarrow\pm\infty}{f(x)}=\pm\infty $.
\item Αν $ \nu<\mu $ τότε $ \displaystyle\lim_{x\rightarrow\pm\infty}{f(x)}=0 $.
\item Αν $ \nu=\mu $ τότε $ \displaystyle\lim_{x\rightarrow\pm\infty}{f(x)}=\dfrac{a_\nu}{\beta_\mu} $.
\end{rlist}
\item \bmath{Όριο εκθετικής - λογαριθμικής για $ a>1 $.}
\begin{multicols}{2}
\begin{enumerate}[itemsep=0mm,label=\alph*.]
\item $ \displaystyle{\lim_{x\rightarrow+\infty}a^x}=+\infty $
\item $ \displaystyle{\lim_{x\rightarrow-\infty}a^x}=0 $
\item $ \displaystyle{\lim_{x\rightarrow0^+}\log_ax}=-\infty $
\item $ \displaystyle{\lim_{x\rightarrow+\infty}\log_ax}=+\infty $
\end{enumerate}
\end{multicols}
\item \bmath{Όριο εκθετικής - λογαριθμικής για $ 0<a<1 $.}
\begin{multicols}{2}
\begin{enumerate}[itemsep=0mm,label=\alph*.]
\item $ \displaystyle{\lim_{x\rightarrow+\infty}a^x}=0 $
\item $ \displaystyle{\lim_{x\rightarrow-\infty}a^x}=+\infty $
\item $ \displaystyle{\lim_{x\rightarrow0^+}\log_ax}=+\infty $
\item $ \displaystyle{\lim_{x\rightarrow+\infty}\log_ax}=-\infty $
\end{enumerate}
\end{multicols}
\item \textbf{Βασικά όρια}
\begin{enumerate}[itemsep=0mm,label=\alph*.]
\begin{multicols}{2}
\item $ \displaystyle{\lim_{x\rightarrow0^+}e^{\frac{1}{x}}}\eq{y=\frac{1}{x}}\displaystyle{\lim_{y\rightarrow+\infty}e^{y}}=+\infty $
\item $ \displaystyle{\lim_{x\rightarrow0^-}e^{\frac{1}{x}}}\eq{y=\frac{1}{x}}\displaystyle{\lim_{y\rightarrow-\infty}e^{y}}=0 $
\item $ \displaystyle{\lim_{x\rightarrow\pm\infty}\textrm{ημ}{\frac{1}{x}}}\eq{y=\frac{1}{x}}\displaystyle{\lim_{y\rightarrow0}\textrm{ημ}{y}}=0 $
\item $ \displaystyle{\lim_{x\rightarrow\pm\infty}\textrm{συν}{\frac{1}{x}}}\eq{y=\frac{1}{x}}\displaystyle{\lim_{y\rightarrow0}\textrm{συν}{y}}=1 $
\item $ \displaystyle{\lim_{x\rightarrow\pm\infty}x\textrm{ημ}{\frac{1}{x}}}\eq{y=\frac{1}{x}}\displaystyle{\lim_{y\rightarrow0}\frac{\textrm{ημ}{y}}{y}}=1 $
\end{multicols}
\item $ \displaystyle{\lim_{x\rightarrow\pm\infty}\frac{\textrm{ημ}x}{x}}=\displaystyle{\lim_{x\rightarrow\pm\infty}\frac{1}{x}{\textrm{ημ}x}}=0 $\quad Μηδενική επί φραγμένη (αποδεικνείεται με κριτήριο παρεμβολής)
\item $ \displaystyle{\lim_{x\rightarrow\pm\infty}(x+\textrm{ημ}x)}= \displaystyle{\lim_{x\rightarrow\pm\infty}x\left( 1+\frac{\textrm{ημ}x}{x}\right) }=\pm\infty $\\\\
Τα παρακάτω όρια ΔΕΝ υπάρχουν
\begin{multicols}{2}
\item $  \displaystyle{\lim_{x\rightarrow\pm\infty}\textrm{ημ}x} $
\item $  \displaystyle{\lim_{x\rightarrow\pm\infty}\textrm{συν}x} $
\end{multicols}
\end{enumerate}
\end{enumerate}
\end{document}



