\documentclass[twoside,nofonts,ektypwsh,math,spyros]{frontisthrio}
\usepackage[amsbb,subscriptcorrection,zswash,mtpcal,mtphrb,mtpfrak]{mtpro2}
\usepackage[no-math,cm-default]{fontspec}
\usepackage{amsmath}
\usepackage{xunicode}
\usepackage{xgreek}
\let\hbar\relax
\defaultfontfeatures{Mapping=tex-text,Scale=MatchLowercase}
\setmainfont[Mapping=tex-text,Numbers=Lining,Scale=1.0,BoldFont={Minion Pro Bold}]{Minion Pro}
\newfontfamily\scfont{GFS Artemisia}
\font\icon = "Webdings"
\usepackage{fontawesome5}
\newfontfamily{\FA}{fontawesome.otf}
\xroma{red!70!black}
%------TIKZ - ΣΧΗΜΑΤΑ - ΓΡΑΦΙΚΕΣ ΠΑΡΑΣΤΑΣΕΙΣ ----
\usepackage{tikz,pgfplots}
\usepackage{tkz-euclide}
\usetkzobj{all}
\usepackage[framemethod=TikZ]{mdframed}
\usetikzlibrary{decorations.pathreplacing}
\tkzSetUpPoint[size=7,fill=white]
%-----------------------
\usepackage{calc,tcolorbox}
\tcbuselibrary{skins,theorems,breakable}
\usepackage{hhline}
\usepackage[explicit]{titlesec}
\usepackage{graphicx}
\usepackage{multicol}
\usepackage{multirow}
\usepackage{tabularx}
\usetikzlibrary{backgrounds}
\usepackage{sectsty}
\sectionfont{\centering}
\usepackage{enumitem}
\usepackage{adjustbox}
\usepackage{mathimatika,gensymb,eurosym,wrap-rl}
\usepackage{systeme,regexpatch}
%-------- ΜΑΘΗΜΑΤΙΚΑ ΕΡΓΑΛΕΙΑ ---------
\usepackage{mathtools}
%----------------------
%-------- ΠΙΝΑΚΕΣ ---------
\usepackage{booktabs}
%----------------------
%----- ΥΠΟΛΟΓΙΣΤΗΣ ----------
\usepackage{calculator}
%----------------------------
%------ ΔΙΑΓΩΝΙΟ ΣΕ ΠΙΝΑΚΑ -------
\usepackage{array}
\newcommand\diag[5]{%
\multicolumn{1}{|m{#2}|}{\hskip-\tabcolsep
$\vcenter{\begin{tikzpicture}[baseline=0,anchor=south west,outer sep=0]
\path[use as bounding box] (0,0) rectangle (#2+2\tabcolsep,\baselineskip);
\node[minimum width={#2+2\tabcolsep-\pgflinewidth},
minimum  height=\baselineskip+#3-\pgflinewidth] (box) {};
\draw[line cap=round] (box.north west) -- (box.south east);
\node[anchor=south west,align=left,inner sep=#1] at (box.south west) {#4};
\node[anchor=north east,align=right,inner sep=#1] at (box.north east) {#5};
\end{tikzpicture}}\rule{0pt}{.71\baselineskip+#3-\pgflinewidth}$\hskip-\tabcolsep}}
%---------------------------------
%---- ΟΡΙΖΟΝΤΙΟ - ΚΑΤΑΚΟΡΥΦΟ - ΠΛΑΓΙΟ ΑΓΚΙΣΤΡΟ ------
\newcommand{\orag}[3]{\node at (#1)
{$ \overcbrace{\rule{#2mm}{0mm}}^{{\scriptsize #3}} $};}
\newcommand{\kag}[3]{\node at (#1)
{$ \undercbrace{\rule{#2mm}{0mm}}_{{\scriptsize #3}} $};}
\newcommand{\Pag}[4]{\node[rotate=#1] at (#2)
{$ \overcbrace{\rule{#3mm}{0mm}}^{{\rotatebox{-#1}{\scriptsize$#4$}}}$};}
%-----------------------------------------
%------------------------------------------
\newcommand{\tss}[1]{\textsuperscript{#1}}
\newcommand{\tssL}[1]{\MakeLowercase{\textsuperscript{#1}}}
%---------- ΛΙΣΤΕΣ ----------------------
\newlist{bhma}{enumerate}{3}
\setlist[bhma]{label=\bf\textit{\arabic*\textsuperscript{o}\;Βήμα :},leftmargin=0cm,itemindent=1.8cm,ref=\bf{\arabic*\textsuperscript{o}\;Βήμα}}
\newlist{rlist}{enumerate}{3}
\setlist[rlist]{itemsep=0mm,label=\roman*.}
\newlist{brlist}{enumerate}{3}
\setlist[brlist]{itemsep=0mm,label=\bf\roman*.}
\newlist{tropos}{enumerate}{3}
\setlist[tropos]{label=\bf\textit{\arabic*\textsuperscript{oς}\;Τρόπος :},leftmargin=0cm,itemindent=2.3cm,ref=\bf{\arabic*\textsuperscript{oς}\;Τρόπος}}
% Αν μπει το bhma μεσα σε tropo τότε
%\begin{bhma}[leftmargin=.7cm]
\tkzSetUpPoint[size=7,fill=white]
\tikzstyle{pl}=[line width=0.3mm]
\tikzstyle{plm}=[line width=0.4mm]
\usepackage{etoolbox}
\makeatletter
\renewrobustcmd{\anw@true}{\let\ifanw@\iffalse}
\renewrobustcmd{\anw@false}{\let\ifanw@\iffalse}\anw@false
\newrobustcmd{\noanw@true}{\let\ifnoanw@\iffalse}
\newrobustcmd{\noanw@false}{\let\ifnoanw@\iffalse}\noanw@false
\renewrobustcmd{\anw@print}{\ifanw@\ifnoanw@\else\numer@lsign\fi\fi}
\makeatother


\begin{document}
\titlos{Γ΄ Λυκείου - Μαθηματικά Προσανατολισμού}{Όρια - Συνέχεια}{Συνάρτηση 1-1}
\vspace{-3mm}
\orismoi
\Orismos{Συνάρτηση 1-1}
Μια συνάρτηση $ f:A\rightarrow\mathbb{R} $ ονομάζεται $ 1-1 $ εάν κάθε στοιχείο $ x\in A $ του πεδίου ορισμού αντιστοιχεί μέσω της συνάρτησης, σε μοναδική τιμή $ f(x) $ του συνόλου τιμών της. Για κάθε ζεύγος αριθμών $ x_1,x_2\in A $ του πεδίου ορισμού της $ f $ θα ισχύει \[ x_1\neq x_2\Rightarrow f(x_1)\neq f(x_2) \]
\thewrhmata
\Thewrhma{Συνάρτηση 1-1}
Μια συνάρτηση $ f:A\rightarrow\mathbb{R} $ είναι μια συνάρτηση $ 1-1 $ αν και μόνο αν για κάθε ζεύγος αριθμών $ x_1,x_2\in A $ του πεδίου ορισμού της, η ισότητα των εικόνων τους συνεπάγεται την ισότητα μεταξύ τους. Δηλαδή θα ισχύει η παρακάτω σχέση \[ f(x_1)=f(x_2)\Rightarrow x_1= x_2 \]
\Thewrhma{Ιδιότητεσ συνάρτησησ 1-1}
Έστω μια συνάρτηση $ f:A\rightarrow\mathbb{R} $. Αν η $ f $ είναι μια συνάρτηση $ 1-1 $ τότε γι αυτήν ισχύουν οι παρακάτω ιδιότητες :\\
\wrapr{-14mm}{8}{3.7cm}{0mm}{\begin{tikzpicture}
\begin{axis}[x=1cm,y=1cm,aks_on,xmin=-.4,xmax=3,
ymin=-.4,ymax=3,ticks=none,xlabel={\footnotesize $ x $},
ylabel={\footnotesize $ y $},belh ar]
\addplot[grafikh parastash,domain=.1:2.7]{2*ln(x+1)};
\end{axis}
\draw[dashed](0,1)--(3.5,1);
\draw[dashed](0,2)--(3.5,2);
\draw[dashed](0,2.5)--(3.5,2.5);
\draw[dashed](0,1.5)--(3.5,1.5);
\tkzDrawPoint[fill=black](1.117,2*ln(2.117))
\node at (0.2,0.2) {\footnotesize$O$};
\node at (1.65,0.7) {\footnotesize$y=\kappa$};
\node at (2.3,3) {\footnotesize$C_f$};
\end{tikzpicture}}{
\begin{rlist}
\item Για κάθε $ x_1,x_2\in A $ ισχύει $ x_1=x_2\Leftrightarrow f(x_1)=f(x_2) $.
\item Κάθε οριζόντια ευθεία της μορφής $ y=\kappa $ με $ \kappa\in\mathbb{R} $ θα έχει το πολύ ένα κοινό σημείο με τη γραφική παράσταση της συνάρτησης $ f $.
\item\label{mon} Εαν η συνάρτηση είναι γνησίως μονότονη σε κάθε διάστημα του πεδίου ορισμού της τότε θα είναι και $ 1-1 $. Το αντίστροφο δεν ισχύει πάντα.
\item Η εξίσωση $ f(x)=0 $ έχει το πολύ μια λύση στο πεδίο ορισμού της $ f $. Εαν $ 0\in f(A) $ τότε η εξίσωση έχει μια λύση ακριβώς.
\end{rlist}
Αν η συνάρτηση $ f $ δεν είναι $ 1-1 $ τότε θα υπάρχει τουλάχιστον ένα ζεύγος αριθμών $ x_1,x_2\in A $ που να έχουν την ίδια τιμή δηλαδή : \[ x_1\neq x_2\Rightarrow f(x_1)=f(x_2) \]}
\end{document}



