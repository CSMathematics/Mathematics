\documentclass[twoside,nofonts,ektypwsh,math,spyros]{frontisthrio}
\usepackage[amsbb,subscriptcorrection,zswash,mtpcal,mtphrb,mtpfrak]{mtpro2}
\usepackage[no-math,cm-default]{fontspec}
\usepackage{amsmath}
\usepackage{xunicode}
\usepackage{xgreek}
\let\hbar\relax
\defaultfontfeatures{Mapping=tex-text,Scale=MatchLowercase}
\setmainfont[Mapping=tex-text,Numbers=Lining,Scale=1.0,BoldFont={Minion Pro Bold}]{Minion Pro}
\newfontfamily\scfont{GFS Artemisia}
\font\icon = "Webdings"
\usepackage{fontawesome5}
\newfontfamily{\FA}{fontawesome.otf}
\xroma{red!70!black}
%------TIKZ - ΣΧΗΜΑΤΑ - ΓΡΑΦΙΚΕΣ ΠΑΡΑΣΤΑΣΕΙΣ ----
\usepackage{tikz,pgfplots}
\usepackage{tkz-euclide}
\usetkzobj{all}
\usepackage[framemethod=TikZ]{mdframed}
\usetikzlibrary{decorations.pathreplacing}
\tkzSetUpPoint[size=7,fill=white]
%-----------------------
\usepackage{calc,tcolorbox}
\tcbuselibrary{skins,theorems,breakable}
\usepackage{hhline}
\usepackage[explicit]{titlesec}
\usepackage{graphicx}
\usepackage{multicol}
\usepackage{multirow,longtable}
\usepackage{tabularx}
\usetikzlibrary{backgrounds}
\usepackage{sectsty}
\sectionfont{\centering}
\usepackage{enumitem}
\usepackage{adjustbox}
\usepackage{mathimatika,gensymb,eurosym,wrap-rl}
\usepackage{systeme,regexpatch}
%-------- ΜΑΘΗΜΑΤΙΚΑ ΕΡΓΑΛΕΙΑ ---------
\usepackage{mathtools}
%----------------------
%-------- ΠΙΝΑΚΕΣ ---------
\usepackage{booktabs}
%----------------------
%----- ΥΠΟΛΟΓΙΣΤΗΣ ----------
\usepackage{calculator}
%----------------------------
%------ ΔΙΑΓΩΝΙΟ ΣΕ ΠΙΝΑΚΑ -------
\usepackage{array}
\newcommand\diag[5]{%
\multicolumn{1}{|m{#2}|}{\hskip-\tabcolsep
$\vcenter{\begin{tikzpicture}[baseline=0,anchor=south west,outer sep=0]
\path[use as bounding box] (0,0) rectangle (#2+2\tabcolsep,\baselineskip);
\node[minimum width={#2+2\tabcolsep-\pgflinewidth},
minimum  height=\baselineskip+#3-\pgflinewidth] (box) {};
\draw[line cap=round] (box.north west) -- (box.south east);
\node[anchor=south west,align=left,inner sep=#1] at (box.south west) {#4};
\node[anchor=north east,align=right,inner sep=#1] at (box.north east) {#5};
\end{tikzpicture}}\rule{0pt}{.71\baselineskip+#3-\pgflinewidth}$\hskip-\tabcolsep}}
%---------------------------------
%---- ΟΡΙΖΟΝΤΙΟ - ΚΑΤΑΚΟΡΥΦΟ - ΠΛΑΓΙΟ ΑΓΚΙΣΤΡΟ ------
\newcommand{\orag}[3]{\node at (#1)
{$ \overcbrace{\rule{#2mm}{0mm}}^{{\scriptsize #3}} $};}
\newcommand{\kag}[3]{\node at (#1)
{$ \undercbrace{\rule{#2mm}{0mm}}_{{\scriptsize #3}} $};}
\newcommand{\Pag}[4]{\node[rotate=#1] at (#2)
{$ \overcbrace{\rule{#3mm}{0mm}}^{{\rotatebox{-#1}{\scriptsize$#4$}}}$};}
%-----------------------------------------
%------------------------------------------
\newcommand{\tss}[1]{\textsuperscript{#1}}
\newcommand{\tssL}[1]{\MakeLowercase{\textsuperscript{#1}}}
%---------- ΛΙΣΤΕΣ ----------------------
\newlist{bhma}{enumerate}{3}
\setlist[bhma]{label=\bf\textit{\arabic*\textsuperscript{o}\;Βήμα :},leftmargin=0cm,itemindent=1.8cm,ref=\bf{\arabic*\textsuperscript{o}\;Βήμα}}
\newlist{rlist}{enumerate}{3}
\setlist[rlist]{itemsep=0mm,label=\roman*.}
\newlist{brlist}{enumerate}{3}
\setlist[brlist]{itemsep=0mm,label=\bf\roman*.}
\newlist{tropos}{enumerate}{3}
\setlist[tropos]{label=\bf\textit{\arabic*\textsuperscript{oς}\;Τρόπος :},leftmargin=0cm,itemindent=2.3cm,ref=\bf{\arabic*\textsuperscript{oς}\;Τρόπος}}
% Αν μπει το bhma μεσα σε tropo τότε
%\begin{bhma}[leftmargin=.7cm]
\tkzSetUpPoint[size=7,fill=white]
\tikzstyle{pl}=[line width=0.3mm]
\tikzstyle{plm}=[line width=0.4mm]
\usepackage{etoolbox}
\makeatletter
\renewrobustcmd{\anw@true}{\let\ifanw@\iffalse}
\renewrobustcmd{\anw@false}{\let\ifanw@\iffalse}\anw@false
\newrobustcmd{\noanw@true}{\let\ifnoanw@\iffalse}
\newrobustcmd{\noanw@false}{\let\ifnoanw@\iffalse}\noanw@false
\renewrobustcmd{\anw@print}{\ifanw@\ifnoanw@\else\numer@lsign\fi\fi}
\makeatother


\begin{document}
\titlos{Γ΄ Λυκείου - Μαθηματικά Προσανατολισμού}{Όρια - Συνέχεια}{Συναρτήσεις}
\orismoi
\Orismos{Συνάρτηση}
Συνάρτηση ονομάζεται ο κανόνας (αντιστοίχηση) με τον οποίο \textbf{κάθε} στοιχείο ενός συνόλου $ A $ αντιστοιχεί σε \textbf{ένα μόνο} στοιχείο ενός συνόλου $ B $.\\Συμβολίζεται με οποιοδήποτε γράμμα του λατινικού ή και του ελληνικού αλφαβήτου $ f, g, h, t, s, \sigma\ldots $ και γράφουμε : \[ f:A\rightarrow B \]
Είναι η σχέση που συνδέει δύο μεταβλητές $ x,y $ όπου κάθε τιμή της πρώτης $ (x\in A) $, στο πρώτο σύνολο, αντιστοιχεί σε μόνο μια τιμή της δεύτερης $ (y\in B) $, στο δεύτερο σύνολο.\vspace{-3mm}
\begin{center}
\begin{figure}[h]
\centering
\begin{tikzpicture}[scale=.6]
\draw(0,0) ellipse (1cm and 1.5cm);
\draw(4,0) ellipse (1cm and 1.5cm);
\draw[fill=\xrwma!30] (4.1,0) ellipse (.6cm and 1.1cm);
\draw[-latex] (0,.2) arc (140:40:2.6);
\tkzDefPoint(0,.2){A}
\tkzDefPoint(4,.2){B}
\tkzDrawPoints(A,B)
\tkzLabelPoint[left](A){{\footnotesize $ x $}}
\tkzLabelPoint[right](B){{\footnotesize $ y $}}
\tkzText(0,1.8){$ A $}
\tkzText(4,1.8){$ B $}
\tkzText(2,1.45){$ f $}
\draw[-latex] (3.5,0) -- (2.7,-1) node[anchor=north east] {\footnotesize $ f\left( A \right)  $};
\end{tikzpicture}
\end{figure}
\end{center}
\vspace{-1.1cm}
\begin{itemize}[itemsep=0mm]
\item Η μεταβλητή $ x $ του συνόλου $ A $ ονομάζεται \textbf{ανεξάρτητη} ενώ η $ y $ \textbf{εξαρτημένη}.
\item Η τιμή της $ y $ ονομάζεται \textbf{τιμή} της $ f $ στο $ x $ και συμβολίζεται $ y=f(x) $.
\item Ο κανόνας της συνάρτησης, με τον οποίο γίνεται η αντιστοίχηση από το $ x $  στο $ f(x) $, εκφράζεται συμβολικά με τη βοήθεια του $ x $ και ονομάζεται \textbf{τύπος της συνάρτησης}.
\item Το σύνολο $ A $ λέγεται \textbf{πεδίο ορισμού} της συνάρτησης $ f $ και συμβολίζεται $ D_f $. Είναι το σύνολο των δυνατών τιμών την ανεξάρτητης μεταβλητής της συνάρτησης.
\item Το σύνολο με στοιχεία όλες τις δυνατές τιμές $ f(x) $ της εξαρτημένης μεταβλητής για κάθε $ x\in D_f $ λέγεται \textbf{σύνολο τιμών} της $ f $, συμβολίζεται $ f\left(D_f\right) $ και ισχύει $ f\left(D_f\right)\subseteq B $.
\item Μια συνάρτηση συμβολίζεται επίσης με τους εξής τρόπους : \[ x\overset{f}{\mapsto}f(x)\;\;,\;\;D_f\overset{f}{\rightarrow}f\left(D_f\right) \]
\item Για το συμβολισμό της ανεξάρτητης μεταβλητής ή της συνάρτησης μπορούμε να χρησιμοποιήσουμε οποιοδήποτε συμβολισμό στη θέση της μεταβλητής $ x $ ή του ονόματος $ f $ της συνάρτησης αντίστοιχα. \[ f(x)\;\;,\;\;g(t)\;\;,\;\;h(s)\ldots \]
\vspace{-3mm}
\item Για να ορίσουμε μια συνάρτηση θα πρέπει να γνωρίζουμε
\vspace{-3mm}
\begin{enumerate}[itemsep=0mm]
\begin{multicols}{2}
\item To πεδίο ορισμού $ D_f $.
\item Το σύνολο $ B $.
\end{multicols}
\vspace{-3mm}
\item Τον τύπο $ f(x) $ της συνάρτησης, για κάθε $ x\in D_f $.
\end{enumerate}
\item Εαν τα σύνολα $ D_f,B $ είναι υποσύνολα του συνόλου των πραγματικών αριθμών τότε μιλάμε για \textbf{πραγματική συνάρτηση πραγματικής μεταβλητής}.
\item Οι συναρτήσεις των οποίων ο τύπος δίνεται από δύο ή περισσότερες αλγεβρικές παραστάσεις ονομάζονται συναρτήσεις \textbf{πολλαπλού τύπου}.
\[ f(x)=\ccases{f_1(x) & \textrm{αν }x\in D_{f_1}\subseteq D_f\\
f_2(x) & \textrm{αν }x\in D_{f_2}\subseteq D_f\\
\;\;\;\;\vdots & \qquad\vdots\\
f_\nu(x) & \textrm{αν }x\in D_{f_\nu}\subseteq D_f} \]
όπου $ D_{f_1},D_{f_2},\ldots,D_{f_\nu} $ είναι υποσύνολα του πεδίου ορισμού ολόκληρης της συνάρτησης $ f $ με $ D_{f_1}\cup D_{f_2}\cup \ldots\cup D_{f_\nu}=D_f $ και $  D_{f_1}\cap D_{f_2}\cap \ldots\cap D_{f_\nu}=\varnothing $.
\end{itemize}
Στον πίνακα βλέπουμε τα βασικά είδη συναρτήσεων τον τύπο τους και το πεδίο ορισμού τους.
\begin{center}
\begin{longtable}{ccc}
\hline \rule[-2ex]{0pt}{5.5ex}\textbf{Είδος} & \textbf{Τύπος} & \textbf{Πεδίο Ορισμού} \\ 
\hhline{===} \rule[-2ex]{0pt}{5.5ex} \textbf{Πολυωνυμική} & $ f(x)=a_\nu x^\nu+\ldots+a_1x+a_0 $ & $ D_f=\mathbb{R} $ \\
\rule[-2ex]{0pt}{5.5ex} \textbf{Ρητή} & $ f(x)=\dfrac{P(x)}{Q(x)} $ & $ D_f=\left\lbrace\left.  x\in\mathbb{R}\right| Q(x)\neq0\right\rbrace $  \\
\rule[-2ex]{0pt}{5.5ex} \textbf{Άρρητη} & $ f(x)=\sqrt{A(x)} $ & $ D_f=\left\lbrace\left. x\in\mathbb{R}\right| A(x)\geq0\right\rbrace $ \\
\hhline{~--}\rule[-2ex]{0pt}{5.5ex} \multirow{5}{*}{\textbf{Τριγωνομετρική}} & $ f(x)=\hm{x}\;\;,\;\;\syn{x} $ & $ D_f=\mathbb{R} $ \\ 
\rule[-2ex]{0pt}{5.5ex}  & $ f(x)=\ef{x} $ & $ D_f=\left\lbrace\left.x\in\mathbb{R}\right| x\neq\kappa\pi+\frac{\pi}{2}\;,\;\kappa\in\mathbb{Z}\right\rbrace $ \\ 
\rule[-2ex]{0pt}{5.5ex}  & $ f(x)=\syf{x} $ & $ D_f=\left\lbrace\left.x\in\mathbb{R}\right| x\neq\kappa\pi\;,\;\kappa\in\mathbb{Z}\right\rbrace $ \\ 
\hhline{~--}\rule[-2ex]{0pt}{5.5ex} \textbf{Εκθετική} & $ f(x)=a^x\;\;,\;\;0<a\neq1 $ & $ D_f=\mathbb{R} $ \\ 
\rule[-2ex]{0pt}{5.5ex} \textbf{Λογαριθμική} & $ f(x)=\log{x}\;\;,\;\;\ln{x} $ & $ D_f=(0,+\infty) $ \\ 
\hline 
\end{longtable}
\end{center}
\vspace{-.8cm}
Επιπλέον, ειδικές περιπτώσεις πολυωνυμικών συναρτήσεων αποτελούν οι παρακάτω συναρτήσεις
\begin{center}
\begin{minipage}{2.5cm}
\textbf{Ταυτοτική}\\$ f(x)=x $
\end{minipage}\qquad
\begin{minipage}{2.5cm}
\textbf{Σταθερή}\\$ f(x)=c $
\end{minipage}\qquad
\begin{minipage}{2.5cm}
\textbf{Μηδενική}\\$ f(x)=0 $
\end{minipage}
\end{center}
\Orismos{Πράξεισ συναρτήσεων}
Αν $ f,g $ δύο συναρτήσεις με πεδία ορισμού $ D_f,D_g $ αντίστοιχα τότε οι πράξεις μεταξύ των δύο συναρτήσεων ορίζονται ως εξής.
\begin{center}
\begin{longtable}{cc}
\hline \rule[-2ex]{0pt}{5.5ex} \textbf{Τύπος} & \textbf{Πεδίο ορισμού} \\ 
\hhline{==} \rule[-2ex]{0pt}{5.5ex} $ (f+g)(x)=f(x)+g(x) $ & $ D_{f+g}=\{x\in\mathbb{R}|x\in D_f\cap D_g\} $ \\ 
\rule[-2ex]{0pt}{5.5ex} $ (f-g)(x)=f(x)-g(x) $ & $ D_{f-g}=\{x\in\mathbb{R}|x\in D_f\cap D_g\} $ \\ 
\rule[-2ex]{0pt}{5.5ex} $ (f\cdot g)(x)=f(x)\cdot g(x) $ & $ D_{f\cdot g}=\{x\in\mathbb{R}|x\in D_f\cap D_g\} $ \\ 
\rule[-2ex]{0pt}{5.5ex} $ \left(\dfrac{f}{g} \right) (x)=\dfrac{f(x)}{g(x)} $ & $ D_{\frac{f}{g}}=\left\lbrace x\in\mathbb{R}|x\in D_f\cap D_g\textrm{ και }g(x)\neq0\right\rbrace  $ \\
\rule[0ex]{0pt}{-.5ex} & \\
\hline
\end{longtable}
\end{center}
\Orismos{Ορθογώνιο - Ορθοκανονικό Σύστημα Συντεταγμένων}
Ορθογώνιο σύστημα συντεταγμένων ονομάζεται το σύστημα αξόνων προσδιορισμού της θέσης ενός σημείου. Στο επίπεδο αποτελείται από δύο κάθετα τοποθετημένους μεταξύ τους άξονες αρίθμησης πάνω στους οποίους παίρνουν τιμές δύο μεταβλητές.
\begin{itemize}[itemsep=0mm]
\item Το σημείο τομής των δύο αξόνων ονομάζεται \textbf{αρχή των αξόνων}.
\item Σε κάθε άξονα του συστήματος επιλέγουμε αυθαίρετα μια μονάδα μέτρησης.
\item Εάν σε κάθε άξονα θέσουμε την ίδια μονάδα μέτρησης το σύστημα ονομάζεται \textbf{ορθοκανονικό}.
\item Ο οριζόντιος άξονας ονομάζεται \textbf{άξονας τετμημένων} και συμβολίζεται με $ x'x $.
\end{itemize}
\begin{minipage}{\linewidth}\mbox{}\\
\vspace{-1.2cm}
\begin{WrapText1}{10}{5cm}
\begin{tikzpicture}[scale=.48,y=1cm]
\tkzInit[xmin=-4,xmax=4.4,ymin=-4,ymax=4.4,ystep=1]
\draw[-latex]  (-4,0) node[left,fill=white] {{\footnotesize $ x' $}} -- coordinate (x axis mid) (4.4,0) node[right,fill=white] {{\footnotesize $ x $}};
\draw[-latex] (0,-4) node[below,fill=white] {{\footnotesize $ y' $}} -- (0,4.4) node[above,fill=white] {{\footnotesize $ y $}};
\draw (1,.15) -- (1,-.15) node[anchor=north] {\scriptsize 1};
\draw (.15,1) -- (-.15,1) node[anchor=east] {\scriptsize 1};
\tkzDefPoint(0,0){O}
\tkzDefPoint(2,1.8){M}
\tkzLabelPoint[below left](O){$ O $}
\tkzLabelPoint[right](M){{\footnotesize $ Μ(x,y) $}}
\draw[dashed] (0,1.8) node[left]{{\scriptsize $ y $}}--(2,1.8)--(2,0) node[below]{{\scriptsize $ x $}};
\tkzDrawPoint[size=7,fill=white](M)
\tkzText(2.2,3.3){{\scriptsize 1\textsuperscript{ο} Τεταρτημόριο}}
\tkzText(-2.2,3.3){{\scriptsize 2\textsuperscript{ο} Τεταρτημόριο}}
\tkzText(-2.2,-2){{\scriptsize 3\textsuperscript{ο} Τεταρτημόριο}}
\tkzText(2.2,-2){{\scriptsize 4\textsuperscript{ο} Τεταρτημόριο}}
\tkzText(2.2,2.7){{\scriptsize $ (+,+) $}}
\tkzText(-2.2,2.7){{\scriptsize $ (-,+) $}}
\tkzText(-2.2,-1.4){{\scriptsize $ (-,-) $}}
\tkzText(2.2,-1.4){{\scriptsize $ (+,-) $}}
\end{tikzpicture}
\end{WrapText1}
\begin{itemize}[itemsep=0mm]
\item Ο κατακόρυφος άξονας ονομάζεται \textbf{άξονας τεταγμένων} και συμβολίζεται με $ y'y $.
\item Κάθε σημείο του επιπέδου του συστήματος συντεταγμένων αντιστοιχεί σε ένα ζευγάρι αριθμών της μορφής $(x,y)$. Αντίστροφα, κάθε ζευγάρι αριθμών $(x,y)$ αντιστοιχεί σε ένα σημείο του επιπέδου.
\item Το ζεύγος αριθμών $(x,y)$ ονομάζεται \textbf{διατεταγμένο ζεύγος αριθμών} διότι έχει σημασία η σειρά με την οποία εμφανίζονται οι αριθμοί.
\item Οι αριθμοί $x,y$ ονομάζονται \textbf{συντεταγμένες} του σημείου. Ο αριθμός $x$ ονομάζεται \textbf{τετμημένη} του σημείου ενώ ο $y$ \textbf{τεταγμένη}.
\end{itemize}\end{minipage}\mbox{}\\
\vspace{-2mm}
\begin{itemize}
\item Στον οριζόντιο άξονα $ x'x $, δεξιά της αρχής των αξόνων, βρίσκονται οι θετικές τιμές της μεταβλητής $x$ ενώ αριστερά, οι αρνητικές.
\item Αντίστοιχα στον κατακόρυφο άξονα $ y'y $, πάνω από την αρχή των αξόνων βρίσκονται οι θετικές τιμές της μεταβλητής $y$, ενώ κάτω οι αρνητικές τιμές.
\item Οι άξονες χωρίζουν το επίπεδο σε τέσσερα μέρη τα οποία ονομάζονται \textbf{τεταρτημόρια}. Ως 1\textsuperscript{ο} τεταρτημόριο ορίζουμε το μέρος στο οποίο ανήκουν οι θετικοί ημιάξονες $ Ox $ και $ Oy $.
\end{itemize}
\Orismos{Γραφική Παράσταση συνάρτησησ}
Γραφική παράσταση μιας συνάρτησης $ f:D_f\rightarrow\mathbb{R} $ ονομάζεται το σύνολο των σημείων του επιπέδου με συντεταγμένες $ M(x,y) $ όπου \[ x\in D_f\;\;,\;\;y=f(x) \]
Το σύνολο των σημείων της γραφικής παράστασης είναι 
\[ C_f=\{M(x,y)|y=f(x)\textrm{ για κάθε }x\in D_f\} \]
\wrapr{-15mm}{5}{4.5cm}{0mm}{\begin{tikzpicture}[scale=.7,domain=.2:4.5,y=1cm]
\tkzInit[xmin=-.5,xmax=7,ymin=-.5,ymax=1.2,ystep=1]
\draw[-latex] (-.5,0) -- coordinate (x axis mid) (5,0) node[right,fill=white] {{\footnotesize $ x $}};
\draw[-latex] (0,-.5) -- (0,4.4) node[above,fill=white] {{\footnotesize $ y $}};
\draw[,domain=.3:3.7,samples=200,line width=.4mm,\xrwma] plot function{(x-2)**3-2*x+6};
\tkzDefPoint(1.5,2.875){A}
\tkzDrawPoint[size=7,fill=\xrwma,color=\xrwma](A)
\draw[dashed] (0,2.875) node[anchor=east]{{\scriptsize $ f(x) $}}  -- (A) -- (1.5,0) node[anchor=north] {{\scriptsize $ x $}};
\tkzLabelPoint[above=1mm](A){{\footnotesize $ M\left( x,f(x)\right)  $}}
\tkzText(4.1,3){{\footnotesize $ C_f $}}
\tkzDefPoint(0,0){O}
\tkzLabelPoint[below left](O){$ O $}
\tkzDefPoint(3,1){B}
\draw[dashed] (3,-.5) -- (3,3.7);
\tkzDrawPoint[size=7,fill=\xrwma,color=\xrwma](B)
\end{tikzpicture}}{
\begin{itemize}[itemsep=0mm]
\item Συμβολίζεται με $ C_f $ και το σύνολο των σημείων της παριστάνει σχήμα.
\item Τα σημεία της γραφικής παράστασης είναι της μορφής $Μ\left(x,f(x)\right) $.
\item Η εξίσωση $ y=f(x) $ είναι η εξίσωση της γραφικής παράστασης την οποία επαληθεύουν οι συντεταγμένες των σημείων της.
\item Κάθε κατακόρυφη ευθεία $ \varepsilon\parallel y'y $ της μορφής $ x=\kappa $ τέμνει τη $ C_f $ \textbf{σε ένα το πολύ} σημείο.
\item Οι τετμημένες $ x $ όλων των σημείων της γραφικής παράστασης σχηματίζουν το πεδίο ορισμού της.
\end{itemize}}

\begin{itemize}[itemsep=0mm]
\item Οι τεταγμένες $ f(x) $ όλων των σημείων της γραφικής παράστασης σχηματίζουν το σύνολο τιμών της.
\item Η $ C_f $ τέμνει τον οριζόντιο άξονα $ x'x $ στα σημεία όπου $ f(x)=0 $, ενώ έχει ένα το πολύ κοινό σημείο με τον κατακόρυφο άξονα, το $ M(0,f(0)) $.
\item Τα σημεία της $ C_f $ που βρίσκονται πάνω από τον οριζόντιο άξονα έχουν $ f(x)>0 $, ενώ όσα βρίσκονται κάτω από τον άξονα έχουν $ f(x)<0 $.
\item Στα σημεία τομής δύο γραφικών παραστάσεων $ C_f,C_g $ δύο συναρτήσεων $ f,g $, ισχύει $ f(x)=g(x) $.
\item Αν η $ C_f $ βρίσκεται πάνω από τη $ C_g $ τότε ισχύει $ f(x)>g(x) $, ενώ όπου η $ C_f $ είναι κάτω από τη $ C_g $ έχουμε $ f(x)<g(x) $.
\end{itemize}
\Orismos{Ίσες συναρτήσεις}
Δύο συναρτήσεις $ f $ και $ g $ είναι μεταξύ τους ίσες όταν έχουν το ίδιο πεδίο ορισμού $ A $ και επιπλέον ισχύει
\[ f(x)=g(x)\ \ ,\ \ \textrm{για κάθε }x\in A \]
Αν έχουν διαφορετικά πεδία ορισμού $ A,B $ αντίστοιχα τότε είναι ίσες στο σύνολο $ A\cap B $.\\\\
\Orismos{Σύνθεση συναρτήσεων}
Η σύνθεση μιας συνάρτησης $ f $ με μια συνάρτηση $ g $ με πεδία ορισμού $ D_f,D_g $ αντίστοιχα, ονομάζεται η συνάρτηση $ g\circ f $ με τύπο και πεδίο ορισμού
\[ (g\circ f)(x)=g(f(x))\ \ ,\ \ D_{g\circ f}=\{x\in\mathbb{R}| x\in D_f\ \textrm{ και }\ f(x)\in D_g\} \]
\begin{center}
\begin{tikzpicture}[scale=.6]
\draw(0,0) ellipse (1cm and 1.5cm);
\draw(4,0) ellipse (1cm and 1.5cm);
\begin{scope}
\draw[clip](4,0) ellipse (1cm and 1.5cm);
\draw[fill=\xrwma!30] (5,0) ellipse (1cm and 1.5cm);
\end{scope}
\draw (5,0) ellipse (1cm and 1.5cm);
\draw (9,0) ellipse (1cm and 1.5cm);
\draw (-.1,0)[fill=\xrwma!30] ellipse (.6cm and 1.1cm);
\draw[-latex] (0,.2) arc (140:40:2.95);
\draw[-latex] (4.5,.2) arc (140:40:2.95);
\draw[-latex] (0,.2) arc (140:40:5.9);
\tkzDefPoint(0,.2){A}
\tkzDefPoint(4.5,.15){B}
\tkzDefPoint(9,.15){C}
\tkzDrawPoints(A,B,C)
\tkzLabelPoint[left](A){{\footnotesize $ x $}}
\tkzLabelPoint[below](B){{\footnotesize $ f(x) $}}
\tkzLabelPoint[below](C){{\footnotesize $ g(f(x)) $}}
\tkzText(0,1.8){\footnotesize$ D_f $}
\tkzText(3.8,1.8){\footnotesize$ f(D_f) $}
\tkzText(5.2,1.8){\footnotesize$ D_g $}
\tkzText(2.4,1.55){\footnotesize$ f $}
\tkzText(6.4,1.55){\footnotesize$ g $}
\tkzText(4.5,2.55){\footnotesize$ g\circ f $}
\tkzText(9,1.8){\footnotesize$ g(D_g) $}
\draw[-latex] (4.5,-0.8) -- (3.2,-1.9) node[anchor=east] {\footnotesize $ f\left( D_f \right)\cap D_g  $};
\draw[-latex] (0,-.5) -- (-.8,-1.7) node[anchor=east,xshift=1mm] {\footnotesize $ D_{g\circ f}  $};
\end{tikzpicture}
\end{center}
\begin{itemize}[itemsep=0mm]
\item Διαβάζεται «σύνθεση της $ f $ με τη $ g $» ή «$ g $ σύνθεση $ f $».
\item Για να ορίζεται η συνάρτηση $ g\circ f $ θα πρέπει να ισχύει $ f(D_f)\cap D_g\neq\varnothing $.
\item Αντίστοιχα ορίζεται και η σύνθεση $ f\circ g $ με πεδίο ορισμού το $ D_{f\circ g}=\{x\in\mathbb{R}|x\in D_g\ \ \textrm{και}\ \ g(x)\in D_f\} $ και τύπο $ (f\circ g)(x)=f(g(x)) $.
\end{itemize}
\end{document}



