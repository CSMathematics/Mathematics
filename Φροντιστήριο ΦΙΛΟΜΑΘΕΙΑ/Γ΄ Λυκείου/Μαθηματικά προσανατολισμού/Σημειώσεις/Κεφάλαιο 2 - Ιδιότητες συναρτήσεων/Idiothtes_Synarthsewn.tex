\documentclass[twoside,nofonts,ektypwsh,math,spyros]{frontisthrio}
\usepackage[amsbb,subscriptcorrection,zswash,mtpcal,mtphrb,mtpfrak]{mtpro2}
\usepackage[no-math,cm-default]{fontspec}
\usepackage{amsmath}
\usepackage{xunicode}
\usepackage{xgreek}
\let\hbar\relax
\defaultfontfeatures{Mapping=tex-text,Scale=MatchLowercase}
\setmainfont[Mapping=tex-text,Numbers=Lining,Scale=1.0,BoldFont={Minion Pro Bold}]{Minion Pro}
\newfontfamily\scfont{GFS Artemisia}
\font\icon = "Webdings"
\usepackage{fontawesome}
\newfontfamily{\FA}{fontawesome.otf}
\xroma{red!70!black}
%------TIKZ - ΣΧΗΜΑΤΑ - ΓΡΑΦΙΚΕΣ ΠΑΡΑΣΤΑΣΕΙΣ ----
\usepackage{tikz,pgfplots}
\usepackage{tkz-euclide}
\usetkzobj{all}
\usepackage[framemethod=TikZ]{mdframed}
\usetikzlibrary{decorations.pathreplacing}
\tkzSetUpPoint[size=7,fill=white]
%-----------------------
\usepackage{calc,tcolorbox}
\tcbuselibrary{skins,theorems,breakable}
\usepackage{hhline}
\usepackage[explicit]{titlesec}
\usepackage{graphicx}
\usepackage{multicol}
\usepackage{multirow}
\usepackage{tabularx}
\usetikzlibrary{backgrounds}
\usepackage{sectsty}
\sectionfont{\centering}
\usepackage{enumitem}
\usepackage{adjustbox}
\usepackage{mathimatika,gensymb,eurosym,wrap-rl}
\usepackage{systeme,regexpatch}
%-------- ΜΑΘΗΜΑΤΙΚΑ ΕΡΓΑΛΕΙΑ ---------
\usepackage{mathtools}
%----------------------
%-------- ΠΙΝΑΚΕΣ ---------
\usepackage{booktabs}
%----------------------
%----- ΥΠΟΛΟΓΙΣΤΗΣ ----------
\usepackage{calculator}
%----------------------------
%------ ΔΙΑΓΩΝΙΟ ΣΕ ΠΙΝΑΚΑ -------
\usepackage{array}
\newcommand\diag[5]{%
\multicolumn{1}{|m{#2}|}{\hskip-\tabcolsep
$\vcenter{\begin{tikzpicture}[baseline=0,anchor=south west,outer sep=0]
\path[use as bounding box] (0,0) rectangle (#2+2\tabcolsep,\baselineskip);
\node[minimum width={#2+2\tabcolsep-\pgflinewidth},
minimum  height=\baselineskip+#3-\pgflinewidth] (box) {};
\draw[line cap=round] (box.north west) -- (box.south east);
\node[anchor=south west,align=left,inner sep=#1] at (box.south west) {#4};
\node[anchor=north east,align=right,inner sep=#1] at (box.north east) {#5};
\end{tikzpicture}}\rule{0pt}{.71\baselineskip+#3-\pgflinewidth}$\hskip-\tabcolsep}}
%---------------------------------
%---- ΟΡΙΖΟΝΤΙΟ - ΚΑΤΑΚΟΡΥΦΟ - ΠΛΑΓΙΟ ΑΓΚΙΣΤΡΟ ------
\newcommand{\orag}[3]{\node at (#1)
{$ \overcbrace{\rule{#2mm}{0mm}}^{{\scriptsize #3}} $};}
\newcommand{\kag}[3]{\node at (#1)
{$ \undercbrace{\rule{#2mm}{0mm}}_{{\scriptsize #3}} $};}
\newcommand{\Pag}[4]{\node[rotate=#1] at (#2)
{$ \overcbrace{\rule{#3mm}{0mm}}^{{\rotatebox{-#1}{\scriptsize$#4$}}}$};}
%-----------------------------------------
%------------------------------------------
\newcommand{\tss}[1]{\textsuperscript{#1}}
\newcommand{\tssL}[1]{\MakeLowercase{\textsuperscript{#1}}}
%---------- ΛΙΣΤΕΣ ----------------------
\newlist{bhma}{enumerate}{3}
\setlist[bhma]{label=\bf\textit{\arabic*\textsuperscript{o}\;Βήμα :},leftmargin=0cm,itemindent=1.8cm,ref=\bf{\arabic*\textsuperscript{o}\;Βήμα}}
\newlist{rlist}{enumerate}{3}
\setlist[rlist]{itemsep=0mm,label=\roman*.}
\newlist{brlist}{enumerate}{3}
\setlist[brlist]{itemsep=0mm,label=\bf\roman*.}
\newlist{tropos}{enumerate}{3}
\setlist[tropos]{label=\bf\textit{\arabic*\textsuperscript{oς}\;Τρόπος :},leftmargin=0cm,itemindent=2.3cm,ref=\bf{\arabic*\textsuperscript{oς}\;Τρόπος}}
% Αν μπει το bhma μεσα σε tropo τότε
%\begin{bhma}[leftmargin=.7cm]
\tkzSetUpPoint[size=7,fill=white]
\tikzstyle{pl}=[line width=0.3mm]
\tikzstyle{plm}=[line width=0.4mm]
\usepackage{etoolbox}
\makeatletter
\renewrobustcmd{\anw@true}{\let\ifanw@\iffalse}
\renewrobustcmd{\anw@false}{\let\ifanw@\iffalse}\anw@false
\newrobustcmd{\noanw@true}{\let\ifnoanw@\iffalse}
\newrobustcmd{\noanw@false}{\let\ifnoanw@\iffalse}\noanw@false
\renewrobustcmd{\anw@print}{\ifanw@\ifnoanw@\else\numer@lsign\fi\fi}
\makeatother


\begin{document}
\titlos{Γ΄ Λυκείου - Μαθηματικά Προσανατολισμού}{Όρια - Συνέχεια}{Μονοτονία - Ακρότατα - Άρτια \& Περιττή συνάρτηση}
\orismoi
\Orismos{Μονοτονία}
Μια συνάρτηση γνησίως αύξουσα ή γνησίως φθίνουσα συνάρτηση ως \textbf{γνησίως μονότονη}. Οι χαρακτηρισμοί αυτοί αφορούν τη \textbf{μονοτονία} μιας συνάρτησης.
\begin{enumerate}[itemsep=0mm,label=\bf\arabic*.]
\item \textbf{Γνησίως αύξουσα}\\ Μια συνάρτηση $ f $ ορισμένη σε ένα διάστημα $ \varDelta $ ονομάζεται γνησίως αύξουσα στο $ \varDelta $ εαν για κάθε ζεύγος αριθμών $ x_1,x_2\in\varDelta $ με $ x_1<x_2 $ ισχύει \[ f(x_1)<f(x_2) \]
Συμβολίζεται με $ f\auxousa\varDelta $.
\item \textbf{Γνησίως φθίνουσα}\\ Μια συνάρτηση $ f $ ορισμένη σε ένα διάστημα $ \varDelta $ ονομάζεται γνησίως φθίνουσα στο $ \varDelta $ εαν για κάθε ζεύγος αριθμών $ x_1,x_2\in\varDelta $ με $ x_1<x_2 $ ισχύει \[ f(x_1)>f(x_2) \]
Συμβολίζεται με $ f\fthinousa\varDelta $.
\begin{center}
\begin{tabular}{p{5cm}p{5cm}}
\begin{tikzpicture}
\draw[dashed] (3.3,1.4) node[anchor=north]{\scriptsize $x_2$} -- 
(3.3,2.58)--(1,2.58) node[left]{\scriptsize $f(x_2)$};
\draw[dashed] (2,1.4) node[anchor=north]{\scriptsize $x_1$}-- 
(2,2.08)--(1,2.08)node[left]{\scriptsize $f(x_1)$};
\begin{axis}[x=1cm,y=1cm,aks_on,xmin=-1,xmax=3,
ymin=-1.4,ymax=2,ticks=none,xlabel={\footnotesize $ x $},
ylabel={\footnotesize $ y $},belh ar]
\addplot[grafikh parastash,domain=-.5:3]{ln(x+1)};
\end{axis}
\tkzDrawPoint[size=7,fill=black](2,2.09)
\tkzDrawPoint[size=7,fill=black](3.3,2.59)
\node[fill=white,inner sep=.1mm] at (2.7,0.6) {\scriptsize $ x_1<x_2\Rightarrow f(x_1)<f(x_2)$};
\end{tikzpicture}	& \begin{tikzpicture}
\draw[dashed] (2.6,1.4) node[anchor=north]{\scriptsize $x_2$} -- 
(2.6,2.02)--(1,2.02) node[left]{\scriptsize $f(x_2)$};
\draw[dashed] (1.5,1.4) node[anchor=north]{\scriptsize $x_1$}-- 
(1.5,2.7)--(1,2.7)node[left]{\scriptsize $f(x_1)$};
\begin{axis}[x=1cm,y=1cm,aks_on,xmin=-1,xmax=3,
ymin=-1.4,ymax=2,ticks=none,xlabel={\footnotesize $ x $},
ylabel={\footnotesize $ y $},belh ar,clip=false]
\addplot[grafikh parastash,domain=-.6:3]{-0.2*(x+.5)^2+1.5};
\end{axis}
\tkzDrawPoint[size=7,fill=\xrwma](2.6,2.02)
\tkzDrawPoint[size=7,fill=\xrwma](1.5,2.7)
\node[fill=white,inner sep=.1mm] at (1.95,0.6) {\scriptsize $ x_1<x_2\Rightarrow f(x_1)>f(x_2)$};
\end{tikzpicture} \\ 
\end{tabular} 
\end{center}
\end{enumerate}
\Orismos{Ολικά Ακρότατα}
Ακρότατα, ή ολικά ακρότατα ονομάζονται οι μέγιστες ή ελάχιστες τιμές μιας συνάρτησης $ f:A\rightarrow\mathbb{R} $ τις οποίες παίρνει σε ολόκληρο το πεδίο ορισμού της.
\begin{enumerate}[itemsep=0mm,label=\bf\arabic*.]
\item \textbf{Ολικό μέγιστο}\\
Μια συνάρτηση $ f:A\rightarrow\mathbb{R} $ παρουσιάζει ολικό μέγιστο σε ένα σημείο $ x_0\in A $ του πεδίου ορισμού της όταν η τιμή $ f(x_0) $ είναι μεγαλύτερη από κάθε άλλη $ f(x) $ για κάθε σημείο $ x_0 $ του πεδίου ορισμού. Συμβολίζεται με $ \max{f(x)} $. \[ f(x)\leq f(x_0)\;\;,\;\;\textrm{για κάθε } x\in A \]
\item \textbf{Ολικό ελάχιστο}\\
Μια συνάρτηση $ f:A\rightarrow\mathbb{R} $ παρουσιάζει ολικό ελάχιστο σε ένα σημείο $ x_0\in A $ του πεδίου ορισμού της όταν η τιμή $ f(x_0) $ είναι μικρότερη από κάθε άλλη $ f(x) $ για κάθε σημείο $ x_0 $ του πεδίου ορισμού. Συμβολίζεται με $ \min{f(x)}$. \[ f(x)\geq f(x_0)\;\;,\;\;\textrm{για κάθε } x\in A \]
\end{enumerate}
\Orismos{Άρτια - Περιττή συνάρτηση}
\vspace{-5mm}
\begin{enumerate}[itemsep=0mm,label=\bf\arabic*.]
\item \textbf{Άρτια συνάρτηση}\\ Άρτια ονομάζεται μια συνάρτηση $ f:A\rightarrow\mathbb{R} $ για την οποία ισχύουν οι παρακάτω συνθήκες :
\begin{enumerate}[itemsep=0mm,label=\roman*.]
\item Για κάθε $ x\in A\Rightarrow -x\in A $
\item $ f(-x)=f(x)\;,\;\textrm{για κάθε } x\in A$
\end{enumerate}
\item \textbf{Περιττή συνάρτηση}\\ Περιττή ονομάζεται μια συνάρτηση $ f:A\rightarrow\mathbb{R} $ για την οποία ισχύουν οι παρακάτω συνθήκες :
\begin{enumerate}[itemsep=0mm,label=\roman*.]
\item Για κάθε $ x\in A\Rightarrow -x\in A $
\item $ f(-x)=-f(x)\;,\;\textrm{για κάθε } x\in A$
\end{enumerate}
\end{enumerate}
\begin{center}
\begin{tabular}{p{5cm}p{5cm}}
\begin{tikzpicture}
\draw[dashed] (0.44,.4) node[anchor=north]{\scriptsize $-x$} -- (0.44,2.96);
\draw[dashed] (3.96,.4) node[anchor=north]{\scriptsize $x$}-- (3.96,2.96);
\draw[dashed] (0.44,2.96) -- (3.96,2.96);
\begin{axis}[x=2.2cm,y=4cm,aks_on,xmin=-1,xmax=1,ymin=-.1,ymax=0.9,ticks=none,xlabel={\footnotesize $ x $},ylabel={\footnotesize $ y $},belh ar]
\addplot[grafikh parastash,domain=-.85:.85]{(x^2)};
\end{axis}
\node[fill=white,inner sep=.1mm] at (2.2,3.2){\scriptsize $f(-x)=f(x)$};
\end{tikzpicture}	& \begin{tikzpicture}
\draw[dashed] (0.44,1.98) node[anchor=south]{\scriptsize $-x$} -- (0.44,0.84);
\draw[dashed] (3.96,2) node[anchor=north]{\scriptsize $x$}-- (3.96,3.1);
\draw[dashed] (2.2,3.1) -- (3.96,3.1);
\draw[dashed] (0.44,0.84) -- (2.2,0.84);
\node at (3.4,4) {\scriptsize $f(-x)=-f(x)$};
\node at (1.85,3.1){\scriptsize $f(x)$};
\node at (2.7,.84){\scriptsize $f(-x)$};
\begin{axis}[x=2.2cm,y=2.2cm,aks_on,xmin=-1,xmax=1,ymin=-.9,ymax=.9,ticks=none,xlabel={\footnotesize $ x $},ylabel={\footnotesize $ y $},belh ar]
\addplot[grafikh parastash,domain=-.9:.9]{(x^3)};
\end{axis}
\end{tikzpicture} \\ 
\end{tabular} 
\end{center}
\begin{itemize}[itemsep=0mm]
\item Η γραφική παράσταση μιας άρτιας συνάρτησης είναι συμμετρική ως προς τον κατακόρυφο άξονα.
\item H γραφική παράσταση μιας περιττής συνάρτησης είναι συμμετρική ως προς την αρχή των αξόνων.
\item Η αρχή των αξόνων για μια περιττή συνάρτηση ονομάζεται \textbf{κέντρο συμμετρίας} της.
\end{itemize}
\end{document}



