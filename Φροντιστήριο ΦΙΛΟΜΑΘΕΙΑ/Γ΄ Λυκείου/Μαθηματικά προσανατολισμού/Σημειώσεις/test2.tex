\documentclass[11pt,a4paper]{article}
\usepackage[utf8]{inputenc}
\usepackage{nimbusserif}
\usepackage[T1]{fontenc}
\usepackage[english,greek]{babel}
\usepackage{amsmath} 
\let\myBbbk\Bbbk 
\let\Bbbk\relax 
\usepackage[amsbb,subscriptcorrection,zswash,mtpcal,mtphrb,mtpfrak]{mtpro2}
\usepackage[left=2.00cm, right=2.00cm, top=2.00cm, bottom=2.00cm]{geometry}
%------TIKZ - ΣΧΗΜΑΤΑ - ΓΡΑΦΙΚΕΣ ΠΑΡΑΣΤΑΣΕΙΣ ---- 
\usepackage{tikz,pgfplots,tkz-tab} 
\usepackage{tkz-euclide} 
\usepackage[framemethod=TikZ]{mdframed} 
\usetikzlibrary{decorations.pathreplacing} 
\tkzSetUpPoint[size=2.9,fill=white]
%----------------------- 
\usepackage{calc,tcolorbox} 
\tcbuselibrary{skins,theorems,breakable} 
\usepackage{hhline} 
\usepackage[explicit]{titlesec} 
\usepackage{graphicx} 
\usepackage{multicol} 
\usepackage{multirow} 
\usepackage{tabularx} 
\usetikzlibrary{backgrounds} 
\usepackage{sectsty} 
\sectionfont{\centering} 
\usepackage{enumitem} 
\usepackage{adjustbox} 
\usepackage{mathimatika,gensymb,eurosym,wrap-rl} 
\usepackage{systeme,regexpatch} 
%-------- ΜΑΘΗΜΑΤΙΚΑ ΕΡΓΑΛΕΙΑ --------- 
\usepackage{mathtools} 
%---------------------- 
%-------- ΠΙΝΑΚΕΣ --------- 
\usepackage{booktabs} 
%---------------------- 
%----- ΥΠΟΛΟΓΙΣΤΗΣ ---------- 
\usepackage{calculator} 
%---------------------------- 
%------------------------------------------ 
\newcommand{\tss}[1]{\textsuperscript{#1}} 
\newcommand{\tssL}[1]{\MakeLowercase{\textsuperscript{#1}}} 
\tikzstyle{pl}=[line width=0.3mm] 
\tikzstyle{plm}=[line width=0.4mm] 
\usepackage{etoolbox} 
\makeatletter 
\renewrobustcmd{\anw@true}{\let\ifanw@\iffalse} 
\renewrobustcmd{\anw@false}{\let\ifanw@\iffalse}\anw@false 
\newrobustcmd{\noanw@true}{\let\ifnoanw@\iffalse} 
\newrobustcmd{\noanw@false}{\let\ifnoanw@\iffalse}\noanw@false 
\renewrobustcmd{\anw@print}{\ifanw@\ifnoanw@\else\numer@lsign\fi\fi} 
\makeatother
\newlist{alist}{enumerate}{3}
\setlist[alist]{itemsep=0mm,label=\alph*.}
\newlist{rlist}{enumerate}{3}
\setlist[rlist]{itemsep=0mm,label=\roman*.}
\newlist{balist}{enumerate}{3}
\setlist[balist]{itemsep=0mm,label=\bf\alph*.}
\newlist{Alist}{enumerate}{3}
\setlist[Alist]{itemsep=0mm,label=\Alph*.}
\newlist{bAlist}{enumerate}{3}
\setlist[bAlist]{itemsep=0mm,label=\bf\Alph*.}
\renewcommand{\textstigma}{\textsigma\texttau}
\makeatletter
\xpatchcmd{\tkzTabLine}
{\node at (Z\thetkz@cnt@impair\thetkz@cnt@lg){$0$};} % search
{\node[fill=white,inner sep=.5mm] at (Z\thetkz@cnt@impair\thetkz@cnt@lg){$0$};} % replace
{}{}
\makeatother
\newcommand{\en}[1]{\selectlanguage{english}{#1}\selectlanguage{greek}}
\newcommand{\roloi}[4][]{
\draw[line width=.5mm,#1] (0,0) circle(2);
\foreach \n in {1,2,...,12}{
\tkzDefPoint(30*\n-90:2){A_\n}
%\tkzDrawPoint(A_\n)
\node at (-30*\n+90:1.65){\n};}
\draw[plm,,#1] (0,0)--(90-30*#2-0.5*#3:1);
\draw[pl,#1] (0,0)--(90-6*#3-0.1*#4:1.5);
\draw[#1](0,0)--(90-6*#4:1.2);
\tkzDrawPoint[fill=#1,color=#1](0,0)
\foreach \s in {1,2,...,12}{
\draw[#1](90-30*\s:1.85)--(90-30*\s:2);}
\foreach \t in {1,2,...,60}{
\draw[#1](90-6*\t:1.93)--(90-6*\t:2);}}


\begin{document}

%@ Κωδικός: Alg-Anis1ou-AnisApT-AA1
%@ Ενότητα: Ανισώσεις 1ου βαθμού
%@ Είδος: Ανισώσεις με απόλυτες τιμές
%@ Δυσκολία: 1
Να λυθούν οι ανισώσεις.
\begin{multicols}{4}
\begin{alist}
\item $ \left|x\right|<4 $
\item $ \left|x\right|>5 $
\item $ \left|x-1\right|<2 $
\item $ \left|x+2\right|>3 $
\item $ \left|2x-1\right|\leq5 $
\item $ \left|3x+4\right|\geq8 $
\item $ \left|1-x\right|<2 $
\item $ \left|3-4x\right|\geq5 $
\end{alist}
\end{multicols}




%@ Κωδικός: Alg-Anis1ou-EpilAnis-AA1
%@ Ενότητα: Ανισώσεις 1ου βαθμού
%@ Είδος: Επίλυση απλής πολυωνυμικής ανίσωσης
%@ Δυσκολία: 1
Να λυθούν οι ανισώσεις και να παρασταθούν γραφικά οι λύσεις.
\begin{multicols}{3}
\begin{alist}
\item $ 2x-3>7-3x $
\item $ 4x+5<2-x+8 $
\item $ 3x-2\leq4-2x+8 $
\item $ -x-4\geq7-3x+2 $
\item $ 7x-3+x<2x+9+5x $
\item $ -3x+8>4-5x+12 $
\end{alist}
\end{multicols}




%@ Κωδικός: Alg-Anis1ou-EpilAnis-AA2
%@ Ενότητα: Ανισώσεις 1ου βαθμού
%@ Είδος: Επίλυση απλής πολυωνυμικής ανίσωσης
%@ Δυσκολία: 1
Να λυθούν οι ανισώσεις και να παρασταθούν γραφικά οι λύσεις.
\begin{multicols}{2}
\begin{alist}
\item $ 2(x-1)+3>4-x $
\item $ 2x-3(4-x)<9+4x $
\item $ 4(3-x)+2(3x-1)<3x+2-(x-1) $
\item $ 3(2x+3)-5>5(x-4)+12 $
\item $ -2-3(4-3x)+5x\leq3-(7-2x) $
\item $ 2-(3x-4)+x\geq3(2x+3)-12-(x-2) $
\end{alist}
\end{multicols}




%@ Κωδικός: Alg-Anis1ou-EpilAnis-AA3
%@ Ενότητα: Ανισώσεις 1ου βαθμού
%@ Είδος: Επίλυση απλής πολυωνυμικής ανίσωσης
%@ Δυσκολία: 1
Να λυθούν οι ανισώσεις και να παρασταθούν γραφικά οι λύσεις.
\begin{multicols}{2}
\begin{alist}
\item $ \dfrac{x}{2}+\dfrac{x+1}{3}>1 $
\item $ \dfrac{2x-1}{3}-\dfrac{x-2}{4}<\dfrac{1}{6} $
\item $ \dfrac{x}{5}+\dfrac{3x-2}{3}\leq\dfrac{x-1}{15} $
\item $ \dfrac{4x-3}{3}-\dfrac{3-2x}{4}\geq1+\dfrac{5x}{12} $
\vfill
\columnbreak
\vfill
\item $ 2x-\dfrac{3x-2}{5}+\dfrac{x-1}{15}\leq\dfrac{1}{3}-\dfrac{2-3x}{15} $
\item $ \dfrac{-2-x}{4}+\dfrac{4x-5}{8}<3x-1-\dfrac{7-4x}{4} $
\item $ \dfrac{1-\dfrac{x}{2}}{3}>2 $
\item $ \dfrac{\dfrac{x-1}{3}+\dfrac{x-2}{4}}{2}-\dfrac{2x-1}{6}>\dfrac{x}{12} $
\end{alist}
\end{multicols}



\end{document}