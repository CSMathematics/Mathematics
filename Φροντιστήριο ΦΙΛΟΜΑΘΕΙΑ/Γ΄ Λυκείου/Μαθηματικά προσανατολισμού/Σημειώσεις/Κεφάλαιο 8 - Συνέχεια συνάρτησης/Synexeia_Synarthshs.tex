\documentclass[twoside,nofonts,ektypwsh,math,spyros]{frontisthrio}
\usepackage[amsbb,subscriptcorrection,zswash,mtpcal,mtphrb,mtpfrak]{mtpro2}
\usepackage[no-math,cm-default]{fontspec}
\usepackage{amsmath}
\usepackage{xunicode}
\usepackage{xgreek}
\let\hbar\relax
\defaultfontfeatures{Mapping=tex-text,Scale=MatchLowercase}
\setmainfont[Mapping=tex-text,Numbers=Lining,Scale=1.0,BoldFont={Minion Pro Bold}]{Minion Pro}
\newfontfamily\scfont{GFS Artemisia}
\font\icon = "Webdings"
\usepackage{fontawesome5}
\newfontfamily{\FA}{fontawesome.otf}
\xroma{red!70!black}
%------TIKZ - ΣΧΗΜΑΤΑ - ΓΡΑΦΙΚΕΣ ΠΑΡΑΣΤΑΣΕΙΣ ----
\usepackage{tikz,pgfplots}
\usepackage{tkz-euclide}
\usetkzobj{all}
\usepackage[framemethod=TikZ]{mdframed}
\usetikzlibrary{decorations.pathreplacing}
\tkzSetUpPoint[size=7,fill=white]
%-----------------------
\usepackage{calc,tcolorbox}
\tcbuselibrary{skins,theorems,breakable}
\usepackage{hhline}
\usepackage[explicit]{titlesec}
\usepackage{graphicx}
\usepackage{multicol}
\usepackage{multirow}
\usepackage{tabularx}
\usetikzlibrary{backgrounds}
\usepackage{sectsty}
\sectionfont{\centering}
\usepackage{enumitem}
\usepackage{adjustbox}
\usepackage{mathimatika,gensymb,eurosym,wrap-rl}
\usepackage{systeme,regexpatch}
%-------- ΜΑΘΗΜΑΤΙΚΑ ΕΡΓΑΛΕΙΑ ---------
\usepackage{mathtools}
%----------------------
%-------- ΠΙΝΑΚΕΣ ---------
\usepackage{booktabs}
%----------------------
%----- ΥΠΟΛΟΓΙΣΤΗΣ ----------
\usepackage{calculator}
%----------------------------
%------ ΔΙΑΓΩΝΙΟ ΣΕ ΠΙΝΑΚΑ -------
\usepackage{array}
\newcommand\diag[5]{%
\multicolumn{1}{|m{#2}|}{\hskip-\tabcolsep
$\vcenter{\begin{tikzpicture}[baseline=0,anchor=south west,outer sep=0]
\path[use as bounding box] (0,0) rectangle (#2+2\tabcolsep,\baselineskip);
\node[minimum width={#2+2\tabcolsep-\pgflinewidth},
minimum  height=\baselineskip+#3-\pgflinewidth] (box) {};
\draw[line cap=round] (box.north west) -- (box.south east);
\node[anchor=south west,align=left,inner sep=#1] at (box.south west) {#4};
\node[anchor=north east,align=right,inner sep=#1] at (box.north east) {#5};
\end{tikzpicture}}\rule{0pt}{.71\baselineskip+#3-\pgflinewidth}$\hskip-\tabcolsep}}
%---------------------------------
%---- ΟΡΙΖΟΝΤΙΟ - ΚΑΤΑΚΟΡΥΦΟ - ΠΛΑΓΙΟ ΑΓΚΙΣΤΡΟ ------
\newcommand{\orag}[3]{\node at (#1)
{$ \overcbrace{\rule{#2mm}{0mm}}^{{\scriptsize #3}} $};}
\newcommand{\kag}[3]{\node at (#1)
{$ \undercbrace{\rule{#2mm}{0mm}}_{{\scriptsize #3}} $};}
\newcommand{\Pag}[4]{\node[rotate=#1] at (#2)
{$ \overcbrace{\rule{#3mm}{0mm}}^{{\rotatebox{-#1}{\scriptsize$#4$}}}$};}
%-----------------------------------------
%------------------------------------------
\newcommand{\tss}[1]{\textsuperscript{#1}}
\newcommand{\tssL}[1]{\MakeLowercase{\textsuperscript{#1}}}
%---------- ΛΙΣΤΕΣ ----------------------
\newlist{bhma}{enumerate}{3}
\setlist[bhma]{label=\bf\textit{\arabic*\textsuperscript{o}\;Βήμα :},leftmargin=0cm,itemindent=1.8cm,ref=\bf{\arabic*\textsuperscript{o}\;Βήμα}}
\newlist{rlist}{enumerate}{3}
\setlist[rlist]{itemsep=0mm,label=\roman*.}
\newlist{brlist}{enumerate}{3}
\setlist[brlist]{itemsep=0mm,label=\bf\roman*.}
\newlist{tropos}{enumerate}{3}
\setlist[tropos]{label=\bf\textit{\arabic*\textsuperscript{oς}\;Τρόπος :},leftmargin=0cm,itemindent=2.3cm,ref=\bf{\arabic*\textsuperscript{oς}\;Τρόπος}}
% Αν μπει το bhma μεσα σε tropo τότε
%\begin{bhma}[leftmargin=.7cm]
\tkzSetUpPoint[size=7,fill=white]
\tikzstyle{pl}=[line width=0.3mm]
\tikzstyle{plm}=[line width=0.4mm]
\usepackage{etoolbox}
\makeatletter
\renewrobustcmd{\anw@true}{\let\ifanw@\iffalse}
\renewrobustcmd{\anw@false}{\let\ifanw@\iffalse}\anw@false
\newrobustcmd{\noanw@true}{\let\ifnoanw@\iffalse}
\newrobustcmd{\noanw@false}{\let\ifnoanw@\iffalse}\noanw@false
\renewrobustcmd{\anw@print}{\ifanw@\ifnoanw@\else\numer@lsign\fi\fi}
\makeatother


\begin{document}
\titlos{Γ΄ Λυκείου - Μαθηματικά προσανατολισμού}{Όρια - Συνέχεια}{Συνέχεια συνάρτησης}
\orismoi
\Orismos{Συνέχεια}
Μια συνάρτηση $ f $ ονομάζεται συνεχής σε ένα σημείο $ x_0 $ του πεδίου ορισμού της όταν το όριο της στο $ x_0 $ είναι ίσο με την τιμή της στο σημείο αυτό. Δηλαδή \[ \lim_{x\rightarrow x_0}{f(x)}=f(x_0) \]
Μια συνάρτηση $ f $ θα λέμε ότι είναι \textbf{συνεχής} εάν είναι συνεχής σε κάθε σημείο του πεδίου ορισμού της.\\\\
\thewrhmata
\Thewrhma{Συνέχεια βασικών συναρτήσεων}
Τα βασικά είδη συναρτήσεων είναι συνεχείς συναρτήσεις στο πεδίο ορισμού τους. Αναλυτικότερα για κάθε είδος ισχύουν τα παρακάτω:
\begin{enumerate}[label=\bf\arabic*.]
\item \textbf{Πολυωνυμικές}\\
Κάθε πολυωνυμική συνάρτηση $ P(x) $ είναι συνεχής στο πεδίο ορισμού της διότι για κάθε $ x\in\mathbb{R} $ ισχύει \[ \lim_{x\rightarrow x_0}{P(x)}=P(x_0) \]
\item \textbf{Ρητές}\\
Κάθε ρητή συνάρτηση $ f(x)=\dfrac{P(x)}{Q(x)} $ είναι συνεχής στο πεδίο ορισμού της $ A $ διότι για κάθε $ x\in A $ ισχύει \[ \lim_{x\rightarrow x_0}{f(x)}=\lim_{x\rightarrow x_0}{\dfrac{P(x)}{Q(x)}}=\dfrac{P(x_0)}{Q(x_0)}=f(x_0) \]
\item \textbf{Άρρητες}\\
Κάθε άρρητη συνάρτηση $ f(x)=\sqrt{A(x)} $ είναι συνεχής στο πεδίο ορισμού της $ A $. Για κάθε $ x\in A $ ισχύει:
\[ \lim_{x\rightarrow x_0}{\sqrt{A(x)}}=\sqrt{A(x_0)} \] 
\item \textbf{Τριγωνομετρικές}\\
Οι βασικές τριγωνομετρικές συναρτήσεις είναι όλες συνεχείς στο πεδίο ορισμού τους. Για κάθε $ x_0 $ ισχύει:
\begin{multicols}{2}
\begin{enumerate}[label=\roman*.]
\item $ \displaystyle{\lim_{x\rightarrow x_0}{\hm{x}}=\hm{x_0}} $
\item $ \displaystyle{\lim_{x\rightarrow x_0}{\syn{x}}=\syn{x_0}} $
\item $ \displaystyle{\lim_{x\rightarrow x_0}{\ef{x}}=\ef{x_0}} $
\item $ \displaystyle{\lim_{x\rightarrow x_0}{\syf{x}}=\syf{x_0}} $
\end{enumerate}
\end{multicols}
\item \textbf{Εκθετικές}\\
Κάθε εκθετική συνάρτηση $ f(x)=a^x $ είναι συνεχής στο πεδίου ορισμού της αφού ισχύει:
\[ \lim_{x\to x_0}{a^x}=a^{x_0} \]
\item \textbf{Λογαριθμικές}\\
Κάθε λογαριθμική συνάρτηση $ f(x)=\log_{a}{x} $ είναι συνεχής στο πεδίου ορισμού της αφού ισχύει:
\[ \lim_{x\to x_0}{\log_{a}{x}}=\log_{a}{x_0} \]
\end{enumerate}
\Thewrhma{Συνέχεια Πράξεων συναρτήσεων}
Εάν οι συναρτήσεις $ f,g $ είναι συνεχείς σε ένα κοινό σημείο $ x_0 $ των πεδίων ορισμού τους τότε και οι συναρτήσεις \[ f+g\;,\;f-g\;,\;c\cdot f\;,\;f\cdot g\;,\;\frac{f}{g}\;,\;|f|\;,\;f^\nu\textrm{ και }\sqrt[\mu]{f} \]
με $ \nu\in\mathbb{Z}\ ,\ \mu\in\mathbb{N} $, είναι συνεχείς στο σημείο $ x_0 $ εφόσον ορίζονται στο σημείο αυτό.\\\\
\Thewrhma{Συνέχεια σύνθεσης συναρτήσεων}
Εάν η συνάρτηση $ f $ είναι συνεχής σε ένα σημείο $ x_0 $ και η συνάρτηση $ g $ είναι συνεχής στο σημείο $ f(x_0) $ η σύνθεση τους $ g\circ f $ είναι συνεχής στο σημείο $ x_0 $.\\\\
\end{document}



