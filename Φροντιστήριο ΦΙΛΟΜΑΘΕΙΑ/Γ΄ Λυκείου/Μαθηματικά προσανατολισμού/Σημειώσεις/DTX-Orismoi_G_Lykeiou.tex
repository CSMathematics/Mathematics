%# Database Document : Orismoi_G_Lykeiou-----------------
%@ Document type: Σημειώσεις
%#--------------------------------------------------

\begin{enumerate}
\item
%# Database File : Analysh-Or-Pragmatikh_Synarthsh
%@ Database source: Mathematics
Πραγματική συνάρτηση με πεδίο ορισμού ένα σύνολο $ A $ είναι μια διαδικασία (αντιστοίχηση) με την οποία \textbf{κάθε} στοιχείο $ x\in A $ αντιστοιχεί σε \textbf{ένα μόνο} πραγματικό αριθμό $ y\in\mathbb{R} $. Το $ y $ λέγεται \textbf{τιμή} της συνάρτησης $ f $ στο $ x $ και συμβολίζεται $ f(x) $.
\begin{center}
\centering
\begin{tikzpicture}[scale=.6]
\draw(0,0) ellipse (1cm and 1.5cm);
\draw(4,0) ellipse (1cm and 1.5cm);
\draw[fill=\xrwma!30] (4.1,0) ellipse (.6cm and 1.1cm);
\draw[-latex] (0,.2) arc (140:40:2.6);
\tkzDefPoint(0,.2){A}
\tkzDefPoint(4,.2){B}
\tkzDrawPoints(A,B)
\tkzLabelPoint[left](A){{\footnotesize $ x $}}
\tkzLabelPoint[right](B){{\footnotesize $ y $}}
\tkzText(0,1.8){$ A $}
\tkzText(4,1.8){$ \mathbb{R} $}
\tkzText(2,1.45){$ f $}
\draw[-latex] (3.5,0) -- (2.7,-1) node[anchor=north east] {\footnotesize $ f\left( A \right)  $};
\end{tikzpicture}
\end{center}
%# End of file Analysh-Or-Pragmatikh_Synarthsh

\item
%# Database File : Analysh-Or-Grafikh_Parastash
%@ Database source: Mathematics
Γραφική παράσταση μιας συνάρτησης $ f:A\rightarrow\mathbb{R} $ ονομάζεται το σύνολο των σημείων του επιπέδου με συντεταγμένες $ M(x,y) $ όπου \[ y=f(x)\ \ \text{ για κάθε }x\in A \]
δηλαδή το σύνολο των σημείων $ M(x,f(x)) $, για κάθε $ x\in A $. Συμβολίζεται με $ C_f $ είναι 
\[ C_f=\{M(x,y)|y=f(x)\textrm{ για κάθε }x\in A\} \]
%# End of file Analysh-Or-Grafikh_Parastash
\item
%# Database File : Analysh-Or-Ises_Synarthseis_1
%@ Database source: Mathematics
Δύο συναρτήσεις $ f,g $ που έχουν το ίδιο πεδίο ορισμού $ A $ ονομάζονται ίσες δηλαδή $ f=g $ όταν ισχύει $ f(x)=g(x) $ για κάθε $ x\in A $.
%# End of file Analysh-Or-Ises_Synarthseis_1
\item
%# Database File : Analysh-Or-Prakseis_Synarthsewn
%@ Database source: Mathematics
Δίνονται δύο συναρτήσεις $ f,g $ με πεδία ορισμού $ A,B $ αντίστοιχα. 
\begin{enumerate}
\item Η συνάρτηση $ f+g $ του αθροίσματος των δύο συναρτήσεων ορίζεται ως η συνάρτηση με τύπο $ (f+g)(x)=f(x)+g(x) $ και πεδίο ορισμού $ D_{f+g}=A\cap B $.
\item Η συνάρτηση $ f-g $ της διαφοράς των δύο συναρτήσεων ορίζεται ως η συνάρτηση με τύπο $ (f-g)(x)=f(x)-g(x) $ και πεδίο ορισμού $ D_{f-g}=A\cap B $.
\item Η συνάρτηση $ f\cdot g $ του γινομένου των δύο συναρτήσεων ορίζεται ως η συνάρτηση με τύπο $ (f\cdot g)(x)=f(x)\cdot g(x) $ και πεδίο ορισμού $ D_{f\cdot g}=A\cap B $.
\item Η συνάρτηση $ \frac{f}{g} $ του πηλίκου των δύο συναρτήσεων ορίζεται ως η συνάρτηση με τύπο $ \left(\frac{f}{g}\right)(x)=\dfrac{f(x)}{g(x)} $ και πεδίο ορισμού $ D_{\frac{f}{g}}=\{x\in A\cap B:g(x)\neq 0\} $.
\end{enumerate}
Αν $ A\cap B=\varnothing $ τότε οι παραπάνω συναρτήσεις δεν ορίζονται.
%# End of file Analysh-Or-Prakseis_Synarthsewn
\item
%# Database File : Analysh-Or-Oliko_Megisto
%@ Database source: Mathematics
Έστω μια συνάρτηση $ f $ με πεδίο ορισμού ένα σύνολο $ A $ και έστω $ x_0\in A $. Η $ f $ θα λέμε ότι παρουσιάζει ολικό μέγιστο στη θέση $ x_0 $, το $ f(x_0) $ όταν 
\[ f(x)\leq f(x_0)\ \ \textrm{για κάθε}\ \ x\in A \]
Το $ x_0 $ λέγεται \textbf{θέση} του μέγιστου.
%# End of file Analysh-Or-Oliko_Megisto
\item
%# Database File : Analysh-Or-Elaxisto
%@ Database source: Mathematics
Έστω μια συνάρτηση $ f $ με πεδίο ορισμού ένα σύνολο $ A $ και έστω $ x_0\in A $. Η $ f $ θα λέμε ότι παρουσιάζει ολικό ελάχιστο στη θέση $ x_0 $, το $ f(x_0) $ όταν 
\[ f(x)\geq f(x_0)\ \ \textrm{για κάθε}\ \ x\in A \]
Το $ x_0 $ λέγεται \textbf{θέση} του ελάχιστου.
%# End of file Analysh-Or-Elaxisto
\item
%# Database File : Analysh-Or-Synarthsh_1-1
%@ Database source: Mathematics
Μια συνάρτηση $ f:A\rightarrow\mathbb{R} $ ονομάζεται $ 1-1 $ εάν για κάθε ζεύγος αριθμών $ x_1,x_2\in A $ του πεδίου ορισμού της θα ισχύει \[ x_1\neq x_2\Rightarrow f(x_1)\neq f(x_2) \]
Δηλαδή κάθε στοιχείο $ x\in A $ του πεδίου ορισμού αντιστοιχεί μέσω της συνάρτησης, σε μοναδική τιμή $ f(x) $ του συνόλου τιμών της.
%# End of file Analysh-Or-Synarthsh_1-1
\item
%# Database File : DTX-Analysh-Synart-AntistrSyn-Def1
%@ Database source: Mathematics
Έστω μια συνάρτηση $ f:A\to\mathbb{R} $ με σύνολο τιμών $ f(A) $. Η συνάρτηση με την οποία κάθε $ y\in f(A) $ αντιστοιχεί σε ένα \textbf{μοναδικό} $ x\in A $ για το οποίο ισχύει $ f(x)=y $, λέγεται αντίστροφη συνάρτηση της $ f $.
\begin{center}
\begin{tikzpicture}[scale=.6]
\draw(0,0) ellipse (1cm and 1.5cm);
\draw(4,0) ellipse (1cm and 1.5cm);
\draw[fill=\xrwma!50] (4.1,0) ellipse (.6cm and 1.1cm);
\draw[latex-] (0,.2) arc (140:40:2.6);
\tkzDefPoint(0,.2){A}
\tkzDefPoint(4,.2){B}
\tkzDrawPoints(A,B)
\tkzLabelPoint[left](A){{\footnotesize $ x $}}
\tkzLabelPoint[right](B){{\footnotesize $ y $}}
\tkzText(0,1.8){$ A $}
\tkzText(4,1.8){$ B $}
\tkzText(2,1.45){$ f^{-1} $}
\draw[-latex] (3.5,0) -- (2.7,-1) node[anchor=north east] {\footnotesize $ f\left( A \right)  $};
\end{tikzpicture}
\end{center}
\begin{itemize}[itemsep=0mm]
\item Συμβολίζεται με $ f^{-1} $ και είναι $ f^{-1}:f(A)\to A $.
\item Το πεδίο ορισμού της $ f^{-1} $ είναι το σύνολο τιμών $ f(A) $ της $ f $, ενώ το σύνολο τιμών της $ f^{-1} $ είναι το πεδίο ορισμού $ A $ της $ f $.
\item Ισχύει ότι $ x=f^{-1}(y) $ για κάθε $ y\in f(A) $.
\end{itemize}
%# End of file DTX-Analysh-Synart-AntistrSyn-Def1
\end{enumerate}
\mbox{}\\\\
\textbf{ΘΕΩΡΗΜΑ}\\
%# Database File : Analysh-Th-Grafikes_Par_Cf_Cf-1
%@ Database source: Mathematics
\bmath{Να αποδείξετε ότι οι γραφικές παραστάσεις $ C_f $ και $ C_{f^{-1}} $ των συναρτήσεων $ f $ και $ f^{-1} $ είναι συμμετρικές ως προς την ευθεία $ y=x $ που διχοτομεί τις γωνίες $ x\hat{O}y $ και $ x'\hat{O}y' $.}\\
\textbf{ΑΠΟΔΕΙΞΗ}\\
Έστω μια συνάρτηση $ f $ η οποία είναι $ 1-1 $ άρα και αντιστρέψιμη. Θα ισχύει γι αυτήν ότι
\[ f(x)=y\Rightarrow x=f^{-1}(y) \]
Αν θεωρήσουμε ένα σημείο $ M(a,\beta) $ που ανήκει στη γραφική παράσταση της $ f $ τότε
\[ f(a)=\beta\Rightarrow a=f^{-1}(\beta) \]
κάτι που σημαίνει ότι το σημείο $ M'(\beta,a) $ ανήκει στη γραφική παράσταση της $ f^{-1} $. Τα σημεία όμως $ M $ και $ M' $ είναι συμμετρικά ως προς της ευθεία $ y=x $ που διχοτομεί τις γωνίες $ x\hat{O}y $ και $ x;\hat{O}y' $. Άρα οι $ C_f $ και $ C_{f^{-1}} $ είναι συμμετρικές ως προς την ευθεία αυτή.
%# End of file Analysh-Th-Grafikes_Par_Cf_Cf-1


