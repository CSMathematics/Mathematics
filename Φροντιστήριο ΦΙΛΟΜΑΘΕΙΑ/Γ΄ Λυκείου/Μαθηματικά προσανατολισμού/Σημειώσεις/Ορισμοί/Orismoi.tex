\documentclass[ektypwsh]{frontisthrio}
\usepackage[T1]{fontenc}
\usepackage[english,greek]{babel}
\usepackage{amsmath,rotate,tikz}
\usepackage{nimbusserif} %
\usepackage[amsbb]{mtpro2}
\usepackage{tikz,pgfplots,fontawesome5}
\newcommand{\kerkissans}[1]{{\fontfamily{maksf}\selectfont #1}}
\newcommand{\en}[1]{{\selectlanguage{english}#1\selectlanguage{greek}}}
\tkzSetUpPoint[size=2.9,fill=white]
\xroma{red!70!black}
\setlist[itemize]{itemsep=0mm}
\newlist{rlist}{enumerate}{3}
\setlist[rlist]{itemsep=0mm,label=\roman*.}
\newlist{brlist}{enumerate}{3}
\setlist[brlist]{itemsep=0mm,label=\bf\roman*.}
\newlist{tropos}{enumerate}{3}
\setlist[tropos]{label=\bf\textit{\arabic*\textsuperscript{oς}\;Τρόπος :},leftmargin=0cm,itemindent=2.3cm,ref=\bf{\arabic*\textsuperscript{oς}\;Τρόπος}}
\newcommand{\tss}[1]{\textsuperscript{#1}}
\newcommand{\tssL}[1]{\MakeLowercase{\textsuperscript{#1}}}
\usepackage{hhline}
\usepackage{multicol,multirow}
\usepackage{wrap-rl,mathimatika,tkz-tab}
\usepackage{fontawesome5}
%\newfontfamily{\FA}{fontawesome.otf}
\xroma{cyan!70!black}
%------TIKZ - ΣΧΗΜΑΤΑ - ΓΡΑΦΙΚΕΣ ΠΑΡΑΣΤΑΣΕΙΣ ----
\usepackage{tikz,pgfplots}
\usepackage{tkz-euclide}
%\usetkzobj{all}
\usepackage[framemethod=TikZ]{mdframed}
\usetikzlibrary{decorations.pathreplacing}
%-----------------------
\usepackage{calc,tcolorbox}
\tcbuselibrary{skins,theorems,breakable}
\usepackage{hhline}
\usepackage[explicit]{titlesec}
\usepackage{graphicx}
\usepackage{multicol}
\usepackage{multirow}
\usepackage{tabularx}
\usetikzlibrary{backgrounds}
\usepackage{sectsty}
\sectionfont{\centering}
\usepackage{enumitem}
\usepackage{adjustbox}
\usepackage{mathimatika,gensymb,eurosym,wrap-rl}
\usepackage{systeme,regexpatch}
%-------- ΜΑΘΗΜΑΤΙΚΑ ΕΡΓΑΛΕΙΑ ---------
\usepackage{mathtools}
%----------------------
%-------- ΠΙΝΑΚΕΣ ---------
\usepackage{booktabs}
%----------------------
%----- ΥΠΟΛΟΓΙΣΤΗΣ ----------
\usepackage{calculator}
%----------------------------

%---------------------------------
%---- ΟΡΙΖΟΝΤΙΟ - ΚΑΤΑΚΟΡΥΦΟ - ΠΛΑΓΙΟ ΑΓΚΙΣΤΡΟ ------
\newcommand{\orag}[3]{\node at (#1)
{$ \overcbrace{\rule{#2mm}{0mm}}^{{\scriptsize #3}} $};}
\newcommand{\kag}[3]{\node at (#1)
{$ \undercbrace{\rule{#2mm}{0mm}}_{{\scriptsize #3}} $};}
\newcommand{\Pag}[4]{\node[rotate=#1] at (#2)
{$ \overcbrace{\rule{#3mm}{0mm}}^{{\rotatebox{-#1}{\scriptsize$#4$}}}$};}
%-----------------------------------------

% Αν μπει το bhma μεσα σε tropo τότε
%\begin{bhma}[leftmargin=.7cm]
\tikzstyle{pl}=[line width=0.3mm]
\tikzstyle{plm}=[line width=0.4mm]
\usepackage{etoolbox}
%\makeatletter
%\renewrobustcmd{\anw@true}{\let\ifanw@\iffalse}
%\renewrobustcmd{\anw@false}{\let\ifanw@\iffalse}\anw@false
%\newrobustcmd{\noanw@true}{\let\ifnoanw@\iffalse}
%\newrobustcmd{\noanw@false}{\let\ifnoanw@\iffalse}\noanw@false
%\renewrobustcmd{\anw@print}{\ifanw@\ifnoanw@\else\numer@lsign\fi\fi}
%\makeatother


\begin{document}
\titlos{Γ΄ Λυκείου - Μαθηματικά Προσανατολισμού}{Ορισμοί}{Από όλη την ύλη}
\Orismos{Πραγματική Συνάρτηση}
Πραγματική συνάρτηση με πεδίο ορισμού ένα σύνολο $ A $ είναι μια διαδικασία (αντιστοίχηση) με την οποία \textbf{κάθε} στοιχείο $ x\in A $ αντιστοιχεί σε \textbf{ένα μόνο} πραγματικό αριθμό $ y\in\mathbb{R} $. Το $ y $ λέγεται \textbf{τιμή} της συνάρτησης $ f $ στο $ x $ και συμβολίζεται $ f(x) $.
\begin{center}
\centering
\begin{tikzpicture}[scale=.6]
\draw(0,0) ellipse (1cm and 1.5cm);
\draw(4,0) ellipse (1cm and 1.5cm);
\draw[fill=\xrwma!30] (4.1,0) ellipse (.6cm and 1.1cm);
\draw[-latex] (0,.2) arc (140:40:2.6);
\tkzDefPoint(0,.2){A}
\tkzDefPoint(4,.2){B}
\tkzDrawPoints(A,B)
\tkzLabelPoint[left](A){{\footnotesize $ x $}}
\tkzLabelPoint[right](B){{\footnotesize $ y $}}
\tkzText(0,1.8){$ A $}
\tkzText(4,1.8){$ B $}
\tkzText(2,1.45){$ f $}
\draw[-latex] (3.5,0) -- (2.7,-1) node[anchor=north east] {\footnotesize $ f\left( A \right)  $};
\end{tikzpicture}
\end{center}
\Orismos{Σύνολο τιμών}
Σύνολο τιμών μιας συνάρτησης $ f $ με πεδίο ορισμού $ A $ λέγεται το σύνολο που περιέχει όλες τις τιμές $ f(x) $ της συνάρτησης για κάθε  $ x\in A $. Συμβολίζεται με $ f(A) $ και είναι
\[ f(A)=\{y\in\mathbb{R}:y=f(x)\ \textrm{για κάθε}\ x\in A\} \]
\Orismos{Γραφική παράσταση}
Γραφική παράσταση μιας συνάρτησης $ f $ με πεδίο ορισμού ένα σύνολο $ A $ ονομάζεται το σύνολο των σημείων της μορφής $ M(x,f(x)) $ για κάθε $ x\in A $. Συμβολίζεται με $ C_f $
\[ C_f=\{M(x,y):y=f(x)\ \textrm{για κάθε}\ x\in A\} \]
\Orismos{Ίσες συναρτήσεις με κοινό πεδίο ορισμού}
Δύο συναρτήσεις $ f,g $ που έχουν το ίδιο πεδίο ορισμού $ A $ ονομάζονται ίσες δηλαδή $ f=g $ όταν ισχύει $ f(x)=g(x) $ για κάθε $ x\in A $.\\\\
\Orismos{Ίσες συναρτήσεις με διαφορετικά πεδία ορισμού}
Δύο συναρτήσεις $ f,g $ με πεδία ορισμού $ A,B $ αντίστοιχα, ονομάζονται ίσες δηλαδή $ f=g $ όταν ισχύει $ f(x)=g(x) $ για κάθε $ x\in A\cap B $. Αν $ A\cap B=\varnothing $ τότε δεν είναι ίσες.\\\\
\Orismos{Πράξεις μεταξύ συναρτήσεων}
Δίνονται δύο συναρτήσεις $ f,g $ με πεδία ορισμού $ A,B $ αντίστοιχα. 
\begin{enumerate}
\item Η συνάρτηση $ f+g $ του αθροίσματος των δύο συναρτήσεων ορίζεται ως η συνάρτηση με τύπο $ f(x)+g(x) $ και πεδίο ορισμού $ D_{f+g}=A\cap B $.
\item Η συνάρτηση $ f-g $ της διαφοράς των δύο συναρτήσεων ορίζεται ως η συνάρτηση με τύπο $ f(x)-g(x) $ και πεδίο ορισμού $ D_{f-g}=A\cap B $.
\item Η συνάρτηση $ f\cdot g $ του γινομένου των δύο συναρτήσεων ορίζεται ως η συνάρτηση με τύπο $ f(x)\cdot g(x) $ και πεδίο ορισμού $ D_{f\cdot g}=A\cap B $.
\item Η συνάρτηση $ \frac{f}{g} $ του πηλίκου των δύο συναρτήσεων ορίζεται ως η συνάρτηση με τύπο $ \frac{f(x)}{g(x)} $ και πεδίο ορισμού $ D_{\frac{f}{g}}=\{x\in A\cap B:g(x)\neq 0\} $.
\end{enumerate}
Αν $ A\cap B=\varnothing $ τότε οι παραπάνω συναρτήσεις δεν ορίζονται.\\\\
\Orismos{Σύνθεση συναρτήσεων}
Αν $f, g$ είναι δύο συναρτήσεις με πεδίο ορισμού $Α, Β$ αντιστοίχως, τότε ονομάζουμε \bmath{σύνθεση της $f$ με την $g$}, και τη συμβολίζουμε με $g\circ f$, τη συνάρτηση με τύπο
\[ (g\circ f)(x)=g(f(x))\]
Το πεδίο ορισμού της $g\circ f$ αποτελείται από όλα τα στοιχεία x του πεδίου ορισμού της $f$ για τα οποία το $f(x)$ ανήκει στο πεδίο ορισμού της g. Δηλαδή είναι το σύνολο
\[ D_{g\circ f}=\{x\in\mathbb{R}| x\in A\ \textrm{ και }\ f(x)\in B\} \]
Είναι φανερό ότι η $g\circ f$ ορίζεται αν $A_1\neq\varnothing$, δηλαδή αν $f(A)∩B\neq\varnothing$.
\begin{center}
\begin{tikzpicture}[scale=.6]
\draw(0,0) ellipse (1cm and 1.5cm);
\draw(4,0) ellipse (1cm and 1.5cm);
\begin{scope}
\draw[clip](4,0) ellipse (1cm and 1.5cm);
\draw[fill=\xrwma!30] (5,0) ellipse (1cm and 1.5cm);
\end{scope}
\draw (5,0) ellipse (1cm and 1.5cm);
\draw (9,0) ellipse (1cm and 1.5cm);
\draw (-.1,0)[fill=\xrwma!30] ellipse (.6cm and 1.1cm);
\draw[-latex] (0,.2) arc (140:40:2.95);
\draw[-latex] (4.5,.2) arc (140:40:2.95);
\draw[-latex] (0,.2) arc (140:40:5.9);
\tkzDefPoint(0,.2){A}
\tkzDefPoint(4.5,.15){B}
\tkzDefPoint(9,.15){C}
\tkzDrawPoints(A,B,C)
\tkzLabelPoint[left](A){{\footnotesize $ x $}}
\tkzLabelPoint[below](B){{\footnotesize $ f(x) $}}
\tkzLabelPoint[below](C){{\footnotesize $ g(f(x)) $}}
\tkzText(0,1.8){\footnotesize$ A $}
\tkzText(3.8,1.8){\footnotesize$ f(A) $}
\tkzText(5.2,1.8){\footnotesize$ B $}
\tkzText(2.4,1.55){\footnotesize$ f $}
\tkzText(6.4,1.55){\footnotesize$ g $}
\tkzText(4.5,2.55){\footnotesize$ g\circ f $}
\tkzText(9,1.8){\footnotesize$ g(B) $}
\draw[-latex] (4.5,-0.8) -- (3.2,-1.9) node[anchor=east] {\footnotesize $ f\left( A \right)\cap B  $};
\draw[-latex] (0,-.5) -- (-.8,-1.7) node[anchor=east,xshift=1mm] {\footnotesize $ D_{g\circ f}  $};
\end{tikzpicture}
\end{center}
Για να ορίζεται η συνάρτηση $ g\circ f $ θα πρέπει να ισχύει $ f(A)\cap B\neq\varnothing $.\\\\
(Αντίστοιχα ορίζεται και η σύνθεση $ f\circ g $ με πεδίο ορισμού το $ D_{f\circ g}=\{x\in\mathbb{R}|x\in B\ \ \textrm{και}\ \ g(x)\in A\} $ και τύπο $ (f\circ g)(x)=f(g(x)) $.)\\\\
\Orismos{Γνησίως μονότονη συνάρτηση}
Δίνεται μια συνάρτηση $ f $ ορισμένη σε ένα διάστημα $ \Delta $ του πεδίου ορισμού της και έστω $ x_1,x_2 $ δύο στοιχεία του $ \Delta $. Η $ f $ θα ονομάζεται
\begin{enumerate}
\item γνησίως αύξουσα στο $\Delta $ αν για κάθε $ x_1,x_2\in\Delta $ με $ x_1<x_2 $ ισχύει $ f(x_1)<f(x_2) $:
\[ x_1<x_2\Rightarrow f(x_1)<f(x_2) \]
\item γνησίως φθίνουσα στο $\Delta $ αν για κάθε $ x_1,x_2\in\Delta $ με $ x_1<x_2 $ ισχύει $ f(x_1)>f(x_2) $:
\[ x_1<x_2\Rightarrow f(x_1)>f(x_2) \]
\end{enumerate}
Η $ f $ σε κάθε περίπτωση λέγεται \textbf{γνησίως μονότονη}.\\\\
\Orismos{Ολικά ακρότατα}
Έστω μια συνάρτηση $ f $ με πεδίο ορισμού ένα σύνολο $ A $ και έστω $ x_0\in A $. Η $ f $ θα λέμε ότι παρουσιάζει
\begin{enumerate}
\item ολικό μέγιστο στο $ x_0 $ το $ f(x_0) $ όταν 
\[ f(x)\leq f(x_0)\ \ \textrm{για κάθε}\ \ x\in A \]
\item ολικό ελάχιστο στο $ x_0 $ το $ f(x_0) $ όταν 
\[ f(x)\geq f(x_0)\ \ \textrm{για κάθε}\ \ x\in A \]
Το ολικό μέγιστο και ολικό ελάχιστο μιας συνάρτησης ονομάζονται \textbf{ολικά ακρότατα}. Το $ x_0 $ λέγεται \textbf{θέση} ακρότατου.
\end{enumerate}
\Orismos{Συνάρτηση $ 1-1 $}
Μια συνάρτηση $ f:A\rightarrow\mathbb{R} $ ονομάζεται $ 1-1 $ εάν κάθε στοιχείο $ x\in A $ του πεδίου ορισμού αντιστοιχεί μέσω της συνάρτησης, σε μοναδική τιμή $ f(x) $ του συνόλου τιμών της. Για κάθε ζεύγος αριθμών $ x_1,x_2\in A $ του πεδίου ορισμού της $ f $ θα ισχύει \[ x_1\neq x_2\Rightarrow f(x_1)\neq f(x_2) \]
\Orismos{Αντίστροφη συνάρτηση}
Έστω μια συνάρτηση $ f:A\to\mathbb{R} $ με σύνολο τιμών $ f(A) $. Η συνάρτηση με την οποία κάθε $ y\in f(A) $ αντιστοιχεί σε ένα \textbf{μοναδικό} $ x\in A $ για το οποίο ισχύει $ f(x)=y $, λέγεται αντίστροφη συνάρτηση της $ f $.
\begin{center}
\begin{tikzpicture}[scale=.6]
\draw(0,0) ellipse (1cm and 1.5cm);
\draw(4,0) ellipse (1cm and 1.5cm);
\draw[fill=\xrwma!50] (4.1,0) ellipse (.6cm and 1.1cm);
\draw[latex-] (0,.2) arc (140:40:2.6);
\tkzDefPoint(0,.2){A}
\tkzDefPoint(4,.2){B}
\tkzDrawPoints(A,B)
\tkzLabelPoint[left](A){{\footnotesize $ x $}}
\tkzLabelPoint[right](B){{\footnotesize $ y $}}
\tkzText(0,1.8){$ A $}
\tkzText(4,1.8){$ B $}
\tkzText(2,1.45){$ f^{-1} $}
\draw[-latex] (3.5,0) -- (2.7,-1) node[anchor=north east] {\footnotesize $ f\left( A \right)  $};
\end{tikzpicture}
\end{center}
\begin{itemize}[itemsep=0mm]
\item Συμβολίζεται με $ f^{-1} $ και είναι $ f^{-1}:f(A)\to A $.
\item Το πεδίο ορισμού της $ f^{-1} $ είναι το σύνολο τιμών $ f(A) $ της $ f $, ενώ το σύνολο τιμών της $ f^{-1} $ είναι το πεδίο ορισμού $ A $ της $ f $.
\item Ισχύει ότι $ x=f^{-1}(y) $ για κάθε $ y\in f(A) $.
\end{itemize}
\Orismos{Συνέχεια συνάρτησης σε σημείο}
Μια συνάρτηση $ f $ ονομάζεται συνεχής σε ένα σημείο $ x_0 $ του πεδίου ορισμού της όταν το όριο της στο $ x_0 $ είναι ίσο με την τιμή της στο σημείο αυτό. Δηλαδή \[ \lim_{x\rightarrow x_0}{f(x)}=f(x_0) \]
\Orismos{Συνέχεια συνάρτησης σε σύνολο}
\begin{enumerate}
\item Μια συνάρτηση $ f $ θα λέμε ότι είναι \textbf{συνεχής} εάν είναι συνεχής σε κάθε σημείο του πεδίου ορισμού της.
\item Μια συνάρτηση $ f $ θα λέγεται συνεχής σε ένα \textbf{ανοιχτό} διάστημα $ (a,\beta) $ εάν είναι συνεχής σε κάθε σημείο του διαστήματος.
\item Μια συνάρτηση $ f $ θα λέγεται συνεχής σε ένα \textbf{κλειστό} διάστημα $ [a,\beta] $ εάν είναι συνεχής σε κάθε σημείο του ανοιχτού διαστήματος και επιπλέον ισχύει
\[ \lim_{x\to a^+}{f(x)}=f(a)\ \ \textrm{και}\ \ \lim_{x\to\beta^-}{f(x)}=f(\beta) \]
\end{enumerate}
\Orismos{Παράγωγος σε σημείο}
Μια συνάρτηση $ f $ λέγεται \textbf{παραγωγίσιμη} σε ένα σημείο $ x_0 $ του πεδίου ορισμού της αν το όριο
\[ \lim_{x\rightarrow x_0}{\frac{f(x)-f(x_0)}{x-x_0}} \]
υπάρχει και είναι πραγματικός αριθμός. Το όριο αυτό ονομάζεται \textbf{παράγωγος} της $ f $ στο $ x_0 $ και συμβολίζεται $ f'(x_0) $.\\\\
\Orismos{Παραγωγίσιμη συνάρτηση}
\vspace{-7mm}
\begin{enumerate}
\item Μια συνάρτηση $ f $ θα λέγεται παραγωγίσιμη στο \textbf{πεδίο ορισμού} της ή απλά παραγωγίσιμη, όταν είναι παραγωγίσιμη σε κάθε σημείο $ x_0\in D_f $.
\item Μια συνάρτηση $ f $ θα λέγεται παραγωγίσιμη σε ένα \textbf{ανοικτό} διάστημα $ (a,\beta) $ του πεδίου ορισμού της όταν είναι παραγωγίσιμη σε κάθε σημείο $ x_0\in(a,\beta) $.
\item Μια συνάρτηση $ f $ θα λέγεται παραγωγίσιμη σε ένα \textbf{κλειστό} διάστημα $ [a,\beta] $ του πεδίου ορισμού της όταν είναι παραγωγίσιμη σε κάθε σημείο $ x_0\in(a,\beta) $ και επιπλέον ισχύει
\[ \lim_{x\to a^+}{\frac{f(x)-f(a)}{x-a}}\in\mathbb{R}\ \ \textrm{ και }\ \ \lim_{x\to \beta^-}{\frac{f(x)-f(\beta)}{x-\beta}}\in\mathbb{R} \]
\end{enumerate}
\Orismos{Πρώτη παράγωγος}
Έστω μια συνάρτηση $ f:Α\to\mathbb{R} $ και έστω $ A_1 $ το σύνολο των σημείων $ x\in A $ για τα οποία η $ f $ είναι παραγωγίσιμη. Η συνάρτηση με την οποία κάθε $ x\in A_1 $ αντιστοιχεί στο $ f'(x) $ ονομάζεται \textbf{πρώτη παράγωγος} της $ f $ η απλά \textbf{παράγωγος} της $ f $. Συμβολίζεται με $ f' $.\\\\
\Orismos{Δεύτερη παράγωγος}
Έστω $ A_1 $  το σύνολο των σημείων για τα οποία η $ f $ είναι παραγωγίσιμη. Αν υποθέσουμε ότι το $ A_1 $ είναι διάστημα ή ένωση διαστημάτων τότε η παράγωγος της $ f' $, αν υπάρχει, λέγεται δεύτερη παράγωγος της $ f $ και συμβολίζεται με $ f'' $. Επαγωγικά ορίζεται και η $ \nu- $οστή παράγωγος της $ f $ και συμβολίζεται με $ f^{(\nu)} $. Δηλαδή
\[ f^{(\nu)}=\left[f^{(\nu-1)}\right]' \]
\Orismos{Ρυθμός μεταβολής}
Αν δύο μεταβλητά μεγέθη  $ x , y $ συνδέονται με τη σχέση $ y = f(x) $ , όταν  $ f  $ είναι μια συνάρτηση παραγωγίσιμη στο $ x_0 $, τότε ονομάζουμε \textbf{ρυθμό μεταβολής} του $ y $ ως προς το $ x $ στο σημείο  $ x_0 $ την παράγωγο $ f '(x_0) $.\\\\
\Orismos{Τοπικό μέγιστο}
Μια συνάρτηση $ f $, με πεδίο ορισμού $ A $, θα λέμε ότι παρουσιάζει τοπικό μέγιστο στο $ x_0\in A $ όταν υπάρχει $ \delta>0 $ τέτοιο ώστε
\[ f(x)\leq f(x_0), \ \ \textrm{για κάθε }x\in A\cap(x_0-\delta,x_0+\delta) \]
Το $ x_0 $ λέγεται \textbf{θέση} η σημείο τοπικού μέγιστου, ενώ το $ f(x_0) $ τοπικό μέγιστο της $ f $.\\\\
\Orismos{Τοπικό ελάχιστο}
Μια συνάρτηση $ f $, με πεδίο ορισμού $ A $, θα λέμε ότι παρουσιάζει τοπικό ελάχιστο στο $ x_0\in A $ όταν υπάρχει $ \delta>0 $ τέτοιο ώστε
\[ f(x)\geq f(x_0), \ \ \textrm{για κάθε }x\in A\cap(x_0-\delta,x_0+\delta) \]
Το $ x_0 $ λέγεται \textbf{θέση} η σημείο τοπικού ελάχιστου, ενώ το $ f(x_0) $ τοπικό ελάχιστο της $ f $.\\\\
\Orismos{Τοπικά ακρότατα}
Τα τοπικά ελάχιστα και τα τοπικά μέγιστα της $ f $ ονομάζονται τοπικά ακρότατα της $ f $.\\\\
\Orismos{Κυρτή συνάρτηση}
Μια συνάρτηση $ f $ λέμε ότι στρέφει τα κοίλα προς τα άνω ή είναι κυρτή στο $ \Delta $, αν η $ f' $ είναι γνησίως αύξουσα στο $ \Delta $.\\\\
\Orismos{Κοίλη συνάρτηση}
Μια συνάρτηση $ f $ λέμε ότι στρέφει τα κοίλα προς τα κάτω ή είναι κοίλη στο $ \Delta $, αν η $ f' $ είναι γνησίως φθίνουσα στο $ \Delta $.\\\\
\Orismos{Σημείο καμπής}
Έστω μια συνάρτηση $ f $ παραγωγίσιμη σε ένα διάστημα $ (a,\beta) $, με εξαίρεση ίσως ένα σημείο $ x_0 $. Αν:
\begin{itemize}[itemsep=0mm]
\item η $ f $ είναι κυρτή στο $ (a,x_0) $ και κοίλη στο $ (x_0,\beta) $ η αντιστρόφως και
\item η $ C_f $ έχει εφαπτομένη στο $ A(x_0,f(x_0)) $
\end{itemize}
τότε το σημείο $ A(x_0,f(x_0)) $ λέγεται \textbf{σημείο καμπής} της $ C_f $.\\\\
\Orismos{Κατακόρυφη ασύμπτωτη}
Αν ένα τουλάχιστον από τα όρια $ \lim\limits_{x\to x_0^-}{f(x)},\lim\limits_{x\to x_0^+}{f(x)} $ ισούται με $ \pm\infty $ τότε η ευθεία $ x=x_0 $ λέγεται \textbf{κατακόρυφη ασύμπτωτη} της $ C_f $.\\\\
\Orismos{Οριζόντια ασύμπτωτη}
Αν $ \lim\limits_{x\to +\infty}{f(x)}=l $ (αντιστοίχως $ \lim\limits_{x\to -\infty}{f(x)}=l $) τότε η ευθεία $ y=l $ λέγεται \textbf{οριζόντια ασύμπτωτη} της $ C_f $ στο $ +\infty $ (αντίστοιχα στο $ -\infty $).\\\\
\Orismos{Πλάγια ασύμπτωτη}
Η ευθεία $ y=\lambda x+\beta $ λέγεται ασύμπτωτη της $ C_f $ στο $ +\infty $ (αντιστοίχως στο $ -\infty $) αν και μόνο αν
\[ \lim\limits_{x\to +\infty}{[f(x)-(\lambda x+\beta)]=0} \]
αντίστοιχα στο $ -\infty $ αν 
\[ \lim_{x\to -\infty}{[f(x)-(\lambda x+\beta)]=0} \]
\begin{itemize}[itemsep=0mm]
\item Αν $ \lambda=0 $ η ασύμπτωτη είναι οριζόντια.
\item Αν $ \lambda\neq 0 $ η ασύμπτωτη είναι πλάγια.
\end{itemize}
\Orismos{Αρχική συνάρτηση}
Αρχική συνάρτηση ή παράγουσα μιας συνάρτησης $f$ σε ένα διάστημα $\varDelta$ του πεδίου ορισμού της, ονομάζεται κάθε παραγωγίσιμη συνάρτηση $F$ για την οποία ισχύει
\[ F'(x)=f(x)\ \ ,\ \ \text{για κάθε }x\in\varDelta \]
\vfill
\begin{flushright}
\begin{minipage}{7.2cm}
{\small Πηγή:\\
Μαθηματικά Προσανατολισμού Γ΄ Λυκείου, Οδηγός προετοιμασίας για τις πανελλαδικές εξετάσεις - Συλλογικό Έργο - Εκδόσεις Ελληνοεκδοτική - 2016\\\en{lysari team}}
\end{minipage}
\end{flushright}
\end{document}



