\documentclass[twoside,nofonts,ektypwsh,math,spyros]{frontisthrio}
\usepackage[amsbb,subscriptcorrection,zswash,mtpcal,mtphrb,mtpfrak]{mtpro2}
\usepackage[no-math,cm-default]{fontspec}
\usepackage{amsmath}
\usepackage{xunicode}
\usepackage{xgreek}
\let\hbar\relax
\defaultfontfeatures{Mapping=tex-text,Scale=MatchLowercase}
\setmainfont[Mapping=tex-text,Numbers=Lining,Scale=1.0,BoldFont={Minion Pro Bold}]{Minion Pro}
\newfontfamily\scfont{GFS Artemisia}
\font\icon = "Webdings"
\usepackage{fontawesome}
\newfontfamily{\FA}{fontawesome.otf}
\xroma{red!70!black}
%------TIKZ - ΣΧΗΜΑΤΑ - ΓΡΑΦΙΚΕΣ ΠΑΡΑΣΤΑΣΕΙΣ ----
\usepackage{tikz,pgfplots}
\usepackage{tkz-euclide}
\usetkzobj{all}
\usepackage[framemethod=TikZ]{mdframed}
\usetikzlibrary{decorations.pathreplacing}
\tkzSetUpPoint[size=7,fill=white]
%-----------------------
\usepackage{calc,tcolorbox}
\tcbuselibrary{skins,theorems,breakable}
\usepackage{hhline}
\usepackage[explicit]{titlesec}
\usepackage{graphicx}
\usepackage{multicol}
\usepackage{multirow}
\usepackage{tabularx}
\usetikzlibrary{backgrounds}
\usepackage{sectsty}
\sectionfont{\centering}
\usepackage{enumitem}
\usepackage{adjustbox}
\usepackage{mathimatika,gensymb,eurosym,wrap-rl}
\usepackage{systeme,regexpatch}
%-------- ΜΑΘΗΜΑΤΙΚΑ ΕΡΓΑΛΕΙΑ ---------
\usepackage{mathtools}
%----------------------
%-------- ΠΙΝΑΚΕΣ ---------
\usepackage{booktabs}
%----------------------
%----- ΥΠΟΛΟΓΙΣΤΗΣ ----------
\usepackage{calculator}
%----------------------------
%------ ΔΙΑΓΩΝΙΟ ΣΕ ΠΙΝΑΚΑ -------
\usepackage{array}
\newcommand\diag[5]{%
\multicolumn{1}{|m{#2}|}{\hskip-\tabcolsep
$\vcenter{\begin{tikzpicture}[baseline=0,anchor=south west,outer sep=0]
\path[use as bounding box] (0,0) rectangle (#2+2\tabcolsep,\baselineskip);
\node[minimum width={#2+2\tabcolsep-\pgflinewidth},
minimum  height=\baselineskip+#3-\pgflinewidth] (box) {};
\draw[line cap=round] (box.north west) -- (box.south east);
\node[anchor=south west,align=left,inner sep=#1] at (box.south west) {#4};
\node[anchor=north east,align=right,inner sep=#1] at (box.north east) {#5};
\end{tikzpicture}}\rule{0pt}{.71\baselineskip+#3-\pgflinewidth}$\hskip-\tabcolsep}}
%---------------------------------
%---- ΟΡΙΖΟΝΤΙΟ - ΚΑΤΑΚΟΡΥΦΟ - ΠΛΑΓΙΟ ΑΓΚΙΣΤΡΟ ------
\newcommand{\orag}[3]{\node at (#1)
{$ \overcbrace{\rule{#2mm}{0mm}}^{{\scriptsize #3}} $};}
\newcommand{\kag}[3]{\node at (#1)
{$ \undercbrace{\rule{#2mm}{0mm}}_{{\scriptsize #3}} $};}
\newcommand{\Pag}[4]{\node[rotate=#1] at (#2)
{$ \overcbrace{\rule{#3mm}{0mm}}^{{\rotatebox{-#1}{\scriptsize$#4$}}}$};}
%-----------------------------------------
%------------------------------------------
\newcommand{\tss}[1]{\textsuperscript{#1}}
\newcommand{\tssL}[1]{\MakeLowercase{\textsuperscript{#1}}}
%---------- ΛΙΣΤΕΣ ----------------------
\newlist{bhma}{enumerate}{3}
\setlist[bhma]{label=\bf\textit{\arabic*\textsuperscript{o}\;Βήμα :},leftmargin=0cm,itemindent=1.8cm,ref=\bf{\arabic*\textsuperscript{o}\;Βήμα}}
\newlist{rlist}{enumerate}{3}
\setlist[rlist]{itemsep=0mm,label=\roman*.}
\newlist{brlist}{enumerate}{3}
\setlist[brlist]{itemsep=0mm,label=\bf\roman*.}
\newlist{tropos}{enumerate}{3}
\setlist[tropos]{label=\bf\textit{\arabic*\textsuperscript{oς}\;Τρόπος :},leftmargin=0cm,itemindent=2.3cm,ref=\bf{\arabic*\textsuperscript{oς}\;Τρόπος}}
% Αν μπει το bhma μεσα σε tropo τότε
%\begin{bhma}[leftmargin=.7cm]
\tkzSetUpPoint[size=7,fill=white]
\tikzstyle{pl}=[line width=0.3mm]
\tikzstyle{plm}=[line width=0.4mm]
\usepackage{etoolbox}
\makeatletter
\renewrobustcmd{\anw@true}{\let\ifanw@\iffalse}
\renewrobustcmd{\anw@false}{\let\ifanw@\iffalse}\anw@false
\newrobustcmd{\noanw@true}{\let\ifnoanw@\iffalse}
\newrobustcmd{\noanw@false}{\let\ifnoanw@\iffalse}\noanw@false
\renewrobustcmd{\anw@print}{\ifanw@\ifnoanw@\else\numer@lsign\fi\fi}
\makeatother


\begin{document}
\titlos{Γ΄ Λυκείου - Μαθηματικά προσανατολισμού}{Όρια - Συνέχεια}{Μη πεπερασμένο όριο}
\orismoi
\Orismos{Μη πεπερασμένο όριο}
Δίνεται μια συνάρτηση $ f $ με πεδίο ορισμού ένα σύνολο $ A $ και $ x_0\in\mathbb{R} $. Το όριο της $ f $ όταν το $ x $ τείνει στο $ x_0 $ λέγεται μη πεπερασμένο όταν είναι ένα από τα $ +\infty $ ή $ -\infty $.
\[ \lim_{x\to x_0}{f(x)}=\pm\infty \]
\thewrhmata
\Thewrhma{Μη πεπερασμένο όριο και διάταξη}
Θεωρούμε μια συνάρτηση $ f:A\to\mathbb{R} $ και $ x_0\in\mathbb{R} $. Θα ισχύει ότι:
\begin{rlist}
\item Αν $ \lim_{x\to x_0}{f(x)}=+\infty $ τότε $ f(x)>0 $ σε μια περιοχή του $ x_0 $.
\item Αν $ \lim_{x\to x_0}{f(x)}=-\infty $ τότε $ f(x)<0 $ σε μια περιοχή του $ x_0 $.
\end{rlist}
\Thewrhma{Ιδιότητες μη πεπερασμένου ορίου}
Θεωρούμε μια συνάρτηση $ f:A\to\mathbb{R} $ και $ x_0\in\mathbb{R} $. Αν η $ f $ έχει μη πεπερασμένο όριο στο σημείο $ x_0 $ τότε ισχύουν γι αυτήν οι ακόλουθες ιδιότητες.
\begin{enumerate}[itemsep=0mm,label=\roman*.]
\item Αν $ \displaystyle{\lim_{x\rightarrow x_0}f(x)}=\pm\infty$ τότε $ \displaystyle{\lim_{x\rightarrow x_0}\frac{1}{f(x)}=0} $.
\item Αν ισχύουν οι σχέσεις $ \ccases{\displaystyle{\lim_{x\rightarrow x_0}f(x)}=0\\f(x)>0 \ \ (<0)\textrm{ κοντά στο }x_0}$ τότε $ \displaystyle{\lim_{x\rightarrow x_0}\frac{1}{f(x)}=+\infty(-\infty)} $.
\item $ \displaystyle{\lim_{x\rightarrow x_0}f(x)\pm\infty}\Rightarrow\displaystyle{\lim_{x\rightarrow x_0}|f(x)|}=+\infty $.
\item $ \displaystyle{\lim_{x\rightarrow x_0}f(x)}=+\infty\Rightarrow\displaystyle{\lim_{x\rightarrow x_0}\!\sqrt[\kappa]{f(x)}=+\infty} $.
\begin{multicols}{2}
\item $ \displaystyle{\lim_{x\rightarrow x_0}\frac{1}{(x-x_0)^{2\nu}}=+\infty} $.
\item $ \displaystyle{\lim_{x\rightarrow x_0}\frac{1}{|x-x_0|}=+\infty} $.
\item $ \displaystyle{\lim_{x\rightarrow x_0^-}\frac{1}{(x-x_0)^{2\nu+1}}=-\infty} $.
\item $ \displaystyle{\lim_{x\rightarrow x_0^+}\frac{1}{(x-x_0)^{2\nu+1}}=+\infty} $.
\end{multicols}
\item Δεν υπάρχουν τα όρια της μορφής $ \displaystyle{\lim_{x\rightarrow x_0}\frac{1}{(x-x_0)^{2\nu+1}}} $.
\end{enumerate}
\Thewrhma{Μη πεπερασμένο όριο και διάταξη 2}
Αν για δύο συναρτήσεις $ f,g $ ισχύει η σχέση $ f(x)\leq g(x) $ κοντά στο $ x_0 $ τότε παίρνουμε ότι:
\begin{enumerate}[itemsep=0mm,label=\roman*.]
\item $ \displaystyle{\lim_{x\rightarrow x_0}f(x)}=+\infty\Rightarrow \displaystyle{\lim_{x\rightarrow x_0}g(x)}=+\infty $.
\item $ \displaystyle{\lim_{x\rightarrow x_0}g(x)}=-\infty\Rightarrow \displaystyle{\lim_{x\rightarrow x_0}f(x)}=-\infty $.
\end{enumerate}
\Thewrhma{Όριο αθροίσματος}
Για το όριο του αθροίσματος δύο συναρτήσεων $ f,g $ έχουμε τις παρακάτω περιπτώσεις:
\begin{center}
\begin{tabular}{c|cccccc}
\hline  Όριο συνάρτησης & \multicolumn{6}{c}{Τιμή ορίου} \rule[-2ex]{0pt}{5.5ex}\\
\hhline{=======} \rule[-2ex]{0pt}{5.5ex} $ \displaystyle{\lim_{x\rightarrow x_0}f(x)} $ & $ a\in\mathbb{R} $ & $ a\in\mathbb{R} $ & $ +\infty $ & $ +\infty $ & $ -\infty $ & $ -\infty $ \\ 
\rule[-2ex]{0pt}{5.5ex} $ \displaystyle{\lim_{x\rightarrow x_0}g(x)} $ & $ +\infty $ & $ -\infty $ & $ +\infty $ & $ -\infty $ & $ +\infty $ & $ -\infty $ \\ 
\hhline{~------}\rule[-2ex]{0pt}{5.5ex} $ \displaystyle{\lim_{x\rightarrow x_0}(f+g)(x)} $ & $ +\infty $ & $ -\infty $ & $ +\infty $ & Απροσδιόριστη & Απροσδιόριστη & $ -\infty $ \\ 
\hline 
\end{tabular}
\end{center}
\Thewrhma{Όριο γινομένου}
Για το όριο του γινομένου δύο συναρτήσεων $ f,g $ έχουμε τις παρακάτω περιπτώσεις:
\begin{center}
\begin{tabular}{c|cccccccccc}
\hline  Όριο συνάρτησης & \multicolumn{10}{c}{Τιμή ορίου} \rule[-2ex]{0pt}{5.5ex}\\
\hhline{===========} \rule[-2ex]{0pt}{5.5ex} $ \displaystyle{\lim_{x\rightarrow x_0}f(x)} $ & $ a>0 $ & $ a>0 $ & $ a<0 $ & $ a<0 $ & $ +\infty $ & $ +\infty $ & $ -\infty $ & $ -\infty $ & 0 & 0 \\ 
\rule[-2ex]{0pt}{5.5ex} $ \displaystyle{\lim_{x\rightarrow x_0}g(x)} $ & $ +\infty $ & $ -\infty $ & $ +\infty $ & $ -\infty $ & $ +\infty $ & $ -\infty $ & $ +\infty $ & $ -\infty $ & $ +\infty $ & $ -\infty $ \\ 
\hhline{~----------}\rule[-2ex]{0pt}{5.5ex} $ \displaystyle{\lim_{x\rightarrow x_0}(f\cdot g)(x)} $ & $ +\infty $ & $ -\infty $ & $ -\infty $ & $ +\infty $ & $ +\infty $ & $ -\infty $ & $ -\infty $ & $ +\infty $ & Απρ. & Απρ. \\ 
\hline 
\end{tabular}
\end{center}
Από τα δύο προηγούμενα θεωρήματα παίρνουμε τις εξής απροσδιόριστες μορφές:
\begin{center}
\begin{tcolorbox}[title=ΑΠΡΟΣΔΙΟΡΙΣΤΕΣ ΜΟΡΦΕΣ,hbox,lifted shadow={1mm}{-2mm}{3mm}{0.3mm}%
{black!50!white},enhanced,
colback=black!5!white,
boxrule=0.1pt,
colframe=\xrwma,
fonttitle=\bfseries,center title,toptitle=3pt,bottomtitle=3pt]
\begin{minipage}{5.5cm}
\vspace{-3mm}
\[ +\infty-\infty\ \ ,\ \ -\infty+\infty \]
\[ 0\cdot(\pm\infty)\ \ ,\ \ \dfrac{0}{0}\ \ ,\ \ \dfrac{\pm\infty}{\pm\infty} \]
\end{minipage}
\end{tcolorbox}
\end{center}
\end{document}