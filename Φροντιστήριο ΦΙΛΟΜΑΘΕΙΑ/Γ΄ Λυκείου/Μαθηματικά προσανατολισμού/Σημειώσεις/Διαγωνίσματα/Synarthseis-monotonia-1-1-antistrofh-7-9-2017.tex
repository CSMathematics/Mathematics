\documentclass[twoside,nofonts,internet,math,spyros]{frontisthrio-diag}
\usepackage[amsbb,subscriptcorrection,zswash,mtpcal,mtphrb,mtpfrak]{mtpro2}
\usepackage[no-math,cm-default]{fontspec}
\usepackage{amsmath}
\usepackage{xunicode}
\usepackage{xgreek}
\let\hbar\relax
\defaultfontfeatures{Mapping=tex-text,Scale=MatchLowercase}
\setmainfont[Mapping=tex-text,Numbers=Lining,Scale=1.0,BoldFont={Minion Pro Bold}]{Minion Pro}
\newfontfamily\scfont{GFS Artemisia}
\font\icon = "Webdings"
\usepackage{fontawesome}
\newfontfamily{\FA}{fontawesome.otf}
\xroma{red!70!black}
%------TIKZ - ΣΧΗΜΑΤΑ - ΓΡΑΦΙΚΕΣ ΠΑΡΑΣΤΑΣΕΙΣ ----
\usepackage{tikz,pgfplots}
\usepackage{tkz-euclide}
\usetkzobj{all}
\usepackage[framemethod=TikZ]{mdframed}
\usetikzlibrary{decorations.pathreplacing}
\tkzSetUpPoint[size=7,fill=white]
%-----------------------
\usepackage{calc,tcolorbox}
\tcbuselibrary{skins,theorems,breakable}
\usepackage{hhline}
\usepackage[explicit]{titlesec}
\usepackage{graphicx}
\usepackage{multicol}
\usepackage{multirow}
\usepackage{tabularx}
\usetikzlibrary{backgrounds}
\usepackage{sectsty}
\sectionfont{\centering}
\usepackage{enumitem}
\usepackage{adjustbox}
\usepackage{mathimatika,gensymb,eurosym,wrap-rl}
\usepackage{systeme,regexpatch}
%-------- ΜΑΘΗΜΑΤΙΚΑ ΕΡΓΑΛΕΙΑ ---------
\usepackage{mathtools}
%----------------------
%-------- ΠΙΝΑΚΕΣ ---------
\usepackage{booktabs}
%----------------------
%----- ΥΠΟΛΟΓΙΣΤΗΣ ----------
\usepackage{calculator}
%----------------------------
%---- ΟΡΙΖΟΝΤΙΟ - ΚΑΤΑΚΟΡΥΦΟ - ΠΛΑΓΙΟ ΑΓΚΙΣΤΡΟ ------
\newcommand{\orag}[3]{\node at (#1)
{$ \overcbrace{\rule{#2mm}{0mm}}^{{\scriptsize #3}} $};}
\newcommand{\kag}[3]{\node at (#1)
{$ \undercbrace{\rule{#2mm}{0mm}}_{{\scriptsize #3}} $};}
\newcommand{\Pag}[4]{\node[rotate=#1] at (#2)
{$ \overcbrace{\rule{#3mm}{0mm}}^{{\rotatebox{-#1}{\scriptsize$#4$}}}$};}
%-----------------------------------------
%------------------------------------------
\newcommand{\tss}[1]{\textsuperscript{#1}}
\newcommand{\tssL}[1]{\MakeLowercase{\textsuperscript{#1}}}
%---------- ΛΙΣΤΕΣ ----------------------
\newlist{bhma}{enumerate}{3}
\setlist[bhma]{label=\bf\textit{\arabic*\textsuperscript{o}\;Βήμα :},leftmargin=0cm,itemindent=1.8cm,ref=\bf{\arabic*\textsuperscript{o}\;Βήμα}}
\newlist{rlist}{enumerate}{3}
\setlist[rlist]{itemsep=0mm,label=\roman*.}
\newlist{brlist}{enumerate}{3}
\setlist[brlist]{itemsep=0mm,label=\bf\roman*.}
\newlist{tropos}{enumerate}{3}
\setlist[tropos]{label=\bf\textit{\arabic*\textsuperscript{oς}\;Τρόπος :},leftmargin=0cm,itemindent=2.3cm,ref=\bf{\arabic*\textsuperscript{oς}\;Τρόπος}}
% Αν μπει το bhma μεσα σε tropo τότε
%\begin{bhma}[leftmargin=.7cm]
\tkzSetUpPoint[size=7,fill=white]
\tikzstyle{pl}=[line width=0.3mm]
\tikzstyle{plm}=[line width=0.4mm]
\usepackage{etoolbox}
\makeatletter
\renewrobustcmd{\anw@true}{\let\ifanw@\iffalse}
\renewrobustcmd{\anw@false}{\let\ifanw@\iffalse}\anw@false
\newrobustcmd{\noanw@true}{\let\ifnoanw@\iffalse}
\newrobustcmd{\noanw@false}{\let\ifnoanw@\iffalse}\noanw@false
\renewrobustcmd{\anw@print}{\ifanw@\ifnoanw@\else\numer@lsign\fi\fi}
\makeatother


\begin{document}
\titlos{Γ΄ Λυκείου - Μαθηματικά προσανατολισμού}{Επαναληπτικό διαγώνισμα}{Συναρτήσεις - Ιδιότητες Συναρτήσεων - Αντίστροφη συνάρτηση}
\begin{thema}
\item \mbox{}\\\vspace{-5mm}
\begin{erwthma}
\item Έστω μια συνάρτηση $ f:A\to\mathbb{R} $ και $ \varDelta $ ένα διάστημα του πεδίου ορισμού της. Πότε η συνάρτηση $ f $ λέγεται γνησίως αύξουσα στο διάστημα $ \varDelta $;
\monades{8}
\item Να δώσετε τον ορισμό του ολικού ελάχιστου μιας συνάρτησης $ f $ με πεδίο ορισμού ένα σύνολο $ A $.\\\monades{7}
\item \swstolathospan
\begin{alist}
\item Αν μια συνάρτηση $ f:A\to\mathbb{R} $ είναι γνησίως μονότονη σε κάθε διάστημα του πεδίου ορισμού της τότε είναι και $ 1-1 $.
\item Έστω $ f $ μια αντιστρέψιμη συνάρτηση. Το σύνολο τιμών της $ f $ είναι το πεδίο ορισμού της $ f^{-1} $.
\item Το πεδίο ορισμού μιας ρητής συνάρτησης $ f(x)=\frac{P(x)}{Q(x)} $ ισούται με $ A=\{x\in\mathbb{R}|Q(x)=0\} $.
\item Αν ένα σημείο $ A(x,y) $ ανήκει στη γραφική παράσταση μιας συνάρτησης $ f $ τότε για τις συντεταγμένες του ισχύει $ y=f(x) $.
\item Στα σημεία όπου η γραφική παράσταση μιας συνάρτησης $ f $ τέμνει τον οριζόντιο άξονα $ x'x $ ισχύει $ x=0 $.
\end{alist}\monades{10}
\end{erwthma}
\item \mbox{}\\
Δίνονται οι ακόλουθες συναρτήσεις $ f:A\to\mathbb{R} $ και $ g:B\to\mathbb{R} $ με τύπους $ f(x)=\frac{x}{x-2} $ και $ g(x)=\ln{(x-1)} $.
\begin{erwthma}
\item Να βρείτε τα πεδία ορισμού $ A,B $ των συναρτήσεων $ f,g $ αντίστοιχα.\monades{5}
\item Να αποδείξετε ότι η συνάρτηση $ f $ είναι γνησίως φθίνουσα στα διαστήματα $ (-\infty,2) $, $ (2,+\infty) $, ενώ η $ g $ είναι γνησίως αύξουσα στο $ (1,+\infty) $.\monades{10}
\item Να εξηγήσετε γιατί είναι αντιστρέψιμες οι δύο παραπάνω συναρτήσεις και να βρείτε τις αντίστροφες συναρτήσεις τους $ f^{-1} $ και $ g^{-1} $.\\\monades{10}
\end{erwthma}
\item 
\item 
\end{thema}

\end{document}
