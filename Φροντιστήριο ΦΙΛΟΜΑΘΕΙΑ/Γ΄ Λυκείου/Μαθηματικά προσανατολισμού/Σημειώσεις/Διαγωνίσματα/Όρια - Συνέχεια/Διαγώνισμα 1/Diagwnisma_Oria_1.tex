\documentclass[twoside,nofonts,ektypwsh,math,spyros]{frontisthrio-diag}
\usepackage[amsbb,subscriptcorrection,zswash,mtpcal,mtphrb,mtpfrak]{mtpro2}
\usepackage[no-math,cm-default]{fontspec}
\usepackage{amsmath}
\usepackage{xunicode}
\usepackage{xgreek}
\let\hbar\relax
\defaultfontfeatures{Mapping=tex-text,Scale=MatchLowercase}
\setmainfont[Mapping=tex-text,Numbers=Lining,Scale=1.0,BoldFont={Minion Pro Bold}]{Minion Pro}
\newfontfamily\scfont{GFS Artemisia}
\font\icon = "Webdings"
\usepackage{fontawesome}
\newfontfamily{\FA}{fontawesome.otf}
\xroma{red!70!black}
%------TIKZ - ΣΧΗΜΑΤΑ - ΓΡΑΦΙΚΕΣ ΠΑΡΑΣΤΑΣΕΙΣ ----
\usepackage{tikz,pgfplots}
\usepackage{tkz-euclide}
\usetkzobj{all}
\usepackage[framemethod=TikZ]{mdframed}
\usetikzlibrary{decorations.pathreplacing}
\tkzSetUpPoint[size=7,fill=white]
%-----------------------
\usepackage{calc,tcolorbox}
\tcbuselibrary{skins,theorems,breakable}
\usepackage{hhline}
\usepackage[explicit]{titlesec}
\usepackage{graphicx}
\usepackage{multicol}
\usepackage{multirow}
\usepackage{tabularx}
\usetikzlibrary{backgrounds}
\usepackage{sectsty}
\sectionfont{\centering}
\usepackage{enumitem}
\usepackage{adjustbox}
\usepackage{mathimatika,gensymb,eurosym,wrap-rl}
\usepackage{systeme,regexpatch}
%-------- ΜΑΘΗΜΑΤΙΚΑ ΕΡΓΑΛΕΙΑ ---------
\usepackage{mathtools}
%----------------------
%-------- ΠΙΝΑΚΕΣ ---------
\usepackage{booktabs}
%----------------------
%----- ΥΠΟΛΟΓΙΣΤΗΣ ----------
\usepackage{calculator}
%----------------------------
%------ ΔΙΑΓΩΝΙΟ ΣΕ ΠΙΝΑΚΑ -------
\usepackage{array}
\newcommand\diag[5]{%
\multicolumn{1}{|m{#2}|}{\hskip-\tabcolsep
$\vcenter{\begin{tikzpicture}[baseline=0,anchor=south west,outer sep=0]
\path[use as bounding box] (0,0) rectangle (#2+2\tabcolsep,\baselineskip);
\node[minimum width={#2+2\tabcolsep-\pgflinewidth},
minimum  height=\baselineskip+#3-\pgflinewidth] (box) {};
\draw[line cap=round] (box.north west) -- (box.south east);
\node[anchor=south west,align=left,inner sep=#1] at (box.south west) {#4};
\node[anchor=north east,align=right,inner sep=#1] at (box.north east) {#5};
\end{tikzpicture}}\rule{0pt}{.71\baselineskip+#3-\pgflinewidth}$\hskip-\tabcolsep}}
%---------------------------------
%---- ΟΡΙΖΟΝΤΙΟ - ΚΑΤΑΚΟΡΥΦΟ - ΠΛΑΓΙΟ ΑΓΚΙΣΤΡΟ ------
\newcommand{\orag}[3]{\node at (#1)
{$ \overcbrace{\rule{#2mm}{0mm}}^{{\scriptsize #3}} $};}
\newcommand{\kag}[3]{\node at (#1)
{$ \undercbrace{\rule{#2mm}{0mm}}_{{\scriptsize #3}} $};}
\newcommand{\Pag}[4]{\node[rotate=#1] at (#2)
{$ \overcbrace{\rule{#3mm}{0mm}}^{{\rotatebox{-#1}{\scriptsize$#4$}}}$};}
%-----------------------------------------
%------------------------------------------
\newcommand{\tss}[1]{\textsuperscript{#1}}
\newcommand{\tssL}[1]{\MakeLowercase{\textsuperscript{#1}}}
%---------- ΛΙΣΤΕΣ ----------------------
\newlist{bhma}{enumerate}{3}
\setlist[bhma]{label=\bf\textit{\arabic*\textsuperscript{o}\;Βήμα :},leftmargin=0cm,itemindent=1.8cm,ref=\bf{\arabic*\textsuperscript{o}\;Βήμα}}
\newlist{rlist}{enumerate}{3}
\setlist[rlist]{itemsep=0mm,label=\roman*.}
\newlist{brlist}{enumerate}{3}
\setlist[brlist]{itemsep=0mm,label=\bf\roman*.}
\newlist{tropos}{enumerate}{3}
\setlist[tropos]{label=\bf\textit{\arabic*\textsuperscript{oς}\;Τρόπος :},leftmargin=0cm,itemindent=2.3cm,ref=\bf{\arabic*\textsuperscript{oς}\;Τρόπος}}
% Αν μπει το bhma μεσα σε tropo τότε
%\begin{bhma}[leftmargin=.7cm]
\tkzSetUpPoint[size=7,fill=white]
\tikzstyle{pl}=[line width=0.3mm]
\tikzstyle{plm}=[line width=0.4mm]
\usepackage{etoolbox}
\makeatletter
\renewrobustcmd{\anw@true}{\let\ifanw@\iffalse}
\renewrobustcmd{\anw@false}{\let\ifanw@\iffalse}\anw@false
\newrobustcmd{\noanw@true}{\let\ifnoanw@\iffalse}
\newrobustcmd{\noanw@false}{\let\ifnoanw@\iffalse}\noanw@false
\renewrobustcmd{\anw@print}{\ifanw@\ifnoanw@\else\numer@lsign\fi\fi}
\makeatother
\ekthetesdeiktes
\usepackage{path}
\pathal
\let\oldlim\lim
\renewcommand{\lim}{\displaystyle\oldlim}


\let\oldccases\ccases
\renewcommand{\ccases}[1]{{%
  \def\arraystretch{1.2}%
  \LEFTRIGHT\lbrace.{\,\array{@{}c@{\quad}l@{}}#1\endarray}%
}}

\begin{document}
\titlos{Γ΄ Λυκείου - Μαθηματικά προσανατολισμού}{Όρια - Συνέχεια}{Όριο σε σημείο - Μη πεπερασμένο όριο - Όριο στο άπειρο}
\begin{thema}
\item \mbox{}\\
\vspace{-7mm}
\begin{erwthma}
\item Δίνεται ένα πολυώνυμο $ P(x)=a_\nu x^\nu+a_{\nu-1}x^{\nu-1}+\ldots+a_1x+a_0 $ και $ x_0\in\mathbb{R} $. Να αποδείξετε ότι $ {\displaystyle{\lim_{x\to x_0}{P(x)}=P(x_0)}} $.\monades{8}
\item Έστω $ f:A\to\mathbb{R} $ μια ρητή συνάρτηση με $ f(x)=\dfrac{P(x)}{Q(x)} $ όπου $ P(x)=a_\nu x^\nu+a_{\nu-1}x^{\nu-1}+\ldots+a_1x+a_0 $ και $ Q(x)=\beta_\mu x^\mu+\beta_{\mu-1}x^{\mu-1}+\ldots+\beta_1x+\beta_0 $ πολυώνυμα βαθμών $ \nu $ και $ \mu $ αντίστοιχα. Να υπολογίσετε το όριο \[ \lim_{x\to +\infty}{f(x)} \] εξετάζοντας περιπτώσεις για τη σχέση μεταξύ των βαθμών $ \nu $ και $ \mu $ των δύο πολυωνύμων.\monades{7}
\item \swstolathospan
\begin{alist}
\item Αν υπάρχει το όριο μιας συνάρτησης $ f $ σε ένα σημείο $ x_0 $ τότε τα πλευρικά όρια $ \lim_{x\to x_0^-}{f(x)} $ και $ \lim_{x\to x_0^+}f(x) $ θα είναι μεταξύ τους ίσα.
\item Αν για δύο συναρτήσεις $ f,g $ ισχύουν οι σχέσεις $ \lim_{x\to x_0}{f(x)}=0 $ και $ \lim_{x\to x_0}{g(x)}=+\infty $ τότε παίρνουμε ότι $ \lim_{x\to x_0}{f(x)\cdot g(x)}=0 $.
\item Αν για μια συνάρτηση $ f $, με πεδίο ορισμού ένα σύνολο $ A $, ισχύει ότι $ \lim_{x\to x_0}{f(x)}>0 $ τότε προκύπτει $ f(x)>0 $ για κάθε $ x\in A $.
\item Δίνεται μια συνάρτηση $ f:A\to\mathbb{R} $ και $ x_0\in\mathbb{R} $. Αν ισχύουν οι σχέσεις $ \lim_{x\to x_0}{f(x)}=0 $ και $ f(x)>0 $ κοντά στο $ x_0 $ τότε $ \lim_{x\to x_0}{\dfrac{1}{f(x)}}=+\infty $.
\item Έστω μια εκθετική συνάρτηση $ f(x)=a^x $ με $ a>1 $. Τότε θα ισχύει ότι $ \lim_{x\to -\infty}{f(x)}=0 $.
\end{alist}\monades{10}
\end{erwthma}
\item \mbox{}\\
Δίνεται η συνάρτηση $ f:\mathbb{R}\to\mathbb{R} $ με
\[ f(x)=\ccases{\dfrac{3x^2-5x+2}{x^2-x}, & \textrm{ αν }\ x>1\\ [4mm]\dfrac{\hm{[a(x-1)]}}{x-1}, & \textrm{ αν }\  x<1} \]
\begin{erwthma}
\item Αν γνωρίζουμε ότι υπάρχει το όριο $ \lim_{x\to 1}{f(x)} $ τότε να αποδείξετε ότι $ a=1 $.\monades{9}
\item Να υπολογίσετε το όριο $ \lim_{x\to+\infty}{f(x)} $.\monades{8}
\item Να υπολογίσετε το όριο $ \lim_{x\to-\infty}{f(x)} $.\monades{8}
\end{erwthma}
\item \mbox{}\\
Δίνεται η συνάρτηση $ f:\mathbb{R}\to\mathbb{R} $ για την οποία ισχύει
\[ \frac{\hm{x}+2x^2+10x}{x+2}\leq f(x)\leq \dfrac{2x^2+8x+7}{x+1} \] για κάθε $ x>0 $.
\begin{erwthma}
\item Να αποδείξετε ότι $ \lim_{x\to +\infty}{\dfrac{f(x)}{x}}=2 $ και $ \lim_{x\to +\infty}{(f(x)-2x)}=6 $.\monades{12}
\item Να υπολογίσετε το όριο $ \lim_{x\to+\infty}{\dfrac{f(x)+3x+x^2\cdot\hm{\frac{1}{x}}}{xf(x)-2x^2-4x+3}} $.\monades{13}
\end{erwthma}\mbox{}\\
\item \mbox{}\\
Δίνεται η συνάρτηση $ f:\mathbb{R}\to\mathbb{R} $, για την οποία ισχύει:
\[ \lim_{x\to 0}{\dfrac{x(f(x)+2)+\hm{3x}}{\sqrt{x+4}-2}}=24 \]
Να βρείτε τα όρια:
\begin{erwthma}
\item $ \lim_{x\to 0}{f(x)} $\monades{7}
\item $ \lim_{x\to 0}{\dfrac{f(x)-4}{|f(x)+1|-|f^2(x)-3f(x)|}} $ \monades{8}
\item Θεωρούμε τη συνάρτηση 
\[ g(x)=\ln{\left(f^2(x)-2f(x)+\syn^2{\left(f(x)-1\right)}\right) }-\ln{\left( f^2(x)-2f(x)+1\right)} \]
Αν η $ f $ είναι $ 1-1 $ και $ f(0)=1 $, να βρείτε
\begin{multicols}{2}
\begin{rlist}
\item το πεδίο ορισμού της $ g $,
\item το όριο $ \lim_{x\to 0}{g(x)} $.
\end{rlist}
\end{multicols}\monades{4+6}
\end{erwthma}
\end{thema}
\kaliepityxia
\end{document}
