\documentclass[ektypwsh]{diag-pan-xelatex}
\usepackage[amsbb]{mtpro2}
\usepackage[no-math,cm-default]{fontspec}
\usepackage{xunicode}
\usepackage{xltxtra}
\usepackage{xgreek}
\defaultfontfeatures{Mapping=tex-text,Scale=MatchLowercase}
\setmainfont[Mapping=tex-text,Numbers=Lining,Scale=1.0,BoldFont={Minion Pro Bold}]{Minion Pro}
\newfontfamily\scfont{GFS Artemisia}
\font\icon = "Webdings"
\usepackage{mtpro2}
\usepackage[left=2.00cm, right=2.00cm, top=2.00cm, bottom=3.00cm]{geometry}
\xroma{cyan}
\usepackage{tikz}
\usepackage{tkz-euclide,tkz-fct}
\usepackage{wrapfig}
\usetkzobj{all}
\usepackage{calc}
\usepackage{systeme,regexpatch}
\usepackage[framemethod=TikZ]{mdframed}
\usepackage{adjustbox}
\usepackage{graphicx}
\usepackage{multicol}
\usepackage{multirow}
\usepackage{enumitem}
\usepackage{tabularx}
\usepackage[decimalsymbol=comma]{siunitx}
\usepackage{mathimatika}
\newlist{rlist}{enumerate}{3}
\setlist[rlist]{itemsep=0mm,label=\roman*.}


\begin{document}
\titlos{ΜΑΘΗΜΑΤΙΚΑ ΚΑΤΕΥΘΥΝΣΗΣ Γ΄ ΕΠΑΛ}{ΔΙΑΦΟΡΙΚΟΣ ΛΟΓΙΣΜΟΣ}
\begin{thema}
\item \mbox{}\\\vspace{-5mm}
\begin{erwthma}
\item Έστω μια συνάρτηση $ f $ ορισμένη σε ένα σύνολο $ A $ και $ x_0 $ ένα σημείο του πεδίου ορισμού της. Να διατυπώσετε τον ορισμό της συνέχειας της συνάρτησης $ f $ στο σημείο $ x_0 $ καθώς και σε όλο το πεδίο ορισμού της. \monades{8}
\item Ποιά συνθήκη θα πρέπει να ισχύει ώστε μια συνάρτηση $ f $ να παρουσιάζει μέγιστη τιμή σε ένα σημείο $ x_0 $ του πεδίου ορισμού της;\monades{7}
\item \swstolathos
\begin{rlist}
\item Η παράγωγος μιας συνάρτησης $ f $ σε ένα σημείο $ x_0 $ του πεδίου ορισμού της είναι πάντα πραγματικός αριθμός.
\item Μια συνάρτηση $ f $ θα είναι αύξουσα σε ένα διάστημα $ \varDelta $ του πεδίου ορισμού της αν ισχύει $ f'(x)>0 $.
\item Αν $ f,g $ είναι δύο παραγωγίσιμες συναρτήσεις τότε η παράγωγος της συνάρτησης $ f(g(x)) $ δίνεται από τον τύπο $ \left[f(g(x))\right]'=f'(g(x))\cdot g'(x) $.
\item Ο κανόνας ο οποίος μας δίνει την παράγωγο του πηλίκου δύο συναρτήσεων είναι  
\[ \left( \frac{f(x)}{g(x)}\right)'=\frac{f'(x)}{g'(x)} \]
\item Η παράγωγος της συνάρτησης $ f(x)=\hm{2\varphi} $ είναι $ f'(x)=2\syn{2\varphi} $.
\end{rlist}\monades{10}
\end{erwthma}
\item \textbf{Συνέχεια συνάρτησης}\\
Έστω μια συνάρτηση $ f:\mathbb{R}\rightarrow\mathbb{R} $ η οποία είναι συνεχής και δίνεται από τον παρακάτω τύπο :
\[ f(x)=\ccases{\dfrac{x^2-16}{x-4} & ,\ x\neq4\\
a+3 & ,\ x=4} \]
\begin{erwthma}
\item Να υπολογιστεί η παράμετρος $ a $.\monades{9}
\item Να υπολογιστεί η παράγωγος της συνάρτησης $ f $.\monades{10}
\item Να υπολογιστεί το όριο $ \displaystyle{\lim_{x\rightarrow 4}{\frac{f(x)}{x+4}}} $\monades{6}
\end{erwthma}
\item \textbf{Μονοτονία - Ακρότατα}\\
Δίνεται η συνάρτηση $ f:\mathbb{R}\rightarrow\mathbb{R} $ με τύπο $ f(x)=\frac{x^2+4}{x} $.
\begin{erwthma}
\item Να βρεθεί το πεδίο ορισμού της συνάρτησης $ f $.\monades{5}
\item Να βρεθεί η παράγωγος της συνάρτησης $ f $.\monades{7}
\item Να εξεταστεί η συνάρτηση $ f $ ως προς τη μονοτονία και τα ακρότατα.\monades{13}
\end{erwthma}
\item \textbf{Σύνθετο θέμα}\\
Μια ομάδα περιβαλλοντολόγων εκτιμά ότι το βάρος $ B $ (σε τόνους) ενός παγόβουνου μεταβάλλεται με το χρόνο $ t $ (σε έτη) σύμφωνα με τον τύπο
\[ B(t)=-\frac{t^3}{3}+2t^2+12t+15\ ,\ \ 0\leq t\leq 10 \]
\begin{erwthma}
\item Να βρεθεί ο ρυθμός μεταβολής του βάρους του παγόβουνου στο δεύτερο χρόνο από την έναρξη της μελέτης.\monades{7}
\item Ποιά χρονική στιγμή το βάρος του παγόβουνου γίνεται μέγιστο;\monades{9}
\item Ποιά χρονική στιγμή ο ρυθμός μεταβολής του βάρους του παγόβουνου γίνεται μέγιστος;\\\monades{9}
\end{erwthma}
\end{thema}
\end{document}

