\documentclass[ektypwsh]{diag-pan-xelatex}
\usepackage[amsbb,subscriptcorrection,zswash,mtpcal,mtphrb]{mtpro2}
\usepackage[no-math,cm-default]{fontspec}
\usepackage{xunicode}
\usepackage{xgreek}
\defaultfontfeatures{Mapping=tex-text,Scale=MatchLowercase}
\setmainfont[Mapping=tex-text,Numbers=Lining,Scale=1.0,BoldFont={Minion Pro Bold}]{Minion Pro}
\newfontfamily\scfont{GFS Artemisia}
\font\icon = "Webdings"
\usepackage[amsbb,subscriptcorrection,zswash,mtpcal,mtphrb]{mtpro2}
\xroma{cyan}
%------TIKZ - ΣΧΗΜΑΤΑ - ΓΡΑΦΙΚΕΣ ΠΑΡΑΣΤΑΣΕΙΣ ----
\usepackage{tikz}
\usepackage{tkz-euclide}
\usetkzobj{all}
\usepackage[framemethod=TikZ]{mdframed}
\usetikzlibrary{decorations.pathreplacing}
\usepackage{pgfplots}
\usetkzobj{all}
%-----------------------
\usepackage{calc}
\usepackage{hhline}
\usepackage[explicit]{titlesec}
\usepackage{graphicx}
\usepackage{multicol}
\usepackage{multirow}
\usepackage{enumitem}
\usepackage{tabularx}
\usepackage[decimalsymbol=comma]{siunitx}
\usetikzlibrary{backgrounds}
\usepackage{sectsty}
\sectionfont{\centering}
\setlist[enumerate]{label=\bf{\large \arabic*.}}
\usepackage{adjustbox}
\usepackage{mathimatika,gensymb,eurosym,wrap-rl}
\usepackage{systeme,regexpatch}
%-------- ΜΑΘΗΜΑΤΙΚΑ ΕΡΓΑΛΕΙΑ ---------
\usepackage{mathtools}
%----------------------
%-------- ΠΙΝΑΚΕΣ ---------
\usepackage{booktabs}
%----------------------
%----- ΥΠΟΛΟΓΙΣΤΗΣ ----------
\usepackage{calculator}
%----------------------------
%------ ΔΙΑΓΩΝΙΟ ΣΕ ΠΙΝΑΚΑ -------
\usepackage{array}
\newcommand\diag[5]{%
\multicolumn{1}{|m{#2}|}{\hskip-\tabcolsep
$\vcenter{\begin{tikzpicture}[baseline=0,anchor=south west,outer sep=0]
\path[use as bounding box] (0,0) rectangle (#2+2\tabcolsep,\baselineskip);
\node[minimum width={#2+2\tabcolsep-\pgflinewidth},
minimum  height=\baselineskip+#3-\pgflinewidth] (box) {};
\draw[line cap=round] (box.north west) -- (box.south east);
\node[anchor=south west,align=left,inner sep=#1] at (box.south west) {#4};
\node[anchor=north east,align=right,inner sep=#1] at (box.north east) {#5};
\end{tikzpicture}}\rule{0pt}{.71\baselineskip+#3-\pgflinewidth}$\hskip-\tabcolsep}}
%---------------------------------
%---- ΟΡΙΖΟΝΤΙΟ - ΚΑΤΑΚΟΡΥΦΟ - ΠΛΑΓΙΟ ΑΓΚΙΣΤΡΟ ------
\newcommand{\orag}[3]{\node at (#1)
{$ \overcbrace{\rule{#2mm}{0mm}}^{{\scriptsize #3}} $};}
\newcommand{\kag}[3]{\node at (#1)
{$ \undercbrace{\rule{#2mm}{0mm}}_{{\scriptsize #3}} $};}
\newcommand{\Pag}[4]{\node[rotate=#1] at (#2)
{$ \overcbrace{\rule{#3mm}{0mm}}^{{\rotatebox{-#1}{\scriptsize$#4$}}}$};}
%-----------------------------------------


%------------------------------------------
\newcommand{\tss}[1]{\textsuperscript{#1}}
\newcommand{\tssL}[1]{\MakeLowercase{\textsuperscript{#1}}}
%---------- ΛΙΣΤΕΣ ----------------------
\newlist{bhma}{enumerate}{3}
\setlist[bhma]{label=\bf\textit{\arabic*\textsuperscript{o}\;Βήμα :},leftmargin=0cm,itemindent=1.8cm,ref=\bf{\arabic*\textsuperscript{o}\;Βήμα}}
\newlist{rlist}{enumerate}{3}
\setlist[rlist]{itemsep=0mm,label=\roman*.}
\newlist{brlist}{enumerate}{3}
\setlist[brlist]{itemsep=0mm,label=\bf\roman*.}
\newlist{tropos}{enumerate}{3}
\setlist[tropos]{label=\bf\textit{\arabic*\textsuperscript{oς}\;Τρόπος :},leftmargin=0cm,itemindent=2.3cm,ref=\bf{\arabic*\textsuperscript{oς}\;Τρόπος}}
% Αν μπει το bhma μεσα σε tropo τότε
%\begin{bhma}[leftmargin=.7cm]
\tkzSetUpPoint[size=7,fill=white]
\tikzstyle{pl}=[line width=0.3mm]
\tikzstyle{plm}=[line width=0.4mm]
\usepackage{etoolbox}
\makeatletter
\renewrobustcmd{\anw@true}{\let\ifanw@\iffalse}
\renewrobustcmd{\anw@false}{\let\ifanw@\iffalse}\anw@false
\newrobustcmd{\noanw@true}{\let\ifnoanw@\iffalse}
\newrobustcmd{\noanw@false}{\let\ifnoanw@\iffalse}\noanw@false
\renewrobustcmd{\anw@print}{\ifanw@\ifnoanw@\else\numer@lsign\fi\fi}
\makeatother

\DeclareMathSizes{10.95}{10.95}{7}{5}


\begin{document}
\fontdimen16\textfont2=2.5pt
\fontdimen17\textfont2=2.5pt
\fontdimen14\textfont2=4.5pt
\fontdimen13\textfont2=4.5pt 
\titlos{ΜΑΘΗΜΑΤΙΚΑ ΓΕΝΙΚΗΣ Γ΄ ΛΥΚΕΙΟΥ}{ΟΡΙΟ - ΣΥΝΕΧΕΙΑ}
\begin{thema}
\item \mbox{}\\\vspace{-5mm}
\begin{erwthma}
\item Έστω μια συνάρτηση $ f $ ορισμένη σε ένα σύνολο $ A $ και $ x_0 $ ένα σημείο του πεδίου ορισμού της. Να διατυπώσετε τον ορισμό της συνέχειας της συνάρτησης $ f $ στο σημείο $ x_0 $ καθώς και σε όλο το πεδίο ορισμού. \monades{10}
\item Να γράψετε τον ορισμό της γραφικής παράστασης μιας συνάρτησης $ f $ με πεδίο ορισμού ένα σύνολο $ A $.\monades{5}
\item \swstolathos
\begin{alist}
\item Αν {$ \displaystyle\lim_{x\to x_0}{f(x)}=\mathcal{l}_1 $ και $ \displaystyle\lim_{x\to x_0}{g(x)}=\mathcal{l}_2 $} τότε {$ \displaystyle\lim_{x\to x_0}{(f(x)+g(x))}=\mathcal{l}_1+\mathcal{l}_2 $}.
\item Το σημείο $ A(2,-1) $ ανήκει στη γραφική παράσταση της συνάρτησης $ f(x)=2x^2-7 $.
\item Η γραφική παράσταση της συνάρτησης $ f(x)=x^2+1 $ βρίσκεται ολόκληρη πάνω από τον οριζόντιο άξονα $ x'x $.
\item Αν {$ \displaystyle\lim_{x\to x_0}{f(x)}=\mathcal{l}_1 $} τότε {$ \displaystyle\lim_{x\to x_0}{\sqrt{f(x)}}=\sqrt{\mathcal{l}_1} $} για κάθε $ x\in D_f $.
\item Οι ρητές συναρτήσεις είναι συνεχείς στο πεδίο ορισμού τους.
\end{alist}\monades{10}
\end{erwthma}
\item \mbox{}\\\vspace{-5mm}
\begin{erwthma}
\item Έστω η συνάρτηση $ f $ με πεδίο ορισμού το σύνολο $ \mathbb{R} $ η οποία δίνεται από τον τύπο :
\[ f(x)=\ccases{\dfrac{\sqrt{x^2+3}-2}{x^2-x}&\ ,\  x\neq1\\
\qquad\ \ 2&\ ,\  x=1} \]
Να εξετάσετε αν η συνάρτηση είναι συνεχής στο $ x_0=1 $.\monades{10}
\item Έστω η συνάρτηση $ f $ με πεδίο ορισμού το σύνολο $ \mathbb{R}^* $ η οποία δίνεται από τον τύπο :
\[ f(x)=\ccases{\dfrac{x^2-3x+2}{x^2-2x}&\ ,\  x\neq2\\
\quad\ 3\lambda-2&\ ,\  x=2} \]
Να βρεθεί η τιμή της παραμέτρου $ \lambda $ ώστε η συνάρτηση να είναι συνεχής στο $ x_0=2 $.\monades{15}
\end{erwthma}
\item Δίνεται η συνεχής συνάρτηση $ f:A\to\mathbb{R} $ η οποία ικανοποιεί τη σχέση
\[ xf(x)+4=\sqrt{x^2+15}+f(x) \]
\begin{erwthma}
\item Να βρεθεί ο τύπος της συνάρτησης $ f $.\monades{10}
\item Να υπολογιστεί το όριο $ {\displaystyle\lim_{x\to 3}{\frac{f(x)-f(3)}{x-3}}} $.\monades{8}
\item Να βρεθούν οι τιμές του $ x $ για τις οποίες η γραφική παράσταση της συνάρτησης $ f $, τέμνει τον άξονα $ x'x $.\\\monades{7}
\end{erwthma}
\newpage
\item \mbox{}\\
Έστω μια συνεχής συνάρτηση $ f:A\to\mathbb{R} $ της οποίας η γραφική παράσταση διέρχεται από τα σημεία $ A(2,a+2\beta) $ και $ B(\beta,4) $. Επίσης δίνεται ότι 
\[ \lim_{x\to 2}{\frac{f(x)+2x}{2x-1}}=3 \ \ \textrm{και}\ \ xf(x)+9=x^2+3f(x) \]
\begin{erwthma}
\item Να βρεθούν οι τιμές των παραμέτρων $ a,\beta $.\monades{15}
\item Αφού βρεθεί ο τύπος της συνάρτησης $ f $ να υπολογίσετε το όριο $ {\displaystyle{\lim_{x\to 1}{\frac{\sqrt{f(x)}-2}{x^2-x}}}} $.\monades{10}
\end{erwthma}
\end{thema}
\end{document}

