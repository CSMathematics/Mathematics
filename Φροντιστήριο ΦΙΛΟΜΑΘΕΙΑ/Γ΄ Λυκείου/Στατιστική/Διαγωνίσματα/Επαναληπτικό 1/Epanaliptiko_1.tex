\documentclass[ektypwsh]{diag-pan-xelatex}
\usepackage[amsbb]{mtpro2}
\usepackage[no-math,cm-default]{fontspec}
\usepackage{xunicode}
\usepackage{xgreek}
\defaultfontfeatures{Mapping=tex-text,Scale=MatchLowercase}
\setmainfont[Mapping=tex-text,Numbers=Lining,Scale=1.0,BoldFont={Minion Pro Bold}]{Minion Pro}
\newfontfamily\scfont{GFS Artemisia}
\font\icon = "Webdings"
\usepackage{mtpro2}
\usepackage[left=2.00cm, right=2.00cm, top=2.00cm, bottom=3.00cm]{geometry}
\xroma{cyan}
\usepackage{tikz,amsmath}
\usepackage{tkz-euclide,tkz-fct}
\usepackage{wrapfig,mathimatika,wrap-rl}
\usetkzobj{all}
\usepackage{calc}
\usepackage{systeme,regexpatch}
\usepackage[framemethod=TikZ]{mdframed}
\usepackage{adjustbox}
\newlist{rlist}{enumerate}{3}
\setlist[rlist]{itemsep=0mm,label=\roman*.}
\tkzSetUpPoint[size=7,fill=white]

\usepackage{graphicx}
\usepackage{multicol}
\usepackage{multirow}
\usepackage{enumitem}
\usepackage{tabularx,gensymb}
\usepackage[decimalsymbol=comma]{siunitx}

\begin{document}
\titlos{ΜΑΘΗΜΑΤΙΚΑ ΚΑΤΕΥΘΥΝΣΗΣ Γ΄ ΕΠΑΛ}{ΕΠΑΝΑΛΗΠΤΙΚΟ}
\begin{thema}
\item \mbox{}\\\vspace{-5mm}
\begin{erwthma}
\item Αν η συνάρτηση $ f $ είναι παραγωγίσιμη στο $ \mathbb{R} $ και $ c $ πραγματική σταθερά, να
αποδείξετε ότι $ (c\cdot f(x))'=cf'(x) $ για κάθε $ x\in\mathbb{R} $.\monades{10}
\item Πότε μια συνάρτηση $ f $ λέγεται συνεχής σε ένα σημείο $ x_0\in D_f $ του πεδίου ορισμού της;\\\monades{5}
\item \swstolathos
\begin{rlist}
\item Η μέση τιμή $ \nu $ σε πλήθος παρατηρήσεων $ t_1,t_2,\ldots,t_\nu $ δίνεται από τον τύπο $ \bar{x}=\displaystyle\frac{1}{\nu}\sum_{i=1}^{\nu}{t_i} $.
\item Αν $ f,g $ είναι παραγωγίσιμες συναρτήσεις και ορίζεται η σύνθεση $ f(g(x)) $ τότε $ \left[f(g(x))\right]'=f'(g(x)) $.
\item Ένα δείγμα χαρακτηρίζεται ομοιογενές αν ο συντελεστής μεταβολής του είναι κάτω από $ 10\% $.
\item Το ιστόγραμμα συχνοτήτων σχεδιάζεται για ποιοτικές μεταβλητές.
\item Αν μια συνάρτηση $ f $ είναι γνησίως μονότονη σε όλο το πεδίο ορισμού της τότε δεν έχει ακρότατα.
\end{rlist}\monades{10}
\end{erwthma}
\item \mbox{}\\
Δίνεται η συνεχής συνάρτηση $ f:[4,+\infty)\rightarrow\mathbb{R} $ με τύπο
\[ f(x)=\ccases{\dfrac{x^3-4ax^2-16x+64a}{2x-8}&,x>4\\\qquad\qquad a-1&,x=4} \]
\begin{erwthma}
\item Να αποδείξετε ότι $ a=1 $.\monades{9}
\item Να βρεθεί η παράγωγος της συνάρτησης $ f $.\monades{9}
\item Να υπολογίσετε το όριο $ \displaystyle{\lim_{x\to-8}\dfrac{2f(x)+6f'(x)}{x+8}} $\monades{8}
\end{erwthma}
\item \mbox{}\\
Θεωρούμε ένα δείγμα $ \nu $ συνδρομητών μιας εταιρείας κινητής τηλεφωνίας. Για τον μήνα Μάιο, οι χρόνοι ομιλίας (σε ώρες) που έχουν χρεωθεί οι συνδρομητές του δείγματος έχουν χωριστεί σε πέντε κλάσεις ίσου πλάτους. Θεωρούμε ότι οι παρατηρήσεις κάθε κλάσης είναι ομοιόμορφα κατανεμημένες. Δίνεται ότι:
\begin{itemize}
\item Η μικρότερη διάρκεια χρόνου ομιλίας που παρατηρήθηκε στο δείγμα είναι μηδέν.
\item Το κέντρο της πέμπτης κλάσης είναι 18.
\item Στο κυκλικό διάγραμμα σχετικών συχνοτήτων, η γωνία του κυκλικού τομέα που αντιστοιχεί στην πέμπτη κλάση ισούται με $ 36\degree $.
\item $ \dfrac{N_1}{4}=\dfrac{N_2}{9}=\dfrac{N_3}{15}=\dfrac{N_4}{18} $ όπου $ N_1,N_2,N_3,N_4 $ είναι οι αθροιστικές συχνότητες της 1ης, 2ης, 3ης και 4ης κλάσης αντίστοιχα.
\end{itemize}
\begin{erwthma}
\item Να αποδείξετε ότι το πλάτος $ c $ της κάθε κλάσης είναι 4.\monades{4}
\item Να μεταφέρετε στο τετράδιό σας τον Πίνακα Ι συμπληρωμένο, αιτιολογώντας την απάντησή σας.
\begin{center}
\begin{tabular}{|c|c|c|}
\hline \rule[-2ex]{0pt}{5.5ex} Κλάσεις & Κεντρικές τιμές $ x_i $ & Σχετικές συχνότητες $ f_i\% $ \\ 
\hline \rule[-2ex]{0pt}{5.5ex} $ [\ \ ,\ ) $ &  &  \\ 
\hline \rule[-2ex]{0pt}{5.5ex} $ [\ \ ,\ ) $ &  &  \\ 
\hline \rule[-2ex]{0pt}{5.5ex} $ [\ \ ,\ ) $ &  &  \\ 
\hline \rule[-2ex]{0pt}{5.5ex} $ [\ \ ,\ ) $ &  &  \\ 
\hline \rule[-2ex]{0pt}{5.5ex} $ [\ \ ,\ ) $ &  &  \\ 
\hline \rule[-2ex]{0pt}{5.5ex} \textbf{Σύνολο} &  &  \\ 
\hline 
\end{tabular} 
\end{center}\monades{10}\mbox{}\\
Για τα ερωτήματα Γ3 και Γ4, δίνεται ότι $ f_1\%=20,f_2\%=25,f_3\%=30,f_4\%=15 $ και $ f_5\%=10 $.
\item Να βρείτε το ποσοστό των συνδρομητών του δείγματος οι οποίοι έχουν χρεωθεί τουλάχιστον 3 ώρες και λιγότερες από 10 ώρες ομιλίας.\monades{5}
\item Υποθέτουμε ότι οι συνδρομητές της εταιρείας δικαιούνται κάθε μήνα μέχρι 4 ώρες δωρεάν χρόνο ομιλίας. Έτσι, πληρώνουν μόνο για το χρόνο ομιλίας που τους έχει χρεωθεί επιπλέον των 4 ωρών. Αφαιρούμε από το δείγμα τους συνδρομητές που χρεώθηκαν λιγότερες από 4 ώρες. Να υπολογίσετε τη μέση τιμή του χρόνου (σε ώρες) που πλήρωσαν οι υπόλοιποι συνδρομητές του δείγματος τον μήνα Μάιο.\monades{6}
\end{erwthma}
\item \mbox{}\\
\wrapr{-5mm}{7}{4cm}{-7mm}{\begin{tikzpicture}[scale=.77]
\tkzDefPoint(0,4){A}
\tkzDefPoint(4,4){B}
\tkzDefPoint(4,0){C}
\tkzDefPoint(0,0){D}
\tkzDefPoint(1,4){K}
\tkzDefPoint(4,3){L}
\tkzDefPoint(3,0){M}
\tkzDefPoint(0,1){N}
\draw[pl] (A) rectangle (C);
\draw[pl] (K)--(L)--(M)--(N)--cycle;
\tkzLabelPoint[above left](A){$A$}
\tkzLabelPoint[above right](B){$B$}
\tkzLabelPoint[below right](C){$\varGamma$}
\tkzLabelPoint[below left](D){$\varDelta$}
\tkzLabelPoint[above](K){$K$}
\tkzLabelPoint[right](L){$\varLambda$}
\tkzLabelPoint[below](M){$M$}
\tkzLabelPoint[left](N){$N$}
\tkzDrawPoints(A,B,C,D,K,L,M,N)
\node at (0.5,4.25) {\footnotesize$x$};
\node at (4.25,3.5) {\footnotesize$x$};
\node at (3.5,-.25) {\footnotesize$x$};
\node at (-.25,0.5) {\footnotesize$x$};
\end{tikzpicture}}{
Δίνεται τετράγωνο $ AB\varGamma\varDelta $ πλευράς 4. Θεωρούμε τα
εσωτερικά σημεία $ K,\varLambda,M $ και $ N $ των πλευρών $ AB,B\varGamma,\varGamma\varDelta,\varDelta A $ αντίστοιχα, έτσι ώστε $ AK = B\varLambda = \varGamma M = \varDelta N = x $, όπως φαίνεται στο σχήμα.
\begin{erwthma}
\item Να αποδείξετε ότι το εμβαδόν του $ K\varLambda MN $, ως συνάρτηση του $ x $, είναι \[ E(x)=2(x^2-4x+8)\ ,\ x\in(0,4)\qquad\monades{7} \]
\end{erwthma}}\mbox{}\\\\
\begin{erwthma}[start=2]
\item Να βρείτε την τιμή του $ x $ για την οποία το εμβαδόν $ E(x) $ γίνεται ελάχιστο.\monades{5}
\item Θεωρούμε τις τιμές $ y_i=E(x_i)\ ,\ x_i\in(0,4)\ ,\ i=1,2,\ldots,19 $ έτσι ώστε :
\begin{itemize}
\item Τα $ x_i $ είναι διαφορετικά μεταξύ τους.
\item Η μέση τιμή των $ x_i $ και η διάμεσος τους είναι ίσες με $ 2 $.
\item Η μέση τιμή των $ y_i $ είναι ίση με $ 8{,}02 $.
\begin{rlist}
\item Να βρεθεί η μέση τιμή των $ x_i^2 $.\monades{6}
\item Να βρείτε την τυπική απόκλιση $ s_x $ και να εξετάσετε αν το δείγμα τους είναι ομοιογενές. Δίνεται ότι
\end{rlist}
\end{itemize}
\end{erwthma}
\end{thema}
\[ s^2=\frac{1}{\nu}\left[\sum_{i=1}^{\nu}{t_i^2}-\dfrac{\left( \displaystyle{\sum_{i=1}^{\nu}{t_i}}\right)^2 }{\nu}\right]  \]\monades{7}
\end{document}
