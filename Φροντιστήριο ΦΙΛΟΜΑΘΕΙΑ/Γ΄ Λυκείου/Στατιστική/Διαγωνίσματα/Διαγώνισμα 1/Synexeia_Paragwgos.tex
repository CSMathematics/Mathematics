\documentclass[ektypwsh]{diag-pan-xelatex}
\usepackage[amsbb]{mtpro2}
\usepackage[no-math,cm-default]{fontspec}
\usepackage{xunicode}
\usepackage{xltxtra}
\usepackage{xgreek}
\defaultfontfeatures{Mapping=tex-text,Scale=MatchLowercase}
\setmainfont[Mapping=tex-text,Numbers=Lining,Scale=1.0,BoldFont={Minion Pro Bold}]{Minion Pro}
\newfontfamily\scfont{GFS Artemisia}
\font\icon = "Webdings"
\usepackage{mtpro2}
\xroma{cyan}
\usepackage{tikz}
\usepackage{tkz-euclide,tkz-fct}
\usepackage{wrapfig}
\usetkzobj{all}
\usepackage{calc}
\usepackage{systeme,regexpatch}
\usepackage[framemethod=TikZ]{mdframed}
\usepackage{adjustbox}
\makeatletter
% change the definition of \sysdelim not to store `\left` and `\right`
\def\sysdelim#1#2{\def\SYS@delim@left{#1}\def\SYS@delim@right{#2}}
\sysdelim\{. % reinitialize

% patch the internal command to use
% \LEFTRIGHT<left delim><right delim>{<system>}
% instead of \left<left delim<system>\right<right delim>
\regexpatchcmd\SYS@systeme@iii
  {\cB.\c{SYS@delim@left}(.*)\c{SYS@delim@right}\cE.}
  {\c{SYS@MT@LEFTRIGHT}\cB\{\1\cE\}}
  {}{}
\def\SYS@MT@LEFTRIGHT{%
  \expandafter\expandafter\expandafter\LEFTRIGHT
  \expandafter\SYS@delim@left\SYS@delim@right}
\makeatother
\newcommand{\synt}[2]{{\scriptsize \begin{matrix}
\times#1\\\\ \times#2
\end{matrix}}}
\usepackage{amsmath}
\usepackage{graphicx}
\usepackage{multicol}
\usepackage{multirow}
\usepackage{enumitem}
\usepackage{tabularx}
\usepackage[decimalsymbol=comma]{siunitx}
%-------- ΤΡΙΓΩΝΟΜΕΤΡΙΚΟΙ ΑΡΙΘΜΟΙ -----------
\newcommand{\hm}[1]{\textrm{ημ}#1}
\newcommand{\syn}[1]{\textrm{συν}#1}
\newcommand{\ef}[1]{\textrm{εφ}#1}
\newcommand{\syf}[1]{\textrm{σφ}#1}
%--------------------------------------------
\newlist{rlist}{enumerate}{3}
\setlist[rlist]{itemsep=0mm,label=\roman*.}


\usepackage{etoolbox}
\makeatletter
\renewrobustcmd{\anw@true}{\let\ifanw@\iffalse}
\renewrobustcmd{\anw@false}{\let\ifanw@\iffalse}\anw@false
\newrobustcmd{\noanw@true}{\let\ifnoanw@\iffalse}
\newrobustcmd{\noanw@false}{\let\ifnoanw@\iffalse}\noanw@false
\renewrobustcmd{\anw@print}{\ifanw@\ifnoanw@\else\numer@lsign\fi\fi}
\makeatother

\begin{document}
\titlos{ΜΑΘΗΜΑΤΙΚΑ Γ΄ ΕΠΑΛ}{ΣΥΝΕΧΕΙΑ - ΠΑΡΑΓΩΓΟΣ}
\begin{thema}
\item \textbf{Θεωρία}\mbox{}\\\vspace{-5mm}
\begin{erwthma}
\item Έστω μια συνάρτηση $ f $ ορισμένη σε ένα σύνολο $ A $ και $ x_0 $ ένα σημείο του πεδίου ορισμού της. Να διατυπώσετε τον ορισμό της συνέχειας της συνάρτησης $ f $ στο σημείο $ x_0 $ καθώς και σε όλο το πεδίο ορισμού της. \monades{10}
\item Να γράψετε τον τύπο με τον οποίο δίνεται η παράγωγος μιας συνάρτησης $ f $ σε ένα σημείο $ x_0 $ του πεδίου ορισμού της. Πότε η $ f $ λέγεται παραγωγίσιμη στο σημείο αυτό;\monades{5}
\item \swstolathos
\begin{rlist}
\item Η παράγωγος μιας συνάρτησης $ f $ σε ένα σημείο $ x_0 $ του πεδίου ορισμού της είναι πάντα πραγματικός αριθμός.
\item Η παράγωγος της συνάρτησης $ f(x)=\hm{x} $ είναι η $ f'(x)=\syn{x} $.
\item Η συνάρτηση $ f(x)=\frac{x}{x+1} $ είναι συνεχής σε όλο το πεδίο ορισμού της.
\item Ο κανόνας ο οποίος μας δίνει την παράγωγο του γινομένου δύο συναρτήσεων είναι $ \left( f(x)\cdot g(x)\right)'=f'(x)\cdot g'(x) $.
\item Η παράγωγος της συνάρτησης $ f(x)=\syn{\varphi} $ είναι $ f'(x)=-\hm{\varphi} $.
\end{rlist}\monades{10}
\end{erwthma}
\item \textbf{Συνέχεια}
\begin{erwthma}
\item Έστω η συνάρτηση $ f $ με πεδίο ορισμού το σύνολο $ \mathbb{R} $ η οποία δίνεται από τον τύπο :
\[ f(x)=\ccases{\dfrac{\sqrt{x^2+3}-2}{x^2-x}&\ ,\  x\neq1\\
\qquad\ \ 2&\ ,\  x=1} \]
Να εξετάσετε αν η συνάρτηση είναι συνεχής στο $ x_0=1 $.\monades{10}
\item Έστω η συνάρτηση $ f $ με πεδίο ορισμού το σύνολο $ \mathbb{R} $ η οποία δίνεται από τον τύπο :
\[ f(x)=\ccases{\dfrac{x^2-3x+2}{x^2-2x}&\ ,\  x\neq2\\
\qquad\ \ 3\lambda-2&\ ,\  x=2} \]
Να βρεθεί η τιμή της παραμέτρου $ \lambda $ ώστε η συνάρτηση ωα είναι συνεχής στο $ x_0=2 $.\monades{15}
\end{erwthma}
\item \textbf{Παράγωγος}
\begin{erwthma}
\item Να υπολογίσετε τις παραγώγους των παρακάτω συναρτήσεων.
\begin{multicols}{2}
\begin{itemize}[itemsep=0mm]
\item $ f(x)=x^2\cdot\hm{x} $
\item $ f(x)=3x^3-5x^2+8x-\ef{\pi} $
\item $ f(x)=\hm{x}\cdot\syn{x} $
\item $ f(x)=x^2\cdot\hm{x}+x\cdot\syn{x} $
\end{itemize}
\end{multicols}\monades{10}
\newpage
\item Να υπολογίσετε τις παραγώγους των παρακάτω συναρτήσεων.
\begin{multicols}{2}
\begin{itemize}[itemsep=0mm]
\item $ f(x)=\dfrac{x-2}{x^2} $
\item $ f(x)=\dfrac{x^2-3x}{x^3-2x^2} $
\item $ f(x)=\dfrac{\hm{x}}{\ef{x}} $
\item $ f(x)=\dfrac{3x^2}{x-2}+\dfrac{x^4}{x^2-x} $
\end{itemize}
\end{multicols}\monades{15}
\end{erwthma}
\item \textbf{Σύνθετο θέμα}\\
Δίνεται μια συνάρτηση $ f $ με πεδίο ορισμού το $ \mathbb{R} $ η οποία δίνεται από τον παρακάτω τύπο
\[ f(x)=\ccases{\dfrac{x^2-ax+6}{x-3}&\ ,\  x\neq2\\
\quad\ \ \beta-4&\ ,\  x=2} \]
\begin{erwthma}
\item Αν γνωρίζουμε οτι η γραφική παράσταση της συνάρτησης $ f $ διέρχεται από το σημείο $ A(1,-1) $ τότε να βρεθεί η τιμή της παραμέτρου $ a $.\monades{8}
\item Για την τιμή της παραμέτρου $ a $ του προηγούμενου ερωτήματος, να βρεθεί η τιμή της παραμέτρου $ \beta $ ώστε η συνάρτηση $ f $ να είναι συνεχής στο $ x_0=2 $.\monades{9}
\item Να υπολογίσετε την παράγωγο $ f'(x) $ της συνάρτησης $ f $ για κάθε $ x\neq2 $.\monades{9}
\end{erwthma}
\end{thema}
\kaliepityxia
\end{document}
