\PassOptionsToPackage{no-math,cm-default}{fontspec}
\documentclass[twoside,nofonts,internet]{askhseis}
\usepackage{amsmath}
\usepackage{xgreek}
\let\hbar\relax
\defaultfontfeatures{Mapping=tex-text,Scale=MatchLowercase}
\setmainfont[Mapping=tex-text,Numbers=Lining,Scale=1.0,BoldFont={Minion Pro Bold}]{Minion Pro}
\newfontfamily\scfont{GFS Artemisia}
\xroma{red!80!black}
\usepackage{mtpro2}
%------TIKZ - ΣΧΗΜΑΤΑ - ΓΡΑΦΙΚΕΣ ΠΑΡΑΣΤΑΣΕΙΣ ----
\usepackage{tikz,pgfplots}
\usepackage{tkz-euclide}
\usetkzobj{all}
\usepackage[framemethod=TikZ]{mdframed}
\usetikzlibrary{decorations.pathreplacing}
\usepackage{pgfplots}
\usetkzobj{all}
%-----------------------
\usepackage{calc}
\usepackage{hhline}
\renewcommand{\thepart}{\arabic{part}}
\usepackage[explicit]{titlesec}
\usepackage{graphicx}
\usepackage{multicol}
\usepackage{multirow}
\usepackage{enumitem}
\usepackage{tabularx}
\usepackage[decimalsymbol=comma]{siunitx}
\usetikzlibrary{backgrounds}
\usepackage{sectsty}
\sectionfont{\centering}
\setlist[enumerate]{label=\bf{\large \arabic*.}}
\usepackage{adjustbox}
\usepackage{mathimatika,wrap-rl}
%-------- ΠΙΝΑΚΕΣ ---------
\usepackage{booktabs}
%----------------------
%----- ΥΠΟΛΟΓΙΣΤΗΣ ----------
\usepackage{calculator}
%----------------------------
%---- ΟΡΙΖΟΝΤΙΟ - ΚΑΤΑΚΟΡΥΦΟ - ΠΛΑΓΙΟ ΑΓΚΙΣΤΡΟ ------
\newcommand{\orag}[3]{\node at (#1)
{$ \overcbrace{\rule{#2mm}{0mm}}^{{\scriptsize #3}} $};}
\newcommand{\kag}[3]{\node at (#1)
{$ \undercbrace{\rule{#2mm}{0mm}}_{{\scriptsize #3}} $};}
\newcommand{\Pag}[4]{\node[rotate=#1] at (#2)
{$ \overcbrace{\rule{#3mm}{0mm}}^{{\rotatebox{-#1}{\scriptsize$#4$}}}$};}
%-----------------------------------------
\newcommand{\tss}[1]{\textsuperscript{#1}}
\newcommand{\tssL}[1]{\MakeLowercase{\textsuperscript{#1}}}
%---------- ΛΙΣΤΕΣ ----------------------
\newlist{bhma}{enumerate}{3}
\setlist[bhma]{label=\bf\textit{\arabic*\textsuperscript{o}\;Βήμα :},leftmargin=0cm,itemindent=1.8cm,ref=\bf{\arabic*\textsuperscript{o}\;Βήμα}}
\newlist{tropos}{enumerate}{3}
\setlist[tropos]{label=\bf\textit{\arabic*\textsuperscript{oς}\;Τρόπος :},leftmargin=0cm,itemindent=2.3cm,ref=\bf{\arabic*\textsuperscript{oς}\;Τρόπος}}
% Αν μπει το bhma μεσα σε tropo τότε
%\begin{bhma}[leftmargin=.7cm]
\tkzSetUpPoint[size=7,fill=white]
\tikzstyle{pl}=[line width=0.3mm]
\tikzstyle{plm}=[line width=0.4mm]








\begin{document}
\twocolkentro{
\titlos{Μαθηματικά και στοιχεία στατιστικής Γ Λυκείου}{Διαφορικός Λογισμός}{Παράγωγος Συνάρτηση}
\thewria}
\begin{enumerate}
\item 
\end{enumerate}
\twocolkentro{\askhseis}
\begin{enumerate}
\item \textbf{Παράγωγος απλών συναρτήσεων}\\
Να υπολογιστούν οι παράγωγοι των παρακάτω συναρτήσεων.
\begin{multicols}{2}
\begin{rlist}
\item $ f(x)=x+2 $
\item $ f(x)=x^3 $
\item $ f(x)=4x^3 $
\item $ f(x)=x^{-2} $
\end{rlist}
\end{multicols}
\item \textbf{Παράγωγος απλών συναρτήσεων}\\
Να υπολογιστούν οι παράγωγοι των παρακάτω συναρτήσεων.
\begin{multicols}{2}
\begin{rlist}[leftmargin=3mm]
\item $ f(x)= x+c $
\item $ f(y)= y^2-x^2 $
\item $ f(t)= t^2-\hm{\pi} $
\item $ f(z)= \hm{z}+\sqrt{x} $
\end{rlist}
\end{multicols}
\item \textbf{Παράγωγος απλών συναρτήσεων}\\
Να υπολογιστούν οι παράγωγοι των παρακάτω συναρτήσεων.
\begin{multicols}{2}
\begin{rlist}[leftmargin=3mm]
\item $ f(x)=x^2+3x $
\item $ f(x)=x^3-x $
\item $ f(x)=4x^2+5x-2 $
\item $ f(x)=2x-x^4 $
\end{rlist}
\end{multicols}
\item \textbf{Παράγωγος απλών συναρτήσεων}\\
Να υπολογιστούν οι παράγωγοι των παρακάτω συναρτήσεων.
\begin{multicols}{2}
\begin{rlist}[leftmargin=3mm]
\item $ f(x)=\sqrt{x}+x $
\item $ f(x)=\sqrt[3]{x}-x^2 $
\item $ f(y)=\sqrt[4]{y}+2\sqrt[3]{y} $
\item $ f(x)=3\sqrt{x}+2\sqrt{t} $
\end{rlist}
\end{multicols}
\item \textbf{Παράγωγος απλών συναρτήσεων}\\
Να υπολογιστούν οι παράγωγοι των παρακάτω συναρτήσεων.
\begin{multicols}{2}
\begin{rlist}[leftmargin=0mm]
\item $ f(x)=\hm{x}-4 $
\item $ f(x)=\syn{x}-\ef{x} $
\item $ f(x)= \syf{x}+\hm{x} $
\item $ f(x)=3\syn{x}-4\hm{x} $
\end{rlist}
\end{multicols}
\item \textbf{Παράγωγος γινομένου}\\
Να υπολογιστούν οι παράγωγοι των παρακάτω συναρτήσεων.
\begin{multicols}{2}
\begin{rlist}[leftmargin=3mm]
\item $ f(x)=x^2\cdot\sqrt{x} $
\item $ f(x)=3x^3\cdot\hm{x} $
\item $ f(x)= \sqrt{x}\cdot\syn{x} $
\item $ f(x)= x\cdot\sqrt[3]{x} $
\end{rlist}
\end{multicols}
\item \textbf{Παράγωγος γινομένου}\\
Να υπολογιστούν οι παράγωγοι των παρακάτω συναρτήσεων.
\begin{rlist}
\item $ f(x)=x^2\cdot\hm{x}+x\cdot\syn{x}$
\item $ f(x)=3x^4\cdot\sqrt{x}-x^3\cdot\sqrt[3]{x} $
\item $ f(x)=3x\cdot\ef{x}+x^2\cdot\syf{x} $
\item $ f(x)=\hm{x}\cdot\syn{x} $
\end{rlist}
\item \textbf{Παράγωγος πηλίκου}\\
Να υπολογιστούν οι παράγωγοι των παρακάτω συναρτήσεων.
\begin{multicols}{2}
\begin{rlist}
\item $ f(x)= \frac{x}{x+1} $
\item $ f(x)= \frac{x^2}{x^3-2} $
\item $ f(x)= \frac{3-x^3}{x} $
\item $ f(x)= \frac{x+\sqrt{x}}{x-2} $
\end{rlist}
\end{multicols}
\item \textbf{Παράγωγος πηλίκου}\\
Να υπολογιστούν οι παράγωγοι των παρακάτω συναρτήσεων.
\begin{multicols}{2}
\begin{rlist}
\item $ f(x)= \frac{\sqrt{x}}{x+1} $
\item $ f(x)= \frac{\sqrt{x}-2x}{x} $
\item $ f(x)= \frac{x}{\hm{x}} $
\item $ f(x)= \frac{x^2}{\syn{x}}$
\end{rlist}
\end{multicols}
\item \textbf{Παράγωγος πηλίκου}\\
Να υπολογιστούν οι παράγωγοι των παρακάτω συναρτήσεων.
\begin{multicols}{2}
\begin{rlist}
\item $ f(x)= \frac{\hm{x}}{x-1} $
\item $ f(x)=\ef{x} $
\item $ f(x)= \frac{\syn{x}}{x^2} $
\item $ f(x)= \frac{\syf{x}}{3-x} $
\end{rlist}
\end{multicols}
\item \textbf{Παράγωγος σύνθετων συναρτήσεων}\\
Να υπολογιστούν οι παράγωγοι των παρακάτω συναρτήσεων.
\begin{multicols}{2}
\begin{rlist}[leftmargin=2mm]
\item $ f(x)=(x+1)^2 $
\item $ f(x)=\left( 3x^2-4x\right) ^4 $
\item $ f(x)= \left( x-x^2\right) ^5 $
\item $ f(x)=\left( 2x^4+1\right) ^{-3} $
\end{rlist}
\end{multicols}
\item \textbf{Παράγωγος σύνθετων συναρτήσεων}\\
Να υπολογιστούν οι παράγωγοι των παρακάτω συναρτήσεων.
\begin{multicols}{2}
\begin{rlist}[leftmargin=5mm]
\item $ f(x)=\sqrt{x+2} $
\item $ f(x)= \sqrt{x^2-1} $
\item $ f(x)= \sqrt{4-x^3} $
\item $ f(x)= \sqrt{3x^2+x} $
\end{rlist}
\end{multicols}
\item \textbf{Παράγωγος σύνθετων συναρτήσεων}\\
Να υπολογιστούν οι παράγωγοι των παρακάτω συναρτήσεων.
\begin{multicols}{2}
\begin{rlist}[leftmargin=5mm]
\item $ f(x)= \hm{(x-3)} $
\item $ f(x)= \syn{\left(x^2\right)} $
\item $ f(x)= \ef{\left(x-x^3\right)} $
\item $ f(x)= \syf{\left(2x-x^2\right)} $
\end{rlist}
\end{multicols}
\end{enumerate}
\end{document}
