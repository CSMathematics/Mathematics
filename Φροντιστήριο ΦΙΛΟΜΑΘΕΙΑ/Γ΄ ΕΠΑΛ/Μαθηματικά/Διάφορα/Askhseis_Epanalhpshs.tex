\documentclass[11pt,a4paper]{article}
\usepackage[english,greek]{babel}
\usepackage[utf8]{inputenc}
\usepackage{nimbusserif}
\usepackage[T1]{fontenc}
\usepackage[left=2.00cm, right=2.00cm, top=3.00cm, bottom=2.00cm]{geometry}
\usepackage{amsmath}
\let\myBbbk\Bbbk
\let\Bbbk\relax
\usepackage[amsbb,subscriptcorrection,zswash,mtpcal,mtphrb,mtpfrak]{mtpro2}
\usepackage{graphicx,multicol,multirow,enumitem,tabularx,mathimatika,gensymb,venndiagram,hhline,longtable,tkz-euclide,fontawesome5,eurosym,tcolorbox}
\tcbuselibrary{skins,theorems,breakable}
\newlist{rlist}{enumerate}{3}
\setlist[rlist]{itemsep=0mm,label=\roman*.}
\newlist{alist}{enumerate}{3}
\setlist[alist]{itemsep=0mm,label=\alph*.}
\newlist{balist}{enumerate}{3}
\setlist[balist]{itemsep=0mm,label=\bf\alph*.}
\newlist{Alist}{enumerate}{3}
\setlist[Alist]{itemsep=0mm,label=\Alph*.}
\newlist{bAlist}{enumerate}{3}
\setlist[bAlist]{itemsep=0mm,label=\bf\Alph*.}
\renewcommand{\textstigma}{\textsigma\texttau}
\newlist{thema}{enumerate}{3}
\setlist[thema]{label=\bf\large{ΘΕΜΑ \textcolor{black}{\Alph*}},itemsep=0mm,leftmargin=0cm,itemindent=18mm}
\newlist{erwthma}{enumerate}{3}
\setlist[erwthma]{label=\bf{\large{\textcolor{black}{\Alph{themai}.\arabic*}}},itemsep=0mm,leftmargin=0.8cm}

\newcommand{\lysh}{\textcolor{black}{\textbf{\faCheck\ \ ΛΥΣΗ}}}
\renewcommand{\textstigma}{\textsigma\texttau}
%----------- ΟΡΙΣΜΟΣ------------------
\newcounter{orismos}[section]
\renewcommand{\theorismos}{\thesection.\arabic{orismos}}   
\newcommand{\Orismos}{\refstepcounter{orismos}{\textbf{\textcolor{black}{\kerkissans{Ορισμός\hspace{2mm}\theorismos}}\;:\;}{}}}

\newenvironment{orismos}[1]
{\begin{tcolorbox}[title=\Orismos {\textcolor{black}{\kerkissans{#1}}},breakable,bottomtitle=-1.5mm,
enhanced standard,titlerule=-.2pt,toprule=0pt, rightrule=0pt, bottomrule=0pt,
colback=white,left=2mm,top=1mm,bottom=0mm,
boxrule=0pt,
colframe=white,borderline west={1.5mm}{0pt}{black},leftrule=2mm,sharp corners,coltitle=black]}
{\end{tcolorbox}}

\newcommand{\kerkissans}[1]{{\fontfamily{maksf}\selectfont \textbf{#1}}}
\renewcommand{\textdexiakeraia}{}

\usepackage[
backend=biber,
style=alphabetic,
sorting=ynt
]{biblatex}
\usepackage[explicit]{titlesec}
\titleformat{\section}[hang]
{\Large\fontfamily{maksf}\selectfont}%
{\colorbox{red!80!black}{%
\raisebox{0pt}[13pt][3pt]{\makebox[80pt]{% height, width
\color{white}{\kerkissans{\textbf{\thesection ο Κεφάλαιο}}}}%
}}}%
{0pt}%
{\colorbox{black}{\raisebox{0pt}[13pt][3pt]{\color{white}\ \textbf{#1}\ }}}

\titleformat{\subsection}[hang]
{\large\bfseries\fontfamily{maksf}\selectfont}%
{\colorbox{red!80!black}{%
\raisebox{0pt}[13pt][3pt]{\makebox[30pt]{% height, width
\color{white}{\kerkissans{\textbf{\thesubsection}}}}%
}}}%
{0pt}%
{\colorbox{black}{\raisebox{0pt}[13pt][3pt]{\color{white}\ \textbf{#1}\ }}}

%------- ΣΤΥΛ ΠΑΡΑΔΕΙΓΜΑΤΟΣ -------
\newcounter{askhsh}[section]
\renewcommand{\theaskhsh}{\kerkissans{\arabic{askhsh}}}   
\newcommand{\Askhsh}[1]{\refstepcounter{askhsh}\kerkissans{\bmath{\textcolor{red!80!black}{\faPen\large \ \ Άσκηση\hspace{2mm}\thesection.\theaskhsh\;:\;}\hspace{1mm}  #1}}\\}{}
%-----------------------------------
\setlength{\parindent}{0pt}
\newcommand{\eng}{\selectlanguage{english}}

\begin{document}
\begin{center}
\kerkissans{\textbf{{\huge Ασκήσεις για επανάληψη}}}
\end{center}
\section{Διαφορικός Λογισμός}
\subsection{Συναρτήσεις}
\Askhsh{Πεδίο ορισμού - Ρητές συναρτήσεις}
Να βρεθεί το πεδίο ορισμού σε καθεμία από τις ακόλουθες συναρτήσεις.
\begin{multicols}{3}
\begin{alist}
\item $f(x)=\dfrac{x^2+1}{x}$
\item $f(x)=\dfrac{x}{x-1}$
\item $f(x)=\dfrac{2x}{x^2-4}$
\item $f(x)=\dfrac{1}{|x|-2}$
\item $f(x)=\dfrac{x^2+1}{x^2+4x-12}$
\item $f(x)=\dfrac{3-2x}{x^3-1}$
\end{alist}
\end{multicols}
\Askhsh{Πεδίο ορισμού - Άρρητες συναρτήσεις}
Να βρεθεί το πεδίο ορισμού σε καθεμία από τις ακόλουθες συναρτήσεις.
\begin{multicols}{3}
\begin{alist}
\item $f(x)=\sqrt{2x-4}$
\item $f(x)=\sqrt{9-3x}$
\item $f(x)=\sqrt{x^2-16}$
\item $f(x)=\sqrt{|x+1|-3}$
\item $f(x)=\sqrt{x^2+x-6}$
\item $f(x)=\sqrt{3-|1-2x|}$
\end{alist}
\end{multicols}
\Askhsh{Πεδίο ορισμού - Ρίζα στον παρονομαστή}
Να βρεθεί το πεδίο ορισμού σε καθεμία από τις ακόλουθες συναρτήσεις.
\begin{multicols}{3}
\begin{alist}
\item $f(x)=\dfrac{1}{\sqrt{x-1}}$
\item $f(x)=\dfrac{x-2}{\sqrt{x^2-9}}$
\item $f(x)=\dfrac{\hm{x}}{\sqrt{|x+2|-5}}$
\item $f(x)=\dfrac{2}{\sqrt{x-1}-3}$
\item $f(x)=\dfrac{\syn{x}}{\sqrt{1-x^2}+2}$
\item $f(x)=\dfrac{4}{\sqrt{x^2+9}-25}$
\end{alist}
\end{multicols}
\subsection{Όρια μορφής \bmath{$\frac{0}{0}$}}
\Askhsh{Ρητές συναρτήσεις}
Να υπολογίσετε τα παρακάτω όρια
\begin{multicols}{3}
\begin{alist}
\item $\lim\limits_{x\to 1}{\dfrac{x-1}{x^2-1}}$
\item $\lim\limits_{x\to 2}{\dfrac{x^2-4}{x^2-2x}}$
\item $\lim\limits_{x\to 3}{\dfrac{9-x^2}{x-3}}$
\item $\lim\limits_{x\to -1}{\dfrac{x^2+x}{x^2-x-2}}$
\item $\lim\limits_{x\to 4}{\dfrac{x^3-4x^2}{x^2-16}}$
\item $\lim\limits_{x\to -2}{\dfrac{x^2-3x-10}{x^2+6x-8}}$
\item $\lim\limits_{x\to 1}{\dfrac{x^3-1}{x^2-1}}$
\item $\lim\limits_{x\to 0}{\dfrac{x^3-4x^2+3x}{x^4-5x}}$
\item $\lim\limits_{x\to \frac{1}{2}}{\dfrac{2x-1}{4x^2-1}}$
\item $\lim\limits_{x\to 2}{\dfrac{x^2-4x+4}{x^3-8}}$
\item $\lim\limits_{x\to 1}{\dfrac{x^3-1}{x^3-7x+6}}$
\item $\lim\limits_{x\to -3}{\dfrac{x^3+6x^2+9x}{x^3+8x^2+21x+18}}$
\end{alist}
\end{multicols}
\Askhsh{Άρρητες συναρτήσεις}
Να υπολογίσετε τα παρακάτω όρια
\begin{multicols}{3}
\begin{alist}
\item $\lim\limits_{x\to 1}{\dfrac{x^2-1}{\sqrt{x}-1}}$
\item $\lim\limits_{x\to 3}{\dfrac{\sqrt{x+1}-2}{x^2-3x}}$
\item $\lim\limits_{x\to 2}{\dfrac{4-x^2}{\sqrt{5x-1}-3}}$
\item $\lim\limits_{x\to 1}{\dfrac{2-\sqrt{x+3}}{x^2-1}}$
\item $\lim\limits_{x\to 2}{\dfrac{\sqrt{3-x}-1}{\sqrt{x+2}-2}}$
\item $\lim\limits_{x\to 4}{\dfrac{\sqrt{x+1}-\sqrt{5}}{x-4}}$
\item $\lim\limits_{x\to -2}{\dfrac{x^2+2x}{\sqrt{x+3}-\sqrt{3x+7}}}$
\item $\lim\limits_{x\to 2}{\dfrac{\sqrt{x^3}-\sqrt{8}}{\sqrt{x}-2}}$
\item $\lim\limits_{x\to 3}{\dfrac{\sqrt{x+1}-2}{\sqrt{x-1}-\sqrt{5-x}}}$
\end{alist}
\end{multicols}
\subsection{Μονοτονία - Ακρότατα}
\Askhsh{Μονοτονία - Ακρότατα}
Να μελετήσετε τις παρακάτω συναρτήσεις ως προς τη μονοτονία και τα ακρότατα.
\begin{multicols}{2}
\begin{alist}
\item $f(x)=x^2+8x-2$
\item $f(x)=x^3-3x^2-9x+4$
\item $f(x)=\dfrac{x^3}{3}+\dfrac{5x^2}{2}-6x+1$
\item $f(x)=\dfrac{x^2}{x-1}$
\item $f(x)=\sqrt{x^2-8x+7}$
\item $f(x)=\sqrt{9-x^2}$
\item $f(x)=\dfrac{x}{x^2-4}$
\item $f(x)=x\cdot\hm{x}\ ,\ x\in[-\pi,\pi]$
\end{alist}
\end{multicols}
\subsection{Εξίσωση εφαπτομένης}
\Askhsh{Εφαπτομένη σε γνωστό σημείο}
Για καθεμία από τις παρακάτω συναρτήσεις, να βρεθεί η εξίσωση της εφαπτόμενης ευθείας, στο δοσμένο σημείο $M(x_0,f(x_0))$.
\begin{multicols}{2}
\begin{alist}
\item $f(x)=x^2-4x\ ,\ M(1,f(1))$
\item $f(x)=\dfrac{1}{x}\ ,\ M(2,f(2))$
\item $f(x)=\hm{x}\ ,\ M\left(\frac{\pi}{3},f\left(\frac{\pi}{3}\right)\right)$
\item $f(x)=\sqrt{x}\ ,\ M(4,f(4))$
\item $f(x)=\dfrac{x+1}{x-2}\ ,\ M(3,f(3))$
\item $f(x)=\syn{2x}\ ,\ M\left(\frac{\pi}{4},f\left(\frac{\pi}{4}\right)\right)$
\item $f(x)=\dfrac{1}{x^2+1}\ ,\ M(-1,f(-1))$
\item $f(x)=\sqrt{x^2-2x+4}\ ,\ M(2,f(2))$
\end{alist}
\end{multicols}
\Askhsh{Εφαπτομένη με γνωστή κλίση}
Δίνεται η συνάρτηση $f(x)=-x^2+3x+4$. Να βρεθεί η εξίσωση της εφαπτομένης της $C_f$ η οποία:
\begin{alist}
\item έχει συντελεστή διεύθυνσης $\lambda=-1$.
\item είναι παράλληλη με την ευθεία $\zeta: y=5x+1$.
\item είναι κάθετη στην ευθεία $\eta: y=\dfrac{1}{3}x+2$.
\item σχηματίζει γωνία $\omega=45\degree$ με τον άξονα $x'x$.
\item είναι παράλληλη με τον άξονα $x'x$.
\end{alist}
\Askhsh{Εφαπτομένη με γνωστή κλίση}
Δίνεται η συνάρτηση $f(x)=x^3+ax-2$ με $a\in\mathbb{R}$. Αν η εφαπτομένη της $C_f$ στο σημείο $M(1,f(1))$ είναι παράλληλη με την ευθεία $\zeta:y=2x+1$
\begin{alist}
\item να δείξετε ότι $a=-1$
\item να βρείτε όλες τις εφαπτομένες της $C_f$ που είναι παράλληλες με την $\zeta$.
\end{alist}
\section{Στατιστική}
\subsection{Μεταβολές των παρατηρήσεων - Κανονική κατανομή}
\Askhsh{Μεταβολές των παρατηρήσεων}
Οι τιμές μιας συγκεκριμένου τύπου τηλεόρασης σε $10$ καταστήματα ηλεκτρονικών ειδών είναι οι ακόλουθες:
\[ 565,567,568,574,575,575,589,589,599,619 \]
(\textit{Πηγή: \eng{skroutz.gr} 15/5/2024}).
\begin{alist}
\item Να υπολογίσετε τη μέση τιμή και την τυπική απόκλιση του δείγματος.
\item Αν για την αγορά αυτής της τηλεόρασης χρησιμοποιήσουμε εκπτωτικό κουπόνι αξίας 35\euro, να υπολογίσετε τη μέση τιμή και τυπική απόκλιση του νέου δείγματος των μειωμένων τιμών.
\item Αν λόγω ζήτησης, οι τιμές του προϊόντος αυτού αυξηθούν κατά 15\%, εξετάστε αν το νέο δείγμα είναι ή όχι ομοιογενές.
\end{alist}
\Askhsh{Μεταβολές των παρατηρήσεων - Εξίσωση ευθείας}
Δίνεται η ευθεία $\varepsilon : y=2x-3$ και τα σημεία $A_1(x_1,y_1),A_2(x_2,y_2),\ldots,A_{10}(x_{10},y_{10})$ τα οποία ανήκουν στην ευθεία αυτή. Αν το δείγμα των τετμημένων $x_i,\ i=1,2,\ldots,10$ των σημείων, έχει μέση τιμή $\bar{x}=10$ και τυπική απόκλιση $s_x=2$, να υπολογίσετε τη μέση τιμή και τυπική απόκλιση του δείγματος των τεταγμένων $y_i$ των σημείων και να εξετάσετε το δείγμα αυτό ως προς την ομοιογένεια.
\end{document}
