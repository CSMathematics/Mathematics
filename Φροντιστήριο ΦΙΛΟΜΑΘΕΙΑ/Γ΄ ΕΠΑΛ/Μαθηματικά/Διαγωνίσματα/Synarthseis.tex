\documentclass[twoside,nofonts,ektypwsh]{frontisthrio-diag}
\usepackage[amsbb,subscriptcorrection,zswash,mtpcal,mtphrb,mtpfrak]{mtpro2}
\usepackage[no-math,cm-default]{fontspec}
\usepackage{amsmath}
\usepackage{xunicode}
\usepackage{xgreek}
\let\hbar\relax
\defaultfontfeatures{Mapping=tex-text,Scale=MatchLowercase}
\setmainfont[Mapping=tex-text,Numbers=Lining,Scale=1.0,BoldFont={Minion Pro Bold}]{Minion Pro}
\newfontfamily\scfont{GFS Artemisia}
\font\icon = "Webdings"
\usepackage{fontawesome5}
\newfontfamily{\FA}{fontawesome.otf}
\xroma{cyan!70!black}
%------TIKZ - ΣΧΗΜΑΤΑ - ΓΡΑΦΙΚΕΣ ΠΑΡΑΣΤΑΣΕΙΣ ----
\usepackage{tikz,pgfplots}
\usepackage{tkz-euclide}
\usetkzobj{all}
\usepackage[framemethod=TikZ]{mdframed}
\usetikzlibrary{decorations.pathreplacing}
\tkzSetUpPoint[size=7,fill=white]
%-----------------------
\usepackage{calc,tcolorbox}
\tcbuselibrary{skins,theorems,breakable}
\usepackage{hhline}
\usepackage[explicit]{titlesec}
\usepackage{graphicx}
\usepackage{multicol}
\usepackage{multirow}
\usepackage{tabularx}
\usetikzlibrary{backgrounds}
\usepackage{sectsty}
\sectionfont{\centering}
\usepackage{enumitem}
\usepackage{adjustbox}
\usepackage{mathimatika,gensymb,eurosym,wrap-rl}
\usepackage{systeme,regexpatch}
%-------- ΜΑΘΗΜΑΤΙΚΑ ΕΡΓΑΛΕΙΑ ---------
\usepackage{mathtools}
%----------------------
%-------- ΠΙΝΑΚΕΣ ---------
\usepackage{booktabs}
%----------------------
%----- ΥΠΟΛΟΓΙΣΤΗΣ ----------
\usepackage{calculator}
%----------------------------
%------------------------------------------
\newcommand{\tss}[1]{\textsuperscript{#1}}
\newcommand{\tssL}[1]{\MakeLowercase{\textsuperscript{#1}}}
%---------- ΛΙΣΤΕΣ ----------------------
\newlist{bhma}{enumerate}{3}
\setlist[bhma]{label=\bf\textit{\arabic*\textsuperscript{o}\;Βήμα :},leftmargin=0cm,itemindent=1.8cm,ref=\bf{\arabic*\textsuperscript{o}\;Βήμα}}
\newlist{tropos}{enumerate}{3}
\setlist[tropos]{label=\bf\textit{\arabic*\textsuperscript{oς}\;Τρόπος :},leftmargin=0cm,itemindent=2.3cm,ref=\bf{\arabic*\textsuperscript{oς}\;Τρόπος}}
% Αν μπει το bhma μεσα σε tropo τότε
%\begin{bhma}[leftmargin=.7cm]
\tkzSetUpPoint[size=7,fill=white]
\tikzstyle{pl}=[line width=0.3mm]
\tikzstyle{plm}=[line width=0.4mm]
\usepackage{etoolbox}
\makeatletter
\renewrobustcmd{\anw@true}{\let\ifanw@\iffalse}
\renewrobustcmd{\anw@false}{\let\ifanw@\iffalse}\anw@false
\newrobustcmd{\noanw@true}{\let\ifnoanw@\iffalse}
\newrobustcmd{\noanw@false}{\let\ifnoanw@\iffalse}\noanw@false
\renewrobustcmd{\anw@print}{\ifanw@\ifnoanw@\else\numer@lsign\fi\fi}
\makeatother

\usepackage{path}
\pathALa

\begin{document}
\titlos{Γ΄ ΕΠΑΛ - Μαθηματικά}{Συναρτήσεις}{Β}
\begin{thema}
\item\mbox{}\\\vspace{-7mm}
\begin{erwthma}
\item Να δώσετε τον ορισμό της πραγματικής συνάρτησης πραγματικής μεταβλητής.\monades{5}
\item Να δώσετε τον ορισμό της γραφικής παράστασης μιας συνάρτησης $ f $.\monades{5}
\item Πότε μια συνάρτηση $ f $ λέγεται συνεχής σε ένα σημείο $ x_0 $ του πεδίου ορισμού της;\monades{5}
\item \swstolathos
\begin{alist}
\item Από τη σχέση $ f(x)\leq 3 $ προκύπτει ότι το $ 3 $ είναι το μέγιστο της συνάρτησης $ f $.
\item Η συνάρτηση $ f(x)=\frac{x-2}{x-3} $ είναι συνεχής σε όλο το πεδίο ορισμού της.
\item Η γραφική παράσταση της συνάρτησης $ f(x)=\sqrt{x-1} $ τέμνει τον άξονα $ y'y $.
\item Για μια γνησίως αύξουσα συνάρτηση $ f $ ισχύει $ f(2)<f(3) $.
\item Για να βρούμε τα σημεία τομής της γραφικής παράστασης μιας συνάρτησης $ f $ με τον άξονα $ x'x $ λύνουμε την εξίσωση $ f(x)=0 $.
\end{alist}\monades{10}
\end{erwthma}
\item\mbox{}\\
Δίνεται η συνάρτηση $ f $ με τύπο
\[ f(x)=\frac{\sqrt{x+3}-2}{x^2+x-2} \]
\begin{erwthma}
\item Να βρεθεί το πεδίο ορισμού της συνάρτησης $ f $.\monades{8}
\item Να βρεθούν τα σημεία τομής της $ C_f $ με τους άξονες $ x'x $ και $ y'y $.\monades{8}
\item Να βρεθεί το $ \lim\limits_{x\to 2}f(x) $.\monades{9}
\end{erwthma}
\item\mbox{}\\
Δίνεται η ακόλουθη συνάρτηση $ f:\mathbb{R}\to\mathbb{R} $ με τύπο
\[ f(x)=\ccases{\frac{x^3-x^2-5x+6}{x-2}& , x\neq 2\\
\lambda-2 & , x=2} \]
η οποία είναι συνεχής στο πεδίο ορισμού της.
\begin{erwthma}
\item Να δείξετε ότι $ \lambda=5 $.\monades{10}
\item Να υπολογίσετε τις τιμές $ f(2),f(-1),f(f(3)) $.\monades{5}
\item Να βρείτε τα σημεία τομής της $ C_f $ με την ευθεία $ y=3 $.\monades{5}
\end{erwthma}
\item\mbox{}\\
Δίνεται η συνάρτηση $ f $ η οποία έχει τύπο
\[ f(x)=\ccases{\frac{x^2-ax+\beta}{x-1}& , x\neq 1\\-2 & ,x=1} \]
για την οποία γνωρίζουμε ότι το σημείο $ A(4,1) $ ανήκει στη γραφική της παράσταση ενώ αυτή τέμνει τον άξονα $ y'y $ στο $ -3 $.
\begin{erwthma}
\item Να δείξετε ότι $ a=4 $ και $ \beta=3 $.\monades{9}
\item Να εξετάσετε αν η συνάρτηση $ f $ είναι συνεχής.\monades{8}
\item Να υπολογίσετε το όριο $ \lim\limits_{x\to 2}{\frac{f^2(x)+f(x)-2}{f(x)+1}} $.\monades{8}
\end{erwthma}
\end{thema}
\diarkeia{3}
\kaliepityxia
\end{document}
