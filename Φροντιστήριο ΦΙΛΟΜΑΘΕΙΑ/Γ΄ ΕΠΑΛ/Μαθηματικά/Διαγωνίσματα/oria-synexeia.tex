\documentclass[ektypwsh]{frontisthrio-diag}
\usepackage[T1]{fontenc}
\usepackage[english,greek]{babel}
\usepackage{amsmath}
\usepackage{libertinus} % txfontsb,libertinus,libertine,kerkis,nimbusserif
\let\Bbbk\relax
\usepackage[amsbb,subscriptcorrection,zswash,mtpcal,mtphrb,mtpfrak]{mtpro2}
\newcommand{\kerkissans}[1]{{\fontfamily{maksf}\selectfont #1}}
\usetikzlibrary{decorations.pathreplacing,backgrounds}
\tkzSetUpPoint[size=2.8,fill=white]
%-----------------------
\usepackage[explicit]{titlesec}
\usepackage{sectsty}
\sectionfont{\centering}
\usepackage{graphicx,tabularx,fontawesome5,pgfplots,calc,tcolorbox,hhline,adjustbox,mathimatika,gensymb,eurosym,wrap-rl,systeme,regexpatch,mathtools,booktabs,calculator}
\let\vary\relax
\usepackage{exsheets}
%----------------------------
\tcbuselibrary{skins,theorems,breakable}
\tikzstyle{pl}=[line width=0.3mm]
\tikzstyle{plm}=[line width=0.4mm]
\usepackage{etoolbox}
\makeatletter
\renewrobustcmd{\anw@true}{\let\ifanw@\iffalse}
\renewrobustcmd{\anw@false}{\let\ifanw@\iffalse}\anw@false
\newrobustcmd{\noanw@true}{\let\ifnoanw@\iffalse}
\newrobustcmd{\noanw@false}{\let\ifnoanw@\iffalse}\noanw@false
\renewrobustcmd{\anw@print}{\ifanw@\ifnoanw@\else\numer@lsign\fi\fi}
\makeatother

\def\TyposDiagvnismatos{A}

\begin{document}
\titlos{Γ ΕΠΑΛ}{ΟΡΙΑ - ΣΥΝΕΧΕΙΑ}
\begin{thema}
\item\mbox{}\\\vspace{-7mm}
\begin{erwthma}
\item Να δώσετε τον ορισμό της συνεχούς συνάρτησης σε ένα σημείο $ x_0 $ του πεδίου ορισμού της.
\item Πότε μια συνάρτηση $ f $ ονομάζεται συνεχής στο πεδίο ορισμού της;
\item Να αναφέρετε τα είδη των συναρτήσεων οι οποίες είναι συνεχείς στο πεδίο ορισμού τους.
\item \swstolathospan\\
Δίνονται οι συναρτήσεις $ f,g $ με πεδίο ορισμού ένα σύνολο $ A $ και $ x_0\in A $ ένα κοινό σημείο του πεδίου ορισμού τους. Τότε ισχύουν τα παρακάτω.
\begin{alist}
\item $ \lim\limits_{x\to x_0}{(f(x)+g(x))}=\lim\limits_{x\to x_0}{f(x)}+\lim\limits_{x\to x_0}{g(x)} $
\item $ \lim\limits_{x\to x_0}{(f(x)\cdot g(x))}=\lim\limits_{x\to x_0}{f(x)}\cdot\lim\limits_{x\to x_0}{g(x)} $
\item $ \lim\limits_{x\to x_0}{\dfrac{f(x)}{g(x)}}=\dfrac{\lim\limits_{x\to x_0}{f(x)}}{\lim\limits_{x\to x_0}{g(x)}} $ για κάθε $ x\in A $
\item $ \lim\limits_{x\to x_0}{(f(x))^{\nu}}=\lim\limits_{x\to x_0}{f(x)} $
\item $ \lim\limits_{x\to x_0}{\sqrt{f(x)}}=\sqrt{\lim\limits_{x\to x_0}{f(x)}} $ για κάθε $ x\in A $.
\end{alist}
\end{erwthma}
\item Δίνεται η συνεχής συνάρτηση $ f:\mathbb{R}\to \mathbb{R} $ για την οποία ισχύει η σχέση
\[ xf(x)+9=3f(x)+x^2 \]
για κάθε $ x\in\mathbb{R} $.
\begin{erwthma}
\item Να αποδείξετε ότι $ f(x)=\dfrac{x^2-9}{x-3} $ για κάθε $ x\neq 3 $.
\item Να υπολογίσετε το όριο $ \lim\limits_{x\to 3}\dfrac{x^2-9}{x-3} $.
\item Να υπολογίσετε το $ f(3) $.
\end{erwthma}
\item Δίνεται η συνάρτηση $ f(x)=x^2+ax+\beta $ με $ a,\beta\in\mathbb{R} $, της οποίας η γραφική παράσταση διέρχεται από τα σημεία $ M(-1,-9) $ και $ N(-5,7) $.
\begin{erwthma}
\item Να δείξετε ότι $ a=2 $ και $ \beta=-8 $.
\item Να βρείτε να διαστήματα στα οποία η $ C_f $ βρίσκεται κάτω από τον άξονα $ x'x $.
\item Να υπολογίσετε το όριο
\[ \lim_{x\to 2}\frac{f(x)}{\sqrt{x-1}-1} \]
\item Να υπολογίσετε το όριο
\[ \lim_{x\to 1}\frac{f(x)-f(1)}{x-1} \]
\end{erwthma}
\item Δίνεται η συνεχής συνάρτηση
\[ f(x)=\begin{cdcases}
\frac{x^2+6x-7}{x-1} & ,x\neq 1\\\lambda & ,x=1
\end{cdcases} \]
όπου $ \lambda\in\mathbb{R} $. 
\begin{erwthma}
\item Να βρείτε το σημείο τομής της $ C_f $ με τον άξονα $ y'y $.
\item Να δείξετε ότι $ \lambda=8 $.
\item Να υπολογίσετε το όριο
\[ \lim_{x\to 1}\frac{f(x)-f(1)}{\sqrt{x^2+3}-2} \]
\end{erwthma}
\end{thema}
\end{document}
