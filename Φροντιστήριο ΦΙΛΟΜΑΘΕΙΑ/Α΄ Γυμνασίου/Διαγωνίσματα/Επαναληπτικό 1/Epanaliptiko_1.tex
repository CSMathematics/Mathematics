\documentclass[ektypwsh]{diag-xelatex}
\usepackage[amsbb]{mtpro2}
\usepackage[no-math,cm-default]{fontspec}
\usepackage{xunicode}
\usepackage{xgreek}
\usepackage{amsmath,mathimatika,gensymb,eurosym}
\defaultfontfeatures{Mapping=tex-text,Scale=MatchLowercase}
\setmainfont[Mapping=tex-text,Numbers=Lining,Scale=1.0,BoldFont={Minion Pro Bold}]{Minion Pro}
\newfontfamily\scfont{GFS Artemisia}
\font\icon = "Webdings"
\usepackage[amsbb]{mtpro2}
\xroma{red!80!black}
\newcommand{\tss}[1]{\textsuperscript{#1}}
\newcommand{\tssL}[1]{\MakeLowercase\textsuperscript{#1}}
\newlist{rlist}{enumerate}{3}
\setlist[rlist]{itemsep=0mm,label=\roman*.}
\usetkzobj{all}
\tkzSetUpPoint[size=7,fill=white]


\begin{document}
\titlos{Α΄ Γυμνασίου}{Επαναληπτικό}
\thewria
\begin{thema}
\item \mbox{}\\
\textbf{Α.} Να απαντήσετε στις παρακάτω ερωτήσεις.
\begin{rlist}
\item Ποιές γωνίες ονομάζονται εφεξής;
\item Ποιά είναι η ισότητα που συνδέει δύο ανάλογα ποσά $ x,y $;
\item Να αναφέρετε τα είδη των γωνιών.
\end{rlist}\monades{3}\mbox{}\\
\textbf{Β.} \swstolathos
\begin{rlist}
\item Οι παραπληρωματικές γωνίες έχουν άθροισμα $ 180\degree $.
\item Ο αριθμός $ 3\frac{5}{4} $ είναι μεικτός αριθμός.
\item Η ισότητα $ 128=14\cdot8+16 $ παριστάνει ισότητα Ευκλείδειας διαίρεσης.
\item Ο αριθμός $ 15 $ είναι σύνθετος.
\item Οι κατακορυφήν γωνίες είναι μεταξύ τους ίσες.
\item Το κλάσμα $ \frac{12}{7} $ είναι μικρότερο της μονάδας.
\end{rlist}\monades{3}
\item \mbox{}\\
\textbf{Α.} Να απαντήσετε στις παρακάτω ερωτήσεις.
\begin{rlist}
\item Ποιές γωνίες ονομάζονται κατακορυφήν;
\item Ποιά κλάσματα ονομάζονται ανάγωγα;
\item Τι γνωρίζουμε για τα σημεία της μεσοκαθέτου ενός ευθύγραμμου τμήματος;
\end{rlist}\monades{3}\mbox{}\\
\textbf{Β.} \swstolathos
\begin{rlist}
\item Οι συμπληρωματικές γωνίες έχουν άθροισμα $ 180\degree $.
\item Τα κλάσματα $ \frac{4}{5},\frac{28}{35} $ είναι ισοδύναμα.
\item Το σημείο $ A(3,1) $ βρίσκεται στην ίδια θέση με το σημείο $ B(1,3) $.
\item Το $ 30\% $ του $ 40 $ ισούται με 12.
\item Το κλάσμα $ \frac{25}{30} $ είναι ανάγωνο.
\item Οι γωνίες $ 87\degree 40' $ και $ 92\degree 20' $ είναι παραπληρωματικές.
\end{rlist}\monades{3}
\end{thema}
\newpage
\noindent
\askhseis
\begin{thema}
\item \mbox{}\\
Δίνονται οι αριθμητικές παραστάσεις $ A=3^2\cdot\left(7^2-5\cdot8 \right)-225:15^2+4 $ και $ B=\left(3^3-2^4\right)\cdot\left(144:4^2-125:5^2\right) $.
\begin{rlist}
\item Να υπολογίσετε τις παραστάσεις $ A,B $.\monades{5}
\item Να απλοποιήσετε το κλάσμα $ \dfrac{A}{B} $.\monades{2}
\end{rlist}
\item \mbox{}\\
Αν γνωρίζουμε ότι οι ευθείες $ \varepsilon_1,\varepsilon_2 $είναι μεταξύ τους παράλληλες τότε να υπολογίσετε τις γωνίες του παρακάτω σχήματος.
\begin{center}
\begin{tikzpicture}[scale=1.3]
\tkzDefPoint(0.25,0){A}
\tkzDefPoint(4,0){B}
\tkzDefPoint(.25,-.75){C}
\tkzDefPoint(2.25,.75){D}
\tkzDefPoint(1.25,0){E}
\tkzDefPoint(2,-.75){Z}
\tkzDefPoint(4,.75){H}
\tkzDefPoint(3,0){J}
\tkzMarkAngle[size=.3](B,E,D)
\tkzMarkAngle[size=.3](A,E,C)
\tkzMarkAngle[size=.25](D,E,A)
\tkzMarkAngle[size=.25](C,E,B)
\tkzMarkAngle[size=.3](B,J,H)
\tkzMarkAngle[size=.25](Z,J,B)
\tkzMarkAngle[size=.3](A,J,Z)
\tkzMarkAngle[size=.25](H,J,A)
\tkzLabelAngle[pos=.4](C,E,B){\footnotesize$138\degree$}
\tkzLabelAngle[pos=.5](B,E,D){\footnotesize$a$}
\tkzLabelAngle[pos=-.5](A,E,C){\footnotesize$\beta$}
\tkzLabelAngle[pos=.4](D,E,A){\footnotesize$\gamma$}
\tkzLabelAngle[pos=.5](B,J,H){\footnotesize$\delta$}
\tkzLabelAngle[pos=.4](Z,J,B){\footnotesize$\varepsilon$}
\tkzLabelAngle[pos=-.5](A,J,Z){\footnotesize$\zeta$}
\tkzLabelAngle[pos=.4](H,J,A){\footnotesize$\eta$}
\draw(A)--(B);
\draw(C)--(D);
\draw(Z)--(H);
\tkzDrawPoints(E,J)
\node at (2.25,.85) {\footnotesize$\varepsilon_1$};
\node at (4,.85) {\footnotesize$\varepsilon_2$};
\end{tikzpicture}
\end{center}\monades{7}
\item \mbox{}\\
Ένας οινοπαραγωγός πουλάει $ 5lt $ κρασί προς $ 45$\officialeuro .\ Ένα βαρέλι περιέχει $ 350lt $ κρασί. Σήμερα ο οινοπαραγωγός πούλησε το $ 72\% $ του περιεχομένου του βαρελιού.
\begin{rlist}
\item Να βρεθεί πόσα λίτρα κρασιού πούλησε ο παραγωγός σήμερα.\monades{5}
\item Αν το κρασί που πούλησε ήταν $ 252lt $ να βρείτε πόσα χρήματα εισέπραξε.\monades{2}
\end{rlist}
\end{thema}
\end{document}

