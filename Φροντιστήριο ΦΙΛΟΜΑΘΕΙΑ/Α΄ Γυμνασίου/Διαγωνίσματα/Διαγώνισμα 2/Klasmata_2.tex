\documentclass[ektypwsh]{diag-xelatex}
\usepackage[amsbb]{mtpro2}
\usepackage[no-math,cm-default]{fontspec}
\usepackage{xunicode}
\usepackage{xltxtra}
\usepackage{xgreek}
\usepackage{amsmath}
\defaultfontfeatures{Mapping=tex-text,Scale=MatchLowercase}
\setmainfont[Mapping=tex-text,Numbers=Lining,Scale=1.0,BoldFont={Minion Pro Bold}]{Minion Pro}
\newfontfamily\scfont{GFS Artemisia}
\font\icon = "Webdings"
\usepackage[amsbb]{mtpro2}
\xroma{red!80!black}
\newcommand{\tss}[1]{\textsuperscript{#1}}
\newcommand{\tssL}[1]{\MakeLowercase\textsuperscript{#1}}
\newlist{rlist}{enumerate}{3}
\setlist[rlist]{itemsep=0mm,label=\roman*.}
%------ ΜΗΚΟΣ ΓΡΑΜΜΗΣ ΚΛΑΣΜΑΤΟΣ ---------
\DeclareRobustCommand{\frac}[3][0pt]{%
{\begingroup\hspace{#1}#2\hspace{#1}\endgroup\over\hspace{#1}#3\hspace{#1}}}
%----------------------------------------
\usepackage{multicol}
\usepackage{eurosym}

\begin{document}
\titlos{Μαθηματικά Α΄ Γυμνασίου}{ΚΛΑΣΜΑΤΑ}
\thewria
\begin{thema}
\item \mbox{}\\
A. Να απαντήσετε στις παρακάτω ερωτήσεις.
\begin{rlist}
\item Ποιά κλάσματα ονομάζονται ομώνυμα;
\item Ανάμεσα σε δύο κλάσματα με κοινό αριθμητή ποιό είναι το μεγαλύτερο;
\item Ποιό κλάσμα ονομάζεται ανάγωγο;
\item Τί ονομάζεται μεικτός αριθμός;\monades{3}
\end{rlist}
B. Να χαρακτηρίσετε τις παρακάτω παραστάσεις ως σωστές (Σ) ή λανθασμένες (Λ).
\begin{rlist}
\item Αν δύο κλάσματα έιναι ίσα τότε και τα χιαστί γινομενά τους είναι ίσα.
\item Ο αριθμός $ 3\frac{5}{2} $ είναι ένας μεικτός αριθμός.
\item Τα κλάσματα $ \frac{a}{\beta} $ και $ \frac{2a}{2\beta} $ είναι ισοδύναμα.
\item Για να πολλαπλασιάσουμε δύο κλάσματα μεταξύ τους πολλαπλασιάζουμε τους όρους τους χιαστί.
\item Το σύνθετο κλάσμα $ \frac[1mm]{a}{\frac{\beta}{\gamma}} $ είναι ίσο με το κλάσμα $ \frac[1mm]{\frac{a}{\beta}}{\gamma} $.\monades{3}
\end{rlist}
\item \mbox{}\\
Α. Να απαντήσετε στις παρακάτω ερωτήσεις.
\begin{rlist}
\item Ποιοί αριθμοί ονομάζονται αντίστροφοι;
\item Ποιά κλάσματα μπορούν να μετατραπούν σε μεικτούς αριθμούς;
\item Πότε ένα κλάσμα είναι μεγαλύτερο της μονάδας;
\item Ποιά κλάσματα ονομάζονται ετερόνυμα;\monades{3}
\end{rlist}
Β. Να επιλέξετε τη σωστή απάντηση για τις παρακάτω προτάσεις.
\begin{rlist}
\item Το κλάσμα $ \frac{3}{4} $ είναι ισοδύναμο με το κλάσμα
\begin{multicols}{4}
\begin{itemize}
\item $ \dfrac{8}{10} $
\item $ \dfrac{4}{3} $
\item $ \dfrac{9}{12} $
\item $ \dfrac{6}{7} $
\end{itemize}
\end{multicols}
\item Το κλάσμα $ \frac{48}{32} $ αν απλοποιηθεί είναι ισοδύναμο με το κλάσμα
\begin{multicols}{4}
\begin{itemize}
\item $ \dfrac{3}{2} $
\item $ \dfrac{4}{3} $
\item $ 2 $
\item $ \dfrac{2}{3} $
\end{itemize}
\end{multicols}
\item Ο αντίστροφος του $ \frac{4}{12}+\frac{7}{12} $ είναι ίσος με 
\begin{multicols}{4}
\begin{itemize}
\item $ \dfrac{11}{12} $
\item $ \dfrac{7}{12} $
\item $ \dfrac{12}{11} $
\item $ \dfrac{11}{24} $\monades{3}
\end{itemize}
\end{multicols}
\end{rlist}
\end{thema}
\newpage
\noindent
\askhseis
\begin{thema}
\item \mbox{}\\
Σε ένα σχολείο με $ 480 $ μαθητές, τα $ \frac{7}{12} $ των παιδιών ανήκουν σε φτωχές οικογένειες. Φέτος ο δήμος θα μοιράσει το ποσό των $ 30.000 $\officialeuro\ στους μαθητές αυτού του σχολείου.
\begin{rlist}
\item Να βρεθούν πόσα παιδιά του σχολείου ανήκουν σε φτωχές οικογένειες.\monades{4}
\item Να βρεθεί το χρηματικό ποσό που αντιστοιχεί σε κάθε παιδί.\monades{3}
\end{rlist}
\item \mbox{}\\
i. Να υπολογίσετε την τιμή της παρακάτω αριθμητικής παράστασης και να απλοποιήσετε και το αποτέλεσμα.
\[ \frac{3}{2}\cdot\left( \frac{4}{5}+\frac{1}{10}-\frac{3}{15}\right)+\frac{8}{5}:\left( 3-\frac{9}{5}\right)   \]\monades{4}\\
ii. Να μετατραπεί η παρακάτω παράσταση σε μεικτό αριθμό
\[ \frac{4}{5}+\frac{2}{3}\cdot\left( \frac{10}{7}-1\right) +3 \]\monades{3}
\item \mbox{}\\
i. Να τοποθετήσετε τους παρακάτω αριθμούς στη σειρά από το μεγαλύτερο στο μικρότερο.
\[ \frac{14}{15},\ \frac{9}{12},\ \frac{7}{5},\ 1\frac{3}{10},\ \frac{34}{30} \]\monades{3}\\
ii. Να εξετάσετε αν από τις παρακάτω παραστάσεις προκύπτουν ισοδύναμα κλάσματα.
\[ A=\frac{5}{3}\cdot\frac{9}{10}+\frac{9}{15}:3\frac{3}{5}\ \ ,\ \ B=\frac{7}{12}+2+2\cdot\frac{3}{\frac{8}{5}}-\frac{14}{3} \]
\end{thema}
\end{document}

