\documentclass[ektypwsh]{diag-xelatex}
\usepackage[amsbb]{mtpro2}
\usepackage[no-math,cm-default]{fontspec}
\usepackage{xunicode}
\usepackage{xltxtra}
\usepackage{xgreek}
\usepackage{amsmath}
\defaultfontfeatures{Mapping=tex-text,Scale=MatchLowercase}
\setmainfont[Mapping=tex-text,Numbers=Lining,Scale=1.0,BoldFont={Minion Pro Bold}]{Minion Pro}
\newfontfamily\scfont{GFS Artemisia}
\font\icon = "Webdings"
\usepackage[amsbb]{mtpro2}
\usepackage[left=2.00cm, right=2.00cm, top=2.00cm, bottom=3.00cm]{geometry}
\xroma{red!80!black}
\newcommand{\tss}[1]{\textsuperscript{#1}}
\newcommand{\tssL}[1]{\MakeLowercase\textsuperscript{#1}}
\newlist{rlist}{enumerate}{3}
\setlist[rlist]{itemsep=0mm,label=\roman*.}
\usepackage{multicol}
\usepackage{amsmath}

\begin{document}
\titlos{Μαθηματικά Α΄ Γυμνασίου}{Φυσικοί Αριθμοί}
\thewria
\begin{thema}
\item \textbf{Ερωτήσεις Θεωρίας}\\
Να απαντήσετε στις παρακάτω ερωτήσεις.
\begin{rlist}
\item Ποιοί αριθμοί ονομάζονται πρώτοι;
\item Τι ονομάζουμε πολλαπλάσιο ενός αριθμού $ a $;
\item Ποιός κανόνας πρέπει να ισχύει ώστε μια διαίρεση μεταξύ δύο αριθμών $ \varDelta $ και $ \delta $ να είναι Ευκλείδεια;
\item Με ποιά σειρά εκτελούνται οι πράξεις σε μια αριθμητική παράσταση;
\item Γράψτε την ισότητα της Ευκλείδειας διαίρεσης.
\end{rlist}\monades{6}
\item \textbf{Σωστό - Λάθος / Πολλαπλής επιλογής}\\
Α. Να χαρακτηρίσετε τις παρακάτω προτάσεις ως σωστές (Σ) ή λανθασμένες (Λ).
\begin{rlist}
\item Σε μια Ευκλείδεια διαίρεση ισχύει πάντα η σχέση $ \upsilon>\delta $.
\item Οι αριθμοί $ 3,7 $ και $ 11 $ είναι πρώτοι.
\item Το Ε.Κ.Π. των αριθμών $ 4,12 $ και $ 16 $ είναι το $ 32 $.
\item Ένα υπόλοιπο της διαίρεσης ενός αριθμού $ \varDelta $ με το $ 7 $ μπορεί να είναι το $ 9 $.
\item Διαιρέτης ενός αριθμού $ a $ ονομάζεται κάθε αριθμός που τον διαιρεί.
\end{rlist}\monades{3}\\
B. Να επιλέξετε τη σωστή απάντηση σε κάθεμία από τις παρακάτω ερωτήσεις.
\begin{rlist}
\item Ποιός από τους παρακάτω αριθμούς είναι πρώτος;
\begin{multicols}{4}
\begin{itemize}
\item $ 23 $
\item $ 25 $
\item $ 21 $
\item $ 27 $
\end{itemize}
\end{multicols}
\item Ποιοί από τους παρακάτω αριθμούς είναι διαιρέτες του $ 24 $;
\begin{multicols}{4}
\begin{itemize}
\item $ 1,2,3,4,6,8,12 $
\item $ 1,2,3,4,6,8,12,24 $
\item $ 1,3,6,9,12,24 $
\item $ 1,24 $
\end{itemize}
\end{multicols}
\item Ποιός από τους παρακάτω αριθμούς διαιρείται συγχρόνως με το $ 3 $ και το $ 2 $;
\begin{multicols}{4}
\begin{itemize}
\item $ 3982 $
\item $ 3671 $
\item $ 4878 $
\item $ 9820 $
\end{itemize}
\end{multicols}
\end{rlist}
\end{thema}
\newpage
\noindent
\askhseis
\begin{thema}
\item \textbf{Αριθμητική παράσταση}\\
Να υπολογίσετε την τιμή κάθεμιάς από τις παρακάτω αριθμητικές παραστάσεις.
\begin{multicols}{2}
\begin{rlist}
\item $ 4^2(5^3-144:3^2)+7^2(200:5^2+2^4:4^2) $
\item $ 5^4-3^5+11^2+7^3 $
\end{rlist}
\end{multicols}\monades{7}
\item \textbf{Ευκλείδεια διαίρεση - Ε.Κ.Π. - Μ.Κ.Δ.}
\begin{rlist}
\item Να κάνετε την Ευκλείδεια διαίρεση $ 4260:27 $ και στη συνέχεια να γράψετε την ισότητα της διαίρεσης.\monades{2}
\item Να υπολογίσετε το Ε.Κ.Π. και το Μ.Κ.Δ. των παρακάτω αριθμών :
\begin{multicols}{2}
\begin{itemize}
\item $ 50,\ 85,\ 75 $
\item $ 120,\ 180,\ 160 $
\end{itemize}
\end{multicols}\monades{5}
\end{rlist}
\item \textbf{Σύνθετο θέμα}\\
Ένας εκδοτικός οίκος εκδίδει ένα περιοδικό Α κάθε 32 μέρες, ένα περιοδικό Β κάθε 48 μέρες και ένα περιοδικό Γ κάθε 60 μέρες. Αν σήμερα είναι \today\  και εκδοθούν και τα τρία περιοδικά μαζί, να βρεθεί
\begin{rlist}
\item μετά από πόσες μέρες θα εκδοθούν ξανά και τα τρία περιοδικά μαζί;\monades{5}
\item τη μέρα που θα εκδοθούν ξανά και τα τρία περιοδικά μαζί, τι μέρα θα είναι;\monades{2}
\end{rlist}
\end{thema}
\gymnasio
\kaliepityxia
\end{document}

