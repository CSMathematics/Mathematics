\documentclass[ektypwsh]{diag-xelatex}
\usepackage[amsbb,subscriptcorrection,zswash,mtpcal,mtphrb]{mtpro2}
\usepackage[no-math,cm-default]{fontspec}
\usepackage{xunicode}
\usepackage{xgreek}
\usepackage{amsmath}
\defaultfontfeatures{Mapping=tex-text,Scale=MatchLowercase}
\setmainfont[Mapping=tex-text,Numbers=Lining,Scale=1.0,BoldFont={Minion Pro Bold}]{Minion Pro}
\newfontfamily\scfont{GFS Artemisia}
\font\icon = "Webdings"
\usepackage[amsbb,subscriptcorrection,zswash,mtpcal,mtphrb]{mtpro2}
\xroma{red!80!black}
%------TIKZ - ΣΧΗΜΑΤΑ - ΓΡΑΦΙΚΕΣ ΠΑΡΑΣΤΑΣΕΙΣ ----
\usepackage{tikz}
\usepackage{tkz-euclide}
\usetkzobj{all}
\usepackage[framemethod=TikZ]{mdframed}
\usetikzlibrary{decorations.pathreplacing}
\usepackage{pgfplots}
\usetkzobj{all}
%-----------------------
\usepackage{calc}
\usepackage{hhline}
\usepackage[explicit]{titlesec}
\usepackage{graphicx}
\usepackage{multicol}
\usepackage{multirow}
\usepackage{enumitem}
\usepackage{tabularx}
\usepackage[decimalsymbol=comma]{siunitx}
\usetikzlibrary{backgrounds}
\usepackage{sectsty}
\sectionfont{\centering}
\setlist[enumerate]{label=\bf{\large \arabic*.}}
\usepackage{adjustbox}
\usepackage{mathimatika,gensymb,eurosym,wrap-rl}
\usepackage{systeme,regexpatch}
%-------- ΜΑΘΗΜΑΤΙΚΑ ΕΡΓΑΛΕΙΑ ---------
\usepackage{mathtools}
%----------------------
%-------- ΠΙΝΑΚΕΣ ---------
\usepackage{booktabs}
%----------------------
%----- ΥΠΟΛΟΓΙΣΤΗΣ ----------
\usepackage{calculator}
%----------------------------
%------ ΔΙΑΓΩΝΙΟ ΣΕ ΠΙΝΑΚΑ -------
\usepackage{array}
\newcommand\diag[5]{%
\multicolumn{1}{|m{#2}|}{\hskip-\tabcolsep
$\vcenter{\begin{tikzpicture}[baseline=0,anchor=south west,outer sep=0]
\path[use as bounding box] (0,0) rectangle (#2+2\tabcolsep,\baselineskip);
\node[minimum width={#2+2\tabcolsep-\pgflinewidth},
minimum  height=\baselineskip+#3-\pgflinewidth] (box) {};
\draw[line cap=round] (box.north west) -- (box.south east);
\node[anchor=south west,align=left,inner sep=#1] at (box.south west) {#4};
\node[anchor=north east,align=right,inner sep=#1] at (box.north east) {#5};
\end{tikzpicture}}\rule{0pt}{.71\baselineskip+#3-\pgflinewidth}$\hskip-\tabcolsep}}
%---------------------------------
%---- ΟΡΙΖΟΝΤΙΟ - ΚΑΤΑΚΟΡΥΦΟ - ΠΛΑΓΙΟ ΑΓΚΙΣΤΡΟ ------
\newcommand{\orag}[3]{\node at (#1)
{$ \overcbrace{\rule{#2mm}{0mm}}^{{\scriptsize #3}} $};}
\newcommand{\kag}[3]{\node at (#1)
{$ \undercbrace{\rule{#2mm}{0mm}}_{{\scriptsize #3}} $};}
\newcommand{\Pag}[4]{\node[rotate=#1] at (#2)
{$ \overcbrace{\rule{#3mm}{0mm}}^{{\rotatebox{-#1}{\scriptsize$#4$}}}$};}
%-----------------------------------------


%------------------------------------------
\newcommand{\tss}[1]{\textsuperscript{#1}}
\newcommand{\tssL}[1]{\MakeLowercase{\textsuperscript{#1}}}
%---------- ΛΙΣΤΕΣ ----------------------
\newlist{bhma}{enumerate}{3}
\setlist[bhma]{label=\bf\textit{\arabic*\textsuperscript{o}\;Βήμα :},leftmargin=0cm,itemindent=1.8cm,ref=\bf{\arabic*\textsuperscript{o}\;Βήμα}}
\newlist{rlist}{enumerate}{3}
\setlist[rlist]{itemsep=0mm,label=\roman*.}
\newlist{brlist}{enumerate}{3}
\setlist[brlist]{itemsep=0mm,label=\bf\roman*.}
\newlist{tropos}{enumerate}{3}
\setlist[tropos]{label=\bf\textit{\arabic*\textsuperscript{oς}\;Τρόπος :},leftmargin=0cm,itemindent=2.3cm,ref=\bf{\arabic*\textsuperscript{oς}\;Τρόπος}}
% Αν μπει το bhma μεσα σε tropo τότε
%\begin{bhma}[leftmargin=.7cm]
\tkzSetUpPoint[size=7,fill=white]
\tikzstyle{pl}=[line width=0.3mm]
\tikzstyle{plm}=[line width=0.4mm]
\usepackage{etoolbox}
\makeatletter
\renewrobustcmd{\anw@true}{\let\ifanw@\iffalse}
\renewrobustcmd{\anw@false}{\let\ifanw@\iffalse}\anw@false
\newrobustcmd{\noanw@true}{\let\ifnoanw@\iffalse}
\newrobustcmd{\noanw@false}{\let\ifnoanw@\iffalse}\noanw@false
\renewrobustcmd{\anw@print}{\ifanw@\ifnoanw@\else\numer@lsign\fi\fi}
\makeatother



\begin{document}
\titlos{Α΄ Γυμνασίου}{ΕΥΚΛΕΙΔΕΙΑ ΔΙΑΙΡΕΣΗ - Ε.Κ.Π. - Μ.Κ.Δ.}
\thewria
\begin{thema}
\item \mbox{}\\\vspace{-7mm}
\begin{erwthma}
\item Να απαντήσετε στις παρακάτω ερωτήσεις.
\begin{rlist}
\item Ποιος κανόνας πρέπει να ισχύει ώστε μια διάιρεση μεταξύ δύο φυσικών αριθμών να είναι Ευκλέιδεια;
\item Ποιοί αριθμοί ονομάζονται πρώτοι;
\item Να γράψετε τον κανόνα με τον οποίο υπολογίζουμε Ε.Κ.Π. δύο ή περισσότερων αριθμών που έχουν αναλυθεί σε γινόμενο πρώτων.
\item Γράψτε την ισότητα της Ευκλείδειας διαίρεσης δύο φυσικών αριθμών $ \varDelta $ και $ \delta $.
\item Τι ονομάζεται διαιρέτης ενός φυσικού αριθμού $ a $;
\end{rlist}\monades{4}
\item \swstolathos
\begin{rlist}
\item Ο αριθμός $ 27 $ είναι πρώτος.
\item Οι αριθμοί $ 8,9 $ είναι πρώτοι μεταξύ τους.
\item Το διπλάσιο ενός πρώτου αριθμού είναι επίσης πρώτος.
\item Η ισότητα $ 128=11\cdot9+29 $ παριστάνει ισότητα Ευκλείδειας διαίρεσης.
\item Η διαίρεση $ \nu:12 $, όπου $ \nu $ είναι φυσικός αριθμός, μπορεί να έχει υπόλοιπο $ 7 $.
\end{rlist}\monades{2}
\end{erwthma}
\item \mbox{}\\\vspace{-7mm}
\begin{erwthma}
\item Να απαντήσετε στις παρακάτω ερωτήσεις.
\begin{rlist}
\item Να γράψετε τον κανόνα με τον οποίο υπολογίζουμε Μ.Κ.Δ. δύο ή περισσότερων αριθμών που έχουν αναλυθεί σε γινόμενο πρώτων.
\item Τι ονομάζεται πολλαπλάσιο ενός αριθμού $ a $;
\item Ποιοί αριθμοί ονομάζονται σύνθετοι;
\item Να γράψετε τους πρώτους δέκα στη σειρά, πρώτους αριθμούς.
\item Πότε μια Ευκλείδεια διαίρεση ονομάζεται τέλεια;
\end{rlist}\monades{4}
\item Να επιλέξετε τη σωστή απάντηση σε κάθεμία από τις παρακάτω ερωτήσεις.
\begin{rlist}[leftmargin=4mm]
\item Ποιά από τις παρακάτω ισότητες παριστάνει σε κάθε περίπτωση ισότητα Ευκλείδειας διαίρεσης;
\begin{multicols}{4}
\begin{alist}
\item $ 127=12\cdot 10+7 $
\item $ 549=35\cdot 15+24 $
\item $ 827=33\cdot24+35 $
\item $ 599=15\cdot 38+29 $
\end{alist}
\end{multicols}
\item Ποιοί από τους παρακάτω αριθμούς είναι διαιρέτες του $ 32 $;
\begin{multicols}{4}
\begin{alist}
\item $ 1,2,3,4,6,8,16,32 $
\item $ 1,32 $
\item $ 1,2,4,8,16,32 $
\item $ 0,1,2,4,8,16,32 $
\end{alist}
\end{multicols}
\item Αν $ a $ και $ \beta $ είναι πρώτοι αριθμοί τότε ο Μ.Κ.Δ. τους είναι:
\begin{multicols}{4}
\begin{alist}
\item $ 1 $
\item $ a $
\item $ \beta $
\item $ a\cdot\beta $
\end{alist}
\end{multicols}
\item Αν $ a $ και $ \beta $ είναι πρώτοι αριθμοί τότε το Ε.Κ.Π. τους είναι:
\begin{multicols}{4}
\begin{alist}
\item $ 1 $
\item $ a $
\item $ \beta $
\item $ a\cdot\beta $
\end{alist}
\end{multicols}
\end{rlist}\monades{2}
\end{erwthma}
\end{thema}
%\newpage
%\noindent
\askhseis
\begin{thema}
\item \mbox{}\\
Δίνονται οι φυσικοί αριθμοί $ a=x5x $ και $ \beta=34x $.
\begin{erwthma}
\item Να βρεθεί η τιμή του ψηφίου $ x $ ώστε ο αριθμός $ a $ να διαιρείται με το $ 9 $ και ο $ \beta $ να διαιρείται με το $ 2 $.\\\monades{3}
\item Να υπολογίσετε το Ε.Κ.Π. και το Μ.Κ.Δ. των φυσικών αριθμών $ a $ και $ \beta $.\monades{4}
\end{erwthma}
\item \mbox{}\\
Ένα σχολείο ετοιμάζεται για να πραγματοποιήσει μια διήμερη εκδρομή.
\begin{erwthma}
\item Αν το σύνολο των μαθητών είναι $ 416 $ πόσα λεωφορεία των $ 52 $ θέσεων
απαιτούνται; \monades{3}
\item Αν για κάθε $ 32 $ παιδιά απαιτείται $ 1 $ συνοδός καθηγητής, πόσοι καθηγητές θα χρειαστεί να συνοδέψουν την εκδρομή;\monades{2}
\item Αν το ξενοδοχείο που θα καταλύσει το σχολείο έχει μόνο τρίκλινα δωμάτια, πόσα δωμάτια θα χρειαστούν τα παιδιά; Θα χωρέσουν ακριβώς στα δωμάτια;\monades{2}
\end{erwthma}
\item \mbox{}\\
Σήμερα η μέρα είναι Πέμπτη 22 Σεπτεμβρίου 2016.
\begin{erwthma}
\item Να βρεθεί η μέρα της εβδομάδας ύστερα από 239 ημέρες.\monades{2}
\item Ένα περιοδικό Α εκδίδεται κάθε 24 μέρες, ενώ ένα περιοδικό Β εκδίδεται κάθε 40 μέρες. Αν σήμερα εκδοθεί το πρώτο τεύχος και των δύο, τότε ποιά θα είναι η μέρα της εβδομάδας όταν θα ξαναβγούν και τα δύο μαζί;\monades{3}
\item Σε ποιό τέυχος θα βρίσκεται το κάθε περιοδικό την ημέρα αυτή; \monades{2}
\end{erwthma}
\end{thema}
\end{document}

