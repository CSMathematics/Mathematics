\documentclass[ektypwsh]{diag-xelatex}
\usepackage[amsbb,subscriptcorrection,zswash,mtpcal,mtphrb]{mtpro2}
\usepackage[no-math,cm-default]{fontspec}
\usepackage{xunicode}
\usepackage{xgreek}
\usepackage{amsmath}
\defaultfontfeatures{Mapping=tex-text,Scale=MatchLowercase}
\setmainfont[Mapping=tex-text,Numbers=Lining,Scale=1.0,BoldFont={Minion Pro Bold}]{Minion Pro}
\newfontfamily\scfont{GFS Artemisia}
\font\icon = "Webdings"
\usepackage[amsbb,subscriptcorrection,zswash,mtpcal,mtphrb]{mtpro2}
\xroma{red!80!black}
%------TIKZ - ΣΧΗΜΑΤΑ - ΓΡΑΦΙΚΕΣ ΠΑΡΑΣΤΑΣΕΙΣ ----
\usepackage{tikz}
\usepackage{tkz-euclide}
\usetkzobj{all}
\usepackage[framemethod=TikZ]{mdframed}
\usetikzlibrary{decorations.pathreplacing}
\usepackage{pgfplots}
\usetkzobj{all}
%-----------------------
\usepackage{calc}
\usepackage{hhline}
\usepackage[explicit]{titlesec}
\usepackage{graphicx}
\usepackage{multicol}
\usepackage{multirow}
\usepackage{enumitem}
\usepackage{tabularx}
\usepackage[decimalsymbol=comma]{siunitx}
\usetikzlibrary{backgrounds}
\usepackage{sectsty}
\sectionfont{\centering}
\setlist[enumerate]{label=\bf{\large \arabic*.}}
\usepackage{adjustbox}
\usepackage{mathimatika,gensymb,eurosym,wrap-rl}
\usepackage{systeme,regexpatch}
%-------- ΜΑΘΗΜΑΤΙΚΑ ΕΡΓΑΛΕΙΑ ---------
\usepackage{mathtools}
%----------------------
%-------- ΠΙΝΑΚΕΣ ---------
\usepackage{booktabs}
%----------------------
%----- ΥΠΟΛΟΓΙΣΤΗΣ ----------
\usepackage{calculator}
%----------------------------
%------ ΔΙΑΓΩΝΙΟ ΣΕ ΠΙΝΑΚΑ -------
\usepackage{array}
\newcommand\diag[5]{%
\multicolumn{1}{|m{#2}|}{\hskip-\tabcolsep
$\vcenter{\begin{tikzpicture}[baseline=0,anchor=south west,outer sep=0]
\path[use as bounding box] (0,0) rectangle (#2+2\tabcolsep,\baselineskip);
\node[minimum width={#2+2\tabcolsep-\pgflinewidth},
minimum  height=\baselineskip+#3-\pgflinewidth] (box) {};
\draw[line cap=round] (box.north west) -- (box.south east);
\node[anchor=south west,align=left,inner sep=#1] at (box.south west) {#4};
\node[anchor=north east,align=right,inner sep=#1] at (box.north east) {#5};
\end{tikzpicture}}\rule{0pt}{.71\baselineskip+#3-\pgflinewidth}$\hskip-\tabcolsep}}
%---------------------------------
%---- ΟΡΙΖΟΝΤΙΟ - ΚΑΤΑΚΟΡΥΦΟ - ΠΛΑΓΙΟ ΑΓΚΙΣΤΡΟ ------
\newcommand{\orag}[3]{\node at (#1)
{$ \overcbrace{\rule{#2mm}{0mm}}^{{\scriptsize #3}} $};}
\newcommand{\kag}[3]{\node at (#1)
{$ \undercbrace{\rule{#2mm}{0mm}}_{{\scriptsize #3}} $};}
\newcommand{\Pag}[4]{\node[rotate=#1] at (#2)
{$ \overcbrace{\rule{#3mm}{0mm}}^{{\rotatebox{-#1}{\scriptsize$#4$}}}$};}
%-----------------------------------------


%------------------------------------------
\newcommand{\tss}[1]{\textsuperscript{#1}}
\newcommand{\tssL}[1]{\MakeLowercase{\textsuperscript{#1}}}
%---------- ΛΙΣΤΕΣ ----------------------
\newlist{bhma}{enumerate}{3}
\setlist[bhma]{label=\bf\textit{\arabic*\textsuperscript{o}\;Βήμα :},leftmargin=0cm,itemindent=1.8cm,ref=\bf{\arabic*\textsuperscript{o}\;Βήμα}}
\newlist{rlist}{enumerate}{3}
\setlist[rlist]{itemsep=0mm,label=\roman*.}
\newlist{brlist}{enumerate}{3}
\setlist[brlist]{itemsep=0mm,label=\bf\roman*.}
\newlist{tropos}{enumerate}{3}
\setlist[tropos]{label=\bf\textit{\arabic*\textsuperscript{oς}\;Τρόπος :},leftmargin=0cm,itemindent=2.3cm,ref=\bf{\arabic*\textsuperscript{oς}\;Τρόπος}}
% Αν μπει το bhma μεσα σε tropo τότε
%\begin{bhma}[leftmargin=.7cm]
\tkzSetUpPoint[size=7,fill=white]
\tikzstyle{pl}=[line width=0.3mm]
\tikzstyle{plm}=[line width=0.4mm]
\usepackage{etoolbox}
\makeatletter
\renewrobustcmd{\anw@true}{\let\ifanw@\iffalse}
\renewrobustcmd{\anw@false}{\let\ifanw@\iffalse}\anw@false
\newrobustcmd{\noanw@true}{\let\ifnoanw@\iffalse}
\newrobustcmd{\noanw@false}{\let\ifnoanw@\iffalse}\noanw@false
\renewrobustcmd{\anw@print}{\ifanw@\ifnoanw@\else\numer@lsign\fi\fi}
\makeatother

\begin{document}
\titlos{ΜΑΘΗΜΑΤΙΚΑ Α΄ Γυμνασίου}{ΘΕΤΙΚΟΙ \& ΑΡΝΗΤΙΚΟΙ ΑΡΙΘΜΟΙ}
\thewria
\begin{thema}
\item \mbox{}\\\vspace{-7mm}
\begin{erwthma}
\item Να απαντήσετε στις παρακάτω ερωτήσεις.
\begin{rlist}
\item Τι ονομάζεται απόλυτη τιμή ενός ρητού αριθμού $ a $;
\item Ποιος αριθμός ονομάζεται αρνητικός;
\item Να διατυπώσετε τον κανόνα της πρόσθεσης για δύο ετερώσημους αριθμούς.
\item Ποιοί αριθμοί ονομάζονται αντίστροφοι;
\item Τι πρόσημο έχει το πηλίκο δύο ομόσημων ρητών αριθμών;
\end{rlist}\monades{4}
\item \swstolathos
\begin{rlist}
\item Αν $ a>0 $ και $ \beta<0 $ τότε $ a\cdot\beta<0 $.
\item Ισχύει ότι $ (-2)^4=-2^4 $.
\item Αν $ a<0 $ τότε $ |a|=-a $.
\item Ισχύει ότι $ \left( \frac{3}{5}\right)^3<\left( \frac{2}{5}\right)^2 $.
\item Οι αριθμοί $ |-7| $ και $ -(-7) $ είναι αντίθετοι.
\end{rlist}\monades{2}
\end{erwthma}
\item \mbox{}\\\vspace{-7mm}
\begin{erwthma}
\item Να απαντήσετε στις παρακάτω ερωτήσεις.
\begin{rlist}
\item Τι ονομάζεται δύναμη ενός ρητού αριθμού $ a $ στον εκθέτη $ \nu $;
\item Ποιος αριθμός ονομάζεται θετικός;
\item Να διατυπώσετε τον κανόνα του πολλαπλασιασμού για δύο ομόσημους αριθμούς.
\item Ποιοί αριθμοί ονομάζονται αντίστροφοι;
\item Τι πρόσημο έχει μια δύναμη με αρνητική βάση και περιττό εκθέτη;
\end{rlist}\monades{4}
\item \swstolathos
\begin{rlist}
\item Ο αντίθετος του $ -8 $ είναι ο $ -\frac{1}{8} $.
\item Οι αντίστροφοι αριθμοί είναι ομόσημοι.
\item Το γινόμενο πέντε αρνητικών αριθμών είναι αρνητικό.
\item Αν $ a\cdot\beta>0 $ τότε $ a>0 $ και $ \beta>0 $.
\item Ισχύει ότι $ 18^5:9^5=32 $.
\end{rlist}\monades{2}
\end{erwthma}
\end{thema}
\newpage
\noindent
\askhseis
\begin{thema}
\item \mbox{}\\\vspace{-7mm}
\begin{erwthma}
\item Να υπολογίσετε την τιμή των παρακάτω παραστάσεων.
\begin{multicols}{2}
\begin{rlist}
\item $ (-3)\cdot(-2)^4+(-2)\cdot\left( 3^3-2^5\right)  $
\item $ \dfrac{4^{12}\cdot \left( 3^5\right)^2}{(3\cdot 4)^{11}} $
\end{rlist}
\end{multicols}\monades{2}
\item Να υπολογίσετε την τιμή της παρακάτω παράστασης.
\[ A=\left[ 2^5-(4\cdot 7-(-8)\cdot5)+12^2:6^2\right] +7 \]\monades{2}
\item Να υπολογίσετε την τιμή της παρακάτω παράστασης.
\[ A=|-(-12)|-3^4+\left|125:5^2-144:12^2\right| +\left| -|-3|\right|  \]\monades{3}
\end{erwthma}
\item \mbox{}\\\vspace{-7mm}
\begin{erwthma}
\item Να συγκρίνετε τους παρακάτω αριθμούς.
\begin{multicols}{3}
\begin{rlist}
\item $ (-1)^7\ldots1^7 $
\item $ -2^{18}\ldots(-2)^{18} $
\item $ (-3)^9\ldots -3^9 $
\item $ |-(-8)|\ldots 8 $
\item $ \left( \frac{4}{5}\right)^3 \ldots\frac{16}{25} $
\item $ 3^4\cdot 3^7\ldots 3^5\cdot 3^6 $
\item $ (-2)^5\ldots |-2|^5 $
\item $ |-(-1)|^8\ldots 125:5^3 $
\item $ \ldots 125:5^3 $
\end{rlist}
\end{multicols}\monades{3}
\item Να γράψετε τις παρακάτω παραστάσεις ως μια δύναμη.
\begin{multicols}{4}
\begin{rlist}
\item $ \dfrac{3^7\cdot3^9\cdot(9\cdot 3)^4}{9^{12}\cdot 3^3} $
\item $ \dfrac{4^8\cdot 2^9\cdot 8^2}{16^3\cdot 4^7} $
\item $ \dfrac{\left( 5^2\right)^4\cdot(-5)^5}{5^{17}:5^9} $
\item $ \dfrac{2^7\cdot2^8\cdot 2^4}{2^9\cdot2^6\cdot2^3} $
\end{rlist}
\end{multicols}\monades{4}
\end{erwthma}
\item \mbox{}\\\vspace{-7mm}
\begin{erwthma}
\item Να υπολογίσετε την τιμή των παρακάτω παραστάσεων.
\begin{rlist}
\item $ |-(-2)^3|-\left( 3\cdot 4-20:|-5|\right) +(-2^3) $.
\item $ (-4)\cdot(-5)\cdot2-3\cdot(-7)\cdot(-1)+10 $.
\item $ -12+(-7)-(-5)+(+8)-12+(-9)-10 $.
\end{rlist}\monades{3}
\item Να υπολογίσετε την τιμή των παρακάτω παραστάσεων.
\begin{multicols}{2}
\begin{rlist}
\item $ -\dfrac{2}{5}\cdot\left(- \dfrac{3}{4}\right) +2\cdot\left( \dfrac{11}{8}-\dfrac{7}{4}\right) $.
\item $ \left( \dfrac{3}{2}\right)^2-\dfrac{1}{3}\cdot\left( 4-\dfrac{7}{5}\right)-1 $.
\item $ \dfrac{4}{5}\cdot\left( -3 +\dfrac{5}{4}\right)\cdot\left(- \dfrac{5}{7}\right) $.
\end{rlist}
\end{multicols}\monades{4}
\end{erwthma}
\end{thema}
\kaliepityxia
\end{document}

