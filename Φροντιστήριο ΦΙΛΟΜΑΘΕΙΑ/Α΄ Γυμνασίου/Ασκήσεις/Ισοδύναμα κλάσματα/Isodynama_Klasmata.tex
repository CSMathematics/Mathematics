\documentclass[11pt,a4paper,twocolumn]{article}
\usepackage[english,greek]{babel}
\usepackage[utf8]{inputenc}
\usepackage{nimbusserif}
\usepackage[T1]{fontenc}
\usepackage[left=1.50cm, right=1.50cm, top=2.00cm, bottom=2.00cm]{geometry}
\usepackage{amsmath}
\let\myBbbk\Bbbk
\let\Bbbk\relax
\usepackage[amsbb,subscriptcorrection,zswash,mtpcal,mtphrb,mtpfrak]{mtpro2}
\usepackage{graphicx,multicol,multirow,enumitem,tabularx,mathimatika,gensymb,venndiagram,hhline,longtable,tkz-euclide,fontawesome5,eurosym,tcolorbox,tabularray,tikzpagenodes,relsize}
\definecolor{xrwma}{HTML}{3d93cd}
\usetikzlibrary{calc}
\usetikzlibrary{positioning}
\usepackage[explicit]{titlesec}
\tcbuselibrary{skins,theorems,breakable}
\newlist{rlist}{enumerate}{3}
\setlist[rlist]{itemsep=0mm,label=\roman*.}
\newlist{alist}{enumerate}{3}
\setlist[alist]{itemsep=0mm,label=\alph*.}
\newlist{balist}{enumerate}{3}
\setlist[balist]{itemsep=0mm,label=\bf\alph*.}
\newlist{Alist}{enumerate}{3}
\setlist[Alist]{itemsep=0mm,label=\Alph*.}
\newlist{bAlist}{enumerate}{3}
\setlist[bAlist]{itemsep=0mm,label=\bf\Alph*.}
\newlist{askhseis}{enumerate}{3}
\setlist[askhseis]{label={\Large\thesection}.\arabic*.}
\renewcommand{\textstigma}{\textsigma\texttau}
\newlist{thema}{enumerate}{3}
\setlist[thema]{label=\bf\large{ΘΕΜΑ \textcolor{black}{\Alph*}},itemsep=0mm,leftmargin=0cm,itemindent=18mm}
\newlist{erwthma}{enumerate}{3}
\setlist[erwthma]{label=\bf{\large{\textcolor{black}{\Alph{themai}.\arabic*}}},itemsep=0mm,leftmargin=0.8cm}

\newcommand{\kerkissans}[1]{{\fontfamily{maksf}\selectfont \textbf{#1}}}
\renewcommand{\textdexiakeraia}{}

\usepackage[
backend=biber,
style=alphabetic,
sorting=ynt
]{biblatex}

\DeclareTblrTemplate{caption}{nocaptemplate}{}
\DeclareTblrTemplate{capcont}{nocaptemplate}{}
\DeclareTblrTemplate{contfoot}{nocaptemplate}{}
\NewTblrTheme{mytabletheme}{
\SetTblrTemplate{caption}{nocaptemplate}{}
\SetTblrTemplate{capcont}{nocaptemplate}{}
\SetTblrTemplate{contfoot}{nocaptemplate}{}
}

\NewTblrEnviron{mytblr}
\SetTblrStyle{firsthead}{font=\bfseries}
\SetTblrStyle{firstfoot}{fg=red2}
\SetTblrOuter[mytblr]{theme=mytabletheme}
\SetTblrInner[mytblr]{
rowspec={t{7mm}},columns = {c},
width = 0.85\linewidth,
row{odd} = {bg=red9,fg=black,ht=8mm},
row{even} = {bg=red7,fg=black,ht=8mm},
hlines={white},vlines={white},
row{1} = {bg=xrwma, fg=white, font=\bfseries\fontfamily{maksf}},rowhead = 1,
hline{2} = {.7mm}, % midrule  
}
\newcounter{askhsh}
\setcounter{askhsh}{1}
\newcommand{\askhsh}{{\large\theaskhsh.}\ \addtocounter{askhsh}{1}}

\titleformat{\section}{\Large}{\kerkissans{\thesection}}{10pt}{\Large\kerkissans{#1}}

\setlength{\columnsep}{5mm}
\titleformat{\paragraph}
{\large}%
{}{0em}%
{\textcolor{xrwma}{\faSquare\ \ \kerkissans{\bmath{#1}}}}
\setlength{\parindent}{0pt}
\titlespacing{\paragraph}{0mm}{2mm}{2mm}
\newcommand{\eng}[1]{\selectlanguage{english}#1\selectlanguage{greek}}

\begin{document}
\twocolumn[{
\begin{tikzpicture}[overlay,remember picture]
\fill[xrwma]($(current page.north west)$)--($(current page.north west)+(2cm,0)$)--($(current page.north west)+(2cm,-2.4cm)$)--($(current page.north west)+(-2cm,-2.4cm)$)--cycle;
%\node (fig1) at ($(current page.north west)+(1cm,-1.2cm)$)
%{\includegraphics[width=0.07\linewidth]{}};
\node[gray2] at ($(0,0)+(-2.4cm,1.5cm)$) {\kerkissans{Φ\,ΡΟΝΤΙΣΤΗΡΙΟ ΜΕΣΗΣ ΕΚΠΑΙΔΕΥΣΗΣ}};
\node[xrwma] at ($(0,0)+(-2.4cm,.9cm)$) {\kerkissans{\huge {\fontsize{28}{33.6}\selectfont Φ\,ΙΛΟΜΑΘΕΙΑ}}};
\draw[gray4] (-5.2,.4)--(.4,.4);
\node at ($(0,0)+(-2.4cm,0cm)$) {\textcolor{xrwma}{\faMapMarker*} Ιακώβου Πολυλά 24, Πεζόδρομος};
\node (title) at ($(current page.north east)+(-3.5cm,-.7cm)$){\textcolor{gray2}{ \kerkissans{\LARGE Φ\,ΥΛΛΑΔΙΟ ΑΣΚΗΣΕΩΝ}}};
\node[below=of title.east,anchor=east,yshift=3mm] (mob) {\textcolor{xrwma}{\faPhone*} 26610 20144 - \textcolor{xrwma}{\faMobile*\ \faTelegram\ \faViber} 693 232 7283};
\node[below=of mob.east,anchor=east,yshift=4mm] (fb) {\textcolor{xrwma}{\faFacebook} Φροντιστήριο Φιλομάθεια - \textcolor{xrwma}{\faInstagram}\ {\eng{front\_filomatheia}}};
\end{tikzpicture}
\vspace{10mm}\mbox{}\\
\centering
\kerkissans{{\huge Άλγεβρα - Α' Γυμνασίου}\\\vspace{2mm} {\LARGE Ισοδύναμα κλάσματα}}\\\vspace{4mm}{\kerkissans{\today}}\\\vspace{3mm}}]
\paragraph{}
\askhsh Να εξετάσετε αν τα παρακάτω κλάσματα είναι ίσα μεταξύ τους.
\begin{multicols}{3}
\begin{alist}
\item $\dfrac{3}{4}$ και $\dfrac{6}{8}$
\item $\dfrac{8}{5}$ και $\dfrac{9}{6}$
\item $\dfrac{12}{15}$ και $\dfrac{8}{10}$
\item $\dfrac{20}{15}$ και $\dfrac{9}{5}$
\item $\dfrac{28}{32}$ και $\dfrac{14}{16}$
\item $\dfrac{40}{56}$ και $\dfrac{15}{21}$
\item $\dfrac{18}{24}$ και $\dfrac{20}{28}$
\item $\dfrac{7}{9}$ και $\dfrac{21}{27}$
\item $\dfrac{5}{2}$ και $\dfrac{60}{24}$
\end{alist}
\end{multicols}
\askhsh Να μετατρέψετε τα παρακάτω κλάσματα σε ομώνυμα.
\begin{multicols}{2}
\begin{alist}
\item $\dfrac{8}{9}$ και $\dfrac{7}{6}$
\item $\dfrac{3}{2}$ και $\dfrac{4}{5}$
\item $\dfrac{5}{4}$ και $\dfrac{1}{8}$
\item $\dfrac{7}{18}$ και $\dfrac{11}{24}$
\item $\dfrac{1}{10},\dfrac{2}{5}$ και $\dfrac{6}{15}$
\item $\dfrac{15}{8},\dfrac{5}{12}$ και $\dfrac{9}{16}$
\item $\dfrac{3}{25},\dfrac{8}{15}$ και $\dfrac{11}{20}$
\end{alist}
\end{multicols}
\end{document}
