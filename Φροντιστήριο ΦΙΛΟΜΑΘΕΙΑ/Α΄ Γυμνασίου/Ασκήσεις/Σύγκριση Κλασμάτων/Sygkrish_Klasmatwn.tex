\PassOptionsToPackage{no-math,cm-default}{fontspec}
\documentclass[twoside,nofonts,internet]{askhseis}
\usepackage{amsmath}
\usepackage{xgreek}
\let\hbar\relax
\defaultfontfeatures{Mapping=tex-text,Scale=MatchLowercase}
\setmainfont[Mapping=tex-text,Numbers=Lining,Scale=1.0,BoldFont={Minion Pro Bold}]{Minion Pro}
\newfontfamily\scfont{GFS Artemisia}
\font\icon = "Webdings"
\usepackage[amsbb,subscriptcorrection,zswash,mtpcal,mtphrb]{mtpro2}
\xroma{red!70!black}
%------TIKZ - ΣΧΗΜΑΤΑ - ΓΡΑΦΙΚΕΣ ΠΑΡΑΣΤΑΣΕΙΣ ----
\usepackage{tikz}
\usepackage{tkz-euclide}
\usetkzobj{all}
\usepackage[framemethod=TikZ]{mdframed}
\usetikzlibrary{decorations.pathreplacing}
\usepackage{pgfplots}
\usetkzobj{all}
%-----------------------
\usepackage{calc}
\usepackage{hhline}
\usepackage[explicit]{titlesec}
\usepackage{graphicx}
\usepackage{multicol}
\usepackage{multirow}
\usepackage{tabularx}
\usepackage[decimalsymbol=comma]{siunitx}
\usetikzlibrary{backgrounds}
\usepackage{sectsty}
\sectionfont{\centering}
\usepackage{enumitem}
\setlist[enumerate]{label=\bf{\large \arabic*.}}
\usepackage{adjustbox}
\usepackage{mathimatika,gensymb,eurosym,wrap-rl}
\usepackage{systeme,regexpatch}
%-------- ΜΑΘΗΜΑΤΙΚΑ ΕΡΓΑΛΕΙΑ ---------
\usepackage{mathtools}
%----------------------
%-------- ΠΙΝΑΚΕΣ ---------
\usepackage{booktabs}
%----------------------
%----- ΥΠΟΛΟΓΙΣΤΗΣ ----------
\usepackage{calculator}
%----------------------------
%------ ΔΙΑΓΩΝΙΟ ΣΕ ΠΙΝΑΚΑ -------
\usepackage{array}
\newcommand\diag[5]{%
\multicolumn{1}{|m{#2}|}{\hskip-\tabcolsep
$\vcenter{\begin{tikzpicture}[baseline=0,anchor=south west,outer sep=0]
\path[use as bounding box] (0,0) rectangle (#2+2\tabcolsep,\baselineskip);
\node[minimum width={#2+2\tabcolsep-\pgflinewidth},
minimum  height=\baselineskip+#3-\pgflinewidth] (box) {};
\draw[line cap=round] (box.north west) -- (box.south east);
\node[anchor=south west,align=left,inner sep=#1] at (box.south west) {#4};
\node[anchor=north east,align=right,inner sep=#1] at (box.north east) {#5};
\end{tikzpicture}}\rule{0pt}{.71\baselineskip+#3-\pgflinewidth}$\hskip-\tabcolsep}}
%---------------------------------
%---- ΟΡΙΖΟΝΤΙΟ - ΚΑΤΑΚΟΡΥΦΟ - ΠΛΑΓΙΟ ΑΓΚΙΣΤΡΟ ------
\newcommand{\orag}[3]{\node at (#1)
{$ \overcbrace{\rule{#2mm}{0mm}}^{{\scriptsize #3}} $};}
\newcommand{\kag}[3]{\node at (#1)
{$ \undercbrace{\rule{#2mm}{0mm}}_{{\scriptsize #3}} $};}
\newcommand{\Pag}[4]{\node[rotate=#1] at (#2)
{$ \overcbrace{\rule{#3mm}{0mm}}^{{\rotatebox{-#1}{\scriptsize$#4$}}}$};}
%-----------------------------------------


%------------------------------------------
\newcommand{\tss}[1]{\textsuperscript{#1}}
\newcommand{\tssL}[1]{\MakeLowercase{\textsuperscript{#1}}}
%---------- ΛΙΣΤΕΣ ----------------------
\newlist{bhma}{enumerate}{3}
\setlist[bhma]{label=\bf\textit{\arabic*\textsuperscript{o}\;Βήμα :},leftmargin=0cm,itemindent=1.8cm,ref=\bf{\arabic*\textsuperscript{o}\;Βήμα}}
\newlist{brlist}{enumerate}{3}
\setlist[brlist]{itemsep=0mm,label=\bf\roman*.}
\newlist{tropos}{enumerate}{3}
\setlist[tropos]{label=\bf\textit{\arabic*\textsuperscript{oς}\;Τρόπος :},leftmargin=0cm,itemindent=2.3cm,ref=\bf{\arabic*\textsuperscript{oς}\;Τρόπος}}
% Αν μπει το bhma μεσα σε tropo τότε
%\begin{bhma}[leftmargin=.7cm]
\tkzSetUpPoint[size=7,fill=white]
\tikzstyle{pl}=[line width=0.3mm]
\tikzstyle{plm}=[line width=0.4mm]
\usepackage{etoolbox}
\makeatletter
\renewrobustcmd{\anw@true}{\let\ifanw@\iffalse}
\renewrobustcmd{\anw@false}{\let\ifanw@\iffalse}\anw@false
\newrobustcmd{\noanw@true}{\let\ifnoanw@\iffalse}
\newrobustcmd{\noanw@false}{\let\ifnoanw@\iffalse}\noanw@false
\renewrobustcmd{\anw@print}{\ifanw@\ifnoanw@\else\numer@lsign\fi\fi}
\makeatother



\begin{document}
\titlos{Α΄ Γυμνασίου}{Κλάσματα}{Σύγκριση κλασμάτων}
\twocolkentro{\thewria}
\begin{enumerate}
\item 
\end{enumerate}
\twocolkentro{\askhseis}
\begin{enumerate}
\item \textbf{Σύγκριση κλασμάτων}\\
Να συγκρίνετε τα παρακάτω κλάσματα μεταξύ τους βάζοντας το σωστό σύμβολο διάταξης ανάμεσά τους $ (<,\ =\ \ \textrm{ή}\ >) $.
\begin{multicols}{3}
\begin{rlist}[leftmargin=3mm]
\item $ \dfrac{4}{3}\ldots \dfrac{7}{3} $
\item $ \dfrac{9}{7}\ldots \dfrac{10}{7} $
\item $ \dfrac{15}{12}\ldots \dfrac{13}{12} $
\item $ \dfrac{4}{5}\ldots \dfrac{4}{5} $
\end{rlist}
\end{multicols}
\item \textbf{Σύγκριση κλασμάτων}\\
Να συγκρίνετε τα παρακάτω κλάσματα μεταξύ τους βάζοντας το σωστό σύμβολο διάταξης ανάμεσά τους $ (<,\ =\ \ \textrm{ή}\ >) $.
\begin{multicols}{3}
\begin{rlist}[leftmargin=3mm]
\item $ \dfrac{4}{5}\ldots \dfrac{4}{3} $
\item $ \dfrac{9}{7}\ldots \dfrac{9}{11} $
\item $ \dfrac{7}{13}\ldots \dfrac{12}{13} $
\item $ \dfrac{5}{21}\ldots \dfrac{21}{21} $
\end{rlist}
\end{multicols}
\item \textbf{Σύγκριση κλασμάτων}\\
Να συγκρίνετε τα παρακάτω κλάσματα μεταξύ τους βάζοντας το σωστό σύμβολο διάταξης ανάμεσά τους $ (<,\ =\ \ \textrm{ή}\ >) $.
\begin{multicols}{3}
\begin{rlist}[leftmargin=3mm]
\item $ \dfrac{3}{2}\ldots \dfrac{4}{3} $
\item $ \dfrac{7}{4}\ldots \dfrac{9}{6} $
\item $ \dfrac{15}{16}\ldots \dfrac{12}{11} $
\item $ \dfrac{8}{5}\ldots \dfrac{16}{10} $
\end{rlist}
\end{multicols}
\item \textbf{Σύγκριση κλασμάτων με τη μονάδα}\\
Να συγκρίνετε τα παρακάτω κλάσματα με τη μονάδα βάζοντας το σωστό σύμβολο διάταξης μεταξύ τους $ (<,\ =\ \ \textrm{ή}\ >) $.
\begin{multicols}{4}
\begin{rlist}[leftmargin=5mm]
\item $ \dfrac{3}{4} $
\item $ \dfrac{5}{3} $
\item $ \dfrac{7}{8} $
\item $ \dfrac{11}{9} $
\item $ \dfrac{17}{17} $
\item $ \dfrac{35}{35} $
\item $ \dfrac{89}{88} $
\item $ \dfrac{1001}{1010} $
\end{rlist}
\end{multicols}
\item \textbf{Σύγκριση κλασμάτων με φυσικό αριθμό}\\
Να συγκρίνετε τους παρακάτω αριθμούς μεταξύ τους βάζοντας το σωστό σύμβολο διάταξης ανάμεσά τους $ (<,\ =\ \ \textrm{ή}\ >) $.
\begin{multicols}{3}
\begin{rlist}[leftmargin=3mm]
\item $ \dfrac{4}{3}\ldots 2 $
\item $ 4\ldots \dfrac{10}{3} $
\item $ \dfrac{15}{3}\ldots 5 $
\item $ 3\dfrac{4}{5}\ldots 1 $
\item $ \dfrac{19}{4}\ldots 5 $
\item $ 7\ldots \dfrac{123}{24} $
\end{rlist}
\end{multicols}
\item \textbf{Σύγκριση κλασμάτων με μεικτό αριθμό}\\
Να συγκρίνετε τους παρακάτω αριθμούς μεταξύ τους βάζοντας το σωστό σύμβολο διάταξης ανάμεσά τους $ (<,\ =\ \ \textrm{ή}\ >) $.
\begin{multicols}{3}
\begin{rlist}[leftmargin=3mm]
\item $ \dfrac{3}{2}\ldots 1\dfrac{3}{4} $
\item $ 4\dfrac{1}{2}\ldots \dfrac{13}{3} $
\item $ \dfrac{17}{3}\ldots 5\dfrac{2}{3} $
\item $ 3\dfrac{4}{5}\ldots \dfrac{21}{4} $
\item $ \dfrac{19}{4}\ldots 5\dfrac{1}{12} $
\item $ 2\dfrac{12}{13}\ldots \dfrac{37}{18} $
\end{rlist}
\end{multicols}
\item \textbf{Σύγκριση μεικτών αριθμών}\\
Να συγκρίνετε τους παρακάτω αριθμούς μεταξύ τους βάζοντας το σωστό σύμβολο διάταξης ανάμεσά τους $ (<,\ =\ \ \textrm{ή}\ >) $.
\begin{multicols}{3}
\begin{rlist}[leftmargin=3mm]
\item $ 2\dfrac{3}{4}\ldots 2\dfrac{5}{7} $
\end{rlist}
\end{multicols}
\item \textbf{Διάταξη κλασμάτων}\\
Να τοποθετήσετε τα παρακάτω κλάσματα στη σειρά από το μικρότερο στο μεγαλύτερο.
\[ \frac{3}{8}\ ,\ \frac{9}{8}\ ,\ \frac{4}{8}\ ,\ \frac{5}{8}\ ,\ \dfrac{11}{8} \]
\item \textbf{Διάταξη κλασμάτων}\\
Να τοποθετήσετε τα παρακάτω κλάσματα στη σειρά από το μικρότερο στο μεγαλύτερο.
\[ \frac{7}{4}\ ,\ \frac{7}{8}\ ,\ \frac{7}{5}\ ,\ \frac{7}{2}\ ,\ \dfrac{7}{12} \]
\item \textbf{Διάταξη κλασμάτων}\\
Να τοποθετήσετε τα παρακάτω κλάσματα στη σειρά από το μικρότερο στο μεγαλύτερο.
\[ \frac{4}{3}\ ,\ \frac{5}{9}\ ,\ \frac{14}{16}\ ,\ \frac{7}{9}\ ,\ \dfrac{11}{12} \]
\item \textbf{Διάταξη κλασμάτων}\\
Να τοποθετήσετε τους παρακάτω αριθμούς στη σειρά από το μικρότερο στο μεγαλύτερο.
\[ 3\frac{5}{8}\ ,\ \frac{23}{6}\ ,\ 2\frac{7}{12}\ ,\ \frac{89}{24}\ ,\ \dfrac{27}{8} \]
\end{enumerate}
\end{document}

