\documentclass[11pt,a4paper,modern]{FFExercises}
\usepackage[english,greek]{babel}
\usepackage[utf8]{inputenc}
\usepackage{nimbusserif}
\usepackage[T1]{fontenc}
\usepackage{amsmath}
\let\myBbbk\Bbbk
\let\Bbbk\relax
\usepackage[amsbb,subscriptcorrection,zswash,mtpcal,mtphrb,mtpfrak]{mtpro2}
\usepackage{graphicx,multicol,multirow,enumitem,tabularx,mathimatika,gensymb,venndiagram,hhline,longtable,tkz-euclide,fontawesome5,eurosym,tcolorbox,tabularray,tikzpagenodes,relsize}
\definecolor{xrwma}{HTML}{aa1212}
\usetikzlibrary{calc}
\usetikzlibrary{positioning}
\tcbuselibrary{skins,theorems,breakable}
\renewcommand{\textstigma}{\textsigma\texttau}
\renewcommand{\textdexiakeraia}{}

\ekthetesdeiktes
\begin{document}

\titlos{Μαθηματικά Α΄ Γυμνασίου}{Κλάσματα}{Πρόσθεση και αφαίρεση κλασμάτων}
\paragraph{Πρόσθεση - Αφαίρεση κλασμάτων}
\askhsh
Να υπολογίσετε τα παρακάτω αθροίσματα.
\begin{multicols}{2}
\begin{alist}
\item $ \dfrac{3}{2}+\dfrac{5}{2} $
\item $ \dfrac{4}{7}+\dfrac{12}{7} $
\item $ \dfrac{15}{8}+\dfrac{25}{8} $
\item $ \dfrac{4}{9}+\dfrac{2}{9}+\dfrac{3}{9} $
\end{alist}
\end{multicols}
\askhsh
Να υπολογίσετε τα παρακάτω αθροίσματα.
\begin{multicols}{2}
\begin{alist}
\item $ \dfrac{3}{5}+\dfrac{4}{3} $
\item $ \dfrac{7}{6}+\dfrac{5}{2} $
\item $ \dfrac{12}{20}+\dfrac{7}{24} $
\item $ \dfrac{5}{4}+\dfrac{6}{8}+\dfrac{9}{12} $
\end{alist}
\end{multicols}
\askhsh
Να υπολογίσετε τα παρακάτω αθροίσματα.
\begin{multicols}{2}
\begin{alist}
\item $ 3+\dfrac{5}{4} $
\item $ \dfrac{9}{5}+2 $
\item $ \dfrac{11}{18}+1 $
\item $ 3+\dfrac{4}{5}+\dfrac{2}{3} $
\end{alist}
\end{multicols}
\askhsh
Να υπολογίσετε τις παρακάτω διαφορές.
\begin{multicols}{2}
\begin{alist}
\item $ \dfrac{7}{2}-\dfrac{3}{2} $
\item $ \dfrac{12}{5}-\dfrac{4}{5} $
\item $ \dfrac{13}{8}-\dfrac{8}{8} $
\item $ \dfrac{5}{9}-\dfrac{2}{9} $
\end{alist}
\end{multicols}
\askhsh
Να υπολογίσετε τις παρακάτω διαφορές.
\begin{multicols}{2}
\begin{alist}
\item $ \dfrac{7}{5}-\dfrac{3}{15} $
\item $ \dfrac{8}{6}-\dfrac{5}{8} $
\item $ \dfrac{13}{18}-\dfrac{4}{16} $
\item $ \dfrac{27}{35}-\dfrac{18}{50} $
\end{alist}
\end{multicols}
\askhsh
Να υπολογίσετε τις παρακάτω διαφορές.
\begin{multicols}{2}
\begin{alist}
\item $ 1-\dfrac{3}{4} $
\item $ \dfrac{15}{6}-2 $
\item $ 4-\dfrac{17}{28} $
\item $ 14-\dfrac{130}{12} $
\end{alist}
\end{multicols}
\paragraph{Μεικτοί αριθμοί}
\askhsh
Να μετατρέψετε τα παρακάτω κλάσματα σε μεικτούς αριθμούς.
\begin{multicols}{4}
\begin{alist}
\item $ \dfrac{12}{5} $
\item $ \dfrac{34}{8} $
\item $ \dfrac{27}{4} $
\item $ \dfrac{59}{12} $
\end{alist}
\end{multicols}
\askhsh
Να μετατρέψετε τους παρακάτω μεικτούς αριθμούς σε κλάσματα.
\begin{multicols}{4}
\begin{alist}
\item $ 4\dfrac{2}{5} $
\item $ 1\dfrac{5}{8} $
\item $ 5\dfrac{4}{9} $
\item $ 7\dfrac{8}{15} $
\end{alist}
\end{multicols}
\paragraph{Αριθμητικές παραστάσεις}
\askhsh
Να υπολογίσετε την τιμή των παρακάτω αθροισμάτων.
\begin{multicols}{2}
\begin{alist}
\item $ 3\dfrac{4}{5}+\dfrac{2}{3} $
\item $ 2+4\dfrac{5}{7} $
\item $ 1\dfrac{2}{3}+4\dfrac{4}{5} $
\item $ 3+\dfrac{4}{5}+1\dfrac{5}{6} $
\end{alist}
\end{multicols}
\askhsh
Να υπολογίσετε την τιμή των παρακάτω διαφορών.
\begin{multicols}{2}
\begin{alist}
\item $ 4\dfrac{2}{3}-\dfrac{2}{3} $
\item $ 4-1\dfrac{2}{6} $
\item $ 7\dfrac{2}{3}-4\dfrac{5}{12} $
\item $ 12\dfrac{5}{8}-11\dfrac{5}{6} $
\end{alist}
\end{multicols}
\askhsh
Να υπολογίσετε την τιμή των παρακάτω αριθμητικών παραστάσεων.
\begin{multicols}{2}
\begin{alist}
\item $ \dfrac{4}{3}+\dfrac{5}{6}-\dfrac{1}{12} $
\item $ \dfrac{7}{8}-\dfrac{14}{12}+\dfrac{5}{6} $
\item $ \dfrac{28}{15}-\dfrac{3}{10}-\dfrac{1}{5} $
\item $ \dfrac{24}{60}-\dfrac{15}{40}+\dfrac{27}{50} $
\end{alist}
\end{multicols}
\askhsh
Να υπολογίσετε την τιμή των παρακάτω αριθμητικών παραστάσεων.
\begin{multicols}{2}
\begin{alist}
\item $ 3+\dfrac{3}{2}-\dfrac{4}{3} $
\item $ \dfrac{5}{4}-1+\dfrac{7}{6} $
\item $ \dfrac{32}{12}-\dfrac{9}{8}-5 $
\item $ \dfrac{40}{35}+7-\dfrac{34}{50} $
\end{alist}
\end{multicols}
\askhsh
Να υπολογίσετε την τιμή των παρακάτω αριθμητικών παραστάσεων.
\begin{multicols}{2}
\begin{alist}
\item $ 2\dfrac{4}{5}-\dfrac{3}{5}+\dfrac{5}{6} $
\item $ 4\dfrac{1}{4}-1\dfrac{3}{8}+\dfrac{7}{12} $
\item $ 5-2\dfrac{7}{8}+\dfrac{3}{4} $
\item $ 1\dfrac{14}{30}-2\dfrac{4}{45}+1\dfrac{17}{60} $
\end{alist}
\end{multicols}
\paragraph{Σύγκριση}
\askhsh
Να συγκρίνετε τους παρακάτω αριθμούς μεταξύ τους βάζοντας το σωστό σύμβολο διάταξης ανάμεσά τους $ (<,\ =\ \ \textrm{ή}\ >) $.
\begin{multicols}{3}
\begin{alist}
\item $ \dfrac{4}{3}\ldots 2 $
\item $ 4\ldots \dfrac{10}{3} $
\item $ \dfrac{15}{3}\ldots 5 $
\item $ 3\dfrac{4}{5}\ldots \dfrac{16}{5} $
\item $ \dfrac{19}{4}\ldots 4\dfrac{3}{4} $
\item $ \dfrac{47}{25}\ldots 1\dfrac{23}{25} $
\end{alist}
\end{multicols}
\askhsh
Να τοποθετήσετε τους παρακάτω αριθμούς στη σειρά από το μικρότερο στο μεγαλύτερο.
\[ 3\frac{5}{8}\ ,\ \frac{23}{6}\ ,\ 2\frac{7}{12}\ ,\ \frac{89}{24}\ ,\ \dfrac{27}{8} \]
\end{document}
