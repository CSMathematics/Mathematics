\PassOptionsToPackage{no-math,cm-default}{fontspec}
\documentclass[twoside,nofonts,internet]{askhseis}
\usepackage{amsmath}
\usepackage{xgreek}
\let\hbar\relax
\defaultfontfeatures{Mapping=tex-text,Scale=MatchLowercase}
\setmainfont[Mapping=tex-text,Numbers=Lining,Scale=1.0,BoldFont={Minion Pro Bold}]{Minion Pro}
\newfontfamily\scfont{GFS Artemisia}
\font\icon = "Webdings"
\usepackage[amsbb]{mtpro2}
\usepackage{tikz,pgfplots}
\tkzSetUpPoint[size=7,fill=white]
\xroma{red!70!black}
%------TIKZ - ΣΧΗΜΑΤΑ - ΓΡΑΦΙΚΕΣ ΠΑΡΑΣΤΑΣΕΙΣ ----
\usepackage{tikz}
\usepackage{tkz-euclide}
\usetkzobj{all}
\usepackage[framemethod=TikZ]{mdframed}
\usetikzlibrary{decorations.pathreplacing}
\usepackage{pgfplots}
\usetkzobj{all}
%-----------------------

%-----ΕΙΚΟΝΑ ΔΙΠΛΑ ΑΠΟ ΚΕΙΜΕΝΟ-------
\usepackage{wrapfig}
\newenvironment{WrapText1}[3][r]
{\wrapfigure[#2]{#1}{#3}}
{\endwrapfigure}

\newenvironment{WrapText2}[3][l]
{\wrapfigure[#2]{#1}{#3}}
{\endwrapfigure}

\newcommand{\wrapr}[6]{
\begin{minipage}{\linewidth}\mbox{}\\
\vspace{#1}
\begin{WrapText1}{#2}{#3}
\vspace{#4}#5\end{WrapText1}#6
\end{minipage}}

\newcommand{\wrapl}[6]{
\begin{minipage}{\linewidth}\mbox{}\\
\vspace{#1}
\begin{WrapText2}{#2}{#3}
\vspace{#4}#5\end{WrapText2}#6
\end{minipage}}
%-------------------------------------------

\usepackage{calc}
\usepackage{hhline}
\renewcommand{\thepart}{\arabic{part}}

\usepackage[explicit]{titlesec}
\usepackage{graphicx}
\usepackage{multicol}
\usepackage{multirow}
\usepackage{enumitem}
\usepackage{tabularx}
\usepackage[decimalsymbol=comma]{siunitx}
\usetikzlibrary{backgrounds}
\usepackage{sectsty}
\sectionfont{\centering}
\usepackage{enumitem}
\setlist[enumerate]{label=\bf{\large \arabic*.}}
\usepackage{adjustbox}
%--------- ΑΓΓΛΙΚΟ ΚΕΙΜΕΝΟ --------------
\newcommand{\eng}[1]{\selectlanguage{english}#1\selectlanguage{greek}}
%----------------------------------------
%------- ΣΥΣΤΗΜΑ -------------------
\usepackage{systeme,regexpatch}
\makeatletter
% change the definition of \sysdelim not to store `\left` and `\right`
\def\sysdelim#1#2{\def\SYS@delim@left{#1}\def\SYS@delim@right{#2}}
\sysdelim\{. % reinitialize

% patch the internal command to use
% \LEFTRIGHT<left delim><right delim>{<system>}
% instead of \left<left delim<system>\right<right delim>
\regexpatchcmd\SYS@systeme@iii
{\cB.\c{SYS@delim@left}(.*)\c{SYS@delim@right}\cE.}
{\c{SYS@MT@LEFTRIGHT}\cB\{\1\cE\}}
{}{}
\def\SYS@MT@LEFTRIGHT{%
\expandafter\expandafter\expandafter\LEFTRIGHT
\expandafter\SYS@delim@left\SYS@delim@right}
\makeatother
\newcommand{\synt}[2]{{\scriptsize \begin{matrix}
\times#1\\\\ \times#2
\end{matrix}}}
%----------------------------------------
%------ ΜΗΚΟΣ ΓΡΑΜΜΗΣ ΚΛΑΣΜΑΤΟΣ ---------
\DeclareRobustCommand{\frac}[3][0pt]{%
{\begingroup\hspace{#1}#2\hspace{#1}\endgroup\over\hspace{#1}#3\hspace{#1}}}
%----------------------------------------
%-------- ΜΑΘΗΜΑΤΙΚΑ ΕΡΓΑΛΕΙΑ ---------
\usepackage{mathtools}
%----------------------

%-------- ΠΙΝΑΚΕΣ ---------
\usepackage{booktabs}
%----------------------
%----- ΥΠΟΛΟΓΙΣΤΗΣ ----------
\usepackage{calculator}
%----------------------------
%------ ΔΙΑΓΩΝΙΟ ΣΕ ΠΙΝΑΚΑ -------
\usepackage{array}
\newcommand\diag[5]{%
\multicolumn{1}{|m{#2}|}{\hskip-\tabcolsep
$\vcenter{\begin{tikzpicture}[baseline=0,anchor=south west,outer sep=0]
\path[use as bounding box] (0,0) rectangle (#2+2\tabcolsep,\baselineskip);
\node[minimum width={#2+2\tabcolsep-\pgflinewidth},
minimum  height=\baselineskip+#3-\pgflinewidth] (box) {};
\draw[line cap=round] (box.north west) -- (box.south east);
\node[anchor=south west,align=left,inner sep=#1] at (box.south west) {#4};
\node[anchor=north east,align=right,inner sep=#1] at (box.north east) {#5};
\end{tikzpicture}}\rule{0pt}{.71\baselineskip+#3-\pgflinewidth}$\hskip-\tabcolsep}}
%---------------------------------

%---- ΟΡΙΖΟΝΤΙΟ - ΚΑΤΑΚΟΡΥΦΟ - ΠΛΑΓΙΟ ΑΓΚΙΣΤΡΟ ------
\newcommand{\orag}[3]{\node at (#1)
{$ \overcbrace{\rule{#2mm}{0mm}}^{{\scriptsize #3}} $};}

\newcommand{\kag}[3]{\node at (#1)
{$ \undercbrace{\rule{#2mm}{0mm}}_{{\scriptsize #3}} $};}

\newcommand{\Pag}[4]{\node[rotate=#1] at (#2)
{$ \overcbrace{\rule{#3mm}{0mm}}^{{\rotatebox{-#1}{\scriptsize$#4$}}}$};}
%-----------------------------------------

%-------- ΤΡΙΓΩΝΟΜΕΤΡΙΚΟΙ ΑΡΙΘΜΟΙ -----------
\newcommand{\hm}[1]{\textrm{ημ}#1}
\newcommand{\syn}[1]{\textrm{συν}#1}
\newcommand{\ef}[1]{\textrm{εφ}#1}
\newcommand{\syf}[1]{\textrm{σφ}#1}
%--------------------------------------------

%--------- ΠΟΣΟΣΤΟ ΤΟΙΣ ΧΙΛΙΟΙΣ ------------
\DeclareRobustCommand{\perthousand}{%
\ifmmode
\text{\textperthousand}%
\else
\textperthousand
\fi}
%------------------------------------------

%------------------------------------------
\usepackage{extarrows}
\newcommand{\eq}[1]{\xlongequal{#1}}
\newcommand{\eqq}[2]{\xlongequal[#2]{#1}}
\DeclareMathOperator*{\Eq}{=}
%------------------------------------------
%------ ΌΡΙΣΜΑ ----------
\newcommand{\Arg}[8]{
\draw[-latex] (#7,#8)-- ++(#1:#2) node[right=#5]{\footnotesize$#4$};
\draw[fill=black!#6] (#7+0.3+#3,#8) arc (0:#1:0.3+#3) -- (#7,#8);}
%------------------------



%--------- ΑΝΙΣΩΣΕΙΣ -------
\tikzset{
thickest/.style={line width=1mm,steelblue},
a/.style={decoration={markings,
mark=at position #1 with {\fill[white,draw=black,thin] circle (3pt);}},postaction={decorate}},
k/.style={decoration={markings,
mark=at position #1 with {\fill[black] circle (3pt);}},postaction={decorate}},
}
%--------- ΔΙΑΣΤΗΜΑ ------------
\newcommand{\diasthma}[7]{
\foreach \x in {#3,#4}
\draw (\x,#7+.2) -- (\x,#7-.2);
\node[anchor=north,fill=white] at (#3,#7)[below=1mm] {$ #1 $};
\node[anchor=north,fill=white] at (#4,#7)[below=1mm] {$ #2 $};
\draw [#5=0,#6=1,thickest] (#3,#7)--(#4,#7);
}
%--------- ΑΞΟΝΑΣ ------------------
\newcommand{\axonas}[3]{
\draw[-latex] (#1,#3) -- (#2,#3)node[below]{$x$};
}
%--------- ΚΑΤΩ ΑΚΡΟ ------------------
\newcommand{\Xapeiro}[5]{
\draw (#2,#5+.2) -- (#2,#5-.2);
\node[anchor=north,fill=white] at (#2,#5)[below=1mm] {$ #1 $};
\draw [#4=0,thickest] (#2,#5)--(#3-.3,#5);
}
%--------- ΠΑΝΩ ΑΚΡΟ ------------------
\newcommand{\apeiroX}[5]{
\draw (#2,#5+.2) -- (#2,#5-.2);
\node[anchor=north,fill=white] at (#2,#5)[below=1mm] {$ #1 $};
\draw [#4=0,thickest] (#2,#5)--(#3+.3,#5);
}
%----- ΔΙΑΚΕΚΟΜΜΕΝΕΣ ΓΡΑΜΜΕΣ ------
\newcommand{\oria}[3]{\draw [dashed] (#1,#2)--(#1,#3);}
%--------------------------------------
\newcommand{\tss}[1]{\textsuperscript{#1}}
\newcommand{\tssL}[1]{\MakeLowercase{\textsuperscript{#1}}}
%---------- ΛΙΣΤΕΣ ----------------------
\newlist{bhma}{enumerate}{3}
\setlist[bhma]{label=\bf\textit{\arabic*\textsuperscript{o}\;Βήμα :},leftmargin=0cm,itemindent=1.8cm,ref=\bf{\arabic*\textsuperscript{o}\;Βήμα}}
\newlist{tropos}{enumerate}{3}
\setlist[tropos]{label=\bf\textit{\arabic*\textsuperscript{oς}\;Τρόπος :},leftmargin=0cm,itemindent=2.3cm,ref=\bf{\arabic*\textsuperscript{oς}\;Τρόπος}}
% Αν μπει το bhma μεσα σε tropo τότε
%\begin{bhma}[leftmargin=.7cm]
\tkzSetUpPoint[size=7,fill=white]
\tikzstyle{pl}=[line width=0.3mm]
\tikzstyle{plm}=[line width=0.4mm]




\begin{document}
\titlos{Μαθηματικά Α΄ Γυμνασίου}{Κλάσματα}{Πρόσθεση και αφαίρεση κλασμάτων}
\twocolkentro{\thewria}
\begin{enumerate}
\item 
\end{enumerate}
\twocolkentro{\askhseis}
\begin{enumerate}
\item \textbf{Πρόσθεση κλασμάτων}\\
Να υπολογίσετε τα παρακάτω αθροίσματα.
\begin{multicols}{2}
\begin{rlist}
\item $ \dfrac{3}{2}+\dfrac{5}{2} $
\item $ \dfrac{4}{7}+\dfrac{12}{7} $
\item $ \dfrac{15}{8}+\dfrac{25}{8} $
\item $ \dfrac{4}{9}+\dfrac{2}{9}+\dfrac{3}{9} $
\end{rlist}
\end{multicols}
\item \textbf{Πρόσθεση κλασμάτων}\\
Να υπολογίσετε τα παρακάτω αθροίσματα.
\begin{multicols}{2}
\begin{rlist}
\item $ \dfrac{3}{5}+\dfrac{4}{3} $
\item $ \dfrac{7}{6}+\dfrac{5}{2} $
\item $ \dfrac{12}{20}+\dfrac{7}{24} $
\item $ \dfrac{5}{4}+\dfrac{6}{8}+\dfrac{9}{12} $
\end{rlist}
\end{multicols}
\item \textbf{Πρόσθεση κλασμάτων}\\
Να υπολογίσετε τα παρακάτω αθροίσματα.
\begin{multicols}{2}
\begin{rlist}
\item $ 3+\dfrac{5}{4} $
\item $ \dfrac{9}{5}+2 $
\item $ \dfrac{11}{18}+1 $
\item $ 3+\dfrac{4}{5}+\dfrac{2}{3} $
\end{rlist}
\end{multicols}
\item \textbf{Αφαίρεση κλασμάτων}\\
Να υπολογίσετε τις παρακάτω διαφορές.
\begin{multicols}{2}
\begin{rlist}
\item $ \dfrac{7}{2}-\dfrac{3}{2} $
\item $ \dfrac{12}{5}-\dfrac{4}{5} $
\item $ \dfrac{13}{8}-\dfrac{8}{8} $
\item $ \dfrac{5}{9}-\dfrac{2}{9} $
\end{rlist}
\end{multicols}
\item \textbf{Αφαίρεση κλασμάτων}\\
Να υπολογίσετε τις παρακάτω διαφορές.
\begin{multicols}{2}
\begin{rlist}
\item $ \dfrac{7}{5}-\dfrac{3}{15} $
\item $ \dfrac{8}{6}-\dfrac{5}{8} $
\item $ \dfrac{13}{18}-\dfrac{4}{16} $
\item $ \dfrac{27}{35}-\dfrac{18}{50} $
\end{rlist}
\end{multicols}
\item \textbf{Αφαίρεση κλασμάτων}\\
Να υπολογίσετε τις παρακάτω διαφορές.
\begin{multicols}{2}
\begin{rlist}
\item $ 1-\dfrac{3}{4} $
\item $ \dfrac{15}{6}-2 $
\item $ 4-\dfrac{17}{28} $
\item $ 14-\dfrac{130}{12} $
\end{rlist}
\end{multicols}
\item \textbf{Μετατροπή κλάσματος σε μεικτό}\\
Να μετατρέψετε τα παρακάτω κλάσματα σε μεικτούς αριθμούς.
\begin{multicols}{4}
\begin{rlist}
\item $ \dfrac{12}{5} $
\item $ \dfrac{34}{8} $
\item $ \dfrac{27}{4} $
\item $ \dfrac{59}{12} $
\end{rlist}
\end{multicols}
\item \textbf{Μετατροπή μεικτού αριθμού σε κλάσμα}\\
Να μετατρέψετε τους παρακάτω μεικτούς αριθμούς σε κλάσματα.
\begin{multicols}{4}
\begin{rlist}
\item $ 4\dfrac{2}{5} $
\item $ 1\dfrac{5}{8} $
\item $ 5\dfrac{4}{9} $
\item $ 7\dfrac{8}{15} $
\end{rlist}
\end{multicols}
\item \textbf{Πρόσθεση κλασμάτων}\\
Να υπολογίσετε την τιμή των παρακάτω αθροισμάτων.
\begin{multicols}{2}
\begin{rlist}
\item $ 3\dfrac{4}{5}+\dfrac{2}{3} $
\item $ 2+4\dfrac{5}{7} $
\item $ 1\dfrac{2}{3}+4\dfrac{4}{5} $
\item $ 3+\dfrac{4}{5}+1\dfrac{5}{6} $
\end{rlist}
\end{multicols}
\item \textbf{Αφαίρεση κλασμάτων}\\
Να υπολογίσετε την τιμή των παρακάτω διαφορών.
\begin{multicols}{2}
\begin{rlist}
\item $ 4\dfrac{2}{3}-\dfrac{2}{3} $
\item $ 4-1\dfrac{2}{6} $
\item $ 7\dfrac{2}{3}-4\dfrac{5}{12} $
\item $ 12\dfrac{5}{8}-11\dfrac{5}{6} $
\end{rlist}
\end{multicols}
\item \textbf{Αριθμητικές παραστάσεις}\\
Να υπολογίσετε την τιμή των παρακάτω αριθμητικών παραστάσεων.
\begin{multicols}{2}
\begin{rlist}
\item $ \dfrac{4}{3}+\dfrac{5}{6}-\dfrac{1}{12} $
\item $ \dfrac{7}{8}-\dfrac{14}{12}+\dfrac{5}{6} $
\item $ \dfrac{28}{15}-\dfrac{3}{10}-\dfrac{1}{5} $
\item $ \dfrac{24}{60}-\dfrac{15}{40}+\dfrac{27}{50} $
\end{rlist}
\end{multicols}
\item \textbf{Αριθμητικές παραστάσεις}\\
Να υπολογίσετε την τιμή των παρακάτω αριθμητικών παραστάσεων.
\begin{multicols}{2}
\begin{rlist}
\item $ 3+\dfrac{3}{2}-\dfrac{4}{3} $
\item $ \dfrac{5}{4}-1+\dfrac{7}{6} $
\item $ \dfrac{32}{12}-\dfrac{9}{8}-5 $
\item $ \dfrac{40}{35}+7-\dfrac{34}{50} $
\end{rlist}
\end{multicols}
\item \textbf{Αριθμητικές παραστάσεις}\\
Να υπολογίσετε την τιμή των παρακάτω αριθμητικών παραστάσεων.
\begin{multicols}{2}
\begin{rlist}
\item $ 2\dfrac{4}{5}-\dfrac{3}{5}+\dfrac{5}{6} $
\item $ 4\dfrac{1}{4}-1\dfrac{3}{8}+\dfrac{7}{12} $
\item $ 5-2\dfrac{7}{8}+\dfrac{3}{4} $
\item $ 1\dfrac{14}{30}-2\dfrac{4}{45}+1\dfrac{17}{60} $
\end{rlist}
\end{multicols}
\item \textbf{Σύγκριση κλασμάτων}\\
Να συγκρίνετε τους παρακάτω αριθμούς μεταξύ τους βάζοντας το σωστό σύμβολο διάταξης ανάμεσά τους $ (<,\ =\ \ \textrm{ή}\ >) $.
\begin{multicols}{3}
\begin{rlist}[leftmargin=3mm]
\item $ \dfrac{4}{3}\ldots 2 $
\item $ 4\ldots \dfrac{10}{3} $
\item $ \dfrac{15}{3}\ldots 5 $
\item $ 3\dfrac{4}{5}\ldots \dfrac{16}{5} $
\item $ \dfrac{19}{4}\ldots 4\dfrac{3}{4} $
\item $ \dfrac{47}{25}\ldots 1\dfrac{23}{25} $
\end{rlist}
\end{multicols}
\item \textbf{Σύγκριση κλασμάτων}\\
Να τοποθετήσετε τους παρακάτω αριθμούς στη σειρά από το μικρότερο στο μεγαλύτερο.
\[ 3\frac{5}{8}\ ,\ \frac{23}{6}\ ,\ 2\frac{7}{12}\ ,\ \frac{89}{24}\ ,\ \dfrac{27}{8} \]
\end{enumerate}
\end{document}