\documentclass[ektypwsh]{diag-xelatex}
\usepackage[amsbb]{mtpro2}
\usepackage[no-math,cm-default]{fontspec}
\usepackage{xunicode}
\usepackage{xltxtra}
\usepackage{xgreek}
\usepackage{amsmath}
\usepackage{enumitem}
\defaultfontfeatures{Mapping=tex-text,Scale=MatchLowercase}
\setmainfont[Mapping=tex-text,Numbers=Lining,Scale=1.0,BoldFont={Minion Pro Bold}]{Minion Pro}
\newfontfamily\scfont{GFS Artemisia}
\font\icon = "Webdings"
\usepackage[amsbb]{mtpro2}
\usepackage[left=2.00cm, right=2.00cm, top=2.00cm, bottom=3.00cm]{geometry}
\xroma{red!80!black}
\newcommand{\tss}[1]{\textsuperscript{#1}}
\newcommand{\tssL}[1]{\MakeLowercase\textsuperscript{#1}}
\newlist{rlist}{enumerate}{3}
\setlist[rlist]{itemsep=0mm,label=\textcolor{\xrwma}{\roman*.}}
\usepackage{multicol}
\usepackage{tikz}

\begin{document}
\titlos{Μαθηματικά Β΄ Γυμνασίου}{Εξισώσεις - Εμβαδά}
\thewria
\begin{thema}
\item \textbf{Θεωρία}\\
Να απαντήσετε στις παρακάτω ερωτήσεις.
\begin{rlist}
\item Τι ονομάζεται εξίσωση;
\item Τι ονομάζεται λύση μιας εξισωσης;
\item Ποιές είναι οι μονάδες μέτρησης επιφάνειας;
\item Ποιά εξίσωση ονομάζεται αόριστη;
\item Πως υπολογίζουμε εμβαδόν ενός ορθογωνίου;
\end{rlist}\monades{6}
\item \textbf{Σωστό - Λάθος / Πολλαπλής επιλογής}\\
Α. \swstolathosd
\begin{rlist}
\item Τα $ 30cm^2 $ αντιστοιχούν σε $ 0{,}3dm^2 $.
\item Η εξίσωση $ 0x=0 $ είναι αδύνατη.
\item Η εξίσωση $ x+1=x+1 $ είναι αόριστη.
\item Τα $ 4km^2 $ είναι ίσα με $ 4.000.000m^2 $.
\item Το εμβαδόν ενός ορθογωνίου τριγώνου δίνεται από τον τύπο $ E=\frac{\beta\cdot\gamma}{2} $ όπου $ \beta,\gamma $ είναι οι κάθετες πλευρές του.
\end{rlist}\monades{3}\\
Β. Να επιλέξετε τη σωστή απάντηση σε κάθεμία από τις παρακάτω ερωτήσεις.
\begin{rlist}
\item Ποιά από τις παρακάτω εξισώσεις είναι αδύνατη;
\begin{multicols}{4}
\begin{itemize}
\item $ 0x=0 $
\item $ 2x+1=4 $
\item $ x=0 $
\item $ x=x+1 $
\end{itemize}
\end{multicols}
\item Ποιό το εμβαδόν ενός τριγώνου με βάση $ 5cm $ και ύψος $ 4cm $;
\begin{multicols}{4}
\begin{itemize}
\item $ 20cm^2 $
\item $ 10cm^2 $
\item $ 25cm^2 $
\item $ 16cm^2 $
\end{itemize}
\end{multicols}
\end{rlist}\monades{3}
\end{thema}
\newpage
\noindent
\askhseis
\begin{thema}
\item \textbf{Εξισώσεις}\\
Να λύσετε τις παρακάτω εξισώσεις.
\begin{multicols}{2}
\begin{rlist}
\item $ 4-3(x-2)+5x=7-4(5-2x)+5 $
\item $ \dfrac{4y-2}{3}+\dfrac{2-5y}{4}=y-2 $
\end{rlist}
\end{multicols}\monades{7}
\item \textbf{Εμβαδά}\\
Να υπολογίσετε τα εμβαδά των παρακάτω σχημάτων.
\begin{multicols}{3}
\begin{rlist}
\item \begin{tikzpicture}
\draw  (-1.5,1.5) rectangle (0.5,-0.5);
\node at (-0.5,-1) {$14$};
\end{tikzpicture}
\item \begin{tikzpicture}
\draw  (-1.5,1.5) rectangle (2,-0.5);
\node at (0.25,-1) {$\frac{14}{5}$};
\node at (2.5,0.5) {$\frac{7}{3}$};
\end{tikzpicture}
\item \begin{tikzpicture}
\draw (-1.5,1)  -- (1,1) -- (0,-0.5) -- (-2.5,-0.5) -- cycle;
\node at (-1.25,-1) {$25$};
\draw (-1.5,1) -- (-1.5,-0.5);
\node at (-1,0.25) {$17$};
\end{tikzpicture}
\item \begin{tikzpicture}
\draw (-1.5,1.5)  -- (1,-0.5) -- (-2.5,-0.5)  -- cycle;
\node at (-1.25,-1) {$2{,}8$};
\draw (-1.5,1.5) -- (-1.5,-0.5);
\node at (-1,0) {$1{,}75$};
\end{tikzpicture}
\item \begin{tikzpicture}
\draw (-2.5,1)  -- (0.5,-0.5) -- (-2.5,-0.5)  -- cycle;
\node at (-1.,-1) {$3^2+4$};
\node[rotate=90] at (-3,0.25) {$5^2-17$};
\end{tikzpicture}
\item \begin{tikzpicture}
\draw (-2.5,1.5)  -- (-0.5,1.5) -- (0.5,-0.5) -- (-2.5,-0.5)  -- cycle;
\node at (-1.,-1) {$500$};
\node[rotate=90] at (-3,0.25) {$200$};
\node at (-1.5,1.75) {$350$};
\end{tikzpicture}
\end{rlist}
\end{multicols}\monades{7}
\item \textbf{Σύνθετο θέμα}\\
Έναν κήπο σχήματος ορθογωνίου τον διασχίζει ένας δρόμος, όπως αυτός φαίνεται στο παρακάτω σχήμα. Το μέρος του κήπου που απομένει φυτεύεται με γκαζόν.
\begin{center}
\begin{tikzpicture}
\fill[black!20] (1.5,1)--(-1.5,1)--(-1.5,-1.5)--cycle;
\fill[black!20] (2.5,-1.5)--(2.5,1)--(-.5,-1.5)--cycle;
\draw  (-1.5,1) rectangle (2.5,-1.5);
\draw (1.5,1) -- (-1.5,-1.5);
\draw (2.5,1) -- (-0.5,-1.5);
\draw[|-|] (-1.5,1.2) -- (2.5,1.2);
\node at (0.5,1.4) {$30m$};
\draw[|-|] (2.8,1) -- (2.8,-1.5);
\node at (3.2,-0.2) {$20m$};
\node at (-1,-1.9) {$x$};
\node at (1.8,0.8) {$x$};
\draw[|-|] (-1.5,-1.7) -- (-0.5,-1.7);
\end{tikzpicture}
\end{center}
\begin{rlist}
\item Να γραφτεί το εμβαδόν του δρόμου με τη βοήθεια του $ x $.\monades{2}
\item Να γραφτεί το εμβαδόν κάθε τμήματος που είναι φυτευμένο με γκαζόν με τη βοήθεια του $ x $.\monades{2}
\item Να βρεθεί η τιμή του $ x $ αν γνωρίζουμε οτι το εμβαδόν του δρόμου είναι το μισό από το εμβαδόν ενός τριγωνικού κήπου.\monades{3}
\end{rlist}
\end{thema}
\gymnasio
\kaliepityxia
\end{document}

