\PassOptionsToPackage{no-math,cm-default}{fontspec}
\documentclass[twoside,nofonts,internet]{askhseis}
\usepackage{amsmath}
\usepackage{xgreek}
\let\hbar\relax
\defaultfontfeatures{Mapping=tex-text,Scale=MatchLowercase}
\setmainfont[Mapping=tex-text,Numbers=Lining,Scale=1.0,BoldFont={Minion Pro Bold}]{Minion Pro}
\newfontfamily\scfont{GFS Artemisia}
\font\icon = "Webdings"
\usepackage[amsbb,subscriptcorrection,zswash,mtpcal,mtphrb]{mtpro2}
\xroma{red!70!black}
%------TIKZ - ΣΧΗΜΑΤΑ - ΓΡΑΦΙΚΕΣ ΠΑΡΑΣΤΑΣΕΙΣ ----
\usepackage{tikz}
\usepackage{tkz-euclide}
\usetkzobj{all}
\usepackage[framemethod=TikZ]{mdframed}
\usetikzlibrary{decorations.pathreplacing}
\usepackage{pgfplots}
\usetkzobj{all}
%-----------------------
\usepackage{calc}
\usepackage{hhline}
\usepackage[explicit]{titlesec}
\usepackage{graphicx}
\usepackage{multicol}
\usepackage{multirow}
\usepackage{enumitem}
\usepackage{tabularx}
\usepackage[decimalsymbol=comma]{siunitx}
\usetikzlibrary{backgrounds}
\usepackage{sectsty}
\sectionfont{\centering}
\usepackage{enumitem}
\setlist[enumerate]{label=\bf{\large \arabic*.}}
\usepackage{adjustbox}
\usepackage{mathimatika,gensymb,eurosym,wrap-rl}
\usepackage{systeme,regexpatch}
%-------- ΜΑΘΗΜΑΤΙΚΑ ΕΡΓΑΛΕΙΑ ---------
\usepackage{mathtools}
%----------------------
%-------- ΠΙΝΑΚΕΣ ---------
\usepackage{booktabs}
%----------------------
%----- ΥΠΟΛΟΓΙΣΤΗΣ ----------
\usepackage{calculator}
%----------------------------
%------ ΔΙΑΓΩΝΙΟ ΣΕ ΠΙΝΑΚΑ -------
\usepackage{array}
\newcommand\diag[5]{%
\multicolumn{1}{|m{#2}|}{\hskip-\tabcolsep
$\vcenter{\begin{tikzpicture}[baseline=0,anchor=south west,outer sep=0]
\path[use as bounding box] (0,0) rectangle (#2+2\tabcolsep,\baselineskip);
\node[minimum width={#2+2\tabcolsep-\pgflinewidth},
minimum  height=\baselineskip+#3-\pgflinewidth] (box) {};
\draw[line cap=round] (box.north west) -- (box.south east);
\node[anchor=south west,align=left,inner sep=#1] at (box.south west) {#4};
\node[anchor=north east,align=right,inner sep=#1] at (box.north east) {#5};
\end{tikzpicture}}\rule{0pt}{.71\baselineskip+#3-\pgflinewidth}$\hskip-\tabcolsep}}
%---------------------------------
%---- ΟΡΙΖΟΝΤΙΟ - ΚΑΤΑΚΟΡΥΦΟ - ΠΛΑΓΙΟ ΑΓΚΙΣΤΡΟ ------
\newcommand{\orag}[3]{\node at (#1)
{$ \overcbrace{\rule{#2mm}{0mm}}^{{\scriptsize #3}} $};}
\newcommand{\kag}[3]{\node at (#1)
{$ \undercbrace{\rule{#2mm}{0mm}}_{{\scriptsize #3}} $};}
\newcommand{\Pag}[4]{\node[rotate=#1] at (#2)
{$ \overcbrace{\rule{#3mm}{0mm}}^{{\rotatebox{-#1}{\scriptsize$#4$}}}$};}
%-----------------------------------------


%------------------------------------------
\newcommand{\tss}[1]{\textsuperscript{#1}}
\newcommand{\tssL}[1]{\MakeLowercase{\textsuperscript{#1}}}
%---------- ΛΙΣΤΕΣ ----------------------
\newlist{brlist}{enumerate}{3}
\setlist[brlist]{itemsep=0mm,label=\bf\roman*.}
\newlist{tropos}{enumerate}{3}
\setlist[tropos]{label=\bf\textit{\arabic*\textsuperscript{oς}\;Τρόπος :},leftmargin=0cm,itemindent=2.3cm,ref=\bf{\arabic*\textsuperscript{oς}\;Τρόπος}}
% Αν μπει το bhma μεσα σε tropo τότε
%\begin{bhma}[leftmargin=.7cm]
\tkzSetUpPoint[size=7,fill=white]
\tikzstyle{pl}=[line width=0.3mm]
\tikzstyle{plm}=[line width=0.4mm]

\begin{document}
\titlos{Μαθηματικά Β΄ Γυμνασίου}{Εξισώσεις - Ανισώσεις}{Εξισώσεις}
\thewria
\begin{enumerate}
\item 
\end{enumerate}
\twocolkentro{\askhseis}
\begin{enumerate}
\item Να λυθούν οι παρακάτω εξισώσεις
\begin{rlist}
\begin{multicols}{2}
\item $ 2x-1=3 $
\item $ 4-3x=1 $
\item $ 5x-4=x $
\item $ 2x-3=-x $
\end{multicols}
\end{rlist}
\item Να λυθούν οι παρακάτω εξισώσεις
\begin{rlist}[leftmargin=4mm]
\begin{multicols}{2}
\item $ 2x-3=x+7 $
\item $ 3x+7=x-5 $
\item $ 7+x=4x-8 $
\item $ 3+4x+5=2x+4 $
\end{multicols}
\end{rlist}
\item Να λυθούν οι παρακάτω εξισώσεις
\begin{rlist}
\item $ 2x-1=4+2x $
\item $ 7-3x=-3x+7 $
\item $ 5x-3=x-3+4x $
\item $ 2x+1-x=x-3+4 $
\end{rlist}
\item Να λυθούν οι παρακάτω εξισώσεις
\begin{rlist}
\item $ 2(x-1)=4 $
\item $ 1-3(1-x)=4 $
\item $ 3(2x-1)=2(1-x) $
\item $ 5(1-x)+7=6-(x+2) $
\end{rlist}
\item Να λυθούν οι παρακάτω εξισώσεις
\begin{rlist}
\item $ 2(x-1)=3(2-x)+7 $
\item $ 4(x-3)-1=3-(3x+2) $
\item $ 5-2(x+3)=7(x-2)+4 $
\item $ 3(2x-5)-(4-x)=3(x+2) $
\end{rlist}
\item Να λυθούν οι παρακάτω εξισώσεις
\begin{rlist}
\item $ 3(x-2)+4=3x-2 $
\item $ 4x-(5+x)=2(x-3)+x $
\item $ 2(4-x)+3(3+2x)=4x-1 $
\item $ 3(1-3x)-(2-x)=4(1-2x)+3 $
\end{rlist}
\item Να λυθούν οι παρακάτω εξισώσεις
\begin{multicols}{2}
\begin{rlist}[leftmargin=4mm]
\item $ \dfrac{x-1}{2}=\dfrac{2x-1}{3} $
\item $ \dfrac{3x-1}{5}=\dfrac{4-x}{2} $
\item $ \dfrac{2x-3}{3}=\dfrac{7}{5} $
\item $ \dfrac{2x-4}{2}=5x $
\end{rlist}
\end{multicols}
\item Να λυθούν οι παρακάτω εξισώσεις
\begin{rlist}
\item $ \dfrac{x-5}{2}+\dfrac{2x-4}{3}=2 $
\item $ \dfrac{3x-8}{4}-\dfrac{1}{2}=\dfrac{7x+8}{10}-\dfrac{x}{2} $
\item $ \dfrac{x+1}{3}=\dfrac{2x-9}{4}+\dfrac{1}{12} $
\item $ \dfrac{1}{4}(x+3)-\dfrac{1}{5}(2x-1)=2+\dfrac{1}{10}x $
\end{rlist}
\item Να λυθούν οι παρακάτω εξισώσεις
\begin{rlist}
\item $ \dfrac{2x-3}{2}-\dfrac{3x+1}{4}=\dfrac{x-3}{4}-1 $
\item $ \dfrac{x-1}{4}+\dfrac{2-x}{3}=1-x $
\item $ \dfrac{2(3-x)}{5}+x=\dfrac{4(x-3)}{7}+\dfrac{x}{35} $
\end{rlist}
\item Να λυθούν οι παρακάτω εξισώσεις
\begin{multicols}{2}
\begin{rlist}
\item $ \dfrac{5+\frac{x-2}{3}}{3}=3 $
\item $ \dfrac{\frac{x-1}{2}+\frac{1}{5}}{4}=\frac{1}{10} $
\end{rlist}
\end{multicols}
\item Δίνεται η παραμετρική εξίσωση \[ (3\lambda-1)x-\lambda x+5=5\lambda x-12 \]
όπου $ \lambda $ είναι γνωστός αριθμός και $ x $ ο άγνωστος. Να βρεθεί η τιμή που πρέπει να έχει το $ \lambda $ ώστε η εξίσωση να έχει λύση το $ x=1 $.
\item Να βρεθεί η τιμή του $ \mu $ ώστε η εξίσωση \[ \dfrac{\mu-1}{2}x+\dfrac{1}{3}=\dfrac{x+1}{3} \] να είναι ταυτότητα. (Να είναι δηλαδή της μορφής $ 0x=0 $).
\item Δίνεται η εξίσωση \[ (\lambda+2)x-(x-1)\lambda=x+\lambda\lambda+1 \]
\begin{enumerate}[label=\roman*.,itemsep=0mm]
\item Αν $ \lambda=3 $ να αποδειχθεί ότι η εξίσωση έχει λύση $ x=1 $.
\item Να λυθεί η εξίσωση για $ \lambda=1 $.
\end{enumerate}
\item Να βρεθεί ο αριθμός $ x $ έτσι ώστε το τρίγωνο $AB\varGamma$ του διπλανού σχήματος να είναι ισοσκελές με 
\begin{enumerate}[itemsep=0mm,label=\roman*.]
\item βάση την πλευρά $B\varGamma$.
\item βάση την πλευρά $AB$.
\end{enumerate}
Να αποδειχθεί επίσης ότι δεν υπάρχει τιμή του $x$ ώστε το τρίγωνο να είναι ισοσκελές με βάση την πλευρά $B\varGamma$.
\end{enumerate}
\end{document}

