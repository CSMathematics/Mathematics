\documentclass[11pt,a4paper,twocolumn]{article}
\usepackage[english,greek]{babel}
\usepackage[utf8]{inputenc}
\usepackage{nimbusserif}
\usepackage[T1]{fontenc}
\usepackage[left=1.50cm, right=1.50cm, top=2.00cm, bottom=2.00cm]{geometry}
\usepackage{amsmath}
\let\myBbbk\Bbbk
\let\Bbbk\relax
\usepackage[amsbb,subscriptcorrection,zswash,mtpcal,mtphrb,mtpfrak]{mtpro2}
\usepackage{graphicx,multicol,multirow,enumitem,tabularx,mathimatika,gensymb,venndiagram,hhline,longtable,tkz-euclide,fontawesome5,eurosym,tcolorbox,tabularray}
\usepackage[explicit]{titlesec}
\tcbuselibrary{skins,theorems,breakable}
\newlist{rlist}{enumerate}{3}
\setlist[rlist]{itemsep=0mm,label=\roman*.}
\newlist{alist}{enumerate}{3}
\setlist[alist]{itemsep=0mm,label=\alph*.}
\newlist{balist}{enumerate}{3}
\setlist[balist]{itemsep=0mm,label=\bf\alph*.}
\newlist{Alist}{enumerate}{3}
\setlist[Alist]{itemsep=0mm,label=\Alph*.}
\newlist{bAlist}{enumerate}{3}
\setlist[bAlist]{itemsep=0mm,label=\bf\Alph*.}
\newlist{askhseis}{enumerate}{3}
\setlist[askhseis]{label={\Large\thesection}.\arabic*.}
\renewcommand{\textstigma}{\textsigma\texttau}
\newlist{thema}{enumerate}{3}
\setlist[thema]{label=\bf\large{ΘΕΜΑ \textcolor{black}{\Alph*}},itemsep=0mm,leftmargin=0cm,itemindent=18mm}
\newlist{erwthma}{enumerate}{3}
\setlist[erwthma]{label=\bf{\large{\textcolor{black}{\Alph{themai}.\arabic*}}},itemsep=0mm,leftmargin=0.8cm}

\newcommand{\kerkissans}[1]{{\fontfamily{maksf}\selectfont \textbf{#1}}}
\renewcommand{\textdexiakeraia}{}

\usepackage[
backend=biber,
style=alphabetic,
sorting=ynt
]{biblatex}

\DeclareTblrTemplate{caption}{nocaptemplate}{}
\DeclareTblrTemplate{capcont}{nocaptemplate}{}
\DeclareTblrTemplate{contfoot}{nocaptemplate}{}
\NewTblrTheme{mytabletheme}{
\SetTblrTemplate{caption}{nocaptemplate}{}
\SetTblrTemplate{capcont}{nocaptemplate}{}
\SetTblrTemplate{contfoot}{nocaptemplate}{}
}

\NewTblrEnviron{mytblr}
\SetTblrStyle{firsthead}{font=\bfseries}
\SetTblrStyle{firstfoot}{fg=red2}
\SetTblrOuter[mytblr]{theme=mytabletheme}
\SetTblrInner[mytblr]{
rowspec={t{7mm}},columns = {c},
width = 0.85\linewidth,
row{odd} = {bg=red9,fg=black,ht=8mm},
row{even} = {bg=red7,fg=black,ht=8mm},
hlines={white},vlines={white},
row{1} = {bg=red4, fg=white, font=\bfseries\fontfamily{maksf}},rowhead = 1,
hline{2} = {.7mm}, % midrule  
}
\newcounter{askhsh}
\setcounter{askhsh}{1}
\newcommand{\askhsh}{\large\theaskhsh.\ \addtocounter{askhsh}{1}}

\titleformat{\section}{\Large}{\kerkissans{\thesection}}{10pt}{\Large\kerkissans{#1}}

\setlength{\columnsep}{5mm}
\titleformat{\paragraph}
{\large}%
{}{0em}%
{\textcolor{red!80!black}{\faSquare\ \ \kerkissans{\bmath{#1}}}}
\setlength{\parindent}{0pt}

\newcommand{\eng}[1]{\selectlanguage{english}#1\selectlanguage{greek}}

\begin{document}
\twocolumn[{
\centering
\kerkissans{{\huge Εξισώσεις 1ου βαθμού}\\\vspace{3mm} {\Large ΑΣΚΗΣΕΙΣ}}\vspace{5mm}}]
\paragraph{Επίλυση εξίσωσης}
\askhsh Να λυθούν οι παρακάτω εξισώσεις
\begin{alist}
\begin{multicols}{2}
\item $ 2x-1=3 $
\item $ 4-3x=1 $
\item $ 5x-4=x $
\item $ 2x-3=-x $
\end{multicols}
\end{alist}
\askhsh Να λυθούν οι παρακάτω εξισώσεις
\begin{alist}[leftmargin=4mm]
\begin{multicols}{2}
\item $ 2x-3=x+7 $
\item $ 3x+7=x-5 $
\item $ 7+x=4x-8 $
\item $ 3+4x+5=2x+4 $
\end{multicols}
\end{alist}
\askhsh Να λυθούν οι παρακάτω εξισώσεις
\begin{alist}
\item $ 2x-1=4+2x $
\item $ 7-3x=-3x+7 $
\item $ 5x-3=x-3+4x $
\item $ 2x+1-x=x-3+4 $
\end{alist}
\askhsh Να λυθούν οι παρακάτω εξισώσεις
\begin{alist}
\item $ 2(x-1)=4 $
\item $ 1-3(1-x)=4 $
\item $ 3(2x-1)=2(1-x) $
\item $ 5(1-x)+7=6-(x+2) $
\end{alist}
\askhsh Να λυθούν οι παρακάτω εξισώσεις
\begin{alist}
\item $ 2(x-1)=3(2-x)+7 $
\item $ 4(x-3)-1=3-(3x+2) $
\item $ 5-2(x+3)=7(x-2)+4 $
\item $ 3(2x-5)-(4-x)=3(x+2) $
\end{alist}
\askhsh Να λυθούν οι παρακάτω εξισώσεις
\begin{alist}
\item $ 3(x-2)+4=3x-2 $
\item $ 4x-(5+x)=2(x-3)+x $
\item $ 2(4-x)+3(3+2x)=4x-1 $
\item $ 3(1-3x)-(2-x)=4(1-2x)+3 $
\end{alist}
\askhsh Να λυθούν οι παρακάτω εξισώσεις
\begin{multicols}{2}
\begin{alist}[leftmargin=4mm]
\item $ \dfrac{x-1}{2}=\dfrac{2x-1}{3} $
\item $ \dfrac{3x-1}{5}=\dfrac{4-x}{2} $
\item $ \dfrac{2x-3}{3}=\dfrac{7}{5} $
\item $ \dfrac{2x-4}{2}=5x $
\end{alist}
\end{multicols}
\askhsh Να λυθούν οι παρακάτω εξισώσεις
\begin{alist}
\item $ \dfrac{x-5}{2}+\dfrac{2x-4}{3}=2 $
\item $ \dfrac{3x-8}{4}-\dfrac{1}{2}=\dfrac{7x+8}{10}-\dfrac{x}{2} $
\item $ \dfrac{x+1}{3}=\dfrac{2x-9}{4}+\dfrac{1}{12} $
\item $ \dfrac{1}{4}(x+3)-\dfrac{1}{5}(2x-1)=2+\dfrac{1}{10}x $
\end{alist}
\askhsh Να λυθούν οι παρακάτω εξισώσεις
\begin{alist}
\item $ \dfrac{2x-3}{2}-\dfrac{3x+1}{4}=\dfrac{x-3}{4}-1 $
\item $ \dfrac{x-1}{4}+\dfrac{2-x}{3}=1-x $
\item $ \dfrac{2(3-x)}{5}+x=\dfrac{4(x-3)}{7}+\dfrac{x}{35} $
\end{alist}
\askhsh Να λυθούν οι παρακάτω εξισώσεις
\begin{multicols}{2}
\begin{alist}
\item $ \dfrac{5+\frac{x-2}{3}}{3}=3 $
\item $ \dfrac{\frac{x-1}{2}+\frac{1}{5}}{4}=\frac{1}{10} $
\end{alist}
\end{multicols}
\askhsh Δίνεται η παραμετρική εξίσωση \[ (3\lambda-1)x-\lambda x+5=5\lambda x-12 \]
όπου $ \lambda $ είναι γνωστός αριθμός και $ x $ ο άγνωστος. Να βρεθεί η τιμή που πρέπει να έχει το $ \lambda $ ώστε η εξίσωση να έχει λύση το $ x=1 $.\\\\
\askhsh Να βρεθεί η τιμή του $ \mu $ ώστε η εξίσωση \[ \dfrac{\mu-1}{2}x+\dfrac{1}{3}=\dfrac{x+1}{3} \] να είναι ταυτότητα. (Να είναι δηλαδή της μορφής $ 0x=0 $).\\\\
\askhsh Δίνεται η εξίσωση \[ (\lambda+2)x-(x-1)\lambda=x+\lambda\lambda+1 \]
\begin{enumerate}[label=\roman*.,itemsep=0mm]
\item Αν $ \lambda=3 $ να αποδειχθεί ότι η εξίσωση έχει λύση $ x=1 $.
\item Να λυθεί η εξίσωση για $ \lambda=1 $.
\end{enumerate}
\askhsh Να βρεθεί ο αριθμός $ x $ έτσι ώστε το τρίγωνο $AB\varGamma$ του διπλανού σχήματος να είναι ισοσκελές με 
\begin{enumerate}[itemsep=0mm,label=\roman*.]
\item βάση την πλευρά $B\varGamma$.
\item βάση την πλευρά $AB$.
\end{enumerate}
Να αποδειχθεί επίσης ότι δεν υπάρχει τιμή του $x$ ώστε το τρίγωνο να είναι ισοσκελές με βάση την πλευρά $B\varGamma$.
\end{document}
