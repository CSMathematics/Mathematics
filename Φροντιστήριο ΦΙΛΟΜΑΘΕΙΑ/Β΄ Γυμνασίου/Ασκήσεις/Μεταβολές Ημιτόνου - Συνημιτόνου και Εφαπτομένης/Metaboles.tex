\PassOptionsToPackage{no-math,cm-default}{fontspec}
\documentclass[twoside,nofonts,internet]{askhseis}
\usepackage{amsmath}
\usepackage{xgreek}
\let\hbar\relax
\defaultfontfeatures{Mapping=tex-text,Scale=MatchLowercase}
\setmainfont[Mapping=tex-text,Numbers=Lining,Scale=1.0,BoldFont={Minion Pro Bold}]{Minion Pro}
\newfontfamily\scfont{GFS Artemisia}
\xroma{red!80!black}
\usepackage{mtpro2}
%------TIKZ - ΣΧΗΜΑΤΑ - ΓΡΑΦΙΚΕΣ ΠΑΡΑΣΤΑΣΕΙΣ ----
\usepackage{tikz,pgfplots}
\usepackage{tkz-euclide}
\usetkzobj{all}
\usepackage[framemethod=TikZ]{mdframed}
\usetikzlibrary{decorations.pathreplacing}
\usepackage{pgfplots}
\usetkzobj{all}
%-----------------------
\usepackage{calc}
\usepackage{hhline}
\renewcommand{\thepart}{\arabic{part}}
\usepackage[explicit]{titlesec}
\usepackage{graphicx}
\usepackage{multicol}
\usepackage{multirow}
\usepackage{enumitem}
\usepackage{tabularx}
\usepackage[decimalsymbol=comma]{siunitx}
\usetikzlibrary{backgrounds}
\usepackage{sectsty}
\sectionfont{\centering}
\setlist[enumerate]{label=\bf{\large \arabic*.}}
\usepackage{adjustbox}
\usepackage{mathimatika,wrap-rl}
%-------- ΠΙΝΑΚΕΣ ---------
\usepackage{booktabs}
%----------------------
%----- ΥΠΟΛΟΓΙΣΤΗΣ ----------
\usepackage{calculator}
%----------------------------
%---- ΟΡΙΖΟΝΤΙΟ - ΚΑΤΑΚΟΡΥΦΟ - ΠΛΑΓΙΟ ΑΓΚΙΣΤΡΟ ------
\newcommand{\orag}[3]{\node at (#1)
{$ \overcbrace{\rule{#2mm}{0mm}}^{{\scriptsize #3}} $};}
\newcommand{\kag}[3]{\node at (#1)
{$ \undercbrace{\rule{#2mm}{0mm}}_{{\scriptsize #3}} $};}
\newcommand{\Pag}[4]{\node[rotate=#1] at (#2)
{$ \overcbrace{\rule{#3mm}{0mm}}^{{\rotatebox{-#1}{\scriptsize$#4$}}}$};}
%-----------------------------------------
\newcommand{\tss}[1]{\textsuperscript{#1}}
\newcommand{\tssL}[1]{\MakeLowercase{\textsuperscript{#1}}}
%---------- ΛΙΣΤΕΣ ----------------------
\newlist{bhma}{enumerate}{3}
\setlist[bhma]{label=\bf\textit{\arabic*\textsuperscript{o}\;Βήμα :},leftmargin=0cm,itemindent=1.8cm,ref=\bf{\arabic*\textsuperscript{o}\;Βήμα}}
\newlist{tropos}{enumerate}{3}
\setlist[tropos]{label=\bf\textit{\arabic*\textsuperscript{oς}\;Τρόπος :},leftmargin=0cm,itemindent=2.3cm,ref=\bf{\arabic*\textsuperscript{oς}\;Τρόπος}}
% Αν μπει το bhma μεσα σε tropo τότε
%\begin{bhma}[leftmargin=.7cm]
\tkzSetUpPoint[size=7,fill=white]
\tikzstyle{pl}=[line width=0.3mm]
\tikzstyle{plm}=[line width=0.4mm]








\begin{document}
\twocolkentro{
\titlos{Μαθηματικά Β' Γυμνασίου}{Τριγωνομετρία}{Ημίτονο - Συνημίτονο οξείας γωνίας}
\thewria}
\begin{enumerate}
\item 
\end{enumerate}
\twocolkentro{\askhseis}
\begin{enumerate}
\item \textbf{Υπολογισμός πλευρών}\\
Δίνεται τρίγωνο $ AB\varGamma $ με $ \hat{B}=65\degree $, $ \hat{\varGamma}=35\degree $ και $ AB=5 $. Να υπολογιστεί το εμβαδόν του τριγώνου.
\vspace{-5mm}
\begin{center}
\begin{tikzpicture}
\tkzDefPoint(0,0){B}
\tkzDefPoint(3.5,0){C}
\tkzDefPoint(1,2){A}
\tkzDefPoint(1,0){D}
\draw[pl] (A) -- (B) -- (C) -- cycle;
\draw[pl] (A) -- (D);
\tkzLabelPoint[above](A){$A$}
\tkzLabelPoint[left](B){$B$}
\tkzLabelPoint[right](C){$\varGamma$}
\tkzLabelPoint[below](D){$\varDelta$}
\tkzMarkRightAngle(C,D,A)
\tkzMarkAngle[size=.33](D,B,A)
\tkzLabelAngle[pos=.7](D,B,A){\footnotesize$65\degree$}
\tkzMarkAngle[size=.45](A,C,D)
\tkzLabelAngle[pos=.75](A,C,D){\footnotesize$35\degree$}
\tkzDrawPoints(A,B,C,D)
\node at (0.3,1.1) {\footnotesize$5$};
\end{tikzpicture}
\end{center}
\item \textbf{Πρόβλημα}\\
Σε κάθε γήπεδο ποδοσφαίρου, η απόσταση από το σημείο του πέναλτι μέχρι το αντίπαλο τέρμα είναι $ 11m $. Επίσης γνωρίζουμε ότι το μήκος του τέρματος είναι $ 7.22m $.
\begin{center}
\begin{tikzpicture}[scale=1.5]
\draw  (1,0.25) rectangle (4.5,-1.5);
\draw  (1.75,0.25) rectangle (3.75,-0.6);
\draw (2.75,-1.1) ++(215:.7) arc (215:325:.7);
\draw[draw=black, fill=white,step=.7mm ]
(2.3,0.25) grid  (3.2,0.45) rectangle (2.3,0.25);
\tkzDefPoint(2.75,-1.1){A}
\tkzDefPoint(3.2,.25){B}
\tkzDefPoint(2.3,0.25){C}
\draw[dashed] (3.2,.25)--(A)--(2.3,0.25);
\tkzDrawPoint(A)
\draw[|-|] (2.3,0.6) -- (3.2,0.6);
\node at (2.75,0.75) {\footnotesize$7.32m$};
\draw[|-|] (3.3,0.22) -- (3.3,-1.1);
\node at (3.5,-0.45) {\footnotesize$11m$};
\draw (A)--(2.75,.25);
\tkzLabelPoint[below left](A){$P$}
\node at (2.15,0.4) {$A$};
\node at (3.35,0.4) {$B$};
\tkzMarkAngle[size=.4](B,A,C)
\end{tikzpicture}
\end{center}
Αν ένας παίκτης στέκεται στο σημείο του πέναλτι και ετοιμάζεται να σουτάρει, τότε να υπολογίσετε τη γωνία $ B\hat{P}A $ υπό την οποία βλέπει το τέρμα και έχει περιθώριο να σκοράρει. (Χρησιμοποιήστε πίνακα τριγωνομετρικών αριθμών).
\end{enumerate}
\end{document}