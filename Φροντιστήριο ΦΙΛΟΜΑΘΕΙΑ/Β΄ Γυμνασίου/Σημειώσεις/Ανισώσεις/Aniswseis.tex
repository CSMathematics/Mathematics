\PassOptionsToPackage{no-math,cm-default}{fontspec}
\documentclass[twoside,nofonts,internet,methodoi]{thewria}
\usepackage{amsmath}
\usepackage{xgreek}
\let\hbar\relax
\defaultfontfeatures{Mapping=tex-text,Scale=MatchLowercase}
\setmainfont[Mapping=tex-text,Numbers=Lining,Scale=1.0,BoldFont={Minion Pro Bold}]{Minion Pro}
\newfontfamily\scfont{GFS Artemisia}
\font\icon = "Webdings"
\usepackage[amsbb,subscriptcorrection,zswash,mtpcal,mtphrb]{mtpro2}
\usepackage{tikz,pgfplots}
\tkzSetUpPoint[size=7,fill=white]
\xroma{red!70!black}
%------TIKZ - ΣΧΗΜΑΤΑ - ΓΡΑΦΙΚΕΣ ΠΑΡΑΣΤΑΣΕΙΣ ----
\usepackage{tikz}
\usepackage{tkz-euclide}
\usetkzobj{all}
\usepackage[framemethod=TikZ]{mdframed}
\usetikzlibrary{decorations.pathreplacing}
\usepackage{pgfplots}
\usetkzobj{all}
%-----------------------
\usepackage{calc}
\usepackage{hhline}
\usepackage[explicit]{titlesec}
\usepackage{graphicx}
\usepackage{multicol}
\usepackage{multirow}
\usepackage{enumitem}
\usepackage{tabularx,textcomp}
\usepackage[decimalsymbol=comma]{siunitx}
\usetikzlibrary{backgrounds}
\usepackage{sectsty}
\sectionfont{\centering}
\setlist[enumerate]{label=\bf{\large \arabic*.}}
\usepackage{adjustbox}
\usepackage{mathimatika,gensymb,eurosym,wrap-rl}
\usepackage{systeme,regexpatch}
%-------- ΜΑΘΗΜΑΤΙΚΑ ΕΡΓΑΛΕΙΑ ---------
\usepackage{mathtools}
%----------------------
%-------- ΠΙΝΑΚΕΣ ---------
\usepackage{booktabs}
%----------------------
%----- ΥΠΟΛΟΓΙΣΤΗΣ ----------
\usepackage{calculator}
%----------------------------
%------ ΔΙΑΓΩΝΙΟ ΣΕ ΠΙΝΑΚΑ -------
\usepackage{array}
\newcommand\diag[5]{%
\multicolumn{1}{|m{#2}|}{\hskip-\tabcolsep
$\vcenter{\begin{tikzpicture}[baseline=0,anchor=south west,outer sep=0]
\path[use as bounding box] (0,0) rectangle (#2+2\tabcolsep,\baselineskip);
\node[minimum width={#2+2\tabcolsep-\pgflinewidth},
minimum  height=\baselineskip+#3-\pgflinewidth] (box) {};
\draw[line cap=round] (box.north west) -- (box.south east);
\node[anchor=south west,align=left,inner sep=#1] at (box.south west) {#4};
\node[anchor=north east,align=right,inner sep=#1] at (box.north east) {#5};
\end{tikzpicture}}\rule{0pt}{.71\baselineskip+#3-\pgflinewidth}$\hskip-\tabcolsep}}
%---------------------------------
%---- ΟΡΙΖΟΝΤΙΟ - ΚΑΤΑΚΟΡΥΦΟ - ΠΛΑΓΙΟ ΑΓΚΙΣΤΡΟ ------
\newcommand{\orag}[3]{\node at (#1)
{$ \overcbrace{\rule{#2mm}{0mm}}^{{\scriptsize #3}} $};}
\newcommand{\kag}[3]{\node at (#1)
{$ \undercbrace{\rule{#2mm}{0mm}}_{{\scriptsize #3}} $};}
\newcommand{\Pag}[4]{\node[rotate=#1] at (#2)
{$ \overcbrace{\rule{#3mm}{0mm}}^{{\rotatebox{-#1}{\scriptsize$#4$}}}$};}
%-----------------------------------------


%------------------------------------------
\newcommand{\tss}[1]{\textsuperscript{#1}}
\newcommand{\tssL}[1]{\MakeLowercase{\textsuperscript{#1}}}
%---------- ΛΙΣΤΕΣ ----------------------
\newlist{bhma}{enumerate}{3}
\setlist[bhma]{label=\bf\textit{\arabic*\textsuperscript{o}\;Βήμα :},leftmargin=0cm,itemindent=1.8cm,ref=\bf{\arabic*\textsuperscript{o}\;Βήμα}}
\newlist{rlist}{enumerate}{3}
\setlist[rlist]{itemsep=0mm,label=\roman*.}
\newlist{brlist}{enumerate}{3}
\setlist[brlist]{itemsep=0mm,label=\bf\roman*.}
\newlist{tropos}{enumerate}{3}
\setlist[tropos]{label=\bf\textit{\arabic*\textsuperscript{oς}\;Τρόπος :},leftmargin=0cm,itemindent=2.3cm,ref=\bf{\arabic*\textsuperscript{oς}\;Τρόπος}}
% Αν μπει το bhma μεσα σε tropo τότε
%\begin{bhma}[leftmargin=.7cm]
\tkzSetUpPoint[size=7,fill=white]
\tikzstyle{pl}=[line width=0.3mm]
\tikzstyle{plm}=[line width=0.4mm]
\usepackage{etoolbox}
\makeatletter
\renewrobustcmd{\anw@true}{\let\ifanw@\iffalse}
\renewrobustcmd{\anw@false}{\let\ifanw@\iffalse}\anw@false
\newrobustcmd{\noanw@true}{\let\ifnoanw@\iffalse}
\newrobustcmd{\noanw@false}{\let\ifnoanw@\iffalse}\noanw@false
\renewrobustcmd{\anw@print}{\ifanw@\ifnoanw@\else\numer@lsign\fi\fi}
\makeatother




\begin{document}
\titlos{Μαθηματικά Β΄ Γυμνασίου}{Εξισώσεις - Ανισώσεις}{Ανισώσεις}
\begin{Methodos}[Επίλυση ανίσωσης]\label{meth:1}
Όπως και για την επίλυση μιας εξίσωσης έτσι και για την επίλυση μιας ανίσωσης με έναν άγνωστο, οποιασδήποτε μορφής, ακολουθούμε τα εξής βήματα:
\begin{bhma}
\item \textbf{Απαλοιφή παρονομαστών}\\
Αν η ανίσωση περιέχει κλάσματα, τότε υπολογίζουμε το Ε.Κ.Π. των παρονομαστών τους, ώστε να πολλαπλασιάσουμε κάθε όρο της μ' αυτο. Στη συνέχεια διαιρούμε κάθε παρονομαστή με το Ε.Κ.Π.
\item \textbf{Απαλοιφή παρενθέσεων}\\
Αν η ανίσωση περιέχει παρενθέσεις τότε εξετάζουμε το λόγο ύπαρξης της κάθε παρένθεσης. Εξετάζουμε δηλαδή αν η παρένθεση υπάρχει λόγω πολλαπλασιασμού, πρόσθεσης ή αφαίρεσης.
\item \textbf{Διαχωρισμός όρων}\\
Χωρίζουμε στα μέλη της ανίσωσης τους γνωστούς από τους άγνωστους όρους αλλάζοντας τα πρόσημά τους. Στη συνέχεια κάνουμε αναγωγή ομοίων όρων και στα δύο μέλη.
\item \textbf{Διαίρεση - Λύσεις ανίσωσης}\\
Τέλος διαιρούμε κάθε μέλος της ανίσωσης με το συντελεστή του αγνώστου, οπότε και προκύπτουν οι λυσεις της.
\item \textbf{Γραφική παράσταση λύσεων (Προαιρετικό)}\\
Αν μας ζητείται, σχεδιάζουμε τις λύσεις της ανίσωσης όπως θα δόυμε στη \textbf{Μέθοδο \ref{meth:2}}.
\end{bhma}
\end{Methodos}
\Paradeigma{Επίλυση απλής ανίσωσης}
\bmath{Να λυθεί η ανίσωση: $ 2x-5<10-3x $.}\\\\
\lysh\\
Στη συγκεκριμένη ανίσωση μπορούμε αμέσως να χωρίσουμε τους γνωστούς από τους άγνωστους όρους. Έτσι θα έχουμε:
\begin{align*}
2x-5&<10-3x\Rightarrow\\
2x+3x&<10+5\Rightarrow\\
\end{align*}
Στη συνέχεια κάνουμε αναγωγή ομοίων όρων σε κάθε μέλος και διαιρούμε με το συντελεστή του αγνώστου.
\[ 5x<15\Rightarrow\frac{5x}{5}<\frac{15}{5}\Rightarrow x<3 \]
Επομένως οι λύσεις της ανίσωσης θα είναι οι $ x<3 $.\\\\
\Paradeigma{Ανίσωση με παρενθέσεις}
\bmath{Να λυθεί η παρακάτω ανίσωση
\[ 5-(2+4x)\geq 2(5-x)+7 \]
Στη συνέχεια να σχεδιαστούν οι λύσεις της στον άξονα των αριθμών.}\\\\
\lysh\\
Ξεκινάμε απαλοίφοντας τις παρενθέσεις εξετάζοντας όσα είδαμε στη μέθοδο και θα έχουμε:
\begin{align*}
5-(2+4x)&\geq 2(5-x)+7\Rightarrow\\
5-2-4x&\geq 10-2x+7
\end{align*}
Συνεχίζουμε με τα υπόλοιπα βήματα όπως και στο προηγούμενο παράδειγμα. Προσέχουμε στο προτελευταίο βήμα να αλλάξουμε τη φορά της ανίσωσης.
\begin{align*}
5-2-4x&\geq 10-2x+7\Rightarrow\\
-4x+2x&\geq 10+7-5+2\Rightarrow\\
-2x&\geq 14\Rightarrow\\
\frac{-2x}{-2}&\leq \frac{14}{-2}\Rightarrow x\leq -7
\end{align*}
Οι λύσεις της ανίσωσης φαίνονται στο παρακάτω σχήμα : \begin{center}
\begin{tikzpicture}
\apeiroX{-7}{2}{-.3}{.3}{\xrwma}
\axonas{-.3}{2.5}
\akro{k}{2}
\end{tikzpicture}
\end{center}
\Paradeigma{Ανίσωση με κλάσματα}
\bmath{Να λυθεί η παρακάτω ανίσωση:
\[ \frac{2x-3}{4}+\frac{x-2}{5}\leq\frac{1}{10}+1 \]}
\lysh\\
Υπολογίζουμε το Ε.Κ.Π. των παρονομαστών και πολλαπλασιάζουμε μ΄αυτό κάθε όρο της ανίσωσης.
\begin{gather*}
\textrm{Ε.Κ.Π.}(4,5,10)=20\ \textrm{ άρα}\\
20\cdot\frac{2x-3}{4}+20\cdot\frac{x-2}{5}\leq20\cdot\frac{1}{10}+20\cdot 1\Rightarrow\\
5(2x-3)+4(x-2)\leq 2\cdot 1+20
\end{gather*}
Συνεχίζουμε λύνοντας την ανίσωση που προέκυψε ακολουθώντας τα υπόλοιπα βήματα όπως στα προηγούμενα παραδείγματα.
\begin{align*}
5(2x-3)+4(x-2)&\leq 2\cdot 1+20\Rightarrow\\
10x-15+4x-8&\leq 2+20\Rightarrow\\
10x+4x&\leq 2+20+15+8\Rightarrow\\
14x&\leq 45\Rightarrow\\ x&\leq \frac{45}{14}
\end{align*}
\begin{Methodos}[Γραφική παράσταση λύσεων]\label{meth:2}
Οι λύσεις μιας ανίσωσης μπορούν να παρασταθούν με τη βοήθεια ενός σχήματος πάνω στην ευθεία των αριθμών. 
\begin{bhma}
\item \textbf{Επίλυση ανισώσεων}\\
Υπολογίζουμε τις λύσεις της δοσμένης ανίσωσης.
\item \textbf{Άξονας λύσεων}\\
Για να σχεδιαστούν οι λύσεις της πάνω στο άξονα αριθμών, τοποθετούμε σ΄αυτόν τον αριθμό που προέκυψε στο 2\tss{ο} μέλος του αποτελέσματος. Στη θέση του αριμού αυτού σχεδιαζουμε
\begin{itemize}
\item έναν κύκλο : $ \circ $ αν έχουμε απλή σχέση ανισότητας.
\item μια κουκίδα : $ \bullet $ αν έχουμε σχέση ανισοισότητας.
\end{itemize}
\item \textbf{Λύσεις ανίσωσης}\\
Τέλος σχεδιάζουμε ένα σκιασμένο ορθογώνιο το οποίο ξεκινά από τη θέση του αριθμού και εκτείνεται προς τη μεριά των λύσεων.
\end{bhma}
\end{Methodos}
\Paradeigma{Γραφική παράσταση λύσεων ανίσωσης}
\bmath{Να σχεδιαστούν πάνω στον άξονα των αριθμών οι παρακάτω λύσεις ανισώσεων:
\begin{multicols}{4}
\begin{rlist}
\item $ x>4 $
\item $ x<-3 $
\item $ x\geq 0 $
\item $ x\leq 2 $
\end{rlist}
\end{multicols}}
\noindent
\lysh\\
Σύμφωνα με τα παραπάνω τα διαγράματα των λύσεων των ανισώσεων αυτών θα είναι τα εξής:
\begin{multicols}{4}
\begin{rlist}
\item \tikzitem\begin{tikzpicture}
\Xapeiro{4}{0}{2.3}{.3}{\xrwma}
\axonas{-.3}{2.5}
\akro{a}{0}
\end{tikzpicture}
\item \tikzitem\begin{tikzpicture}
\apeiroX{-3}{2}{-.3}{.3}{\xrwma}
\axonas{-.3}{2.5}
\akro{a}{2}
\end{tikzpicture}
\item \tikzitem\begin{tikzpicture}
\Xapeiro{0}{0}{2.3}{.3}{\xrwma}
\axonas{-.3}{2.5}
\akro{k}{0}
\end{tikzpicture}
\item \tikzitem\begin{tikzpicture}
\apeiroX{2}{2}{-.3}{.3}{\xrwma}
\axonas{-.3}{2.5}
\akro{k}{2}
\end{tikzpicture}
\end{rlist}
\end{multicols}
\begin{Methodos}[Κοινές λύσεις ανισώσεων]
Για την εύρεση κοινών λύσεων δύο ή περισσοτέρων ανισώσεων εργαζόμαστε ως εξής.
\begin{bhma}
\item \textbf{Επίλυση ανισώσεων}\\
Υπολογίζουμε τις λύσεις όλων των ανισώσεων σύμφωνα με τη \textbf{Μέθοδο \ref{meth:1}}.
\item \textbf{Άξονας λύσεων - Κοινές λύσεις}\\
Σχεδιάζουμε στον ίδιο άξονα αριθμών τα ορθογώνια των λύσεων όλων των ανισώσεων. Το κοινό μέρος των σχημάτων αυτών μας δίνει τις κοινές λύσεις τους.
\end{bhma}
\end{Methodos}
\Paradeigma{Κοινές λύσεις ανισώσεων}
\bmath{Να βρεθούν οι κοινές λύσεις των παρακάτω ανισώσεων
\[ 3(x-1)-8\leq x+5\qquad\textrm{και}\qquad 2x-4>3(2-x)+5 \]}
\lysh\\
Λύνοντας ξεχωριστά τις δύο ανισώσεις βρίσκουμε αντίστοιχα τις λύσεις $ x\leq 8 $ και $ x>3 $.
\begin{center}
\begin{tikzpicture}
\Xapeiro{3}{0}{2.5}{.3}{\xrwma}
\apeiroX{8}{2}{-.4}{.35}{\xrwma}
\axonas{-.3}{2.5}
\akro{k}{2}
\akro{a}{0}
\end{tikzpicture}
\end{center}
Σχεδιάζοντας και τα δύο σχήματα στον ίδιο άξονα βλέπουμε ότι οι κοινές λύσεις θα είναι οι: $ 3<x\leq 8 $.
\end{document}

