\PassOptionsToPackage{no-math,cm-default}{fontspec}
\documentclass[twoside,nofonts,internet,methodoi]{thewria}
\usepackage{amsmath}
\usepackage{xgreek}
\let\hbar\relax
\defaultfontfeatures{Mapping=tex-text,Scale=MatchLowercase}
\setmainfont[Mapping=tex-text,Numbers=Lining,Scale=1.0,BoldFont={Minion Pro Bold}]{Minion Pro}
\newfontfamily\scfont{GFS Artemisia}
\font\icon = "Webdings"
\usepackage[amsbb,subscriptcorrection,zswash,mtpcal,mtphrb]{mtpro2}
\usepackage{tikz,pgfplots}
\tkzSetUpPoint[size=7,fill=white]
\xroma{red!70!black}
%------TIKZ - ΣΧΗΜΑΤΑ - ΓΡΑΦΙΚΕΣ ΠΑΡΑΣΤΑΣΕΙΣ ----
\usepackage{tikz}
\usepackage{tkz-euclide}
\usetkzobj{all}
\usepackage[framemethod=TikZ]{mdframed}
\usetikzlibrary{decorations.pathreplacing}
\usepackage{pgfplots}
\usetkzobj{all}
%-----------------------
\usepackage{calc}
\usepackage{hhline}
\usepackage[explicit]{titlesec}
\usepackage{graphicx}
\usepackage{multicol}
\usepackage{multirow}
\usepackage{enumitem}
\usepackage{tabularx}
\usepackage[decimalsymbol=comma]{siunitx}
\usetikzlibrary{backgrounds}
\usepackage{sectsty}
\sectionfont{\centering}
\setlist[enumerate]{label=\bf{\large \arabic*.}}
\usepackage{adjustbox}
\usepackage{mathimatika,gensymb,eurosym,wrap-rl}
\usepackage{systeme,regexpatch}
%-------- ΜΑΘΗΜΑΤΙΚΑ ΕΡΓΑΛΕΙΑ ---------
\usepackage{mathtools}
%----------------------
%-------- ΠΙΝΑΚΕΣ ---------
\usepackage{booktabs}
%----------------------
%----- ΥΠΟΛΟΓΙΣΤΗΣ ----------
\usepackage{calculator}
%----------------------------
%------ ΔΙΑΓΩΝΙΟ ΣΕ ΠΙΝΑΚΑ -------
\usepackage{array}
\newcommand\diag[5]{%
\multicolumn{1}{|m{#2}|}{\hskip-\tabcolsep
$\vcenter{\begin{tikzpicture}[baseline=0,anchor=south west,outer sep=0]
\path[use as bounding box] (0,0) rectangle (#2+2\tabcolsep,\baselineskip);
\node[minimum width={#2+2\tabcolsep-\pgflinewidth},
minimum  height=\baselineskip+#3-\pgflinewidth] (box) {};
\draw[line cap=round] (box.north west) -- (box.south east);
\node[anchor=south west,align=left,inner sep=#1] at (box.south west) {#4};
\node[anchor=north east,align=right,inner sep=#1] at (box.north east) {#5};
\end{tikzpicture}}\rule{0pt}{.71\baselineskip+#3-\pgflinewidth}$\hskip-\tabcolsep}}
%---------------------------------
%---- ΟΡΙΖΟΝΤΙΟ - ΚΑΤΑΚΟΡΥΦΟ - ΠΛΑΓΙΟ ΑΓΚΙΣΤΡΟ ------
\newcommand{\orag}[3]{\node at (#1)
{$ \overcbrace{\rule{#2mm}{0mm}}^{{\scriptsize #3}} $};}
\newcommand{\kag}[3]{\node at (#1)
{$ \undercbrace{\rule{#2mm}{0mm}}_{{\scriptsize #3}} $};}
\newcommand{\Pag}[4]{\node[rotate=#1] at (#2)
{$ \overcbrace{\rule{#3mm}{0mm}}^{{\rotatebox{-#1}{\scriptsize$#4$}}}$};}
%-----------------------------------------


%------------------------------------------
\newcommand{\tss}[1]{\textsuperscript{#1}}
\newcommand{\tssL}[1]{\MakeLowercase{\textsuperscript{#1}}}
%---------- ΛΙΣΤΕΣ ----------------------
\newlist{bhma}{enumerate}{3}
\setlist[bhma]{label=\bf\textit{\arabic*\textsuperscript{o}\;Βήμα :},leftmargin=0cm,itemindent=1.8cm,ref=\bf{\arabic*\textsuperscript{o}\;Βήμα}}
\newlist{rlist}{enumerate}{3}
\setlist[rlist]{itemsep=0mm,label=\roman*.}
\newlist{brlist}{enumerate}{3}
\setlist[brlist]{itemsep=0mm,label=\bf\roman*.}
\newlist{tropos}{enumerate}{3}
\setlist[tropos]{label=\bf\textit{\arabic*\textsuperscript{oς}\;Τρόπος :},leftmargin=0cm,itemindent=2.3cm,ref=\bf{\arabic*\textsuperscript{oς}\;Τρόπος}}
% Αν μπει το bhma μεσα σε tropo τότε
%\begin{bhma}[leftmargin=.7cm]
\tkzSetUpPoint[size=7,fill=white]
\tikzstyle{pl}=[line width=0.3mm]
\tikzstyle{plm}=[line width=0.4mm]
\usepackage{etoolbox}
\makeatletter
\renewrobustcmd{\anw@true}{\let\ifanw@\iffalse}
\renewrobustcmd{\anw@false}{\let\ifanw@\iffalse}\anw@false
\newrobustcmd{\noanw@true}{\let\ifnoanw@\iffalse}
\newrobustcmd{\noanw@false}{\let\ifnoanw@\iffalse}\noanw@false
\renewrobustcmd{\anw@print}{\ifanw@\ifnoanw@\else\numer@lsign\fi\fi}
\makeatother



\begin{document}
\titlos{Μαθηματικά Β΄ Γυμνασίου}{Εξισώσεις - Ανισώσεις}{Εξισώσεις}
\begin{Methodos}[Επίλυση εξισώσεων]
Για την επίλυση κάθε εξίσωση με έναν άγνωστο, οποιασδήποτε μορφής, ακολουθούμε τα εξής βήματα:
\begin{bhma}
\item \textbf{Απαλοιφή παρονομαστών}\\
Αν η εξίσωση περιέχει κλάσματα, τότε υπολογίζουμε το Ε.Κ.Π. των παρονομαστών τους, ώστε να πολλαπλασιάσουμε κάθε όρο της μ' αυτο. Στη συνέχεια διαιρούμε κάθε παρονομαστή με το Ε.Κ.Π.
\item \textbf{Απαλοιφή παρενθέσεων}\\
Αν η εξίσωση περιέχει παρενθέσεις τότε εξετάζουμε το λόγο ύπαρξης της κάθε παρένθεσης. Εξετάζουμε δηλαδή αν η παρένθεση υπάρχει λόγω πολλαπλασιασμού, πρόσθεσης ή αφαίρεσης.
\item \textbf{Διαχωρισμός όρων}\\
Χωρίζουμε στα μέλη της εξίσωσης τους γνωστούς από τους άγνωστους όρους αλλάζοντας τα πρόσημά τους. Στη συνέχεια κάνουμε αναγωγή ομοίων όρων και στα δύο μέλη.
\item \textbf{Διαίρεση - Λύση εξίσωσης}\\
Τέλος διαιρούμε κάθε μέλος της εξίσωσης με το συντελεστή του αγνώστου, οπότε και προκύπτει η λύση της εξίσωσης.
\end{bhma}
\end{Methodos}
\Paradeigma{Επίλυση απλής εξίσωσης}
\bmath{Να λυθεί η εξίσωση : $ 4x-5+x=2x+7 $.}\\\\
\lysh\\
Στην εξίσωση αυτή μπορούμε άμεσα να χωρίσουμε στα δύο μέλη τηςτους γνωστούς από τους άγνωστους όρους της. Οπότε θα έχουμε:
\begin{align*}
4x-5+x&=2x+7\Rightarrow\\
4x+x-2x&=7+5
\end{align*}
Προσθέτοντας τους όμοιους όρους σε κάθε μέλος θα προκύψει:
\[ 3x=12 \]
Τέλος, διαιρούμε κάθε μέλος της εξίσωσης με τον αριθμό $ 3 $ και παίρνουμε:
\[ \frac{3x}{3}=\frac{12}{3}\Rightarrow x=4 \]
\Paradeigma{Εξίσωση με παρενθέσεις}
\bmath{Να λυθεί η παρακάτω εξίσωση:
\[ 3(x-2)+4=(3+2x)-5-(3x-2) \]}
\lysh\\
Βλέπουμε αρχικά ότι η εξίσωση περιέχει τρεις παρενθέσεις και αν παρατηρήσουμε καλύτερα θα δούμε ότι καθεμία έχει διαφορετικό λόγο ύπαρξης. Η πρώτη έχει απ΄έξω πολλαπλασιασμό, η δεύτερη το πρόσημο $ + $, το οποίο έχουμε παραλείψει και η τρίτη το πρόσημο $ - $. Έτσι θα έχουμε:
\begin{align*}
3(x-2)+4&=(3+2x)-5-(3x-2)\Rightarrow\\
3x-6+4&=3+2x-5-3x+2
\end{align*}
Μετά το βήμα αυτό συνεχίζουμε τα υπόλοιπα βήματα όπως και στο προηγούμενο παράδειγμα οπότε:
\begin{align*}
3x-6+4&=3+2x-5-3x+2\Rightarrow\\
3x-2x+3x&=3-5+2+6-4\Rightarrow\\
4x&=2\Rightarrow\\
\frac{4x}{4}&=\frac{2}{4}\Rightarrow x=\frac{1}{2}
\end{align*}
\Paradeigma{Εξίσωση με κλάσματα}
\bmath{Να λυθεί η παρακάτω εξίσωση
\[ \frac{x-1}{3}-\frac{x}{4}=1 \]}
\lysh\\
Ξεκινάμε υπολογίζοντας το Ε.Κ.Π. των παρονομαστών των κλασμάτων, ώστε να πολλαπλασιάσουμε όλους τους όρους της εξίσωσης. Έχουμε λοιπόν
\begin{gather*}
\textrm{Ε.Κ.Π.}(3,4)=12\ \textrm{ άρα}\\
12\cdot\frac{x-1}{3}-12\cdot\frac{x}{4}=12\cdot 1\Rightarrow 4(x-1)-3x=12
\end{gather*}
Η εξίσωση που προέκυψε περιέχει παρενθέσεις άρα συνεχίζουμε όπως και στο προηγούμενο παράδειγμα. Έτσι
\begin{align*}
4(x-1)-3x&=12\Rightarrow\\4x-4-3x&=12\Rightarrow\\
4x-3x&=12+4\Rightarrow\\ x&=16
\end{align*}
\Paradeigma{Εξίσωση με κλάσματα}
\bmath{Να λυθεί η εξίσωση: $ \frac{2x-1}{3}=\frac{4-x}{2} $.}\\\\
\lysh\\
Ένας εναλλακτικός και σύντομος τρόπος να γίνει απαλοιφή παρονομαστών, που εφαρμόζεται στις εξισώσεις με δύο όρους, είναι να πολλαπλασιάσουμε χιαστί τους όρους των κλασμάτων. Δηλαδή θα έχουμε
\begin{gather*}
\frac{2x-1}{3}=\frac{4-x}{2}\Rightarrow 2(2x-1)=3(4-x)\Rightarrow 4x-2=12-3x\Rightarrow\\ 4x+3x=12+2\Rightarrow 7x=14\Rightarrow x=2
\end{gather*}
\end{document}

