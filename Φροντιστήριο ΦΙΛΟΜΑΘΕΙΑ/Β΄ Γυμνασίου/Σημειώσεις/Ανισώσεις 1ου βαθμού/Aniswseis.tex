\PassOptionsToPackage{no-math,cm-default}{fontspec}
\documentclass[twoside,nofonts,internet,shmeiwseis]{thewria}
\usepackage{amsmath}
\usepackage{xgreek}
\let\hbar\relax
\defaultfontfeatures{Mapping=tex-text,Scale=MatchLowercase}
\setmainfont[Mapping=tex-text,Numbers=Lining,Scale=1.0,BoldFont={Minion Pro Bold}]{Minion Pro}
\newfontfamily\scfont{GFS Artemisia}
\font\icon = "Webdings"
\usepackage[amsbb]{mtpro2}
\usepackage{tikz,pgfplots}
\tkzSetUpPoint[size=7,fill=white]
\xroma{red!70!black}
%------TIKZ - ΣΧΗΜΑΤΑ - ΓΡΑΦΙΚΕΣ ΠΑΡΑΣΤΑΣΕΙΣ ----
\usepackage{tikz}
\usepackage{tkz-euclide}
\usetkzobj{all}
\usepackage[framemethod=TikZ]{mdframed}
\usetikzlibrary{decorations.pathreplacing}
\usepackage{pgfplots}
\usetkzobj{all}
%-----------------------

%-----ΕΙΚΟΝΑ ΔΙΠΛΑ ΑΠΟ ΚΕΙΜΕΝΟ-------
\usepackage{wrapfig}
\newenvironment{WrapText1}[3][r]
{\wrapfigure[#2]{#1}{#3}}
{\endwrapfigure}

\newenvironment{WrapText2}[3][l]
{\wrapfigure[#2]{#1}{#3}}
{\endwrapfigure}

\newcommand{\wrapr}[6]{
\begin{minipage}{\linewidth}\mbox{}\\
\vspace{#1}
\begin{WrapText1}{#2}{#3}
\vspace{#4}#5\end{WrapText1}#6
\end{minipage}}

\newcommand{\wrapl}[6]{
\begin{minipage}{\linewidth}\mbox{}\\
\vspace{#1}
\begin{WrapText2}{#2}{#3}
\vspace{#4}#5\end{WrapText2}#6
\end{minipage}}
%-------------------------------------------

\usepackage{calc}

\renewcommand{\thepart}{\arabic{part}}

\usepackage[explicit]{titlesec}
\usepackage{graphicx}
\usepackage{multicol}
\usepackage{multirow}
\usepackage{enumitem}
\usepackage{tabularx}
\usepackage{hhline}
\usepackage[decimalsymbol=comma]{siunitx}
\usetikzlibrary{backgrounds}
\usepackage{sectsty}
\sectionfont{\centering}
\usepackage{enumitem}
\setlist[enumerate]{label=\bf{\large \arabic*.}}
\usepackage{adjustbox}
%--------- ΑΓΓΛΙΚΟ ΚΕΙΜΕΝΟ --------------
\newcommand{\eng}[1]{\selectlanguage{english}#1\selectlanguage{greek}}
%----------------------------------------
%------- ΣΥΣΤΗΜΑ -------------------
\usepackage{systeme,regexpatch}
\makeatletter
% change the definition of \sysdelim not to store `\left` and `\right`
\def\sysdelim#1#2{\def\SYS@delim@left{#1}\def\SYS@delim@right{#2}}
\sysdelim\{. % reinitialize

% patch the internal command to use
% \LEFTRIGHT<left delim><right delim>{<system>}
% instead of \left<left delim<system>\right<right delim>
\regexpatchcmd\SYS@systeme@iii
{\cB.\c{SYS@delim@left}(.*)\c{SYS@delim@right}\cE.}
{\c{SYS@MT@LEFTRIGHT}\cB\{\1\cE\}}
{}{}
\def\SYS@MT@LEFTRIGHT{%
\expandafter\expandafter\expandafter\LEFTRIGHT
\expandafter\SYS@delim@left\SYS@delim@right}
\makeatother
\newcommand{\synt}[2]{{\scriptsize \begin{matrix}
\times#1\\\\ \times#2
\end{matrix}}}
%----------------------------------------
%------ ΜΗΚΟΣ ΓΡΑΜΜΗΣ ΚΛΑΣΜΑΤΟΣ ---------
\DeclareRobustCommand{\frac}[3][0pt]{%
{\begingroup\hspace{#1}#2\hspace{#1}\endgroup\over\hspace{#1}#3\hspace{#1}}}
%----------------------------------------
%-------- ΜΑΘΗΜΑΤΙΚΑ ΕΡΓΑΛΕΙΑ ---------
\usepackage{mathtools}
%----------------------

%-------- ΠΙΝΑΚΕΣ ---------
\usepackage{booktabs}
%----------------------
%----- ΥΠΟΛΟΓΙΣΤΗΣ ----------
\usepackage{calculator}
%----------------------------
%------ ΔΙΑΓΩΝΙΟ ΣΕ ΠΙΝΑΚΑ -------
\usepackage{array}
\newcommand\diag[5]{%
\multicolumn{1}{|m{#2}|}{\hskip-\tabcolsep
$\vcenter{\begin{tikzpicture}[baseline=0,anchor=south west,outer sep=0]
\path[use as bounding box] (0,0) rectangle (#2+2\tabcolsep,\baselineskip);
\node[minimum width={#2+2\tabcolsep-\pgflinewidth},
minimum  height=\baselineskip+#3-\pgflinewidth] (box) {};
\draw[line cap=round] (box.north west) -- (box.south east);
\node[anchor=south west,align=left,inner sep=#1] at (box.south west) {#4};
\node[anchor=north east,align=right,inner sep=#1] at (box.north east) {#5};
\end{tikzpicture}}\rule{0pt}{.71\baselineskip+#3-\pgflinewidth}$\hskip-\tabcolsep}}
%---------------------------------

%---- ΟΡΙΖΟΝΤΙΟ - ΚΑΤΑΚΟΡΥΦΟ - ΠΛΑΓΙΟ ΑΓΚΙΣΤΡΟ ------
\newcommand{\orag}[3]{\node at (#1)
{$ \overcbrace{\rule{#2mm}{0mm}}^{{\scriptsize #3}} $};}

\newcommand{\kag}[3]{\node at (#1)
{$ \undercbrace{\rule{#2mm}{0mm}}_{{\scriptsize #3}} $};}

\newcommand{\Pag}[4]{\node[rotate=#1] at (#2)
{$ \overcbrace{\rule{#3mm}{0mm}}^{{\rotatebox{-#1}{\scriptsize$#4$}}}$};}
%-----------------------------------------

%-------- ΤΡΙΓΩΝΟΜΕΤΡΙΚΟΙ ΑΡΙΘΜΟΙ -----------
\newcommand{\hm}[1]{\textrm{ημ}#1}
\newcommand{\syn}[1]{\textrm{συν}#1}
\newcommand{\ef}[1]{\textrm{εφ}#1}
\newcommand{\syf}[1]{\textrm{σφ}#1}
%--------------------------------------------

%--------- ΠΟΣΟΣΤΟ ΤΟΙΣ ΧΙΛΙΟΙΣ ------------
\DeclareRobustCommand{\perthousand}{%
\ifmmode
\text{\textperthousand}%
\else
\textperthousand
\fi}
%------------------------------------------

%------------------------------------------
\usepackage{extarrows}
\newcommand{\eq}[1]{\xlongequal{#1}}
%------------------------------------------
%------ ΌΡΙΣΜΑ ----------
\newcommand{\Arg}[8]{
\draw[-latex] (#7,#8)-- ++(#1:#2) node[right=#5]{\footnotesize$#4$};
\draw[fill=black!#6] (#7+0.3+#3,#8) arc (0:#1:0.3+#3) -- (#7,#8);}
%------------------------


\newcommand{\pinakasdyo}[8]{
\begin{tikzpicture}
\foreach \x in {#6,#7}{
\draw (-3,0) -- (#8,0);
\draw (\x,-.5)--(\x,.0);
\node[fill=white,inner sep=1pt] at (\x,-0.25) {$0$};}
\draw (-.5,0.5) -- (-.5,-0.5);
\node at (-.15,0.25) {$-\infty$};
\node at (#8-.3,0.25) {$+\infty$};
\node at (-1.75,0.25) {$x$};
\node at (-1.75,-0.3) {$#1$};
\node[fill=white,inner sep=1pt] at (#6,0.25) {$#2$};
\node[fill=white,inner sep=1pt] at (#7,0.25) {$#3$};
\node at (0.5*#6-0.25,-0.3) {$#4$};
\node at (0.5*#7+0.5*#6,-0.3) {$#5$};
\node at (0.5*#7+0.5*#8,-0.3) {$#4$};
\end{tikzpicture}}

\newcommand{\pinakasmia}[5]{
\begin{tikzpicture}
\draw (-3,0) -- (#5,0);
\draw (#4,-.5)--(#4,.0);
\node[fill=white,inner sep=1pt] at (#4,-0.25) {$0$};
\draw (-.5,0.5) -- (-.5,-0.5);
\node at (-.15,0.25) {$-\infty$};
\node at (#5-0.3,0.25) {$+\infty$};
\node at (-1.75,0.25) {$x$};
\node at (-1.75,-0.3) {$#1$};
\node[fill=white,inner sep=1pt] at (#4,0.25) {$#2$};
\node at (0.5*#4-0.25,-.3) {$#3$};
\node at (0.5*#4+0.5*#5,-0.3) {$#3$};
\end{tikzpicture}}

\newcommand{\pinakaskamia}[2]{
\begin{tikzpicture}
\draw (-3,0) -- (5,0);
\draw (-.5,0.5) -- (-.5,-0.5);
\node at (-.15,0.25) {$-\infty$};
\node at (4.7,0.25) {$+\infty$};
\node at (-1.75,0.25) {$x$};
\node at (-1.75,-0.3) {$#1$};
\node[fill=white,inner sep=1pt] at (2.25,-0.3) {$#2$};
\end{tikzpicture}}


%----------- ΓΡΑΦΙΚΕΣ ΠΑΡΑΣΤΑΣΕΙΣ ---------
\pgfkeys{/pgfplots/aks_on/.style={axis lines=center,
xlabel style={at={(current axis.right of origin)},xshift=1.5ex, anchor=center},
ylabel style={at={(current axis.above origin)},yshift=1.5ex, anchor=center}}}
\pgfkeys{/pgfplots/grafikh parastash/.style={black,line width=.4mm,samples=200}}
\pgfkeys{/pgfplots/belh ar/.style={tick label style={font=\scriptsize},axis line style={-latex}}}
%------------------------------------------

\newlist{rlist}{enumerate}{3}
\setlist[rlist]{itemsep=0mm,label=\roman*.}
\newlist{tropos}{enumerate}{3}
\setlist[tropos]{label=\bf\textit{\arabic*\textsuperscript{oς}\;Τρόπος :},leftmargin=0cm,itemindent=2.3cm,ref=\bf{\arabic*\textsuperscript{oς}\;Τρόπος}}
% Αν μπει το bhma μεσα σε tropo τότε
%\begin{bhma}[leftmargin=.7cm]
\newcommand{\tss}[1]{\textsuperscript{#1}}
\newcommand{\tssL}[1]{\MakeLowercase{\textsuperscript{#1}}}


\begin{document}
\titlos{ΜΑΘΗΜΑΤΙΚΑ Β΄ ΓΥΜΝΑΣΙΟΥ}{Εξισώσεις - Ανισώσεις}{Ανισώσεις 1\tssL{ου} Βαθμού}
\orismoi
\Orismos{Ανίσωση}
Ανίσωση ονομάζεται κάθε ανισότητα η οποία περιέχει τουλάχιστον μια μεταβλητή. Μια ανίσωση μιας μεταβλητής θα έχει τη μορφή
\[ ax+\beta>0\textrm{ ή }ax+\beta<0 \]
όπου $ a,\beta $ οποιοιδήποτε αριθμοί.
\begin{itemize}[itemsep=0mm]
\item Ανισώσεις αποτελούν και οι σχέσεις με σύμβολα ανισοϊσότητας $ \leq,\geq $.
\item Κάθε αριθμός που επαληθεύει μια ανίσωση ονομάζεται \textbf{λύση} της. Κάθε ανίσωση έχει λύσεις ένα \textbf{σύνολο αριθμών}.
\item Αν μια ανίσωση έχει λύσεις όλους τους αριθμούς ονομάζεται \textbf{αόριστη}.
\item Αν μια ανίσωση δεν έχει καθόλου λύσεις ονομάζεται \textbf{αδύνατη}.
\item Σχέσεις τις μορφής $ Q(x)\leq P(x)\leq R(x) $ λέγονται \textbf{διπλές ανισώσεις} όπου $ P(x),Q(x),\\R(x) $ είναι αλγεβρικές παρατάσεις. Αποτελείται από δύο ανισώσεις, με κοινό μέλος την παράσταση $ P(x) $, οι οποίες συναληθεύουν.
\item \textbf{Κοινές λύσεις} μιας διπλής ανίσωσης ή δύο ή περισσότερων ανισώσεων ονομάζονται οι αριθμοί που επαληθεύουν όλες τις ανισώσεις συγχρόνως.
\end{itemize}

\thewrhmata
\Thewrhma{ΙΔΙΟΤΗΤΕΣ ΑΝΙΣΟΤΗΤΩΝ}\label{th:idan}
\vspace{-5mm}
\begin{enumerate}
\item Εαν σε μια ανισότητα προσθέσουμε ή αφαιρέσουμε τον ίδιο αριθμό και απ' τα δύο μέλη της, προκύπτει ξανά ανισότητα με την ίδια φορά της αρχικής.
\[ a>\beta\Leftrightarrow\ccases{a+\gamma>\beta+\gamma\\a-\gamma>\beta-\gamma} \]
\item Για να πολλαπλασιάσουμε ή να διαιρέσουμε και τα δύο μέλη μιας ανισότητας με τον ίδιο αριθμό διακρίνουμε τις εξής περπτώσεις :
\begin{rlist}
\item Εαν πολλαπλασιάσουμε ή διαιρέσουμε και τα δύο μέλη μιας ανισότητας με τον ίδιο \textbf{θετικό} αριθμό, τότε προκύπτει ανισότητα με την \textbf{ίδια} φορά της αρχικής.
\item Εαν πολλαπλασιάσουμε ή διαιρέσουμε και τα δύο μέλη μιας ανισότητας με τον ίδιο \textbf{αρνητικό} αριθμό, τότε προκύπτει ανισότητα με φορά \textbf{αντίθετη} της αρχικής.
\end{rlist}
\begin{gather*}
\textrm{Αν }\gamma>0\textrm{ τότε }a>\beta\Leftrightarrow a\cdot\gamma>\beta\cdot\gamma\textrm{ και }\dfrac{a}{\gamma}>\dfrac{\beta}{\gamma}\\
\textrm{Αν }\gamma<0\textrm{ τότε }a>\beta\Leftrightarrow a\cdot\gamma<\beta\cdot\gamma\textrm{ και }\dfrac{a}{\gamma}<\dfrac{\beta}{\gamma}
\end{gather*}
\end{enumerate}
Ανάλογα συμπεράσματα ισχύουν και για τις ανισότητες $ a<\beta,a\geq\beta $ και $ a\leq\beta $.\\\\
\Thewrhma{Πράξεισ κατά μέλη ανισοτήτων}
Προσθέτοντας κατά μέλη κάθε ζεύγος ανισοτήτων προκύπτει ανισότητα, με 1\textsuperscript{ο} μέλος το άθροισμα των 1\textsuperscript{ων} μελών τους και 2\textsuperscript{ο} μέλος το άθροισμα των 2\textsuperscript{ων} μελών τους με φορά ίδια της αρχικής. Ομοίως πολλαπλασιάζοντας κατά μέλη δύο ανισότητες προκύπτει ανισότητα με φορά ίδια της αρχικής. Για να πολλαπλασιαστούν δύο ανισότητες κατά μέλη πρέπει όλοι οι όροι τους να είναι θετικοί.
\[ a>\beta\;\;\textrm{και}\;\;\gamma>\delta\Rightarrow\ccases{\textrm{\textbf{\textit{1. Πρόσθεση κατά μέλη }}}& a+\gamma>\beta+\delta\\\textrm{\textbf{\textit{2. Πολλαπλασιασμός κατά μέλη }}}& a\cdot\gamma>\beta\cdot\delta\;\;,\;\;\textrm{με }a,\beta,\gamma,\delta>0} \]
\textbf{Δεν} μπορούμε να αφαιρέσουμε ή να διαιρέσουμε ανισότητες κατά μέλη.\\\\
\end{document}