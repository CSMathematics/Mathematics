\documentclass[ektypwsh]{diag-xelatex}
\usepackage[amsbb]{mtpro2}
\usepackage[no-math,cm-default]{fontspec}
\usepackage{xunicode}
\usepackage{xltxtra}
\usepackage{xgreek}
\usepackage{amsmath}
\defaultfontfeatures{Mapping=tex-text,Scale=MatchLowercase}
\setmainfont[Mapping=tex-text,Numbers=Lining,Scale=1.0,BoldFont={Minion Pro Bold}]{Minion Pro}
\newfontfamily\scfont{GFS Artemisia}
\font\icon = "Webdings"
\usepackage[amsbb]{mtpro2}
\xroma{red!80!black}
\newcommand{\tss}[1]{\textsuperscript{#1}}
\newcommand{\tssL}[1]{\MakeLowercase\textsuperscript{#1}}
\newlist{rlist}{enumerate}{3}
\setlist[rlist]{itemsep=0mm,label=\roman*.}
\usepackage{multicol}
\usepackage{tikz}
\tikzstyle{pl}=[line width=0.3mm]

\begin{document}
\titlos{Μαθηματικά Β΄ Γυμνασίου}{Εξισώσεις}
\thewria
\begin{thema}
\item \textbf{Θεωρία}\\
Να απαντήσετε στις παρακάτω ερωτήσεις.
\begin{rlist}
\item Τι ονομάζεται εξίσωση;
\item Ποιά εξίσωση ονομάζεται αδύνατη;
\item Τι ονομάζεται λύση μιας ανίσωσης;
\item Ποιά εξίσωση ονομάζεται αόριστη;
\end{rlist}\monades{6}
\item \textbf{Σωστό - Λάθος / Πολλαπλαπλής επιλογής}\\
Α. \swstolathosd
\begin{rlist}
\item Η εξίσωση $ x+2=5 $ έχει λύση τον αριθμό $ 3 $.
\item Η ανίσωση $ x>x $ είναι αδύνατη.
\item Ο αριθμός $ 0 $ είναι λύση της εξίσωσης $ 0x=3 $.
\item Η εξίσωση $ x=x+2 $ δεν έχει λύσεις.
\item Η εξίσωση $ 2x+1>3 $ έχει λύση τον αριθμό $ 1 $.
\end{rlist}\monades{3}\\
Β. Να επιλέξετε τη σωστή απάντηση σε κάθεμία από τις παρακάτω ερωτήσεις.
\begin{rlist}
\item Ποιά από τις παρακάτω εξισώσεις είναι αδύνατη;
\begin{multicols}{4}
\begin{itemize}
\item $ 0x=0 $
\item $ 2x+1=4 $
\item $ x=0 $
\item $ x=x+1 $
\end{itemize}
\end{multicols}
\item Ποιά από τις παρακάτω ανισώσεις είναι αόριστη;
\begin{multicols}{4}
\begin{itemize}
\item $ 0x\leq-2 $
\item $ x+1>x+1 $
\item $ x<1 $
\item $ x+3>x+2 $
\end{itemize}
\end{multicols}
\item Ποιά από τις παρακάτω εξισώσεις έχει λύση τον αριθμό $ 1 $.
\begin{multicols}{4}
\begin{itemize}
\item $ 0x=1 $
\item $ 2x+1=2 $
\item $ 0x=0 $
\item $ 3x=0 $
\end{itemize}
\end{multicols}
\end{rlist}\monades{3}
\end{thema}
\newpage
\noindent
\askhseis
\begin{thema}
\item \mbox{}\\
Να λύσετε τις παρακάτω εξισώσεις και ανισώσεις.
\begin{rlist}
\item $ 5-3(x-3)=4-(3x+2)+5x $
\item $ 4x-3(x+2)\leq7-(x-4) $
\end{rlist}\monades{7}
\item \mbox{}\\
Να λύσετε τις παρακάτω εξισώσεις και ανισώσεις.
\begin{rlist}
\item $ \dfrac{3x-1}{5}-\dfrac{2x-4}{3}=4-x $
\item $ \dfrac{1}{2}(3-2x)-\dfrac{2}{5}(x+3)\geq\dfrac{4}{10}+x $
\end{rlist}\monades{7}
\item \mbox{}\\
Δίνεται το ορθογώνιο $ AB\varGamma\varDelta $ όπως αυτό φαίνεται στο παρακάτω σχήμα. Να βρεθούν οι αριθμοί $ x,y $ και $ \omega $.
\begin{center}
\begin{tikzpicture}
\draw[pl]  (-1.5,1) rectangle (3,-1.2);
\node at (0.8,1.4) {$\frac{x-3}{4}+\frac{9+2x}{5}$};
\node at (0.8,-1.6) {$2x-3$};
\node[rotate=90] at (-1.8,-0.2) {$3(y-2)+5$};
\node[rotate=-90] at (3.4,-0.2) {$4(3-y)-6$};
\draw  (2.6,-0.8) rectangle (3,-1.2);
\node at (1.5,-0.6) {{\footnotesize $4(\omega-20^o)+10^o$}};
\end{tikzpicture}
\end{center}\monades{7}
\end{thema}
\end{document}

