\documentclass[11pt,a4paper]{article}
\usepackage[amsbb,subscriptcorrection,zswash,mtpcal,mtphrb]{mtpro2}
\usepackage[no-math,cm-default]{fontspec}
\usepackage{xunicode}
\usepackage{xgreek}
\defaultfontfeatures{Mapping=tex-text,Scale=MatchLowercase}
\def\xrwma{red!70!black}
\def\xrwma{red!90!black}
\setmainfont[Mapping=tex-text,Numbers=Lining,Scale=1.0]{Minion Pro}
\newfontfamily\mpro{Minion Pro}
\usepackage{amsmath}
\usepackage[left=2.00cm, right=2.00cm, top=3.00cm, bottom=2.00cm]{geometry}
\usepackage{makeidx}
\usepackage{longtable}
\usepackage{etoolbox}
\makeatletter
\newif\ifLT@nocaption
\preto\longtable{\LT@nocaptiontrue}
\appto\endlongtable{%
\ifLT@nocaption
\addtocounter{table}{\m@ne}%
\fi}
\preto\LT@caption{%
\noalign{\global\LT@nocaptionfalse}}
\makeatother
\makeindex
\usepackage{tikz,pgfplots}
\usepackage{tkz-euclide,tkz-fct}
\usepackage{wrapfig}
\usetkzobj{all}
\usepackage{calc}
\usepackage{cleveref}
\usepackage[colorlinks=false, pdfborder={0 0 0}]{hyperref}
\usepackage[framemethod=TikZ]{mdframed}
\newcommand{\ypogrammisi}[1]{\underline{\smash{#1}}}
\usetikzlibrary{backgrounds}
\renewcommand{\thepart}{\arabic{part}}
\definecolor{steelblue}{cmyk}{.7,.278,0,.294}
\definecolor{doc}{cmyk}{1,0.455,0,0.569}
\definecolor{olivedrab}{cmyk}{0.25,0,0.75,0.44}
\usepackage{capt-of}
\usepackage{titletoc}
\usepackage[explicit]{titlesec}
\usepackage{graphicx}
\usepackage{multicol}
\usepackage{multirow}
\usepackage{enumitem}
\usepackage{tabularx}
\usepackage{mathimatika,tkz-tab,gensymb}
\usepackage[decimalsymbol=comma]{siunitx}
\tikzset{>=latex}
\makeatletter
\pretocmd{\@part}{\gdef\parttitle{#1}}{}{}
\pretocmd{\@spart}{\gdef\parttitle{#1}}{}{}
\makeatother
\usepackage[titletoc]{appendix}


\usepackage{venndiagram}
%-------- ΣΤΥΛ ΚΕΦΑΛΑΙΟΥ ---------
\newcommand*\chapterlabel{}
\newcommand{\fancychapter}{%
\titleformat{\chapter}
{
\normalfont\Huge}
{\gdef\chapterlabel{\thechapter\ }}{0pt}
{\begin{tikzpicture}[remember picture,overlay]
\node[yshift=-7cm] at (current page.north west)
{\begin{tikzpicture}[remember picture, overlay]
%\node[inner sep=0pt] at ($(current page.north) +			(0cm,-1.38in)$) {\includegraphics[width=17cm]{Kefalaio}};
\node[anchor=west,xshift=.08\paperwidth,yshift=.1\paperheight,rectangle]
{{\color{white}\fontsize{30}{20}\textbf{\textcolor{black}{\contour{white}{ΚΕΦΑΛΑΙΟ}}}}};
\node[anchor=west,xshift=.07\paperwidth,yshift=.05\paperheight,rectangle] {\fontsize{27}{20} {\color{black}{{\textcolor{black}{\contour{white}{\sc##1}}}}}};
%\fill[fill=black] (12.2,2) rectangle (14.8,4.7);
\node[anchor=west,xshift=.77\paperwidth,yshift=.077\paperheight,rectangle]
{\fontsize{80}{20}\textbf{\textit{\contour{black}{\thechapter}}}};
\end{tikzpicture}
};
\end{tikzpicture}
}
\titlespacing*{\chapter}{0pt}{20pt}{30pt}
}
%------------------------------------------------

\usepackage[outline]{contour}
\newcommand{\regularchapter}{%
\titleformat{\chapter}[display]
{\normalfont\huge\bfseries}{\chaptertitlename\ \thechapter}{20pt}{\Huge##1}
\titlespacing*{\chapter}
{0pt}{-20pt}{40pt}
}

\apptocmd{\mainmatter}{\fancychapter}{}{}
\apptocmd{\backmatter}{\regularchapter}{}{}
\apptocmd{\frontmatter}{\regularchapter}{}{}

\titlespacing*{\section}
{0pt}{30pt}{0pt}
\usepackage{booktabs}
\usepackage{hhline}
\DeclareRobustCommand{\perthousand}{%
\ifmmode
\text{\textperthousand}%
\else
\textperthousand
\fi}



\contentsmargin{0cm}
\titlecontents{part}[-1pc]
{\addvspace{10pt}%
\bf\Large ΜΕΡΟΣ\quad }%
{}
{}
{\;\dotfill\;\normalsize\ Σελίδα}%
%------------------------------------------
\titlecontents{chapter}[0pc]
{\addvspace{30pt}%
\begin{tikzpicture}[remember picture, overlay]%
\draw[fill=black,draw=black] (-.3,.5) rectangle (3.7,1.1); %
\pgftext[left,x=0cm,y=0.75cm]{\color{white}\sc\Large\bfseries Κεφάλαιο\ \thecontentslabel};%
\end{tikzpicture}\large\sc}%
{}
{}
{\hspace*{-2.3em}\hfill\normalsize Σελίδα \thecontentspage}%
\titlecontents{section}[2.4pc]
{\addvspace{1pt}}
{\contentslabel[\thecontentslabel]{2pc}}
{}
{\;\dotfill\;\small \thecontentspage}
[]
\titlecontents*{subsection}[4pc]
{\addvspace{-1pt}\small}
{}
{}
{\ --- \small\thecontentspage}
[ \textbullet\ ][]

\makeatletter
\renewcommand{\tableofcontents}{%
\chapter*{%
\vspace*{-20\p@}%
\begin{tikzpicture}[remember picture, overlay]%
\pgftext[right,x=12cm,y=0.2cm]{\Huge\sc\bfseries \contentsname};%
\draw[fill=black,draw=black] (9.5,-.75) rectangle (12.5,1);%
\clip (9.5,-.75) rectangle (15,1);
\pgftext[right,x=12cm,y=0.2cm]{\color{white}\Huge\bfseries \contentsname};%
\end{tikzpicture}}%
\@starttoc{toc}}
\makeatother
\pgfmathdeclarefunction{gauss}{2}{%
\pgfmathparse{1/(#2*sqrt(2*pi))*exp(-((x-#1)^2)/(2*#2^2))}%
}
\usepackage[contents={},scale=1,opacity=1,color=black,angle=0]{background}

\newcommand\blfootnote[1]{%
\begingroup
\renewcommand\thefootnote{}\footnote{#1}%
\addtocounter{footnote}{-1}%
\endgroup
}
\usepackage{epstopdf}
\epstopdfsetup{update}
\usepackage{textcomp}



\usepackage[labelfont={footnotesize,it,bf},font={footnotesize}]{caption}

\usepackage{wrapfig,wrap-rl}
%-------- ΜΑΘΗΜΑΤΙΚΑ ΕΡΓΑΛΕΙΑ ---------
\usepackage{mathtools}
%----------------------
%-------- ΠΙΝΑΚΕΣ ---------
\usepackage{booktabs}
%----------------------
%----- ΥΠΟΛΟΓΙΣΤΗΣ ----------
%\usepackage{calculator}
%----------------------------
\newcommand{\tss}[1]{\textsuperscript{#1}}
\newcommand{\tssL}[1]{\MakeLowercase{\textsuperscript{#1}}}
%----- ΟΡΙΖΟΝΤΙΑ ΛΙΣΤΑ ------
\usepackage{xparse}
\newcounter{answers}
\renewcommand\theanswers{\arabic{answers}}
\ExplSyntaxOn
\NewDocumentCommand{\results}{m}
{
\seq_set_split:Nnn \l_results_a_seq {,}{#1}
\par\nobreak\noindent\setcounter{answers}{0}
\seq_map_inline:Nn \l_results_a_seq
{
\makebox[.18\linewidth][l]{\stepcounter{answers}\theanswers.~##1}\hfill
}
\par
}
\seq_new:N \l_results_a_seq
\ExplSyntaxOff
%----------------------------

\usepackage{microtype}
\usepackage{float}
\usepackage{caption}
%----------- ΓΡΑΦΙΚΕΣ ΠΑΡΑΣΤΑΣΕΙΣ ---------
\pgfkeys{/pgfplots/aks_on/.style={axis lines=center,
xlabel style={at={(current axis.right of origin)},xshift=1.5ex, anchor=center},
ylabel style={at={(current axis.above origin)},yshift=1.5ex, anchor=center}}}
\pgfkeys{/pgfplots/grafikh parastash/.style={black,line width=.4mm,samples=200}}
\pgfkeys{/pgfplots/belh ar/.style={tick label style={font=\scriptsize},axis line style={-latex}}}
%-----------------------------------------

%---- ΟΡΙΖΟΝΤΙΟ - ΚΑΤΑΚΟΡΥΦΟ - ΠΛΑΓΙΟ ΑΓΚΙΣΤΡΟ ------
\newcommand{\orag}[3]{\node at (#1)
{$ \overcbrace{\rule{#2mm}{0mm}}^{{\scriptsize #3}} $};}

\newcommand{\kag}[3]{\node at (#1)
{$ \undercbrace{\rule{#2mm}{0mm}}_{{\scriptsize #3}} $};}

\newcommand{\Pag}[4]{\node[rotate=#1] at (#2)
{$ \overcbrace{\rule{#3mm}{0mm}}^{{\rotatebox{-#1}{\scriptsize$#4$}}}$};}
%-----------------------------------------
\tikzstyle{pl}=[line width=0.3mm]
\tikzstyle{plm}=[line width=0.4mm]
\tkzSetUpPoint[size=7,fill=white]
\newlist{rlist}{enumerate}{3}
\setlist[rlist]{itemsep=0mm,label=\roman*.}
\setlist[itemize]{itemsep=0mm}





\begin{document}
\begin{enumerate}
\item \textbf{Εξίσωση} : Η ισότητα που περιέχει τουλάχιστον μια μεταβλητή.
\item \textbf{Λύση εξίσωσης} : Ο αριθμός που την επαληθεύει.
\item \textbf{Αόριστη} : Η εξίσωση με λύσεις όλους τους αριθμούς.
\item \textbf{Αδύνατη} : Η εξίσωση χωρίς λύση.
\item \textbf{Τετραγωνική Ρίζα του $ x $} : Ο \textbf{θετικός} αριθμός $ a $ που αν υψωθεί στο τετράγωνο δίνει τον αριθμό $ x $.
\[ \left(\sqrt{x}\;\right)^2=x\;\;,\;\; x\geq0 \]
\item \textbf{Συνάρτηση} : H σχέση που συνδέει δύο ποσά $ x,y $ όπου \textbf{κάθε} τιμή της μεταβλητής $ x $ αντιστοιχεί σε \textbf{μια μόνο} τιμή της μεταβλητής $ y $.
\item \textbf{Η συνάρτηση $ y=ax $} Η συνάρτηση που συνδέει δύο \textbf{ανάλογα} ποσά $ x,y $. \begin{rlist}
\item Η γραφική της παράσταση είναι ευθεία γραμμή που διέρχεται από την αρχή των αξόνων.
\item Ο $ a $ ονομάζεται \textbf{κλίση} της ευθείας. Ισούται με $ a=\frac{y}{x} $.
\end{rlist}
\item \textbf{Η συνάρτηση $ y=ax+\beta $} : Παριστάνει ευθεία γραμμή η οποία είναι παράλληλη με την ευθεία $ y=ax $. 
\item \textbf{Εμβαδά βασικών σχημάτων} :
\begin{center}
\begin{tikzpicture}[scale=.7]
\tkzDefPoint(0,0){D}
\tkzDefPoint(3,0){C}
\tkzDefPoint(3,3){B}
\tkzDefPoint(0,3){A}
\draw[pl] (A)--(B)--(C)--(D)--cycle;
\tkzDrawPoints(A,B,C,D)
\tkzLabelPoint[above left](A){$A$}
\tkzLabelPoint[above right](B){$B$}
\tkzLabelPoint[below right](C){$\varGamma$}
\tkzLabelPoint[below left](D){$\varDelta$}
\node at (1.5,1.5) {$E=a^2$};
\node at (1.5,-0.25) {$a$};
\node at (3.25,1.5) {$a$};
\node at (1.5,3.25) {$a$};
\node at (-0.25,1.5) {$a$};
\node at (1.5,4) {Τετράγωνο};
\end{tikzpicture}\quad\begin{tikzpicture}[scale=.7]
\tkzDefPoint(0,0){D}
\tkzDefPoint(4,0){C}
\tkzDefPoint(4,3){B}
\tkzDefPoint(0,3){A}
\draw[pl] (A)--(B)--(C)--(D)--cycle;
\tkzDrawPoints(A,B,C,D)
\tkzLabelPoint[above left](A){$A$}
\tkzLabelPoint[above right](B){$B$}
\tkzLabelPoint[below right](C){$\varGamma$}
\tkzLabelPoint[below left](D){$\varDelta$}
\node at (2,1.5) {$E=a\cdot\beta$};
\node at (2,-0.5) {$\beta$};
\node at (4.25,1.5) {$a$};
\node at (2,3.25) {$\beta$};
\node at (-0.25,1.5) {$a$};
\node at (2,4) {Ορθογώνιο};
\end{tikzpicture}\quad\begin{tikzpicture}[scale=.7]
\tkzDefPoint(0,0){D}
\tkzDefPoint(4,0){C}
\tkzDefPoint(5,3){B}
\tkzDefPoint(1,3){A}
\tkzDefPoint(1.5,0){a}
\tkzDefPoint(1.5,3){b}
\tkzDefPoint(4.5,1.5){c}
\tkzDefPoint(.9,2.7){d}
\tkzMarkRightAngle(C,a,b)
\tkzMarkRightAngle(D,d,c)
\draw[pl] (A)--(B)--(C)--(D)--cycle;
\tkzDrawPoints(A,B,C,D)
\tkzLabelPoint[above left](A){$A$}
\tkzLabelPoint[above right](B){$B$}
\tkzLabelPoint[below right](C){$\varGamma$}
\tkzLabelPoint[below left](D){$\varDelta$}
\node at (3,1.5) {$E=a\cdot\upsilon_a$};
\node at (3,1) {$E=\beta\cdot\upsilon_\beta$};
\node at (2,-0.5) {$\beta$};
\node at (5,1.5) {$a$};
\node at (3,3.25) {$\beta$};
\node at (0,1.5) {$a$};
\draw (a) -- (b);
\draw (c) -- (d);
\node at (2.5,2.35) {\footnotesize$\upsilon_a$};
\node at (1.25,0.5) {\footnotesize$\upsilon_\beta$};
\node at (2.5,4) {Παραλληλόγραμμο};
\end{tikzpicture}
\end{center}
\begin{center}
\begin{tikzpicture}
\clip (-1.5,-1.5) rectangle (4.5,3.5);
\begin{scope}
\clip (-1,-1.5) rectangle (4.5,3.5);
\tkzDefPoint(0,0){B}
\tkzDefPoint(3.5,0){C}
\tkzDefPoint(1.,2.1){A}
\tkzDefPointBy[projection = onto A--B](C) \tkzGetPoint{M}
\tkzDefPointBy[projection = onto A--C](B) \tkzGetPoint{L}
\tkzDefPoint(1,0){K}
\tkzInterLL(A,K)(B,L)\tkzGetPoint{H}
\tkzMarkRightAngle[size=.2](C,K,A)
\tkzMarkRightAngle[size=.2](C,M,A)
\tkzMarkRightAngle[size=.2](B,L,A)
\draw[pl](A)--(B)--(C)--cycle;
\tkzDrawAltitude[draw=black](A,B)(C)
\tkzDrawAltitude[draw=black](A,C)(B)
\tkzDrawAltitude[draw=black](B,C)(A)
\tkzDrawPoints(A,B,C,K,L,M)
\tkzLabelPoint[above](A){$A$}
\tkzLabelPoint[left](B){$B$}
\tkzLabelPoint[right](C){$\varGamma$}
\tkzLabelPoint[below](K){$K$}
\tkzLabelPoint[right,yshift=1mm](L){$\varLambda$}
\tkzLabelPoint[left](M){$M$}
\node at (1.25,0.5) {\footnotesize$\upsilon_a$};
\node at (1.35,1.25) {\footnotesize$\upsilon_\beta$};
\node at (2,0.5) {\footnotesize$\upsilon_\gamma$};
\end{scope}
\node at (1.5,-0.75) {$ E=\dfrac{1}{2}a\cdot\upsilon_a=\dfrac{1}{2}\beta\cdot\upsilon_\beta=\dfrac{1}{2}\gamma\cdot\upsilon_\gamma $};
\node at (1,3) {Τρίγωνο};
\end{tikzpicture}\quad
\begin{tikzpicture}
\tkzDefPoint(0,0){B}
\tkzDefPoint(3.5,0){C}
\tkzDefPoint(0,2.1){A}
\tkzMarkRightAngle[size=.2](C,B,A)
\draw[pl](A)--(B)--(C)--cycle;
\tkzDrawPoints(A,B,C)
\tkzLabelPoint[above left](A){$B$}
\tkzLabelPoint[left](B){$A$}
\tkzLabelPoint[right](C){$\varGamma$}
\node at (1.75,-0.25) {\footnotesize$\beta$};
\node at (-0.25,1) {\footnotesize$\gamma$};
\node at (1,-1){$ E=\dfrac{\beta\cdot\gamma}{2} $};
\node at (1.5,3) {Ορθογώνιο τρίγωνο};
\end{tikzpicture}\quad
\begin{tikzpicture}
\tkzDefPoint(0,-1.5){D}
\tkzDefPoint(0.5,.5){A}
\tkzDefPoint(2.5,.5){B}
\tkzDefPoint(3.5,-1.5){C}
\tkzDefPoint(0.9,0.5){E}
\tkzDefPoint(0.9,-1.5){Z}
\tkzMarkRightAngle(C,Z,E)
\draw (0.9,0.5) -- (0.9,-1.5);
\draw[pl] (0,-1.5) -- (0.5,0.5) -- (2.5,0.5) -- (3.5,-1.5) -- cycle;
\tkzLabelPoint[above](A){$A$}
\tkzLabelPoint[above](B){$B$}
\tkzLabelPoint[below](C){$\varGamma$}
\tkzLabelPoint[below](D){$\varDelta$}
\tkzDrawPoints(A,B,C,D)
\node at (1.5,0.7) {\footnotesize$\beta$};
\node at (1.7,-1.8) {\footnotesize$B$};
\node at (.7,-.2) {\footnotesize$ \upsilon $};
\node at (1.75,-2.7){$ E=\dfrac{(\beta+B)\cdot\upsilon}{2} $};
\node at (1.5,1.5) {Τραπέζιο};
\end{tikzpicture}
\end{center}
\item \textbf{Πυθαγόρειο θεώρημα}\\
\wrapr{-7mm}{7}{4.5cm}{-9mm}{\begin{tikzpicture}
\tkzDefPoint(0,0){B}
\tkzDefPoint(3.5,0){C}
\tkzDefPoint(0,2.1){A}
\tkzMarkRightAngle[size=.2](C,B,A)
\draw[pl](A)--(B)--(C)--cycle;
\tkzDrawPoints(A,B,C)
\tkzLabelPoint[above left](A){$B$}
\tkzLabelPoint[left](B){$A$}
\tkzLabelPoint[right](C){$\varGamma$}
\node at (1.75,-0.25) {\footnotesize$\beta$};
\node at (-0.25,1) {\footnotesize$\gamma$};
\node at (2,1.25) {\footnotesize$a$};
\end{tikzpicture}}{
Σε κάθε ορθογώνιο τρίγωνο το τετράγωνο της υποτείνουσας ισούται με το άθροισμα των τετραγώνων των δύο κάθετων πλευρών.
\[ B\varGamma^2=AB^2+A\varGamma^2\ \ \textrm{ή}\ \ a^2=\beta^2+\gamma^2 \]}
\item \textbf{Αντίστροφο Πυθαγορείου θεωρήματος}\\
Αν το τετράγωνο της μεγαλύτερης πλευράς ενός τριγώνου ισούται με το άθροισμα των τετραγώνων των δύο άλλων πλευρών τότε το τρίγωνο έιναι ορθογώνιο. Η ορθή γωνία βρίσκεται απέναντι από τη μεγαλύτερη πλευρά.
\[ \textrm{Αν }\ B\varGamma^2=AB^2+A\varGamma^2\Rightarrow\ \hat{A}=90\degree \]
\item \textbf{Τριγωνομετρικοί αριθμοί} : \\
\wrapr{-11mm}{7}{4.5cm}{0mm}{\begin{tikzpicture}
\tkzDefPoint(0,0){B}
\tkzDefPoint(3.5,0){C}
\tkzDefPoint(0,2.1){A}
\tkzMarkRightAngle[size=.2](C,B,A)
\tkzMarkAngle[size=.3](B,A,C)
\draw[pl](A)--(B)--(C)--cycle;
\tkzDrawPoints(A,B,C)
\tkzLabelPoint[above left](A){$B$}
\tkzLabelPoint[left](B){$A$}
\tkzLabelPoint[right](C){$\varGamma$}
\node at (1.75,-0.25) {\footnotesize$\beta$};
\node at (-0.25,1) {\footnotesize$\gamma$};
\node at (2,1.25) {\footnotesize$a$};
\node at (0.2589,1.6348) {\footnotesize$\omega$};
\end{tikzpicture}}{
\begin{rlist}
\item $ \textrm{Εφαπτομένη}=\dfrac{\textrm{Απέναντι Κάθετη}}{\textrm{Προσκείμενη Κάθετη}}\;\;,\;\;\ef{\omega}=\dfrac{A\varGamma}{AB} $
\item $ \textrm{Ημίτονο}=\dfrac{\textrm{Απέναντι Κάθετη}}{\textrm{Υποτείνουσα}}\;\;,\;\;\hm{\omega}=\dfrac{A\varGamma}{B\varGamma} $
\item $ \textrm{Συνημίτονο}=\dfrac{\textrm{Προσκείμενη Κάθετη}}{\textrm{Υποτείνουσα}}\;\;,\;\;\syn{\omega}=\dfrac{AB}{B\varGamma} $
\end{rlist}}
\end{enumerate}
\end{document}