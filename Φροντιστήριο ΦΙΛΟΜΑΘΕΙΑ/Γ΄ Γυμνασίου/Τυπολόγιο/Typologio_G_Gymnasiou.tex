\PassOptionsToPackage{no-math,cm-default}{fontspec}
\documentclass[twoside,nofonts,internet,shmeiwseis]{thewria}
\usepackage{amsmath}
\usepackage{xgreek}
\let\hbar\relax
\defaultfontfeatures{Mapping=tex-text,Scale=MatchLowercase}
\setmainfont[Mapping=tex-text,Numbers=Lining,Scale=1.0,BoldFont={Minion Pro Bold}]{Minion Pro}
\newfontfamily\scfont{GFS Artemisia}
\font\icon = "Webdings"
\usepackage[amsbb,subscriptcorrection,zswash,mtpcal,mtphrb]{mtpro2}
\usepackage{tikz,pgfplots}
\tkzSetUpPoint[size=7,fill=white]
\xroma{red!70!black}
%------TIKZ - ΣΧΗΜΑΤΑ - ΓΡΑΦΙΚΕΣ ΠΑΡΑΣΤΑΣΕΙΣ ----
\usepackage{tikz}
\usepackage{tkz-euclide}
\usetkzobj{all}
\usepackage[framemethod=TikZ]{mdframed}
\usetikzlibrary{decorations.pathreplacing}
\usepackage{pgfplots}
\usetkzobj{all}
%-----------------------
\usepackage{calc}
\usepackage{hhline}
\usepackage[explicit]{titlesec}
\usepackage{graphicx}
\usepackage{multicol}
\usepackage{multirow}
\usepackage{enumitem}
\usepackage{tabularx}
\usepackage[decimalsymbol=comma]{siunitx}
\usetikzlibrary{backgrounds}
\usepackage{sectsty}
\sectionfont{\centering}
\setlist[enumerate]{label=\bf{\large \arabic*.}}
\usepackage{adjustbox}
\usepackage{mathimatika,gensymb,eurosym,wrap-rl}
\usepackage{systeme,regexpatch}
%-------- ΜΑΘΗΜΑΤΙΚΑ ΕΡΓΑΛΕΙΑ ---------
\usepackage{mathtools}
%----------------------
%-------- ΠΙΝΑΚΕΣ ---------
\usepackage{booktabs}
%----------------------
%----- ΥΠΟΛΟΓΙΣΤΗΣ ----------
\usepackage{calculator}
%----------------------------
%------ ΔΙΑΓΩΝΙΟ ΣΕ ΠΙΝΑΚΑ -------
\usepackage{array}
\newcommand\diag[5]{%
\multicolumn{1}{|m{#2}|}{\hskip-\tabcolsep
$\vcenter{\begin{tikzpicture}[baseline=0,anchor=south west,outer sep=0]
\path[use as bounding box] (0,0) rectangle (#2+2\tabcolsep,\baselineskip);
\node[minimum width={#2+2\tabcolsep-\pgflinewidth},
minimum  height=\baselineskip+#3-\pgflinewidth] (box) {};
\draw[line cap=round] (box.north west) -- (box.south east);
\node[anchor=south west,align=left,inner sep=#1] at (box.south west) {#4};
\node[anchor=north east,align=right,inner sep=#1] at (box.north east) {#5};
\end{tikzpicture}}\rule{0pt}{.71\baselineskip+#3-\pgflinewidth}$\hskip-\tabcolsep}}
%---------------------------------
%---- ΟΡΙΖΟΝΤΙΟ - ΚΑΤΑΚΟΡΥΦΟ - ΠΛΑΓΙΟ ΑΓΚΙΣΤΡΟ ------
\newcommand{\orag}[3]{\node at (#1)
{$ \overcbrace{\rule{#2mm}{0mm}}^{{\scriptsize #3}} $};}
\newcommand{\kag}[3]{\node at (#1)
{$ \undercbrace{\rule{#2mm}{0mm}}_{{\scriptsize #3}} $};}
\newcommand{\Pag}[4]{\node[rotate=#1] at (#2)
{$ \overcbrace{\rule{#3mm}{0mm}}^{{\rotatebox{-#1}{\scriptsize$#4$}}}$};}
%-----------------------------------------
%------------------------------------------
\newcommand{\tss}[1]{\textsuperscript{#1}}
\newcommand{\tssL}[1]{\MakeLowercase{\textsuperscript{#1}}}
%---------- ΛΙΣΤΕΣ ----------------------
\newlist{bhma}{enumerate}{3}
\setlist[bhma]{label=\bf\textit{\arabic*\textsuperscript{o}\;Βήμα :},leftmargin=0cm,itemindent=1.8cm,ref=\bf{\arabic*\textsuperscript{o}\;Βήμα}}
\newlist{rlist}{enumerate}{3}
\setlist[rlist]{itemsep=0mm,label=\roman*.}
\newlist{brlist}{enumerate}{3}
\setlist[brlist]{itemsep=0mm,label=\bf\roman*.}
\newlist{tropos}{enumerate}{3}
\setlist[tropos]{label=\bf\textit{\arabic*\textsuperscript{oς}\;Τρόπος :},leftmargin=0cm,itemindent=2.3cm,ref=\bf{\arabic*\textsuperscript{oς}\;Τρόπος}}
% Αν μπει το bhma μεσα σε tropo τότε
%\begin{bhma}[leftmargin=.7cm]
\tkzSetUpPoint[size=7,fill=white]
\tikzstyle{pl}=[line width=0.3mm]
\tikzstyle{plm}=[line width=0.4mm]
\usepackage{etoolbox}
\makeatletter
\renewrobustcmd{\anw@true}{\let\ifanw@\iffalse}
\renewrobustcmd{\anw@false}{\let\ifanw@\iffalse}\anw@false
\newrobustcmd{\noanw@true}{\let\ifnoanw@\iffalse}
\newrobustcmd{\noanw@false}{\let\ifnoanw@\iffalse}\noanw@false
\renewrobustcmd{\anw@print}{\ifanw@\ifnoanw@\else\numer@lsign\fi\fi}
\makeatother

\begin{document}
\section{Αλγεβρικές Παραστάσεις}
\begin{enumerate}
\item \textbf{Μονώνυμο:} Αλγεβρική παράσταση που περιέχει μόνο πολλαπλασιασμό. Παράδειγμα : $ 2x^3y^4 $
\begin{rlist}
\item Ο αριθμός λέγεται \textbf{συντελεστής}.
\item Οι μεταβλητές αποτελούν το \textbf{κύριο μέρος}.
\end{rlist}
\item \textbf{Όμοια μονώνυμα:} Τα μονώνυμα με ίδιο κύριο μέρος.
\item \textbf{Ίσα μονώνυμα:} Τα μονώνυμα με ίδιο κύριο μέρος και ίδιους συντελεστές.
\item \textbf{Αντίθετα μονώνυμα:} Τα μονώνυμα με ίδιο κύριο μέρος και αντίθετους συντελεστές.
\item \textbf{Πολυώνυμο:} Άθροισμα ανόμοιων μονωνύμων. Παράδειγμα:
\[ 3x^2y+4x^3z-xy^4 \]
\begin{rlist}
\item Κάθε προσθετέος μέσα σε ένα πολυώνυμο λέγεται \textbf{όρος}.
\item Ο μεγαλύτερος εκθέτης μιας μεταβλητής λέγεται \textbf{βαθμός} του πολυωνύμου.
\end{rlist}
\item \textbf{Ταυτότητα:} Μια ισότητα που έχει μεταβλητές και επαληθεύεται πάντα. Βασικές ταυτότητες:
\begin{rlist}
\item $ \left(a+\beta \right)^2=a^2+2a\beta+\beta^2  $
\item $ \left(a-\beta \right)^2=a^2-2a\beta+\beta^2  $
\item $ \left(a+\beta \right)^3=a^3+3a^2\beta+3a\beta^2+\beta^3 $
\item $ \left(a-\beta \right)^3=a^3-3a^2\beta+3a\beta^2-\beta^3 $
\item $ \left(a+\beta \right)\cdot\left(a-\beta \right)=a^2-\beta^2  $
\item $ \left(a+\beta \right)\cdot\left(a^2-a\cdot\beta+\beta^2\right)=a^3+\beta^3  $
\item $ \left(a-\beta \right)\cdot\left(a^2+a\cdot\beta+\beta^2\right)=a^3-\beta^3  $
\end{rlist}
\item \textbf{Παραγοντοποίηση:} Διαδικασία με την οποία μετατρέπουμε το άθροισμα σε γινόμενο. Βασικοί τρόποι:
\begin{rlist}
\item \textbf{Κοινός παράγοντας:} Από αριθμούς βγάζουμε το Μ.Κ.Δ. και από μεταβλητές βγάζουμε τις κοινές στη μικρότερη δύναμη.
\item \textbf{Ομαδοποίηση:} Μοιράζουμε την παράσταση σε ομάδες και από κάθε ομάδα βγάζουμε κοινό παράγοντα.
\item \textbf{Διαφορά τετραγώνων:} $ a^2-\beta^2=(a-\beta)(a+\beta) $.
\item \textbf{Ανάπτυγμα τετραγώνου:} $ a^2\pm 2a\beta+\beta^2=(a\pm\beta)^2 $
\end{rlist}
\item \begin{rlist}
\item \textbf{Ε.Κ.Π. αλγεβρικών παραστάσεων:} Επιλέγουμε όλους τους παράγοντες στη μεγαλύτερη δύναμη.
\item \textbf{Μ.Κ.Δ. αλγεβρικών παραστάσεων:} Επιλέγουμε τους κοινούς παράγοντες στη μικρότερη δύναμη.
\end{rlist}
Αν έχουμε πολυώνυμα πρώτα παραγοντοποιούμε.
\item \textbf{Ρητή αλγεβρική παράσταση:} Μια αλγ. παράσταση που έχει τη μορφή κλάσματος. (Πρέπει ο παρονομαστής να είναι διάφορος του μηδέν.)
\end{enumerate}
\newpage
\noindent
\section{Εξισώσεις - Ανισώσεις}
\begin{enumerate}
\item \textbf{Εξίσωση 2ου βαθμού:} $ ax^2+\beta x+\gamma=0\ \ ,\ \ a\neq0 $.
\begin{rlist}
\item  Οι $ a,\beta,\gamma $ λέγονται \textbf{συντελεστές}.
\item Ο αριθμός $ \varDelta=\beta^2-4a\gamma $ λέγεται \textbf{διακρίνουσα}.
\item Οι λύσεις μια εξίσωσης 2ου βαθμού φαίνονται στον παρακάτω πίνακα:
\begin{center}
\begin{tabular}{ccc}
\hline\textbf{Διακρίνουσα} & \textbf{Πλήθος λύσεων} & \textbf{Λύσεις} \rule[-2ex]{0pt}{5.5ex}\\ 
\hhline{===}\rule[-2ex]{0pt}{7ex} $ \varDelta>0 $ &  2 λύσεις & $ x_{1,2}=\dfrac{-\beta\pm\!\sqrt{\varDelta}}{2a} $  \\
\rule[-2ex]{0pt}{5.5ex} $ \varDelta=0 $ & 1 διπλή λύση & $ x=-\dfrac{\beta}{2a} $\\
\rule[-2ex]{0pt}{5.5ex} $ \varDelta<0 $ & \multicolumn{2}{c}{Καμία λύση}\\
\hline 
\end{tabular}
\end{center}
\end{rlist}
\item \textbf{Ανίσωση:}  Μια ανισότητα που περιέχει μεταβλητές.\\
\textbf{Ιδιότητες ανισοτήτων:}
\begin{rlist}
\item $ a>\beta\Leftrightarrow\ccases{a+\gamma>\beta+\gamma\\a-\gamma>\beta-\gamma} $
\item $
\textrm{Αν }\gamma>0\textrm{ τότε }a>\beta\Leftrightarrow a\cdot\gamma>\beta\cdot\gamma\textrm{ και }\dfrac{a}{\gamma}>\dfrac{\beta}{\gamma}$ \\ $
\textrm{Αν }\gamma<0\textrm{ τότε }a>\beta\Leftrightarrow a\cdot\gamma<\beta\cdot\gamma\textrm{ και }\dfrac{a}{\gamma}<\dfrac{\beta}{\gamma}
$
\end{rlist}
\end{enumerate}\mbox{}\\
\section{Συστήματα Εξισώσεων}
\begin{enumerate}
\item \textbf{Γραμμική εξίσωση:} Μια εξίσωση της μορφής $ ax+\beta y=\gamma $.
\begin{rlist}
\item Έχει $ 2 $ μεταβλητές $ x,y $.
\item Οι $ a,\beta,\gamma $ είναι γνωστοί αριθμοί.
\item Οι λύσεις είναι ζευγάρια αριθμών $ (x,y) $.
\item Αν $ a\neq0 $ ή $ \beta\neq0 $ τότε η εξίσωση παριστάνει ευθεία γραμμή.
\item Κάθε εξίσωση $ y=a $ παριστάνει \textbf{οριζόντια} ευθεία που περνάει από το σημείο $ (0,a) $.
\item Κάθε εξίσωση $ x=a $ παριστάνει \textbf{κατακόρυφη} ευθεία που περνάει από το σημείο $ (a,0) $.
\end{rlist}
\item \textbf{Σύστημα γραμμικών εξισώσεων:} Δύο εξισώσεις με δύο άγνωστους $ x,y $ γραμμένες μαζί. Παράδειγμα:
\[ \systeme[xy]{2x+3 y=5,x-4y=-3} \]
\begin{rlist}
\item Το ζευγάρι αριθμών $ (x,y) $ που επαληθεύει και τις δύο εξισώσεις είναι η \textbf{λύση του συστήματος}.
\item Κάθε εξίσωση είναι ευθεία γραμμή. Το κοινό σημείο των ευθειών είναι η λύση.
\item Αν ένα σύστημα δεν έχει λύσεις είναι \textbf{αδύνατο}. (Οι ευθείες είναι παράλληλες.)
\item Αν έχει άπειρες λύσεις είναι \textbf{αόριστο}. (Οι ευθείες ταυτίζονται.)
\end{rlist}
\item \textbf{Μέθοδοι για να λύσουμε σύστημα:}
\begin{rlist}
\item Μέθοδος της αντικατάστασης
\item Μέθοδος των αντίθετων συντελεστών
\end{rlist}
\end{enumerate}
\newpage
\noindent
\section{Πιθανότητες}
\begin{enumerate}
\item \textbf{Σύνολο:} Μια ομάδα από όμοια αντικείμενα. Τα αντικείμενα λέγονται \textbf{στοιχεία}.
\item \textbf{Τρόποι παράστασης συνόλου:}
\begin{brlist}[leftmargin=4mm]
\item \textbf{Αναγραφή:} Γράφουμε όλα τα στοιχεία μέσα σε άγκιστρα. Π.χ. $ A=\{1,2,3\} $.
\item \textbf{Περιγραφή:} Γράφουμε την ιδιότητα που έχουν τα στοιχεία και που ανήκουν. Π.χ. $ A=\{x\in\mathbb{N}/x<4\} $.
\item \textbf{Διάγραμμα Venn:} Σχεδιάζουμε με κύκλους τα σύνολα μέσα στο βασικό σύνολο $ \Omega $ που είναι ορθογώνιο.
\end{brlist}
\item \textbf{Πράξεις με σύνολα:}
\begin{brlist}
\item \textbf{Ένωση:} Το σύνολο που περιέχει όλα τα στοιχεία από δύο σύνολα. $ A\cup B $.
\item \textbf{Τομή:} 
\end{brlist}
\end{enumerate}
\end{document}
