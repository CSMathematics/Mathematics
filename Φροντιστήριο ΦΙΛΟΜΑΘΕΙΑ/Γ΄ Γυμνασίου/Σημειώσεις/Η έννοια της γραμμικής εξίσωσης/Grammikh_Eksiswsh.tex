\PassOptionsToPackage{no-math,cm-default}{fontspec}
\documentclass[twoside,nofonts,internet,shmeiwseis]{thewria}
\usepackage{amsmath}
\usepackage{xgreek}
\let\hbar\relax
\defaultfontfeatures{Mapping=tex-text,Scale=MatchLowercase}
\setmainfont[Mapping=tex-text,Numbers=Lining,Scale=1.0,BoldFont={Minion Pro Bold}]{Minion Pro}
\newfontfamily\scfont{GFS Artemisia}
\font\icon = "Webdings"
\usepackage[amsbb,subscriptcorrection,zswash,mtpcal,mtphrb]{mtpro2}
\usepackage{tikz,pgfplots}
\tkzSetUpPoint[size=7,fill=white]
\xroma{red!70!black}
%------TIKZ - ΣΧΗΜΑΤΑ - ΓΡΑΦΙΚΕΣ ΠΑΡΑΣΤΑΣΕΙΣ ----
\usepackage{tikz}
\usepackage{tkz-euclide}
\usetkzobj{all}
\usepackage[framemethod=TikZ]{mdframed}
\usetikzlibrary{decorations.pathreplacing}
\usepackage{pgfplots}
\usetkzobj{all}
%-----------------------
\usepackage{calc}
\usepackage{hhline}
\usepackage[explicit]{titlesec}
\usepackage{graphicx}
\usepackage{multicol}
\usepackage{multirow}
\usepackage{enumitem}
\usepackage{tabularx}
\usepackage[decimalsymbol=comma]{siunitx}
\usetikzlibrary{backgrounds}
\usepackage{sectsty}
\sectionfont{\centering}
\setlist[enumerate]{label=\bf{\large \arabic*.}}
\usepackage{adjustbox}
\usepackage{mathimatika,gensymb,eurosym,wrap-rl}
\usepackage{systeme,regexpatch}
%-------- ΜΑΘΗΜΑΤΙΚΑ ΕΡΓΑΛΕΙΑ ---------
\usepackage{mathtools}
%----------------------
%-------- ΠΙΝΑΚΕΣ ---------
\usepackage{booktabs}
%----------------------
%----- ΥΠΟΛΟΓΙΣΤΗΣ ----------
\usepackage{calculator}
%----------------------------
%------ ΔΙΑΓΩΝΙΟ ΣΕ ΠΙΝΑΚΑ -------
\usepackage{array}
\newcommand\diag[5]{%
\multicolumn{1}{|m{#2}|}{\hskip-\tabcolsep
$\vcenter{\begin{tikzpicture}[baseline=0,anchor=south west,outer sep=0]
\path[use as bounding box] (0,0) rectangle (#2+2\tabcolsep,\baselineskip);
\node[minimum width={#2+2\tabcolsep-\pgflinewidth},
minimum  height=\baselineskip+#3-\pgflinewidth] (box) {};
\draw[line cap=round] (box.north west) -- (box.south east);
\node[anchor=south west,align=left,inner sep=#1] at (box.south west) {#4};
\node[anchor=north east,align=right,inner sep=#1] at (box.north east) {#5};
\end{tikzpicture}}\rule{0pt}{.71\baselineskip+#3-\pgflinewidth}$\hskip-\tabcolsep}}
%---------------------------------
%---- ΟΡΙΖΟΝΤΙΟ - ΚΑΤΑΚΟΡΥΦΟ - ΠΛΑΓΙΟ ΑΓΚΙΣΤΡΟ ------
\newcommand{\orag}[3]{\node at (#1)
{$ \overcbrace{\rule{#2mm}{0mm}}^{{\scriptsize #3}} $};}
\newcommand{\kag}[3]{\node at (#1)
{$ \undercbrace{\rule{#2mm}{0mm}}_{{\scriptsize #3}} $};}
\newcommand{\Pag}[4]{\node[rotate=#1] at (#2)
{$ \overcbrace{\rule{#3mm}{0mm}}^{{\rotatebox{-#1}{\scriptsize$#4$}}}$};}
%-----------------------------------------
%------------------------------------------
\newcommand{\tss}[1]{\textsuperscript{#1}}
\newcommand{\tssL}[1]{\MakeLowercase{\textsuperscript{#1}}}
%---------- ΛΙΣΤΕΣ ----------------------
\newlist{bhma}{enumerate}{3}
\setlist[bhma]{label=\bf\textit{\arabic*\textsuperscript{o}\;Βήμα :},leftmargin=0cm,itemindent=1.8cm,ref=\bf{\arabic*\textsuperscript{o}\;Βήμα}}
\newlist{rlist}{enumerate}{3}
\setlist[rlist]{itemsep=0mm,label=\roman*.}
\newlist{brlist}{enumerate}{3}
\setlist[brlist]{itemsep=0mm,label=\bf\roman*.}
\newlist{tropos}{enumerate}{3}
\setlist[tropos]{label=\bf\textit{\arabic*\textsuperscript{oς}\;Τρόπος :},leftmargin=0cm,itemindent=2.3cm,ref=\bf{\arabic*\textsuperscript{oς}\;Τρόπος}}
% Αν μπει το bhma μεσα σε tropo τότε
%\begin{bhma}[leftmargin=.7cm]
\tkzSetUpPoint[size=7,fill=white]
\tikzstyle{pl}=[line width=0.3mm]
\tikzstyle{plm}=[line width=0.4mm]
\usepackage{etoolbox}
\makeatletter
\renewrobustcmd{\anw@true}{\let\ifanw@\iffalse}
\renewrobustcmd{\anw@false}{\let\ifanw@\iffalse}\anw@false
\newrobustcmd{\noanw@true}{\let\ifnoanw@\iffalse}
\newrobustcmd{\noanw@false}{\let\ifnoanw@\iffalse}\noanw@false
\renewrobustcmd{\anw@print}{\ifanw@\ifnoanw@\else\numer@lsign\fi\fi}
\makeatother

\begin{document}
\titlos{Μαθηματικά Γ΄ Γυμνασίου}{Γραμμικά Συστήματα}{Η έννοια της γραμμικής εξίσωσης}
\orismoi
\Orismos{Γραμμική Εξίσωση}
\wrapr{-4mm}{10}{4.5cm}{-4mm}{
\begin{tikzpicture}[domain=-.2:4,y=1cm,scale=.8]
\draw[-latex] (-.5,0) -- coordinate (x axis mid) (4.4,0) node[right,fill=white] {{\footnotesize $ x $}};
\draw[-latex] (0,-.5) -- (0,3.5) node[above,fill=white] {{\footnotesize $ y $}};
\draw[domain=-.2:3.4,samples=100,line width=.4mm,\xrwma] plot function{-0.8*x+2.5};
\tkzText(2.5,2.7){$ ax+\beta y=\gamma $}
\tkzText(2.5,2.2){{\footnotesize $ a,\beta,\gamma\in\mathbb{R} $}}
\tkzText(2.5,1.7){{\footnotesize $ a\neq0 $ ή $ \beta\neq0 $}}
\tkzDefPoint(0,0){O}
\tkzLabelPoint[below left](O){$ O $}
\end{tikzpicture}}{
Γραμμική εξίσωση δύο μεταβλητών, ονομάζεται κάθε πολυωνυμική εξίσωση στην οποία κάθε όρος της είναι μονώνυμο 1\textsuperscript{ου} βαθμού μιας μεταβλητής. Έχει τη μορφή \[ ax+\beta y=\gamma \]
όπου οι συντελεστές $ a,\beta $ και ο σταθερός όρος $ \gamma $ είναι πραγματικοί αριθμοί.
Η καμπύλη της εξίσωσης είναι ευθεία γραμμή αν οι συντελεστές $ a,\beta $ των μεταβλητών $ x,y $ αντίστοιχα, δεν μηδενίζονται συγχρόνως δηλ. $ a\neq0 $ ή $ \beta\neq0 $.}
\begin{itemize}[itemsep=0mm]
\item Οι ευθείες της μορφής $ x=\kappa $ ονομάζονται \textbf{κατακόρυφες} ευθείες ενώ οι ευθείες της μορφής $ y=\kappa $ οριζόντιες ευθείες.
\item Ο πραγματικός αριθμός $ \lambda=-\frac{a}{\beta} $ ονομάζεται \textbf{συντελεστής διεύθυνσης} της ευθείας $ ax+\beta y=\gamma $.
\end{itemize}
\Orismos{Λύση γραμμικήσ εξίσωσησ}
Λύση μιας γραμμικής εξίσωσης της μορφής \[ ax+\beta y=\gamma \] ονομάζεται κάθε διατεταγμένο ζεύγος αριθμών $ \left( x_0,y_0\right)  $ το οποίο επαληθεύει την εξίσωση.\\\\
\end{document}
