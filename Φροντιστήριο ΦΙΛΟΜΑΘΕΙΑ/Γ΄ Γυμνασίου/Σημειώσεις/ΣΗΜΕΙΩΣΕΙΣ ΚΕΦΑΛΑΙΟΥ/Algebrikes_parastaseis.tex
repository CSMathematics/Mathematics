\PassOptionsToPackage{no-math,cm-default}{fontspec}
\documentclass[twoside,nofonts,internet,shmeiwseis]{thewria}
\usepackage{amsmath}
\usepackage{xgreek}
\let\hbar\relax
\defaultfontfeatures{Mapping=tex-text,Scale=MatchLowercase}
\setmainfont[Mapping=tex-text,Numbers=Lining,Scale=1.0,BoldFont={Minion Pro Bold}]{Minion Pro}
\newfontfamily\scfont{GFS Artemisia}
\font\icon = "Webdings"
\usepackage[amsbb]{mtpro2}
\usepackage{tikz,pgfplots}
\tkzSetUpPoint[size=7,fill=white]
\xroma{red!70!black}
%------- ΣΥΣΤΗΜΑ -------------------
\usepackage{systeme,regexpatch}
\makeatletter
% change the definition of \sysdelim not to store `\left` and `\right`
\def\sysdelim#1#2{\def\SYS@delim@left{#1}\def\SYS@delim@right{#2}}
\sysdelim\{. % reinitialize

% patch the internal command to use
% \LEFTRIGHT<left delim><right delim>{<system>}
% instead of \left<left delim<system>\right<right delim>
\regexpatchcmd\SYS@systeme@iii
{\cB.\c{SYS@delim@left}(.*)\c{SYS@delim@right}\cE.}
{\c{SYS@MT@LEFTRIGHT}\cB\{\1\cE\}}
{}{}
\def\SYS@MT@LEFTRIGHT{%
\expandafter\expandafter\expandafter\LEFTRIGHT
\expandafter\SYS@delim@left\SYS@delim@right}
\makeatother
\newcommand{\synt}[2]{{\scriptsize \begin{matrix}
\times#1\\\\ \times#2
\end{matrix}}}
%----------------------------------------
%------ ΜΗΚΟΣ ΓΡΑΜΜΗΣ ΚΛΑΣΜΑΤΟΣ ---------
\DeclareRobustCommand{\frac}[3][0pt]{%
{\begingroup\hspace{#1}#2\hspace{#1}\endgroup\over\hspace{#1}#3\hspace{#1}}}
%----------------------------------------

\newlist{rlist}{enumerate}{3}
\setlist[rlist]{itemsep=0mm,label=\roman*.}
\newlist{brlist}{enumerate}{3}
\setlist[brlist]{itemsep=0mm,label=\bf\roman*.}
\newlist{tropos}{enumerate}{3}
\setlist[tropos]{label=\bf\textit{\arabic*\textsuperscript{oς}\;Τρόπος :},leftmargin=0cm,itemindent=2.3cm,ref=\bf{\arabic*\textsuperscript{oς}\;Τρόπος}}
\newcommand{\tss}[1]{\textsuperscript{#1}}
\newcommand{\tssL}[1]{\MakeLowercase{\textsuperscript{#1}}}

\usepackage{hhline}
%----------- ΓΡΑΦΙΚΕΣ ΠΑΡΑΣΤΑΣΕΙΣ ---------
\pgfkeys{/pgfplots/aks_on/.style={axis lines=center,
xlabel style={at={(current axis.right of origin)},xshift=1.5ex, anchor=center},
ylabel style={at={(current axis.above origin)},yshift=1.5ex, anchor=center}}}
\pgfkeys{/pgfplots/grafikh parastash/.style={\xrwma,line width=.4mm,samples=200}}
\pgfkeys{/pgfplots/belh ar/.style={tick label style={font=\scriptsize},axis line style={-latex}}}
%-----------------------------------------
\usepackage{multicol}
\usepackage{wrap-rl}
\pgfplotsset{compat=newest}
\usepackage{longtable}
\usepackage{hhline}
\pgfplotsset{compat=1.9}
    \usepgfplotslibrary{polar}
\usepackage{varwidth}




\begin{document}
\pagenumbering{gobble}% Remove page numbers (and reset to 1)
\clearpage
\Titlos{Μαθηματικά Γ΄ Γυμνασίου}{Άλγεβρα}{Αλγεβρικές Παραστάσεις}
\begin{center}
$(x^2+y^2)^2=4y^3-12yx^2+4\sqrt{x^2+y^2}^3$\\
\begin{tikzpicture}
\begin{polaraxis}
\addplot[red!70!black,line width=.4mm,domain=0:360,samples=360,smooth] (x,{4-4*sin(3*x)});
\addplot[dashed,domain=0:2*pi] (deg(x),7);
\addplot[dashed,domain=0:2*pi] (deg(x),3);
\addplot[dashed,domain=0:2*pi] (deg(x),5);
\addplot[mark=*,only marks] coordinates {(0,0)};
\end{polaraxis}
\end{tikzpicture}
\end{center}
\newpage\phantom{}
\clearpage
\pagenumbering{arabic}
\orismoi
\Orismos{Σύνολα αριθμών}
\vspace{-5mm}
\begin{enumerate}[itemsep=0mm,label=\bf\arabic*.]
\item \textbf{Φυσικοί Αριθμοί} : Το σύνολο των αριθμών από το 0 εως το άπειρο όπου κάθε αριθμός έχει διαφορά μιας μονάδας από τον προηγούμενο.
\item \textbf{Ακέραιοι Αριθμοί}: Το σύνολο των φυσικών αριθμών μαζί με τους αντίθετους τους.
\item \textbf{Ρητοί Αριθμοί}: Όλοι οι αριθμοί που μπορούν να γραφτούν με τη μορφή κλάσματος με ακέραιους όρους.
\item \textbf{Άρρητοι Αριθμοί}: Κάθε αριθμός ο οποίος δεν είναι ρητός.
\item \textbf{Πραγματικοί Αριθμοί} : Οι ρητοί μαζί με το σύνολο των άρρητων μας δίνουν τους πραγματικούς αριθμούς, όλους τους αριθμούς που γνωρίζουμε.
\end{enumerate}
\Orismos{Αντίθετοι - Αντίστροφοι αριθμοι}
\begin{enumerate}[itemsep=0mm,label=\bf\arabic*.]
\vspace{-5mm}
\item \textbf{Αντίθετοι αριθμοί}\\
Αντίθετοι ονομάζονται οι αριθμοί οι οποίοι έχουν άθροισμα μηδέν. Οι αντίθετοι αριθμοί έχουν ίσες απόλυτες τιμές και αντίθετα πρόσημα.
\[ a+(-a)=0 \]
\item \textbf{Αντίστροφοι αριθμοί}\\
Αντίστροφοι ονομάζονται δύο πραγματικοί αριθμοί οι οποίοι έχουν γινόμενο ίσο με τη μονάδα. \[ a\cdot \frac{1}{a}=1 \]
\end{enumerate}
\Orismos{Πράξεισ αριθμών}
Στον παρακάτω πίνακα φαίνονται τα ονόματα των αριθμών που αποτελούν μια πράξη, τα ονόματα των αποτελεσμάτων και ο συμβολισμός κάθε πράξης.
\begin{center}
\begin{tabular}{cccc}
\hline \rule[-2ex]{0pt}{5.5ex} \textbf{Πράξη} & \textbf{Όροι} & \textbf{Αποτέλεσμα} & \textbf{Συμβολισμός} \\ 
\hhline{====} \rule[-2ex]{0pt}{5.5ex} \textbf{Πρόσθεση} & Προσθετέοι & Άθροισμα & $ a+\beta $ \\ 
\rule[-2ex]{0pt}{5.5ex} \textbf{Αφαίρεση} & Μειωτέος - Αφαιρετέος & Διαφορά & $ a-\beta $ \\ 
\rule[-2ex]{0pt}{5.5ex} \textbf{Πολλαπλασιασμός} & Παράγοντες & Γινόμενο & $ a\cdot\beta $ \\ 
\rule[-2ex]{0pt}{5.5ex} \textbf{Διαίρεση} & Διαιρετέος - Διαιρέτης & Πηλίκο & $ a:\beta $ \\ 
\hline\end{tabular}
\end{center}
Η αφαίρεση $ a-\beta $ και η διαίρεση $ a:\beta $ δύο αρθμών $ a,\beta\in\mathbb{R} $ είναι οι πράξεις που προκύπτουν από την πρόσθεση και τον πολλαπλασιασμό αντίστοιχα και μπορούν να γραφτούν με τη βοήθεια τους.
\[ a-\beta=a+(-\beta)\;\;,\;\;a:\beta=\frac{a}{\beta}=a\cdot\frac{1}{\beta} \]
\Orismos{Απόλυτη Τιμή Πραγματικού Αριθμού}
Απόλυτη τιμή ενός πραγματικού αριθμού  ορίζεται να είναι η απόσταση της εικόνας του αριθμού αυτού απο το 0 και συμβολίζεται με $ |a| $.
\begin{center}
\begin{tabular}{c >{\centering\arraybackslash}m{6cm}}
$ |a|=\LEFTRIGHT\{.{\begin{aligned}
a & \;,\;a\geq0\\
-a & \;,\;a<0
\end{aligned}} $  & \begin{tikzpicture}
\draw[-latex] (-1,0) -- coordinate (x axis mid) (4.4,0) node[right,fill=white] {{\footnotesize $ x $}};
\foreach \x in {-1,0,...,4}
\draw (\x,.5mm) -- (\x,-.5mm) node[anchor=north,fill=white] {{\scriptsize \x}};
\draw[line width=.7mm,\xrwma] (0,0) -- (3,0);
\tkzText[\xrwma](1.5,.34){$ \overcbrace{\rule{27mm}{0mm}}^{{\scriptsize |3|=3}} $}
\tkzDefPoint(3,0){A}
\tkzDrawPoint[size=7,fill=white](A)
\tkzLabelPoint[above right](A){{\scriptsize $A(3)$}}
\end{tikzpicture}
\end{tabular} 
\end{center}
\begin{itemize}[itemsep=0mm]
\item Η απόλυτη τιμή ενός θετικού αριθμού $ a $ είναι ίση με τον ίδιο τον αριθμό.
\item Η απόλυτη τιμή ενός αρνητικού αριθμού $ a $ είναι ίση με τον αντίθετο του αριθμού $ a $.
\end{itemize}
\Orismos{Δύναμη πραγματικου αριθμου}
Δύναμη ενός φυσικού αριθμού $ a $ ονομάζεται το γινόμενο $ \nu $ ίσων παραγόντων του αριθμού αυτού. Συμβολίζεται με $ a^\nu $ όπου $ \nu\in\mathbb{N} $ είναι το πλήθος των ίσων παραγόντων. 
\[ \undercbrace{a\cdot a\cdot\ldots a}_{\nu\textrm{ παράγοντες }}=a^\nu \]
\begin{itemize}[itemsep=0mm]
\item Ο αριθμός $ a $ ονομάζεται \textbf{βάση} και ο αριθμός $ \nu $ \textbf{εκθέτης} της δύναμης.
\item Η δύναμη $ a^2 $ ονομάζεται και $ \mathbold{a} $ \textbf{στο τετράγωνο}.
\item Η δύναμη $ a^3 $ ονομάζεται και  {\boldmath $ a $}  \textbf{στον κύβο}.
\item Σε μία αριθμητική παράσταση, η σειρά με την οποία γίνονται οι πράξεις είναι
\begin{enumerate}[itemsep=0mm]
\item Δυνάμεις
\item Πολλαπλασιασμοί - Διαιρέσεις
\item Προσθέσεις - Αφαιρέσεις
\end{enumerate}
\item Οι πράξεις εκτελούνται μ' αυτή τη σειρά πρώτα μέσα στις παρενθέσεις αν υπάρχουν και ύστερα έξω απ' αυτές.
\end{itemize}
\Orismos{Τετραγωνική Ρίζα}
Τετραγωνική ρίζα ενός θετικού αριθμού $ x $ ονομάζεται ο \textbf{θετικός} αριθμός $ a $ που αν υψωθεί στο τετράγωνο δίνει τον αριθμό $ x $ και συμβολίζεται με $ \sqrt{x} $.
\[ \sqrt{x}=a\;\;,\;\;\textrm{ όπου }x\geq0\textrm{ και }a\geq0 \]
\begin{itemize}[itemsep=0mm]
\item Δεν ορίζεται ρίζα αρνητικού αριθμού.
\item Ο θετικός αριθμός $ x $ ονομάζεται \textbf{υπόριζο}.
\end{itemize}
\Orismos{Αλγεβρική παράσταση - Ακέραια αλγεβρική παράσταση}
Μια μαθηματική παράσταση ονομάζεται αλγεβρική όταν αυτή περιέχει αριθμούς και μεταβλητές με πράξεις μεταξύ τους. \begin{itemize}
\item Μια αλγεβρική παράσταση θα ονομάζεται \textbf{ακέραια} εαν μεταξύ των μεταβλητών της υπάρχουν μόνο οι πράξεις του πολλαπλασιασμού και της πρόσθεσης, ενώ οι εκθέτες είναι \textbf{φυσικοί αριθμοί}.
\item \textbf{Τιμή} μιας αλγεβρικής παράστασης ονομάζεται ο αριθμός που θα προκύψει ύστερα από πράξεις εαν αντικατασταθούν οι μεταβλητές της με αριθμούς.
\end{itemize}
\Orismos{Μονώνυμο}
Μονώνυμο ονομάζεται μια ακέραια αλγεβρική παράσταση εαν μεταξύ των μεταβλητών υπάρχει μόνο η πράξη του πολαπλασιασμού.
\[ \textrm{{\scriptsize Συντελεστής} }\longrightarrow a\cdot \undercbrace{x^{\nu_1}y^{\nu_2}\cdot \ldots\cdot z^{\nu_\kappa}}_{\textrm{κύριο μέρος}}\;\;,\;\;\nu_1,\nu_2,\ldots,\nu_\kappa\in\mathbb{Z} \]
\begin{itemize}[itemsep=0mm]
\item Το γινόμενο των μεταβλητών ενός μονωνύμου ονομάζεται \textbf{κύριο μέρος}.
\item  Ο σταθερός αριθμός με τον οποίο πολλαπλασιάζουμε το κύριο μέρος ενός μονωνύμου ονομάζεται \textbf{συντελεστής}.
\end{itemize}
\Orismos{ΒΑΘΜΌΣ ΜΟΝΩΝΎΜΟΥ}
Βαθμός μονωνύμου, ως προς μια μεταβλητή, ονομάζεται ο εκθέτης της μεταβλητής.
\begin{itemize}[itemsep=0mm]
\item Βαθμός ενός μονωνύμου ως προς όλες τις μεταβλητές είναι το άθροισμα των βαθμών κάθε μεταβλητής.
\item Οι πραγματικοί αριθμοί ονομάζονται \textbf{σταθερά} μονώνυμα και είναι μηδενικού βαθμού, ενώ το 0 ονομάζεται \textbf{μηδενικό} μονώνυμο και δεν έχει βαθμό.
\end{itemize}
\Orismos{ΌΜΟΙΑ - ΊΣΑ - ΑΝΤΊΘΕΤΑ ΜΟΝΏΝΥΜΑ}
\vspace{-5mm}
\begin{itemize}[itemsep=0mm]
\item Όμοια ονομάζονται τα μονώνυμα που έχουν το ίδιο κύριο μέρος.
\item Ίσα ονομάζονται δύο ή περισσότερα όμοια μονώνυμα που έχουν ίσους συντελεστές.
\item Αντίθετα ονομάζονται δύο όμοια μονώνυμα που έχουν αντίθετους συντελεστές.
\end{itemize}
\Orismos{ΠΟΛΥΏΝΥΜΟ}	Πολυώνυμο ονομάζεται η ακέραια αλγεβρική παράσταση η οποία αποτελεί άθροισμα
\textbf{ανόμοιων} μονωνύμων.
\begin{itemize}[itemsep=0mm]
\item Κάθε μονώνυμο μέσα σ' ένα πολυώνυμο ονομάζεται \textbf{όρος} του πολυωνύμου.
\item Το πολυώνυμο με 3 όρους ονομάζεται \textbf{τριώνυμο}.
\item Οι αριθμοί ονομάζονται \textbf{σταθερά} πολυώνυμα ενώ το 0 \textbf{μηδενικό} πολυώνυμο.
\item  Κάθε πολυώνυμο συμβολίζεται με ένα κεφαλαίο γράμμα όπως : $ P, Q, A, B\ldots $ τοποθετώντας δίπλα από το όνομα μια παρένθεση στην οποία θα βρίσκονται οι μεταβλητές του δηλαδή : $ P(x), Q(x,y), A(z,w), \\B(x_1,x_2,\ldots,x_\nu) $.
\item Βαθμός ενός πολυωνύμου είναι ο βαθμός του \textbf{μεγιστοβάθμιου} όρου.
\item Τα πολυώνυμα μιας μεταβλητής τα γράφουμε κατά \textbf{φθίνουσες δυνάμεις της μεταβλητής} δηλαδή από τη μεγαλύτερη στη μικρότερη. Έχουν τη μορφή :
\[ P(x)=a_\nu x^\nu+a_{\nu-1}x^{\nu-1}+\ldots+a_1x+a_0 \]
\item Τιμή ενός πολυωνύμου ονομάζεται ο πραγματικός αριθμός που προκύπτει ύστερα από πράξεις αν αντικατασταθούν οι μεταβλητές του πολυωνύμου με πραγματικούς αριθμούς. Η τιμή ενός πολυωνύμου $ P(x) $,  για $ x=x_0 $ συμβολίζεται με $ P(x_0) $ και είναι ίση με :
\[ P(x_0)=a_\nu x_0^\nu+a_{\nu-1}x_0^{\nu-1}+\ldots+a_1x_0+a_0 \]
\end{itemize}
\Orismos{ΑΝΑΓΩΓΉ ΟΜΟΊΩΝ ΌΡΩΝ}
Αναγωγή ομοίων όρων ονομάζεται η διαδικασία με την οποία απλοποιούμε μια αλγεβρική παράσταση προσθέτοντας ή αφαιρώντας τoυς όμοιους όρους της.\\\\
\Orismos{Ταυτότητα}
Ταυτότητα ονομάζεται κάθε ισότητα η οποία περιέχει μεταβλητές και επαληθεύεται για κάθε τιμή των μεταβλητών της.
\begin{center}
\textbf{ΑΞΙΟΣΗΜΕΙΩΤΕΣ ΤΑΥΤΟΤΗΤΕΣ}
\end{center}
\begin{multicols}{2}
\begin{enumerate}[itemsep=0mm,label=\bf\arabic*.]
\item \parbox[t]{7cm}{\textbf{Άθροισμα στο τετράγωνο}\\$ (a+\beta)^2=a^2+2a\beta+\beta^2 $}
\item \parbox[t]{7cm}{\textbf{Διαφορά στο τετράγωνο}\\$ (a-\beta)^2=a^2-2a\beta+\beta^2 $}
\item \parbox[t]{7cm}{\textbf{Άθροισμα στον κύβο}\\$ (a+\beta)^3=a^3+3a^2\beta+3a\beta^2+\beta^3 $}
\item \parbox[t]{7cm}{\textbf{Διαφορά στον κύβο}\\$ (a-\beta)^3=a^3-3a^2\beta+3a\beta^2-\beta^3 $}
\item \parbox[t]{7cm}{\textbf{Γινόμενο αθροίσματος επί διαφορά}\\$ (a+\beta)(a-\beta)=a^2-\beta^2 $}
\item \parbox[t]{7cm}{\textbf{Άθροισμα κύβων}\\$ (a+\beta)\left(a^2-a\beta+\beta^2 \right)=a^3+\beta^3 $}
\item \parbox[t]{7cm}{\textbf{Διαφορά κύβων}\\$ (a-\beta)\left(a^2+a\beta+\beta^2 \right)=a^3-\beta^3 $}
\end{enumerate}
\end{multicols}
\Orismos{Παραγοντοποίηση}
Παραγοντοποίηση ονομάζεται η διαδικασία με την οποία μια αλγεβρική παράσταση, μετατρέπεται από άθροισμα σε γινόμενο παραγόντων.\\\\
\begin{center}
\textbf{ΒΑΣΙΚΟΙ ΚΑΝΟΝΕΣ ΠΑΡΑΓΟΝΤΟΠΟΙΗΣΗΣ}
\end{center}
\begin{enumerate}[itemsep=0mm,label=\bf\arabic*.]
\item \textbf{Κοινός Παράγοντας}\\
Η διαδικασία αυτή εφαρμόζεται όταν σ' όλους τους όρους της παράστασης υπάρχει κοινός παράγοντας. Κοινό παράγοντα βγάζουμε 
\begin{rlist}
\item το Μ.Κ.Δ. των συντελεστών και 
\item τις κοινές μεταβλητές ή παραστάσεις στο μικρότερο εκθέτη.
\end{rlist}
Οι όροι μέσα στην παρένθεση προκύπτουν διαιρώντας κάθε όρο με τον κοινό παράγοντα.
\item \textbf{Ομαδοποίηση}\\
Χρησιμοποιείται στην περίπτωση που δεν υπάρχει σε όλους τους όρους μιας παράστασης κοινός παράγοντας οπότε μοιράζονται οι όροι σε ομάδες έτσι ώστε κάθε ομάδα να έχει δικό της κοινό παράγοντα.
\item \textbf{Διαφορά Τετραγώνων}\\
Κάθε σχέση της μορφής $ a^2-\beta^2 $ παραγοντοποιείται ως εξής : \[ a^2-\beta^2=(a-\beta)(a+\beta) \]
\item \textbf{Διαφορά - Άθροισμα Κύβων}\\
Κάθε σχέση της μορφής $ a^3-\beta^3 $ ή $ a^3+\beta^3 $ παραγοντοποιείται ως εξής : \begin{gather*}
a^3-\beta^3=(a-\beta)\left(a^2+a\beta+\beta^2 \right)\quad\textrm{ και }\quad
a^3+\beta^3=(a+\beta)\left(a^2-a\beta+\beta^2 \right)
\end{gather*}
\item \textbf{Ανάπτυγμα Τετραγώνου}\\
Κάθε σχέση της μορφής $ a^2\pm2a\beta+\beta^2 $ παραγοντοποιείται ως εξής :
\begin{gather*}
a^2+2a\beta+\beta^2=(a+\beta)^2\quad\textrm{ και }\quad
a^2-2a\beta+\beta^2=(a-\beta)^2
\end{gather*}
\item \textbf{Τριώνυμο}\\
Κάθε σχέση της μορφής $ x^2+(a+\beta)x+a\beta $ παραγοντοποιείται ως εξής : \[ x^2+(a+\beta)x+a\beta=(x+a)(x+\beta) \]
\end{enumerate}
\Orismos{Ευκλειδεια διαίρεση}
Ευκλείδεια διαίρεση μεταξύ δύο πολυωνύμων $ \varDelta(x) $ και $ \delta(x) $ ονομάζεται η διαδικασία με την οποία διαρώντας τα πολυώνυμα αυτά προκύπτεί μοναδικό ζεύγος πολυωνύμων $ \pi(x) $ και $ \upsilon(x) $ για τα οποία ισχύει
\[ \varDelta(x)=\delta(x)\cdot\pi(x)+\upsilon(x) \]
\begin{itemize}
\item Τα πολυώνυμα $ \varDelta(x),\delta(x),\pi(x),\upsilon(x) $ ονομάζονται \textbf{Διαιρετέος, διαιρέτης, πηλίκο} και \textbf{υπόλοιπο} αντίστοιχα.
\item Η ισότητα $ \varDelta(x)=\delta(x)\cdot\pi(x)+\upsilon(x) $ ονομάζεται \textbf{ισότητα της Ευκλείδειας διαίρεσης}.
\item Αν το υπόλοιπο της διαίρεσης είναι μηδενικό $ (\upsilon(x)=0) $ η διαίρεση ονομάζεται τέλεια και ισχύει :
\[ \varDelta(x)=\delta(x)\cdot\pi(x) \]
Στην τέλεια διαίρεση, τα πολυώνυμα $ \delta(x) $ και $ \pi(x) $ ονομάζονται \textbf{παράγοντες} ή \textbf{διαιρέτες} του $ \varDelta(x) $.
\end{itemize}
\Orismos{Ε.Κ.Π. Αλγεβρικών παραστάσεων}
Ε.Κ.Π. δύο ή περισσότερων αλγεβρικών παραστάσεων ονομάζεται το γινόμενο των κοινών και μη κοινών παραγόντων τους, τον καθένα υψωμένο στη μεγαλύτερη δύναμη.\\\\\\
\Orismos{Μ.Κ.Δ. Αλγεβρικών παραστάσεων}
Μ.Κ.Δ. δύο ή περισσότερων αλγεβρικών παραστάσεων ονομάζεται το γινόμενο των κοινών παραγόντων τους, τον καθένα υψωμένο στη μικρότερη δύναμη.\\\\
\Orismos{Ρητή αλγεβρική παράσταση}
Ρητή ονομάζεται κάθε αλγεβρική παράσταση η οποία έχει τη μορφή κλάσματος με τουλάχιστον μια μεταβλητη στον παρονομαστή της. Είναι της μορφής : $ \frac{A(x)}{B(x)} $.
\begin{itemize}[itemsep=0mm]
\item Μια ρητή αλγεβρική παράσταση ορίζεται αν ο παρονομαστής της είναι \textbf{διάφορος του μηδενός} : $ B(x)\neq0 $.
\item Μια αλγεβρική παράσταση απλοποιείται \textbf{μόνο} αν και οι δύο όροι της αποτελούν \textbf{γινόμενο} παραγόντων.
\end{itemize}\mbox{}\\
\thewrhmata
\Thewrhma{Ιδιότητεσ των Πράξεων}
Στον παρακάτω πίνακα βλέπουμε τις βασικές ιδιότητες της πρόσθεσης και του πολλαπλασιασμού στο σύνολο των πραγματικών αριθμών.
\begin{center}
\begin{tabular}{ccc}
	\hline \rule[-2ex]{0pt}{5.5ex} \textbf{Ιδιότητα} & \textbf{Πρόσθεση} & \textbf{Πολλαπλασιασμός} \\ 
	\hhline{===} \rule[-2ex]{0pt}{5.5ex} \textbf{Αντιμεταθετική} & $ a+\beta=\beta+a $ & $ a\cdot\beta=\beta\cdot a $ \\
	\rule[-2ex]{0pt}{5ex} \textbf{Προσεταιριστική} & $ a+\left( \beta+\gamma\right) =\left( a+\beta\right) +\gamma $ & $ a\cdot\left( \beta\cdot\gamma\right) =\left( a\cdot\beta\right)\cdot\gamma $\\
	\rule[-2ex]{0pt}{5ex} \textbf{Ουδέτερο στοιχείο} & $ a+0=a $ & $ a\cdot1= a $\\
	\rule[-2ex]{0pt}{5ex} \textbf{Αντίθετοι / Αντίστροφοι} & $ a+(-a)=0 $ & $ a\cdot\frac{1}{a}= 1 $\\
	\rule[-2ex]{0pt}{5ex} \textbf{Επιμεριστική} & \multicolumn{2}{c}{$ a\cdot\left( \beta\pm\gamma\right)=a\cdot\beta\pm a\cdot\gamma  $}\\
	\hline
\end{tabular}
\end{center}
Ισχύουν επίσης :
\begin{itemize}[itemsep=0mm]
\item Για κάθε πραγματικό αριθμό $ a $ ισχύει $ a\cdot0=0 $
\item Δύο αριθμοί που έχουν άθροισμα 0 λέγονται \textbf{αντίθετοι}.
\item Το 0 λέγεται \textbf{ουδέτερο στοιχείο της πρόσθεσης}.
\item Δύο αριθμοί που έχουν γινόμενο 1 λέγονται \textbf{αντίστροφοι}.
\item Το 1 λέγεται \textbf{ουδέτερο στοιχείο του πολλαπλασιασμού}.
\item Το 0 δεν έχει αντίστροφο.
\end{itemize}
\Thewrhma{Γινόμενο - Πηλίκο πραγματικων αριθμών}
Για οποιουσδήποτε δύο πραγματικούς $ a,\beta\in\mathbb{R} $ ισχύουν οι παρακάτω προτάσεις :
\begin{itemize}[itemsep=0mm]
\item Το γινόμενο και το πηλίκο δύο ομόσημων πραγματικών αριθμών $ a,\beta $ είναι θετικό.
\item Το γινόμενο και το πηλίκο δύο ετερόσημων πραγματικών αριθμών $ a,\beta $ είναι αρνητικό.
\end{itemize}
\begin{gather*}
a,\beta\textrm{ ομόσημοι }\Rightarrow a\cdot\beta>0\textrm{ και }\dfrac{a}{\beta}>0\\
a,\beta\textrm{ ετερόσημοι }\Rightarrow a\cdot\beta<0\textrm{ και }\dfrac{a}{\beta}<0
\end{gather*}
\Thewrhma{Ιδιότητεσ δυνάμεων}
Για κάθε δυναμη με βάση έναν πραγματικό αριθμό $ a $ ισχύει :
\[ a^1=a\;\;,\;\;a^0=1\;,\;\textrm{όπου }a\neq0\;\;,\;\;a^{-\nu}=\dfrac{1}{a^\nu}\;,\;\textrm{όπου }a\neq0 \]
Επίσης για κάθε δυναμη με βάση οποιουσδήποτε πραγματικούς αριθμούς $ a,\beta $ και φυσικούς εκθέτες $ \nu,\mu $ ισχύουν οι παρακάτω ιδιότητες :
\begin{center}
\begin{longtable}{ccc}
\hline \rule[-2ex]{0pt}{5.5ex} & \textbf{Ιδιότητα} & \textbf{Συνθήκη} \\
\hhline{===}\rule[-2ex]{0pt}{5.5ex} \textbf{1} & Γινόμενο δυνάμεων με κοινή βάση & $ a^\nu\cdot a^\mu=a^{\nu+\mu} $ \\
\rule[-2ex]{0pt}{5.5ex} \textbf{2} & Πηλίκο δυνάμεων με κοινή βάση & $ a^\nu: a^\mu=a^{\nu-\mu} $\\
\rule[-2ex]{0pt}{5.5ex} \textbf{3} & Γινόμενο δυνάμεων με κοινό εκθέτη & $ \left(a\cdot\beta\right)^\nu=a^\nu\cdot\beta^\nu $ \\
\rule[-2ex]{0pt}{5.5ex} \textbf{4} & Πηλίκο δυνάμεων με κοινό εκθέτη & $ \left(\dfrac{a}{\beta}\right)^\nu=\dfrac{a^\nu}{\beta^\nu}\;\;,\;\;\beta\neq0 $ \\
\rule[-2ex]{0pt}{5.5ex} \textbf{5} & Δύναμη υψωμένη σε δύναμη & $ \left( a^\nu\right)^\mu=a^{\nu\cdot\mu} $ \\
\rule[-2ex]{0pt}{5.5ex} \textbf{6} & Κλάσμα με αρνητικό εκθέτη & $ \left( \dfrac{a}{\beta}\right)^{-\nu}=\left(\dfrac{\beta}{a}\right)^\nu\;\;,\;\;a,\beta\neq0 $ \\
&&\\
\hline
\end{longtable}
\end{center}
Οι ιδιότητες 1 και 3 ισχύουν και για γινόμενο περισσότερων των δύο παραγόντων.
\begin{gather*}
a^{\nu_1}\cdot a^{\nu_2}\cdot\ldots\cdot a^{\nu_\kappa}=a^{\nu_1+\nu_2+\ldots+\nu_\kappa}\\
\left( a_1\cdot a_2\cdot\ldots\cdot a_\kappa\right)^\nu=a_1^\nu\cdot a_2^\nu\cdot\ldots\cdot a_\kappa^\nu
\end{gather*}
\Thewrhma{Ιδιότητεσ Ριζών}
Για οποιουσδήποτε πραγματικούς αριθμούς $ x,y $ ισχύουν οι παρακάτω ιδιότητες για την τετραγωνική ρίζα.
\begin{center}
\begin{longtable}{ccc}
\hline \rule[-2ex]{0pt}{5.5ex} & \textbf{Ιδιότητα} & \textbf{Συνθήκη} \\
\hhline{===}\rule[-2ex]{0pt}{5.5ex} \textbf{1} & Τετράγωνο ρίζας & $ \left(\!\sqrt{x}\;\right)^2=x\;\;,\;\; x\geq0  $ \\
\rule[-2ex]{0pt}{5.5ex} \textbf{2} & Ρίζα τετραγώνου & $ \sqrt{x^2}=|x|\;\;,\;\; x\;\textrm{ πραγματικός} $\\
\rule[-2ex]{0pt}{5.5ex} \textbf{3} & Ρίζα γινομένου & $ \sqrt{x\cdot y}=\!\sqrt{x}\cdot\!\sqrt{y}\;\;,\;\; x,y\geq0 $ \\
\rule[-2ex]{0pt}{6.5ex}\textbf{4} & Ρίζα πηλίκου & $ \SQRT{\dfrac{x}{y}}=\dfrac{\sqrt{x}}{\sqrt{y}}\;\;,\;\;x\geq0\textrm{ και }y>0 $ \vspace{1mm}\\
\hline
\end{longtable}
\end{center}
Η ιδιότητα 3 ισχύει και για γινόμενο περισσότερων των δύο παραγόντων. \[ \sqrt{x_1\cdot x_2\cdot\ldots\cdot x_\nu}=\!\sqrt{x_1}\cdot\!\sqrt{x_2}\cdot\ldots\cdot\!\sqrt{x_\nu} \] όπου $ x_1,x_2,\ldots x_\nu\geq0 $ και $ \nu\;\textrm{ φυσικός} $.\\\\
\Thewrhma{Πρόσθεση μονωνύμων}
Το άθροισμα όμοιων μονωνύμων είναι ένα \textbf{μονώνυμο} με κοινό κύριο μέρος και συντελεστή το άθροισμα των συντελεστών τους.\\\\
\Thewrhma{Γινόμενο μονωνύμων}
Το γινόμενο μονωνύμων είναι ένα \textbf{μονώνυμο} με κύριο μέρος το γινόμενο των κύριων μερών τους και συντελεστή το γινόμενο των συντελεστών τους. Ο βαθμός του γινομένου ως προς κάθε μεταβλητή είναι το άθροισμα των αντίστοιχων βαθμών.\\\\
\Thewrhma{Ευκλείδεια διαίρεση}
Δίνονται τα πολυώνυμα $ \varDelta(x),\delta(x),\pi(x),\upsilon(x) $ τα οποία συνδέονται με τη σχέση :
\[ \varDelta(x)=\delta(x)\cdot\pi(x)+\upsilon(x) \]
\begin{itemize}
\item Η ισότητα αυτή παριστάνει ταυτότητα Ευκλέιδειας διαίρεσης αν και μόνο αν ο βαθμός του υπολοίπου $ \upsilon(x) $ είναι μικρότερος από το βαθμό του διαιρέτη $ \delta(x) $.
\item Ένα πολυώνυμο $ \delta(x) $ είναι παράγοντας ενός πολυωνύμου $ \varDelta(x) $ αν υπάρχει πολυώνυμο $ \pi(x) $ ώστε να ισχύει $ \varDelta(x)=\delta(x)\cdot\pi(x) $.
\end{itemize}
\end{document}
