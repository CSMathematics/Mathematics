\PassOptionsToPackage{no-math,cm-default}{fontspec}
\documentclass[twoside,nofonts,internet,shmeiwseis]{thewria}
\usepackage{amsmath}
\usepackage{xgreek}
\let\hbar\relax
\defaultfontfeatures{Mapping=tex-text,Scale=MatchLowercase}
\setmainfont[Mapping=tex-text,Numbers=Lining,Scale=1.0,BoldFont={Minion Pro Bold}]{Minion Pro}
\newfontfamily\scfont{GFS Artemisia}
\font\icon = "Webdings"
\usepackage[amsbb]{mtpro2}
\usepackage{tikz,pgfplots,tkz-euclide}
\tkzSetUpPoint[size=7,fill=white]
\xroma{red!70!black}
%------- ΣΥΣΤΗΜΑ -------------------
\usepackage{systeme,regexpatch}
\makeatletter
% change the definition of \sysdelim not to store `\left` and `\right`
\def\sysdelim#1#2{\def\SYS@delim@left{#1}\def\SYS@delim@right{#2}}
\sysdelim\{. % reinitialize

% patch the internal command to use
% \LEFTRIGHT<left delim><right delim>{<system>}
% instead of \left<left delim<system>\right<right delim>
\regexpatchcmd\SYS@systeme@iii
{\cB.\c{SYS@delim@left}(.*)\c{SYS@delim@right}\cE.}
{\c{SYS@MT@LEFTRIGHT}\cB\{\1\cE\}}
{}{}
\def\SYS@MT@LEFTRIGHT{%
\expandafter\expandafter\expandafter\LEFTRIGHT
\expandafter\SYS@delim@left\SYS@delim@right}
\makeatother
\newcommand{\synt}[2]{{\scriptsize \begin{matrix}
\times#1\\\\ \times#2
\end{matrix}}}
%----------------------------------------
%------ ΜΗΚΟΣ ΓΡΑΜΜΗΣ ΚΛΑΣΜΑΤΟΣ ---------
\DeclareRobustCommand{\frac}[3][0pt]{%
{\begingroup\hspace{#1}#2\hspace{#1}\endgroup\over\hspace{#1}#3\hspace{#1}}}
%----------------------------------------

\newlist{rlist}{enumerate}{3}
\setlist[rlist]{itemsep=0mm,label=\roman*.}
\newlist{brlist}{enumerate}{3}
\setlist[brlist]{itemsep=0mm,label=\bf\roman*.}
\newlist{tropos}{enumerate}{3}
\setlist[tropos]{label=\bf\textit{\arabic*\textsuperscript{oς}\;Τρόπος :},leftmargin=0cm,itemindent=2.3cm,ref=\bf{\arabic*\textsuperscript{oς}\;Τρόπος}}
\newcommand{\tss}[1]{\textsuperscript{#1}}
\newcommand{\tssL}[1]{\MakeLowercase{\textsuperscript{#1}}}
\usetkzobj{all}
\usepackage{hhline}
%----------- ΓΡΑΦΙΚΕΣ ΠΑΡΑΣΤΑΣΕΙΣ ---------
\pgfkeys{/pgfplots/aks_on/.style={axis lines=center,
xlabel style={at={(current axis.right of origin)},xshift=1.5ex, anchor=center},
ylabel style={at={(current axis.above origin)},yshift=1.5ex, anchor=center}}}
\pgfkeys{/pgfplots/grafikh parastash/.style={\xrwma,line width=.4mm,samples=200}}
\pgfkeys{/pgfplots/belh ar/.style={tick label style={font=\scriptsize},axis line style={-latex}}}
%-----------------------------------------
\usepackage{multicol}
\usepackage{wrap-rl}
\tkzSetUpPoint[size=7,fill=white]
\tikzstyle{pl}=[line width=0.3mm]
\tikzstyle{plm}=[line width=0.4mm]
\usepackage{gensymb}


\begin{document}
\titlos{ΜΑΘΗΜΑΤΙΚΑ Γ΄ ΓΥΜΝΑΣΙΟΥ}{Γεωμετρία}{ΙΣΟΤΗΤΑ ΤΡΙΓΩΝΩΝ}
\orismoi
\Orismos{Τρίγωνο - Κύρια στοιχεία τριγώνου}
\wrapr{-4mm}{2}{4cm}{-3mm}{\begin{tikzpicture}[x=1cm,y=1cm]
\draw[pl] (-0.5,1.2) node(A){} -- (-1.5,-0.5) node(B){} 
-- (1.5,-0.5) node(C){}--cycle;
\tkzMarkAngle[size=4mm](B,A,C)
\tkzMarkAngle[size=4mm](A,C,B)
\tkzMarkAngle[size=4mm](C,B,A)
\tkzDrawPoints(A,B,C)
\tkzLabelPoint[above](A){$A$}
\tkzLabelPoint[left](B){$B$}
\tkzLabelPoint[right](C){$\varGamma$}
\node at (-1.25,0.5) {$\gamma$};
\node at (.75,0.5) {$\beta$};
\node at (0,-0.75) {$a$};
\end{tikzpicture}}{
Τρίγωνο ονομάζεται το κυρτό πολύγωνο που έχει τρεις πλευρές και τρεις γωνίες. \begin{itemize}
\item Τα κύρια στοιχεία ενός τριγώνου είναι οι πλευρές, οι γωνίες και οι κορυφές του.
\item Κάθε τρίγωνο συμβολίζεται με τη χρήση των ονομάτων των τριών κορυφών του για παράδειγμα $ AB\varGamma $.
\end{itemize}
\[ B\varGamma\rightarrow a\;\;,\;\;A\varGamma\rightarrow \beta\;\;,\;\;AB\rightarrow \gamma \]
\begin{itemize}
\item Οι πλευρές ενός τριγώνου, εκτός από το συνηθισμένο συμβολισμό ενός ευθύγραμμου τμήματος, μπορούν εναλλακτικά να συμβολιστούν με ένα μικρό γράμμα, αντίστοιχο του ονόματος της απέναντι κορυφής.
\end{itemize}}\mbox{}\\\\\\
\Orismos{Δευτερεύοντα στοιχεία τριγώνου}
Τα δευτερεύοντα στοιχεία κάθε τριγώνου είναι η διάμεσος, η διχοτόμος και το ύψος του. Αναλυτικά ορίζονται ως εξής :
\begin{enumerate}[label=\bf\arabic*.]
\item \textbf{Διάμεσος}\\
Διαμεσος ενός τριγώνου ονομάζεται το ευθύγραμμο τμήμα το οποίο ενώνει μια κορυφή του τριγώνου με το μέσο της απέναντι πλευράς. \begin{itemize}
\item Κάθε διάμεσος συμβολίζεται είτε με τα γράμματα των δύο άκρων της είναι με το γράμμα $ \mu $ το οποίο θα έχει δείκτη, το όνομα της πλευράς στην οποία αντιστοιχεί η διάμεσος. 
\item Οι διάμεσοι για ένα τρίγωνο $ AB\varGamma $ θα συμβολίζονται $ \mu_a,\mu_\beta,\mu_\gamma $.
\end{itemize}
\item \textbf{Διχοτόμος}\\
Διχοτόμος ενός τριγώνου ονομάζεται το ευθύγραμμο τμήμα το οποίο χωρίζει μια γωνία του τριγώνου σε δύο ίσα μέρη.
\begin{itemize}
\item Κάθε διχοτόμος συμβολίζεται εναλλακτικά με το γράμμα $ \delta $ το οποίο θα έχει δείκτη, το όνομα της πλευράς στην οποία αντιστοιχεί η διχοτόμος. 
\item Οι διχοτόμοι για ένα τρίγωνο $ AB\varGamma $ θα συμβολίζονται $ \delta_a,\delta_\beta,\delta_\gamma $.
\end{itemize}
\end{enumerate}
\begin{center}
\begin{tabular}{p{3.5cm}cp{3.5cm}cp{3.5cm}}
\begin{tikzpicture}[x=1cm,y=1cm]
\draw[pl] (-0.5,1.25) node(A){} -- (-1.5,-0.5) node(B){} 
-- (1.5,-0.5) node(C){}--cycle;
\tkzDefPoint(0,-.5){M}
\draw[\xrwma,plm] (-0.5,1.25)--(M);
\tkzMarkSegment[mark=|](B,M)
\tkzMarkSegment[mark=|](M,C)
\tkzLabelPoint[above](A){$A$}
\tkzLabelPoint[left](B){$B$}
\tkzLabelPoint[right](C){$\varGamma$}
\tkzLabelPoint[below](M){$M$}
\tkzDrawPoints(A,B,C,M)
\node at (-0.5,0.25) {$\mu_a$};
\end{tikzpicture} &  & \begin{tikzpicture}[x=1cm,y=1cm]
\clip (-2,-.98) rectangle (2,1.75);
\draw[pl] (-0.5,1.25) node(A){} -- (-1.5,-0.5) node(B){} 
-- (1.5,-0.5) node(C){}--cycle;
\tkzDefLine[bisector](B,A,C) \tkzGetPoint{a}
\tkzInterLL(A,a)(B,C) \tkzGetPoint{D}
\tkzDrawSegment[plm,\xrwma](A,D)
\tkzMarkAngle[size=4mm,mark=|](B,A,D)
\tkzMarkAngle[size=5mm,mark=|](D,A,C)
\tkzLabelPoint[above](A){$A$}
\tkzLabelPoint[left](B){$B$}
\tkzLabelPoint[right](C){$\varGamma$}
\tkzLabelPoint[below](D){$\varDelta$}
\tkzDrawPoints(A,B,C,D)
\node at (-0.6,0.25) {$\delta_a$};
\end{tikzpicture} &  & \begin{tikzpicture}[x=1cm,y=1cm]
\draw[pl] (-0.5,1.25) node(A){} -- (-1.5,-0.5) node(B){} 
-- (1.5,-0.5) node(C){}--cycle;
\tkzDefPoint(-.5,-.5){M}
\tkzMarkRightAngle(C,M,A)
\draw[\xrwma,plm] (-0.5,1.25)--(M);
\tkzLabelPoint[above](A){$A$}
\tkzLabelPoint[left](B){$B$}
\tkzLabelPoint[right](C){$\varGamma$}
\tkzLabelPoint[below](M){$H$}
\tkzDrawPoints(A,B,C,M)
\node at (-0.2,0.25) {$\upsilon_a$};
\end{tikzpicture} \\ 
\end{tabular} 
\end{center}
\begin{enumerate}[label=\bf\arabic*.,start=3]
\item \textbf{Ύψος}\\
Ύψος ενός τριγώνου ονομάζεται το ευθύγραμμο τμήμα το οποίο έχει το ένα άκρο του σε μια κορυφή του τριγώνου και είναι κάθετο με την απέναντι πλευρά.
\begin{itemize}
\item Τα ύψη ενός τριγώνου συμβολίζονται με το γράμμα $ \upsilon $ το οποίο θα έχει δείκτη, το όνομα της πλευράς στην οποία αντιστοιχεί η διχοτόμος. 
\item Τα ύψη για ένα τρίγωνο $ AB\varGamma $ θα συμβολίζονται $ \upsilon_a,\upsilon_\beta,\upsilon_\gamma $.
\end{itemize}
\end{enumerate}
\Orismos{Είδη τριγώνων}
Τα τρίγωνα μπορούν να χωριστούν σε κατηγορίες ως προς το είδος των γωνιών που περιέχουν και ως προς τη σχέση των πλευρων μεταξύ τους.
\begin{enumerate}[label=\bf\arabic*.]
\item \textbf{Είδη τριγώνων ως προς τις γωνίες}\\
Με κριτήριο το είδος των γωνιών που περιέχει ένα τρίγωνο διακρίνουμε τα παρακάτω τρία είδη τριγώνων.
\begin{center}
\begin{tabular}{>{\centering\arraybackslash}m{4.5cm}|>{\centering\arraybackslash}m{4.5cm}|>{\centering\arraybackslash}m{4.5cm}}
\hline \rule[-2ex]{0pt}{5.5ex} \textbf{Οξυγώνιο} & \textbf{Ορθογώνιο} & \textbf{Αμβλυγώνιο} \\ 
\hhline{===} \vspace{2mm}\begin{tikzpicture}
\tkzDefPoint(1,1.5){A}
\tkzDefPoint(0,0){B}
\tkzDefPoint(2.7,0){C}
\tkzMarkAngle[size=3.5mm,fill=\xrwma!50](B,A,C)
\tkzMarkAngle[size=4mm,fill=\xrwma!50](A,C,B)
\tkzMarkAngle[size=3.4mm,fill=\xrwma!50](C,B,A)
\tkzDrawPolygon[pl](A,B,C)
\tkzDrawPoints(A,B,C)
\tkzLabelPoint[above](A){$A$}
\tkzLabelPoint[left](B){$B$}
\tkzLabelPoint[right](C){$\varGamma$}
\node at (1.35,-.4){$\hat{A},\hat{B},\hat{\varGamma}<90\degree$};
\end{tikzpicture}\vspace{2mm} & \begin{tikzpicture}
\tkzDefPoint(0,1.5){A}
\tkzDefPoint(0,0){B}
\tkzDefPoint(2.7,0){C}
\tkzMarkAngle[size=4mm](B,A,C)
\tkzMarkAngle[size=4mm](A,C,B)
\tkzMarkRightAngle[fill=\xrwma!50](C,B,A)
\tkzDrawPolygon[pl](A,B,C)
\tkzDrawPoints(A,B,C)
\tkzLabelPoint[above](A){$A$}
\tkzLabelPoint[left](B){$B$}
\tkzLabelPoint[right](C){$\varGamma$}
\node at (1.35,-.4){$\hat{B}=90\degree$};
\end{tikzpicture} & \begin{tikzpicture}
\tkzDefPoint(0,1.5){A}
\tkzDefPoint(0.5,0){B}
\tkzDefPoint(2.7,0){C}
\tkzMarkAngle[size=4mm](B,A,C)
\tkzMarkAngle[size=4mm](A,C,B)
\tkzMarkAngle[size=3mm,fill=\xrwma!50](C,B,A)
\tkzDrawPolygon[pl](A,B,C)
\tkzDrawPoints(A,B,C)
\tkzLabelPoint[above](A){$A$}
\tkzLabelPoint[left](B){$B$}
\tkzLabelPoint[right](C){$\varGamma$}
\node at (1.35,-.4){$\hat{B}>90\degree$};
\end{tikzpicture}
\\ \hline \vspace{2mm}Ένα τρίγωνο ονομάζεται
\textbf{οξυγώνιο} εαν έχει \textbf{όλες} τις γωνίες του οξείες.\vspace{2mm} & Ένα τρίγωνο ονομάζεται \textbf{ορθογώνιο} εαν έχει μια ορθή γωνία. & Ένα τρίγωνο ονομάζεται \textbf{αμβλυγώνιο} εαν έχει μια αμβλεία γωνία.\\ 
\hline 
\end{tabular}
\end{center}
\item \textbf{Είδη τριγώνων ως προς τις πλευρές}\\
Με βάση τη σχέση μεταξύ των πλευρών ενός τριγώνου χωρίζουμε τα τρίγωνα στις παρακάτω τρεις κατηγορίες.
\begin{center}
\begin{tabular}{>{\centering\arraybackslash}m{4.7cm}|>{\centering\arraybackslash}m{4.7cm}|>{\centering\arraybackslash}m{4.7cm}}
\hline \rule[-2ex]{0pt}{5.5ex} \textbf{Σκαληνό} & \textbf{Ισοσκελές} & \textbf{Ισόπλευρο} \\ 
\hhline{===} \vspace{2mm}\begin{tikzpicture}
\tkzDefPoint(1,1.5){A}
\tkzDefPoint(0,0){B}
\tkzDefPoint(2.7,0){C}
\tkzDrawPolygon[pl](A,B,C)
\tkzDrawPoints(A,B,C)
\tkzLabelPoint[above](A){$A$}
\tkzLabelPoint[left](B){$B$}
\tkzLabelPoint[right](C){$\varGamma$}
\node at (1.35,-.4){$AB\neq A\varGamma\neq B\varGamma$};
\end{tikzpicture}\vspace{2mm} & \begin{tikzpicture}
\tkzDefPoint(1.35,1.5){A}
\tkzDefPoint(0,0){B}
\tkzDefPoint(2.7,0){C}
\tkzDrawSegments[plm,\xrwma](A,B A,C)
\tkzMarkSegments[mark=|](A,B A,C)
\tkzDrawSegments[pl](B,C)
\tkzDrawPoints(A,B,C)
\tkzLabelPoint[above](A){$A$}
\tkzLabelPoint[left](B){$B$}
\tkzLabelPoint[right](C){$\varGamma$}
\node at (1.35,-.4){$AB=A\varGamma$};
\end{tikzpicture} & \begin{tikzpicture}
\tkzDefPoint(0.86,1.5){A}
\tkzDefPoint(0,0){B}
\tkzDefPoint(1.73,0){C}
\tkzDrawSegments[pl](A,B A,C B,C)
\tkzMarkSegments[mark=|](A,B A,C B,C)
\tkzDrawPoints(A,B,C)
\tkzLabelPoint[above](A){$A$}
\tkzLabelPoint[left](B){$B$}
\tkzLabelPoint[right](C){$\varGamma$}
\node at (0.86,-.4){$AB=A\varGamma=B\varGamma$};
\end{tikzpicture}
\\ \hline \vspace{2mm}Ένα τρίγωνο ονομάζεται
\textbf{σκαληνό} εαν όλες οι πλευρές του είναι μεταξύ τους άνισες.\vspace{2mm} & Ένα τρίγωνο ονομάζεται \textbf{ισοσκελές} εαν έχει δύο πλευρές ίσες. Η τρίτη πλευρά ονομάζεται \textbf{βάση}. & Ένα τρίγωνο ονομάζεται \textbf{ισόπλευρο} εαν έχει όλες τις πλευρές του ίσες.\\ 
\hline 
\end{tabular}
\end{center}
\end{enumerate}
\thewrhmata
\Thewrhma{1\tssL{o} Κριτήριο Ισότητασ Τριγώνων}
Αν ένα τρίγωνο έχει δύο πλευρές τους ίσες μια προς μια και τις περιεχόμενες σ' αυτές γωνίες μεταξύ τους ίσες τότε έιναι ίσα.
\begin{center}
\begin{tikzpicture}[scale=.8]
\coordinate [label=left:{$ B $}] (B) at (0,0);
\coordinate [label=right:{$ \varGamma $}] (C) at (4,0);
\coordinate[label=above:{$ A $}] (A) at (1,2);
\coordinate [label=left:{$ B' $}] (B') at (5.5,0);
\coordinate [label=right:{$ \varGamma' $}] (C') at (9.5,0);
\coordinate [label=above:{$ A' $}] (A') at (6.5,2);
\tkzDrawPolygon[pl](A,B,C)
\tkzDrawPolygon[pl](A',B',C')
\tkzMarkAngle[fill=\xrwma,%
size=0.4](B,A,C)
\tkzMarkAngle[fill=\xrwma,%
size=0.4,](B',A',C')
\tkzMarkSegments[mark=|,color=\xrwma](A,B A',B')
\tkzMarkSegments[mark=||,color=\xrwma](A,C A',C')
\tkzDrawPoints(A,B,C,A',B',C')
\node at (12,1.5) {$AB=A'B'$};
\node at (12,1) {$A\varGamma=A'\varGamma'$};
\node at (12,.5) {$\hat{A}=\hat{A'}$};
\end{tikzpicture}
\end{center}
\Thewrhma{2\tssL{o} Κριτήριο Ισότητασ Τριγώνων}
Αν δυο τριγωνα έχουν μια πλευρά και τις προσκείμενες σ' αυτήν γωνίες ίσες, τότε ειναι ίσα.
\begin{center}
\begin{tikzpicture}[scale=.8]
\tkzDefPoint(-1.4,0){E}
\coordinate [label=left:{$ B $}] (B) at (0,0);
\coordinate [label=right:{$ \varGamma $}] (C) at (4,0);
\coordinate[label=above:{$ A $}] (A) at (1,2);
\coordinate [label=left:{$ B' $}] (B') at (5.5,0);
\coordinate [label=right:{$ \varGamma' $}] (C') at (9.5,0);
\coordinate [label=above:{$ A' $}] (A') at (6.5,2);
\tkzDrawPolygon[pl](A,B,C)
\tkzDrawPolygon[pl](A',B',C')
\tkzMarkAngle[fill=\xrwma,%
size=0.4,mark=|](C',B',A')
\tkzMarkAngle[fill=\xrwma,%
size=0.4,mark=|](C,B,A)
\tkzMarkAngle[fill=\xrwma,%
size=0.5,mark=||](A,C,B)
\tkzMarkAngle[fill=\xrwma,%
size=0.5,mark=||](A',C',B')
\tkzMarkSegments[mark=|,color=\xrwma](B,C B',C')
\tkzDrawPoints(A,B,C,A',B',C')
\node at (12,1.5) {$B\varGamma=B'\varGamma'$};
\node at (12,1) {$\hat{B}=\hat{B'}$};
\node at (12,.5) {$\hat{\varGamma}=\hat{\varGamma'}$};
\end{tikzpicture}
\end{center}
\Thewrhma{3\tssL{o} Κριτήριο Ισότητασ Τριγώνων}
Αν δυο τριγωνα έχουν όλες τις πλευρές τους ίσες μια προς μια, τότε ειναι ίσα.
\begin{center}
\begin{tikzpicture}[scale=.8]
\tkzDefPoint(-1.4,0){E}
\coordinate [label=left:{$ B $}] (B) at (0,0);
\coordinate [label=right:{$ \varGamma $}] (C) at (4,0);
\coordinate[label=above:{$ A $}] (A) at (1,2);
\coordinate [label=left:{$ B' $}] (B') at (5.5,0);
\coordinate [label=right:{$ \varGamma' $}] (C') at (9.5,0);
\coordinate [label=above:{$ A' $}] (A') at (6.5,2);
\tkzDrawPolygon[pl](A,B,C)
\tkzDrawPolygon[pl](A',B',C')
\tkzMarkSegments[mark=|,color=\xrwma](A,B A',B')
\tkzMarkSegments[mark=||,color=\xrwma](A,C A',C')
\tkzMarkSegments[mark=|||,color=\xrwma](B,C B',C')
\tkzDrawPoints(A,B,C,A',B',C')
\node at (12,1.5) {$AB=A'B'$};
\node at (12,1) {$A\varGamma=A'\varGamma'$};
\node at (12,.5) {$B\varGamma=B'\varGamma'$};
\end{tikzpicture}
\end{center}
\Thewrhma{1\tssL{o} Πόρισμα για το ισοσκελές τρίγωνο}
Σε κάθε ισοσκελές τριγωνο\\
\wrapr{-11mm}{5}{3.8cm}{-9mm}{\begin{tikzpicture}[scale=.9]
\coordinate [label=left:$ B $] (B) at (0,0);
\coordinate [label=right:$ \varGamma $] (C) at (3,0);
\coordinate[label=above:$ A $] (A) at (1.5,2.7);
\tkzDrawPolygon[pl](A,B,C)
\tkzDefMidPoint(B,C) \tkzGetPoint{D}
\tkzDrawSegment[pl](A,D)
\tkzMarkAngle[mark=|,fill=\xrwma,%
size=0.5,   
opacity=0.7](A,C,B)
\tkzMarkAngle[mark=|,fill=\xrwma,%
size=0.5,   
opacity=0.7](C,B,A)
\tkzMarkSegments[mark=|,color=\xrwma](A,B A,C)
\tkzLabelPoint[below](D){$ \varDelta $}
\tkzMarkRightAngles(C,D,A)
\tkzDrawPoints(A,B,C,D)
\end{tikzpicture}}{
\begin{itemize}[itemsep=0mm]
\item Οι προσκείμενες γωνίες στη βάση είναι ίσες.
\item Η διάμεσος το ύψος και η διχοτόμος της γωνίας της κορυφης του ισοσκελούς τριγώνου συμπίπτουν.
\end{itemize}}\mbox{}\\\\\\
\Thewrhma{1\tssL{ο} Πόρισμα για τη μεσοκάθετο}
\wrapr{-4mm}{7}{4cm}{-10mm}{\begin{tikzpicture}[scale=.8]
\coordinate [label=left:$ A $] (A) at (0,0);
\coordinate [label=right:$ B $] (B) at (4,0);
\coordinate[label=left:$ M $] (M) at (2,3);
\tkzDrawPolygon[pl](A,B,M)
\tkzDefMidPoint(A,B) \tkzGetPoint{K}
\tkzDrawLine(M,K)
\tkzMarkSegments[mark=|,color=\xrwma](A,K K,B)
\tkzLabelPoint[below left](K){$ K $}
\tkzText(2.2,4.2){$ \varepsilon $}
\tkzMarkRightAngle(B,K,M)
\tkzDrawPoints(A,B,M,K)
\end{tikzpicture}}{
Κάθε σημείο της μεσοκαθέτου ενός ευθυγράμμου τμήματος ισαπέχει από τα άκρα του.\\\\
\Thewrhma{2\tssL{ο} Πόρισμα για τη μεσοκάθετο}
Κάθε σημείο το οποίο ισαπέχει από τα άκρα ενός ευθυγράμμου τμήματος, θα ανήκει στη μεσοκάθετό του.}\mbox{}\\\\\\
\Thewrhma{1\tssL{ο} Κριτήριο ισότητας ορθογωνίων τριγώνων}
Αν δύο ορθογώνια τρίγωνα έχουν δύο πλευρές ίσες τότε έιναι ίσα.
\begin{center}
\begin{tikzpicture}[scale=.9]
\tkzDefPoint(0,0){A}
\tkzDefPoint(3,0){B}
\tkzDefPoint(0,2){C}
\tkzLabelPoint[left](A){$A$}
\tkzLabelPoint[right](B){$B$}
\tkzLabelPoint[above](C){$\varGamma$}
\tkzMarkRightAngle(B,A,C)
\tkzDrawSegments[pl](A,B B,C C,A)
\tkzDrawPoints(A,B,C)
\tkzDefPoint(4.3,0){A'}
\tkzDefPoint(7.3,0){B'}
\tkzDefPoint(4.3,2){C'}
\tkzLabelPoint[left](A'){$A'$}
\tkzLabelPoint[right](B'){$B'$}
\tkzLabelPoint[above](C'){$\varGamma'$}
\tkzMarkRightAngle(B',A',C')
\tkzDrawSegments[pl](A',B' B',C' C',A')
\tkzDrawPoints(A',B',C')
\tkzMarkSegments[mark=|,\xrwma](A,B A',B')
\tkzMarkSegments[mark=||,\xrwma](C,B C',B')
\end{tikzpicture}
\end{center}
\Thewrhma{2\tssL{ο} Κριτήριο ισότητας ορθογωνίων τριγώνων}
Αν δύο τρίγωνα έχουν μια πλευρά και μια οξεία γωνία ίσες αντίστοιχα τότε είναι ίσα.
\begin{center}
\begin{tikzpicture}[scale=.9]
\tkzDefPoint(0,0){A}
\tkzDefPoint(3,0){B}
\tkzDefPoint(0,2){C}
\tkzLabelPoint[left](A){$A$}
\tkzLabelPoint[right](B){$B$}
\tkzLabelPoint[above](C){$\varGamma$}
\tkzMarkRightAngle(B,A,C)
\tkzMarkAngle[size=5mm,mark=|,fill=\xrwma](C,B,A)
\tkzDrawSegments[pl](A,B B,C C,A)
\tkzDrawPoints(A,B,C)
\tkzDefPoint(4.3,0){A'}
\tkzDefPoint(7.3,0){B'}
\tkzDefPoint(4.3,2){C'}
\tkzLabelPoint[left](A'){$A'$}
\tkzLabelPoint[right](B'){$B'$}
\tkzLabelPoint[above](C'){$\varGamma'$}
\tkzMarkRightAngle(B',A',C')
\tkzMarkAngle[size=5mm,mark=|,fill=\xrwma](C',B',A')
\tkzDrawSegments[pl](A',B' B',C' C',A')
\tkzDrawPoints(A',B',C')
\tkzMarkSegments[mark=|](A,B A',B')
\end{tikzpicture}
\end{center}
\Thewrhma{Διχοτόμος γωνίας}
Τα σημεία της διχοτόμου μιας γωνίας ισαπέχουν από τις πλευρές της. Αντίστροφα, κάθε σημείο που ισαπέχει από τις πλευρές μιας γωνίας θα ανήκει στη διχοτόμο της.
\begin{center}
\begin{tikzpicture}[scale=1.2]
\clip (-.5,-.5) rectangle (5.2,2.2);
\tkzDefPoint(0,0){A}
\tkzDefPoint(3,0){B}
\tkzDefPoint(2.5,1.7){C}
\draw (B) -- (A) -- (C);
\tkzDrawBisector[draw=\xrwma](B,A,C)\tkzGetPoint{a}
\tkzDefPointWith[linear,K=0.8](A,a) \tkzGetPoint{D}
\tkzDefPointBy[projection=onto A--C](D)
\tkzGetPoint{h}
\tkzDrawSegment(D,h)
\tkzMarkRightAngle[fill=\xrwma](A,h,D)
\tkzDefPointBy[projection=onto A--B](D)
\tkzGetPoint{f}
\tkzDrawSegment(D,f)
\tkzMarkRightAngle[fill=\xrwma](A,f,D)
\tkzLabelPoint[left](A){$A$}
\tkzLabelPoint[above,xshift=2mm](D){$M$}
\tkzLabelPoint[above](h){$B$}
\tkzLabelPoint[below](f){$\varGamma$}
\tkzLabelPoint[above](C){$y$}
\tkzLabelPoint[right](B){$x$}
\tkzDrawPoints(A,h,f,D)
\node at (4,.8){$ MB=M\varGamma $};
\end{tikzpicture}
\end{center}
Προκύπτει λοιπόν ότι η διχοτόμος μιας γωνίας είναι ο γεωμετρικός τόπος των σημείων του επιπέδου που ισαπέχουν από τις πλευρές της γωνίας.\\\\
\end{document}
