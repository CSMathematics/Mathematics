\PassOptionsToPackage{no-math,cm-default}{fontspec}
\documentclass[twoside,nofonts,internet,shmeiwseis]{thewria}
\usepackage{amsmath}
\usepackage{xgreek}
\let\hbar\relax
\defaultfontfeatures{Mapping=tex-text,Scale=MatchLowercase}
\setmainfont[Mapping=tex-text,Numbers=Lining,Scale=1.0,BoldFont={Minion Pro Bold}]{Minion Pro}
\newfontfamily\scfont{GFS Artemisia}
\font\icon = "Webdings"
\usepackage[amsbb]{mtpro2}
\usepackage{tikz,pgfplots}
\tkzSetUpPoint[size=7,fill=white]
\xroma{red!70!black}
%------- ΣΥΣΤΗΜΑ -------------------
\usepackage{systeme,regexpatch}
\makeatletter
% change the definition of \sysdelim not to store `\left` and `\right`
\def\sysdelim#1#2{\def\SYS@delim@left{#1}\def\SYS@delim@right{#2}}
\sysdelim\{. % reinitialize

% patch the internal command to use
% \LEFTRIGHT<left delim><right delim>{<system>}
% instead of \left<left delim<system>\right<right delim>
\regexpatchcmd\SYS@systeme@iii
{\cB.\c{SYS@delim@left}(.*)\c{SYS@delim@right}\cE.}
{\c{SYS@MT@LEFTRIGHT}\cB\{\1\cE\}}
{}{}
\def\SYS@MT@LEFTRIGHT{%
\expandafter\expandafter\expandafter\LEFTRIGHT
\expandafter\SYS@delim@left\SYS@delim@right}
\makeatother
\newcommand{\synt}[2]{{\scriptsize \begin{matrix}
\times#1\\\\ \times#2
\end{matrix}}}
%----------------------------------------
%------ ΜΗΚΟΣ ΓΡΑΜΜΗΣ ΚΛΑΣΜΑΤΟΣ ---------
\DeclareRobustCommand{\frac}[3][0pt]{%
{\begingroup\hspace{#1}#2\hspace{#1}\endgroup\over\hspace{#1}#3\hspace{#1}}}
%----------------------------------------

\newlist{rlist}{enumerate}{3}
\setlist[rlist]{itemsep=0mm,label=\roman*.}
\newlist{brlist}{enumerate}{3}
\setlist[brlist]{itemsep=0mm,label=\bf\roman*.}
\newlist{tropos}{enumerate}{3}
\setlist[tropos]{label=\bf\textit{\arabic*\textsuperscript{oς}\;Τρόπος :},leftmargin=0cm,itemindent=2.3cm,ref=\bf{\arabic*\textsuperscript{oς}\;Τρόπος}}
\newcommand{\tss}[1]{\textsuperscript{#1}}
\newcommand{\tssL}[1]{\MakeLowercase{\textsuperscript{#1}}}

\usepackage{hhline}
%----------- ΓΡΑΦΙΚΕΣ ΠΑΡΑΣΤΑΣΕΙΣ ---------
\pgfkeys{/pgfplots/aks_on/.style={axis lines=center,
xlabel style={at={(current axis.right of origin)},xshift=1.5ex, anchor=center},
ylabel style={at={(current axis.above origin)},yshift=1.5ex, anchor=center}}}
\pgfkeys{/pgfplots/grafikh parastash/.style={\xrwma,line width=.4mm,samples=200}}
\pgfkeys{/pgfplots/belh ar/.style={tick label style={font=\scriptsize},axis line style={-latex}}}
%-----------------------------------------
\usepackage{multicol}
\usepackage{wrap-rl}


\begin{document}
\titlos{Μαθηματικά Γ΄ Γυμνασίου}{Εξισώσεις}{Εξισώσεις 2\tssL{ου} Βαθμού}
\orismoi
\Orismos{ΤΡΙΩΝΥΜΟ 2\MakeLowercase{\textsuperscript{ου}} ΒΑΘΜΟΥ} Τριώνυμο 2\textsuperscript{ου} βαθμού ονομάζεται κάθε πολυώνυμο 2\textsuperscript{ου} βαθμού με τρεις όρους και είναι της μορφής \[ ax^2+\beta x+\gamma\;\textrm{ με }\;a\neq0 \]
\begin{itemize}[itemsep=0mm]
\item Οι πραγματικοί αριθμοί $ a,\beta,\gamma\in\mathbb{R} $ ονομάζονται \textbf{συντελεστές} του τριωνύμου.
\item Ο συντελεστής $ \gamma\in\mathbb{R} $ ονομάζεται \textbf{σταθερός όρος}.
\end{itemize}
\Orismos{εξίσωση 2\textsuperscript{\MakeLowercase{ου}} βαθμού}
Εξίσωση 2\textsuperscript{ου} βαθμού με έναν άγνωστο ονομάζεται κάθε πολυωνυμική εξίσωση της οποίας η αλγεβρική παράσταση είναι τριώνυμο 2\textsuperscript{ου} βαθμού. Είναι της μορφής :
\[ ax^2+\beta x+\gamma=0\;\;,\;\;a\neq0 \]
\Orismos{Διακρίνουσα}
Διακρίνουσα ενός τριωνύμου 2\textsuperscript{ου} βαθμού ονομάζεται ο πραγματικός αριθμός
\[ \varDelta=\beta^2-4a\gamma \]
Το πρόσημό της μας επιτρέπει να διακρίνουμε το πλήθος των ριζών του τριωνύμου.\\\\
\thewrhmata
\Thewrhma{λυσεισ εξισωσησ 2\textsuperscript{\MakeLowercase{ου}} βαθμου}
Αν $ ax^2+\beta x+\gamma=0 $ με $ a\neq0 $ μια εξίσωση 2\textsuperscript{ου} βαθμού τότε με βάση το πρόσημο της διακρίνουσας έχουμε τις παρακάτω περιπτώσεις για το πλήθος των λύσεων της :\\
\wrapr{-10mm}{10}{9.2cm}{4mm}{\begin{tabular}{ccc}
\hline\textbf{Διακρίνουσα} & \textbf{Πλήθος λύσεων} & \textbf{Λύσεις} \rule[-2ex]{0pt}{5.5ex}\\ 
\hhline{===}\rule[-2ex]{0pt}{7ex} $ \varDelta>0 $ &  2 λύσεις & $ x_{1,2}=\dfrac{-\beta\pm\!\sqrt{\varDelta}}{2a} $  \\
\rule[-2ex]{0pt}{5.5ex} $ \varDelta=0 $ & 1 διπλή λύση & $ x=-\dfrac{\beta}{a} $\\
\rule[-2ex]{0pt}{5.5ex} $ \varDelta<0 $ & \multicolumn{2}{c}{Καμία λύση}\\
\hline 
\end{tabular}}{\begin{enumerate}[itemsep=0mm]
\item Αν $ \varDelta>0 $ τότε η εξίσωση έχει δύο άνισες λύσεις οι οποίες δίνονται από τον τύπο : \[ x_{1,2}=\frac{-\beta\pm\!\sqrt{\varDelta}}{2a} \]
\item Αν $ \varDelta=0 $ τότε η εξίσωση έχει μια διπλή λύση την \[ x=-\frac{\beta}{a} \]
\end{enumerate}}
\begin{enumerate}[itemsep=0mm,start=3]
\item Αν $ \varDelta<0 $ τότε η εξίσωση είναι αδύνατη στο σύνολο $ \mathbb{R} $. Οι περιπτώσεις αυτές φαίνονται επίσης στον παραπάνω πίνακα :
\end{enumerate}
\Thewrhma{Παραγοντοποίηση Τριωνύμου}
Ένα τριώνυμο της μορφής $ ax^2+\beta x+\gamma=0 $ με $ a\neq0 $ μπορεί να γραφτεί ως γινόμενο παραγόντων σύμφωνα με τον παρακάτω κανόνα :
\begin{enumerate}[itemsep=0mm]
\item Αν η διακρίνουσα του τριωνύμου είναι θετική $\left( \varDelta>0\right)  $ τότε το τριώνυμο παραγοντοποιείται ως εξής \[ ax^2+\beta x+\gamma=a(x-x_1)(x-x_2) \]
όπου $ x_1,x_2 $ είναι οι ρίζες του τριωνύμου.
\item Αν η διακρίνουσα είναι μηδενική $\left( \varDelta=0\right)  $ τότε το τριώνυμο παραγοντοποιείται ως εξής : \[ ax^2+\beta x+\gamma=a\left(x-x_0\right)^2  \]
όπου $ x_0 $ είναι η διπλή ρίζα του τριωνύμου.
\item Αν η διακρίνουσα είναι αρνητική $\left( \varDelta<0\right)  $ τότε το τριώνυμο δεν γράφεται ως γινόμενο πρώτων παραγόντων.
\end{enumerate}
\end{document}
