\PassOptionsToPackage{no-math,cm-default}{fontspec}
\documentclass[twoside,nofonts,internet,shmeiwseis]{thewria}
\usepackage{amsmath}
\usepackage{xgreek}
\let\hbar\relax
\defaultfontfeatures{Mapping=tex-text,Scale=MatchLowercase}
\setmainfont[Mapping=tex-text,Numbers=Lining,Scale=1.0,BoldFont={Minion Pro Bold}]{Minion Pro}
\newfontfamily\scfont{GFS Artemisia}
\font\icon = "Webdings"
\usepackage[amsbb,subscriptcorrection,zswash,mtpcal,mtphrb]{mtpro2}
\usepackage{tikz,pgfplots}
\tkzSetUpPoint[size=7,fill=white]
\xroma{red!70!black}
%------TIKZ - ΣΧΗΜΑΤΑ - ΓΡΑΦΙΚΕΣ ΠΑΡΑΣΤΑΣΕΙΣ ----
\usepackage{tikz}
\usepackage{tkz-euclide}
\usetkzobj{all}
\usepackage[framemethod=TikZ]{mdframed}
\usetikzlibrary{decorations.pathreplacing}
\usepackage{pgfplots}
\usetkzobj{all}
%-----------------------
\usepackage{calc}
\usepackage{hhline}
\usepackage[explicit]{titlesec}
\usepackage{graphicx}
\usepackage{multicol}
\usepackage{multirow}
\usepackage{enumitem}
\usepackage{tabularx}
\usepackage[decimalsymbol=comma]{siunitx}
\usetikzlibrary{backgrounds}
\usepackage{sectsty}
\sectionfont{\centering}
\setlist[enumerate]{label=\bf{\large \arabic*.}}
\usepackage{adjustbox}
\usepackage{mathimatika,gensymb,eurosym,wrap-rl}
\usepackage{systeme,regexpatch}
%-------- ΜΑΘΗΜΑΤΙΚΑ ΕΡΓΑΛΕΙΑ ---------
\usepackage{mathtools}
%----------------------
%-------- ΠΙΝΑΚΕΣ ---------
\usepackage{booktabs}
%----------------------
%----- ΥΠΟΛΟΓΙΣΤΗΣ ----------
\usepackage{calculator}
%----------------------------
%------ ΔΙΑΓΩΝΙΟ ΣΕ ΠΙΝΑΚΑ -------
\usepackage{array}
\newcommand\diag[5]{%
\multicolumn{1}{|m{#2}|}{\hskip-\tabcolsep
$\vcenter{\begin{tikzpicture}[baseline=0,anchor=south west,outer sep=0]
\path[use as bounding box] (0,0) rectangle (#2+2\tabcolsep,\baselineskip);
\node[minimum width={#2+2\tabcolsep-\pgflinewidth},
minimum  height=\baselineskip+#3-\pgflinewidth] (box) {};
\draw[line cap=round] (box.north west) -- (box.south east);
\node[anchor=south west,align=left,inner sep=#1] at (box.south west) {#4};
\node[anchor=north east,align=right,inner sep=#1] at (box.north east) {#5};
\end{tikzpicture}}\rule{0pt}{.71\baselineskip+#3-\pgflinewidth}$\hskip-\tabcolsep}}
%---------------------------------
%---- ΟΡΙΖΟΝΤΙΟ - ΚΑΤΑΚΟΡΥΦΟ - ΠΛΑΓΙΟ ΑΓΚΙΣΤΡΟ ------
\newcommand{\orag}[3]{\node at (#1)
{$ \overcbrace{\rule{#2mm}{0mm}}^{{\scriptsize #3}} $};}
\newcommand{\kag}[3]{\node at (#1)
{$ \undercbrace{\rule{#2mm}{0mm}}_{{\scriptsize #3}} $};}
\newcommand{\Pag}[4]{\node[rotate=#1] at (#2)
{$ \overcbrace{\rule{#3mm}{0mm}}^{{\rotatebox{-#1}{\scriptsize$#4$}}}$};}
%-----------------------------------------
%------------------------------------------
\newcommand{\tss}[1]{\textsuperscript{#1}}
\newcommand{\tssL}[1]{\MakeLowercase{\textsuperscript{#1}}}
%---------- ΛΙΣΤΕΣ ----------------------
\newlist{bhma}{enumerate}{3}
\setlist[bhma]{label=\bf\textit{\arabic*\textsuperscript{o}\;Βήμα :},leftmargin=0cm,itemindent=1.8cm,ref=\bf{\arabic*\textsuperscript{o}\;Βήμα}}
\newlist{rlist}{enumerate}{3}
\setlist[rlist]{itemsep=0mm,label=\roman*.}
\newlist{brlist}{enumerate}{3}
\setlist[brlist]{itemsep=0mm,label=\bf\roman*.}
\newlist{tropos}{enumerate}{3}
\setlist[tropos]{label=\bf\textit{\arabic*\textsuperscript{oς}\;Τρόπος :},leftmargin=0cm,itemindent=2.3cm,ref=\bf{\arabic*\textsuperscript{oς}\;Τρόπος}}
% Αν μπει το bhma μεσα σε tropo τότε
%\begin{bhma}[leftmargin=.7cm]
\tkzSetUpPoint[size=7,fill=white]
\tikzstyle{pl}=[line width=0.3mm]
\tikzstyle{plm}=[line width=0.4mm]
\usepackage{etoolbox}
\makeatletter
\renewrobustcmd{\anw@true}{\let\ifanw@\iffalse}
\renewrobustcmd{\anw@false}{\let\ifanw@\iffalse}\anw@false
\newrobustcmd{\noanw@true}{\let\ifnoanw@\iffalse}
\newrobustcmd{\noanw@false}{\let\ifnoanw@\iffalse}\noanw@false
\renewrobustcmd{\anw@print}{\ifanw@\ifnoanw@\else\numer@lsign\fi\fi}
\makeatother

\begin{document}
\titlos{Μαθηματικά Γ΄ Γυμνασίου}{Γεωμετρία}{Λόγος ευθυγράμμων τμημάτων}
\orismoi
\Orismos{Λόγοσ ευθυγράμμων τμημάτων}
Λόγος δύο ευθυγράμμων τμημάτων $ AB $ και $ \varGamma\varDelta $ ονομάζεται ο θετικός αριθμός $ \lambda $ ο οποίος είναι ίσος με το πηλίκο τους ή ισοδύναμα το πηλίκο των μέτρων τους.
\[ \lambda=\frac{AB}{\varGamma\varDelta} \]
\Orismos{Αναλογία}
Αναλογία ευθυγράμμων τμημάτων ονομάζεται η ισότητα δύο ή περισσότερων λόγων ευθυγράμμων τμημάτων. Αν $ a,\beta,\gamma,\delta $ είναι ευθύγραμμα τμήματα τότε η αναλογία έχει ως εξής
\[ \frac{a}{\beta}=\frac{\gamma}{\delta}=\lambda \]
\begin{itemize}[itemsep=0mm]
\item Τα ευθύγραμμα τμήματα $ a,\beta,\gamma,\delta $ ονομάζονται \textbf{όροι} της αναλογίας.
\item Οι αριθμητές της αναλογίας είναι ανάλογοι προς τους παρονομαστές της δηλαδή τα ευθύγραμμα τμήματα $ a,\gamma $ είναι ανάλογα προς τα $ \beta,\delta $.
\item Τα ευθύγραμμα τμήματα $ a $ και $ \delta $ ονομάζονται \textbf{άκροι όροι} ενώ τα $ \beta,\gamma $ \textbf{μέσοι όροι} της αναλογίας.
\item Τα ευθύγραμμα τμήματα που βρίσκονται μέσα στον ίδιο λόγο (κλάσμα) ονομάζονται \textbf{ομόλογα} ή \textbf{αντίστοιχα}.
\end{itemize}
\thewrhmata
\Thewrhma{Ίσα τμήματα από παράλληλες ευθείες}
\wrapr{-4mm}{5}{4.3cm}{-8mm}{\begin{tikzpicture}[scale=1.3]
\tkzDefPoint(0,0){A}
\tkzDefPoint(3,0){B}
\tkzDefPoint(0,.5){C}
\tkzDefPoint(3,0.5){D}
\tkzDefPoint(0,1){E}
\tkzDefPoint(3,1){Z}
\tkzDefPoint(.7,1.3){H}
\tkzDefPoint(.4,-.3){I}
\tkzDefPoint(1.7,1.3){K}
\tkzDefPoint(2.7,-.3){L}
\draw[pl] (A)--(B);
\draw[pl] (C)--(D);
\draw[pl] (E)--(Z);
\draw[pl,\xrwma] (H)--(I);
\draw[pl,\xrwma] (K)--(L);
\tkzInterLL(E,Z)(H,I)\tkzGetPoint{S}
\tkzInterLL(C,D)(H,I)\tkzGetPoint{T}
\tkzInterLL(A,B)(H,I)\tkzGetPoint{Y}
\tkzInterLL(E,Z)(K,L)\tkzGetPoint{O}
\tkzInterLL(C,D)(K,L)\tkzGetPoint{P}
\tkzInterLL(A,B)(K,L)\tkzGetPoint{Q}
\tkzDrawPoints(S,T,Y,O,P,Q)
\tkzLabelPoint[above left](S){$A$}
\tkzLabelPoint[above left](T){$B$}
\tkzLabelPoint[above left](Y){$\varGamma$}
\tkzLabelPoint[above right](O){$\varDelta$}
\tkzLabelPoint[above right](P){$E$}
\tkzLabelPoint[above right](Q){$Z$}
\node at (3.2,1) {\footnotesize$\varepsilon_1$};
\node at (3.2,0.5) {\footnotesize$\varepsilon_2$};
\node at (3.2,0) {\footnotesize$\varepsilon_3$};
\node at (0.6,-0.2) {\footnotesize$\varepsilon$};
\node at (2.4,-0.2) {\footnotesize$\zeta$};
\end{tikzpicture}}{
Αν τρεις ή περισσότερες παράλληλες ευθείες ορίζουν ίσα τμήματα σε μια τέμνουσα, τότε θα ορίζουν ίσα τμήματα και σε οποιαδήποτε άλλη τέμουσα ευθεία.
\[ \varepsilon_1\parallel\varepsilon_2\parallel\varepsilon_3\textrm{ και }AB=B\varGamma\Rightarrow\varDelta E=EZ \]
}\mbox{}\\\\\\
\Thewrhma{Ιδιότητεσ των αναλογιών}
Για κάθε αναλογία με όρους τα ευθύγραμμα τμήματα $ a,\beta,\gamma,\delta $ θα ισχύουν οι παρακάτω ιδιότητες :
\begin{center}
\begin{tabular}{ccc}
\hline \rule[-2ex]{0pt}{5.5ex} & \textbf{Ιδιότητα} & \textbf{Συνθήκη} \\
\hhline{===}\rule[-2ex]{0pt}{5.5ex} \textbf{1} & Χιαστί γινόμενα & $ \dfrac{a}{\beta}=\dfrac{\gamma}{\delta}\Leftrightarrow a\cdot\delta=\beta\cdot\gamma $ \\
\rule[-2ex]{0pt}{7ex} \textbf{2} & Εναλλαγή μέσων και άκρων όρων & $ \dfrac{a}{\beta}=\dfrac{\gamma}{\delta}\Leftrightarrow \dfrac{a}{\gamma}=\dfrac{\beta}{\delta}\;\;\textrm{ και }\;\;\dfrac{\delta}{\beta}=\dfrac{\gamma}{a} $\\
\rule[-2ex]{0pt}{7ex} \textbf{5} & Άθροισμα - Διαφορά αριθμ. και παρονομ. & $ \dfrac{a}{\beta}=\dfrac{\gamma}{\delta}=\dfrac{a\pm\beta}{\gamma\pm\delta} $\\
\hline
\end{tabular}
\end{center}\mbox{}\\
\Thewrhma{Τμήμα από τα μέσα δύο πλευρών}
\wrapr{-4mm}{7}{4.5cm}{-4mm}{\begin{tikzpicture}
\tkzDefPoint(0,0){B}
\tkzDefPoint(3.5,0){C}
\tkzDefPoint(1.2,2){A}
\tkzDefPoint(.6,1){M}
\tkzDefPoint(2.35,1){N}
\draw[pl](A)--(B)--(C)--cycle;
\draw[dashed] (0,1)--(3.3,1);
\draw[plm,\xrwma] (M)--(N);
\tkzDrawPoints(A,B,C,M,N)
\tkzLabelPoint[above](A){$A$}
\tkzLabelPoint[left](B){$B$}
\tkzLabelPoint[right](C){$\varGamma$}
\tkzLabelPoint[left,yshift=2mm](M){$M$}
\tkzLabelPoint[right,yshift=2mm](N){$N$}
\end{tikzpicture}}{
Το ευθύγραμμο τμήμα που ενώνει τα μέσα των δύο πλευρών ενός τριγώνου είναι παράλληλο με την τρίτη πλευρά και ισούται με το μισό της. Θα ισχύει
\[ MN\parallel=\frac{B\varGamma}{2} \]
για ένα τρίγωνο $ AB\varGamma $ με $ M,N $ τα μέσα των πλευρών $ AB,A\varGamma $ αντίστοιχα.}\mbox{}\\\\\\
\Thewrhma{Διάμεσος από ορθή γωνία}
\wrapr{-4mm}{7}{3cm}{-11mm}{\begin{tikzpicture}
\tkzDefPoint(0,0){A}
\tkzDefPoint(2,0){B}
\tkzDefPoint(0,2.2){C}
\tkzDefPoint(1,1.1){M}
\draw[pl,\xrwma](A)--(M);
\tkzMarkRightAngle[size=.3](B,A,C)
\draw[pl](A)--(B)--(C)--cycle;
\tkzMarkSegments[mark=|](C,M M,B M,A)
\tkzLabelPoint[left](A){$A$}
\tkzLabelPoint[right](B){$B$}
\tkzLabelPoint[left](C){$\varGamma$}
\tkzLabelPoint[right,yshift=1mm](M){$M$}
\tkzDrawPoints(A,B,C,M)
\end{tikzpicture}}{
Η διάμεσος που άγεται από την ορθή γωνία προς την υποτείνουσα σε κάθε ορθογώνιο τρίγωνο, ισούται με το μισό της υποτείνουσας.}
\end{document}
