\PassOptionsToPackage{no-math,cm-default}{fontspec}
\documentclass[twoside,nofonts,internet,shmeiwseis]{thewria}
\usepackage{amsmath}
\usepackage{xgreek}
\let\hbar\relax
\defaultfontfeatures{Mapping=tex-text,Scale=MatchLowercase}
\setmainfont[Mapping=tex-text,Numbers=Lining,Scale=1.0,BoldFont={Minion Pro Bold}]{Minion Pro}
\newfontfamily\scfont{GFS Artemisia}
\font\icon = "Webdings"
\usepackage[amsbb]{mtpro2}
\usepackage{tikz,pgfplots}
\tkzSetUpPoint[size=7,fill=white]
\xroma{red!70!black}
%------TIKZ - ΣΧΗΜΑΤΑ - ΓΡΑΦΙΚΕΣ ΠΑΡΑΣΤΑΣΕΙΣ ----
\usepackage{tikz}
\usepackage{tkz-euclide}
\usetkzobj{all}
\usepackage[framemethod=TikZ]{mdframed}
\usetikzlibrary{decorations.pathreplacing}
\usepackage{pgfplots}
\usetkzobj{all}
%-----------------------

%-----ΕΙΚΟΝΑ ΔΙΠΛΑ ΑΠΟ ΚΕΙΜΕΝΟ-------
\usepackage{wrapfig}
\newenvironment{WrapText1}[3][r]
{\wrapfigure[#2]{#1}{#3}}
{\endwrapfigure}

\newenvironment{WrapText2}[3][l]
{\wrapfigure[#2]{#1}{#3}}
{\endwrapfigure}

\newcommand{\wrapr}[6]{
\begin{minipage}{\linewidth}\mbox{}\\
\vspace{#1}
\begin{WrapText1}{#2}{#3}
\vspace{#4}#5\end{WrapText1}#6
\end{minipage}}

\newcommand{\wrapl}[6]{
\begin{minipage}{\linewidth}\mbox{}\\
\vspace{#1}
\begin{WrapText2}{#2}{#3}
\vspace{#4}#5\end{WrapText2}#6
\end{minipage}}
%-------------------------------------------

\usepackage{calc}

\renewcommand{\thepart}{\arabic{part}}

\usepackage[explicit]{titlesec}
\usepackage{graphicx}
\usepackage{multicol}
\usepackage{multirow}
\usepackage{enumitem}
\usepackage{tabularx}
\usepackage{hhline}
\usepackage[decimalsymbol=comma]{siunitx}
\usetikzlibrary{backgrounds}
\usepackage{sectsty}
\sectionfont{\centering}
\usepackage{enumitem}
\setlist[enumerate]{label=\bf{\large \arabic*.}}
\usepackage{adjustbox}
%--------- ΑΓΓΛΙΚΟ ΚΕΙΜΕΝΟ --------------
\newcommand{\eng}[1]{\selectlanguage{english}#1\selectlanguage{greek}}
%----------------------------------------
%------- ΣΥΣΤΗΜΑ -------------------
\usepackage{systeme,regexpatch}
\makeatletter
% change the definition of \sysdelim not to store `\left` and `\right`
\def\sysdelim#1#2{\def\SYS@delim@left{#1}\def\SYS@delim@right{#2}}
\sysdelim\{. % reinitialize

% patch the internal command to use
% \LEFTRIGHT<left delim><right delim>{<system>}
% instead of \left<left delim<system>\right<right delim>
\regexpatchcmd\SYS@systeme@iii
{\cB.\c{SYS@delim@left}(.*)\c{SYS@delim@right}\cE.}
{\c{SYS@MT@LEFTRIGHT}\cB\{\1\cE\}}
{}{}
\def\SYS@MT@LEFTRIGHT{%
\expandafter\expandafter\expandafter\LEFTRIGHT
\expandafter\SYS@delim@left\SYS@delim@right}
\makeatother
\newcommand{\synt}[2]{{\scriptsize \begin{matrix}
\times#1\\\\ \times#2
\end{matrix}}}
%----------------------------------------
%------ ΜΗΚΟΣ ΓΡΑΜΜΗΣ ΚΛΑΣΜΑΤΟΣ ---------
\DeclareRobustCommand{\frac}[3][0pt]{%
{\begingroup\hspace{#1}#2\hspace{#1}\endgroup\over\hspace{#1}#3\hspace{#1}}}
%----------------------------------------
%-------- ΜΑΘΗΜΑΤΙΚΑ ΕΡΓΑΛΕΙΑ ---------
\usepackage{mathtools}
%----------------------

%-------- ΠΙΝΑΚΕΣ ---------
\usepackage{booktabs}
%----------------------
%----- ΥΠΟΛΟΓΙΣΤΗΣ ----------
\usepackage{calculator}
%----------------------------
%------ ΔΙΑΓΩΝΙΟ ΣΕ ΠΙΝΑΚΑ -------
\usepackage{array}
\newcommand\diag[5]{%
\multicolumn{1}{|m{#2}|}{\hskip-\tabcolsep
$\vcenter{\begin{tikzpicture}[baseline=0,anchor=south west,outer sep=0]
\path[use as bounding box] (0,0) rectangle (#2+2\tabcolsep,\baselineskip);
\node[minimum width={#2+2\tabcolsep-\pgflinewidth},
minimum  height=\baselineskip+#3-\pgflinewidth] (box) {};
\draw[line cap=round] (box.north west) -- (box.south east);
\node[anchor=south west,align=left,inner sep=#1] at (box.south west) {#4};
\node[anchor=north east,align=right,inner sep=#1] at (box.north east) {#5};
\end{tikzpicture}}\rule{0pt}{.71\baselineskip+#3-\pgflinewidth}$\hskip-\tabcolsep}}
%---------------------------------

%---- ΟΡΙΖΟΝΤΙΟ - ΚΑΤΑΚΟΡΥΦΟ - ΠΛΑΓΙΟ ΑΓΚΙΣΤΡΟ ------
\newcommand{\orag}[3]{\node at (#1)
{$ \overcbrace{\rule{#2mm}{0mm}}^{{\scriptsize #3}} $};}

\newcommand{\kag}[3]{\node at (#1)
{$ \undercbrace{\rule{#2mm}{0mm}}_{{\scriptsize #3}} $};}

\newcommand{\Pag}[4]{\node[rotate=#1] at (#2)
{$ \overcbrace{\rule{#3mm}{0mm}}^{{\rotatebox{-#1}{\scriptsize$#4$}}}$};}
%-----------------------------------------

%-------- ΤΡΙΓΩΝΟΜΕΤΡΙΚΟΙ ΑΡΙΘΜΟΙ -----------
\newcommand{\hm}[1]{\textrm{ημ}#1}
\newcommand{\syn}[1]{\textrm{συν}#1}
\newcommand{\ef}[1]{\textrm{εφ}#1}
\newcommand{\syf}[1]{\textrm{σφ}#1}
%--------------------------------------------

%--------- ΠΟΣΟΣΤΟ ΤΟΙΣ ΧΙΛΙΟΙΣ ------------
\DeclareRobustCommand{\perthousand}{%
\ifmmode
\text{\textperthousand}%
\else
\textperthousand
\fi}
%------------------------------------------

%------------------------------------------
\usepackage{extarrows}
\newcommand{\eq}[1]{\xlongequal{#1}}
%------------------------------------------
%------ ΌΡΙΣΜΑ ----------
\newcommand{\Arg}[8]{
\draw[-latex] (#7,#8)-- ++(#1:#2) node[right=#5]{\footnotesize$#4$};
\draw[fill=black!#6] (#7+0.3+#3,#8) arc (0:#1:0.3+#3) -- (#7,#8);}
%------------------------

\newcommand{\pinakasdyo}[8]{
\begin{tikzpicture}
\foreach \x in {#6,#7}{
\draw (-3,0) -- (#8,0);
\draw (\x,-.5)--(\x,.0);
\node[fill=white,inner sep=1pt] at (\x,-0.25) {$0$};}
\draw (-.5,0.5) -- (-.5,-0.5);
\node at (-.15,0.25) {$-\infty$};
\node at (#8-.3,0.25) {$+\infty$};
\node at (-1.75,0.25) {$x$};
\node at (-1.75,-0.3) {$#1$};
\node[fill=white,inner sep=1pt] at (#6,0.25) {$#2$};
\node[fill=white,inner sep=1pt] at (#7,0.25) {$#3$};
\node at (0.5*#6-0.25,-0.3) {$#4$};
\node at (0.5*#7+0.5*#6,-0.3) {$#5$};
\node at (0.5*#7+0.5*#8,-0.3) {$#4$};
\end{tikzpicture}}

\newcommand{\pinakasmia}[5]{
\begin{tikzpicture}
\draw (-3,0) -- (#5,0);
\draw (#4,-.5)--(#4,.0);
\node[fill=white,inner sep=1pt] at (#4,-0.25) {$0$};
\draw (-.5,0.5) -- (-.5,-0.5);
\node at (-.15,0.25) {$-\infty$};
\node at (#5-0.3,0.25) {$+\infty$};
\node at (-1.75,0.25) {$x$};
\node at (-1.75,-0.3) {$#1$};
\node[fill=white,inner sep=1pt] at (#4,0.25) {$#2$};
\node at (0.5*#4-0.25,-.3) {$#3$};
\node at (0.5*#4+0.5*#5,-0.3) {$#3$};
\end{tikzpicture}}

\newcommand{\pinakaskamia}[2]{
\begin{tikzpicture}
\draw (-3,0) -- (5,0);
\draw (-.5,0.5) -- (-.5,-0.5);
\node at (-.15,0.25) {$-\infty$};
\node at (4.7,0.25) {$+\infty$};
\node at (-1.75,0.25) {$x$};
\node at (-1.75,-0.3) {$#1$};
\node[fill=white,inner sep=1pt] at (2.25,-0.3) {$#2$};
\end{tikzpicture}}


%----------- ΓΡΑΦΙΚΕΣ ΠΑΡΑΣΤΑΣΕΙΣ ---------
\pgfkeys{/pgfplots/aks_on/.style={axis lines=center,
xlabel style={at={(current axis.right of origin)},xshift=1.5ex, anchor=center},
ylabel style={at={(current axis.above origin)},yshift=1.5ex, anchor=center}}}
\pgfkeys{/pgfplots/grafikh parastash/.style={black,line width=.4mm,samples=200}}
\pgfkeys{/pgfplots/belh ar/.style={tick label style={font=\scriptsize},axis line style={-latex}}}
%------------------------------------------

\newlist{rlist}{enumerate}{3}
\setlist[rlist]{itemsep=0mm,label=\roman*.}
\newlist{tropos}{enumerate}{3}
\setlist[tropos]{label=\bf\textit{\arabic*\textsuperscript{oς}\;Τρόπος :},leftmargin=0cm,itemindent=2.3cm,ref=\bf{\arabic*\textsuperscript{oς}\;Τρόπος}}
% Αν μπει το bhma μεσα σε tropo τότε
%\begin{bhma}[leftmargin=.7cm]



\begin{document}
\titlos{ΜΑΘΗΜΑΤΙΚΑ Γ΄ ΓΥΜΝΑΣΙΟΥ}{Αλγεβρικές Παραστάσεις}{Μονώνυμα}
\orismoi
\Orismos{Αλγεβρική παράσταση - Ακέραια αλγεβρική παράσταση}
Μια μαθηματική παράσταση ονομάζεται αλγεβρική όταν αυτή περιέχει αριθμούς και μεταβλητές με πράξεις μεταξύ τους. \begin{itemize}
\item Μια αλγεβρική παράσταση θα ονομάζεται \textbf{ακέραια} εαν μεταξύ των μεταβλητών της υπάρχουν μόνο οι πράξεις του πολλαπλασιασμού και της πρόσθεσης, ενώ οι εκθέτες είναι \textbf{φυσικοί αριθμοί}.
\item \textbf{Τιμή} μιας αλγεβρικής παράστασης ονομάζεται ο αριθμός που θα προκύψει ύστερα από πράξεις εαν αντικατασταθούν οι μεταβλητές της με αριθμούς.
\end{itemize}
\Orismos{Μονώνυμο}
Μονώνυμο ονομάζεται μια ακέραια αλγεβρική παράσταση εαν μεταξύ των μεταβλητών υπάρχει μόνο η πράξη του πολαπλασιασμού.
\[ \textrm{{\scriptsize Συντελεστής} }\longrightarrow a\cdot \undercbrace{x^{\nu_1}y^{\nu_2}\cdot \ldots\cdot z^{\nu_\kappa}}_{\textrm{κύριο μέρος}}\;\;,\;\;\nu_1,\nu_2,\ldots,\nu_\kappa\in\mathbb{N} \]
\begin{itemize}[itemsep=0mm]
\item Το γινόμενο των μεταβλητών ενός μονωνύμου ονομάζεται \textbf{κύριο μέρος}.
\item  Ο σταθερός αριθμός με τον οποίο πολλαπλασιάζουμε το κύριο μέρος ενός μονωνύμου ονομάζεται \textbf{συντελεστής}.
\end{itemize}
\Orismos{ΒΑΘΜΌΣ ΜΟΝΩΝΎΜΟΥ}
Βαθμός μονωνύμου, ως προς μια μεταβλητή, ονομάζεται ο εκθέτης της μεταβλητής.
\begin{itemize}[itemsep=0mm]
\item Βαθμός ενός μονωνύμου ως προς όλες τις μεταβλητές είναι το άθροισμα των βαθμών κάθε μεταβλητής.
\item Οι πραγματικοί αριθμοί ονομάζονται \textbf{σταθερά} μονώνυμα και είναι μηδενικού βαθμού, ενώ το 0 ονομάζεται \textbf{μηδενικό} μονώνυμο και δεν έχει βαθμό.
\end{itemize}
\begin{center}
\begin{tabular}{c>{\centering\arraybackslash}m{4cm}}
$ a\cdot x^\nu y^\mu $ & \begin{tikzpicture}[box/.style={minimum height=1cm,draw,rounded corners,text width=5cm,align=center}]
\node[box] (b) {{\footnotesize $ \nu $ βαθμού ως προς $ x $\\$ \mu $ βαθμού ως προς $ y $\\$ \nu+\mu $ βαθμού ως προς $ x $ και $ y $}};
\draw[-latex] (b.180) -- (-3,0);
\end{tikzpicture} 
\end{tabular} 
\end{center}
\Orismos{ΌΜΟΙΑ - ΊΣΑ - ΑΝΤΊΘΕΤΑ ΜΟΝΏΝΥΜΑ}
\vspace{-5mm}
\begin{itemize}[itemsep=0mm]
\item Όμοια ονομάζονται τα μονώνυμα που έχουν το ίδιο κύριο μέρος.
\item Ίσα ονομάζονται δύο ή περισσότερα όμοια μονώνυμα που έχουν ίσους συντελεστές.
\item Αντίθετα ονομάζονται δύο όμοια μονώνυμα που έχουν αντίθετους συντελεστές.
\end{itemize}
\begin{center}
\begin{tabular}{c>{\centering\arraybackslash}m{4cm}}
$ a\cdot x^\nu y^\mu\;\;,\;\;\beta\cdot x^\nu y^\mu $ & \begin{tikzpicture}[box/.style={minimum height=.7cm,draw,rounded corners,text width=2.5cm,align=center}]
\node[box] (b) {\footnotesize Όμοια μονώνυμα};
\draw[-latex] (b.180) -- (-2,0);
\end{tikzpicture} 
\\
$ a\cdot x^\nu y^\mu\;\;,\;\;a\cdot x^\nu y^\mu $ & \begin{tikzpicture}[box/.style={minimum height=.7cm,draw,rounded corners,text width=2.5cm,align=center}]
\node[box] (b) {\footnotesize Ίσα μονώνυμα};
\draw[-latex] (b.180) -- (-2,0);
\end{tikzpicture} \\
$ a\cdot x^\nu y^\mu\;\;,\;\;-a\cdot x^\nu y^\mu $ & \begin{tikzpicture}[box/.style={minimum height=.7cm,draw,rounded corners,text width=3cm,align=center}]
\node[box] (b) {\footnotesize Αντίθετα μονώνυμα};
\draw[-latex] (b.180) -- (-2,0);
\end{tikzpicture} 
\end{tabular} 
\end{center}
\thewrhmata
\Thewrhma{Πρόσθεση μονωνύμων}
Το άθροισμα όμοιων μονωνύμων είναι ένα μονώνυμο με κοινό κύριο μέρος και συντελεστή το άθροισμα των συντελεστών τους.\\\\
\Thewrhma{Γινόμενο μονωνύμων}
Το γινόμενο μονωνύμων είναι ένα μονώνυμο με κύριο μέρος το γινόμενο των κύριων μερών τους και συντελεστή το γινόμενο των συντελεστών τους. Ο βαθμός του γινομένου ως προς κάθε μεταβλητή είναι το άθροισμα των αντίστοιχων βαθμών.
\end{document}