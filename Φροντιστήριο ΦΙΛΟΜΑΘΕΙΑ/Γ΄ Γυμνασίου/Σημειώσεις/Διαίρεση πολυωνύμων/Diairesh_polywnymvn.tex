\PassOptionsToPackage{no-math,cm-default}{fontspec}
\documentclass[twoside,nofonts,internet,shmeiwseis]{thewria}
\usepackage{amsmath}
\usepackage{xgreek}
\let\hbar\relax
\defaultfontfeatures{Mapping=tex-text,Scale=MatchLowercase}
\setmainfont[Mapping=tex-text,Numbers=Lining,Scale=1.0,BoldFont={Minion Pro Bold}]{Minion Pro}
\newfontfamily\scfont{GFS Artemisia}
\font\icon = "Webdings"
\usepackage[amsbb]{mtpro2}
\usepackage{tikz,pgfplots}
\tkzSetUpPoint[size=7,fill=white]
\xroma{red!70!black}
%------- ΣΥΣΤΗΜΑ -------------------
\usepackage{systeme,regexpatch}
\makeatletter
% change the definition of \sysdelim not to store `\left` and `\right`
\def\sysdelim#1#2{\def\SYS@delim@left{#1}\def\SYS@delim@right{#2}}
\sysdelim\{. % reinitialize

% patch the internal command to use
% \LEFTRIGHT<left delim><right delim>{<system>}
% instead of \left<left delim<system>\right<right delim>
\regexpatchcmd\SYS@systeme@iii
{\cB.\c{SYS@delim@left}(.*)\c{SYS@delim@right}\cE.}
{\c{SYS@MT@LEFTRIGHT}\cB\{\1\cE\}}
{}{}
\def\SYS@MT@LEFTRIGHT{%
\expandafter\expandafter\expandafter\LEFTRIGHT
\expandafter\SYS@delim@left\SYS@delim@right}
\makeatother
\newcommand{\synt}[2]{{\scriptsize \begin{matrix}
\times#1\\\\ \times#2
\end{matrix}}}
%----------------------------------------
%------ ΜΗΚΟΣ ΓΡΑΜΜΗΣ ΚΛΑΣΜΑΤΟΣ ---------
\DeclareRobustCommand{\frac}[3][0pt]{%
{\begingroup\hspace{#1}#2\hspace{#1}\endgroup\over\hspace{#1}#3\hspace{#1}}}
%----------------------------------------

\newlist{rlist}{enumerate}{3}
\setlist[rlist]{itemsep=0mm,label=\roman*.}
\newlist{brlist}{enumerate}{3}
\setlist[brlist]{itemsep=0mm,label=\bf\roman*.}
\newlist{tropos}{enumerate}{3}
\setlist[tropos]{label=\bf\textit{\arabic*\textsuperscript{oς}\;Τρόπος :},leftmargin=0cm,itemindent=2.3cm,ref=\bf{\arabic*\textsuperscript{oς}\;Τρόπος}}
\newcommand{\tss}[1]{\textsuperscript{#1}}
\newcommand{\tssL}[1]{\MakeLowercase{\textsuperscript{#1}}}

\usepackage{hhline}
%----------- ΓΡΑΦΙΚΕΣ ΠΑΡΑΣΤΑΣΕΙΣ ---------
\pgfkeys{/pgfplots/aks_on/.style={axis lines=center,
xlabel style={at={(current axis.right of origin)},xshift=1.5ex, anchor=center},
ylabel style={at={(current axis.above origin)},yshift=1.5ex, anchor=center}}}
\pgfkeys{/pgfplots/grafikh parastash/.style={\xrwma,line width=.4mm,samples=200}}
\pgfkeys{/pgfplots/belh ar/.style={tick label style={font=\scriptsize},axis line style={-latex}}}
%-----------------------------------------
\usepackage{multicol}
\usepackage{wrap-rl}
\setlist[itemize]{itemsep=0mm}

\begin{document}
\titlos{Μαθηματικά Γ΄ Γυμνασίου}{Αλγεβρικές παραστάσεις}{Ευκλείδεια Διαίρεση Πολυωνύμων}
\orismoi
\Orismos{Ευκλειδεια διαίρεση}
Ευκλείδεια διαίρεση μεταξύ δύο πολυωνύμων $ \varDelta(x) $ και $ \delta(x) $ ονομάζεται η διαδικασία με την οποία διαρώντας τα πολυώνυμα αυτά προκύπτεί μοναδικό ζεύγος πολυωνύμων $ \pi(x) $ και $ \upsilon(x) $ για τα οποία ισχύει
\[ \varDelta(x)=\delta(x)\cdot\pi(x)+\upsilon(x) \]
\begin{itemize}
\item Τα πολυώνυμα $ \varDelta(x),\delta(x),\pi(x),\upsilon(x) $ ονομάζονται \textbf{Διαιρετέος, διαιρέτης, πηλίκο} και \textbf{υπόλοιπο} αντίστοιχα.
\item Η ισότητα $ \varDelta(x)=\delta(x)\cdot\pi(x)+\upsilon(x) $ ονομάζεται \textbf{ισότητα της Ευκλείδειας διαίρεσης}.
\item Αν το υπόλοιπο της διαίρεσης είναι μηδενικό $ (\upsilon(x)=0) $ η διαίρεση ονομάζεται τέλεια και ισχύει :
\[ \varDelta(x)=\delta(x)\cdot\pi(x) \]
Στην τέλεια διαίρεση, τα πολυώνυμα $ \delta(x) $ και $ \pi(x) $ ονομάζονται \textbf{παράγοντες} ή \textbf{διαιρέτες} του $ \varDelta(x) $.
\end{itemize}
\thewrhmata
\Thewrhma{Ευκλείδεια διαίρεση}
Δίνονται τα πολυώνυμα $ \varDelta(x),\delta(x),\pi(x),\upsilon(x) $ τα οποία συνδέονται με τη σχέση :
\[ \varDelta(x)=\delta(x)\cdot\pi(x)+\upsilon(x) \]
\begin{itemize}
\item Η ισότητα αυτή παριστάνει ταυτότητα Ευκλέιδειας διαίρεσης αν και μόνο αν ο βαθμός του υπολοίπου $ \upsilon(x) $ είναι μικρότερος από το βαθμό του διαιρέτη $ \delta(x) $.
\item Ένα πολυώνυμο $ \delta(x) $ είναι παράγοντας ενός πολυωνύμου $ \varDelta(x) $ αν υπάρχει πολυώνυμο $ \pi(x) $ ώστε να ισχύει $ \varDelta(x)=\delta(x)\cdot\pi(x) $.
\end{itemize}

\end{document}
