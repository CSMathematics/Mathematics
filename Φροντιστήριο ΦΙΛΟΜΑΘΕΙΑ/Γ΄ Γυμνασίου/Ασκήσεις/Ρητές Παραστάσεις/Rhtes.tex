\documentclass{askhseis}
\usepackage[amsbb,zswash]{mtpro2}
\usepackage[no-math,cm-default]{fontspec}
\usepackage{xunicode}
\usepackage{xltxtra}
\usepackage{xgreek}
\defaultfontfeatures{Mapping=tex-text,Scale=MatchLowercase}
\setmainfont[Mapping=tex-text,Numbers=Lining,Scale=1.0,BoldFont={Minion Pro Bold}]{Minion Pro}
\newfontfamily\mpro{Minion Pro}
\usepackage[amsbb,zswash]{mtpro2}
\usepackage[left=2.00cm, right=2.00cm, top=3.00cm, bottom=3.00cm]{geometry}
%------TIKZ - ΣΧΗΜΑΤΑ - ΓΡΑΦΙΚΕΣ ΠΑΡΑΣΤΑΣΕΙΣ ----
\usepackage{tikz}
\usepackage{tkz-euclide}
\usetkzobj{all}
\usetkzobj{all}
%-----------------------
\usepackage{aniswseis}
%-----ΕΙΚΟΝΑ ΔΙΠΛΑ ΑΠΟ ΚΕΙΜΕΝΟ-------
\usepackage{wrapfig}
\newenvironment{WrapText1}[3][r]
{\wrapfigure[#2]{#1}{#3}}
{\endwrapfigure}

\newenvironment{WrapText2}[3][l]
{\wrapfigure[#2]{#1}{#3}}
{\endwrapfigure}

\newcommand{\wrapr}[6]{
\begin{minipage}{\linewidth}\mbox{}\\
\vspace{#1}
\begin{WrapText1}{#2}{#3}
\vspace{#4}#5\end{WrapText1}#6
\end{minipage}}

\newcommand{\wrapl}[6]{
\begin{minipage}{\linewidth}\mbox{}\\
\vspace{#1}
\begin{WrapText2}{#2}{#3}
\vspace{#4}#5\end{WrapText2}#6
\end{minipage}}
%-------------------------------------------
\usepackage{calc}
\usepackage{hhline}
\usepackage{graphicx}
\usepackage{multicol}
\usepackage{multirow}
\usepackage{tabularx}
\usepackage[decimalsymbol=comma]{siunitx}
\usetikzlibrary{backgrounds}
\usepackage{enumitem}
\setlist[enumerate]{label=\bf{\large \arabic*.}}
%------- ΣΥΣΤΗΜΑ -------------------
\usepackage{systeme,regexpatch}
\makeatletter
% change the definition of \sysdelim not to store `\left` and `\right`
\def\sysdelim#1#2{\def\SYS@delim@left{#1}\def\SYS@delim@right{#2}}
\sysdelim\{. % reinitialize

% patch the internal command to use
% \LEFTRIGHT<left delim><right delim>{<system>}
% instead of \left<left delim<system>\right<right delim>
\regexpatchcmd\SYS@systeme@iii
{\cB.\c{SYS@delim@left}(.*)\c{SYS@delim@right}\cE.}
{\c{SYS@MT@LEFTRIGHT}\cB\{\1\cE\}}
{}{}
\def\SYS@MT@LEFTRIGHT{%
\expandafter\expandafter\expandafter\LEFTRIGHT
\expandafter\SYS@delim@left\SYS@delim@right}
\makeatother
\newcommand{\synt}[2]{{\scriptsize \begin{matrix}
\times#1\\\\ \times#2
\end{matrix}}}
%----------------------------------------
%------ ΜΗΚΟΣ ΓΡΑΜΜΗΣ ΚΛΑΣΜΑΤΟΣ ---------
\DeclareRobustCommand{\frac}[3][0pt]{%
{\begingroup\hspace{#1}#2\hspace{#1}\endgroup\over\hspace{#1}#3\hspace{#1}}}
%----------------------------------------
%-------- ΜΑΘΗΜΑΤΙΚΑ ΕΡΓΑΛΕΙΑ ---------
\usepackage{mathtools}
%----------------------
%------ ΔΙΑΓΩΝΙΟ ΣΕ ΠΙΝΑΚΑ -------
\usepackage{array}
\newcommand\diag[5]{%
\multicolumn{1}{|m{#2}|}{\hskip-\tabcolsep
$\vcenter{\begin{tikzpicture}[baseline=0,anchor=south west,outer sep=0]
\path[use as bounding box] (0,0) rectangle (#2+2\tabcolsep,\baselineskip);
\node[minimum width={#2+2\tabcolsep-\pgflinewidth},
minimum  height=\baselineskip+#3-\pgflinewidth] (box) {};
\draw[line cap=round] (box.north west) -- (box.south east);
\node[anchor=south west,align=left,inner sep=#1] at (box.south west) {#4};
\node[anchor=north east,align=right,inner sep=#1] at (box.north east) {#5};
\end{tikzpicture}}\rule{0pt}{.71\baselineskip+#3-\pgflinewidth}$\hskip-\tabcolsep}}
%---------------------------------

%---- ΟΡΙΖΟΝΤΙΟ - ΚΑΤΑΚΟΡΥΦΟ - ΠΛΑΓΙΟ ΑΓΚΙΣΤΡΟ ------
\newcommand{\orag}[3]{\node at (#1)
{$ \overcbrace{\rule{#2mm}{0mm}}^{{\scriptsize #3}} $};}

\newcommand{\kag}[3]{\node at (#1)
{$ \undercbrace{\rule{#2mm}{0mm}}_{{\scriptsize #3}} $};}

\newcommand{\Pag}[4]{\node[rotate=#1] at (#2)
{$ \overcbrace{\rule{#3mm}{0mm}}^{{\rotatebox{-#1}{\scriptsize$#4$}}}$};}
%-----------------------------------------

%-------- ΤΡΙΓΩΝΟΜΕΤΡΙΚΟΙ ΑΡΙΘΜΟΙ -----------
\newcommand{\hm}[1]{\textrm{ημ}#1}
\newcommand{\syn}[1]{\textrm{συν}#1}
\newcommand{\ef}[1]{\textrm{εφ}#1}
\newcommand{\syf}[1]{\textrm{σφ}#1}
%--------------------------------------------

%--------- ΠΟΣΟΣΤΟ ΤΟΙΣ ΧΙΛΙΟΙΣ ------------
\DeclareRobustCommand{\perthousand}{%
\ifmmode
\text{\textperthousand}%
\else
\textperthousand
\fi}
%------------------------------------------

%------------------------------------------
\usepackage{extarrows}
\newcommand{\eq}[1]{\xlongequal{#1}}
\newcommand{\eqq}[2]{\xlongequal[#2]{#1}}
\DeclareMathOperator*{\Eq}{=}
%------------------------------------------
\newcommand{\tss}[1]{\textsuperscript{#1}}
\newcommand{\tssL}[1]{\MakeLowercase{\textsuperscript{#1}}}
%---------- ΛΙΣΤΕΣ ----------------------
\newlist{rlist}{enumerate}{3}
\setlist[rlist]{itemsep=0mm,label=\roman*.}
\tkzSetUpPoint[size=7,fill=white]
\tikzstyle{pl}=[line width=0.3mm]
\tikzstyle{plm}=[line width=0.4mm]

\begin{document}
\titlos{ΑΛΓΕΒΡΙΚΕΣ ΠΑΡΑΣΤΑΣΕΙΣ}{ΡΗΤΕΣ ΠΑΡΑΣΤΑΣΕΙΣ}
\paragraph{ΘΕΩΡΙΑ}
\begin{enumerate}
\item \textbf{Ερωτήσεις θεωρίας}\\
Να απαντήσετε στις παρακάτω ερωτήσεις.
\begin{rlist}
\item Ποια αλγεβρική παράσταση ονομάζεται ρητή;
\item Ποια συνθήκη πρέπει να ισχύει ώστε να ορίζεται μια ρητή αλγεβρική παράσταση;
\item Πότε μπορεί να απλοποιηθεί μια ρητή αλγεβρική παράσταση;
\item Οι περιορισμοί που αφορούν μια ρητή αλγεβρική παράσταση είναι περισσότεροι ή ίδιοι μ' αυτούς που αφορούν την ίδια παράσταση απλοποιημένη;
\end{rlist}
\item \textbf{Σωστό - Λάθος}\\
Να χαρακτηρίσετε τις παρακάτω προτάσεις ως σωστές (Σ) ή λανθασμένες (Λ).
\begin{rlist}
\item Μια ρητή αλγεβρική παράσταση ορίζεται για κάθε τιμή των μεταβλητών που περιέχει.
\item Για να απλοποιηθεί μια ρητή αλγεβρική παράσταση θα πρέπει και οι δύο όροι της να αποτελούν γινόμενο παραγόντων.
\end{rlist}
\end{enumerate}
\paragraph{ΑΣΚΗΣΕΙΣ}
\begin{center}
\textbf{ΠΕΡΙΟΡΙΣΜΟΙ}
\end{center}
\begin{enumerate}
\item \textbf{Περιορισμοί}\\
Να βρεθούν οι τιμές των μεταβλητών για τις οποίες ορίζεται κάθε μια από τις παρακάτω παραστάσεις.
\begin{multicols}{3}
\begin{rlist}
\item $ \dfrac{1}{x} $
\item $ \dfrac{3}{x-1} $
\item $ \dfrac{y}{2y-4} $
\item $ \dfrac{z-3}{4-z} $
\item $ \dfrac{2x-1}{3x-6} $
\item $ \dfrac{4}{x-y} $
\end{rlist}
\end{multicols}
\item \textbf{Περιορισμοί}\\
Να βρεθούν οι τιμές των μεταβλητών για τις οποίες ορίζεται κάθε μια από τις παρακάτω παραστάσεις.
\begin{multicols}{2}
\begin{rlist}
\item $ \dfrac{3}{x^2} $
\item $ \dfrac{3x}{x(x-3)} $
\item $ \dfrac{7}{x(x+1)} $
\item $ \dfrac{x-3}{(x-2)(x+1)} $
\end{rlist}
\end{multicols}
\item \textbf{Περιορισμοί}\\
Να βρεθούν οι τιμές των μεταβλητών για τις οποίες ορίζεται κάθε μια από τις παρακάτω παραστάσεις.
\begin{multicols}{2}
\begin{rlist}
\item $ \dfrac{1}{x^2-x} $
\item $ \dfrac{2}{x^3-4x^2} $
\item $ \dfrac{3-x}{2x-x^2} $
\item $ \dfrac{x+y}{xy-y} $
\end{rlist}
\end{multicols}
\item \textbf{Περιορισμοί}\\
Να βρεθούν οι τιμές των μεταβλητών για τις οποίες ορίζεται κάθε μια από τις παρακάτω παραστάσεις.
\begin{multicols}{3}
\begin{rlist}[leftmargin=5mm]
\item $ \dfrac{x}{x^2-4} $
\item $ \dfrac{5}{4y^2-25} $
\item $ \dfrac{3-2x}{3x^2-18} $
\item $ \dfrac{3x+4}{x^3-16x} $
\item $ \dfrac{y-2}{y^2-x^2} $
\item $ \dfrac{z^2}{z^4-1} $
\end{rlist}
\end{multicols}
\item \textbf{Περιορισμοί}\\
Να βρεθούν οι τιμές των μεταβλητών για τις οποίες ορίζεται κάθε μια από τις παρακάτω παραστάσεις.
\begin{multicols}{2}
\begin{rlist}[leftmargin=5mm]
\item $ \dfrac{x+1}{x^2-4x+4} $
\item $ \dfrac{2}{x^2+2x+1} $
\item $ \dfrac{4+y}{25y^2+10y+1} $
\item $ \dfrac{z-5}{z^3-6z^2+9z} $
\end{rlist}
\end{multicols}
\begin{center}
\textbf{ΑΠΛΟΠΟΙΗΣΗ}
\end{center}
\item \textbf{Απλοποίηση}\\
Να απλοποιηθούν οι παρακάτω ρητές αλγεβρικές παραστάσεις.
\begin{multicols}{3}
\begin{rlist}
\item $ \dfrac{3x}{4x^2} $
\item $ \dfrac{2x}{8xy} $
\item $ \dfrac{12x^2}{4x^3y^2} $
\item $ \dfrac{5xy}{25yz} $
\item $ \dfrac{20x^2z^2}{10xz^3} $
\item $ \dfrac{8y^3z}{16yz^3} $
\end{rlist}
\end{multicols}
\item \textbf{Απλοποίηση}\\
Να απλοποιηθούν οι παρακάτω ρητές αλγεβρικές παραστάσεις.
\begin{multicols}{2}
\begin{rlist}
\item $ \dfrac{x-1}{x(x-1)^2} $
\item $ \dfrac{3x-6}{x^2-2x} $
\item $ \dfrac{4-2y}{(2-y)^2} $
\item $ \dfrac{2z+1}{4z^2-2z} $
\end{rlist}
\end{multicols}
\item \textbf{Απλοποίηση}\\
Να απλοποιηθούν οι παρακάτω ρητές αλγεβρικές παραστάσεις αφού γραφτούν οι απαραίτητοι περιορισμοί.
\begin{multicols}{2}
\begin{rlist}
\item $ \dfrac{x-3}{3x-9} $
\item $ \dfrac{2-x}{4-x^2} $
\item $ \dfrac{y^2}{y^3-y} $
\item $ \dfrac{4z^2+3z}{16z^3-9z} $
\end{rlist}
\end{multicols}
\item \textbf{Απλοποίηση}\\
Να απλοποιηθούν οι παρακάτω ρητές αλγεβρικές παραστάσεις.
\begin{multicols}{2}
\begin{rlist}
\item $ \dfrac{2x+4}{x^2+4x+4} $
\item $ \dfrac{y^2-25}{y^3-5y^2} $
\item $ \dfrac{4z^2-9}{4z^2-12z+9} $
\item $ \dfrac{t^2+10t+25}{2t^2-50} $
\end{rlist}
\end{multicols}
\item \textbf{Απλοποίηση}\\
Να απλοποιηθούν οι παρακάτω ρητές αλγεβρικές παραστάσεις.
\begin{multicols}{2}
\begin{rlist}
\item $ \dfrac{(x+y)^2}{x^2-y^2} $
\item $ \dfrac{4z-2y}{3y^2-6yz} $
\item $ \dfrac{xy^2-x^2y}{x^2-2xy+y^2} $
\item $ \dfrac{z^2-9x^2}{2zx-6x^2} $
\end{rlist}
\end{multicols}
\item \textbf{Απλοποίηση}\\
Να απλοποιηθούν οι παρακάτω ρητές αλγεβρικές παραστάσεις.
\begin{multicols}{2}
\begin{rlist}
\item $ \dfrac{x^2-3x+2}{(x-1)(x-2)} $
\item $ \dfrac{z^2-4z}{z^2-z-12} $
\item $ \dfrac{x^2-10x+25}{x^2-3x-10} $
\item $ \dfrac{y^2-5y-14}{y^2-49} $
\end{rlist}
\end{multicols}
\begin{center}
\textbf{ΣΥΝΘΕΤΕΣ ΑΣΚΗΣΕΙΣ}
\end{center}
\item \textbf{Πολυώνυμα - Απλοποίηση}\\
Δίνονται τα παρακάτω πολυώνυμα\\
$ \begin{aligned}
& A(x)=2x-6\\& B(x)=x^2-9\\& \varGamma(x)=x^2-7x+12
\end{aligned} $\\
Με τη βοήθεια των παραπάνω πολυωνύμων να σχηματιστούν και στη συνέχεια να απλοποιηθούν οι ακόλουθες παράστασεις :
\begin{multicols}{3}
\begin{rlist}
\item $ \dfrac{A(x)}{B(x)} $
\item $ \dfrac{B(x)}{\varGamma(x)} $
\item $ \dfrac{A(x)}{B(x)} $
\end{rlist}
\end{multicols}
\item \textbf{Ρητές - Πολυώνυμα - Τιμές - Απλοποίηση}\\
Δίνεται η ρητή παράσταση $ P(x)=\frac{A(x)}{B(x)} $ όπου οι όροι της παράστασης $ A(x), B(x) $ δίνονται από τους τύπους : \begin{gather*}
 A(x)=4-x\\ B(x)=x^2-16
\end{gather*}
\begin{rlist}
\item Να βρεθούν οι τιμές $ P(2), P(-1) $ και $ P(3) $ της παράστασης $ P(x) $.
\item Εξετάστε αν μπορεί να υπολογιστεί η τιμή $ P(4) $ της παράστασης.
\item Απλοποιήστε την παράσταση $ P(x) $ και στη συνέχεια προσπαθήστε ξανά να εξετάσετε αν υπολογίζεται η τιμή $ P(4) $. Αιτιολογίστε την απάντησή σας.
\end{rlist}
\item \textbf{Πολυώνυμα - Περιορισμοί - Απλοποίηση}\\
Με τη βοήθεια των παρακάτω πολυωνύμων\\
$ \begin{aligned}
& A(x)=4x-8\\& B(x)=x^2-4\\& \varGamma(x)=x^2-5x+6
\end{aligned} $\\
να σχηματιστούν οι ακόλουθες παράστασεις. Υπολογίστε τις τιμές της μεταβλητής για τις οποίες ορίζεται κάθε παράσταση και ύστερα απλοποιήστε τες.
\begin{multicols}{2}
\begin{rlist}
\item $ \dfrac{B(x)-A(x)}{B(x)} $
\item $ \dfrac{B(x)-\varGamma(x)}{A(x)} $
\item $ \dfrac{A(x)+\varGamma(x)}{B(x)} $
\item $ \dfrac{A(2x)}{B(-2x)} $
\end{rlist}
\end{multicols}
\item \textbf{}
\end{enumerate}
\end{document}
