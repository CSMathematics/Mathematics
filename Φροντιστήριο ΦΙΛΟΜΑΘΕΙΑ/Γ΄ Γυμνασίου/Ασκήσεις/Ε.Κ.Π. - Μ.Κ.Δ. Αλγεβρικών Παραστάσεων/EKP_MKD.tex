\PassOptionsToPackage{no-math,cm-default}{fontspec}
\documentclass[twoside,nofonts,internet]{askhseis}
\usepackage{amsmath}
\usepackage{xgreek}
\let\hbar\relax
\defaultfontfeatures{Mapping=tex-text,Scale=MatchLowercase}
\setmainfont[Mapping=tex-text,Numbers=Lining,Scale=1.0,BoldFont={Minion Pro Bold}]{Minion Pro}
\newfontfamily\scfont{GFS Artemisia}
\font\icon = "Webdings"
\usepackage[amsbb]{mtpro2}
\usepackage{tikz,pgfplots}
\tkzSetUpPoint[size=7,fill=white]
\xroma{red!70!black}
%------TIKZ - ΣΧΗΜΑΤΑ - ΓΡΑΦΙΚΕΣ ΠΑΡΑΣΤΑΣΕΙΣ ----
\usepackage{tikz}
\usepackage{tkz-euclide}
\usetkzobj{all}
\usepackage[framemethod=TikZ]{mdframed}
\usetikzlibrary{decorations.pathreplacing}
\usepackage{pgfplots}
\usetkzobj{all}
%-----------------------

%-----ΕΙΚΟΝΑ ΔΙΠΛΑ ΑΠΟ ΚΕΙΜΕΝΟ-------
\usepackage{wrapfig}
\newenvironment{WrapText1}[3][r]
{\wrapfigure[#2]{#1}{#3}}
{\endwrapfigure}

\newenvironment{WrapText2}[3][l]
{\wrapfigure[#2]{#1}{#3}}
{\endwrapfigure}

\newcommand{\wrapr}[6]{
\begin{minipage}{\linewidth}\mbox{}\\
\vspace{#1}
\begin{WrapText1}{#2}{#3}
\vspace{#4}#5\end{WrapText1}#6
\end{minipage}}

\newcommand{\wrapl}[6]{
\begin{minipage}{\linewidth}\mbox{}\\
\vspace{#1}
\begin{WrapText2}{#2}{#3}
\vspace{#4}#5\end{WrapText2}#6
\end{minipage}}
%-------------------------------------------

\usepackage{calc}
\usepackage{hhline}
\renewcommand{\thepart}{\arabic{part}}

\usepackage[explicit]{titlesec}
\usepackage{graphicx}
\usepackage{multicol}
\usepackage{multirow}
\usepackage{enumitem}
\usepackage{tabularx}
\usepackage[decimalsymbol=comma]{siunitx}
\usetikzlibrary{backgrounds}
\usepackage{sectsty}
\sectionfont{\centering}
\usepackage{enumitem}
\setlist[enumerate]{label=\bf{\large \textcolor{\xrwma}{\arabic*.}}}
\usepackage{adjustbox}
%--------- ΑΓΓΛΙΚΟ ΚΕΙΜΕΝΟ --------------
\newcommand{\eng}[1]{\selectlanguage{english}#1\selectlanguage{greek}}
%----------------------------------------
%------- ΣΥΣΤΗΜΑ -------------------
\usepackage{systeme,regexpatch}
\makeatletter
% change the definition of \sysdelim not to store `\left` and `\right`
\def\sysdelim#1#2{\def\SYS@delim@left{#1}\def\SYS@delim@right{#2}}
\sysdelim\{. % reinitialize

% patch the internal command to use
% \LEFTRIGHT<left delim><right delim>{<system>}
% instead of \left<left delim<system>\right<right delim>
\regexpatchcmd\SYS@systeme@iii
{\cB.\c{SYS@delim@left}(.*)\c{SYS@delim@right}\cE.}
{\c{SYS@MT@LEFTRIGHT}\cB\{\1\cE\}}
{}{}
\def\SYS@MT@LEFTRIGHT{%
\expandafter\expandafter\expandafter\LEFTRIGHT
\expandafter\SYS@delim@left\SYS@delim@right}
\makeatother
\newcommand{\synt}[2]{{\scriptsize \begin{matrix}
\times#1\\\\ \times#2
\end{matrix}}}
%----------------------------------------
%------ ΜΗΚΟΣ ΓΡΑΜΜΗΣ ΚΛΑΣΜΑΤΟΣ ---------
\DeclareRobustCommand{\frac}[3][0pt]{%
{\begingroup\hspace{#1}#2\hspace{#1}\endgroup\over\hspace{#1}#3\hspace{#1}}}
%----------------------------------------
%-------- ΜΑΘΗΜΑΤΙΚΑ ΕΡΓΑΛΕΙΑ ---------
\usepackage{mathtools}
%----------------------

%-------- ΠΙΝΑΚΕΣ ---------
\usepackage{booktabs}
%----------------------
%----- ΥΠΟΛΟΓΙΣΤΗΣ ----------
\usepackage{calculator}
%----------------------------
%------ ΔΙΑΓΩΝΙΟ ΣΕ ΠΙΝΑΚΑ -------
\usepackage{array}
\newcommand\diag[5]{%
\multicolumn{1}{|m{#2}|}{\hskip-\tabcolsep
$\vcenter{\begin{tikzpicture}[baseline=0,anchor=south west,outer sep=0]
\path[use as bounding box] (0,0) rectangle (#2+2\tabcolsep,\baselineskip);
\node[minimum width={#2+2\tabcolsep-\pgflinewidth},
minimum  height=\baselineskip+#3-\pgflinewidth] (box) {};
\draw[line cap=round] (box.north west) -- (box.south east);
\node[anchor=south west,align=left,inner sep=#1] at (box.south west) {#4};
\node[anchor=north east,align=right,inner sep=#1] at (box.north east) {#5};
\end{tikzpicture}}\rule{0pt}{.71\baselineskip+#3-\pgflinewidth}$\hskip-\tabcolsep}}
%---------------------------------

%---- ΟΡΙΖΟΝΤΙΟ - ΚΑΤΑΚΟΡΥΦΟ - ΠΛΑΓΙΟ ΑΓΚΙΣΤΡΟ ------
\newcommand{\orag}[3]{\node at (#1)
{$ \overcbrace{\rule{#2mm}{0mm}}^{{\scriptsize #3}} $};}

\newcommand{\kag}[3]{\node at (#1)
{$ \undercbrace{\rule{#2mm}{0mm}}_{{\scriptsize #3}} $};}

\newcommand{\Pag}[4]{\node[rotate=#1] at (#2)
{$ \overcbrace{\rule{#3mm}{0mm}}^{{\rotatebox{-#1}{\scriptsize$#4$}}}$};}
%-----------------------------------------

%-------- ΤΡΙΓΩΝΟΜΕΤΡΙΚΟΙ ΑΡΙΘΜΟΙ -----------
\newcommand{\hm}[1]{\textrm{ημ}#1}
\newcommand{\syn}[1]{\textrm{συν}#1}
\newcommand{\ef}[1]{\textrm{εφ}#1}
\newcommand{\syf}[1]{\textrm{σφ}#1}
%--------------------------------------------

%--------- ΠΟΣΟΣΤΟ ΤΟΙΣ ΧΙΛΙΟΙΣ ------------
\DeclareRobustCommand{\perthousand}{%
\ifmmode
\text{\textperthousand}%
\else
\textperthousand
\fi}
%------------------------------------------

%------------------------------------------
\usepackage{extarrows}
\newcommand{\eq}[1]{\xlongequal{#1}}
\newcommand{\eqq}[2]{\xlongequal[#2]{#1}}
\DeclareMathOperator*{\Eq}{=}
%------------------------------------------
%------ ΌΡΙΣΜΑ ----------
\newcommand{\Arg}[8]{
\draw[-latex] (#7,#8)-- ++(#1:#2) node[right=#5]{\footnotesize$#4$};
\draw[fill=black!#6] (#7+0.3+#3,#8) arc (0:#1:0.3+#3) -- (#7,#8);}
%------------------------



%--------- ΑΝΙΣΩΣΕΙΣ -------
\tikzset{
thickest/.style={line width=1mm,steelblue},
a/.style={decoration={markings,
mark=at position #1 with {\fill[white,draw=black,thin] circle (3pt);}},postaction={decorate}},
k/.style={decoration={markings,
mark=at position #1 with {\fill[black] circle (3pt);}},postaction={decorate}},
}
%--------- ΔΙΑΣΤΗΜΑ ------------
\newcommand{\diasthma}[7]{
\foreach \x in {#3,#4}
\draw (\x,#7+.2) -- (\x,#7-.2);
\node[anchor=north,fill=white] at (#3,#7)[below=1mm] {$ #1 $};
\node[anchor=north,fill=white] at (#4,#7)[below=1mm] {$ #2 $};
\draw [#5=0,#6=1,thickest] (#3,#7)--(#4,#7);
}
%--------- ΑΞΟΝΑΣ ------------------
\newcommand{\axonas}[3]{
\draw[-latex] (#1,#3) -- (#2,#3)node[below]{$x$};
}
%--------- ΚΑΤΩ ΑΚΡΟ ------------------
\newcommand{\Xapeiro}[5]{
\draw (#2,#5+.2) -- (#2,#5-.2);
\node[anchor=north,fill=white] at (#2,#5)[below=1mm] {$ #1 $};
\draw [#4=0,thickest] (#2,#5)--(#3-.3,#5);
}
%--------- ΠΑΝΩ ΑΚΡΟ ------------------
\newcommand{\apeiroX}[5]{
\draw (#2,#5+.2) -- (#2,#5-.2);
\node[anchor=north,fill=white] at (#2,#5)[below=1mm] {$ #1 $};
\draw [#4=0,thickest] (#2,#5)--(#3+.3,#5);
}
%----- ΔΙΑΚΕΚΟΜΜΕΝΕΣ ΓΡΑΜΜΕΣ ------
\newcommand{\oria}[3]{\draw [dashed] (#1,#2)--(#1,#3);}
%--------------------------------------
\newcommand{\tss}[1]{\textsuperscript{#1}}
\newcommand{\tssL}[1]{\MakeLowercase{\textsuperscript{#1}}}
%---------- ΛΙΣΤΕΣ ----------------------
\newlist{bhma}{enumerate}{3}
\setlist[bhma]{label=\bf\textit{\arabic*\textsuperscript{o}\;Βήμα :},leftmargin=0cm,itemindent=1.8cm,ref=\bf{\arabic*\textsuperscript{o}\;Βήμα}}
\newlist{tropos}{enumerate}{3}
\setlist[tropos]{label=\bf\textit{\arabic*\textsuperscript{oς}\;Τρόπος :},leftmargin=0cm,itemindent=2.3cm,ref=\bf{\arabic*\textsuperscript{oς}\;Τρόπος}}
% Αν μπει το bhma μεσα σε tropo τότε
%\begin{bhma}[leftmargin=.7cm]
\tkzSetUpPoint[size=7,fill=white]
\tikzstyle{pl}=[line width=0.3mm]
\tikzstyle{plm}=[line width=0.4mm]




\begin{document}
\titlos{Μαθηματικά Γ΄ Γυμνασίου}{Αλγεβρικές Παραστάσεις}{Ε.Κ.Π. - Μ.Κ.Δ. Αλγεβρικών Παραστάσεων}
\thewria
\begin{multicols}{2}
\begin{enumerate}
\item \textbf{Ερωτήσεις Θεωρίας}\\
Να απαντήσετε στις παρακάτω ερωτήσεις.
\vspace{-2mm}
\begin{rlist}
\item Τι ονομάζουμε Ε.Κ.Π. δύο ή περισσότερων αλγεβρικών παραστάσεων;
\item Τι ονομάζουμε Μ.Κ.Δ. δύο ή περισσότερων αλγεβρικών παραστάσεων;
\item Με ποιόν κανόνα υπολογίζουμε Ε.Κ.Π. δύο ή περισσότερων αλγεβρικών παραστάσεων;
\item Με ποιόν κανόνα υπολογίζουμε Μ.Κ.Δ. δύο ή περισσότερων αλγεβρικών παραστάσεων;
\end{rlist}
\item \textbf{Πολλαπλής Επιλογής}\\
Να επιλέξετε το πολυώνυμο το οποίο αποτελεί το Ε.Κ.Π. των αλγεβρικών παραστάσεων $ 4x^2(x-1)^3 $ και $ 8x^3(x-1) $
\begin{multicols}{2}
\begin{rlist}
\item $ 4x^3(x-1) $
\item $ 8x^2(x-1)^2 $
\item $ 8x^3(x-1)^2 $
\item $ 4x^2(x-1) $
\end{rlist}
\end{multicols}
\item \textbf{Πολλαπλής Επιλογής}\\
Να επιλέξετε το πολυώνυμο το οποίο αποτελεί το Μ.Κ.Δ. των αλγεβρικών παραστάσεων $ 5x^2(x-2)^3 $ και $ 4x(x-2)^2 $
\begin{multicols}{2}
\begin{rlist}
\item $ 4x^3(x-2) $
\item $ x(x-2)^2 $
\item $ 5x^3(x-2)^2 $
\item $ 20x^2(x-2)^3 $
\end{rlist}
\end{multicols}
\end{enumerate}
\end{multicols}
\askhseis
\begin{multicols*}{2}
\begin{enumerate}
\item \textbf{Ε.Κ.Π. - Μ.Κ.Δ. Μονωνύμων}\\
Να υπολογιστεί το Ε.Κ.Π. και ο Μ.Κ.Δ. των παρακάτω μονωνύμων.
\begin{rlist}
\item $ 3x^2\ ,\ 9x^3\ ,\ 6x^4 $
\item $ 2x^4y^2\ ,\ 8x^3y^2z^5\ ,\ 4y^3z^4 $
\item $ 12x^4y^2\ ,\ 36yz^4\ ,\ 24x^5z^2 $
\end{rlist}
\item \textbf{Ε.Κ.Π. - Μ.Κ.Δ. Μονωνύμων}\\
Να υπολογιστεί το Ε.Κ.Π. και ο Μ.Κ.Δ. των παρακάτω μονωνύμων.
\begin{rlist}
\item $ 30x^4y^5\ ,\ 45y^5z^7\ ,\ 50x^3z^4 $
\item $ 200w^5z^7\ ,\ 250z^2v^4\ ,\ 150w^7 $
\item $ 180a^2r^4\ ,\ 210r^3s^5\ ,\ 150a^5s^3 $
\end{rlist}
\item \textbf{Ε.Κ.Π. - Μ.Κ.Δ. Πολυωνύμων}\\
Να υπολογιστεί το Ε.Κ.Π. και ο Μ.Κ.Δ. των παρακάτω πολυωνύμων.
\begin{rlist}
\item $ 2x(x-2)\ ,\ (x-2)^2\ ,\ 3x(x-2) $
\item $ 4x^2(x+2)\ ,\ 3x(x+2)^3\ ,\ 6x^4 $
\item $ 5(x-4)(x+4)^2\ ,\ 3(x+4)^4\ ,\ 4(2x+8) $
\end{rlist}
\item \textbf{Ε.Κ.Π. - Μ.Κ.Δ. Πολυωνύμων}\\
Να υπολογιστεί το Ε.Κ.Π. και ο Μ.Κ.Δ. των παρακάτω πολυωνύμων.
\begin{rlist}
\item $ x^2+x\ ,\ x^2-1\ ,\ x^3-x $
\item $ 3x-6\ ,\ 4-2x\ ,\ x^2-2x $
\item $ x^2-9\ ,\ 3x+9\ ,\ 4x^2+12x $
\end{rlist}
\item \textbf{Ε.Κ.Π. - Μ.Κ.Δ. Πολυωνύμων}\\
Να υπολογιστεί το Ε.Κ.Π. και ο Μ.Κ.Δ. των παρακάτω πολυωνύμων.
\begin{rlist}
\item $ x^2-2x+1\ ,\ 3x^2-9x\ ,\ 9-x^2 $
\item $ 2x^2-8x-x+4\ ,\ 4x^2-4x+1\ ,\ x^2-4x $
\item $ 3x^3-12x\ ,\ x^3-4x^2+4x\ ,\ 12x-24 $
\end{rlist}
\item \textbf{Ε.Κ.Π. - Μ.Κ.Δ. Πολυωνύμων}\\
Να υπολογιστεί το Ε.Κ.Π. και ο Μ.Κ.Δ. των παρακάτω πολυωνύμων.
\begin{rlist}
\item $ y^3-2y^2\ ,\ y^2-3y+2\ ,\ y^2-4y+4 $
\item $ 5z^2-125 \ ,\ z^2-10z+25\ ,\ z^2-7z+10 $
\item $ 2r^3-18r\ ,\ 4r^2+4r-8\ ,\ 27-3r^2 $
\end{rlist}
\item \textbf{Ε.Κ.Π. Πολυωνύμων}\\
Να υπολογιστεί το Ε.Κ.Π και ο Μ.Κ.Δ. των παρακάτω πολυωνύμων.
\begin{rlist}
\item $ 20x^4-320\ ,\ 25x^3-100x^2+100x\ ,\\ 40x^5+160x^3 $
\item $ 45-5x^2\ ,\ 50x^3-150x^2+200x-600 \ ,\\30x-90 $
\end{rlist}
\end{enumerate}
\end{multicols*}
\end{document}
