\PassOptionsToPackage{no-math,cm-default}{fontspec}
\documentclass[twoside,nofonts,internet,methodoi]{thewria}
\usepackage{amsmath}
\usepackage{xgreek}
\let\hbar\relax
\defaultfontfeatures{Mapping=tex-text,Scale=MatchLowercase}
\setmainfont[Mapping=tex-text,Numbers=Lining,Scale=1.0,BoldFont={Minion Pro Bold}]{Minion Pro}
\newfontfamily\scfont{GFS Artemisia}
\font\icon = "Webdings"
\usepackage[amsbb]{mtpro2}
\usepackage{tikz,pgfplots}
\tkzSetUpPoint[size=7,fill=white]
\xroma{blue!50!green}
%------- ΣΥΣΤΗΜΑ -------------------
\usepackage{systeme,regexpatch}
\makeatletter
% change the definition of \sysdelim not to store `\left` and `\right`
\def\sysdelim#1#2{\def\SYS@delim@left{#1}\def\SYS@delim@right{#2}}
\sysdelim\{. % reinitialize
% patch the internal command to use
% \LEFTRIGHT<left delim><right delim>{<system>}
% instead of \left<left delim<system>\right<right delim>
\regexpatchcmd\SYS@systeme@iii
{\cB.\c{SYS@delim@left}(.*)\c{SYS@delim@right}\cE.}
{\c{SYS@MT@LEFTRIGHT}\cB\{\1\cE\}}
{}{}
\def\SYS@MT@LEFTRIGHT{%
\expandafter\expandafter\expandafter\LEFTRIGHT
\expandafter\SYS@delim@left\SYS@delim@right}
\makeatother
\newcommand{\synt}[2]{{\scriptsize \begin{matrix}
\times#1\\\\ \times#2
\end{matrix}}}
%----------------------------------------
%------ ΜΗΚΟΣ ΓΡΑΜΜΗΣ ΚΛΑΣΜΑΤΟΣ ---------
\DeclareRobustCommand{\frac}[3][0pt]{%
{\begingroup\hspace{#1}#2\hspace{#1}\endgroup\over\hspace{#1}#3\hspace{#1}}}
%----------------------------------------
\newlist{rlist}{enumerate}{3}
\setlist[rlist]{itemsep=0mm,label=\roman*.}
\newlist{brlist}{enumerate}{3}
\setlist[brlist]{itemsep=0mm,label=\bf\roman*.}
\newlist{tropos}{enumerate}{3}
\setlist[tropos]{label=\bf\textit{\arabic*\textsuperscript{oς}\;Τρόπος :},leftmargin=0cm,itemindent=2.3cm,ref=\bf{\arabic*\textsuperscript{oς}\;Τρόπος}}
\newcommand{\tss}[1]{\textsuperscript{#1}}
\newcommand{\tssL}[1]{\MakeLowercase{\textsuperscript{#1}}}
\usepackage{hhline}
%----------- ΓΡΑΦΙΚΕΣ ΠΑΡΑΣΤΑΣΕΙΣ ---------
\pgfkeys{/pgfplots/aks_on/.style={axis lines=center,
xlabel style={at={(current axis.right of origin)},xshift=1.5ex, anchor=center},
ylabel style={at={(current axis.above origin)},yshift=1.5ex, anchor=center}}}
\pgfkeys{/pgfplots/grafikh parastash/.style={\xrwma,line width=.4mm,samples=200}}
\pgfkeys{/pgfplots/belh ar/.style={tick label style={font=\scriptsize},axis line style={-latex}}}
%-----------------------------------------
\usepackage{multicol}
\usepackage{mathtools}
\usepackage{wrap-rl}
\usepackage{extarrows}
\newcommand{\eq}[1]{\xlongequal{#1}}
\newcommand{\eqq}[2]{\xlongequal[#2]{#1}}
\DeclareMathOperator*{\Eq}{=}



\begin{document}
\pagenumbering{gobble}% Remove page numbers (and reset to 1)
\clearpage
\titlos{ΜΑΘΗΜΑΤΙΚΑ Γ΄ ΓΥΜΝΑΣΙΟΥ}{Αλγεβρικές Παραστάσεις}{Πολυώνυμα}
\clearpage
\pagenumbering{arabic}
\begin{Methodos}[Τιμή πολυωνύμου]
Αν $ A $ είναι ένα πολυώνυμο μιας ή περισσότερων μεταβλητών τότε προκειμένου να υπολογίσουμε την τιμή του για δοσμένες τιμές των μεταβλητών του
\begin{bhma}
\item \textbf{Αντικατάσταση τιμών}\\
Αντικαθιστούμε τις τιμές των μεταβλητών που μας δίνονται στο πολυώνυμο, οπότε μετατρέπεται από αλγεβρική σε αριθμιτική παράσταση.
\item \textbf{Πράξεις}\\
Εκτελούμε τις πράξεις μέσα στην αριθμιτική παράσταση που προέκυψε με τη γνωστή σειρά και υπολογίζουμε το αποτέλεσμα.
\end{bhma}
\end{Methodos}
\Paradeigma{Υπολογισμόσ τμήσ}
\textbf{Να υπολογιστεί η τιμή του παρακάτω πολυωνύμου}
{\boldmath \[ A=3x^2y^3-4xy^2+xz^4 \]}
\textbf{όταν γνωρίζουμε οτι {\boldmath$ x=-1, y=2 $} και {\boldmath$ z=3 $}}.\\\\
\lysh\\
Αν θέσουμε όπου $ x=-1, y=2 $ και $ z=3 $ στη θέση των μεταβλητών του πολυωνύμου τότε προκύπτει αριθμητική παράσταση.
\begin{align*}
A=3x^2y^3-4xy^2+xz^4&\eqq{x=-1,y=2}{z=3} 3\cdot(-1)^2\cdot2^3-4\cdot(-1)\cdot2^2+(-1)\cdot3^4\\&=3\cdot 1\cdot 8-4\cdot(-1)\cdot4+(-1)\cdot81\\
&=24+16-81=-41
\end{align*}
Η τιμή λοιπόν του πολυωνύμου για τις δοσμένες τιμές των μεταβλητών του θα είναι ίση με $ -41 $.\\\\
\Paradeigma{Υπολογισμόσ τιμήσ}
\textbf{Να υπολογιστεί η τιμή του παρακάτω πολυωνύμου}
{\boldmath \[ P(x)=x^3-4x^2+3x-7 \]}
\textbf{εαν μας δίνεται οτι {\boldmath$ x=-2$}}.\\\\
\lysh\\
Το πολυώνυμο που μας δίνεται είναι μιας μεταβλητής. Θέτοντας λοιπόν όπου $ x=-2 $ η τιμή του θα συμβολιστεί με $ P(-2) $. Θα έχουμε λοιπόν
\begin{align*} P(x)=x^3-4x^2+3x-7\xRightarrow{x=-2}P(-2)&=(-2)^3-4\cdot(-2)^2+3\cdot(-2)-7\\
&=-8-4\cdot4+3\cdot(-2)-7\\
&=-8-16-6-7=-37 
\end{align*}
Προέκυψε λοιπόν η τιμή του πολυωνύμου $ P(-2)=-37 $.
\begin{Methodos}[Αλλαγή μεταβλητήσ]
Όπως και στην προηγούμενη μέθοδο αντικαταστήσαμε στη θέση των μεταβλητών σταθερούς αριθμούς με τον ίδιο τρόπο μπορούμε να θέσουμε στη θέση των αρχικών μεταβλητών, νέες μεταβλητές.
\begin{bhma}
\item \textbf{Αντικατάσταση}\\
Αντικαθιστούμε στη θέση των αρχικών μεταβλητών τις νέες μεταβλητές που μας δίνονται.
\item \textbf{Απλοποίηση}\\
Προκύπτει τότε μια νέα αλγεβρική παράσταση την οποία απλοποιούμε εκτελώνας όλες τις δυνατές πράξεις.
\end{bhma}
\end{Methodos}
\Paradeigma{Αλλαγή μεταβλητήσ}
\textbf{Δίνεται το πολυώνυμο {\boldmath$ P(x)=2x^2-3x+5 $}. Να βρεθούν τα πολυώνυμα}
{\boldmath
\begin{multicols}{3}
\begin{brlist}
\item $ P(t) $
\item $ P(2x) $
\item $ P(-3s) $
\end{brlist}\end{multicols}}
\lysh
\begin{rlist}
\item Αντικαθιστώντας τη μεταβλητή $ t $ στη θέση της μεταβλητής $ x $ του πολυωνύμου $ P $ παρατηρούμε οτι γίνεται μόνο αλλαγή του συμβολισμού της πράγμα που σημαίνει οτι η δομή του πολυωνύμου δεν θα αλλάξει. Έχουμε λοιπόν
\[ P(x)=2x^2-3x+5\xRightarrow{x\rightarrow t}P(t)=2t^2-3t+5 \]
\item Θέτοντας στη θέση της μεταβλητής $ x $ το μονώνυμο $ 2x $ στο πολυώνυμο $ P $ θα προκύψει
\begin{align*}
 P(x)=2x^2-3x+5\xRightarrow{x\rightarrow 2x}P(2x)&=2(2x)^2-3\cdot(2x)+5\\
&=2\cdot 4x^2-6x+5=8x^2-6x+5
\end{align*}
\item Θέτοντας όπου $ x $ το μονώνυμο $ -3x $ έχουμε
\begin{align*}
 P(x)=2x^2-3x+5\xRightarrow{x\rightarrow -3x}P(-3x)&=2(-3x)^2-3\cdot(-3x)+5\\
&=2\cdot 9x^2+9x+5=18x^2+9x+5
\end{align*}
\end{rlist}
\begin{Methodos}[Πρόσθεση πολυωνύμων]
Για να προσθέσουμε ή να αφαιρέσουμε δύο ή περισσότερα πολυώνυμα μεταξύ τους εκτελούμε τις πράξεις μεταξύ των συντελεστών των όμοιων μονωνύμων τους κάνοντας αναγωγή ομοίων όρων.
\end{Methodos}
\Paradeigma{Πρόσθεση πολυωνύμων}
\textbf{Δίνονται τα πολυώνυμα {\boldmath$ A(x)=x^3-5x^2+2x+1 $} και {\boldmath$ B(x)=3x^3-x^2+5x+4$}. Να βρεθούν τα πολυώνυμα}
{\boldmath
\begin{multicols}{2}
\begin{brlist}
\item $ A(x)+B(x) $
\item $ B(x)-A(x) $
\end{brlist}
\end{multicols}}
\lysh\\
Όπως και στην πρόσθεση έτσι και στην αφαίρεση των πολυωνύμων θα χρειαστεί να ξεχωρίσουμε τους όμοιους μεταξύ τους όρους.
\begin{rlist}
\item Έχουμε λοιπόν
\begin{gather*}
A(x)+B(x)=\left( x^3-5x^2+2x+1\right) +\left( 3x^3-x^2+5x+4\right)=\\
x^3+3x^3-5x^2-x^2+2x+5x+1+4=4x^3-6x^2+7x+5
\end{gather*}
\item Για τη διαφορά των δύο πολυωνύμων θα χρειαστεί να αλλάξουμε τα πρόσημα του δεύτερου πολυωνύμου.
\begin{gather*}
B(x)-A(x)=\left( 3x^3-x^2+5x+4\right)-\left( x^3-5x^2+2x+1\right)=\\
3x^3-x^2+5x+4-x^3+5x^2-2x-1=2x^3+4x^2+3x+3
\end{gather*}
\end{rlist}
\begin{Methodos}[Πολλαπλασιασμόσ πολυωνύμων]
Για τον πολλαπλασιασμό πολυωνύμων κάνουμε χρήση της επιμεριστικής ιδιότητας.
\begin{bhma}
\item \textbf{Πολλαπλασιασμός}\\
Για να πολλαπλασιάσουμε δύο πολυώνυμα μεταξύ τους πολλαπλασιάζουμε κάνοντας χρήση της επιμεριστικής ιδιότητας κάθε όρο του πρώτου με κάθεναν από τους όρους του δεύτερο πολυωνύμου.
\item \textbf{Αναγωγή ομοίων όρων}\\
Αφού βρεθεί το ανάπτυγμα του γινομένου προσθέτουμε αν υπάρχουν τους όμοιους όρους που θα προκύψουν μεταξύ τους ώστε να απλοποιηθεί η παράσταση.
\end{bhma}
\end{Methodos}
\Paradeigma{Πολλαπλασιασμόσ πολυωνύμων}
\end{document}