\documentclass[twoside]{article}
\usepackage[amsbb,subscriptcorrection,zswash,mtpcal,mtphrb,mtpfrak]{mtpro2}
\usepackage[no-math,cm-default]{fontspec}
\usepackage{amsmath}
\usepackage{xunicode}
\usepackage{polyglossia}
\let\hbar\relax
\defaultfontfeatures{Mapping=tex-text,Scale=MatchLowercase}
\setmainfont[Mapping=tex-text,Numbers=Lining,Scale=1.0,BoldFont={Nimbus Roman Bold}]{Nimbus Roman}
\newfontfamily\scfont{KerkisSans}

\usepackage{fontawesome5}
\newcommand{\xrwma}{red!70!black}
%------TIKZ - ΣΧΗΜΑΤΑ - ΓΡΑΦΙΚΕΣ ΠΑΡΑΣΤΑΣΕΙΣ ----
\usepackage{tikz,pgfplots}
\usepackage{tkz-euclide}
\usepackage[framemethod=TikZ]{mdframed}
\usetikzlibrary{decorations.pathreplacing}
\tkzSetUpPoint[size=7,fill=white]
%-----------------------
\usepackage{calc}
\usepackage{hhline}
\usepackage[explicit]{titlesec}
\usepackage{graphicx}
\usepackage{multicol}
\usepackage{multirow}
\usepackage{tabularx}
\usetikzlibrary{backgrounds}
\usepackage{sectsty}
\sectionfont{\centering}
\usepackage{enumitem}
\setlist[enumerate]{label=\bf{\large \arabic*.}}
\usepackage{adjustbox}
\usepackage{mathimatika,gensymb,eurosym}
%-------- ΜΑΘΗΜΑΤΙΚΑ ΕΡΓΑΛΕΙΑ ---------
\usepackage{mathtools}
%----------------------
%-------- ΠΙΝΑΚΕΣ ---------
\usepackage{booktabs}
%----------------------
%----- ΥΠΟΛΟΓΙΣΤΗΣ ----------
\usepackage{calculator}
%----------------------------
%------ ΔΙΑΓΩΝΙΟ ΣΕ ΠΙΝΑΚΑ -------
\usepackage{array}
\newcommand\diag[5]{%
\multicolumn{1}{|m{#2}|}{\hskip-\tabcolsep
$\vcenter{\begin{tikzpicture}[baseline=0,anchor=south west,outer sep=0]
\path[use as bounding box] (0,0) rectangle (#2+2\tabcolsep,\baselineskip);
\node[minimum width={#2+2\tabcolsep-\pgflinewidth},
minimum  height=\baselineskip+#3-\pgflinewidth] (box) {};
\draw[line cap=round] (box.north west) -- (box.south east);
\node[anchor=south west,align=left,inner sep=#1] at (box.south west) {#4};
\node[anchor=north east,align=right,inner sep=#1] at (box.north east) {#5};
\end{tikzpicture}}\rule{0pt}{.71\baselineskip+#3-\pgflinewidth}$\hskip-\tabcolsep}}
%---------------------------------
%---- ΟΡΙΖΟΝΤΙΟ - ΚΑΤΑΚΟΡΥΦΟ - ΠΛΑΓΙΟ ΑΓΚΙΣΤΡΟ ------
\newcommand{\orag}[3]{\node at (#1)
{$ \overcbrace{\rule{#2mm}{0mm}}^{{\scriptsize #3}} $};}
\newcommand{\kag}[3]{\node at (#1)
{$ \undercbrace{\rule{#2mm}{0mm}}_{{\scriptsize #3}} $};}
\newcommand{\Pag}[4]{\node[rotate=#1] at (#2)
{$ \overcbrace{\rule{#3mm}{0mm}}^{{\rotatebox{-#1}{\scriptsize$#4$}}}$};}
%-----------------------------------------


%------------------------------------------
\newcommand{\tss}[1]{\textsuperscript{#1}}
\newcommand{\tssL}[1]{\MakeLowercase{\textsuperscript{#1}}}
%---------- ΛΙΣΤΕΣ ----------------------
\newlist{bhma}{enumerate}{3}
\setlist[bhma]{label=\bf\textit{\arabic*\textsuperscript{o}\;Βήμα :},leftmargin=0cm,itemindent=1.8cm,ref=\bf{\arabic*\textsuperscript{o}\;Βήμα}}
\newlist{brlist}{enumerate}{3}
\setlist[brlist]{itemsep=0mm,label=\bf\roman*.}
\newlist{tropos}{enumerate}{3}
\setlist[tropos]{label=\bf\textit{\arabic*\textsuperscript{oς}\;Τρόπος :},leftmargin=0cm,itemindent=2.3cm,ref=\bf{\arabic*\textsuperscript{oς}\;Τρόπος}}
% Αν μπει το bhma μεσα σε tropo τότε
%\begin{bhma}[leftmargin=.7cm]
\tkzSetUpPoint[size=7,fill=white]
\tikzstyle{pl}=[line width=0.3mm]
\tikzstyle{plm}=[line width=0.4mm]
\usepackage{etoolbox}
\makeatletter
%\renewrobustcmd{\anw@true}{\let\ifanw@\iffalse}
%\renewrobustcmd{\anw@false}{\let\ifanw@\iffalse}\anw@false
%\newrobustcmd{\noanw@true}{\let\ifnoanw@\iffalse}
%\newrobustcmd{\noanw@false}{\let\ifnoanw@\iffalse}\noanw@false
%\renewrobustcmd{\anw@print}{\ifanw@\ifnoanw@\else\numer@lsign\fi\fi}
%\makeatother


\begin{document}
%\titlos{Μαθηματικά Γ΄ Γυμνασίου}{Εξισώσεις}{Εξισώσεις 2\tssL{ου} βαθμού}
%\thewria
\begin{enumerate}
\item
\begin{enumerate}[label=\roman*.]
\item Τι ονομάζουμε εξίσωση 2\textsuperscript{ου} βαθμού;
\item Ποιός αριθμός μας δείχνει το πλήθος των ριζών μιας εξίσωσης 2\textsuperscript{ου} βαθμού;
\item Πότε μια εξίσωση έχει 2 ρίζες, πότε μια και πότε είναι αδύνατη;
\end{enumerate}
\item \textbf{Να χαρακτηριστούν οι παρακάτω εξισώσεις ως σωστές (Σ) ή λανθασμένες (Λ).}
\begin{enumerate}[label=\roman*.]
\item Αν για μια εξίσωση 2\textsuperscript{ου} βαθμού έχουμε $ \varDelta>0 $ τότε έχει 2 άνισες λύσεις.
\item Αν για μια εξίσωση 2\textsuperscript{ου} βαθμού έχουμε $ \varDelta<0 $ τότε έχει μια διπλή λύση.
\item Η εξίσωση $ ax^2+\beta x+\gamma=0 $ παριστάνει μια εξίσωση 2\textsuperscript{ου} βαθμού για κάθε τιμή του $ a $.
\end{enumerate}
\end{enumerate}
\newpage
%\askhseis
\begin{enumerate}
\item Να λυθούν οι παρακάτω εξισώσεις 2\textsuperscript{ου} βαθμού με παραγοντοποίηση.
\begin{multicols}{3}
\begin{enumerate}[label=\roman*.]
\item $ x^2+4x=0 $
\item $ x^2-5x=0 $
\item $ 2x^2-4x=0 $
\item $ 4x^2-3x=0 $
\item $ x^2-4=0 $
\item $ x^2-25=0 $
\item $ 2x^2-32=0 $
\item $ x^2+16=0 $
\item $ 2x^3-72x=0 $
\end{enumerate}
\end{multicols}
\item Να λυθούν οι παρακάτω εξισώσεις 2\textsuperscript{ου} βαθμού με τη βοήθεια του τύπου.
\begin{multicols}{3}
\begin{enumerate}[label=\roman*.]
\item $ x^2-3x+2=0 $
\item $ x^2-5x+6=0 $
\item $ x^2-7x+12=0 $
\item $ y^2-y-2=0 $
\item $ -z^2+3z+4=0 $
\item $ 2x^2-5x+3=0 $
\item $ \frac{1}{2}x^2-x-4=0 $
\item $ 0{,}1x^2-0{,}7x+1{,}2=0 $
\item $ -y^2+y+3=0 $
\end{enumerate}
\end{multicols}
\item Να λυθούν οι παρακάτω εξισώσεις 2\textsuperscript{ου} βαθμού με τη βοήθεια του τύπου.
\begin{multicols}{3}
\begin{enumerate}[label=\roman*.]
\item $ x^2-2x+1=0 $
\item $ x^2+4x+4=0 $
\item $ -x^2+6x-9=0 $
\item $ 25y^2+10y+1=0 $
\item $ z^2-z+\frac{1}{4}=0 $
\item $ \frac{x^2}{9}-\frac{2x}{3}+1=0 $
\end{enumerate}
\end{multicols}
\item Να λυθούν οι παρακάτω εξισώσεις 2\textsuperscript{ου} βαθμού με τη βοήθεια του τύπου.
\begin{multicols}{3}
\begin{enumerate}[label=\roman*.]
\item $ x^2-x+1=0 $
\item $ -x^2+x-3=0 $
\item $ 4x^2+1=0 $
\end{enumerate}
\end{multicols}
\item Να λυθούν οι παρακάτω εξισώσεις.
\begin{multicols}{3}
\begin{enumerate}[label=\roman*.]
\item $ x^2-x-4=2 $
\item $ y^2-3y+6=2y $
\item $ x^2-3x+1=x-2 $
\item $ 2z^2-z-2=z^2 $
\item $ x^2+2x+4=4x+3 $
\item $ x^2-8x+5=2x^2-7 $
\item $ 2x^2-5x+3=(x-1)^2 $
\item $ x^2+5x-4=2x^2 $
\item $ (x-3)^2+x=2x-1 $
\end{enumerate}
\end{multicols}
\item Να λυθούν οι παρακάτω εξισώσεις.
\begin{multicols}{3}
\begin{enumerate}[label=\roman*.]
\item $ \dfrac{x^2-3}{4}-\dfrac{2x+1}{3}=2+\dfrac{3x}{8} $
\item $ \dfrac{(x-2)^2}{3}-x=\dfrac{x}{5}-3 $
\item $ x\left(\dfrac{x}{2}-\dfrac{3}{4} \right)+\dfrac{1-x}{5}=\dfrac{37}{20}  $
\end{enumerate}
\end{multicols}
\item Να παραγοντοποιηθούν τα παρακάτω τριώνυμα.
\begin{multicols}{3}
\begin{enumerate}[label=\roman*.]
\item $ x^2-3x+2 $
\item $ x^2-5x+6 $
\item $ y^2-y-2 $
\item $ z^2+2z+1 $
\item $ 2y^2-5y+3 $
\item $ x^2+x+4 $
\item $ -4y^2+4y-1 $
\item $ 3z^2+10z-8 $
\item $ 4x^2+20x+25 $
\end{enumerate}
\end{multicols}
\item Για καθένα από τα παρακάτω σχήματα 
\end{enumerate}
\end{document}

