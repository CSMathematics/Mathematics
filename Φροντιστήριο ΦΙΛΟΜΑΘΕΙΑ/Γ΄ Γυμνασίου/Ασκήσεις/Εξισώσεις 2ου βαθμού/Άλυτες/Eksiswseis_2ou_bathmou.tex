\documentclass[11pt,a4paper,twocolumn]{article}
\usepackage[english,greek]{babel}
\usepackage[utf8]{inputenc}
\usepackage{nimbusserif}
\usepackage[T1]{fontenc}
\usepackage[left=1.50cm, right=1.50cm, top=2.00cm, bottom=2.00cm]{geometry}
\usepackage{amsmath}
\let\myBbbk\Bbbk
\let\Bbbk\relax
\usepackage[amsbb,subscriptcorrection,zswash,mtpcal,mtphrb,mtpfrak]{mtpro2}
\usepackage{graphicx,multicol,multirow,enumitem,tabularx,mathimatika,gensymb,venndiagram,hhline,longtable,tkz-euclide,fontawesome5,eurosym,tcolorbox,tabularray}
\usepackage[explicit]{titlesec}
\tcbuselibrary{skins,theorems,breakable}
\newlist{rlist}{enumerate}{3}
\setlist[rlist]{itemsep=0mm,label=\roman*.}
\newlist{alist}{enumerate}{3}
\setlist[alist]{itemsep=0mm,label=\alph*.}
\newlist{balist}{enumerate}{3}
\setlist[balist]{itemsep=0mm,label=\bf\alph*.}
\newlist{Alist}{enumerate}{3}
\setlist[Alist]{itemsep=0mm,label=\Alph*.}
\newlist{bAlist}{enumerate}{3}
\setlist[bAlist]{itemsep=0mm,label=\bf\Alph*.}
\newlist{askhseis}{enumerate}{3}
\setlist[askhseis]{label={\Large\thesection}.\arabic*.}
\renewcommand{\textstigma}{\textsigma\texttau}
\newlist{thema}{enumerate}{3}
\setlist[thema]{label=\bf\large{ΘΕΜΑ \textcolor{black}{\Alph*}},itemsep=0mm,leftmargin=0cm,itemindent=18mm}
\newlist{erwthma}{enumerate}{3}
\setlist[erwthma]{label=\bf{\large{\textcolor{black}{\Alph{themai}.\arabic*}}},itemsep=0mm,leftmargin=0.8cm}

\newcommand{\kerkissans}[1]{{\fontfamily{maksf}\selectfont \textbf{#1}}}
\renewcommand{\textdexiakeraia}{}

\usepackage[
backend=biber,
style=alphabetic,
sorting=ynt
]{biblatex}

\DeclareTblrTemplate{caption}{nocaptemplate}{}
\DeclareTblrTemplate{capcont}{nocaptemplate}{}
\DeclareTblrTemplate{contfoot}{nocaptemplate}{}
\NewTblrTheme{mytabletheme}{
\SetTblrTemplate{caption}{nocaptemplate}{}
\SetTblrTemplate{capcont}{nocaptemplate}{}
\SetTblrTemplate{contfoot}{nocaptemplate}{}
}

\NewTblrEnviron{mytblr}
\SetTblrStyle{firsthead}{font=\bfseries}
\SetTblrStyle{firstfoot}{fg=red2}
\SetTblrOuter[mytblr]{theme=mytabletheme}
\SetTblrInner[mytblr]{
rowspec={t{7mm}},columns = {c},
width = 0.85\linewidth,
row{odd} = {bg=red9,fg=black,ht=8mm},
row{even} = {bg=red7,fg=black,ht=8mm},
hlines={white},vlines={white},
row{1} = {bg=red4, fg=white, font=\bfseries\fontfamily{maksf}},rowhead = 1,
hline{2} = {.7mm}, % midrule  
}
\newcounter{askhsh}
\setcounter{askhsh}{1}
\newcommand{\askhsh}{\large\theaskhsh.\ \addtocounter{askhsh}{1}}

\titleformat{\section}{\Large}{\kerkissans{\thesection}}{10pt}{\Large\kerkissans{#1}}

\setlength{\columnsep}{5mm}
\titleformat{\paragraph}
{\large}%
{}{0em}%
{\textcolor{red!80!black}{\faSquare\ \ \kerkissans{\bmath{#1}}}}
\setlength{\parindent}{0pt}

\newcommand{\eng}[1]{\selectlanguage{english}#1\selectlanguage{greek}}

\begin{document}
\twocolumn[{
\centering
\kerkissans{{\huge Εξισώσεις 2ου βαθμού}\\\vspace{3mm} {\Large ΑΣΚΗΣΕΙΣ}}\vspace{5mm}}]
\paragraph{Ερωτήσεις θεωρίας}
\begin{alist}
\item Τι ονομάζουμε εξίσωση 2\textsuperscript{ου} βαθμού;
\item Ποιός αριθμός μας δείχνει το πλήθος των ριζών μιας εξίσωσης 2\textsuperscript{ου} βαθμού;
\item Πότε μια εξίσωση έχει 2 ρίζες, πότε μια και πότε είναι αδύνατη;
\end{alist}
\askhsh Να χαρακτηριστούν οι παρακάτω εξισώσεις ως σωστές (Σ) ή λανθασμένες (Λ).
\begin{alist}
\item Αν για μια εξίσωση 2\textsuperscript{ου} βαθμού έχουμε $ \varDelta>0 $ τότε έχει 2 άνισες λύσεις.
\item Αν για μια εξίσωση 2\textsuperscript{ου} βαθμού έχουμε $ \varDelta<0 $ τότε έχει μια διπλή λύση.
\item Η εξίσωση $ ax^2+\beta x+\gamma=0 $ παριστάνει μια εξίσωση 2\textsuperscript{ου} βαθμού για κάθε τιμή του $ a $.
\end{alist}
\paragraph{Επίλυση εξισώσεων - Ειδικές περιπτώσεις}
\askhsh Να λυθούν οι παρακάτω εξισώσεις 2\textsuperscript{ου} βαθμού με παραγοντοποίηση.
\begin{multicols}{2}
\begin{alist}
\item $ x^2+4x=0 $
\item $ x^2-5x=0 $
\item $ 2x^2-4x=0 $
\item $ 4x^2-3x=0 $
\item $ 2x^2-15x=0 $
\end{alist}
\end{multicols}
\askhsh Να λυθούν οι παρακάτω εξισώσεις 2\textsuperscript{ου} βαθμού με παραγοντοποίηση.
\begin{multicols}{2}
\begin{alist}
\item $ x^2-4=0 $
\item $ x^2-25=0 $
\item $ 2x^2-32=0 $
\item $ x^2+16=0 $
\item $2x^2-3=0$
\item $3x^2=48$
\end{alist}
\end{multicols}
\paragraph{Επίλυση εξισώσεων - Τύπος}
\askhsh Να λυθούν οι παρακάτω εξισώσεις 2\textsuperscript{ου} βαθμού με τη βοήθεια του τύπου.
\begin{multicols}{2}
\begin{alist}[leftmargin=5mm]
\item $ x^2-3x+2=0 $
\item $ x^2-5x+6=0 $
\item $ x^2-7x+12=0 $
\item $ y^2-y-2=0 $
\item $ -z^2+3z+4=0 $
\item $ 2x^2-5x+3=0 $
\item $ \frac{1}{2}x^2-x-4=0 $
\item {\small $ 0{,}1x^2-0{,}7x+1{,}2=0 $}
\item $ -y^2+y+3=0 $
\end{alist}
\end{multicols}
\askhsh Να λυθούν οι παρακάτω εξισώσεις 2\textsuperscript{ου} βαθμού με τη βοήθεια του τύπου.
\begin{multicols}{2}
\begin{alist}
\item $ x^2-2x+1=0 $
\item $ x^2+4x+4=0 $
\item $ -x^2+6x-9=0 $
\item $ 25y^2+10y+1=0 $
\item $ z^2-z+\frac{1}{4}=0 $
\item $ \frac{x^2}{9}-\frac{2x}{3}+1=0 $
\end{alist}
\end{multicols}
\askhsh Να λυθούν οι παρακάτω εξισώσεις 2\textsuperscript{ου} βαθμού με τη βοήθεια του τύπου.
\begin{multicols}{2}
\begin{alist}
\item $ x^2-x+1=0 $
\item $ -x^2+x-3=0 $
\item $ 4x^2+1=0 $
\item $2x^2+x+1=0$
\end{alist}
\end{multicols}
\askhsh Να λυθούν οι παρακάτω εξισώσεις.
\begin{alist}
\item $ x^2-x-4=2 $
\item $ y^2-3y+6=2y $
\item $ x^2-3x+1=x-2 $
\item $ 2z^2-z-2=z^2 $
\item $ x^2+2x+4=4x+3 $
\item $ x^2-8x+5=2x^2-7 $
\item $ 2x^2-5x+3=(x-1)^2 $
\item $ x^2+5x-4=2x^2 $
\item $ (x-3)^2+x=2x-1 $
\end{alist}
\askhsh Να λυθούν οι παρακάτω εξισώσεις.
\begin{alist}
\item $\dfrac{x^2+2}{3}=\dfrac{x+5}{3}-1$
\item $\dfrac{x^2}{4}+\dfrac{3x-2}{2}=3$
\item $ \dfrac{x^2-3}{4}-\dfrac{2x+1}{3}=2+\dfrac{3x}{8} $
\item $ \dfrac{(x-2)^2}{3}-x=\dfrac{x}{5}-3 $
\item $ x\left(\dfrac{x}{2}-\dfrac{3}{4} \right)+\dfrac{1-x}{5}=\dfrac{37}{20}  $
\end{alist}
\paragraph{Παραγοντοποίηση τριωνύμου}
\askhsh Να παραγοντοποιηθούν τα παρακάτω τριώνυμα.
\begin{multicols}{2}
\begin{alist}
\item $ x^2-3x+2 $
\item $ x^2-5x+6 $
\item $ y^2-y-2 $
\item $ z^2+2z+1 $
\item $ 2y^2-5y+3 $
\item $ x^2+x+4 $
\item $ -4y^2+4y-1 $
\item $ 3z^2+10z-8 $
\item $ 4x^2+20x+25 $
\end{alist}
\end{multicols}
\askhsh Απλοποιήστε τα παρακάτω κλάσματα.
\begin{multicols}{2}
\begin{alist}
\item $\dfrac{x^2-3x+2}{(x-1)^2}$
\item $\dfrac{x^2-4}{x^2-x-2}$
\item $\dfrac{x^2+3x-4}{x^2-16}$
\item $\dfrac{x^2+7x+12}{x^2+x-6}$
\item $\dfrac{2x^2+3x-2}{4x^2-1}$
\item $\dfrac{-x^2+2x+3}{x^2-3x}$
\item $\dfrac{x^2-4x+4}{-x^2+3x-2}$
\item $\dfrac{2x^2-5x+3}{4x^2-12x+9}$
\end{alist}
\end{multicols}
\askhsh Για καθένα από τα παρακάτω σχήματα 

\end{document}
