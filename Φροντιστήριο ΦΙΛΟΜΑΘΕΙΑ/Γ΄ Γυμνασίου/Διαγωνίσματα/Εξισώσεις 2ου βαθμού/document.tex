\documentclass[internet]{diag-xelatex}
\usepackage[amsbb]{mtpro2}
\usepackage[no-math,cm-default]{fontspec}
\usepackage{xunicode}
\usepackage{xltxtra}
\usepackage{xgreek}
\usepackage{amsmath}
\defaultfontfeatures{Mapping=tex-text,Scale=MatchLowercase}
\setmainfont[Mapping=tex-text,Numbers=Lining,Scale=1.0,BoldFont={Minion Pro Bold}]{Minion Pro}
\newfontfamily\scfont{GFS Artemisia}
\font\icon = "Webdings"
\usepackage[amsbb]{mtpro2}
\usepackage[left=2.00cm, right=2.00cm, top=2.00cm, bottom=3.00cm]{geometry}
\xroma{red}
\newcommand{\tss}[1]{\textsuperscript{#1}}
\newcommand{\tssL}[1]{\MakeLowercase\textsuperscript{#1}}



\begin{document}
\titlos{ΜΑΘΗΜΑΤΙΚΑ Γ΄ ΓΥΜΝΑΣΙΟΥ}{ΕΞΙΣΩΣΕΙΣ 2\tss{ου} ΒΑΘΜΟΥ}
\thewria
\begin{thema}
\item Να απαντήσεις στις παρακάτω ερωτήσεις.
\begin{enumerate}[label=\bf\roman*.]
\item Πότε μια εξίσωση με έναν άγνωστο ονομάζεται 2\tss{ου} βαθμού;
\item Για ποιές τιμές του πραγματικού αριθμού $ a $, η εξίσωση $ ax^2+\beta x+\gamma=0 $, παριστάνει εξίσωση 2\tss{ου} βαθμού;
\item Ποιά συνθήκη πρέπει να ισχύει ώστε η εξίσωση $ ax^2+\beta x+\gamma=0 $ να έχει 1 διπλή λύση;
\item Πόσες λύσεις έχει μια εξίσωση 2\textsuperscript{ου} βαθμού όταν $ \varDelta<0 $;
\end{enumerate}
\item \textbf{Α.} Να συμπληρώσεις τα παρακάτω κενά.
\begin{enumerate}[label=\bf\roman*.]
\item 
\end{enumerate}
\textbf{Β.} Να χαρακτηρίσεις τις παρακάτω προτάσεις ως σωστές (Σ) ή λάνθασμένες (Λ).
\begin{enumerate}[label=\bf\roman*.]
\item Μια εξίσωση 2\textsuperscript{ου} βαθμού έχει πάντα 2 πραγματικές λύσεις.
\item Εαν μια εξίσωση 2\textsuperscript{ου} βαθμού έχει θετική διακρίνουσα τότε οι λύσεις της δίνονται από τον τύπο $ x_{1,2}=\dfrac{\beta\pm\sqrt{\varDelta}}{2a} $
\item 
\end{enumerate}
\end{thema}
\newpage
\noindent
\askhseis
\begin{thema}
\item 
\end{thema}
\end{document}
