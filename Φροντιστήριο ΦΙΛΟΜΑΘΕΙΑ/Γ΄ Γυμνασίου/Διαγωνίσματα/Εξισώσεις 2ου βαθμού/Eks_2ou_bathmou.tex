\documentclass[internet]{diag-xelatex}
\usepackage[amsbb,subscriptcorrection,zswash,mtpcal,mtphrb]{mtpro2}
\usepackage[no-math,cm-default]{fontspec}
\usepackage{xunicode}
\usepackage{xgreek}
\usepackage{amsmath}
\defaultfontfeatures{Mapping=tex-text,Scale=MatchLowercase}
\setmainfont[Mapping=tex-text,Numbers=Lining,Scale=1.0,BoldFont={Minion Pro Bold}]{Minion Pro}
\newfontfamily\scfont{GFS Artemisia}
\font\icon = "Webdings"
\usepackage[amsbb,subscriptcorrection,zswash,mtpcal,mtphrb]{mtpro2}
\xroma{red!80!black}
%------TIKZ - ΣΧΗΜΑΤΑ - ΓΡΑΦΙΚΕΣ ΠΑΡΑΣΤΑΣΕΙΣ ----
\usepackage{tikz}
\usepackage{tkz-euclide}
\usetkzobj{all}
\usepackage[framemethod=TikZ]{mdframed}
\usetikzlibrary{decorations.pathreplacing}
\usepackage{pgfplots}
\usetkzobj{all}
%-----------------------
\usepackage{calc}
\usepackage{hhline}
\usepackage[explicit]{titlesec}
\usepackage{graphicx}
\usepackage{multicol}
\usepackage{multirow}
\usepackage{enumitem}
\usepackage{tabularx}
\usepackage[decimalsymbol=comma]{siunitx}
\usetikzlibrary{backgrounds}
\usepackage{sectsty}
\sectionfont{\centering}
\setlist[enumerate]{label=\bf{\large \arabic*.}}
\usepackage{adjustbox}
\usepackage{mathimatika,gensymb,eurosym,wrap-rl}
\usepackage{systeme,regexpatch}
%-------- ΜΑΘΗΜΑΤΙΚΑ ΕΡΓΑΛΕΙΑ ---------
\usepackage{mathtools}
%----------------------
%-------- ΠΙΝΑΚΕΣ ---------
\usepackage{booktabs}
%----------------------
%----- ΥΠΟΛΟΓΙΣΤΗΣ ----------
\usepackage{calculator}
%----------------------------
%------ ΔΙΑΓΩΝΙΟ ΣΕ ΠΙΝΑΚΑ -------
\usepackage{array}
\newcommand\diag[5]{%
\multicolumn{1}{|m{#2}|}{\hskip-\tabcolsep
$\vcenter{\begin{tikzpicture}[baseline=0,anchor=south west,outer sep=0]
\path[use as bounding box] (0,0) rectangle (#2+2\tabcolsep,\baselineskip);
\node[minimum width={#2+2\tabcolsep-\pgflinewidth},
minimum  height=\baselineskip+#3-\pgflinewidth] (box) {};
\draw[line cap=round] (box.north west) -- (box.south east);
\node[anchor=south west,align=left,inner sep=#1] at (box.south west) {#4};
\node[anchor=north east,align=right,inner sep=#1] at (box.north east) {#5};
\end{tikzpicture}}\rule{0pt}{.71\baselineskip+#3-\pgflinewidth}$\hskip-\tabcolsep}}
%---------------------------------
%---- ΟΡΙΖΟΝΤΙΟ - ΚΑΤΑΚΟΡΥΦΟ - ΠΛΑΓΙΟ ΑΓΚΙΣΤΡΟ ------
\newcommand{\orag}[3]{\node at (#1)
{$ \overcbrace{\rule{#2mm}{0mm}}^{{\scriptsize #3}} $};}
\newcommand{\kag}[3]{\node at (#1)
{$ \undercbrace{\rule{#2mm}{0mm}}_{{\scriptsize #3}} $};}
\newcommand{\Pag}[4]{\node[rotate=#1] at (#2)
{$ \overcbrace{\rule{#3mm}{0mm}}^{{\rotatebox{-#1}{\scriptsize$#4$}}}$};}
%-----------------------------------------


%------------------------------------------
\newcommand{\tss}[1]{\textsuperscript{#1}}
\newcommand{\tssL}[1]{\MakeLowercase{\textsuperscript{#1}}}
%---------- ΛΙΣΤΕΣ ----------------------
\newlist{bhma}{enumerate}{3}
\setlist[bhma]{label=\bf\textit{\arabic*\textsuperscript{o}\;Βήμα :},leftmargin=0cm,itemindent=1.8cm,ref=\bf{\arabic*\textsuperscript{o}\;Βήμα}}
\newlist{rlist}{enumerate}{3}
\setlist[rlist]{itemsep=0mm,label=\roman*.}
\newlist{brlist}{enumerate}{3}
\setlist[brlist]{itemsep=0mm,label=\bf\roman*.}
\newlist{tropos}{enumerate}{3}
\setlist[tropos]{label=\bf\textit{\arabic*\textsuperscript{oς}\;Τρόπος :},leftmargin=0cm,itemindent=2.3cm,ref=\bf{\arabic*\textsuperscript{oς}\;Τρόπος}}
% Αν μπει το bhma μεσα σε tropo τότε
%\begin{bhma}[leftmargin=.7cm]
\tkzSetUpPoint[size=7,fill=white]
\tikzstyle{pl}=[line width=0.3mm]
\tikzstyle{plm}=[line width=0.4mm]
\usepackage{etoolbox}
\makeatletter
\renewrobustcmd{\anw@true}{\let\ifanw@\iffalse}
\renewrobustcmd{\anw@false}{\let\ifanw@\iffalse}\anw@false
\newrobustcmd{\noanw@true}{\let\ifnoanw@\iffalse}
\newrobustcmd{\noanw@false}{\let\ifnoanw@\iffalse}\noanw@false
\renewrobustcmd{\anw@print}{\ifanw@\ifnoanw@\else\numer@lsign\fi\fi}
\makeatother

\begin{document}
\titlos{ΜΑΘΗΜΑΤΙΚΑ Γ΄ ΓΥΜΝΑΣΙΟΥ}{ΕΞΙΣΩΣΕΙΣ 2\tss{ου} ΒΑΘΜΟΥ}
\thewria
\begin{thema}
\item\mbox{}\\\vspace{-7mm}
\begin{erwthma}
\item Να απαντήσετε στις παρακάτω ερωτήσεις.
\begin{enumerate}[label=\bf\roman*.]
\item Πότε μια εξίσωση με έναν άγνωστο ονομάζεται 2\tss{ου} βαθμού;
\item Για ποιές τιμές του πραγματικού αριθμού $ a $, η εξίσωση $ ax^2+\beta x+\gamma=0 $, παριστάνει εξίσωση 2\tss{ου} βαθμού;
\item Ποιά συνθήκη πρέπει να ισχύει ώστε η εξίσωση $ ax^2+\beta x+\gamma=0 $ να έχει μια διπλή λύση;
\item Πόσες λύσεις έχει μια εξίσωση 2\textsuperscript{ου} βαθμού όταν $ \varDelta<0 $;
\end{enumerate}\monades{3}
\item \swstolathos
\begin{enumerate}[label=\bf\roman*.]
\item Μια εξίσωση 2\textsuperscript{ου} βαθμού έχει πάντα 2 πραγματικές λύσεις.
\item Εαν μια εξίσωση 2\textsuperscript{ου} βαθμού έχει θετική διακρίνουσα τότε οι λύσεις της δίνονται από τον τύπο \[ x_{1,2}=\dfrac{\beta\pm\sqrt{\varDelta}}{2a} \]
\item Το τριώνυμο $ x^2+x+1 $ έχει μηδενική διακρίνουσα.
\item Η εξίσωση $ x^2+7x-8=0 $ έχει λύσεις τους αριθμούς $ x=1 $ και $ x=-8 $.
\item Αν ένα τριώνυμο έχει μηδενική διακρίνουσα τότε αποτελεί ανάπτυγμα ταυτότητας.
\end{enumerate}
\end{erwthma}\monades{3}
\item\mbox{}\\\vspace{-7mm}
\begin{erwthma}
\item Να απαντήσετε στις παρακάτω ερωτήσεις.
\begin{enumerate}[label=\bf\roman*.]
\item Αν $ x_1,x_2 $ είναι οι ρίζες του τριωνύμου $ ax^2+\beta x+\gamma $ τότε πως παραγοντοποιείται το τριώνυμο;
\item Σε ποια περίπτωση μια εξίσωση 2\tss{ου} βαθμού είναι αδύνατη;
\item Ποια συνθήκη πρέπει να ισχύει ώστε η εξίσωση $ ax^2+\beta x+\gamma=0 $ να έχει δύο λύσεις άνισες;
\item Πως παραγοντοποιείται το τριώνυμο $ ax^2+\beta x+\gamma $ αν έχει μηδενική διακρίνουσα;
\end{enumerate}\monades{3}
\item \swstolathos
\begin{enumerate}[label=\bf\roman*.]
\item Η εξίσωση $ \lambda x^2+2\lambda x-1=0 $ είναι μια εξίσωση 2\tss{ου} βαθμού για κάθε τιμή του πραγματικού αριθμού $ \lambda $.
\item Οι αριθμοί $ -3 $ και $ 4 $ είναι λύσεις της εξίσωσης $ x^2+x-12=0 $.
\item Το τριώνυμο $ x^2+6x+9 $ αποτελεί ανάπτυγμα ταυτότητας.
\item Η εξίσωση $ x^2+x+10=0 $ είναι αδύνατη.
\item Αν για ένα τριώνυμο ισχύει $ \varDelta\geq 0 $ τότε έχει πραγματικές λύσεις.
\end{enumerate}\monades{3}
\end{erwthma}
\end{thema}
\newpage
\noindent
\askhseis
\begin{thema}
\item \mbox{}\\\vspace{-7mm}
\begin{erwthma}
\item Να λυθούν οι παρακάτω εξισώσεις.
\begin{multicols}{3}
\begin{rlist}
\item $ x^2-4x+3=0 $
\item $ 2y^2-y-3=0 $
\item $ z^2-z+1=0 $
\end{rlist}
\end{multicols}\monades{4,5}
\item Να παραγοντοποιηθούν τα παρακάτω πολυώνυμα.
\begin{multicols}{2}
\begin{rlist}
\item $ x^2-7x+10 $
\item $ y^2+5x-14 $
\end{rlist}
\end{multicols}\monades{2,5}
\end{erwthma}
\item \mbox{}\\\vspace{-7mm}
\begin{erwthma}
\item Να λυθούν οι παρακάτω εξισώσεις.
\begin{multicols}{2}
\begin{rlist}
\item $ (x-2)^2+3x=5x-1 $
\item $ 4(3-x)+12=x^2-(x-5)+9 $
\end{rlist}
\end{multicols}\monades{4}
\item Να παραγοντοποιηθούν τα παρακάτω πολυώνυμα.
\begin{multicols}{2}
\begin{rlist}
\item $ 2y^2+8y+8 $
\item $ x^2-x+6 $
\end{rlist}
\end{multicols}\monades{3}
\end{erwthma}
\item \mbox{}\\\vspace{-7mm}
\begin{erwthma}
\item Να λυθούν οι παρακάτω εξισώσεις.
\begin{multicols}{3}
\begin{rlist}
\item $ 3x^2-x-10=0 $
\item $ (z-1)^2+2=4z-6 $
\item $ \frac{x^2-1}{4}=3x-7 $
\end{rlist}
\end{multicols}\monades{4,5}
\item Να παραγοντοποιηθούν τα παρακάτω πολυώνυμα.
\begin{multicols}{2}
\begin{rlist}
\item $ 4x^2-4x+1 $
\item $ y^2+8y+15 $
\end{rlist}
\end{multicols}\monades{2,5}
\end{erwthma}
\end{thema}
\kaliepityxia
\end{document}

