\documentclass[ektypwsh]{diag-xelatex}
\usepackage[amsbb]{mtpro2}
\usepackage[no-math,cm-default]{fontspec}
\usepackage{xunicode}
\usepackage{xltxtra}
\usepackage{xgreek}
\usepackage{amsmath}
\defaultfontfeatures{Mapping=tex-text,Scale=MatchLowercase}
\setmainfont[Mapping=tex-text,Numbers=Lining,Scale=1.0,BoldFont={Minion Pro Bold}]{Minion Pro}
\newfontfamily\scfont{GFS Artemisia}
\font\icon = "Webdings"
\usepackage[amsbb]{mtpro2}
\xroma{red!80!black}
\newcommand{\tss}[1]{\textsuperscript{#1}}
\newcommand{\tssL}[1]{\MakeLowercase\textsuperscript{#1}}
\newlist{rlist}{enumerate}{3}
\setlist[rlist]{itemsep=0mm,label=\roman*.}
\usepackage{multicol}
%---------------------------------
\makeatletter
\renewcommand*{\@alph}[1]{%
  \ifcase#1\or α\or β\or γ\or
    δ\or ε\or στ\or ζ\or η\or θ\or ι\or κ\or
    λ\or μ\or ν\or ξ\or ο\or π\or ρ\or σ\or
    τ\or υ\or φ\or χ\or ψ\or
    ω\else\@ctrerr\fi
}
\renewcommand*{\@Alph}[1]{%
  \ifcase#1\or Α\or Β\or Γ\or
    Δ\or Ε\or Ζ\or Η\or Θ\or Ι\or Κ\or
    Λ\or Μ\or Ν\or Ξ\or Ο\or Π\or Ρ\or Σ\or
    Τ\or Υ\or Φ\or Χ\or Ψ\or
    Ω\else\@ctrerr\fi
}
\makeatother
%--------------------------------



\begin{document}
\titlos{Μαθηματικά Γ΄ Γυμνασίου}{ΠΟΛΥΩΝΥΜΑ - ΤΑΥΤΟΤΗΤΕΣ}
\thewria
\begin{thema}
\item \textbf{Ερωτήσεις Θεωρίας}\\
Να απαντήσετε στις παρακάτω ερωτήσεις.
\begin{rlist}
\item Τι ονομάζεται πολυώνυμο;
\item Τι ονομάζεται ταυτότητα;
\item Να αποδείξετε τις παρακάτω ταυτότητες :
\begin{enumerate}[itemsep=0mm]
\item $ (a-\beta)^2=a^2-2a\beta+\beta^2 $
\item $ (a+\beta)^3=a^3+3a^2\beta+3a\beta^2+\beta^3 $
\end{enumerate}
\end{rlist}\monades{6}
\item \textbf{Σωστό - Λάθος / Πολλαπλής Επιλογής}\\
Α. Να χαρακτηρίσετε τις παρακάτω προτάσεις ως σωστές (Σ) ή λανθασμένες (Λ).
\begin{rlist}
\item Το πολυώνυμο $ A(x)=3x^2-x^3+5x-1+x^3 $ είναι 3ου βαθμού.
\item Η ισότητα $ x^2\cdot x=x^3 $ αποτελεί ταυτοτητα.
\item Αν το πολυώνυμο $ P(x) $ είναι 4ου βαθμού και το πολυώνυμο $ Q(x) $ είναι 3ου βαθμού τότε το πολυώνυμο $ P(x)\cdot Q(x) $ είναι 12ου βαθμού.
\item Το ανάπτυγμα της ταυτότητας $ (x-2)^2 $ είναι $ x^2-4x+4 $.
\item Αν το πολυώνυμο $ A(x) $ είναι 2ου βαθμού και το πολυώνυμο $ B(x) $ είναι 3ου βαθμού τότε το άθροισμά τους $ A(x)+B(x) $ είναι πολυώνυμο 3ου βαθμού.
\end{rlist}\monades{3}\\
Β. Να επιλέξετε τη σωστή απάντηση σε κάθεμία από τις παρακάτω ερωτήσεις.
\begin{rlist}
\item Ποιά από τις παρακάτω σχέσεις αποτελεί ταυτότητα;
\begin{multicols}{4}
\begin{itemize}
\item $ x+y=2 $
\item $ x^3+x^2=x^5 $
\item $ x+x=2x $
\item $ (x-1)^2=x^2-1 $
\end{itemize}
\end{multicols}
\item Ποιό από τα παρακάτω πολυώνυμα είναι το ανάπτυγμα της παράστασης $ (2x+1)^2 $;
\begin{multicols}{4}
\begin{itemize}
\item $ 4x^2+4x+1 $
\item $ 4x^2+1 $
\item $ 2x^2+4x+1 $
\item $ 4x^2-4x+1 $
\end{itemize}
\end{multicols}
\item Ποιό από τα παρακάτω πολυώνυμα είναι το ανάπτυγμα του γινομένου $ (3x-2)(2-3x) $;
\begin{multicols}{4}
\begin{itemize}
\item $ 4-9x^2 $
\item $ -9x^2+12x-4 $
\item $ 9x^2-12x+4 $
\item $ 9x^2-4 $
\end{itemize}
\end{multicols}
\item Ποιό από τα παρακάτω πολυώνυμα είναι το ανάπτυγμα της παράστασης $ (y-3)^3 $;
\begin{multicols}{4}
\begin{itemize}
\item $ y^3-3y^2+3y-1 $
\item $ y^3-9y^2+27y-27 $
\item $ y^3-9y^2-27y+27 $
\item $ y^3-27 $
\end{itemize}
\end{multicols}
\end{rlist}\monades{3}
\end{thema}
\newpage
\noindent
\askhseis
\begin{thema}
\item \textbf{Ταυτότητες}\\
Έστω $ x,y $ δύο πραγματικοί αριθμοί για τους οποίους ισχύουν οι σχέσεις $ \frac{x}{y}+\frac{y}{x}=2 $ και $ \frac{x}{y}-\frac{y}{x}=3 $.
\begin{rlist}
\item Να αποδειχθεί η ταυτότητα $ (a+\beta)^2\cdot(a-\beta)^2+2a^2\beta^2=a^4+\beta^4 $.\monades{4}
\item Να υπολογιστεί η τιμή της παράστασης $ \left( \frac{x}{y}\right)^4+\Big(\frac{y}{x}\Big)^4 $.\monades{3}
\end{rlist}
\item \textbf{Ταυτότητες}\\
Να αναπτύξετε τις παρακάτω παραστάσεις.
\begin{multicols}{2}
\begin{rlist}
\item $ (4x-3)^2 $
\item $ (5x-4)^3 $
\item $ (x^2+x)^2 $
\item $ (2x^2-3y)(3y+2x^2) $
\end{rlist}
\end{multicols}\monades{7}
\item \textbf{Πολλαπλασιασμός πολυωνύμων - Πολυώνυμα}\\
Δίνονται τα πολυώνυμα $ P(x)=x^2-5x $ και $ Q(x)=2x^2-4x+3 $
\begin{rlist}
\item Να βρεθούν οι τιμές $ P(-2),\ P(3) $ και $ Q(-3) $ των πολυωνύμων.\monades{2}
\item Να βρεθεί το γινόμενο $ P(x)\cdot Q(x) $.\monades{3}
\item Να βρεθεί η παράσταση $ P(-2x)+Q(-x) $.\monades{2}
\end{rlist}
\end{thema}
\kaliepityxia
\end{document}

