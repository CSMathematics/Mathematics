\documentclass[internet]{diag-xelatex}
\usepackage[amsbb]{mtpro2}
\usepackage[no-math,cm-default]{fontspec}
\usepackage{xunicode}
\usepackage{xltxtra}
\usepackage{xgreek}
\usepackage{amsmath}
\defaultfontfeatures{Mapping=tex-text,Scale=MatchLowercase}
\setmainfont[Mapping=tex-text,Numbers=Lining,Scale=1.0,BoldFont={Minion Pro Bold}]{Minion Pro}
\newfontfamily\scfont{GFS Artemisia}
\font\icon = "Webdings"
\usepackage[amsbb]{mtpro2}
\usepackage[left=2.00cm, right=2.00cm, top=2.00cm, bottom=3.00cm]{geometry}
\xroma{red!80!black}
\newcommand{\tss}[1]{\textsuperscript{#1}}
\newcommand{\tssL}[1]{\MakeLowercase\textsuperscript{#1}}
\newlist{rlist}{enumerate}{3}
\setlist[rlist]{itemsep=0mm,label=\roman*.}
\newlist{brlist}{enumerate}{3}
\setlist[brlist]{itemsep=0mm,label=\bf\roman*.}
\newlist{tropos}{enumerate}{3}
\setlist[tropos]{label=\bf\textit{\arabic*\textsuperscript{oς}\;Τρόπος :},leftmargin=0cm,itemindent=2.3cm,ref=\bf{\arabic*\textsuperscript{oς}\;Τρόπος}}
\usepackage{multicol}
%------- ΣΥΣΤΗΜΑ -------------------
\usepackage{systeme,regexpatch}
\makeatletter
% change the definition of \sysdelim not to store `\left` and `\right`
\def\sysdelim#1#2{\def\SYS@delim@left{#1}\def\SYS@delim@right{#2}}
\sysdelim\{. % reinitialize

% patch the internal command to use
% \LEFTRIGHT<left delim><right delim>{<system>}
% instead of \left<left delim<system>\right<right delim>
\regexpatchcmd\SYS@systeme@iii
{\cB.\c{SYS@delim@left}(.*)\c{SYS@delim@right}\cE.}
{\c{SYS@MT@LEFTRIGHT}\cB\{\1\cE\}}
{}{}
\def\SYS@MT@LEFTRIGHT{%
\expandafter\expandafter\expandafter\LEFTRIGHT
\expandafter\SYS@delim@left\SYS@delim@right}
\makeatother
\newcommand{\synt}[2]{{\scriptsize \begin{matrix}
\times#1\\\\ \times#2
\end{matrix}}}
%----------------------------------------



\begin{document}
\titlos{ΜΑΘΗΜΑΤΙΚΑ Γ΄ ΓΥΜΝΑΣΙΟΥ}{ΣΥΣΤΗΜΑΤΑ}
\thewria
\begin{thema}
\item \textbf{ΕΡΩΤΗΣΕΙΣ ΑΝΑΠΤΥΞΗΣ}\\
Να απαντήσετε στις παρακάτω ερωτήσεις.
\begin{rlist}
\item Για ποιές τιμές των συντελεστών $ a,\beta $ η γραμμική εξίσωση $ ax+\beta y=\gamma $ παριστάνει ευθεία γραμμή;
\item Τι ονομάζεται γραμμικό σύστημα;
\item Ποιές μεθόδους αλγεβρικής επίλυσης ενός γραμμικού συστήματος γνωρίζετε;
\item Τι ονομάζουμε λύση ενός γραμμικού συστήματος;
\end{rlist}\monades{6}
\item \textbf{A : ΣΩΣΤΟ - ΛΑΘΟΣ}\\
Να χαρακτηρίσετε τις παρακάτω προτάσεις ως σωστές (Σ) ή λανθασμένες (Λ).
\begin{rlist}
\item Το σημείο $ A(-2,1) $ ανήκει στην ευθεία $ 3x-y=4 $.
\item Η εξίσωση $ x=4 $ παριστάνει ευθεία παράλληλη με τον κατακόρυφο άξονα $ y'y $.
\item Ένα γραμμικό σύστημα έχει πάντα λύση.
\item 
\end{rlist}\monades{3}\\
\textbf{Β : ΠΟΛΛΑΠΛΗΣ ΕΠΙΛΟΓΗΣ}\\
Να επιλέξετε τη σωστή απάντηση σε καθεμια από τις παρακάτω προτάσεις.
\end{thema}
\newpage
\noindent
\askhseis
\begin{thema}
\item \textbf{ΓΡΑΦΙΚΗ ΕΠΙΛΥΣΗ}\\
Να λυθούν γραφικά τα παρακάτω γραμμικά συστήματα.
\begin{multicols}{2}
\begin{rlist}
\item $ \systeme{3x+y=6,2x-4y=-10} $
\item $ \systeme{x-4y=-7,3x+y=5} $
\end{rlist}
\end{multicols}\monades{7}
\item \textbf{ΑΛΓΕΒΡΙΚΗ ΕΠΙΛΥΣΗ}\\
Να λυθούν αλγεβρικά τα παρακάτω γραμμικά συστήματα με οποιαδήποτε μέθοδο.
\begin{multicols}{2}
\begin{rlist}
\item $ \systeme{2x+5y=-1,3x+4y=2} $
\item $ \systeme{4x-3y=2,x+7y=16} $
\end{rlist}
\end{multicols}\monades{7}
\item \textbf{ΣΥΝΘΕΤΟ ΣΥΣΤΗΜΑ}\\
Να λυθεί το παρακάτω γραμμικό σύστημα με οποιαδήποτε μέθοδο γραφική ή αλγεβρική.
\[ \ccases{\dfrac{2x-y}{3}+\dfrac{x+2}{4}=1-\dfrac{2y-1}{12}\\[3mm]
\dfrac{x-3}{4}+\dfrac{y+2}{5}=\dfrac{x-3y}{20}+\dfrac{1}{2}} \]
\end{thema}
\kaliepityxia
\end{document}
