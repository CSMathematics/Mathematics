\documentclass[11pt,a4paper]{article}
\usepackage[english,greek]{babel}
\usepackage[utf8]{inputenc}
\usepackage{nimbusserif}
\usepackage[T1]{fontenc}
\usepackage[left=2.00cm, right=2.00cm, top=2.00cm, bottom=2.00cm]{geometry}
\usepackage{amsmath}
\let\myBbbk\Bbbk
\let\Bbbk\relax
\usepackage[amsbb,subscriptcorrection,zswash,mtpcal,mtphrb,mtpfrak]{mtpro2}
\usepackage{siunitx, graphicx,multicol,multirow,enumitem,tabularx,mathimatika,gensymb,venndiagram,hhline,longtable,tkz-euclide,fontawesome5,eurosym,tcolorbox,wrap-rl}
\tcbuselibrary{skins,theorems,breakable}
\newlist{rlist}{enumerate}{3}
\setlist[rlist]{itemsep=0mm,label=\roman*.}
\newlist{alist}{enumerate}{3}
\setlist[alist]{itemsep=0mm,label=\alph*.}
\newlist{balist}{enumerate}{3}
\setlist[balist]{itemsep=0mm,label=\bf\alph*.}
\newlist{Alist}{enumerate}{3}
\setlist[Alist]{itemsep=0mm,label=\Alph*.}
\newlist{bAlist}{enumerate}{3}
\setlist[bAlist]{itemsep=0mm,label=\bf\Alph*.}
\renewcommand{\textstigma}{\textsigma\texttau}
\newlist{thema}{enumerate}{3}
\setlist[thema]{label=\bf\large{ΘΕΜΑ \textcolor{black}{\Alph*}},itemsep=0mm,leftmargin=0cm,itemindent=18mm}
\newlist{erwthma}{enumerate}{3}
\setlist[erwthma]{label=\bf{\large{\textcolor{black}{\Alph{themai}.\arabic*}}},itemsep=0mm,leftmargin=0.8cm}

\newcommand{\lysh}{\textcolor{black}{\textbf{\faCheck\ \ ΛΥΣΗ}}}
\renewcommand{\textstigma}{\textsigma\texttau}
%----------- ΟΡΙΣΜΟΣ------------------
\newcounter{orismos}[section]
\renewcommand{\theorismos}{\thesection.\arabic{orismos}}   
\newcommand{\Orismos}{\refstepcounter{orismos}{\textbf{\textcolor{black}{\kerkissans{Ορισμός\hspace{2mm}\theorismos}}\;:\;}{}}}

\newenvironment{orismos}[1]
{\begin{tcolorbox}[title=\Orismos {\textcolor{black}{\kerkissans{#1}}},breakable,bottomtitle=-1.5mm,
enhanced standard,titlerule=-.2pt,toprule=0pt, rightrule=0pt, bottomrule=0pt,
colback=white,left=2mm,top=1mm,bottom=0mm,
boxrule=0pt,
colframe=white,borderline west={1.5mm}{0pt}{black},leftrule=2mm,sharp corners,coltitle=black]}
{\end{tcolorbox}}

\newcommand{\kerkissans}[1]{{\fontfamily{maksf}\selectfont \textbf{#1}}}
\renewcommand{\textdexiakeraia}{}

\usepackage[
backend=biber,
style=alphabetic,
sorting=ynt
]{biblatex}

\begin{document}
\begin{center}
{\LARGE \kerkissans{Μαθηματικά Γ' Γυμνασίου}}\\
{\large \kerkissans{Ασκήσεις - Ταυτότητες\\\today}}
\end{center}
\begin{enumerate}
\item Τι ονομάζεται ταυτότητα?
\item Να βρείτε τα αναπτύγματα των παρακάτω ταυτοτήτων
\begin{multicols}{3}
\begin{enumerate}[label=\roman*.]
\item $ (x+3)^2 $
\item $ (3x-2y)^2 $
\item $ \left(\sqrt{x}+\sqrt{y}\right)^2 $
\item $ \left(\dfrac{x}{4}-2 \right)^2 $
\end{enumerate}
\end{multicols}
\item Να βρείτε τα αναπτύγματα των παρακάτω ταυτοτήτων
\begin{multicols}{2}
\begin{enumerate}[label=\roman*.]
\item $ (x+2)^3 $
\item $ (3x-4y)^3 $
\end{enumerate}
\end{multicols}
\item Να βρείτε τα αναπτύγματα των παρακάτω ταυτοτήτων
\begin{multicols}{2}
\begin{enumerate}[label=\roman*.]
\item $ (x+4)(x-4) $
\item $ \left(\sqrt{x}+\sqrt{y}\right)\left(\sqrt{x}-\sqrt{y}\right) $
\item $ \left(\dfrac{x}{2}-\dfrac{y}{3}\right)\left(\dfrac{x}{2}+\dfrac{y}{3}\right) $
\end{enumerate}
\end{multicols}
\item Να μετατρέψετε τα παρακάτω κλάσματα που έχουν άρρητο παρονομαστή σε ισοδύναμα κλάσματα με ρητό παρονομαστή.
\begin{multicols}{2}
\begin{enumerate}[label=\roman*.]
\item $ \dfrac{1}{\sqrt{5}-1} $
\item $ \dfrac{2}{\sqrt{7}+\sqrt{3}} $
\end{enumerate}
\end{multicols}
\item \begin{enumerate}
\item[$ a $.] Να αποδείξετε την παρακάτω ταυτότητα \[ (x+y)^2-(x-y)^2=4xy \]
\item[$ \beta $.] Να βρείτε την τιμή της παράστασης $ \left(2023+\dfrac{1}{2023}\right)^2-\left(2023-\dfrac{1}{2023}\right)^2 $.
\end{enumerate}
\end{enumerate}
\mbox{}\\
\end{document}