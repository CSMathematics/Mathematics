\PassOptionsToPackage{no-math,cm-default}{fontspec}
\documentclass[twoside,nofonts,internet,shmeiwseis]{thewria}
\usepackage{amsmath}
\usepackage{xgreek}
\let\hbar\relax
\defaultfontfeatures{Mapping=tex-text,Scale=MatchLowercase}
\setmainfont[Mapping=tex-text,Numbers=Lining,Scale=1.0,BoldFont={Minion Pro Bold}]{Minion Pro}
\newfontfamily\scfont{GFS Artemisia}
\font\icon = "Webdings"
\usepackage[amsbb,subscriptcorrection,zswash,mtpcal,mtphrb]{mtpro2}
\usepackage{tikz,pgfplots}
\tkzSetUpPoint[size=7,fill=white]
\xroma{red!70!black}
%------TIKZ - ΣΧΗΜΑΤΑ - ΓΡΑΦΙΚΕΣ ΠΑΡΑΣΤΑΣΕΙΣ ----
\usepackage{tikz}
\usepackage{tkz-euclide}
\usetkzobj{all}
\usepackage[framemethod=TikZ]{mdframed}
\usetikzlibrary{decorations.pathreplacing}
\usepackage{pgfplots}
\usetkzobj{all}
%-----------------------
\usepackage{calc}
\usepackage{hhline}
\usepackage[explicit]{titlesec}
\usepackage{graphicx}
\usepackage{multicol}
\usepackage{multirow}
\usepackage{enumitem}
\usepackage{tabularx}
\usepackage[decimalsymbol=comma]{siunitx}
\usetikzlibrary{backgrounds}
\usepackage{sectsty}
\sectionfont{\centering}
\setlist[enumerate]{label=\bf{\large \arabic*.}}
\usepackage{adjustbox}
\usepackage{mathimatika,gensymb,eurosym,wrap-rl}
\usepackage{systeme,regexpatch}
%-------- ΜΑΘΗΜΑΤΙΚΑ ΕΡΓΑΛΕΙΑ ---------
\usepackage{mathtools}
%----------------------
%-------- ΠΙΝΑΚΕΣ ---------
\usepackage{booktabs}
%----------------------
%----- ΥΠΟΛΟΓΙΣΤΗΣ ----------
\usepackage{calculator}
%----------------------------
%------ ΔΙΑΓΩΝΙΟ ΣΕ ΠΙΝΑΚΑ -------
\usepackage{array}
\newcommand\diag[5]{%
\multicolumn{1}{|m{#2}|}{\hskip-\tabcolsep
$\vcenter{\begin{tikzpicture}[baseline=0,anchor=south west,outer sep=0]
\path[use as bounding box] (0,0) rectangle (#2+2\tabcolsep,\baselineskip);
\node[minimum width={#2+2\tabcolsep-\pgflinewidth},
minimum  height=\baselineskip+#3-\pgflinewidth] (box) {};
\draw[line cap=round] (box.north west) -- (box.south east);
\node[anchor=south west,align=left,inner sep=#1] at (box.south west) {#4};
\node[anchor=north east,align=right,inner sep=#1] at (box.north east) {#5};
\end{tikzpicture}}\rule{0pt}{.71\baselineskip+#3-\pgflinewidth}$\hskip-\tabcolsep}}
%---------------------------------
%---- ΟΡΙΖΟΝΤΙΟ - ΚΑΤΑΚΟΡΥΦΟ - ΠΛΑΓΙΟ ΑΓΚΙΣΤΡΟ ------
\newcommand{\orag}[3]{\node at (#1)
{$ \overcbrace{\rule{#2mm}{0mm}}^{{\scriptsize #3}} $};}
\newcommand{\kag}[3]{\node at (#1)
{$ \undercbrace{\rule{#2mm}{0mm}}_{{\scriptsize #3}} $};}
\newcommand{\Pag}[4]{\node[rotate=#1] at (#2)
{$ \overcbrace{\rule{#3mm}{0mm}}^{{\rotatebox{-#1}{\scriptsize$#4$}}}$};}
%-----------------------------------------
%------------------------------------------
\newcommand{\tss}[1]{\textsuperscript{#1}}
\newcommand{\tssL}[1]{\MakeLowercase{\textsuperscript{#1}}}
%---------- ΛΙΣΤΕΣ ----------------------
\newlist{bhma}{enumerate}{3}
\setlist[bhma]{label=\bf\textit{\arabic*\textsuperscript{o}\;Βήμα :},leftmargin=0cm,itemindent=1.8cm,ref=\bf{\arabic*\textsuperscript{o}\;Βήμα}}
\newlist{rlist}{enumerate}{3}
\setlist[rlist]{itemsep=0mm,label=\roman*.}
\newlist{brlist}{enumerate}{3}
\setlist[brlist]{itemsep=0mm,label=\bf\roman*.}
\newlist{tropos}{enumerate}{3}
\setlist[tropos]{label=\bf\textit{\arabic*\textsuperscript{oς}\;Τρόπος :},leftmargin=0cm,itemindent=2.3cm,ref=\bf{\arabic*\textsuperscript{oς}\;Τρόπος}}
% Αν μπει το bhma μεσα σε tropo τότε
%\begin{bhma}[leftmargin=.7cm]
\tkzSetUpPoint[size=7,fill=white]
\tikzstyle{pl}=[line width=0.3mm]
\tikzstyle{plm}=[line width=0.4mm]
\usepackage{etoolbox}
\makeatletter
\renewrobustcmd{\anw@true}{\let\ifanw@\iffalse}
\renewrobustcmd{\anw@false}{\let\ifanw@\iffalse}\anw@false
\newrobustcmd{\noanw@true}{\let\ifnoanw@\iffalse}
\newrobustcmd{\noanw@false}{\let\ifnoanw@\iffalse}\noanw@false
\renewrobustcmd{\anw@print}{\ifanw@\ifnoanw@\else\numer@lsign\fi\fi}
\makeatother

\begin{document}
\titlos{Μαθηματικά Γ΄ Γυμνασίου}{Εξισώσεις - Ανισώσεις}{Ανισότητες - Ανισώσεις}
\orismoi
\Orismos{Διάταξη}
Διάταξη ονομάζεται η ιδιότητα του συνόλου των πραγματικών αριθμών κατά την οποία μπορούμε να τους συγκρίνουμε και να τους τοποθετήσουμε σε αύξουσα ή φθίνουσα σειρά. Οι σχέσεις διάταξης που χρησιμοποιούμε είναι
\begin{center}
$ < $ : μικρότερο  \;,\;  $ > $ : μεγαλύτερο  \;,\; $ \leq $  μικρότερο ίσο \;,\; $ \geq $  μεγαλύτερο ισο  
\end{center}
Δύο ή περισσότεροι αριθμοί που είναι τοποθετημένοι πάνω στην ευθεία των πραγμωτικών αριθμών ονομάζονται \textbf{διατεταγμένοι}.\mbox{}\\\\
\Orismos{Μεγαλύτεροσ - Μικρότεροσ}
Ένας αριθμός $ a $ είναι \textbf{μεγαλύτερος} απο έναν αριθμό $ \beta $, και γράφουμε $a>\beta$, όταν η διαφορά $ a-\beta$ είναι θετικός αριθμός.
\[ a>\beta\Leftrightarrow a-\beta>0 \]
Ένας αριθμός $ a $ είναι \textbf{μικρότερος} απο έναν αριθμό $ \beta $, και γράφουμε $a<\beta$, όταν η διαφορά $a-\beta$ είναι αρνητικός αριθμός.
\[ a<\beta\Leftrightarrow a-\beta<0 \]
\Orismos{Ανίσωση}
Ανίσωση ονομάζεται κάθε ανισότητα η οποία περιέχει τουλάχιστον μια μεταβλητή, κάθε σχέση της μορφής :
\[ P(x,y,\ldots,z)>0\;\;,\;\;P(x,y,\ldots,z)<0 \]
όπου $ P(x,y,\ldots,z) $ είναι μια αλγεβρική παράσταση πολλών μεταβλητών.
\begin{itemize}[itemsep=0mm]
\item Ανισώσεις αποτελούν και οι σχέσεις με σύμβολα ανισοϊσότητας $ \leq,\geq $.
\item Κάθε αριθμός που επαληθεύει μια ανίσωση ονομάζεται \textbf{λύση} της. Κάθε ανίσωση έχει λύσεις ένα \textbf{σύνολο αριθμών}.
\item Αν μια ανίσωση έχει λύσεις όλους τους αριθμούς ονομάζεται \textbf{αόριστη}.
\item Αν μια ανίσωση δεν έχει καθόλου λύσεις ονομάζεται \textbf{αδύνατη}.
\item Σχέσεις τις μορφής $ Q(x)\leq P(x)\leq R(x) $ λέγονται \textbf{διπλές ανισώσεις} όπου $ P(x),Q(x),R(x) $ αλγεβρικές παρατάσεις. Αποτελείται από δύο ανισώσεις, με κοινό μέλος την παράσταση $ P(x) $, οι οποίες συναληθεύουν.
\item \textbf{Κοινές λύσεις} μιας διπλής ανίσωσης ή δύο ή περισσότερων ανισώσεων ονομάζονται οι αριθμοί που επαληθεύουν όλες τις ανισώσεις συγχρόνως.
\end{itemize}
\Orismos{ανισωση 1\textsuperscript{\MakeLowercase{ου}} βαθμου}
Ανίσωση 1\textsuperscript{ου} βαθμού με έναν άγνωστο ονομάζεται κάθε πολυωνυμική ανίσωση της οποίας η αλγεβρική παράσταση είναι πολυώνυμο 1\textsuperscript{ου} βαθμού. Είναι της μορφής :
\[ ax+\beta>0\;\;,\;\;ax+\beta<0 \] με πραγματικούς συντελεστές $ a,\beta $.\\\\
\thewrhmata
\Thewrhma{Ιδιότητες Διάταξης}
\vspace{-5mm}
\begin{enumerate}
\item Εαν σε μια ανισότητα προσθέσουμε ή αφαιρέσουμε τον ίδιο αριθμό και απ' τα δύο μέλη της, προκύπτει ξανά ανισότητα με την ίδια φορά της αρχικής.
\[ a>\beta\Leftrightarrow\ccases{a+\gamma>\beta+\gamma\\α-\gamma>\beta-\gamma} \]
\item Για να πολλαπλασιάσουμε ή να διαιρέσουμε και τα δύο μέλη μιας ανισότητας με τον ίδιο αριθμό διακρίνουμε τις εξής περπτώσεις :
\begin{rlist}
\item Εαν πολλαπλασιάσουμε ή διαιρέσουμε και τα δύο μέλη μιας ανισότητας με τον ίδιο \textbf{θετικό} αριθμό, τότε προκύπτει ανισότητα με την \textbf{ίδια} φορά της αρχικής.
\item Εαν πολλαπλασιάσουμε ή διαιρέσουμε και τα δύο μέλη μιας ανισότητας με τον ίδιο \textbf{αρνητικό} αριθμό, τότε προκύπτει ανισότητα με φορά \textbf{αντίθετη} της αρχικής.
\end{rlist}
\begin{gather*}
\textrm{Αν }\gamma>0\textrm{ τότε }a>\beta\Leftrightarrow a\cdot\gamma>\beta\cdot\gamma\textrm{ και }\dfrac{a}{\gamma}>\dfrac{\beta}{\gamma}\\
\textrm{Αν }\gamma<0\textrm{ τότε }a>\beta\Leftrightarrow a\cdot\gamma<\beta\cdot\gamma\textrm{ και }\dfrac{a}{\gamma}<\dfrac{\beta}{\gamma}
\end{gather*}
\end{enumerate}
Ανάλογα συμπεράσματα ισχύουν και για τις ανισότητες $ a<\beta,a\geq\beta $ και $ a\leq\beta $.\\\\
\Thewrhma{Πράξεισ κατά μέλη ανισοτήτων}
Μπορούμε να προσθέτουμε κατά μέλη κάθε ζεύγος ανισοτήτων με ίδια φορά και να πολλαπλασιάσουμε κατά μέλη δύο ανισότητες ίδιας φοράς αρκεί όλοι οι όροι τους να είναι θετικοί.
\[ a>\beta\;\;\textrm{και}\;\;\gamma>\delta\Rightarrow\begin{cases}
\textrm{\textbf{{1. Πρόσθεση κατά μέλη }}}& a+\gamma>\beta+\delta\\\textrm{\textbf{{2. Πολλαπλασιασμός κατά μέλη }}}& a\cdot\gamma>\beta\cdot\delta\;\;,\;\;\textrm{με }a,\beta,\gamma,\delta>0
\end{cases} \]
\textbf{Δεν} μπορούμε να αφαιρέσουμε ή να διαιρέσουμε ανισότητες κατά μέλη.\\\\
\Thewrhma{Δύναμη με άρτιο εκθέτη}
Το τετράγωνο κάθε πραγματικού αριθμού $ a $ είναι μη αρνητικός αριθμός :
\[ a^2\geq0\;\;,\;\;\kappa\ \textrm{ακέραιος} \]
Αν για δύο πραγματικούς αριθμούς $ a,\beta $ ισχύει $ a^2+\beta^2=0 $ τότε $ a=0 $ και $ \beta=0 $.
\end{document}
