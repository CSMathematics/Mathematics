\PassOptionsToPackage{no-math,cm-default}{fontspec}
\documentclass[twoside,nofonts,internet,shmeiwseis]{thewria}
\usepackage{amsmath}
\usepackage{xgreek}
\let\hbar\relax
\defaultfontfeatures{Mapping=tex-text,Scale=MatchLowercase}
\setmainfont[Mapping=tex-text,Numbers=Lining,Scale=1.0,BoldFont={Minion Pro Bold}]{Minion Pro}
\newfontfamily\scfont{GFS Artemisia}
\font\icon = "Webdings"
\usepackage[amsbb,subscriptcorrection,zswash,mtpcal,mtphrb]{mtpro2}
\usepackage{tikz,pgfplots}
\tkzSetUpPoint[size=7,fill=white]
\xroma{red!70!black}
%------TIKZ - ΣΧΗΜΑΤΑ - ΓΡΑΦΙΚΕΣ ΠΑΡΑΣΤΑΣΕΙΣ ----
\usepackage{tikz}
\usepackage{tkz-euclide}
\usetkzobj{all}
\usepackage[framemethod=TikZ]{mdframed}
\usetikzlibrary{decorations.pathreplacing}
\usepackage{pgfplots}
\usetkzobj{all}
%-----------------------
\usepackage{calc}
\usepackage{hhline}
\usepackage[explicit]{titlesec}
\usepackage{graphicx}
\usepackage{multicol}
\usepackage{multirow}
\usepackage{enumitem}
\usepackage{tabularx}
\usepackage[decimalsymbol=comma]{siunitx}
\usetikzlibrary{backgrounds}
\usepackage{sectsty}
\sectionfont{\centering}
\setlist[enumerate]{label=\bf{\large \arabic*.}}
\usepackage{adjustbox}
\usepackage{mathimatika,gensymb,eurosym,wrap-rl}
\usepackage{systeme,regexpatch}
%-------- ΜΑΘΗΜΑΤΙΚΑ ΕΡΓΑΛΕΙΑ ---------
\usepackage{mathtools}
%----------------------
%-------- ΠΙΝΑΚΕΣ ---------
\usepackage{booktabs}
%----------------------
%----- ΥΠΟΛΟΓΙΣΤΗΣ ----------
\usepackage{calculator}
%----------------------------
%------ ΔΙΑΓΩΝΙΟ ΣΕ ΠΙΝΑΚΑ -------
\usepackage{array}
\newcommand\diag[5]{%
\multicolumn{1}{|m{#2}|}{\hskip-\tabcolsep
$\vcenter{\begin{tikzpicture}[baseline=0,anchor=south west,outer sep=0]
\path[use as bounding box] (0,0) rectangle (#2+2\tabcolsep,\baselineskip);
\node[minimum width={#2+2\tabcolsep-\pgflinewidth},
minimum  height=\baselineskip+#3-\pgflinewidth] (box) {};
\draw[line cap=round] (box.north west) -- (box.south east);
\node[anchor=south west,align=left,inner sep=#1] at (box.south west) {#4};
\node[anchor=north east,align=right,inner sep=#1] at (box.north east) {#5};
\end{tikzpicture}}\rule{0pt}{.71\baselineskip+#3-\pgflinewidth}$\hskip-\tabcolsep}}
%---------------------------------
%---- ΟΡΙΖΟΝΤΙΟ - ΚΑΤΑΚΟΡΥΦΟ - ΠΛΑΓΙΟ ΑΓΚΙΣΤΡΟ ------
\newcommand{\orag}[3]{\node at (#1)
{$ \overcbrace{\rule{#2mm}{0mm}}^{{\scriptsize #3}} $};}
\newcommand{\kag}[3]{\node at (#1)
{$ \undercbrace{\rule{#2mm}{0mm}}_{{\scriptsize #3}} $};}
\newcommand{\Pag}[4]{\node[rotate=#1] at (#2)
{$ \overcbrace{\rule{#3mm}{0mm}}^{{\rotatebox{-#1}{\scriptsize$#4$}}}$};}
%-----------------------------------------
%------------------------------------------
\newcommand{\tss}[1]{\textsuperscript{#1}}
\newcommand{\tssL}[1]{\MakeLowercase{\textsuperscript{#1}}}
%---------- ΛΙΣΤΕΣ ----------------------
\newlist{bhma}{enumerate}{3}
\setlist[bhma]{label=\bf\textit{\arabic*\textsuperscript{o}\;Βήμα :},leftmargin=0cm,itemindent=1.8cm,ref=\bf{\arabic*\textsuperscript{o}\;Βήμα}}
\newlist{rlist}{enumerate}{3}
\setlist[rlist]{itemsep=0mm,label=\roman*.}
\newlist{brlist}{enumerate}{3}
\setlist[brlist]{itemsep=0mm,label=\bf\roman*.}
\newlist{tropos}{enumerate}{3}
\setlist[tropos]{label=\bf\textit{\arabic*\textsuperscript{oς}\;Τρόπος :},leftmargin=0cm,itemindent=2.3cm,ref=\bf{\arabic*\textsuperscript{oς}\;Τρόπος}}
% Αν μπει το bhma μεσα σε tropo τότε
%\begin{bhma}[leftmargin=.7cm]
\tkzSetUpPoint[size=7,fill=white]
\tikzstyle{pl}=[line width=0.3mm]
\tikzstyle{plm}=[line width=0.4mm]
\usepackage{etoolbox}
\makeatletter
\renewrobustcmd{\anw@true}{\let\ifanw@\iffalse}
\renewrobustcmd{\anw@false}{\let\ifanw@\iffalse}\anw@false
\newrobustcmd{\noanw@true}{\let\ifnoanw@\iffalse}
\newrobustcmd{\noanw@false}{\let\ifnoanw@\iffalse}\noanw@false
\renewrobustcmd{\anw@print}{\ifanw@\ifnoanw@\else\numer@lsign\fi\fi}
\makeatother

\begin{document}
\titlos{Μαθηματικά Γ΄ Γυμνασίου}{Τριγωνομετρία}{Τριγωνομετρικοί αριθμοί γωνίας {\MakeLowercase{$ \mathbold{\omega} $}} με {\MakeLowercase{$ \mathbold{0\degree\leq\omega\leq 180\degree} $}}}
\orismoi
\Orismos{Τριγωνομετρικοί αριθμοί}
Έστω $ AB\varGamma $ ένα ορθογώνιο τρίγωνο, με $ A=90\degree $ τότε οι τριγωνομετρικοί αριθμοί των οξείων γωνιών του τριγώνου ορίζονται ως εξής :\\
\begin{minipage}{\linewidth}\mbox{}\\
\vspace{-1cm}
\begin{WrapText1}{12}{3.3cm}
\vspace{0mm}
\begin{tikzpicture}[scale=.8]
\tkzDefPoint(0,0){A}
\tkzDefPoint(3,0){B}
\tkzDefPoint(0,4){C}
\tkzMarkAngle[fill=\xrwma!50,size=.5](C,B,A)
\tkzMarkRightAngle[size=.3](B,A,C)
\tkzDrawPolygon[pl](A,B,C)
\tkzText(2.2,.2){$ \omega $}
\tkzLabelPoint[left](A){$ A $}
\tkzLabelPoint[right](B){$ B $}
\tkzLabelPoint[left](C){$ \varGamma $}
\tkzDrawPoints[size=7,fill=white](A,B,C)
\end{tikzpicture}
\end{WrapText1}
\begin{enumerate}[itemsep=0mm,label=\bf\arabic*.]
\item \textbf{Ημίτονο}\\
Ημίτονο μιας οξέιας γωνίας ενός ορθογωνίου τριγώνου ονομάζεται ο λόγος της απέναντι κάθετης πλευράς προς την υποτείνουσα.
\[ \textrm{Ημίτονο}=\frac{\textrm{Απέναντι Κάθετη}}{\textrm{Υποτείνουσα}}\;\;,\;\;\hm{\omega}=\frac{A\varGamma}{B\varGamma} \]
\item \textbf{Συνημίτονο}\\
Συνημίτονο μιας οξέιας γωνίας ενός ορθογωνίου τριγώνου ονομάζεται ο λόγος της προσκείμενης κάθετης πλευράς προς την υποτείνουσα.
\[ \textrm{Συνημίτονο}=\frac{\textrm{Προσκείμενη Κάθετη}}{\textrm{Υποτείνουσα}}\;\;,\;\;\syn{\omega}=\frac{AB}{B\varGamma} \]
\end{enumerate}

\begin{enumerate}[itemsep=0mm,label=\bf\arabic*.,start=3]
\item \textbf{Εφαπτομένη}\\
Εφαπτομένη μιας οξέιας γωνίας ενός ορθογωνίου τριγώνου ονομάζεται ο λόγος της απέναντι κάθετης πλευράς προς την προσκείμενη κάθετη.
\[ \textrm{Εφαπτομένη}=\frac{\textrm{Απέναντι Κάθετη}}{\textrm{Προσκείμενη Κάθετη}}\;\;,\;\;\ef{\omega}=\frac{A\varGamma}{AB} \]
\end{enumerate}
\end{minipage}\mbox{}\\\\\\
\Orismos{τριγ. αρ. γωνιασ σε συστημα συντεταγμενων}
Έστω $ Oxy $ ένα ορθογώνιο σύστημα συντεταγμένων και $ M(x,y) $ ένα σημείο του. Ενώνοντας το σημείο $ M $ με την αρχή των αξόνων, το ευθύγραμμο τμήμα που προκύπτει δημιουργεί μια γωνία $ \omega $ με το θετικό οριζόντιο ημιάξονα $ Ox $.
Το μήκος του ευθύγραμμου τμήματος $ OM $ είναι :
\[ OM=\rho=\sqrt{x^2+y^2} \]
Οι τριγωνομετρικοί αριθμοί της γωνίας $ x\hat{O}y $ ορίζονται με τη βοήθεια των συντεταγμένων του σημείου και είναι\\
\begin{minipage}{\linewidth}\mbox{}\\
\vspace{-1cm}
\begin{WrapText1}{14}{4.7cm}
\vspace{.5cm}
\begin{tikzpicture}[y=.8cm,x=.9cm]
\draw[draw=black,fill=\xrwma!50] (0,0) -- (.5,0) arc (0:40:.5) -- cycle;
\draw[-latex]  (-.4,0)  -- coordinate (x axis mid) (4,0) node[right,fill=white] {{\footnotesize $ x $}};
\draw[-latex] (0,-.4) -- (0,3.5) node[above,fill=white] {{\footnotesize $ y $}};
\draw (3,.1) -- (3,-.1) node[anchor=north] {\scriptsize $ x $};
\draw (.1,2.5) -- (-.1,2.5) node[anchor=east] {\scriptsize $ y $};
\draw[dashed] (3,0) -- (3,2.5) -- (0,2.5);
\tkzDefPoint(0,0){O}
\tkzDefPoint(3,2.5){M}
\tkzDefPoint(3,0){A}
\tkzDefPoint(0,2.5){B}
\tkzDrawSegment(O,M)
\tkzDrawPoint[size=7,fill=white](M)
\tkzDrawPoint[size=7,fill=white](A)
\tkzDrawPoint[size=7,fill=white](B)
\tkzLabelPoint[below left](O){$ O $}
\tkzLabelPoint[above](M){{\footnotesize $ M(x,y) $}}
\tkzLabelPoint[above right](A){{\footnotesize $ A(x,0) $}}
\tkzLabelPoint[above right](B){{\footnotesize $ B(0,y) $}}
\tkzText(1.5,-.4){$ \undercbrace{\rule{23mm}{0mm}}_{{\scriptsize x}} $}
\tkzText(-.3,1.25){{{\scriptsize $ y $}}$\LEFTRIGHT\{.{ \rule{0pt}{20mm} } $}
\tkzText[fill=white,inner sep=.2mm](2.7,1){{\footnotesize $ \rho=\sqrt{x^2+y^2} $}}
\tkzText(.7,.2){{\footnotesize $ \omega $}}
\end{tikzpicture}\end{WrapText1}
\begin{enumerate}[itemsep=0mm,label=\bf\arabic*.]
\item \textbf{Ημίτονο}\\
Ημίτονο της γωνίας  ονομάζεται ο λόγος της τεταγμένης του σημείου προς την απόσταση του από την αρχή των αξόνων.
\[ \hm{\omega}=\frac{AM}{OM}=\frac{y}{\rho} \]
\item \textbf{Συνημίτονο}\\
Συνημίτονο της γωνίας  ονομάζεται ο λόγος της τετμημένης του σημείου προς την απόσταση του από την αρχή των αξόνων.
\[ \syn{\omega}=\frac{BM}{OM}=\frac{x}{\rho} \]
\end{enumerate}\end{minipage}
\vspace{-3mm}
\begin{enumerate}[itemsep=0mm,label=\bf\arabic*.,start=3]
\item \textbf{Εφαπτομένη}\\
Εφαπτομένη της γωνίας ονομάζεται ο λόγος της τεταγμένης του σημείου προς την τετμημένη του.
\[ \ef{\omega}=\frac{AM}{BM}=\frac{y}{x}\;\;,\;\;x\neq0 \]
\end{enumerate}
Στον παρακάτω πίνακα βλέπουμε το μέτρο μερικών βασικών γωνιών δοσμένο σε μοίρες και ακτίνια αλλά και τους τριγωνομετρικούς αριθμούς των γωνιών αυτών.
\begin{center}
\begin{tabular}{c||>{\centering\arraybackslash}m{.8cm}>{\centering\arraybackslash}m{.8cm}>{\centering\arraybackslash}m{.8cm}>{\centering\arraybackslash}m{.8cm}>{\centering\arraybackslash}m{.8cm}>{\centering\arraybackslash}m{.8cm}}
\hline  \multicolumn{7}{c}{\textbf{Βασικές Γωνίες}} \rule[-2ex]{0pt}{5.5ex}  \\ 
\hhline{=======} \rule[-2ex]{0pt}{5.5ex} \textbf{Μοίρες} & $ 0\degree $ & $ 30\degree $ & $ 45\degree $ & $ 60\degree $ & $ 90\degree $ & $ 180\degree $ \\ 
\rule[-2ex]{0pt}{4ex} \textbf{Ακτίνια} & $ 0 $ & $ \frac{\pi}{6} $ & $ \frac{\pi}{4} $ & $ \frac{\pi}{3} $ & $ \frac{\pi}{2} $ & $ \pi $ \\ 
\hline \rule[-2ex]{0pt}{5.5ex} \textbf{Σχήμα} & \begin{tikzpicture}
\fill[fill=\xrwma!50] (0,0) -- (.3,0) arc (0:0:.3) -- cycle;
\draw (-.35,0) -- (.35,0);
\draw (0,-.35) -- (0,.35);
\draw (0,0) circle (.3);
\coordinate (A) at (0:.3);
\draw (0,0) -- (A);
\end{tikzpicture} & \begin{tikzpicture}
\fill[fill=\xrwma!50] (0,0) -- (.3,0) arc (0:30:.3) -- cycle;
\draw (-.35,0) -- (.35,0);
\draw (0,-.35) -- (0,.35);
\draw (0,0) circle (.3);
\coordinate (A) at (30:.3);
\draw (0,0) -- (A);
\end{tikzpicture} & \begin{tikzpicture}
\fill[fill=\xrwma!50] (0,0) -- (.3,0) arc (0:45:.3) -- cycle;
\draw (-.35,0) -- (.35,0);
\draw (0,-.35) -- (0,.35);
\draw (0,0) circle (.3);
\coordinate (A) at (45:.3);
\draw (0,0) -- (A);
\end{tikzpicture} & \begin{tikzpicture}
\fill[fill=\xrwma!50] (0,0) -- (.3,0) arc (0:60:.3) -- cycle;
\draw (-.35,0) -- (.35,0);
\draw (0,-.35) -- (0,.35);
\draw (0,0) circle (.3);
\coordinate (A) at (60:.3);
\draw (0,0) -- (A);
\end{tikzpicture} & \begin{tikzpicture}
\fill[fill=\xrwma!50] (0,0) -- (.3,0) arc (0:90:.3) -- cycle;
\draw (-.35,0) -- (.35,0);
\draw (0,-.35) -- (0,.35);
\draw (0,0) circle (.3);
\coordinate (A) at (90:.3);
\draw (0,0) -- (A);
\end{tikzpicture} & \begin{tikzpicture}
\fill[fill=\xrwma!50] (0,0) -- (.3,0) arc (0:180:.3) -- cycle;
\draw (-.35,0) -- (.35,0);
\draw (0,-.35) -- (0,.35);
\draw (0,0) circle (.3);
\coordinate (A) at (180:.3);
\draw (0,0) -- (A);
\end{tikzpicture} \\ 
\hline \rule[-2ex]{0pt}{5ex} $ \hm{\omega} $ & $ 0 $ & $ \frac{1}{2} $ & $ \frac{\sqrt{2}}{2} $ & $ \frac{\sqrt{3}}{2} $ & $ 1 $ & $ 0 $  \\ 
\rule[-2ex]{0pt}{4ex} $ \syn{\omega} $ & $ 1 $ & $ \frac{\sqrt{3}}{2} $ & $ \frac{\sqrt{2}}{2} $ & $ \frac{1}{2} $ & $ 0 $ & $ -1 $  \\ 
\rule[-2ex]{0pt}{4ex} $ \ef{\omega} $ & $ 0 $ & $ \frac{\sqrt{3}}{3} $ & $ 1 $ & $ \sqrt{3} $ & \begin{minipage}{.8cm}
\begin{center}
{\scriptsize Δεν\\\vspace{-1mm}ορίζεται}
\end{center}
\end{minipage} & $ 0 $  \\
\hline 
\end{tabular}
\end{center}\mbox{}\\\\
\thewrhmata
\Thewrhma{Πρόσημα τριγωνομετρικών αριθμών}
Τα πρόσημα των τριγωνομετρικών αριθμών μιας γωνίας $ \omega $ εξαρτόνται από το είδος της γωνίας:
\begin{rlist}
\item Αν η γωνία $ \omega $ είναι οξεία τότε $ \hm{\omega}>0,\syn{\omega}>0,\ef{\omega}>0 $.
\item Αν η γωνία $ \omega $ είναι αμβλεία τότε $ \hm{\omega}>0,\syn{\omega}<0,\ef{\omega}>0 $.
\end{rlist}
\end{document}
