\PassOptionsToPackage{no-math,cm-default}{fontspec}
\documentclass[twoside,nofonts,internet,shmeiwseis]{thewria}
\usepackage{amsmath}
\usepackage{xgreek}
\let\hbar\relax
\defaultfontfeatures{Mapping=tex-text,Scale=MatchLowercase}
\setmainfont[Mapping=tex-text,Numbers=Lining,Scale=1.0,BoldFont={Minion Pro Bold}]{Minion Pro}
\newfontfamily\scfont{GFS Artemisia}
\font\icon = "Webdings"
\usepackage[amsbb]{mtpro2}
\usepackage{tikz,pgfplots}
\tkzSetUpPoint[size=7,fill=white]
\xroma{red!70!black}
%------- ΣΥΣΤΗΜΑ -------------------
\usepackage{systeme,regexpatch}
\makeatletter
% change the definition of \sysdelim not to store `\left` and `\right`
\def\sysdelim#1#2{\def\SYS@delim@left{#1}\def\SYS@delim@right{#2}}
\sysdelim\{. % reinitialize

% patch the internal command to use
% \LEFTRIGHT<left delim><right delim>{<system>}
% instead of \left<left delim<system>\right<right delim>
\regexpatchcmd\SYS@systeme@iii
{\cB.\c{SYS@delim@left}(.*)\c{SYS@delim@right}\cE.}
{\c{SYS@MT@LEFTRIGHT}\cB\{\1\cE\}}
{}{}
\def\SYS@MT@LEFTRIGHT{%
\expandafter\expandafter\expandafter\LEFTRIGHT
\expandafter\SYS@delim@left\SYS@delim@right}
\makeatother
\newcommand{\synt}[2]{{\scriptsize \begin{matrix}
\times#1\\\\ \times#2
\end{matrix}}}
%----------------------------------------
%------ ΜΗΚΟΣ ΓΡΑΜΜΗΣ ΚΛΑΣΜΑΤΟΣ ---------
\DeclareRobustCommand{\frac}[3][0pt]{%
{\begingroup\hspace{#1}#2\hspace{#1}\endgroup\over\hspace{#1}#3\hspace{#1}}}
%----------------------------------------

\newlist{rlist}{enumerate}{3}
\setlist[rlist]{itemsep=0mm,label=\roman*.}
\newlist{brlist}{enumerate}{3}
\setlist[brlist]{itemsep=0mm,label=\bf\roman*.}
\newlist{tropos}{enumerate}{3}
\setlist[tropos]{label=\bf\textit{\arabic*\textsuperscript{oς}\;Τρόπος :},leftmargin=0cm,itemindent=2.3cm,ref=\bf{\arabic*\textsuperscript{oς}\;Τρόπος}}
\newcommand{\tss}[1]{\textsuperscript{#1}}
\newcommand{\tssL}[1]{\MakeLowercase{\textsuperscript{#1}}}

\usepackage{hhline}
%----------- ΓΡΑΦΙΚΕΣ ΠΑΡΑΣΤΑΣΕΙΣ ---------
\pgfkeys{/pgfplots/aks_on/.style={axis lines=center,
xlabel style={at={(current axis.right of origin)},xshift=1.5ex, anchor=center},
ylabel style={at={(current axis.above origin)},yshift=1.5ex, anchor=center}}}
\pgfkeys{/pgfplots/grafikh parastash/.style={\xrwma,line width=.4mm,samples=200}}
\pgfkeys{/pgfplots/belh ar/.style={tick label style={font=\scriptsize},axis line style={-latex}}}
%-----------------------------------------
\usepackage{multicol,multirow}
\usepackage{wrap-rl}


\begin{document}
\titlos{Μαθηματικά Γ΄ Γυμνασίου}{Εξισώσεις}{Εξισώσεις \tssL{ου} βαθμού}
\orismoi
\Orismos{εξισωση 1\textsuperscript{\MakeLowercase{ου}} βαθμου}
Εξίσωση 1\textsuperscript{ου} βαθμού με έναν άγνωστο ονομάζεται κάθε πολυωνυμική εξίσωση της οποίας η αλγεβρική παράσταση είναι πολυώνυμο 1\textsuperscript{ου} βαθμού. Είναι της μορφής :
\[ ax+\beta=0 \]
Όπου $ a,\beta\in\mathbb{R} $. Αν ο συντελεστής της μεταβλητής $ x $ είναι διάφορος του 0 τότε η εξίσωση έχει μοναδική λύση την $ x=-\frac{\beta}{a} $. Σε αντίθετη περίπτωση θα είναι είτε αδύνατη είτε αόριστη.\\\\

\thewrhmata
\Thewrhma{λυσεισ εξισωσησ 1\textsuperscript{\MakeLowercase{ου}} βαθμου}
Έστω $ ax+\beta=0 $ μια εξίσωση 1\textsuperscript{ου} βαθμού με $ a,\beta\in\mathbb{R} $ τότε διακρίνουμε τις παρακάτω περιπτώσεις για τις λύσεις της ανάλογα με την τιμή των συντελεστών της $ a,\beta $ :
\begin{enumerate}
\item Αν $ a\neq0 $ τότε η εξίσωση έχει \textbf{μοναδική λύση} την $ x=-\frac{\beta}{a} $.
\item Αν $ a=0 $ και 
\begin{rlist}
\item $ \beta=0 $ τότε η εξίσωση παίρνει τη μορφή $ 0x=0 $ η οποία έχει λύσεις όλους τους αριθμούς οπότε είναι \textbf{αόριστη}.
\item $ \beta\neq0 $ τότε η εξίσωση παίρνει τη μορφή $ 0x=\beta $ η οποία δεν έχει καμία λύση άρα είναι \textbf{αδύνατη}.
\end{rlist}
\end{enumerate}
\begin{center}
\begin{tabular}{c|c|c}
\hline\multicolumn{2}{c}{\textbf{Συντελεστές}} & \textbf{Λύσεις} \rule[-2ex]{0pt}{5.5ex}\\ 
\hhline{===}  \multicolumn{2}{c}{$a\neq0$} &  $ x=-\frac{\beta}{a} $ μοναδική λύση \rule[-2ex]{0pt}{5.5ex}\\ 
\hline\rule[-2ex]{0pt}{5.5ex} \multirow{3}{*}{$a=0$}  & $ \beta=0 $ & $ 0x=0 $ αόριστη - άπειρες λύσεις \\
\hhline{~--} \rule[-2ex]{0pt}{5.5ex}   & $ \beta\neq0 $ & $ 0x=\beta $ αδύνατη - καμία λύση \\ 
\hline 
\end{tabular}
\end{center}
\end{document}
