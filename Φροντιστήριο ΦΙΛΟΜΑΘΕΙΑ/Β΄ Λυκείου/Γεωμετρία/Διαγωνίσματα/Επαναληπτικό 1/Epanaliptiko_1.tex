\documentclass[twoside,nofonts,ektypwsh,math,spyros]{frontisthrio-diag}
\usepackage[amsbb,subscriptcorrection,zswash,mtpcal,mtphrb,mtpfrak]{mtpro2}
\usepackage[no-math,cm-default]{fontspec}
\usepackage{amsmath}
\usepackage{xunicode}
\usepackage{xgreek}
\let\hbar\relax
\defaultfontfeatures{Mapping=tex-text,Scale=MatchLowercase}
\setmainfont[Mapping=tex-text,Numbers=Lining,Scale=1.0,BoldFont={Minion Pro Bold}]{Minion Pro}
\newfontfamily\scfont{GFS Artemisia}
\font\icon = "Webdings"
\usepackage{fontawesome}
\newfontfamily{\FA}{fontawesome.otf}
\xroma{red!70!black}
%------TIKZ - ΣΧΗΜΑΤΑ - ΓΡΑΦΙΚΕΣ ΠΑΡΑΣΤΑΣΕΙΣ ----
\usepackage{tikz,pgfplots}
\usepackage{tkz-euclide}
\usetkzobj{all}
\usepackage[framemethod=TikZ]{mdframed}
\usetikzlibrary{decorations.pathreplacing}
\tkzSetUpPoint[size=7,fill=white]
%-----------------------
\usepackage{calc,tcolorbox}
\tcbuselibrary{skins,theorems,breakable}
\usepackage{hhline}
\usepackage[explicit]{titlesec}
\usepackage{graphicx}
\usepackage{multicol}
\usepackage{multirow}
\usepackage{tabularx}
\usetikzlibrary{backgrounds}
\usepackage{sectsty}
\sectionfont{\centering}
\usepackage{enumitem}
\usepackage{adjustbox}
\usepackage{mathimatika,gensymb,eurosym,wrap-rl}
\usepackage{systeme,regexpatch}
%-------- ΜΑΘΗΜΑΤΙΚΑ ΕΡΓΑΛΕΙΑ ---------
\usepackage{mathtools}
%----------------------
%-------- ΠΙΝΑΚΕΣ ---------
\usepackage{booktabs}
%----------------------
%----- ΥΠΟΛΟΓΙΣΤΗΣ ----------
\usepackage{calculator}
%----------------------------
%------ ΔΙΑΓΩΝΙΟ ΣΕ ΠΙΝΑΚΑ -------
\usepackage{array}
\newcommand\diag[5]{%
\multicolumn{1}{|m{#2}|}{\hskip-\tabcolsep
$\vcenter{\begin{tikzpicture}[baseline=0,anchor=south west,outer sep=0]
\path[use as bounding box] (0,0) rectangle (#2+2\tabcolsep,\baselineskip);
\node[minimum width={#2+2\tabcolsep-\pgflinewidth},
minimum  height=\baselineskip+#3-\pgflinewidth] (box) {};
\draw[line cap=round] (box.north west) -- (box.south east);
\node[anchor=south west,align=left,inner sep=#1] at (box.south west) {#4};
\node[anchor=north east,align=right,inner sep=#1] at (box.north east) {#5};
\end{tikzpicture}}\rule{0pt}{.71\baselineskip+#3-\pgflinewidth}$\hskip-\tabcolsep}}
%---------------------------------
%---- ΟΡΙΖΟΝΤΙΟ - ΚΑΤΑΚΟΡΥΦΟ - ΠΛΑΓΙΟ ΑΓΚΙΣΤΡΟ ------
\newcommand{\orag}[3]{\node at (#1)
{$ \overcbrace{\rule{#2mm}{0mm}}^{{\scriptsize #3}} $};}
\newcommand{\kag}[3]{\node at (#1)
{$ \undercbrace{\rule{#2mm}{0mm}}_{{\scriptsize #3}} $};}
\newcommand{\Pag}[4]{\node[rotate=#1] at (#2)
{$ \overcbrace{\rule{#3mm}{0mm}}^{{\rotatebox{-#1}{\scriptsize$#4$}}}$};}
%-----------------------------------------
%------------------------------------------
\newcommand{\tss}[1]{\textsuperscript{#1}}
\newcommand{\tssL}[1]{\MakeLowercase{\textsuperscript{#1}}}
%---------- ΛΙΣΤΕΣ ----------------------
\newlist{bhma}{enumerate}{3}
\setlist[bhma]{label=\bf\textit{\arabic*\textsuperscript{o}\;Βήμα :},leftmargin=0cm,itemindent=1.8cm,ref=\bf{\arabic*\textsuperscript{o}\;Βήμα}}
\newlist{rlist}{enumerate}{3}
\setlist[rlist]{itemsep=0mm,label=\roman*.}
\newlist{brlist}{enumerate}{3}
\setlist[brlist]{itemsep=0mm,label=\bf\roman*.}
\newlist{tropos}{enumerate}{3}
\setlist[tropos]{label=\bf\textit{\arabic*\textsuperscript{oς}\;Τρόπος :},leftmargin=0cm,itemindent=2.3cm,ref=\bf{\arabic*\textsuperscript{oς}\;Τρόπος}}
% Αν μπει το bhma μεσα σε tropo τότε
%\begin{bhma}[leftmargin=.7cm]
\tkzSetUpPoint[size=7,fill=white]
\tikzstyle{pl}=[line width=0.3mm]
\tikzstyle{plm}=[line width=0.4mm]
\usepackage{etoolbox}
\makeatletter
\renewrobustcmd{\anw@true}{\let\ifanw@\iffalse}
\renewrobustcmd{\anw@false}{\let\ifanw@\iffalse}\anw@false
\newrobustcmd{\noanw@true}{\let\ifnoanw@\iffalse}
\newrobustcmd{\noanw@false}{\let\ifnoanw@\iffalse}\noanw@false
\renewrobustcmd{\anw@print}{\ifanw@\ifnoanw@\else\numer@lsign\fi\fi}
\makeatother

\usepackage{path}
\pathal

\begin{document}
\titlos{Γεωμετρία Β΄ Λυκείου}{Διαγώνισμα}{Επαναληπτικό}
\begin{thema}
\item\mbox{}\\\vspace{-7mm}
\begin{erwthma}
\item Να απαντήσετε στις παρακάτω ερωτήσεις.
\begin{rlist}
\item Γράψτε τους τύπους από τους οποίους δίνονται τα εμβαδά των παρακάτω βασικών σχημάτων : Τετράγωνο, ορθογώνιο, παραλληλόγραμμο, τρίγωνο και τραπέζιο.
\item Να διατυπώσετε το γενικευμένο πυθαγόρειο θεώρημα για πλευρά που βρίσκεται απέναντι από οξεία γωνία.
\item Γράψτε τους τύπους από τους οποίους δίνεται το εμβαδόν ενός κύκλου, ενός κυκλικού τομέα και ενός κυκλικού τμήματος.
\item Ποιο πολύγωνο ονομάζεται κανονικό;
\item Με τι ισούται ο λόγος των εμβαδών δύο ομοίων τριγώνων;
\end{rlist}\monades{3}
\item \swstolathos
\begin{rlist}
\item Η κεντρική γωνία ενός κανονικού οκταγώνου ισούται με $ \omega_8=45\degree $.
\item Η πλευρά ενός κανονικού εξαγώνου εγγεγραμμένο σε κύκλο ακτίνας $ R $ ισούται με $ \lambda_6=R\sqrt{2} $.
\item Το εμβαδόν ενός τραπεζίου ισούται με το γινόμενο της διαμέσου επί το ύψος του τραπεζίου.
\item Αν για ένα τρίγωνο $ AB\varGamma $ ισχύει η σχέση $ a^2>\beta^2+\gamma^2 $ τότε το τρίγωνο είναι αμβλυγώνιο.
\item Το εμβαδόν ενός κυκλικού τομέα $ O\widearc{AB} $ μέτρου $ \mu $ δίνεται από τον τύπο
\[ (O\widearc{AB})=\frac{\pi R\mu}{360\degree} \]
\end{rlist}\monades{2}
\end{erwthma}
\item\mbox{}\\
Δίνεται ένα τρίγωνο $ AB\varGamma $ όπως αυτό φαίνεται στο παρακάτω σχήμα, με πλευρές $ a=7,\beta=8 $ και $ \gamma=10 $.
\begin{center}
\begin{tikzpicture}[scale=.4]
\tkzDefPoint(0,0){B}
\tkzDefPoint(10,0){C}
\tkzDefPoint(4.25,5.56){A}
\tkzDefPoint(4.25,0){D}
\draw[pl] (A) -- (B) -- (C) -- cycle;
\tkzLabelPoint[above](A){$A$}
\tkzLabelPoint[left](B){$B$}
\tkzLabelPoint[right](C){$\varGamma$}
\tkzLabelPoint[below](D){$\varDelta$}
\draw (A) -- (D);
\node at (4.9,2.4){$\upsilon_a$};
\tkzDrawPoints(A,B,C,D)
\end{tikzpicture}
\end{center}
\begin{erwthma}
\item Να αποδείξετε ότι το τρίγωνο είναι οξυγώνιο.\monades{1}
\item Να υπολογίσετε την προβολή της πλευράς $ \beta $ στην πλευρά $ \gamma $.\monades{2}
\item Να υπολογίσετε το ύψος που αντιστοιχεί στην πλευρά $ a $.\monades{2}
\end{erwthma}
\item\mbox{}\\
\wrapr{-4mm}{7}{5cm}{-7mm}{\begin{tikzpicture}
\tkzDefPoint(-1,.5){A}
\tkzDefPoint(2,.5){B}
\tkzDefPoint(1,-.5){C}
\tkzDefPoint(-2,-.5){D}
\draw[pl] (-1,0.5) -- (-2,-0.5) -- (1,-0.5) -- (2,0.5) --cycle;
\tkzLabelPoint[above](A){$A$}
\tkzLabelPoint[above right](B){$B$}
\tkzLabelPoint[right](C){$\varGamma$}
\tkzLabelPoint[left](D){$\varDelta$}
\tkzMarkAngle[scale=.5](D,A,B)
\tkzMarkAngle[scale=.55](A,B,C)
\tkzDrawPoints(A,B,C,D)
\end{tikzpicture}}{
Σε παραλληλόγραμμο $ AB\varGamma\varDelta $ είναι $ \hat{A}=5\hat{B} $ και $ (AB\varGamma\varDelta) = 40$. Αν η περίμετρος είναι δωδεκαπλάσια της $ A\varDelta $, να δείξετε ότι:
\begin{erwthma}
\item $ AB=20 $ και $ B\varGamma=4 $.\monades{3}
\item Τα ύψη του $ AB\varGamma\varDelta $ είναι $ 2 $ και $ 10 $.\monades{2}
\end{erwthma}}\mbox{}\\\\
\item\mbox{}\\
Δίνεται τετράγωνο $ AB\varGamma\varDelta $ πλευράς $ a $ και τα τόξα $ \widearc{B\varDelta} $ ακτίνας $ B\varGamma $ και $ \widearc{\varGamma\varDelta} $ διαμέτρου $ \varGamma\varDelta $ .\\
\begin{center}
\begin{tikzpicture}
\polygon[]{4}{2}{A,B,\varGamma,\varDelta}{pl,fill=white}
\draw[fill=\xrwma!30] (1.4142,1.4142) arc (90:180:2.8251)--(1.4142,-1.4142)--(1.4142,1.4142);
\draw[fill=white] (1.4142,-1.4142) arc (0:180:1.4142)--(1.4142,-1.4142);
\end{tikzpicture}
\end{center}
\begin{erwthma}
\item Να βρεθεί η περίμετρος του καμπυλόγραμμου τριγώνου $ B\varGamma\varDelta $ συναρτήσει της πλευράς $ a $ του τετραγώνου.\\\monades{2}
\item Να βρεθεί το εμβαδόν του χρωματισμένου μέρους συναρτήσει της πλευράς $ a $ του τετραγώνου.\\\monades{3}
\end{erwthma}
\end{thema}
\kaliepityxia
\end{document}
