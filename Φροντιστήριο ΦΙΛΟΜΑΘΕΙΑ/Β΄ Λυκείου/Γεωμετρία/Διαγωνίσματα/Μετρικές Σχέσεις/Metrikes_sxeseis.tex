\documentclass[ektypwsh]{diag-xelatex}
\usepackage[amsbb]{mtpro2}
\usepackage[no-math,cm-default]{fontspec}
\usepackage{xunicode}
\usepackage{xltxtra}
\usepackage{xgreek}
\usepackage{amsmath}
\defaultfontfeatures{Mapping=tex-text,Scale=MatchLowercase}
\setmainfont[Mapping=tex-text,Numbers=Lining,Scale=1.0,BoldFont={Minion Pro Bold}]{Minion Pro}
\newfontfamily\scfont{GFS Artemisia}
\font\icon = "Webdings"
\usepackage[amsbb]{mtpro2}
\usepackage[left=2.00cm, right=2.00cm, top=2.00cm, bottom=3.00cm]{geometry}
\xroma{red!80!black}
\newcommand{\tss}[1]{\textsuperscript{#1}}
\newcommand{\tssL}[1]{\MakeLowercase\textsuperscript{#1}}
\newlist{rlist}{enumerate}{3}
\setlist[rlist]{itemsep=0mm,label=\roman*.}
\usepackage{tikz,pgfplots,tkz-euclide}
\usepackage{mathimatika,gensymb,wrap-rl}
\tkzSetUpPoint[size=7,fill=white]




\begin{document}
\titlos{Γεωμετρία Β' Λυκείου}{Μετρικές Σχέσεις}
\begin{thema}
\item \textbf{Θεωρία}\\
Να διατυπώσετε και να αποδείξετε το 1\tss{ο} Θεώρημα διαμέσων.\monades{5}
\item \textbf{Γενικευμένο Πυθαγόρειο θεώρημα}\\
Δίνεται ένα τρίγωνο $ AB\varGamma $ όπως αυτό φαίνεται στο παρακάτω σχήμα, με πλευρές $ a=7,\beta=8 $ και $ \gamma=10 $.
\begin{center}
\begin{tikzpicture}[scale=.4]
\tkzDefPoint(0,0){B}
\tkzDefPoint(10,0){C}
\tkzDefPoint(4.25,5.56){A}
\tkzDefPoint(4.25,0){D}
\draw[pl] (A) -- (B) -- (C) -- cycle;
\tkzDrawPoints(A,B,C,D)
\tkzLabelPoint[above](A){$A$}
\tkzLabelPoint[left](B){$B$}
\tkzLabelPoint[right](C){$\varGamma$}
\tkzLabelPoint[below](D){$\varDelta$}
\draw (A) -- (D);
\node at (4.9,2.4){$\upsilon_a$};
\end{tikzpicture}
\end{center}
\begin{rlist}
\item Να αποδείξετε οτι το τρίγωνο είναι οξυγώνιο.\monades{2}
\item Να υπολογίσετε την προβολή της πλευράς $ \beta $ στην πλευρά $ \gamma $.\monades{2}
\item Να υπολογίσετε το ύψος που αντιστοιχεί στην πλευρά $ a $.\monades{1}
\end{rlist}
\item \textbf{Θεώρημα διαμέσων}\\
Επιλέγουμε ένα τυχαίο σημείο $ M $ στο εξωτερικό μέρος ενός ορθογωνίου $ AB\varGamma\varDelta $. Να αποδείξετε ότι ισχύουν οι παρακάτω σχέσεις.\\
\wrapr{-8mm}{8}{4.2cm}{-7mm}{\begin{tikzpicture}[scale=.8]
\tkzDefPoint(0,0){D}
\tkzDefPoint(4,0){C}
\tkzDefPoint(4,2.5){B}
\tkzDefPoint(0,2.5){A}
\tkzDefPoint(2,1.25){O}
\tkzDefPoint(3,3){M}
\draw[pl] (A)--(B)--(C)--(D) -- cycle;
\tkzLabelPoint[above left](A){$A$}
\tkzLabelPoint[above right](B){$B$}
\tkzLabelPoint[right](C){$\varGamma$}
\tkzLabelPoint[left](D){$\varDelta$}
\tkzLabelPoint[below](O){$O$}
\tkzLabelPoint[above](M){$M$}
\draw (A) -- (C);
\draw (B) -- (D);
\draw (M) -- (D);
\draw (M) -- (A);
\draw (M) -- (B);
\draw (M) -- (C);
\draw[plm] (M) -- (O);
\tkzDrawPoints(A,B,C,D,O,M)
\end{tikzpicture}}{
\begin{rlist}
\item $ MA^2+M\varGamma^2=MB^2+M\varDelta^2 $\monades{2}
\item $ MA^2+M\varGamma^2+MB^2+M\varDelta^2=A\varGamma^2+4MO^2 $\monades{3}
\end{rlist}}\mbox{}\\\\
\item \textbf{Σύνθετο θέμα}\\
Έστω $ A\varDelta $ η διχοτόμος της γωνίας $ \hat{A} $ ενός τριγώνου $ AB\varGamma $. Στην προέκταση της $ \varGamma B $ παίρνουμε τμήμα $ BE $ τέτοιο ώστε $ BE=A\varGamma $.
\begin{center}
\begin{tikzpicture}
\tkzDefPoint(0,0){B}
\tkzDefPoint(3,0){C}
\tkzDefPoint(1,1.5){A}
\tkzDefPoint(-1.8,0){E}
\tkzDefPoint(1.26,0){D}
\tkzDefPoint(-.7,-1.04){Z}
\draw[pl] (A) -- (B) -- (C) -- cycle;
\draw[plm] (A) -- (D);
\draw (A) -- (D);
\draw (B) -- (Z);
\draw (B) -- (E);
\draw (A) -- (E);
\draw (-.27,.51) circle (1.61);
\tkzLabelPoint[above right](A){$A$}
\tkzLabelPoint[below right](B){$B$}
\tkzLabelPoint[right](C){$\varGamma$}
\tkzLabelPoint[below](D){$\varDelta$}
\tkzLabelPoint[left](E){$E$}
\tkzLabelPoint[below](Z){$Z$}
\tkzDrawPoints(A,B,C,D,E,Z)
\end{tikzpicture}
\end{center}
Η ευθεία $ AB $ τέμνει τον περιγεγραμμένο κύκλο του τριγώνου $ A\varDelta E $ στο σημείο $ Z $. Να αποδείξετε ότι ισχύει $ \varGamma\varDelta=BZ $.\monades{5}
\end{thema}
\end{document}

