\documentclass[ektypwsh]{diag-xelatex}
\usepackage[amsbb]{mtpro2}
\usepackage[no-math,cm-default]{fontspec}
\usepackage{xunicode}
\usepackage{xltxtra}
\usepackage{xgreek}
\usepackage{amsmath}
\defaultfontfeatures{Mapping=tex-text,Scale=MatchLowercase}
\setmainfont[Mapping=tex-text,Numbers=Lining,Scale=1.0,BoldFont={Minion Pro Bold}]{Minion Pro}
\newfontfamily\scfont{GFS Artemisia}
\font\icon = "Webdings"
\usepackage[amsbb]{mtpro2}
\usepackage[left=2.00cm, right=2.00cm, top=2.00cm, bottom=3.00cm]{geometry}
\xroma{red!80!black}
\newcommand{\tss}[1]{\textsuperscript{#1}}
\newcommand{\tssL}[1]{\MakeLowercase\textsuperscript{#1}}
\newlist{rlist}{enumerate}{3}
\setlist[rlist]{itemsep=0mm,label=\roman*.}
\usepackage{mathimatika,gensymb,wrap-rl,multicol}
\usetkzobj{all}
\tkzSetUpPoint[size=7,fill=white]


\begin{document}
\titlos{Γεωμετρία Β΄ Λυκείου}{Εμβαδά}
\begin{thema}
\item \textbf{Θεωρία}\\
Να απαντήσετε στις παρακάτω ερωτήσεις.
\begin{rlist}
\item Γράψτε τους τύπους από τους οποίους δίνονται τα εμβαδά των παρακάτω βασικών σχημάτων : Τετράγωνο, ορθογώνιο, παραλληλόγραμμο, τρίγωνο και τραπέζιο.\monades{2}
\item Από ποιούς τύπους δίνονται τα εμβαδά του ρόμβου, του ισόπλευρου τριγώνου και του ορθογωνίου τριγώνου;\\\monades{1}
\item Γράψτε τους επιπλέον τύπους που δίνουν το εμβαδόν ενός τριγώνου.\monades{1}
\item Με τι ισούται ο λόγος των εμβαδών δύο τριγώνων που έχουν δύο πλευρές τους ίσες;\monades{1}
\end{rlist}
\item \textbf{Εμβαδά βασικών σχημάτων}\\
\wrapr{-4mm}{7}{5cm}{-7mm}{\begin{tikzpicture}
\tkzDefPoint(-1,.5){A}
\tkzDefPoint(2,.5){B}
\tkzDefPoint(1,-.5){C}
\tkzDefPoint(-2,-.5){D}
\draw[pl] (-1,0.5) -- (-2,-0.5) -- (1,-0.5) -- (2,0.5) --cycle;
\tkzLabelPoint[above](A){$A$}
\tkzLabelPoint[above right](B){$B$}
\tkzLabelPoint[right](C){$\varGamma$}
\tkzLabelPoint[left](D){$\varDelta$}
\tkzMarkAngle[scale=.5](D,A,B)
\tkzMarkAngle[scale=.55](A,B,C)
\tkzDrawPoints(A,B,C,D)
\end{tikzpicture}}{
Σε παραλληλόγραμμο $ AB\varGamma\varDelta $ είναι $ \hat{A}=5\hat{B} $ και $ (AB\varGamma\varDelta) = 40$. Αν η περίμετρος είναι δωδεκαπλάσια της $ A\varDelta $, να δείξετε ότι:
\begin{rlist}
\item $ AB=20 $ και $ B\varGamma=4 $.\monades{3}
\item Τα ύψη του $ AB\varGamma\varDelta $ είναι $ 2 $ και $ 10 $.\monades{2}
\end{rlist}}\mbox{}\\\\
\item \textbf{Λόγος εμβαδών}\\
Σε τρίγωνο $ AB\varGamma $ με $ \hat{A}\neq90\degree $ σχεδιάζουμε τα ύψη $ BZ $ και $ \varGamma H $. Να αποδείξετε ότι $ (AZH)=(AB\varGamma)\syn^2{A} $.\monades{5}
\item \textbf{Σύνθετο θέμα}\\
Δίνεται τρίγωνο $ AB\varGamma $ και $ \varTheta $ το βαρύκεντρό του. Από σημείο $ \varSigma $ της διαμέσου $ A\varDelta $ φέρουμε κάθετες $ \varSigma E,\varSigma Z $ στις $ A\Gamma,AB $ αντίστοιχα. Να αποδείξετε ότι
\begin{multicols}{2}
\begin{rlist}
\item $ (AB\varSigma)=(A\varGamma\varSigma) $\monades{1}
\item $ AB\cdot \varSigma Z=A\varGamma\cdot\varSigma E $.\monades{2}
\item $ (AB\varTheta)=(B\varTheta\varGamma)=\frac{1}{3}(AB\varGamma) $\monades{2}
\end{rlist}
\end{multicols}
\end{thema}
\end{document}

