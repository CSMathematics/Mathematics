\PassOptionsToPackage{no-math,cm-default}{fontspec}
\documentclass[twoside,nofonts,internet,shmeiwseis]{thewria}
\usepackage{amsmath}
\usepackage{xgreek}
\let\hbar\relax
\defaultfontfeatures{Mapping=tex-text,Scale=MatchLowercase}
\setmainfont[Mapping=tex-text,Numbers=Lining,Scale=1.0,BoldFont={Minion Pro Bold}]{Minion Pro}
\newfontfamily\scfont{GFS Artemisia}
\font\icon = "Webdings"
\usepackage[amsbb,subscriptcorrection,zswash,mtpcal,mtphrb]{mtpro2}
\usepackage{tikz,pgfplots}
\tkzSetUpPoint[size=7,fill=white]
\xroma{red!70!black}
%------TIKZ - ΣΧΗΜΑΤΑ - ΓΡΑΦΙΚΕΣ ΠΑΡΑΣΤΑΣΕΙΣ ----
\usepackage{tikz}
\usepackage{tkz-euclide}
\usetkzobj{all}
\usepackage[framemethod=TikZ]{mdframed}
\usetikzlibrary{decorations.pathreplacing,intersections}
\usepackage{pgfplots}
\usetkzobj{all}
%-----------------------
\usepackage{calc}
\usepackage{hhline}
\usepackage[explicit]{titlesec}
\usepackage{graphicx}
\usepackage{multicol}
\usepackage{multirow}
\usepackage{enumitem}
\usepackage{tabularx}
\usepackage[decimalsymbol=comma]{siunitx}
\usetikzlibrary{backgrounds}
\usepackage{sectsty}
\sectionfont{\centering}
\setlist[enumerate]{label=\bf{\large \arabic*.}}
\usepackage{adjustbox}
\usepackage{mathimatika,gensymb,eurosym,wrap-rl}
\usepackage{systeme,regexpatch}
%-------- ΜΑΘΗΜΑΤΙΚΑ ΕΡΓΑΛΕΙΑ ---------
\usepackage{mathtools}
%----------------------
%-------- ΠΙΝΑΚΕΣ ---------
\usepackage{booktabs}
%----------------------
%----- ΥΠΟΛΟΓΙΣΤΗΣ ----------
\usepackage{calculator}
%----------------------------
%------ ΔΙΑΓΩΝΙΟ ΣΕ ΠΙΝΑΚΑ -------
\usepackage{array}
\newcommand\diag[5]{%
\multicolumn{1}{|m{#2}|}{\hskip-\tabcolsep
$\vcenter{\begin{tikzpicture}[baseline=0,anchor=south west,outer sep=0]
\path[use as bounding box] (0,0) rectangle (#2+2\tabcolsep,\baselineskip);
\node[minimum width={#2+2\tabcolsep-\pgflinewidth},
minimum  height=\baselineskip+#3-\pgflinewidth] (box) {};
\draw[line cap=round] (box.north west) -- (box.south east);
\node[anchor=south west,align=left,inner sep=#1] at (box.south west) {#4};
\node[anchor=north east,align=right,inner sep=#1] at (box.north east) {#5};
\end{tikzpicture}}\rule{0pt}{.71\baselineskip+#3-\pgflinewidth}$\hskip-\tabcolsep}}
%---------------------------------
%---- ΟΡΙΖΟΝΤΙΟ - ΚΑΤΑΚΟΡΥΦΟ - ΠΛΑΓΙΟ ΑΓΚΙΣΤΡΟ ------
\newcommand{\orag}[3]{\node at (#1)
{$ \overcbrace{\rule{#2mm}{0mm}}^{{\scriptsize #3}} $};}
\newcommand{\kag}[3]{\node at (#1)
{$ \undercbrace{\rule{#2mm}{0mm}}_{{\scriptsize #3}} $};}
\newcommand{\Pag}[4]{\node[rotate=#1] at (#2)
{$ \overcbrace{\rule{#3mm}{0mm}}^{{\rotatebox{-#1}{\scriptsize$#4$}}}$};}
%-----------------------------------------
%------------------------------------------
\newcommand{\tss}[1]{\textsuperscript{#1}}
\newcommand{\tssL}[1]{\MakeLowercase{\textsuperscript{#1}}}
%---------- ΛΙΣΤΕΣ ----------------------
\newlist{bhma}{enumerate}{3}
\setlist[bhma]{label=\bf\textit{\arabic*\textsuperscript{o}\;Βήμα :},leftmargin=0cm,itemindent=1.8cm,ref=\bf{\arabic*\textsuperscript{o}\;Βήμα}}
\newlist{rlist}{enumerate}{3}
\setlist[rlist]{itemsep=0mm,label=\roman*.}
\newlist{brlist}{enumerate}{3}
\setlist[brlist]{itemsep=0mm,label=\bf\roman*.}
\newlist{tropos}{enumerate}{3}
\setlist[tropos]{label=\bf\textit{\arabic*\textsuperscript{oς}\;Τρόπος :},leftmargin=0cm,itemindent=2.3cm,ref=\bf{\arabic*\textsuperscript{oς}\;Τρόπος}}
% Αν μπει το bhma μεσα σε tropo τότε
%\begin{bhma}[leftmargin=.7cm]
\tkzSetUpPoint[size=7,fill=white]
\tikzstyle{pl}=[line width=0.3mm]
\tikzstyle{plm}=[line width=0.4mm]
\usepackage{etoolbox}
\makeatletter
\renewrobustcmd{\anw@true}{\let\ifanw@\iffalse}
\renewrobustcmd{\anw@false}{\let\ifanw@\iffalse}\anw@false
\newrobustcmd{\noanw@true}{\let\ifnoanw@\iffalse}
\newrobustcmd{\noanw@false}{\let\ifnoanw@\iffalse}\noanw@false
\renewrobustcmd{\anw@print}{\ifanw@\ifnoanw@\else\numer@lsign\fi\fi}
\makeatother

\newcommand{\miniskos}[7][]{
 \coordinate (#6) at (#2);
  \coordinate (#7) at (#3);
  \begin{scope}[overlay]
  \path [name path=#6] (#6) circle [radius=#4];
  \path [name path=#7] (#7) circle [radius=#5];
  \path [name intersections={of=#6 and #7, by={p1,p2}}];
  \end{scope}
  \filldraw [#1] let
    \p1=(#6),\p2=(#7),\p3=(p1),\p4=(p2),
    \n1={veclen(\x3-\x1,\y3-\y1)},
    \n2={atan2(\y3-\y1,\x3-\x1)}, \n3={atan2(\y4-\y1,\x4-\x1)},
    \n4={veclen(\x3-\x2,\y3-\y2)},
    \n5={atan2(\y3-\y2,\x3-\x2)}, \n6={atan2(\y4-\y2,\x4-\x2)} in
    ($(#6)+(\n2:\n1)$) arc (\n2:\n3:\n1) arc(\n6:\n5:\n4) -- cycle;
}




\begin{document}
\titlos{Γεωμετρία Β΄ Λυκείου}{Μέτρηση κύκλου}{Εμβαδόν κύκλου - Κυκλικού τομέα - Κυκλικού τμήματος}
\orismoi
\Orismos{Εμβαδόν κύκλου}
Εμβαδόν ενός κύκλου $ (O,R) $ ονομάζεται ο θετικός αριθμός $ E $ ο οποίος είναι το όριο των ακολουθιών $ (E_\nu) $ των εγγεγραμμένων και $ (E_\nu') $ των περιγεγραμμένων κανονικών πολυγώνων, καθώς το πλήθος $ \nu $ των πλευρών αυξάνεται.\\\\
\Orismos{Κυκλικός τομέας}
\wrapr{-5mm}{5}{2.4cm}{-12mm}{\begin{tikzpicture}
\draw (0,0) circle (1.2);
\coordinate (O)  at (0,0);
\coordinate (A)  at (240:1.2);
\coordinate (B)  at (300:1.2);
\tkzLabelPoint[below left](A){$A$}
\tkzLabelPoint[below right](B){$B$}
\tkzLabelPoint[above](O){$O$}
\draw[fill=\xrwma!30,pl,draw=\xrwma] (A)--(O)--(B) arc[start angle=300, end angle=240, radius=1.2] (A);
\tkzMarkAngle[size=.35](A,O,B)
\tkzDrawPoints(A,B,O)
\node at (0,-0.5) {\footnotesize$ \mu $};
\end{tikzpicture}}{
Κυκλικός τομέας κέντρου $ O $ και ακτίνας $ R $ ενός κύκλου $ (O,R) $ ονομάζεται το σύνολο των σημείων που περικλείει μια επίκεντρη $ \hat{O} $ γωνία και το αντίστοιχο τόξο της $ \widearc{AB} $. Συμβολίζεται με $ O\widearc{AB} $.}\mbox{}\\\\\\
\Orismos{Κυκλικό τμήμα}
\wrapr{-5mm}{5}{2.4cm}{-12mm}{\begin{tikzpicture}
\draw (0,0) circle (1.2);
\coordinate (O)  at (0,0);
\coordinate (A)  at (240:1.2);
\coordinate (B)  at (300:1.2);
\tkzLabelPoint[below left](A){$A$}
\tkzLabelPoint[below right](B){$B$}
\tkzLabelPoint[above](O){$O$}
\draw (A)--(O)--(B);
\draw[fill=\xrwma!30,pl,draw=\xrwma] (A)--(B) arc[start angle=300, end angle=240, radius=1.2cm] (A);
\tkzMarkAngle[size=.35](A,O,B)
\tkzDrawPoints(A,B,O)
\node at (0,-0.5) {\footnotesize$ \mu $};
\end{tikzpicture}}{
Κυκλικό τμήμα ονομάζεται το σύνολο των σημείων που περικλείονται μεταξύ ενός τόξου και της αντίστοιχης χορδής του, σε έναν κύκλο $ (O,R) $.}\mbox{}\\\\\\
\Orismos{Μηνισκος}
\wrapr{-5mm}{5}{2.8cm}{-5mm}{\begin{tikzpicture}
\draw (0,0) circle (1.2);
\draw (.55,.55) circle (.7);
\miniskos[\xrwma!30,draw=\xrwma,line width=.3mm]{0,0}{.55,.55}{1.2}{.7}{A}{B}
\draw (p1)--(p2);
\tkzLabelPoint[left](A){$ K $}
\tkzLabelPoint[left](B){$ \varLambda $}
\tkzLabelPoint[above left](p1){$ A $}
\tkzLabelPoint[right](p2){$ B $}
\tkzDrawPoints(p1,p2,A,B)
\end{tikzpicture}}{
Μηνίσκος ονομάζεται το σύνοο των σημείων του επιπέδου που βρίσκονται μεταξύ δύο τόξων με κοινή χορδή. Τα τόξα αυτά βρίσκονται προς το ίδιο μέρος της χορδής.}\mbox{}\\\\
\thewrhmata
\Thewrhma{Εμβαδόν κύκλου}
Το εμβαδόν $ E $ ενός κύκλου ακτίνας $ R $ ισούται με $ E=\pi R^2 $.\\\\
\Thewrhma{Εμβαδόν κυκλικού τομέα}\label{th2}
Το εμβαδόν ενός κυκλικού τομέα $ O\widearc{AB} $ κέντρου $ O $ και ακτίνας $ R $ ισούται με \[ ( O\widearc{AB} )=\pi R^2\cdot\frac{\mu}{360\degree}=\frac{1}{2}aR^2 \]
όπου $ \mu $ είναι το μέτρο του τομέα σε μοίρες και $ a $ το μέτρο του σε ακτίνια.\\\\\\
\Thewrhma{Εμβαδόν κυκλικού τμήματος}\label{th3}
Το εμβαδόν ενός κυκλικού τμήματος $ \varepsilon $ που βρίσκεται μεταξύ ενός τόξου $ AB $ και της αντίστοιχης χορδής του δίνεται από τον τύπο:
\[ \varepsilon=(O\widearc{AB})-(OAB)=\frac{\pi R^2\mu}{360\degree}-\frac{R^2\hm{\mu}}{2}=\frac{R^2}{2}(a-\hm{a}) \]
\Thewrhma{Εμβαδόν μηνισκου}
\wrapr{-5mm}{7}{3.1cm}{-10mm}{\begin{tikzpicture}[scale=1.59]
\clip (-.5,-.5) rectangle (1.6,1.6);
\draw[dashed] (0,0) circle (1.2);
\draw[dashed] (.55,.55) circle (.7);
\miniskos[\xrwma!30,draw=\xrwma,line width=.3mm]{0,0}{.55,.55}{1.2}{.7}{A}{B}
\tkzMarkAngle[size=.15](p2,B,p1)
\tkzMarkAngle[size=.19](p2,A,p1)
\draw (p1)--(A)--(p2)--(B)--cycle;
\draw (p1)--(p2);
\tkzLabelPoint[left](A){$ K $}
\tkzLabelPoint[left](B){$ \varLambda $}
\tkzLabelPoint[above left](p1){$ A $}
\tkzLabelPoint[right](p2){$ B $}
\tkzDrawPoints(p1,p2,A,B)
\node at (.23,.23){\footnotesize$ \theta $};
\node[fill=white,inner sep=.2mm] at (.74,.74){\footnotesize$ \varphi $};
\node at (.94,.94){\footnotesize$ \mu $};
\node at (.57,0.01){\footnotesize$ R $};
\node at (.75,0.33){\footnotesize$ \rho $};
\end{tikzpicture}}{Το εμβαδόν ενός μηνίσκου $ \mu $ που ορίζεται από δύο κυκλικά τόξα κοινής χορδής $ AB $ ισούται με τη διαφορά των εμβαδών των δύο κυκλικών τμημάτων που ορίζει η χορδή στους δύο κύκλους. 
\[ \mu=(K\widearc{AB})-(\varLambda\widearc{AB})+(K AB)-(\varLambda AB) \]
Με τη βοήθεια των \textbf{Θεωρημάτων \ref{th2} και \ref{th3}} προκύπτουν επιπλέον τύποι για τον υπολογισμό του εμβαδού του μηνίσκου:
\[ \mu=\frac{\pi R^2(\theta-\varphi)}{360\degree}+\frac{R^2(\hm{\theta}-\hm{\varphi})}{2}=\frac{R^2}{2}(a-\beta+\hm{a}-\hm{\beta}) \]
όπου $ a $ και $ \beta $ είναι τα μέτρα των γωνιών $ \theta $ και $ \varphi $ αντίστοιχα, δοσμένα σε ακτίνια.}\mbox{}\\\\
\end{document}
