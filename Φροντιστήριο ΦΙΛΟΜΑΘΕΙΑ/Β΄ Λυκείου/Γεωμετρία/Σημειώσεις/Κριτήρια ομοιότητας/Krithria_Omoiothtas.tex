\PassOptionsToPackage{no-math,cm-default}{fontspec}
\documentclass[twoside,nofonts,internet,shmeiwseis]{thewria}
\usepackage{amsmath}
\usepackage{xgreek}
\let\hbar\relax
\defaultfontfeatures{Mapping=tex-text,Scale=MatchLowercase}
\setmainfont[Mapping=tex-text,Numbers=Lining,Scale=1.0,BoldFont={Minion Pro Bold}]{Minion Pro}
\newfontfamily\scfont{GFS Artemisia}
\font\icon = "Webdings"
\usepackage[amsbb,subscriptcorrection,zswash,mtpcal,mtphrb]{mtpro2}
\usepackage{tikz,pgfplots}
\tkzSetUpPoint[size=7,fill=white]
\xroma{red!70!black}
%------TIKZ - ΣΧΗΜΑΤΑ - ΓΡΑΦΙΚΕΣ ΠΑΡΑΣΤΑΣΕΙΣ ----
\usepackage{tikz}
\usepackage{tkz-euclide}
\usetkzobj{all}
\usepackage[framemethod=TikZ]{mdframed}
\usetikzlibrary{decorations.pathreplacing}
\usepackage{pgfplots}
\usetkzobj{all}
%-----------------------
\usepackage{calc}
\usepackage{hhline}
\usepackage[explicit]{titlesec}
\usepackage{graphicx}
\usepackage{multicol}
\usepackage{multirow}
\usepackage{enumitem}
\usepackage{tabularx}
\usepackage[decimalsymbol=comma]{siunitx}
\usetikzlibrary{backgrounds}
\usepackage{sectsty}
\sectionfont{\centering}
\setlist[enumerate]{label=\bf{\large \arabic*.}}
\usepackage{adjustbox}
\usepackage{mathimatika,gensymb,eurosym,wrap-rl}
\usepackage{systeme,regexpatch}
%-------- ΜΑΘΗΜΑΤΙΚΑ ΕΡΓΑΛΕΙΑ ---------
\usepackage{mathtools}
%----------------------
%-------- ΠΙΝΑΚΕΣ ---------
\usepackage{booktabs}
%----------------------
%----- ΥΠΟΛΟΓΙΣΤΗΣ ----------
\usepackage{calculator}
%----------------------------
%------ ΔΙΑΓΩΝΙΟ ΣΕ ΠΙΝΑΚΑ -------
\usepackage{array}
\newcommand\diag[5]{%
\multicolumn{1}{|m{#2}|}{\hskip-\tabcolsep
$\vcenter{\begin{tikzpicture}[baseline=0,anchor=south west,outer sep=0]
\path[use as bounding box] (0,0) rectangle (#2+2\tabcolsep,\baselineskip);
\node[minimum width={#2+2\tabcolsep-\pgflinewidth},
minimum  height=\baselineskip+#3-\pgflinewidth] (box) {};
\draw[line cap=round] (box.north west) -- (box.south east);
\node[anchor=south west,align=left,inner sep=#1] at (box.south west) {#4};
\node[anchor=north east,align=right,inner sep=#1] at (box.north east) {#5};
\end{tikzpicture}}\rule{0pt}{.71\baselineskip+#3-\pgflinewidth}$\hskip-\tabcolsep}}
%---------------------------------
%---- ΟΡΙΖΟΝΤΙΟ - ΚΑΤΑΚΟΡΥΦΟ - ΠΛΑΓΙΟ ΑΓΚΙΣΤΡΟ ------
\newcommand{\orag}[3]{\node at (#1)
{$ \overcbrace{\rule{#2mm}{0mm}}^{{\scriptsize #3}} $};}
\newcommand{\kag}[3]{\node at (#1)
{$ \undercbrace{\rule{#2mm}{0mm}}_{{\scriptsize #3}} $};}
\newcommand{\Pag}[4]{\node[rotate=#1] at (#2)
{$ \overcbrace{\rule{#3mm}{0mm}}^{{\rotatebox{-#1}{\scriptsize$#4$}}}$};}
%-----------------------------------------
%------------------------------------------
\newcommand{\tss}[1]{\textsuperscript{#1}}
\newcommand{\tssL}[1]{\MakeLowercase{\textsuperscript{#1}}}
%---------- ΛΙΣΤΕΣ ----------------------
\newlist{bhma}{enumerate}{3}
\setlist[bhma]{label=\bf\textit{\arabic*\textsuperscript{o}\;Βήμα :},leftmargin=0cm,itemindent=1.8cm,ref=\bf{\arabic*\textsuperscript{o}\;Βήμα}}
\newlist{rlist}{enumerate}{3}
\setlist[rlist]{itemsep=0mm,label=\roman*.}
\newlist{brlist}{enumerate}{3}
\setlist[brlist]{itemsep=0mm,label=\bf\roman*.}
\newlist{tropos}{enumerate}{3}
\setlist[tropos]{label=\bf\textit{\arabic*\textsuperscript{oς}\;Τρόπος :},leftmargin=0cm,itemindent=2.3cm,ref=\bf{\arabic*\textsuperscript{oς}\;Τρόπος}}
% Αν μπει το bhma μεσα σε tropo τότε
%\begin{bhma}[leftmargin=.7cm]
\tkzSetUpPoint[size=7,fill=white]
\tikzstyle{pl}=[line width=0.3mm]
\tikzstyle{plm}=[line width=0.4mm]
\usepackage{etoolbox}
\makeatletter
\renewrobustcmd{\anw@true}{\let\ifanw@\iffalse}
\renewrobustcmd{\anw@false}{\let\ifanw@\iffalse}\anw@false
\newrobustcmd{\noanw@true}{\let\ifnoanw@\iffalse}
\newrobustcmd{\noanw@false}{\let\ifnoanw@\iffalse}\noanw@false
\renewrobustcmd{\anw@print}{\ifanw@\ifnoanw@\else\numer@lsign\fi\fi}
\makeatother

\begin{document}
\titlos{Γεωμετρία Β΄ Λυκείου}{Ομοιότητα}{Κριτήρια ομοιότητας}
\thewrhmata
\Thewrhma{1\tssL{ο} Κριτήριο ομοιότητας τριγώνων}
Δύο τρίγωνα $ AB\varGamma $ και $ A'B'\varGamma' $ είναι όμοια αν έχουν δύο γωνίες ίσες μια προς μια.
\begin{center}
\begin{tikzpicture}[scale=1]
\coordinate (O)  at (0,0);
\coordinate (A)  at (110:1.4);
\coordinate (B) at (0:1.2);
\coordinate (C) at (180:1.2);
\tkzMarkAngle[fill=\xrwma!30,size=.25,mark=|](B,C,A)
\tkzMarkAngle[fill=\xrwma!30,size=.42,mark=||](A,B,C)
\draw[pl] (A)--(B)--(C)--cycle;
\tkzLabelPoint[above](A){$A$}
\tkzLabelPoint[right](B){$\varGamma$}
\tkzLabelPoint[left](C){$B$}
\tkzDrawPoints(A,B,C)
\end{tikzpicture}\quad\begin{tikzpicture}[scale=.8]
\coordinate (O)  at (0,0);
\coordinate (A)  at (110:1.4);
\coordinate (B) at (0:1.2);
\coordinate (C) at (180:1.2);
\tkzMarkAngle[fill=\xrwma!30,size=.25,mark=|](B,C,A)
\tkzMarkAngle[fill=\xrwma!30,size=.42,mark=||](A,B,C)
\draw[pl] (A)--(B)--(C)--cycle;
\tkzLabelPoint[above](A){$A'$}
\tkzLabelPoint[right](B){$\varGamma'$}
\tkzLabelPoint[left](C){$B'$}
\tkzDrawPoints(A,B,C)
\end{tikzpicture}
\end{center}
\Thewrhma{Πορίσματα ομοιότητας τριγώνων}
\vspace{-5mm}
\begin{rlist}
\item Δύο ορθογώνια τρίγωνα είναι όμοια αν έχουν μια αντίστοιχη οξεία γωνία ίση.
\item Όλα τα ισόπλευρα τρίγωνα είναι μεταξύ τους όμοια.
\item Αν δύο ισοσκελή τρίγωνα έχουν μια αντίστοιχη γωνία ίση τότε είναι όμοια.
\end{rlist}
\Thewrhma{2\tssL{ο} Κριτήριο ομοιότητας τριγώνων}
Δύο τρίγωνα που έχουν δύο πλευρές ανάλογες μια προς μια και τις περιεχόμενες γωνίες ίσες είναι όμοια.
\begin{center}
\begin{tikzpicture}[scale=1]
\coordinate (O)  at (0,0);
\coordinate (A)  at (110:1.4);
\coordinate (B) at (0:1.2);
\coordinate (C) at (180:1.2);
\tkzMarkAngle[fill=\xrwma!30,size=.25,mark=|](C,A,B)
\draw[pl] (A)--(B)--(C)--cycle;
\tkzLabelPoint[above](A){$A$}
\tkzLabelPoint[right](B){$\varGamma$}
\tkzLabelPoint[left](C){$B$}
\tkzDrawPoints(A,B,C)
\end{tikzpicture}\quad\begin{tikzpicture}[scale=.8]
\coordinate (O)  at (0,0);
\coordinate (A)  at (110:1.4);
\coordinate (B) at (0:1.2);
\coordinate (C) at (180:1.2);
\tkzMarkAngle[fill=\xrwma!30,size=.25,mark=|](C,A,B)
\draw[pl] (A)--(B)--(C)--cycle;
\tkzLabelPoint[above](A){$A'$}
\tkzLabelPoint[right](B){$\varGamma'$}
\tkzLabelPoint[left](C){$B'$}
\tkzDrawPoints(A,B,C)
\end{tikzpicture}
\end{center}
\Thewrhma{3\tssL{ο} Κριτήριο ομοιότητας τριγώνων}
Δύο τρίγωνα που έχουν τις πλευρές τους ανάλογες μια προς μια είναι όμοια.\\\\
\Thewrhma{Λόγος δευτερευόντων στοιχείων τριγώνων}
Ο λόγος ομοιότητας δύο όμοιων τριγώνων ισούται με το λόγο 
\begin{multicols}{3}
\begin{rlist}
\item δύο ομόλογων υψών
\item δύο ομόλογων διχοτόμων και
\item δύο ομόλογων διαμέσων.
\end{rlist}
\end{multicols}
\Thewrhma{Πόρισμα για το ορθογώνιο τρίγωνο}
Σε κάθε ορθογώνιο τρίγωνο το γινόμενο των κάθετων πλευρών ισούται με το γινόμενο της υποτείνουσας επί το αντίστοιχο ύψος της.
\end{document}
