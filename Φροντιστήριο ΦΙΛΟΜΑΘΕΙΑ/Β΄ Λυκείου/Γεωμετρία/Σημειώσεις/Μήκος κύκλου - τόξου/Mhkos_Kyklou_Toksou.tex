\PassOptionsToPackage{no-math,cm-default}{fontspec}
\documentclass[twoside,nofonts,internet,shmeiwseis]{thewria}
\usepackage{amsmath}
\usepackage{xgreek}
\let\hbar\relax
\defaultfontfeatures{Mapping=tex-text,Scale=MatchLowercase}
\setmainfont[Mapping=tex-text,Numbers=Lining,Scale=1.0,BoldFont={Minion Pro Bold}]{Minion Pro}
\newfontfamily\scfont{GFS Artemisia}
\font\icon = "Webdings"
\usepackage[amsbb,subscriptcorrection,zswash,mtpcal,mtphrb]{mtpro2}
\usepackage{tikz,pgfplots}
\tkzSetUpPoint[size=7,fill=white]
\xroma{red!70!black}
%------TIKZ - ΣΧΗΜΑΤΑ - ΓΡΑΦΙΚΕΣ ΠΑΡΑΣΤΑΣΕΙΣ ----
\usepackage{tikz}
\usepackage{tkz-euclide}
\usetkzobj{all}
\usepackage[framemethod=TikZ]{mdframed}
\usetikzlibrary{decorations.pathreplacing}
\usepackage{pgfplots}
\usetkzobj{all}
%-----------------------
\usepackage{calc}
\usepackage{hhline}
\usepackage[explicit]{titlesec}
\usepackage{graphicx}
\usepackage{multicol}
\usepackage{multirow}
\usepackage{enumitem}
\usepackage{tabularx}
\usepackage[decimalsymbol=comma]{siunitx}
\usetikzlibrary{backgrounds}
\usepackage{sectsty}
\sectionfont{\centering}
\setlist[enumerate]{label=\bf{\large \arabic*.}}
\usepackage{adjustbox}
\usepackage{mathimatika,gensymb,eurosym,wrap-rl}
\usepackage{systeme,regexpatch}
%-------- ΜΑΘΗΜΑΤΙΚΑ ΕΡΓΑΛΕΙΑ ---------
\usepackage{mathtools}
%----------------------
%-------- ΠΙΝΑΚΕΣ ---------
\usepackage{booktabs}
%----------------------
%----- ΥΠΟΛΟΓΙΣΤΗΣ ----------
\usepackage{calculator}
%----------------------------
%------ ΔΙΑΓΩΝΙΟ ΣΕ ΠΙΝΑΚΑ -------
\usepackage{array}
\newcommand\diag[5]{%
\multicolumn{1}{|m{#2}|}{\hskip-\tabcolsep
$\vcenter{\begin{tikzpicture}[baseline=0,anchor=south west,outer sep=0]
\path[use as bounding box] (0,0) rectangle (#2+2\tabcolsep,\baselineskip);
\node[minimum width={#2+2\tabcolsep-\pgflinewidth},
minimum  height=\baselineskip+#3-\pgflinewidth] (box) {};
\draw[line cap=round] (box.north west) -- (box.south east);
\node[anchor=south west,align=left,inner sep=#1] at (box.south west) {#4};
\node[anchor=north east,align=right,inner sep=#1] at (box.north east) {#5};
\end{tikzpicture}}\rule{0pt}{.71\baselineskip+#3-\pgflinewidth}$\hskip-\tabcolsep}}
%---------------------------------
%---- ΟΡΙΖΟΝΤΙΟ - ΚΑΤΑΚΟΡΥΦΟ - ΠΛΑΓΙΟ ΑΓΚΙΣΤΡΟ ------
\newcommand{\orag}[3]{\node at (#1)
{$ \overcbrace{\rule{#2mm}{0mm}}^{{\scriptsize #3}} $};}
\newcommand{\kag}[3]{\node at (#1)
{$ \undercbrace{\rule{#2mm}{0mm}}_{{\scriptsize #3}} $};}
\newcommand{\Pag}[4]{\node[rotate=#1] at (#2)
{$ \overcbrace{\rule{#3mm}{0mm}}^{{\rotatebox{-#1}{\scriptsize$#4$}}}$};}
%-----------------------------------------
%------------------------------------------
\newcommand{\tss}[1]{\textsuperscript{#1}}
\newcommand{\tssL}[1]{\MakeLowercase{\textsuperscript{#1}}}
%---------- ΛΙΣΤΕΣ ----------------------
\newlist{bhma}{enumerate}{3}
\setlist[bhma]{label=\bf\textit{\arabic*\textsuperscript{o}\;Βήμα :},leftmargin=0cm,itemindent=1.8cm,ref=\bf{\arabic*\textsuperscript{o}\;Βήμα}}
\newlist{rlist}{enumerate}{3}
\setlist[rlist]{itemsep=0mm,label=\roman*.}
\newlist{brlist}{enumerate}{3}
\setlist[brlist]{itemsep=0mm,label=\bf\roman*.}
\newlist{tropos}{enumerate}{3}
\setlist[tropos]{label=\bf\textit{\arabic*\textsuperscript{oς}\;Τρόπος :},leftmargin=0cm,itemindent=2.3cm,ref=\bf{\arabic*\textsuperscript{oς}\;Τρόπος}}
% Αν μπει το bhma μεσα σε tropo τότε
%\begin{bhma}[leftmargin=.7cm]
\tkzSetUpPoint[size=7,fill=white]
\tikzstyle{pl}=[line width=0.3mm]
\tikzstyle{plm}=[line width=0.4mm]
\usepackage{etoolbox}
\makeatletter
\renewrobustcmd{\anw@true}{\let\ifanw@\iffalse}
\renewrobustcmd{\anw@false}{\let\ifanw@\iffalse}\anw@false
\newrobustcmd{\noanw@true}{\let\ifnoanw@\iffalse}
\newrobustcmd{\noanw@false}{\let\ifnoanw@\iffalse}\noanw@false
\renewrobustcmd{\anw@print}{\ifanw@\ifnoanw@\else\numer@lsign\fi\fi}
\makeatother

\begin{document}
\titlos{Γεωμετρία Β΄ Λυκείου}{Μέτρηση κύκλου}{Μήκος κύκλου - Μήκος τόξου}
\orismoi
\Orismos{Μήκος κύκλου}
Μήκος ενός κύκλου $ (O,R) $ ονομάζεται ο θετικός αριθμός $ L $ ο οποίος είναι το όριο των ακολουθιών των περιμέτρων $ (P_\nu) $ των εγγεγραμμένων και $ (P_{\nu}') $ των περιγεγραμμένων κανονικών $ \nu- $γωνων καθών το πλήθος $ \nu $ των πλευρών αυξάνεται. Ισούται με 
\[ L=2\pi R \]
\Orismos{Εγγεγραμμένη - Περιγεγραμμένη τεθλασμένη γραμμή}
\vspace{-5mm}
\begin{enumerate}[label=\bf\arabic*.,itemsep=0mm]
\item \textbf{Εγγεγραμμένη τεθλασμένη γραμμή}\\
\wrapr{-5mm}{7}{3cm}{-15mm}{\begin{tikzpicture}
\draw (0,0) circle (1.2);
\coordinate (A)  at (140:1.2);
\coordinate (B) at (90:1.2);
\coordinate (C) at (50:1.2);
\coordinate (D) at (-10:1.2);
\coordinate (E) at (-80:1.2);
\coordinate (F) at (180:1.2);
\draw[pl,\xrwma] (A)--(B)--(C)--(D);
\draw[dashed,\xrwma,pl] (D)--(E);
\draw[dashed,\xrwma,pl] (A)--(F);
\tkzLabelPoint[above left](A){$A$}
\tkzLabelPoint[above ](B){$B$}
\tkzLabelPoint[above right](C){$\varGamma$}
\tkzLabelPoint[right](D){$\varDelta$}
\tkzDrawPoints(A,B,C,D)
\end{tikzpicture}}{
Εγγεγραμμένη σε έναν κύκλο $ (O,R) $ ονομάζεται μια τεθλασμένη γραμμή η οποία αποτελείται από χορδές του κύκλου.}
\wrapl{-5mm}{7}{3cm}{-5mm}{\begin{tikzpicture}
\draw (0,0) circle (1.1);
\tkzDefPoint(0,0){O};
\coordinate (A)  at (150:1.1);
\tkzTangent[at=A](O)\tkzGetPoint{a}
\coordinate (B)  at (95:1.1);
\tkzTangent[at=B](O)\tkzGetPoint{b}
\coordinate (C)  at (50:1.1);
\tkzTangent[at=C](O)\tkzGetPoint{c}
\coordinate (D)  at (-10:1.1);
\tkzTangent[at=D](O)\tkzGetPoint{d}
\tkzInterLL(A,a)(B,b)\tkzGetPoint{k}
\tkzInterLL(C,c)(B,b)\tkzGetPoint{m}
\tkzInterLL(C,c)(D,d)\tkzGetPoint{l}
\draw[pl,\xrwma] (a)--(k)--(m)--(l);
\tkzDrawLine[add=0 and .8,color=\xrwma](l,D)
\tkzLabelPoint[above left](k){$A$}
\tkzLabelPoint[above ](m){$B$}
\tkzLabelPoint[right](l){$\varGamma$}
\tkzDrawPoints(k,m,l)
\end{tikzpicture}}{\item \textbf{Περιγεγραμμένη τεθλασμένη γραμμή}\\
Περιγεγραμμένη σε έναν κύκλο $ (O,R) $ ονομάζεται μια τεθλασμένη γραμμή η οποία αποτελείται από εφαμτόμενα τμήματα του κύκλου.}
\end{enumerate}\mbox{}\\\\\\
\Orismos{Μήκος τόξου}
Μήκος ενός τόξου $ \widearc{AB} $ ονομάζεται ο θετικός αριθμός $ \mathcal{l} $ ο οποίος είναι το όριο των ακολουθιών των μηκών $ (P_\nu) $ των εγγεγραμμένων και $ (P_{\nu}') $ των περιγεγραμμένων τεθλασμένων γραμμών του τόξου καθώς αυξάνεται το πλήθος των τμημάτων τους. Ισούται με \[ \mathcal{l}=\pi R\cdot\frac{\mu}{180}=aR \]
όπου $ \mu $ είναι το μέτρο του τόξου σε μοίρες και $ a $ το μέτρο του σε ακτίνια.
\thewrhmata
\Thewrhma{Μετατροπή μοιρών σε ακτίνια}
Αν $ \mu $ είναι το μέτρο μιας γωνίας σε μοίρες και $ a $ το μέτρο της ίδιας γωνίας σε ακτίνια, η σχέση που τα συνδέει και με την οποία μπορούμε να μετατρέψουμε το μέτρο μιας γωνίας από μοίρες σε ακτίνια και αντίστροφα είναι :
\[ \frac{\mu}{180\degree}=\frac{a}{\pi} \]
\end{document}
