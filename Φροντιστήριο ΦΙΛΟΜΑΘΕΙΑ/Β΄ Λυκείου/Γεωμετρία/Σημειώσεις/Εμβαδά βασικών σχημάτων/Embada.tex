\PassOptionsToPackage{no-math,cm-default}{fontspec}
\documentclass[twoside,nofonts,internet,shmeiwseis]{thewria}
\usepackage{amsmath}
\usepackage{xgreek}
\let\hbar\relax
\defaultfontfeatures{Mapping=tex-text,Scale=MatchLowercase}
\setmainfont[Mapping=tex-text,Numbers=Lining,Scale=1.0,BoldFont={Minion Pro Bold}]{Minion Pro}
\newfontfamily\scfont{GFS Artemisia}
\font\icon = "Webdings"
\usepackage[amsbb]{mtpro2}
\usepackage{tikz,pgfplots}
\tkzSetUpPoint[size=7,fill=white]
\xroma{red!70!black}
\usetkzobj{all}
\newlist{rlist}{enumerate}{3}
\setlist[rlist]{itemsep=0mm,label=\roman*.}
\newlist{brlist}{enumerate}{3}
\setlist[brlist]{itemsep=0mm,label=\bf\roman*.}
\newlist{tropos}{enumerate}{3}
\setlist[tropos]{label=\bf\textit{\arabic*\textsuperscript{oς}\;Τρόπος :},leftmargin=0cm,itemindent=2.3cm,ref=\bf{\arabic*\textsuperscript{oς}\;Τρόπος}}
\newcommand{\tss}[1]{\textsuperscript{#1}}
\newcommand{\tssL}[1]{\MakeLowercase{\textsuperscript{#1}}}

\usepackage{hhline}
%----------- ΓΡΑΦΙΚΕΣ ΠΑΡΑΣΤΑΣΕΙΣ ---------
\pgfkeys{/pgfplots/aks_on/.style={axis lines=center,
xlabel style={at={(current axis.right of origin)},xshift=1.5ex, anchor=center},
ylabel style={at={(current axis.above origin)},yshift=1.5ex, anchor=center}}}
\pgfkeys{/pgfplots/grafikh parastash/.style={\xrwma,line width=.4mm,samples=200}}
\pgfkeys{/pgfplots/belh ar/.style={tick label style={font=\scriptsize},axis line style={-latex}}}
%-----------------------------------------
\usepackage{multicol}
\usepackage{wrap-rl,gensymb,mathimatika}


\begin{document}
\titlos{Γεωμετρία Β΄ Λυκείου}{Εμβαδά}{Εμβαδά βασικών σχημάτων}
\orismoi
\Orismos{Πολυγωνικό χωρίο}
Πολυγωνικό χωρίο ονομάζεται το σύνολο των σημείων ενός πολυγώνου μαζί με τα εσωτερικά του σημεία.
\begin{itemize}[itemsep=0mm]
\item Κάθε πολυγωνικό χωρίο παόρνει το όνομά του από το όνομα του αντίστοιχου πολυγώνου : τριγωνικό, τετραπλευρικό, πενταγωνικό και γενικά ν-γωνικό χωρίο.
\item Η επιφάνεια που αποτελείται από πεπερασμένο πλήθος πολυγωνικών χωρίων με κοινές πλευρές χωρίς κοινά εσωτερικά σημεία ονομάζεται \textbf{πολυγωνική επιφάνεια}.
\end{itemize}
\Orismos{Μονάδα μέτρησης επιφάνειας}
Μονάδα μέτρησης επιφάνειας ονομάζεται το μέγεθος ενός πολυγωνικού χωρίου το οποίο χρησιμοποιείται για τη μέτρηση και σύγκριση όλων των πολυγωνικών χωρίων.\\\\
\Orismos{Εμβαδόν}
Εμβαδόν ενός πολυγωνικού χωρίου ονομάζεται ο θετικός αριθμός με τον οποίο πολλαπλασιάζουμε τη μονάδα μέτρησης επιφάνειας ώστε να καλύψουμε το χωρίο αυτό.\\\\
\Orismos{Ισοδύναμα χωρία}
Ισοδύναμα ή ισεμβαδικά ονομάζονται τα χωρία τα οποία έχουν ίσα εμβαδά.
\thewrhmata
\Thewrhma{Αξιώματα πολυγωνικών χωρίων}
Δεχόμαστε τις εξής προτάσεις που αφορούν τα πολυγωνικά χωρία και τις πολυγωνικές επιφάνειες.
\begin{rlist}
\item Ίσα πολυγωνικά χωρία έχουν ίσα εμβαδά.
\item Αν ένα πολυγωνικό χωρίο ή πολυγωνική επιφάνεια χωριστεί σε πεπερασμένο πλήθος χωρίων χωρίς εσωτερικά σημεία, το εμβαδόν του ισούται με το άθροισμα των εμβαδών των επιμέρους χωρίων.
\item Το εμβαδόν τετραγώνου πλευράς 1 ισούται με 1.
\item Αν ένα χωρίο $ P $ βρίσκεται στο εσωτερικό ενός χωρίου $ Q $ τότε το εμβαδόν του $ P $ είναι μικρότερο από το εμβαδόν του $ Q $.
\end{rlist}
\newpage
\noindent
\Thewrhma{Εμβαδά βασικών σχημάτων}
Τα βασικά πολυγωνικά χωρία που συναντάμε είναι το τετράγωνο, το ορθογώνιο, το παραλληλόγραμμο, το τρίγωνο, το τραπέζιο και ο ρόμβος. Τα εμβαδά τους είναι τα εξής :
\begin{enumerate}[itemsep=0mm,label=\bf\arabic*.]
\item \textbf{Τετράγωνο}\\
Το εμβαδόν ενός τετραγώνου πλευράς $ a $ ισούται με το τετράγωνο της πλευράς του: $ E=a^2 $.
\item \textbf{Ορθογώνιο}\\
Το εμβαδόν ενός ορθογωνίου με διαστάσεις $ a,\beta $ ισούται με το γινόμενο του μήκους επί του πλάτους του.
\[ E=a\cdot \beta \]
\item \textbf{Παραλληλόγραμμο}\\
Το εμβαδόν ενός παραλληλογράμμου ισούται με το γινόμενο μιας πλευράς επί το αντίστοιχο ύψος της
\[ E=a\cdot\upsilon_a=\beta\cdot\upsilon_\beta \]
\begin{center}
\begin{tikzpicture}[scale=.7]
\tkzDefPoint(0,0){D}
\tkzDefPoint(3,0){C}
\tkzDefPoint(3,3){B}
\tkzDefPoint(0,3){A}
\draw[pl] (A)--(B)--(C)--(D)--cycle;
\tkzDrawPoints(A,B,C,D)
\tkzLabelPoint[above left](A){$A$}
\tkzLabelPoint[above right](B){$B$}
\tkzLabelPoint[below right](C){$\varGamma$}
\tkzLabelPoint[below left](D){$\varDelta$}
\node at (1.5,1.5) {$E=a^2$};
\node at (1.5,-0.25) {$a$};
\node at (3.25,1.5) {$a$};
\node at (1.5,3.25) {$a$};
\node at (-0.25,1.5) {$a$};
\end{tikzpicture}\quad\begin{tikzpicture}[scale=.7]
\tkzDefPoint(0,0){D}
\tkzDefPoint(4,0){C}
\tkzDefPoint(4,3){B}
\tkzDefPoint(0,3){A}
\draw[pl] (A)--(B)--(C)--(D)--cycle;
\tkzDrawPoints(A,B,C,D)
\tkzLabelPoint[above left](A){$A$}
\tkzLabelPoint[above right](B){$B$}
\tkzLabelPoint[below right](C){$\varGamma$}
\tkzLabelPoint[below left](D){$\varDelta$}
\node at (2,1.5) {$E=a\cdot\beta$};
\node at (2,-0.5) {$\beta$};
\node at (4.25,1.5) {$a$};
\node at (2,3.25) {$\beta$};
\node at (-0.25,1.5) {$a$};
\end{tikzpicture}\quad\begin{tikzpicture}[scale=.7]
\tkzDefPoint(0,0){D}
\tkzDefPoint(4,0){C}
\tkzDefPoint(5,3){B}
\tkzDefPoint(1,3){A}
\tkzDefPoint(1.5,0){a}
\tkzDefPoint(1.5,3){b}
\tkzDefPoint(4.5,1.5){c}
\tkzDefPoint(.9,2.7){d}
\tkzMarkRightAngle(C,a,b)
\tkzMarkRightAngle(D,d,c)
\draw[pl] (A)--(B)--(C)--(D)--cycle;
\tkzDrawPoints(A,B,C,D)
\tkzLabelPoint[above left](A){$A$}
\tkzLabelPoint[above right](B){$B$}
\tkzLabelPoint[below right](C){$\varGamma$}
\tkzLabelPoint[below left](D){$\varDelta$}
\node at (3,1.5) {$E=a\cdot\upsilon_a$};
\node at (3,1) {$E=\beta\cdot\upsilon_\beta$};
\node at (2,-0.5) {$\beta$};
\node at (5,1.5) {$a$};
\node at (3,3.25) {$\beta$};
\node at (0,1.5) {$a$};
\draw (a) -- (b);
\draw (c) -- (d);
\node at (2.5,2.35) {\footnotesize$\upsilon_a$};
\node at (1.25,0.5) {\footnotesize$\upsilon_\beta$};
\end{tikzpicture}
\end{center}
\item \textbf{Τρίγωνο}\\
\wrapr{-7mm}{5}{4.4cm}{-11mm}{\begin{tikzpicture}
\clip (-.5,-.52) rectangle (4,2.5);
\tkzDefPoint(0,0){B}
\tkzDefPoint(3.5,0){C}
\tkzDefPoint(1.,2.1){A}
\tkzDefPointBy[projection = onto A--B](C) \tkzGetPoint{M}
\tkzDefPointBy[projection = onto A--C](B) \tkzGetPoint{L}
\tkzDefPoint(1,0){K}
\tkzInterLL(A,K)(B,L)\tkzGetPoint{H}
\tkzMarkRightAngle[size=.2](C,K,A)
\tkzMarkRightAngle[size=.2](C,M,A)
\tkzMarkRightAngle[size=.2](B,L,A)
\draw[pl](A)--(B)--(C)--cycle;
\tkzDrawAltitude[draw=\xrwma](A,B)(C)
\tkzDrawAltitude[draw=\xrwma](A,C)(B)
\tkzDrawAltitude[draw=\xrwma](B,C)(A)
\tkzDrawPoints(A,B,C,K,L,M)
\tkzLabelPoint[above](A){$A$}
\tkzLabelPoint[left](B){$B$}
\tkzLabelPoint[right](C){$\varGamma$}
\tkzLabelPoint[below](K){$K$}
\tkzLabelPoint[right,yshift=1mm](L){$\varLambda$}
\tkzLabelPoint[left](M){$M$}
\node at (1.25,0.5) {\footnotesize$\upsilon_a$};
\node at (1.35,1.25) {\footnotesize$\upsilon_\beta$};
\node at (2,0.5) {\footnotesize$\upsilon_\gamma$};
\end{tikzpicture}}{
Το εμβαδόν ενός τριγώνου ισούται με το μισό του γινομένου μιας πλευράς επί το αντίστοιχο ύψος της.
\[ E=\frac{1}{2}a\cdot\upsilon_a=\frac{1}{2}\beta\cdot\upsilon_\beta=\frac{1}{2}\gamma\cdot\upsilon_\gamma \]
\begin{itemize}[itemsep=0mm]
\item Το εμβαδόν ενός ορθογωνίου τριγώνου ισούται με το ημιγινόμενο των κάθετων πλευρών του.
\item Το εμβαδόν ενός ισόπλευρου τριγώνου πλευράς $ a $ ισούται με $ E=\frac{a^2\sqrt{3}}{4} $.
\end{itemize}}
\item \textbf{Τραπέζιο}\\
\wrapr{-7mm}{5}{4.1cm}{-4mm}{\begin{tikzpicture}
\tkzDefPoint(0,-1.5){D}
\tkzDefPoint(0.5,.5){A}
\tkzDefPoint(2.5,.5){B}
\tkzDefPoint(3.5,-1.5){C}
\tkzDefPoint(.25,-.5){M}
\tkzDefPoint(3,-.5){N}
\tkzDefPoint(0.9,0.5){E}
\tkzDefPoint(0.9,-1.5){Z}
\tkzMarkRightAngle(C,Z,E)
\draw (0.9,0.5) -- (0.9,-1.5);
\draw[pl] (0,-1.5) -- (0.5,0.5) -- (2.5,0.5) -- (3.5,-1.5) -- cycle;
\draw[plm,\xrwma](M)--(N);
\tkzLabelPoint[above](A){$A$}
\tkzLabelPoint[above](B){$B$}
\tkzLabelPoint[below](C){$\varGamma$}
\tkzLabelPoint[below](D){$\varDelta$}
\tkzLabelPoint[left](M){$M$}
\tkzLabelPoint[right](N){$N$}
\tkzDrawPoints(A,B,C,D,M,N)
\node at (1.5,0.7) {\footnotesize$\beta$};
\node at (1.7,-1.8) {\footnotesize$B$};
\node at (.7,-.2) {\footnotesize$ \upsilon $};
\node at (1.75,-.35) {\footnotesize$ \delta $};
\end{tikzpicture}}{
Το εμβαδόν ενός τραπεζίου ισούται με το γινόμενο του αθροίσματος των βάσεων επί το μισό του ύψους του.
\[ E=\frac{(\beta+B)\cdot\upsilon}{2}=\delta\cdot\upsilon \]
Ισούται επίσης με το γινόμενο της διαμέσου επί το ύψος του.}
\item \textbf{Ρόμβος}\\
\wrapr{-7mm}{7}{5.2cm}{-9mm}{\begin{tikzpicture}[scale=.7]
\tkzDefPoint(0,1.5){D}
\tkzDefPoint(3,3){A}
\tkzDefPoint(6,1.5){B}
\tkzDefPoint(3,0){C}
\tkzDefPoint(3,1.5){O}
\tkzMarkRightAngle[size=.4](B,O,A)
\draw[pl] (A)--(B)--(C)--(D) -- cycle;
\draw[pl] (A)--(C);
\draw[pl] (B)--(D);
\tkzLabelPoint[above](A){$A$}
\tkzLabelPoint[right](B){$B$}
\tkzLabelPoint[below](C){$\varGamma$}
\tkzLabelPoint[left](D){$\varDelta$}
\tkzLabelPoint[above left](O){$O$}
\tkzDrawPoints(A,B,C,D,O)
\node at (2,1.8) {\footnotesize$\delta_1$};
\node at (3.4,.8) {\footnotesize$\delta_2$};
\end{tikzpicture}}{
Το εμβαδόν ενός ρόμβου ισούται με το ημιγινόμενο των διαγωνίων του.
\[ E=\frac{\delta_1\cdot\delta_2}{2} \]
Γενικότερα το εμβαδόν οποιουδήποτε τετραπλεύρου με κάθετες διαγώνιους ισούται με το ημιγινόμενο των διαγωνίων του.}
\end{enumerate}
\Thewrhma{Διάμεσος - Ισεμβαδικά τρίγωνα}
\wrapr{-5mm}{4}{4.1cm}{-12mm}{\begin{tikzpicture}
\tkzDefPoint(0,0){B}
\tkzDefPoint(1,2){A}
\tkzDefPoint(3,0){C}
\tkzDefPoint(1.5,0){M}
\draw[pl] (A)--(B)--(C)--cycle;
\draw[pl,\xrwma] (A)--(M);
\tkzDrawPoints(A,B,C,M)
\tkzLabelPoint[above](A){$A$}
\tkzLabelPoint[left](B){$B$}
\tkzLabelPoint[right](C){$\varGamma$}
\tkzLabelPoint[below](M){$M$}
\end{tikzpicture}}{
Σε κάθε τρίγωνο, οποιαδήποτε διάμεσος χωρίζει το τρίγωνο σε ισεμβαδικά μέρη.
\[ (AMB)=(AM\varGamma) \]}
\end{document}
