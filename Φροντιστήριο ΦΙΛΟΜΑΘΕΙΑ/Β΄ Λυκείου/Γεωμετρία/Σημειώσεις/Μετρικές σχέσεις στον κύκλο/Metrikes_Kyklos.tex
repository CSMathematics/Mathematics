\PassOptionsToPackage{no-math,cm-default}{fontspec}
\documentclass[twoside,nofonts,internet,shmeiwseis]{thewria}
\usepackage{amsmath}
\usepackage{xgreek}
\let\hbar\relax
\defaultfontfeatures{Mapping=tex-text,Scale=MatchLowercase}
\setmainfont[Mapping=tex-text,Numbers=Lining,Scale=1.0,BoldFont={Minion Pro Bold}]{Minion Pro}
\newfontfamily\scfont{GFS Artemisia}
\font\icon = "Webdings"
\usepackage[amsbb]{mtpro2}
\usepackage{tikz,pgfplots}
\tkzSetUpPoint[size=7,fill=white]
\xroma{red!70!black}
\newlist{rlist}{enumerate}{3}
\setlist[rlist]{itemsep=0mm,label=\roman*.}
\newlist{brlist}{enumerate}{3}
\setlist[brlist]{itemsep=0mm,label=\bf\roman*.}
\newlist{tropos}{enumerate}{3}
\setlist[tropos]{label=\bf\textit{\arabic*\textsuperscript{oς}\;Τρόπος :},leftmargin=0cm,itemindent=2.3cm,ref=\bf{\arabic*\textsuperscript{oς}\;Τρόπος}}
\newcommand{\tss}[1]{\textsuperscript{#1}}
\newcommand{\tssL}[1]{\MakeLowercase{\textsuperscript{#1}}}
\usepackage{hhline}
\usepackage{multicol,gensymb,mathimatika}
\usepackage{wrap-rl}


\begin{document}
\titlos{Γεωμετρία Β΄ Λυκείου}{Μετρικές Σχέσεις}{Μετρικές σχέσεις στον κύκλο}
\orismoi
\Orismos{Δύναμη σημείου ως προς κύκλο}
Δύναμη ενός σημείου $ M $ ως προς ένα κύκλο $ (O,R) $ ονομάζεται η διαφορά
\[ \Delta_{_{(O,R)}}^{^{M}}=\delta^2-R^2 \]
όπου $ \delta $ είναι η απόσταση του σημείου $ M $ από το κέντρο του κύκλου : $ OM=\delta $. Συμβολίζεται με $ \Delta_{_{(O,R)}}^{^{M}} $.
\thewrhmata
\Thewrhma{Τέμνουσες κύκλου}
Έστω $ AB $ και $ \varGamma\varDelta $ δύο χορδές ενός κύκλου $ (O,\rho) $. Αν $ M $ είναι το σημείο τομής αυτών ή των προεκτάσεών τους τότε τα γινόμενα των τμημάτων που ορίζει το σημείο τομής με τα άκρα κάθε χορδής είναι μεταξύ τους ίσα.
\[ MA\cdot MB=M\varGamma\cdot M\varDelta \]
\begin{center}
\begin{tabular}{p{4cm}p{4cm}}
\begin{tikzpicture}
\tkzDefPoint(0,0){O}
\tkzDefPoint(30:1.25){A}
\tkzDefPoint(160:1.25){B}
\tkzDefPoint(110:1.25){C}
\tkzDefPoint(290:1.25){D}
\tkzInterLL(A,B)(C,D)\tkzGetPoint{M}
\draw[pl]  (O) circle (1.25);
\draw[pl,\xrwma](A)--(B);
\draw[pl,\xrwma](C)--(D);
\tkzDrawPoints(A,B,C,D,M)
\tkzLabelPoint[above right](A){$A$}
\tkzLabelPoint[above left](B){$B$}
\tkzLabelPoint[above](C){$\varGamma$}
\tkzLabelPoint[below](D){$\varDelta$}
\tkzLabelPoint[above right](M){$M$}
\end{tikzpicture} & \begin{tikzpicture}
\tkzDefPoint(0,0){O}
\tkzDefPoint(30:1.25){A}
\tkzDefPoint(150:1.25){B}
\tkzDefPoint(110:1.25){C}
\tkzDefPoint(290:1.25){D}
\tkzInterLL(A,C)(B,D)\tkzGetPoint{M}
\draw[pl]  (O) circle (1.25);
\draw[pl,\xrwma](A)--(M);
\draw[pl,\xrwma](D)--(M);
\tkzDrawPoints(A,B,C,D,M)
\tkzLabelPoint[above right](A){$A$}
\tkzLabelPoint[left,yshift=-1mm](B){$\varGamma$}
\tkzLabelPoint[above](C){$B$}
\tkzLabelPoint[below](D){$\varDelta$}
\tkzLabelPoint[above right](M){$M$}
\end{tikzpicture} \\ 
\end{tabular} 
\end{center}
\Thewrhma{Ακρα εγγράψιμου τετραπλεύρου}
Έστω δύο ευθύγραμμα τμήματα $ AB $ και $ \varGamma\varDelta $ και $ M $ το σημείο τομής αυτών ή των προεκτάσεων τους. Αν τα γινόμενα των τμημάτων που ορίζει το σημείο τομής με τα άκρα κάθε τμήματος να είναι ίσα τότε το τετράπλευρο $ AB\varGamma\varDelta $ είναι εγγράψιμο.
\[ MA\cdot MB=M\varGamma\cdot M\varDelta\Rightarrow AB\varGamma\varDelta\ \ \textrm{εγγράψιμο} \]
\Thewrhma{Τέμνουσα και εφαπτόμενη}
\wrapr{-5mm}{5}{4cm}{-12mm}{\begin{tikzpicture}
\tkzDefPoint(0,0){O}
\tkzDefPoint(-50:1.25){A}
\tkzDefPoint(170:1.25){B}
\tkzDefPoint(110:1.25){C}
\tkzTangent[at=C](O)
\tkzGetPoint{h}
\tkzInterLL(C,h)(A,B)\tkzGetPoint{M}
\draw[pl,\xrwma](A)--(M)--(C);
\draw[pl]  (O) circle (1.25);
\tkzDrawPoints(A,B,C,M)
\tkzLabelPoint[below right](A){$A$}
\tkzLabelPoint[left,yshift=-1mm](B){$B$}
\tkzLabelPoint[above](C){$\varGamma$}
\tkzLabelPoint[left](M){$M$}
\end{tikzpicture}}{
Έστω $ AB $ μια χορδή ενός κύκλου $ (O,\rho) $ και $ \varGamma $ ένα σημείο του κύκλου. Αν $ M $ είναι ένα εξωτερικό σημείο του κύκλου τότε το γινόμενο των τμημάτων που ορίζει το σημείο τομής με τα άκρα της χορδής είναι ίσο με το τετράγωνο του εφαπτόμενου τμήματος.
\[ M\varGamma^2=MA\cdot MB \]}\mbox{}\\\\\\
\Thewrhma{Δύναμη σημείου ως προς κύκλο}
Έστω ένας κύκλος $ (O,R) $ και $ M $ ένα σημείο του επιπέδου του κύκλου.
\begin{rlist}
\item Η δύναμη $ \Delta_{_{(O,R)}}^{^{M}} $ του σημείου $ M $ ως προς τον κύκλο είναι θετική αν και μόνο αν το σημείο είναι εξωτερικό του κύκλου.
\[ \Delta_{_{(O,R)}}^{^{M}}>0\Leftrightarrow \delta>R \]
\item Η δύναμη $ \Delta_{_{(O,R)}}^{^{M}} $ του σημείου $ M $ ως προς τον κύκλο είναι μηδενική αν και μόνο αν το σημείο είναι πάνω στον κύκλο.
\[ \Delta_{_{(O,R)}}^{^{M}}=0\Leftrightarrow \delta=R \]
\item Η δύναμη $ \Delta_{_{(O,R)}}^{^{M}} $ του σημείου $ M $ ως προς τον κύκλο είναι αρνητική αν και μόνο αν το σημείο είναι εσωτερικό του κύκλου.
\[ \Delta_{_{(O,R)}}^{^{M}}<0\Leftrightarrow \delta<R \]
\end{rlist}
\end{document}
