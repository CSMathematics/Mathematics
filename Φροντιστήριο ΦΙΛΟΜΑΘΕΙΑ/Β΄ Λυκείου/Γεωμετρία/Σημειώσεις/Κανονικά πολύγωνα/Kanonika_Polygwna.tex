\PassOptionsToPackage{no-math,cm-default}{fontspec}
\documentclass[twoside,nofonts,internet,shmeiwseis]{thewria}
\usepackage{amsmath}
\usepackage{xgreek}
\let\hbar\relax
\defaultfontfeatures{Mapping=tex-text,Scale=MatchLowercase}
\setmainfont[Mapping=tex-text,Numbers=Lining,Scale=1.0,BoldFont={Minion Pro Bold}]{Minion Pro}
\newfontfamily\scfont{GFS Artemisia}
\font\icon = "Webdings"
\usepackage[amsbb,subscriptcorrection,zswash,mtpcal,mtphrb]{mtpro2}
\usepackage{tikz,pgfplots}
\tkzSetUpPoint[size=7,fill=white]
\xroma{red!70!black}
%------TIKZ - ΣΧΗΜΑΤΑ - ΓΡΑΦΙΚΕΣ ΠΑΡΑΣΤΑΣΕΙΣ ----
\usepackage{tikz}
\usepackage{tkz-euclide}
\usetkzobj{all}
\usepackage[framemethod=TikZ]{mdframed}
\usetikzlibrary{decorations.pathreplacing}
\usepackage{pgfplots}
\usetkzobj{all}
%-----------------------
\usepackage{calc}
\usepackage{hhline}
\usepackage[explicit]{titlesec}
\usepackage{graphicx}
\usepackage{multicol}
\usepackage{multirow}
\usepackage{enumitem}
\usepackage{tabularx}
\usepackage[decimalsymbol=comma]{siunitx}
\usetikzlibrary{backgrounds}
\usepackage{sectsty}
\sectionfont{\centering}
\setlist[enumerate]{label=\bf{\large \arabic*.}}
\usepackage{adjustbox}
\usepackage{mathimatika,gensymb,eurosym,wrap-rl}
\usepackage{systeme,regexpatch}
%-------- ΜΑΘΗΜΑΤΙΚΑ ΕΡΓΑΛΕΙΑ ---------
\usepackage{mathtools}
%----------------------
%-------- ΠΙΝΑΚΕΣ ---------
\usepackage{booktabs}
%----------------------
%----- ΥΠΟΛΟΓΙΣΤΗΣ ----------
\usepackage{calculator}
%----------------------------
%------ ΔΙΑΓΩΝΙΟ ΣΕ ΠΙΝΑΚΑ -------
\usepackage{array}
\newcommand\diag[5]{%
\multicolumn{1}{|m{#2}|}{\hskip-\tabcolsep
$\vcenter{\begin{tikzpicture}[baseline=0,anchor=south west,outer sep=0]
\path[use as bounding box] (0,0) rectangle (#2+2\tabcolsep,\baselineskip);
\node[minimum width={#2+2\tabcolsep-\pgflinewidth},
minimum  height=\baselineskip+#3-\pgflinewidth] (box) {};
\draw[line cap=round] (box.north west) -- (box.south east);
\node[anchor=south west,align=left,inner sep=#1] at (box.south west) {#4};
\node[anchor=north east,align=right,inner sep=#1] at (box.north east) {#5};
\end{tikzpicture}}\rule{0pt}{.71\baselineskip+#3-\pgflinewidth}$\hskip-\tabcolsep}}
%---------------------------------
%---- ΟΡΙΖΟΝΤΙΟ - ΚΑΤΑΚΟΡΥΦΟ - ΠΛΑΓΙΟ ΑΓΚΙΣΤΡΟ ------
\newcommand{\orag}[3]{\node at (#1)
{$ \overcbrace{\rule{#2mm}{0mm}}^{{\scriptsize #3}} $};}
\newcommand{\kag}[3]{\node at (#1)
{$ \undercbrace{\rule{#2mm}{0mm}}_{{\scriptsize #3}} $};}
\newcommand{\Pag}[4]{\node[rotate=#1] at (#2)
{$ \overcbrace{\rule{#3mm}{0mm}}^{{\rotatebox{-#1}{\scriptsize$#4$}}}$};}
%-----------------------------------------
%------------------------------------------
\newcommand{\tss}[1]{\textsuperscript{#1}}
\newcommand{\tssL}[1]{\MakeLowercase{\textsuperscript{#1}}}
%---------- ΛΙΣΤΕΣ ----------------------
\newlist{bhma}{enumerate}{3}
\setlist[bhma]{label=\bf\textit{\arabic*\textsuperscript{o}\;Βήμα :},leftmargin=0cm,itemindent=1.8cm,ref=\bf{\arabic*\textsuperscript{o}\;Βήμα}}
\newlist{rlist}{enumerate}{3}
\setlist[rlist]{itemsep=0mm,label=\roman*.}
\newlist{brlist}{enumerate}{3}
\setlist[brlist]{itemsep=0mm,label=\bf\roman*.}
\newlist{tropos}{enumerate}{3}
\setlist[tropos]{label=\bf\textit{\arabic*\textsuperscript{oς}\;Τρόπος :},leftmargin=0cm,itemindent=2.3cm,ref=\bf{\arabic*\textsuperscript{oς}\;Τρόπος}}
% Αν μπει το bhma μεσα σε tropo τότε
%\begin{bhma}[leftmargin=.7cm]
\tkzSetUpPoint[size=7,fill=white]
\tikzstyle{pl}=[line width=0.3mm]
\tikzstyle{plm}=[line width=0.4mm]
\usepackage{etoolbox}
\makeatletter
\renewrobustcmd{\anw@true}{\let\ifanw@\iffalse}
\renewrobustcmd{\anw@false}{\let\ifanw@\iffalse}\anw@false
\newrobustcmd{\noanw@true}{\let\ifnoanw@\iffalse}
\newrobustcmd{\noanw@false}{\let\ifnoanw@\iffalse}\noanw@false
\renewrobustcmd{\anw@print}{\ifanw@\ifnoanw@\else\numer@lsign\fi\fi}
\makeatother

\begin{document}
\titlos{Γεωμετρία Β΄ Λυκείου}{Μέτρηση κύκλου}{Κανονικά πολύγωνα}
\orismoi
\Orismos{Κανονικό πολύγωνο (\MakeLowercase{$ \mathbold\nu $}-γωνο)}
\wrapr{-4mm}{9}{3.5cm}{-4mm}{\begin{tikzpicture}
\draw(15:1.2) arc (15:-290:1.2);
\coordinate (O)  at (0,0);
\coordinate (A)  at (90:1.2);
\coordinate (B) at (135:1.2);
\coordinate (C) at (180:1.2);
\coordinate (D) at (225:1.2);
\coordinate (E) at (270:1.2);
\coordinate (F) at (315:1.2);
\coordinate (G) at (0:1.2);
\coordinate (H) at (45:1.2);
\draw[pl,\xrwma] (A)--(B)--(C)--(D)--(E)--(F)--(G);
\tkzDrawSegments[dashed,add=0 and -.4](A,H G,H);
\tkzMarkSegments[mark=|,size=.7mm](A,B B,C C,D D,E E,F F,G);
\tkzLabelPoint[above](A){$A$}
\tkzLabelPoint[above left](B){$B$}
\tkzLabelPoint[left](C){$\varGamma$}
\tkzLabelPoint[below left](D){$\varDelta$}
\tkzLabelPoint[below](E){$E$}
\tkzLabelPoint[below right](F){$Z$}
\tkzLabelPoint[right](G){$H$}
\tkzLabelPoint[above](O){$O$}
\tkzDrawPoints(O,A,B,C,D,E,F,G)
\end{tikzpicture}}{
Κανονικό ονομάζεται κάθε πολύγωνο το οποίο έχει όλες τις πλευρές του ίσες και όλες τις γωνίες του ίσες μεταξύ τους.
\begin{itemize}
\item Ένα κανονικό πολύγωνο συμβολίζεται $ \nu $-γωνο, όπου $ \nu $ είναι ο φυσικός αριθμός που καθορίζει το πλήθος των πλευρών του πολυγώνου με $ \nu\geq3 $.
\item Κάθε κανονικό πολύγωνο εγγράφεται σε έναν κύκλο και ο κύκλος αυτός ονομάζεται \textbf{κύκλος του πολυγώνου}.
\item Το κέντρο του περιγεγραμμένου κύκλου ονομάζεται \textbf{κέντρο του πολυγώνου}
\end{itemize}}\mbox{}\\\\\\
\Orismos{Στοιχεία πολυγώνου}
\wrapr{-5mm}{7}{4cm}{-8mm}{\begin{tikzpicture}
\draw (0,0) circle (1.5);
\draw (0,0) circle (1.299);
\coordinate (O)  at (0,0);
\coordinate (A)  at (120:1.5);
\coordinate (B) at (60:1.5);
\coordinate (C) at (0:1.5);
\coordinate (D) at (-60:1.5);
\coordinate (E) at (-120:1.5);
\coordinate (F) at (180:1.5);
\coordinate (G) at (30:1.299);
\tkzMarkAngle[size=.3](B,O,A)
\tkzMarkAngle[size=.25,fill=white,draw=black](E,F,A)
\tkzMarkRightAngle[size=.2](O,G,C)
\draw[pl,\xrwma] (A)--(B)--(C)--(D)--(E)--(F)--cycle;
\draw (B)--(O)--(A);
\draw (O)--(G);
\tkzLabelPoint[above left](A){$A$}
\tkzLabelPoint[above right](B){$B$}
\tkzLabelPoint[right](C){$\varGamma$}
\tkzLabelPoint[below right](D){$\varDelta$}
\tkzLabelPoint[below left](E){$E$}
\tkzLabelPoint[left](F){$Z$}
\tkzLabelPoint[below](O){$O$}
\tkzDrawPoints(O,A,B,C,D,E,F,G)
\node at (0,0.4979) {\footnotesize$\omega_\nu$};
\node at (-1,0) {\footnotesize$\varphi_\nu$};
\node at (0.6928,0.1444) {\footnotesize$a_\nu$};
\node at (0,-1.1) {\footnotesize$\lambda_\nu$};
\node at (-0.6547,0.783) {\footnotesize$R$};
\end{tikzpicture}}{
Τα στοιχεία ενός κανονικού $ \nu- $γωνου είναι τα εξής:
\begin{enumerate}[label=\bf\arabic*.]
\item \textbf{Κεντρική γωνία}\\
Η κεντρική γωνία είναι η γωνία που σχηματίζουν δύο ακτίνες του κύκλου του πολυγώνου που ενώνουν το κέντρο με δύο διαδοχικές κορυφές του. Συμβολίζεται με $ \omega_\nu $.
\end{enumerate}}\mbox{}\\
\vspace{-3mm}
\begin{enumerate}[label=\bf\arabic*.,start=2,itemsep=0mm]
\item \textbf{Γωνία πολυγώνου}\\
Η γωνία του πολυγώνου είναι η γωνία που σχηματίζουν δύο διαδοχικές πλευρές του. Συμβολίζεται $ \varphi_\nu $.
\item \textbf{Πλευρά πολυγώνου}\\
Η πλευρά ενός κανονικού πολυγώνου συμβολίζεται με $ \lambda_\nu $.
\item \textbf{Απόστημα πολυγώνου}\\
Το απόστημα ενός πολυγώνου είναι η ακτίνα του εγγεγραμμένου κύκλου του. Συμβολίζεται με $ a_\nu $.
\item \textbf{Κέντρο πολυγώνου}\\
Το κέντρο ενός κανονικού πολυγώνου είναι το κέντρο του περιγεγραμμένου κύκλου.
\item \textbf{Ακτίνα πολυγώνου}\\
Ακτίνα ενός κανονικού πολυγώνου ονομάζεται η ακτίνα του περιγεγραμμένου κύκλου. Συμβολίζεται $ R $.
\item \textbf{Περίμετρος - Εμβαδόν πολυγώνου}\\
Η περίμετρος ενός καονικού πολυγώνου συμβολίζεται με $ P_\nu $ ενώ το εμβαδόν του με $ E_\nu $.
\end{enumerate}
\thewrhmata
\Thewrhma{Σχέσεις στοιχείων πολυγώνου}
Για τα στοιχεία ενός κανονικού $ \nu- $γωνου ισχύουν οι παρακάτω σχέσεις:
\begin{multicols}{4}
\begin{rlist}
\item $ \omega_\nu=\dfrac{360\degree}{\nu} $
\item $ \varphi_\nu=180\degree-\omega_\nu $
\item $ a_\nu^2+\dfrac{\lambda_\nu^2}{4}=R^2 $
\item $ a_\nu=R\cdot\syn{\left( \frac{\omega_\nu}{2}\right) } $
\item $ \lambda_\nu=2R\cdot\hm{\left( \frac{\omega_\nu}{2}\right) } $
\item $ \lambda_\nu=2a_\nu\cdot\ef{\left( \frac{\omega_\nu}{2}\right)} $
\item $ P_\nu=\nu\cdot\lambda_\nu $
\item $ E_\nu=\dfrac{1}{2}P_\nu\cdot a_\nu $
\end{rlist}
\end{multicols}
\Thewrhma{Λόγος στοιχείων κανονικού πολυγώνου}
Ο λόγος των πλευρών, ο λόγος των ακτίνων και ο λόγος των αποστημάτων δύο κανονικών $ \nu- $γωνων ισούνται με το λόγο ομοιότητας τους.
\[ \frac{\lambda_\nu}{\lambda_\nu'}=\frac{R}{R'}=\frac{a_\nu}{a_\nu'} \]
\end{document}
