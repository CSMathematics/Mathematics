\PassOptionsToPackage{no-math,cm-default}{fontspec}
\documentclass[twoside,nofonts,internet,shmeiwseis]{thewria}
\usepackage{amsmath}
\usepackage{xgreek}
\let\hbar\relax
\defaultfontfeatures{Mapping=tex-text,Scale=MatchLowercase}
\setmainfont[Mapping=tex-text,Numbers=Lining,Scale=1.0,BoldFont={Minion Pro Bold}]{Minion Pro}
\newfontfamily\scfont{GFS Artemisia}
\font\icon = "Webdings"
\usepackage[amsbb]{mtpro2}
\usepackage{tikz,pgfplots,tkz-euclide}
\usetkzobj{all}
\tkzSetUpPoint[size=7,fill=white]
\xroma{red!70!black}
\usepackage{multicol,mathimatika,gensymb}
\usepackage{wrap-rl}
\newcommand{\tss}[1]{\textsuperscript{#1}}
\newcommand{\tssL}[1]{\MakeLowercase{\textsuperscript{#1}}}

\begin{document}
\titlos{Γεωμετρία Β΄ Λυκείου}{Μετρικές Σχέσεις}{Θεωρήματα διαμέσων}
\thewrhmata
\Thewrhma{1\tssL{o} Θεώρημα διαμέσων}
\wrapr{-5mm}{7}{4cm}{-7mm}{\begin{tikzpicture}
\tkzDefPoint(0,0){B}
\tkzDefPoint(3,0){C}
\tkzDefPoint(1,2){A}
\tkzDefPoint(.5,1){c}
\tkzDefPoint(1.5,0){a}
\tkzDefPoint(2,1){b}
\draw[pl] (A)--(B)--(C)--cycle;
\draw(A)--(a);
\draw(C)--(c);
\draw(B)--(b);
\tkzDrawPoints(A,B,C,a,b,c)
\tkzLabelPoint[above](A){$A$}
\tkzLabelPoint[left](B){$B$}
\tkzLabelPoint[right](C){$\varGamma$}
\tkzLabelPoint[below](a){$\varDelta$}
\tkzLabelPoint[left](c){$Z$}
\tkzLabelPoint[right](b){$E$}
\end{tikzpicture}}{
Το άθροισμα των τετραγώνων δύο πλευρών ενός τριγώνου ισούται με το διπλάσιο τετράγωνο της περιεχόμενης διαμέσου συν το μισό τετράγωνο της τρίτης πλευράς.
\[ a^2+\beta^2=2\mu_\gamma^2+\frac{\gamma^2}{2}\ \ ,\ \ a^2+\gamma^2=2\mu_\beta^2+\frac{\beta^2}{2}\ \ ,\ \ \beta^2+\gamma^2=2\mu_a^2+\frac{a^2}{2} \]}\mbox{}\\\\\\
Από τους παραπάνω τύπους μπορούμε να εκφράσουμε τις διαμέσους του τριγώνου με τη βοήθεια των πλευρών του ως εξής :
\[ \mu_a^2=\frac{2\beta^2+2\gamma^2-a^2}{4}\ \ ,\ \ \mu_\beta^2=\frac{2a^2+2\gamma^2-\beta^2}{4}\ \ ,\ \ \mu_\gamma^2=\frac{2a^2+2\beta^2-\gamma^2}{4} \]
\Thewrhma{2\tssL{ο} Θεώρημα Διαμέσων}
\wrapr{-5mm}{7}{4cm}{-7mm}{\begin{tikzpicture}
\tkzDefPoint(0,0){B}
\tkzDefPoint(3,0){C}
\tkzDefPoint(1,2){A}
\tkzDefPoint(.5,1){c}
\tkzDefPoint(1.5,0){a}
\tkzDefPoint(2,1){b}
\tkzDefPoint(1,0){D}
\tkzDefPointBy[projection=onto A--B](C)\tkzGetPoint{e}
\tkzDefPointBy[projection=onto A--C](B)\tkzGetPoint{d}
\tkzMarkRightAngle[size=.2](C,D,A)
\tkzMarkRightAngle[size=.2](B,d,C)
\tkzMarkRightAngle[size=.2](C,e,A)
\draw[pl] (A)--(B)--(C)--cycle;
\draw(D)--(A)--(a);
\draw(e)--(C)--(c);
\draw(d)--(B)--(b);
\draw[pl,\xrwma](D)--(a);
\draw[pl,\xrwma](b)--(d);
\draw[pl,\xrwma](e)--(c);
\tkzDrawPoints(A,B,C,a,b,c,D,d,e)
\tkzLabelPoint[above](A){$A$}
\tkzLabelPoint[left](B){$B$}
\tkzLabelPoint[right](C){$\varGamma$}
\tkzLabelPoint[below](a){$\varDelta$}
\tkzLabelPoint[left](c){$Z$}
\tkzLabelPoint[above right](b){$E$}
\tkzLabelPoint[below](D){$M$}
\tkzLabelPoint[above right](d){$K$}
\tkzLabelPoint[above left,xshift=1mm](e){$\varLambda$}
\end{tikzpicture}}{
Η διαφορά των τετραγώνων δύο πλευρών ισούται με το διπλάσιο γινόμενο της τρίτης πλευράς επί την προβολή της περιεχόμενης διαμέσου στην τρίτη πλευρά.
\[ a^2-\beta^2=2\gamma\cdot\varLambda Z\ \ ,\ \ a^2-\gamma^2=2\beta\cdot KE\ \ ,\ \ \beta^2-\gamma^2=2a\cdot M\varDelta \]}
\end{document}
