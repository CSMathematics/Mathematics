\documentclass[twoside,nofonts,internet,shmeiwseis]{thewria}
\usepackage[amsbb,subscriptcorrection,zswash,mtpcal,mtphrb,mtpfrak]{mtpro2}
\usepackage[no-math,cm-default]{fontspec}
\usepackage{amsmath}
\usepackage{xgreek}
\let\hbar\relax
\defaultfontfeatures{Mapping=tex-text,Scale=MatchLowercase}
\setmainfont[Mapping=tex-text,Numbers=Lining,Scale=1.0,BoldFont={Minion Pro Bold}]{Minion Pro}
\newfontfamily\scfont{GFS Artemisia}
\font\icon = "Webdings"
\usepackage{tikz,pgfplots}
\tkzSetUpPoint[size=7,fill=white]
\xroma{red!70!black}
%------TIKZ - ΣΧΗΜΑΤΑ - ΓΡΑΦΙΚΕΣ ΠΑΡΑΣΤΑΣΕΙΣ ----
\usepackage{tikz}
\usepackage{tkz-euclide}
\usetkzobj{all}
\usepackage[framemethod=TikZ]{mdframed}
\usetikzlibrary{decorations.pathreplacing}
\usepackage{pgfplots}
\usetikzlibrary{intersections,calc}
%-----------------------
\usepackage{calc}
\usepackage{hhline}
\usepackage[explicit]{titlesec}
\usepackage{graphicx}
\usepackage{multicol}
\usepackage{multirow}
\usepackage{enumitem}
\usepackage{tabularx}
\usetikzlibrary{backgrounds}
\usepackage{sectsty}
\sectionfont{\centering}
\setlist[enumerate]{label=\bf{\large \arabic*.}}
\usepackage{adjustbox}
\usepackage{mathimatika,gensymb,eurosym,wrap-rl}
\usepackage{systeme,regexpatch}
%-------- ΜΑΘΗΜΑΤΙΚΑ ΕΡΓΑΛΕΙΑ ---------
\usepackage{mathtools}
%----------------------
%-------- ΠΙΝΑΚΕΣ ---------
\usepackage{booktabs}
%----------------------
%----- ΥΠΟΛΟΓΙΣΤΗΣ ----------
\usepackage{calculator}
%----------------------------
%------ ΔΙΑΓΩΝΙΟ ΣΕ ΠΙΝΑΚΑ -------
\usepackage{array}
\newcommand\diag[5]{%
\multicolumn{1}{|m{#2}|}{\hskip-\tabcolsep
$\vcenter{\begin{tikzpicture}[baseline=0,anchor=south west,outer sep=0]
\path[use as bounding box] (0,0) rectangle (#2+2\tabcolsep,\baselineskip);
\node[minimum width={#2+2\tabcolsep-\pgflinewidth},
minimum  height=\baselineskip+#3-\pgflinewidth] (box) {};
\draw[line cap=round] (box.north west) -- (box.south east);
\node[anchor=south west,align=left,inner sep=#1] at (box.south west) {#4};
\node[anchor=north east,align=right,inner sep=#1] at (box.north east) {#5};
\end{tikzpicture}}\rule{0pt}{.71\baselineskip+#3-\pgflinewidth}$\hskip-\tabcolsep}}
%---------------------------------
%---- ΟΡΙΖΟΝΤΙΟ - ΚΑΤΑΚΟΡΥΦΟ - ΠΛΑΓΙΟ ΑΓΚΙΣΤΡΟ ------
\newcommand{\orag}[3]{\node at (#1)
{$ \overcbrace{\rule{#2mm}{0mm}}^{{\scriptsize #3}} $};}
\newcommand{\kag}[3]{\node at (#1)
{$ \undercbrace{\rule{#2mm}{0mm}}_{{\scriptsize #3}} $};}
\newcommand{\Pag}[4]{\node[rotate=#1] at (#2)
{$ \overcbrace{\rule{#3mm}{0mm}}^{{\rotatebox{-#1}{\scriptsize$#4$}}}$};}
%-----------------------------------------
%------------------------------------------
\newcommand{\tss}[1]{\textsuperscript{#1}}
\newcommand{\tssL}[1]{\MakeLowercase{\textsuperscript{#1}}}
%---------- ΛΙΣΤΕΣ ----------------------
\newlist{bhma}{enumerate}{3}
\setlist[bhma]{label=\bf\textit{\arabic*\textsuperscript{o}\;Βήμα :},leftmargin=0cm,itemindent=1.8cm,ref=\bf{\arabic*\textsuperscript{o}\;Βήμα}}
\newlist{rlist}{enumerate}{3}
\setlist[rlist]{itemsep=0mm,label=\roman*.}
\newlist{brlist}{enumerate}{3}
\setlist[brlist]{itemsep=0mm,label=\bf\roman*.}
\newlist{tropos}{enumerate}{3}
\setlist[tropos]{label=\bf\textit{\arabic*\textsuperscript{oς}\;Τρόπος :},leftmargin=0cm,itemindent=2.3cm,ref=\bf{\arabic*\textsuperscript{oς}\;Τρόπος}}
% Αν μπει το bhma μεσα σε tropo τότε
%\begin{bhma}[leftmargin=.7cm]
\tkzSetUpPoint[size=7,fill=white]
\tikzstyle{pl}=[line width=0.3mm]
\tikzstyle{plm}=[line width=0.4mm]
\usepackage{etoolbox}
\makeatletter
\renewrobustcmd{\anw@true}{\let\ifanw@\iffalse}
\renewrobustcmd{\anw@false}{\let\ifanw@\iffalse}\anw@false
\newrobustcmd{\noanw@true}{\let\ifnoanw@\iffalse}
\newrobustcmd{\noanw@false}{\let\ifnoanw@\iffalse}\noanw@false
\renewrobustcmd{\anw@print}{\ifanw@\ifnoanw@\else\numer@lsign\fi\fi}
\makeatother

\begin{document}
\begin{center}
\textbf{ΤΥΠΟΛΟΓΙΟ ΓΕΩΜΕΤΡΙΑΣ}
\end{center}
\begin{enumerate}
\item Θεωρήματα ορθογωνίων τριγώνων.\\
\wrapr{-12mm}{5}{3.5cm}{-5mm}{\begin{tikzpicture}
\tkzDefPoint(0,0){A}
\tkzDefPoint(2.5,0){B}
\tkzDefPoint(0,1.7){C}
\tkzMarkRightAngle[size=.2](B,A,C)
\tkzDefPointBy[projection=onto C--B](A)\tkzGetPoint{D}
\tkzMarkRightAngle[size=.2](A,D,B)
\draw[pl] (B)--(A)--(C);
\draw[pl] (A)--(D);
\draw[pl] (B)--(C);
\tkzDrawPoints(A,B,C,D)
\tkzLabelPoint[left](A){$A$}
\tkzLabelPoint[right](B){$B$}
\tkzLabelPoint[above left](C){$\varGamma$}
\tkzLabelPoint[right,yshift=1mm](D){$\varDelta$}
\end{tikzpicture}}{
\begin{align*}
\textrm{Θεώρημα προβολής }\ &AB^2=B\varGamma\cdot B\varDelta\ \ ,\ \ A\varGamma^2=B\varGamma\cdot \varGamma\varDelta\\
\textrm{Πυθαγόρειο θεώρημα }\ &\hat{A}=90\degree\Rightarrow AB^2+A\varGamma^2=B\varGamma^2\\
\textrm{Αντίστροφο Πυθαγορείου }\ &AB^2+A\varGamma^2=B\varGamma^2\Rightarrow\hat{A}=90\degree\\
\textrm{Θεώρημα ύψους }\ &A\varDelta^2=B\varDelta\cdot\varGamma\varDelta
\end{align*}}
\item Γενικευμένο Πυθαγόρειο για οξεία γωνία
\begin{align*}
\hat{A}<90\degree \Rightarrow a^2=\beta^2+\gamma^2-2\beta\cdot AE \quad\textrm{και }\ \ \  & a^2=\beta^2+\gamma^2-2\gamma\cdot AZ\\
\hat{B}<90\degree \Rightarrow\beta^2=a^2+\gamma^2-2a\cdot B\varDelta \quad\textrm{και }\ \ \  & \beta^2=a^2+\gamma^2-2\gamma\cdot BZ\\
\hat{\varGamma}<90\degree \Rightarrow\gamma^2=a^2+\beta^2-2a\cdot \varGamma\varDelta \quad\textrm{και }\ \ \  & \gamma^2=a^2+\beta^2-2\beta\cdot \varGamma E
\end{align*}
\begin{center}
\begin{tabular}{p{5cm}p{5cm}}
 \begin{tikzpicture}
\tkzDefPoint(0,0){B}
\tkzDefPoint(3,0){C}
\tkzDefPoint(1,2){A}
\tkzDefPoint(1,0){D}
\tkzDefPointBy[projection=onto A--B](C)\tkzGetPoint{c}
\tkzDefPointBy[projection=onto A--C](B)\tkzGetPoint{b}
\tkzMarkRightAngle[size=.2](C,D,A)
\tkzMarkRightAngle[size=.2](B,c,C)
\tkzMarkRightAngle[size=.2](B,b,C)
\draw[pl] (A)--(B)--(C)--cycle;
\draw(A)--(D);
\draw(C)--(c);
\draw(B)--(b);
\tkzDrawPoints(A,B,C,D,b,c)
\tkzLabelPoint[above](A){$A$}
\tkzLabelPoint[left](B){$B$}
\tkzLabelPoint[right](C){$\varGamma$}
\tkzLabelPoint[below](D){$\varDelta$}
\tkzLabelPoint[left](c){$Z$}
\tkzLabelPoint[right,yshift=1mm](b){$E$}
\end{tikzpicture} & \begin{tikzpicture}
\clip (-1.2,-1.5) rectangle (3.5,2.5);
\tkzDefPoint(0,0){B}
\tkzDefPoint(3,0){C}
\tkzDefPoint(-1,2){A}
\tkzDefPointBy[projection=onto A--B](C)\tkzGetPoint{c}
\tkzDefPointBy[projection=onto C--B](A)\tkzGetPoint{a}
\tkzDefPointBy[projection=onto A--C](B)\tkzGetPoint{b}
\tkzMarkRightAngle[size=.2](C,a,A)
\tkzMarkRightAngle[size=.2](B,c,C)
\tkzMarkRightAngle[size=.2](B,b,C)
\draw[pl] (A)--(B)--(C)--cycle;
\draw(A)--(a);
\draw(C)--(c);
\draw(B)--(b);
\draw[dashed](B)--(a);
\draw[dashed](B)--(c);
\tkzDrawPoints(A,B,C,a,b,c)
\tkzLabelPoint[above](A){$A$}
\tkzLabelPoint[below left](B){$B$}
\tkzLabelPoint[right](C){$\varGamma$}
\tkzLabelPoint[below](a){$\varDelta$}
\tkzLabelPoint[left](c){$Z$}
\tkzLabelPoint[above right](b){$E$}
\end{tikzpicture} \\ 
\end{tabular} 
\end{center}
\item Γενικευμένο Πυθαγόρειο για αμβλεία γωνία
\begin{align*}
\hat{A}<90\degree \Rightarrow a^2=\beta^2+\gamma^2-2\beta\cdot AE \quad\textrm{και }\ \ \  & a^2=\beta^2+\gamma^2-2\gamma\cdot AZ\\
\hat{B}>90\degree \Rightarrow\beta^2=a^2+\gamma^2+2a\cdot B\varDelta \quad\textrm{και }\ \ \  & \beta^2=a^2+\gamma^2+2\gamma\cdot BZ\\
\hat{\varGamma}<90\degree \Rightarrow\gamma^2=a^2+\beta^2-2a\cdot \varGamma\varDelta \quad\textrm{και }\ \ \  & \gamma^2=a^2+\beta^2-2\beta\cdot \varGamma E
\end{align*}
\item Είδος τριγώνου ως προς τις γωνίες:
\begin{gather*}
a^2>\beta^2+\gamma^2\Rightarrow \hat{A}>90\degree\\
a^2=\beta^2+\gamma^2\Rightarrow \hat{A}=90\degree\\
a^2<\beta^2+\gamma^2\Rightarrow \hat{A}<90\degree
\end{gather*}
\begin{multicols}{2}
\item Νόμος συνημίτονων:\\
 \begin{tikzpicture}
\node at (5.8,1.5) {$a^2=\beta^2+\gamma^2-2\beta\gamma\syn{\hat{A}}$};
\node at (5.8,1) {$\beta^2=a^2+\gamma^2-2a\gamma\syn{\hat{B}}$};
\node at (5.8,.5) {$\gamma^2=a^2+\beta^2-2a\beta\syn{\hat{\varGamma}}$};
\end{tikzpicture}
\item Νόμος ημίτονων:\\
$ \dfrac{a}{\hm{\hat{A}}}=\dfrac{\beta}{\hm{\hat{B}}}=\dfrac{\gamma}{\hm{\hat{\varGamma}}}=2R $.
\end{multicols}
\item 1ο Θεώρημα διαμέσων:\\
\wrapr{-12mm}{7}{4cm}{0mm}{\begin{tikzpicture}
\tkzDefPoint(0,0){B}
\tkzDefPoint(3,0){C}
\tkzDefPoint(1,2){A}
\tkzDefPoint(.5,1){c}
\tkzDefPoint(1.5,0){a}
\tkzDefPoint(2,1){b}
\draw[pl] (A)--(B)--(C)--cycle;
\draw(A)--(a);
\draw(C)--(c);
\draw(B)--(b);
\tkzDrawPoints(A,B,C,a,b,c)
\tkzLabelPoint[above](A){$A$}
\tkzLabelPoint[left](B){$B$}
\tkzLabelPoint[right](C){$\varGamma$}
\tkzLabelPoint[below](a){$\varDelta$}
\tkzLabelPoint[left](c){$Z$}
\tkzLabelPoint[right](b){$E$}
\end{tikzpicture}}{
\begin{gather*}
a^2+\beta^2=2\mu_\gamma^2+\frac{\gamma^2}{2}\ \ ,\ \ a^2+\gamma^2=2\mu_\beta^2+\frac{\beta^2}{2}\ \ ,\ \ \beta^2+\gamma^2=2\mu_a^2+\frac{a^2}{2}\\
\mu_a^2=\frac{2\beta^2+2\gamma^2-a^2}{4}\ \ ,\ \ \mu_\beta^2=\frac{2a^2+2\gamma^2-\beta^2}{4}\ \ ,\ \ \mu_\gamma^2=\frac{2a^2+2\beta^2-\gamma^2}{4}
\end{gather*}}\mbox{}\\\\\\
\item 2ο Θεώρημα Διαμέσων\\
\wrapr{-5mm}{7}{4cm}{-7mm}{\begin{tikzpicture}
\tkzDefPoint(0,0){B}
\tkzDefPoint(3,0){C}
\tkzDefPoint(1,2){A}
\tkzDefPoint(.5,1){c}
\tkzDefPoint(1.5,0){a}
\tkzDefPoint(2,1){b}
\tkzDefPoint(1,0){D}
\tkzDefPointBy[projection=onto A--B](C)\tkzGetPoint{e}
\tkzDefPointBy[projection=onto A--C](B)\tkzGetPoint{d}
\tkzMarkRightAngle[size=.2](C,D,A)
\tkzMarkRightAngle[size=.2](B,d,C)
\tkzMarkRightAngle[size=.2](C,e,A)
\draw[pl] (A)--(B)--(C)--cycle;
\draw(D)--(A)--(a);
\draw(e)--(C)--(c);
\draw(d)--(B)--(b);
\draw[pl,\xrwma](D)--(a);
\draw[pl,\xrwma](b)--(d);
\draw[pl,\xrwma](e)--(c);
\tkzDrawPoints(A,B,C,a,b,c,D,d,e)
\tkzLabelPoint[above](A){$A$}
\tkzLabelPoint[left](B){$B$}
\tkzLabelPoint[right](C){$\varGamma$}
\tkzLabelPoint[below](a){$\varDelta$}
\tkzLabelPoint[left](c){$Z$}
\tkzLabelPoint[above right](b){$E$}
\tkzLabelPoint[below](D){$M$}
\tkzLabelPoint[above right](d){$K$}
\tkzLabelPoint[above left,xshift=1mm](e){$\varLambda$}
\end{tikzpicture}}{
\begin{align*}
&a^2-\beta^2=2\gamma\cdot\varLambda Z\\ &a^2-\gamma^2=2\beta\cdot KE\\ &\beta^2-\gamma^2=2a\cdot M\varDelta
\end{align*}}\mbox{}\\\\
\item Λόγος εμβαδών με ίσες βάσεις - ίσα ύψη:\\
\[ \textrm{Ίσες βάσεις }\Rightarrow \frac{E}{E'}=\frac{\upsilon}{\upsilon'}\qquad\textrm{Ίσα ύψη }\Rightarrow \frac{E}{E'}=\frac{\beta}{\beta'} \]
\item Λόγος εμβαδών όμοιων τριγώνων - πολυγώνων:\\
\[ AB\varGamma\approx A'B'\varGamma'\Rightarrow\frac{(AB\varGamma)}{(A'B'\varGamma')}=\lambda^2\ \ (\lambda=\textrm{λόγος ομοιότητας}) \]
\item Λόγος εμβαδών τριγώνων με ίσες - παραπληρωματικές γωνίες:\\
\[ \textrm{Αν }\hat{A}=\hat{A'}\ \textrm{ ή }\ \hat{A}+\hat{A'}=180\degree\Rightarrow \frac{(AB\varGamma)}{(A'B'\varGamma')}=\frac{\beta\gamma}{\beta'\gamma'} \]
\end{enumerate}
\begin{center}
\textbf{ΕΜΒΑΔΑ ΒΑΣΙΚΩΝ ΣΧΗΜΑΤΩΝ}
\end{center}
\vspace{-4mm}
\hrulefill
\begin{enumerate}[resume]
\begin{multicols}{2}
\item Εμβαδόν τετραγώνου: $ E=a^2 $.
\item Εμβαδόν ορθογωνίου: $ E=a\cdot\beta $.
\item Εμβαδόν παραλληλογράμμου: \\$ E=a\cdot\upsilon_a=\beta\cdot\upsilon_\beta $.
\item Εμβαδόν τραπεζίου: $ E=\dfrac{(B+\beta)\cdot\upsilon}{2}=\delta\cdot\upsilon $.
\item Εμβαδόν ισόπλευρου τριγώνου: $ E=\dfrac{a^2\sqrt{3}}{4} $.
\item Εμβαδόν ρόμβου: $ E=\dfrac{1}{2}\delta_1\cdot\delta_2 $.
\end{multicols}
\item Εμβαδόν τριγώνου: 
\begin{multicols}{2}
\begin{rlist}
\item $ E=\dfrac{1}{2}a\cdot\upsilon_a=\dfrac{1}{2}\beta\cdot\upsilon_\beta=\dfrac{1}{2}\gamma\cdot\upsilon_\gamma $.
\item $ E=\sqrt{\tau(\tau-a)(\tau-\beta)(\tau-\gamma)} $.
\item $ E=\tau\cdot\rho $.
\item $ E=\dfrac{a\beta\gamma}{4R} $.
\item $ E=\dfrac{1}{2}a\cdot\beta\cdot\hm{\varGamma}=\dfrac{1}{2}\beta\cdot\gamma\cdot\hm{A}=\dfrac{1}{2}a\cdot\gamma\cdot\hm{B} $.
\end{rlist}
\end{multicols}
$ \left( \tau=\dfrac{a+\beta+\gamma}{2}\right)  $\end{enumerate}
\hrulefill
\begin{enumerate}[resume]
\item Στοιχεία κανονικού πολυγώνου:
\begin{multicols}{2}
\begin{rlist}
\item Κεντρική γωνία: $ \omega_\nu=\dfrac{360\degree}{\nu} $
\item Γωνία: $ \varphi_\nu=180\degree-\omega_\nu $
\item $ a_\nu^2+\dfrac{\lambda_\nu^2}{4}=R^2 $
\item $ a_\nu=R\cdot\syn{\left( \dfrac{\omega_\nu}{2}\right) } $
\item $ \lambda_\nu=2R\cdot\hm{\left( \dfrac{\omega_\nu}{2}\right) } $
\item $ \lambda_\nu=2a_\nu\cdot\ef{\left( \dfrac{\omega_\nu}{2}\right)} $
\item Περίμετρος: $ P_\nu=\nu\cdot\lambda_\nu $
\item Εμβαδόν: $ E_\nu=\dfrac{1}{2}P_\nu\cdot a_\nu $
\end{rlist}
\end{multicols}
\item Λόγος στοιχείων κανονικού πολυγώνου
\[ \nu-\textrm{γωνο}\approx\nu-\textrm{γωνο}\Rightarrow\frac{\lambda_\nu}{\lambda_\nu'}=\frac{R}{R'}=\frac{a_\nu}{a_\nu'} \]
\item Εγγραφή κανονικών πολυγώνων σε κύκλο:
\begin{center}\begin{tabular}{c|c|c|c}
\hline \rule[-2ex]{0pt}{5ex}
 & \textbf{Ισόπλευρο τρίγωνο} & \textbf{Τετράγωνο} & \textbf{Κανονικό εξάγωνο} \\ 
 & {\boldmath$ \nu=3 $} & {\boldmath$ \nu=4 $} & {\boldmath$ \nu=6 $} \\
\hhline{====} \rule[-2ex]{0pt}{5ex}
Πλευρά $ \lambda_\nu $ & $ R\sqrt{3} $ & $ R\sqrt{2} $ & $ R $ \\ 
\rule[-2ex]{0pt}{5ex}
Απόστημα $ a_\nu $ & $ \dfrac{R}{2} $ & $ \dfrac{R\sqrt{2}}{2} $ & $ \dfrac{R\sqrt{3}}{2} $ \\ 
\hline 
\end{tabular} 
\end{center} 
\wrapr{-5mm}{5}{7.5cm}{0mm}{\begin{tikzpicture}
\draw (0,0) circle (1.2);
\coordinate (O)  at (0,0);
\coordinate (A)  at (240:1.2);
\coordinate (B)  at (300:1.2);
\tkzLabelPoint[below left](A){$A$}
\tkzLabelPoint[below right](B){$B$}
\tkzLabelPoint[above](O){$O$}
\draw[fill=black!30,pl,draw=black] (A)--(O)--(B) arc[start angle=300, end angle=240, radius=1.2] (A);
\tkzMarkAngle[size=.35](A,O,B)
\tkzDrawPoints(A,B,O)
\node at (0,-0.5) {\footnotesize$ \mu $};
\end{tikzpicture}\qquad\begin{tikzpicture}
\draw (0,0) circle (1.2);
\coordinate (O)  at (0,0);
\coordinate (A)  at (240:1.2);
\coordinate (B)  at (300:1.2);
\tkzLabelPoint[below left](A){$A$}
\tkzLabelPoint[below right](B){$B$}
\tkzLabelPoint[above](O){$O$}
\draw (A)--(O)--(B);
\draw[fill=black!30,pl,draw=black] (A)--(B) arc[start angle=300, end angle=240, radius=1.2cm] (A);
\tkzMarkAngle[size=.35](A,O,B)
\tkzDrawPoints(A,B,O)
\node at (0,-0.5) {\footnotesize$ \mu $};
\end{tikzpicture}}{
\item Μήκος Κύκλου: $ L=2\pi R $.
\item Μήκος τόξου: $ \mathcal{l}=\dfrac{\pi R \mu}{180\degree
}=a R $.
\item Εμβαδόν κυκλικού δίσκου: $ E=\pi R^2 $.
\item Εμβαδόν κυκλικού τομέα: $ E=\dfrac{\pi R^2 \mu}{360\degree}=\dfrac{1}{2}aR^2 $.
\item Εμβαδόν κυκλικού τμήματος: \[ \varepsilon=(\textrm{τομέα})-(\textrm{τριγώνου})=\undercbrace{\dfrac{\pi R^2\mu}{360\degree}-\dfrac{R^2\hm{\mu}}{2}}_{\textrm{με μοίρες}}=\undercbrace{\dfrac{R^2}{2}(a-\hm{a})}_{\textrm{με ακτίνια}} \]}
\end{enumerate}

\end{document}
