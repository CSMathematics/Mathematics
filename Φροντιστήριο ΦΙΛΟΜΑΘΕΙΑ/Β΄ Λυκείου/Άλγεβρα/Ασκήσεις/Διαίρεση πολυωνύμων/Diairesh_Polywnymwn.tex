\documentclass[11pt,a4paper,twocolumn]{article}
\usepackage[english,greek]{babel}
\usepackage[utf8]{inputenc}
\usepackage{nimbusserif}
\usepackage[T1]{fontenc}
\usepackage[left=1.50cm, right=1.50cm, top=2.00cm, bottom=2.00cm]{geometry}
\usepackage{amsmath}
\let\myBbbk\Bbbk
\let\Bbbk\relax
\usepackage[amsbb,subscriptcorrection,zswash,mtpcal,mtphrb,mtpfrak]{mtpro2}
\usepackage{graphicx,multicol,multirow,enumitem,tabularx,mathimatika,gensymb,venndiagram,hhline,longtable,tkz-euclide,fontawesome5,eurosym,tcolorbox,tabularray}
\usepackage[explicit]{titlesec}
\tcbuselibrary{skins,theorems,breakable}
\newlist{rlist}{enumerate}{3}
\setlist[rlist]{itemsep=0mm,label=\roman*.}
\newlist{alist}{enumerate}{3}
\setlist[alist]{itemsep=0mm,label=\alph*.}
\newlist{balist}{enumerate}{3}
\setlist[balist]{itemsep=0mm,label=\bf\alph*.}
\newlist{Alist}{enumerate}{3}
\setlist[Alist]{itemsep=0mm,label=\Alph*.}
\newlist{bAlist}{enumerate}{3}
\setlist[bAlist]{itemsep=0mm,label=\bf\Alph*.}
\newlist{askhseis}{enumerate}{3}
\setlist[askhseis]{label={\Large\thesection}.\arabic*.}
\renewcommand{\textstigma}{\textsigma\texttau}
\newlist{thema}{enumerate}{3}
\setlist[thema]{label=\bf\large{ΘΕΜΑ \textcolor{black}{\Alph*}},itemsep=0mm,leftmargin=0cm,itemindent=18mm}
\newlist{erwthma}{enumerate}{3}
\setlist[erwthma]{label=\bf{\large{\textcolor{black}{\Alph{themai}.\arabic*}}},itemsep=0mm,leftmargin=0.8cm}

\newcommand{\kerkissans}[1]{{\fontfamily{maksf}\selectfont \textbf{#1}}}
\renewcommand{\textdexiakeraia}{}

\usepackage[
backend=biber,
style=alphabetic,
sorting=ynt
]{biblatex}

\DeclareTblrTemplate{caption}{nocaptemplate}{}
\DeclareTblrTemplate{capcont}{nocaptemplate}{}
\DeclareTblrTemplate{contfoot}{nocaptemplate}{}
\NewTblrTheme{mytabletheme}{
\SetTblrTemplate{caption}{nocaptemplate}{}
\SetTblrTemplate{capcont}{nocaptemplate}{}
\SetTblrTemplate{contfoot}{nocaptemplate}{}
}

\NewTblrEnviron{mytblr}
\SetTblrStyle{firsthead}{font=\bfseries}
\SetTblrStyle{firstfoot}{fg=red2}
\SetTblrOuter[mytblr]{theme=mytabletheme}
\SetTblrInner[mytblr]{
rowspec={t{7mm}},columns = {c},
width = 0.85\linewidth,
row{odd} = {bg=red9,fg=black,ht=8mm},
row{even} = {bg=red7,fg=black,ht=8mm},
hlines={white},vlines={white},
row{1} = {bg=red4, fg=white, font=\bfseries\fontfamily{maksf}},rowhead = 1,
hline{2} = {.7mm}, % midrule  
}
\newcounter{askhsh}
\setcounter{askhsh}{1}
\newcommand{\askhsh}{{\large\theaskhsh.}\ \addtocounter{askhsh}{1}}

\titleformat{\section}{\Large}{\kerkissans{\thesection}}{10pt}{\Large\kerkissans{#1}}

\setlength{\columnsep}{5mm}
\titleformat{\paragraph}
{\large}%
{}{0em}%
{\textcolor{red!80!black}{\faSquare\ \ \kerkissans{\bmath{#1}}}}
\setlength{\parindent}{0pt}

\newcommand{\eng}[1]{\selectlanguage{english}#1\selectlanguage{greek}}

\begin{document}
\twocolumn[{
\centering
\kerkissans{{\huge Διαίρεση πολυωνύμων}\\\vspace{3mm} {\Large ΑΣΚΗΣΕΙΣ}}\vspace{5mm}}]
\paragraph{Διαίρεση πολυωνύμων}
\askhsh Για καθεμία από τις παρακάτω διαιρέσεις, να βρεθεί, με κάθετη διαίρεση, το πηλίκο και το υπόλοιπο. Στη συνέχεια να γράψετε την ταυτότητα της διαίρεσης.
\begin{alist}
\item $\left(x^3-4x^2+5x-12\right):\left(x^2+2x+3\right)$
\item $\left(x^3+7x^2-8x+4\right):\left(x^2-x\right)$
\item $\left(2x^3+x^2+3x-9\right):\left(x^2-3\right)$
\item $\left(x^3+5x^2-11x+10\right):\left(x+2\right)$
\item $\left(x^4-2x^3+5x^2+4x-8\right):\left(x^2+4x-2\right)$
\item $\left(x^4-2x^2+3x+2\right):\left(x^2-3x-5\right)$
\item $\left(x^4+x-+8\right):\left(x-4\right)$
\item $\left(2x^4+6x^3-5x^2+x+7\right):\left(2x^2+7\right)$
\end{alist}
\askhsh Για καθεμία από τις παρακάτω διαιρέσεις, να βρεθεί, με τη χρήση του σχήματος \eng{Horner}, το πηλίκο και το υπόλοιπο. Στη συνέχεια να γράψετε την ταυτότητα της διαίρεσης.
\begin{alist}
\item $\left(x^3+3x^2+4x-2\right):\left(x-1\right)$
\item $\left(x^3+9x^2+2x-8\right):\left(x-2\right)$
\item $\left(x^3+4x^2-7x+10\right):\left(x+3\right)$
\item $\left(2x^3-5x^2+8x+4\right):\left(x+2\right)$
\item $\left(x^3+x-2\right):\left(x+1\right)$
\item $\left(x^4+6x^2+4\right):\left(x-1\right)$
\item $\left(x^4+2x^3-5x^2-7x+8\right):\left(x-4\right)$
\item $\left(2x^4-9x^3-4x+5\right):\left(x+2\right)$
\end{alist}
\paragraph{Βασικά θεωρήματα}
\askhsh Δίνεται το πολυώνυμο
\[ P(x)=x^3-4x^2+5x-2 \]
Να εξετάσετε ποιο από τα παρακάτω πολυώνυμα είναι παράγοντας του $P(x)$.
\begin{multicols}{4}
\begin{alist}
\item $x-1$
\item $x+2$
\item $x-3$
\item $x-2$
\end{alist}
\end{multicols}
\askhsh Δίνεται το πολυώνυμο \[ P(x)=(\lambda-1)x^3-5x^2+(\lambda^2-2)x+8 \] όπου $ \lambda\in\mathbb{R} $.
\begin{alist}
\item Να βρεθεί η τιμή της παραμέτρου $ \lambda $ για την οποία το πολυώνυμο έχει παράγοντα το $ x-2 $.
\item Να βρεθούν οι τιμές της μεταβλητής $ x $ για τις οποίες η γραφική παράσταση της συνάρτησης $ P(x) $ βρίσκεται πάνω από τον άξονα $ x'x $.
\item Να βρεθούν οι τιμές της παραμέτρου $ \lambda $ για τις οποίες το πολυώνυμο $ P(x) $ αν διαιρεθεί με το $ x-1 $ δίνει υπόλοιπο $ 2 $.
\end{alist}
\paragraph{Τράπεζα Θεμάτων - ΙΕΠ}
\askhsh \textbf{Θέμα 2 - 15643}\\
Δίνεται το πολυώνυμο
\[ P(x)=2x^3-3x^3-11x+6 \]
\begin{alist}
\item \begin{rlist}
\item Να δείξετε ότι το πολυώνυμο $P(x)$ έχει παράγοντα το $x-3$.
\item Να γράψετε την ταυτότητα της Ευκλείδειας διαίρεσης $P(x):(x-3)$.
\end{rlist}
\item Να δείξετε ότι το πολυώνυμο $P(x)$ έχει παράγοντα το $(x-3)(2x-1)$.
\end{alist}
\end{document}
