\documentclass[11pt,a4paper,twocolumn]{article}
\usepackage[english,greek]{babel}
\usepackage[utf8]{inputenc}
\usepackage{nimbusserif}
\usepackage[T1]{fontenc}
\usepackage[left=1.50cm, right=1.50cm, top=2.00cm, bottom=2.00cm]{geometry}
\usepackage{amsmath}
\let\myBbbk\Bbbk
\let\Bbbk\relax
\usepackage[amsbb,subscriptcorrection,zswash,mtpcal,mtphrb,mtpfrak]{mtpro2}
\usepackage{xcolor,graphicx,multicol,multirow,enumitem,tabularx,mathimatika,gensymb,venndiagram,hhline,longtable,tkz-euclide,fontawesome5,eurosym,tcolorbox,tabularray}
\usepackage[explicit]{titlesec}
\tcbuselibrary{skins,theorems,breakable}
\newlist{rlist}{enumerate}{3}
\setlist[rlist]{itemsep=0mm,label=\roman*.}
\newlist{alist}{enumerate}{3}
\setlist[alist]{itemsep=0mm,label=\alph*.}
\newlist{balist}{enumerate}{3}
\setlist[balist]{itemsep=0mm,label=\bf\alph*.}
\newlist{Alist}{enumerate}{3}
\setlist[Alist]{itemsep=0mm,label=\Alph*.}
\newlist{bAlist}{enumerate}{3}
\setlist[bAlist]{itemsep=0mm,label=\bf\Alph*.}
\newlist{askhseis}{enumerate}{3}
\setlist[askhseis]{label={\Large\thesection}.\arabic*.}
\renewcommand{\textstigma}{\textsigma\texttau}
\newlist{thema}{enumerate}{3}
\setlist[thema]{label=\bf\large{ΘΕΜΑ \textcolor{black}{\Alph*}},itemsep=0mm,leftmargin=0cm,itemindent=18mm}
\newlist{erwthma}{enumerate}{3}
\setlist[erwthma]{label=\bf{\large{\textcolor{black}{\Alph{themai}.\arabic*}}},itemsep=0mm,leftmargin=0.8cm}

\newcommand{\kerkissans}[1]{{\fontfamily{maksf}\selectfont \textbf{#1}}}
\renewcommand{\textdexiakeraia}{}

\usepackage[
backend=biber,
style=alphabetic,
sorting=ynt
]{biblatex}

\DeclareTblrTemplate{caption}{nocaptemplate}{}
\DeclareTblrTemplate{capcont}{nocaptemplate}{}
\DeclareTblrTemplate{contfoot}{nocaptemplate}{}
\NewTblrTheme{mytabletheme}{
\SetTblrTemplate{caption}{nocaptemplate}{}
\SetTblrTemplate{capcont}{nocaptemplate}{}
\SetTblrTemplate{contfoot}{nocaptemplate}{}
}

\NewTblrEnviron{mytblr}
\SetTblrStyle{firsthead}{font=\bfseries}
\SetTblrStyle{firstfoot}{fg=red2}
\SetTblrOuter[mytblr]{theme=mytabletheme}
\SetTblrInner[mytblr]{
rowspec={t{7mm}},columns = {c},
width = 0.85\linewidth,
row{odd} = {bg=red9,fg=black,ht=8mm},
row{even} = {bg=red7,fg=black,ht=8mm},
hlines={white},vlines={white},
row{1} = {bg=red4, fg=white, font=\bfseries\fontfamily{maksf}},rowhead = 1,
hline{2} = {.7mm}, % midrule  
}
\newcounter{askhsh}
\setcounter{askhsh}{1}
\newcommand{\askhsh}{\large\theaskhsh.\ \addtocounter{askhsh}{1}}

\titleformat{\section}{\Large}{\kerkissans{\thesection}}{10pt}{\Large\kerkissans{#1}}

\setlength{\columnsep}{5mm}
\titleformat{\paragraph}
{\large}%
{}{0em}%
{\textcolor{red!80!black}{\faSquare\ \ \kerkissans{\bmath{#1}}}}
\setlength{\parindent}{0pt}

\newcommand{\eng}[1]{\selectlanguage{english}#1\selectlanguage{greek}}

\begin{document}
\twocolumn[{
\centering
\kerkissans{{\huge Πολυωνυμικές εξισώσεις}\\\vspace{3mm} {\Large ΑΣΚΗΣΕΙΣ}}\vspace{5mm}}]
\paragraph{Επίλυση εξισώσεων}
\askhsh Να λυθούν οι ακόλουθες εξισώσεις.
\begin{alist}
\item $x^3+2x^2=0$
\item $x^3-4x^2+x-4=0$
\item $x^3-3x^2-4x+12=0$
\item $x^3+3x=2x^2+6$
\item $x^3-8=0$
\end{alist}
\askhsh Να λυθούν οι ακόλουθες εξισώσεις.
\begin{alist}
\item $x^4-9x^2=0$
\item $x^4+3x^3-10x^2=0$
\item $x^4+2x^3-5x^2=10x$
\item $x^5+4x^4-12x^3=0$
\end{alist}
\askhsh Να λυθούν οι ακόλουθες εξισώσεις.
\begin{alist}
\item $x^3-2x^2-5x+6=0$
\item $x^3-4x^2+x+6=0$
\item $x^3-x^2-10x-8=0$
\item $x^3+6x^2+5x-12=0$
\item $x^3-7x+6=0$
\item $x^3+x^2-5x+3=0$
\end{alist}
\paragraph{Επίλυση ανισώσεων}
\askhsh Να λυθούν οι παρακάτω ανισώσεις :
\begin{alist}
\item $(x-1)(x^2-9)>0$
\item $(x+3)(x^2-4x-12)\leq 0$
\item $ (x+2)(5-x)(x^2-6x-7)>0 $
\item $ (x-2)(x+1)(x^2-7x+12)\geq0 $
\item $ (x^2+x-2)(x^2-4x+4)<0 $
\item $ (x^3+x)(4-x^2)\leq 0 $
\end{alist}
\askhsh Να λυθούν οι παρακάτω ανισώσεις :
\begin{alist}
\item $ x^3-3x^2+4x-12>0 $
\item $ x^3+2x^2-8x<0 $
\item $ x^3-7x+6\geq 0 $
\item $ 2x^3-x^2-5x-2\leq 0 $
\end{alist}
\askhsh Να λυθούν οι παρακάτω ανισώσεις :
\begin{alist}
\item $ x^3-3x^2-6x+8>0 $
\item $ x^3+2x^2-3x-10<0 $
\item $ x^3+4x^2+5x+2\leq 0 $
\item $ 2x^3-5x^2+4x+21\geq 0 $
\item $ x^3-5x^2+3x+9<0 $
\item $ x^3-3x^2-2x-8\leq 0 $
\end{alist}
\askhsh Να λυθούν οι παρακάτω ανισώσεις :
\begin{alist}%4ου βαθμού
\item $ x^4 $
\end{alist}
\paragraph{Παραμετρικές}
\askhsh Δίνεται το πολυώνυμο \[ P(x)=x^4+ax^3+\beta x^2+\gamma x-2 \] το οποίο έχει ρίζα τον αριθμό $ -1 $ με πολλαπλότητα 3 (τριπλή λύση).
\begin{alist}
\item Να βρεθούν οι τιμές των παραμέτρων $ a,\beta,\gamma $.
\item Να λυθεί η εξίσωση $ P(x)=0 $.
\end{alist}
\askhsh Το πολυώνυμο $ P(x)=x^3-4x^2+ax-2 $ έχει παράγοντα το $ x-2 $, ενώ αν διαιρεθεί με το $ x-3 $ αφήνει υπόλοιπο $ 4 $.
\begin{alist}
\item Να βρεθεί η τιμή της παραμέτρου $ a $.
\item Να βρεθεί το υπόλοιπο της διαίρεσης του $ P(x) $ με το $ x^2-5x+6 $.
\item Να λυθεί η εξίσωση $ P(x)=18 $.
\item Να λυθεί η ανίσωση $ P(x)^2-P(x)\geq 0 $.
\end{alist}
\askhsh Δίνεται το πολυώνυμο \[ P(x)=x^4-2x^3+ax^2+\beta x+4 \] Να βρεθούν οι τιμές των παραμέτρων $ a $ και $ \beta $ και το πολυώνυμο $ Q(x) $ ώστε $ Q^2(x)=P(x) $.\\\\
\askhsh Αν τα πολυώνυμα \[ P(x)=x^2+(a-1)x-\beta-5
\;\;\textrm{ και }\]
\[Q(x) = x^3+\beta x^2+(a-6)x-4 \] έχουν κοινή ρίζα το $ x=2 $, να βρεθούν οι τιμές των παραμέτρων $ a,\beta $.\\\\
\askhsh Δίνεται το πολυώνυμο \[ P(x)=ax^4-4x^3+\beta x+2 \] το οποίο αν διαιρεθεί με το $ x-1 $ αφήνει υπόλοιπο $ -6 $ ενώ το $ x+1 $ είναι παράγοντας του.
\begin{alist}
\item Να βρεθούν οι τιμές των παραμέτρων $ a,\beta $.
\item Να αποδειχτεί ότι το $ 2x-1 $ είναι παράγοντας του $ P(x) $.
\item Να λυθεί η εξίσωση $ P(x)=0 $.
\item Να λυθεί η ανίσωση $ P(x)\leq0 $.
\end{alist}
\askhsh Δίνεται το πολυώνυμο \[ P(x)=x^3+ax^2+\beta x+1 \] το οποίο έχει παράγοντα το $ \left(x-1\right)^2 $.
\begin{alist}
\item Να βρεθούν οι τιμές των παραμέτρων $ a,\beta $.
\item Να βρεθούν όλες οι ρίζες του $ P(x) $.
\item Να γραφτεί το $P(x)$ ως γινόμενο παραγόντων.
\end{alist}
\askhsh Δίνεται το πολυώνυμο \[ P(x)=ax^3-2x^2-5x+\beta \] 
Αν το υπόλοιπο της διαίρεσης του $ P(x) $ με το $ x-2 $ είναι $ -4 $ και το $ x-1 $ είναι παράγοντας του $ P(x) $ τότε
\begin{alist}
\item Να αποδείξετε ότι $ a=1 $ και $ \beta=6 $.
\item Να λυθεί η εξίσωση $ P(x)=0 $.
\item Να λυθεί η ανίσωση $ P(x)\leq0 $.
\end{alist}
\end{document}
