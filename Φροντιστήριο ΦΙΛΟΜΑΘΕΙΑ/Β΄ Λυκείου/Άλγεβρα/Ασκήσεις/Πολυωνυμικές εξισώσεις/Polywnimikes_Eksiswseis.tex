\PassOptionsToPackage{no-math,cm-default}{fontspec}
\documentclass[twoside,nofonts,internet]{askhseis}
\usepackage{amsmath}
\usepackage{xgreek}
\let\hbar\relax
\defaultfontfeatures{Mapping=tex-text,Scale=MatchLowercase}
\setmainfont[Mapping=tex-text,Numbers=Lining,Scale=1.0,BoldFont={Minion Pro Bold}]{Minion Pro}
\newfontfamily\scfont{GFS Artemisia}
\xroma{red!80!black}
\usepackage{mtpro2}
%------TIKZ - ΣΧΗΜΑΤΑ - ΓΡΑΦΙΚΕΣ ΠΑΡΑΣΤΑΣΕΙΣ ----
\usepackage{tikz,pgfplots}
\usepackage{tkz-euclide}
\usetkzobj{all}
\usepackage[framemethod=TikZ]{mdframed}
\usetikzlibrary{decorations.pathreplacing}
\usepackage{pgfplots}
\usetkzobj{all}
%-----------------------
\usepackage{calc}
\usepackage{hhline}
\renewcommand{\thepart}{\arabic{part}}
\usepackage[explicit]{titlesec}
\usepackage{graphicx}
\usepackage{multicol}
\usepackage{multirow}
\usepackage{enumitem}
\usepackage{tabularx}
\usepackage[decimalsymbol=comma]{siunitx}
\usetikzlibrary{backgrounds}
\usepackage{sectsty}
\sectionfont{\centering}
\setlist[enumerate]{label=\bf{\large \arabic*.}}
\usepackage{adjustbox}
\usepackage{mathimatika,wrap-rl}
%-------- ΠΙΝΑΚΕΣ ---------
\usepackage{booktabs}
%----------------------
%----- ΥΠΟΛΟΓΙΣΤΗΣ ----------
\usepackage{calculator}
%----------------------------
%---- ΟΡΙΖΟΝΤΙΟ - ΚΑΤΑΚΟΡΥΦΟ - ΠΛΑΓΙΟ ΑΓΚΙΣΤΡΟ ------
\newcommand{\orag}[3]{\node at (#1)
{$ \overcbrace{\rule{#2mm}{0mm}}^{{\scriptsize #3}} $};}
\newcommand{\kag}[3]{\node at (#1)
{$ \undercbrace{\rule{#2mm}{0mm}}_{{\scriptsize #3}} $};}
\newcommand{\Pag}[4]{\node[rotate=#1] at (#2)
{$ \overcbrace{\rule{#3mm}{0mm}}^{{\rotatebox{-#1}{\scriptsize$#4$}}}$};}
%-----------------------------------------
\newcommand{\tss}[1]{\textsuperscript{#1}}
\newcommand{\tssL}[1]{\MakeLowercase{\textsuperscript{#1}}}
%---------- ΛΙΣΤΕΣ ----------------------
\newlist{bhma}{enumerate}{3}
\setlist[bhma]{label=\bf\textit{\arabic*\textsuperscript{o}\;Βήμα :},leftmargin=0cm,itemindent=1.8cm,ref=\bf{\arabic*\textsuperscript{o}\;Βήμα}}
\newlist{tropos}{enumerate}{3}
\setlist[tropos]{label=\bf\textit{\arabic*\textsuperscript{oς}\;Τρόπος :},leftmargin=0cm,itemindent=2.3cm,ref=\bf{\arabic*\textsuperscript{oς}\;Τρόπος}}
% Αν μπει το bhma μεσα σε tropo τότε
%\begin{bhma}[leftmargin=.7cm]
\tkzSetUpPoint[size=7,fill=white]
\tikzstyle{pl}=[line width=0.3mm]
\tikzstyle{plm}=[line width=0.4mm]








\begin{document}
\twocolkentro{
\titlos{Άλγεβρα Β΄ Λυκείου}{Πολυώνυμα}{Πολυωνυμικές εξισώσεις}
\thewria}

\twocolkentro{\askhseis}
\begin{enumerate}
\item 
\item Δίνεται το πολυώνυμο \[ P(x)=x^4-2x^3+ax^2+\beta x+4 \] Να βρεθούν οι τιμές των παραμέτρων $ a $ και $ \beta $ και το πολυώνυμο $ Q(x) $ ώστε $ Q^2(x)=P(x) $
\item Αν τα πολυώνυμα \[ P(x)=x^2+(a-1)x-\beta-5
\;\;\textrm{ και }\]
\[Q(x) = x^3+\beta x^2+(a-6)x-4 \] έχουν κοινή ρίζα το $ x=2 $, να βρεθούν οι τιμές των παραμέτρων $ a,\beta $.
\item Δίνεται το πολυώνυμο \[ P(x)=ax^4-4x^3+\beta x+2 \] το οποίο αν διαιρεθεί με το $ x-1 $ αφήνει υπόλοιπο $ -6 $ ενώ το $ x+1 $ είναι παράγοντας του.
\begin{rlist}
\item Να βρεθούν οι τιμές των παραμέτρων $ a,\beta $.
\item Να αποδειχτεί ότι το $ 2x-1 $ είναι παράγοντας του $ P(x) $.
\item Να λυθεί η εξίσωση $ P(x)=0 $.
\item Να λυθεί η ανίσωση $ P(x)\leq0 $.
\end{rlist}
\item Δίνεται το πολυώνυμο \[ P(x)=x^3+ax^2+\beta x+1 \] το οποίο έχει παράγοντα το $ \left(x-1\right)^2 $.
\begin{rlist}
\item Να βρεθούν οι τιμές των παραμέτρων $ a,\beta $.
\item Να βρεθούν όλες οι ρίζες του $ P(x) $.
\item Να γραφτεί το $P(x)$ σαν γινόμενο παραγόντων.
\end{rlist}
\newpage
\noindent
\item Να λυθούν οι παρακάτω ανισώσεις :
\begin{rlist}
\item $ (x+2)(5-x)(x^2-6x-7)>0 $
\item $ (x-2)^2(x+1)(x^2-7x+12)\geq0 $
\item $ 3x^4-x^3-9x^2+9x-2<0 $
\item $ (x^2-4)\leq8x^2+16x $
\end{rlist}
\item Δίνεται το πολυώνυμο \[ P(x)=x^4+ax^3+\beta x^2+\gamma x-2 \] το οποίο έχει ρίζα τον αριθμό $ -1 $ με πολλαπλότητα 3 (τριπλή λύση).
\begin{rlist}
\item Να βρεθούν οι τιμές των παραμέτρων $ a,\beta,\gamma $.
\item Να λυθεί η εξίσωση $ P(x)=0 $.
\end{rlist}
\item Το πολυώνυμο $ P(x)=x^3-4x^2+ax-2 $ έχει παράγοντα το $ x-2 $, ενώ αν διαιρεθεί με το $ x-3 $ αφήνει υπόλοιπο $ 4 $.
\begin{rlist}
\item Να βρεθεί η τιμή της παραμέτρου $ a $.
\item Να βρεθεί το υπόλοιπο της διαίρεσης του $ P(x) $ με το $ x^2-5x+6 $.
\item Να λυθεί η εξίσωση $ P(x)=18 $.
\item Να λυθεί η ανίσωση $ P(x)^2-P(x)\geq 0 $.
\end{rlist}
\end{enumerate}
\end{document}