\documentclass[11pt,a4paper,twocolumn]{article}
\usepackage[english,greek]{babel}
\usepackage[utf8]{inputenc}
\usepackage{nimbusserif}
\usepackage[T1]{fontenc}
\usepackage[left=1.50cm, right=1.50cm, top=2.00cm, bottom=2.00cm]{geometry}
\usepackage{amsmath}
\let\myBbbk\Bbbk
\let\Bbbk\relax
\usepackage[amsbb,subscriptcorrection,zswash,mtpcal,mtphrb,mtpfrak]{mtpro2}
\usepackage{graphicx,multicol,multirow,enumitem,tabularx,mathimatika,gensymb,venndiagram,hhline,longtable,tkz-euclide,fontawesome5,eurosym,tcolorbox,tabularray}
\usepackage[explicit]{titlesec}
\tcbuselibrary{skins,theorems,breakable}
\newlist{rlist}{enumerate}{3}
\setlist[rlist]{itemsep=0mm,label=\roman*.}
\newlist{alist}{enumerate}{3}
\setlist[alist]{itemsep=0mm,label=\alph*.}
\newlist{balist}{enumerate}{3}
\setlist[balist]{itemsep=0mm,label=\bf\alph*.}
\newlist{Alist}{enumerate}{3}
\setlist[Alist]{itemsep=0mm,label=\Alph*.}
\newlist{bAlist}{enumerate}{3}
\setlist[bAlist]{itemsep=0mm,label=\bf\Alph*.}
\newlist{askhseis}{enumerate}{3}
\setlist[askhseis]{label={\Large\thesection}.\arabic*.}
\renewcommand{\textstigma}{\textsigma\texttau}
\newlist{thema}{enumerate}{3}
\setlist[thema]{label=\bf\large{ΘΕΜΑ \textcolor{black}{\Alph*}},itemsep=0mm,leftmargin=0cm,itemindent=18mm}
\newlist{erwthma}{enumerate}{3}
\setlist[erwthma]{label=\bf{\large{\textcolor{black}{\Alph{themai}.\arabic*}}},itemsep=0mm,leftmargin=0.8cm}

\newcommand{\kerkissans}[1]{{\fontfamily{maksf}\selectfont \textbf{#1}}}
\renewcommand{\textdexiakeraia}{}

\usepackage[
backend=biber,
style=alphabetic,
sorting=ynt
]{biblatex}

\DeclareTblrTemplate{caption}{nocaptemplate}{}
\DeclareTblrTemplate{capcont}{nocaptemplate}{}
\DeclareTblrTemplate{contfoot}{nocaptemplate}{}
\NewTblrTheme{mytabletheme}{
\SetTblrTemplate{caption}{nocaptemplate}{}
\SetTblrTemplate{capcont}{nocaptemplate}{}
\SetTblrTemplate{contfoot}{nocaptemplate}{}
}

\NewTblrEnviron{mytblr}
\SetTblrStyle{firsthead}{font=\bfseries}
\SetTblrStyle{firstfoot}{fg=red2}
\SetTblrOuter[mytblr]{theme=mytabletheme}
\SetTblrInner[mytblr]{
rowspec={t{7mm}},columns = {c},
width = 0.85\linewidth,
row{odd} = {bg=red9,fg=black,ht=8mm},
row{even} = {bg=red7,fg=black,ht=8mm},
hlines={white},vlines={white},
row{1} = {bg=red4, fg=white, font=\bfseries\fontfamily{maksf}},rowhead = 1,
hline{2} = {.7mm}, % midrule  
}
\newcounter{askhsh}
\setcounter{askhsh}{1}
\newcommand{\askhsh}{\large\theaskhsh.\ \addtocounter{askhsh}{1}}

\titleformat{\section}{\Large}{\kerkissans{\thesection}}{10pt}{\Large\kerkissans{#1}}

\setlength{\columnsep}{5mm}
\titleformat{\paragraph}
{\large}%
{}{0em}%
{\textcolor{red!80!black}{\faSquare\ \ \kerkissans{\bmath{#1}}}}
\setlength{\parindent}{0pt}

\newcommand{\eng}[1]{\selectlanguage{english}#1\selectlanguage{greek}}
\newcommand{\tss}[1]{\textsuperscript{#1}}

\begin{document}
\twocolumn[{
\centering
\kerkissans{{\huge Αναγωγή στο 1ο τεταρτημόριο}\\\vspace{3mm} {\Large ΑΣΚΗΣΕΙΣ}}\vspace{5mm}}]
\paragraph{Υπολογισμός τριγωνομετρικών αριθμών}
\askhsh
Να υπολογίσετε τους τριγωνομετρικούς αριθμούς των παρακάτω γωνιών κάνοντας αναγωγή στο 1\tss{ο} τεταρτημόριο.
\begin{multicols}{4}
\begin{alist}[leftmargin=5mm]
\item $ 120\degree $
\item $ 150\degree $
\item $ 135\degree $
\item $ 495\degree $
\item $ 480\degree $
\item $ 510\degree $
\item $ 840\degree $
\item $ 1935\degree $
\end{alist}
\end{multicols}
\askhsh
Να υπολογίσετε τους τριγωνομετρικούς αριθμούς των παρακάτω γωνιών κάνοντας αναγωγή στο 1\tss{ο} τεταρτημόριο.
\begin{multicols}{3}
\begin{alist}
\item $ \dfrac{3\pi}{4} $
\item $ \dfrac{5\pi}{6} $
\item $ \dfrac{2\pi}{3} $
\item $ \dfrac{9\pi}{4} $
\item $ \dfrac{32\pi}{3} $
\item $ \dfrac{89\pi}{6} $
\end{alist}
\end{multicols}
\askhsh
Να υπολογίσετε τους τριγωνομετρικούς αριθμούς των παρακάτω γωνιών κάνοντας αναγωγή στο 1\tss{ο} τεταρτημόριο.
\begin{multicols}{4}
\begin{alist}[leftmargin=5mm]
\item $ -45\degree $
\item $ -30\degree $
\item $ -60\degree $
\item $ 330\degree $
\item $ 300\degree $
\item $ 315\degree $
\item $ 1020\degree $
\item $ 1395\degree $
\end{alist}
\end{multicols}
\askhsh
Να υπολογίσετε τους τριγωνομετρικούς αριθμούς των παρακάτω γωνιών κάνοντας αναγωγή στο 1\tss{ο} τεταρτημόριο.
\begin{multicols}{4}
\begin{alist}[leftmargin=5mm]
\item $ -\dfrac{\pi}{3}$
\item $ -\dfrac{\pi}{4} $
\item $ \dfrac{11\pi}{6} $
\item $ \dfrac{7\pi}{4} $
\item $ \dfrac{5\pi}{3} $
\item $ \dfrac{31\pi}{4} $
\item $ \dfrac{83\pi}{6} $
\item $ \dfrac{29\pi}{3} $
\end{alist}
\end{multicols}
\askhsh
Να υπολογίσετε τους τριγωνομετρικούς αριθμούς των παρακάτω γωνιών κάνοντας αναγωγή στο 1\tss{ο} τεταρτημόριο.
\begin{multicols}{4}
\begin{alist}[leftmargin=4mm]
\item $ 210\degree $
\item $ 240\degree $
\item $ 225\degree $
\item $ 570\degree $
\item $ 600\degree $
\item $ 945\degree $
\item $ 1680\degree $
\item $ 2760\degree $
\end{alist}
\end{multicols}
\askhsh
Να υπολογίσετε τους τριγωνομετρικούς αριθμούς των παρακάτω γωνιών κάνοντας αναγωγή στο 1\tss{ο} τεταρτημόριο.
\begin{multicols}{3}
\begin{alist}
\item $ \dfrac{7\pi}{6} $
\item $ \dfrac{5\pi}{4} $
\item $ \dfrac{4\pi}{3} $
\item $ \dfrac{29\pi}{4} $
\item $ \dfrac{55\pi}{6} $
\item $ \dfrac{34\pi}{3} $
\end{alist}
\end{multicols}
\askhsh Να υπολογίσετε τους τριγωνομετρικούς αριθμούς των παρακάτω γωνιών κάνοντας αναγωγή στο 1\tss{ο} τεταρτημόριο.
\begin{multicols}{3}
\begin{alist}
\item $ -120\degree $
\item $ -135\degree $
\item $ -210\degree $
\item $ -225\degree $
\item $ -315\degree $
\item $ -300\degree $
\end{alist}
\end{multicols}
\askhsh Να υπολογίσετε τους τριγωνομετρικούς αριθμούς των παρακάτω γωνιών κάνοντας αναγωγή στο 1\tss{ο} τεταρτημόριο.
\begin{multicols}{3}
\begin{alist}
\item $ -\dfrac{4\pi}{3} $
\item $ -\dfrac{7\pi}{6} $
\item $ -\dfrac{3\pi}{4} $
\item $ -\dfrac{5\pi}{6} $
\item $ -\dfrac{7\pi}{4} $
\item $ -\dfrac{5\pi}{3} $
\end{alist}
\end{multicols}
\askhsh Να υπολογίσετε την τιμή καθεμιάς από τις παρακάτω παραστάσεις.
\begin{alist}
\item $ \hm{40\degree}+\hm{140\degree}-2\syn{50}\degree $
\item $ \hm{50\degree}\cdot\syn{70\degree}+\hm{130\degree}\cdot\syn{110\degree} $
\item $ \ef{45\degree}\cdot \syf{135}-\hm^2{225\degree} $
\item $ \hm^2{35\degree}+\syn^2{145\degree} $
\item $ \ef^2{330\degree}+\syf^2{240\degree} $
\end{alist}
\askhsh Να υπολογίσετε την τιμή των παρακάτω παραστάσεων.
\begin{alist}
\item $\dfrac{\hm{120\degree\cdot\syn{240\degree-\ef{330\degree}}}}{\syn{150\degree\cdot\syf{240\degree+\hm{315\degree}}}}$
\end{alist}
\paragraph{Τριγωνομετρικές ταυτότητες}
\askhsh
Να αποδειχθούν οι παρακάτω τριγωνομετρικές ταυτότητες.
\begin{alist}
\item $ \hm{\left( \pi-x\right) }-\hm{x}=0 $
\item $ \syn^2{(\pi+x)}+\hm^2{(\pi-x)}=1 $
\item $ \ef{\left( \frac{\pi}{2}-x\right) }\cdot\ef{\left( \pi+x\right) }=1 $
\item $ \hm^2{(\pi-x)}+\syn^2{(-x)}=1 $
\end{alist}
\paragraph{Ταυτότητες τριγώνου}
\askhsh
Να δειχθεί ότι σε κάθε τρίγωνο $ AB\varGamma $ ισχύουν οι παρακάτω τριγωνομετρικές ταυτότητες.
\begin{alist}
\item $ \hm{(A+B)}=\hm{\varGamma} $
\item $ \syn{(B+\varGamma)}=\syn{(\pi-A)} $
\item $ \ef{(\pi-\varGamma-A)}=\syf{\left(\frac{\pi}{2}-B\right) } $
\item $ \hm{(A+B)}=\syn{\left(\varGamma-\frac{\pi}{2}\right) } $
\end{alist}
\askhsh
Να υπολογιστούν οι ζητούμενες γωνίες του τριγώνου $ AB\varGamma $ από τις παρακάτω εξισώσεις.
\begin{alist}
\item $ \hm{(A-B)}=\hm{\left(\varGamma+\frac{\pi}{2} \right) } $
\end{alist}
\end{document}
