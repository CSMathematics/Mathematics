\PassOptionsToPackage{no-math,cm-default}{fontspec}
\documentclass[twoside,nofonts,internet]{askhseis}
\usepackage{amsmath}
\usepackage{xgreek}
\let\hbar\relax
\defaultfontfeatures{Mapping=tex-text,Scale=MatchLowercase}
\setmainfont[Mapping=tex-text,Numbers=Lining,Scale=1.0,BoldFont={Minion Pro Bold}]{Minion Pro}
\newfontfamily\scfont{GFS Artemisia}
\font\icon = "Webdings"
\usepackage[amsbb]{mtpro2}
\usepackage{tikz,pgfplots}
\tkzSetUpPoint[size=7,fill=white]
\xroma{red!70!black}
\usepackage{tikz}
\usepackage{tkz-euclide}
\usepackage{wrapfig}
\usetkzobj{all}
\usepackage{calc}
\usepackage[framemethod=TikZ]{mdframed}
\newcommand{\ypogrammisi}[1]{\underline{\smash{#1}}}

\renewcommand{\thepart}{\arabic{part}}
\definecolor{steelblue}{cmyk}{.7,.278,0,.294}
\definecolor{royalblue}{RGB}{0,0,0}
\definecolor{doc}{cmyk}{1,0.455,0,0.569}
\definecolor{olivedrab}{cmyk}{0.25,0,0.75,0.44}

\usepackage{titletoc}
\usepackage[explicit]{titlesec}
\usepackage{graphicx}
\usepackage{multicol}
\usepackage{multirow}
\usepackage{enumitem}
\usepackage{tabularx}
\usepackage[decimalsymbol=comma]{siunitx}
\usepackage{eurosym}

\usepackage{sectsty}
\sectionfont{\centering}

\usepackage{systeme,regexpatch}
\makeatletter
% change the definition of \sysdelim not to store `\left` and `\right`
\def\sysdelim#1#2{\def\SYS@delim@left{#1}\def\SYS@delim@right{#2}}
\sysdelim\{. % reinitialize

% patch the internal command to use
% \LEFTRIGHT<left delim><right delim>{<system>}
% instead of \left<left delim<system>\right<right delim>
\regexpatchcmd\SYS@systeme@iii
  {\cB.\c{SYS@delim@left}(.*)\c{SYS@delim@right}\cE.}
  {\c{SYS@MT@LEFTRIGHT}\cB\{\1\cE\}}
  {}{}
\def\SYS@MT@LEFTRIGHT{%
  \expandafter\expandafter\expandafter\LEFTRIGHT
  \expandafter\SYS@delim@left\SYS@delim@right}
\makeatother
\newcommand{\synt}[2]{{\scriptsize \begin{matrix}
\times#1\\\\ \times#2
\end{matrix}}}

\begin{document}
\titlos{Άλγεβρα Β΄ Λυκείου}{Συστήματα}{Γραμμικά Συστήματα}
\begin{enumerate}[label=\bf\textcolor{royalblue}{{\large \arabic*.}},
itemsep=5mm]
\item Δίνεται η ευθεία $ 3x-2y=4 $.
\begin{enumerate}[label=\roman*.]
\item Να εξεταστεί αν τα σημεία $ A(2,1) $ και $ B(-1,3) $ ανήκουν στην ευθεία.
\item Να βρεθούν τα σημεία τομής της ευθείας με τους άξονες $ x'x $ και $ y'y $.
\item Να βρεθούν όλες οι λύσεις
\end{enumerate}
\begin{center}
\textcolor{royalblue}{\textbf{ΓΡΑΜΜΙΚΑ ΣΥΣΤΗΜΑΤΑ}}
\end{center}
\item Να λυθούν τα παρακάτω συστήματα με τη μέθοδο της αντικατάστασης.
\begin{multicols}{3}
\begin{enumerate}[label=\roman*.,itemsep=0mm]
\item $\ccases{
\;x+2y=3\\
\;4x-3y=1}$
\item $\ccases{
\;3x+2y=5\\
\;-x+y=-5}$
\item $\ccases{
\;2(x-1)-(y-2)=9\\
\;-(1-x)+3y=0}$
\end{enumerate}\end{multicols}
\item Να λυθούν τα παρακάτω συστήματα με τη μέθοδο των αντίθετων συντελεστών.
\begin{multicols}{3}
\begin{enumerate}[label=\roman*.,itemsep=0mm]
\item $\ccases{
\;4x-y=3\\
\;2x+3y=5}$
\item $\ccases{
\;2(x-2)+3(y+1)=1\\
\;4x-(2-y)=2}$
\item $\ccases{
\;\dfrac{x+2}{3}+\dfrac{1-y}{2}=2\\
\;\dfrac{2x-1}{5}+\dfrac{y}{3}=-\dfrac{2}{15}}$
\end{enumerate}\end{multicols}
\item Να λυθούν τα παρακάτω συστήματα με τη μέθοδο των οριζουσών.
\begin{multicols}{3}
\begin{enumerate}[label=\roman*.,itemsep=0mm]
\item $\ccases{
\;2x-y=3\\
\;x+3y=5}$
\item $\ccases{
\;2(x-1)+3(y+2)=13\\
\;x-(2y-1)=2}$
\item $\ccases{
\;6x-4y=2\\
\;3x-2y=4}$
\end{enumerate}\end{multicols}
\item Να λυθούν τα παρακάτω συστήματα
\begin{multicols}{2}
\begin{enumerate}[label=\roman*.,itemsep=3mm]
\item $\ccases{
\;(2x-1)(y+1)-(x+4)(2y-3)=1\\
\;(1-x)(3y+1)+(x+2)(3y+4)=2}$
\item $\ccases{
\;\dfrac{x+1}{3}-\dfrac{3(y-2)}{4}=1\\[3mm]
\;\dfrac{x}{2}-\dfrac{2-y}{2}=x+y}$
\item $\ccases{
\;\dfrac{x-1}{2}+\dfrac{x-y}{3}=1-2x\\[3mm]
\;\dfrac{3y-x}{4}-\dfrac{3(y-2x)}{2}=\dfrac{1}{8}}$
\item $\ccases{
\;\dfrac{3x^2-x+1}{3}-\dfrac{2x^2-y}{2}=-2\\[3mm]
\;\dfrac{5y^2-x}{5}-\dfrac{y(3y-2)}{3}=\dfrac{1}{8}}$
\end{enumerate}\end{multicols}
\begin{center}
\textcolor{royalblue}{\textbf{ΠΑΡΑΜΕΤΡΙΚΑ ΣΥΣΤΗΜΑΤΑ}}
\end{center}
\item Να λυθούν τα παρακάτω συστήματα
\begin{multicols}{3}
\begin{enumerate}[label=\roman*.,itemsep=0mm]
\item $\ccases{
\;(\lambda^2+1)x-y=2\\
\;2\lambda x+y=4}$
\item $\ccases{
\;(\lambda+2)x-3y=\lambda+2\\
\;\lambda x+(\lambda-2)y=1}$
\item $\ccases{
\;\lambda^2x+4y=2\lambda\\
\;(\lambda-1) x+y=\lambda-1}$
\end{enumerate}\end{multicols}
\item Να βρεθεί η τιμή του $ \lambda\in\mathbb{R} $ ώστε το παρακάτω σύστημα να είναι αδύνατο.
\begin{multicols}{2}
\begin{enumerate}[label=\roman*.,itemsep=0mm]
\item $\ccases{
\;\lambda x+(\lambda-3)y=-1\\
\;2x+(\lambda-3)y=1}$
\item $\ccases{
\;(\lambda+1) x-3y=-1\\
\;x+(\lambda-3)y=1}$
\end{enumerate}\end{multicols}
\item Να βρεθεί η τιμή του $ \lambda\in\mathbb{R} $ ώστε το παρακάτω σύστημα να έχει άπειρες λύσεις οι οποίες να βρεθούν.
\begin{multicols}{2}
\begin{enumerate}[label=\roman*.,itemsep=0mm]
\item $\ccases{
\;2\lambda x+(\lambda+3)y=2\\
\;x+\lambda y=-1}$
\item $\ccases{
\;x+\lambda y=2-\lambda\\
\;\lambda x+y=\lambda}$
\end{enumerate}\end{multicols}
\item Να βρεθεί η εξίσωση της ευθείας $ y=ax+\beta $ η οποία διέρχεται από τα σημεία
\begin{multicols}{2}
\begin{enumerate}[label=\roman*.,itemsep=0mm]
\item $ A(-2,1) $ και $ B(3,0) $
\item $ A(-1,1) $ και $ B(4,-2) $
\item $ A(0,-5) $ και $ B(2,4) $
\item $ A\PARENS{\dfrac{3}{2},-1} $ και $ B\PARENS{0,\dfrac{1}{2}} $
\end{enumerate}\end{multicols}
\begin{center}
\textcolor{royalblue}{\textbf{ΣΥΣΤΗΜΑΤΑ 3$\times$3}}
\end{center}
\item Να λυθούν τα παρακάτω συστήματα
\begin{multicols}{3}
\begin{enumerate}[label=\roman*.,itemsep=0mm]
\item $\ccases{
\;3x-2y+z=6\\
\;x-3y-z=3\\
\;2x+y-4z=-3}$
\item $\ccases{
\;x-2y+z=4\\
\;x-y-z=2\\
\;2x+3y-3z=0}$
\item $\ccases{
\;x-2y+3z=6\\
\;2x-4y+6z=12\\
\;x+y-z=0}$
\end{enumerate}\end{multicols}
\item Να βρεθεί η εξίσωση της παραβολής $ y=ax^2+\beta x+\gamma $ η οποία διέρχεται από τα σημεία
\begin{multicols}{2}
\begin{enumerate}[label=\roman*.,itemsep=0mm]
\item $ A(-2,1) $ , $ B(3,0) $ και $ \varGamma(1,-2) $
\item $ A(-1,1) $ , $ B(1,3) $ και $ \varGamma(0,-2) $
\item $ A(-4,3) $ , $ B(1,2) $ και $ \varGamma(0,1) $
\item $ A(-2,4) $ , $ B(3,9) $ και $ \varGamma(1,1) $
\end{enumerate}\end{multicols}
\item Αν $ D, D_{x}, D_{y} $ είναι οι ορίζουσες ενός συστήματος $ 2\times2 $ και ισχύει \[ \ccases{
\;D-2D_{x}-2D_{y}=-6\\
\;4D-3D_{x}-2D_{y}=-1\\
\;2D+3D_{x}-D_{y}=-4} \]
να βρεθούν οι λύσεις $ x,y\in\mathbb{R} $ του $ 2\times2 $ συστήματος.
\newpage
\noindent
\begin{center}
\textcolor{royalblue}{\textbf{ΣΥΣΤΗΜΑΤΑ ΜΕ ΑΝΑΘΕΣΗ}}
\end{center}
\item Να λυθούν τα παρακάτω συστήματα
\begin{multicols}{2}
\begin{enumerate}[label=\roman*.,itemsep=3mm]
\item $\ccases{
\;|x|-|y|=3\\
\;2|x|+3|y|=11}$
\item $\ccases{
\;|x-1|+2|y+2|=7\\
\;3|x-1|-4|y+2|=1}$
\item $\ccases{
\;x^2+2y^3=0\\
\;3x^2+5y^3=11}$
\item $\ccases{
\;\sqrt{x}-3\sqrt{y}=-1\\
\;2\sqrt{x}+9\sqrt{y}=13}$
\item $\ccases{
\;\dfrac{1}{x}+\dfrac{1}{y}=7\\
\;\dfrac{2}{x}-\dfrac{1}{2y}=4}$
\item $\ccases{
\;2(x^2+3x-3)+3(y^2-5y+7)=5\\
\;-(x^2+3x-3)+2(y^2-5y+7)=1}$
\item $\ccases{
\;2\textrm{ημ}x+\textrm{συν}y=2 & , x\in(0,\pi/2)\\
\;3\textrm{ημ}x-4\textrm{συν}y=\dfrac{5}{2} & , y\in(-\pi/2,\pi/2)}$
\end{enumerate}\end{multicols}
\begin{center}
\textcolor{royalblue}{\textbf{ΠΡΟΒΛΗΜΑΤΑ}}
\end{center}
\item Ένα ξενοδοχείο έχει 30 δωμάτια, άλλα δίκλινα και άλλα τρίκλινα. Συνολικά τα κρεβάτια είναι 80. Πόσα είναι τα δίκλινα και πόσα τα τρίκλινα δωμάτια;
\item Ένας μαθητής έχει στο πορτοφόλι του 15 χαρτονομίσματα. Κάποια είναι των 5\officialeuro\; και κάποια των 10\officialeuro. Με τα χρήματα αυτά αγοράζει ένα κινητό τηλέφωνο αξίας 112\officialeuro\; και παίρνει ρέστα 8\officialeuro. Πόσα χαρτονομίσματα είναι των 5\officialeuro\; και πόσα των 10\officialeuro?
\item Μια εταιρία κινητής τηλεφωνίας έχει τις εξής χρεώσεις : 0{,}07\officialeuro/sms και 0{,}09\officialeuro/1' ομιλίας. Ένας συνδρομητής, με μια κάρτα των 10\officialeuro\;ξόδεψε συνολικά 120 λεπτά και μηνύματα. Πόσα ήταν τα λεπτά ομιλίας και πόσα τα μηνύματα?
\item Ένας πατέρας είναι 32 χρόνια μεγαλύτερος από το γιό του. Σε 8 χρόνια ο πατέρας θα έχει τα 3πλάσια χρόνια από το γιό του. Ποιά είναι η ηλικία του πατέρα και του γιού?
\item Σε ένα κουτί υπάρχουν κόκκινες και πράσινες μπάλες. Αν προσθέσουμε στο κουτί 3 κόκκινες μπάλες, οι πράσινες θα είναι διπλάσιες από τις κόκκινες ενώ αν προσθέσουμε 4 πράσινες τότε, κόκκινες και πράσινες θα είναι ίσες. Πόσες μπάλες από το κάθε χρώμα υπάρχουν?
\item Σε μια φάρμα ζούν κότες και αγελάδες που είναι 80. Αν έχουν συνολικά 260 πόδια να βρεθούν πόσες κότες και πόσες αγελάδες ζούν στη φάρμα.
\item Σε ένα ορθογώνιο, το μήκος είναι διπλάσιο του πλάτους και η περίμετρος είναι ίση με το μήκος αυξημένο κατα 12 μέτρα. Να βρεθούν οι πλευρές του ορθογωνίου.
\item \begin{minipage}{\textwidth}
   \begin{wrapfigure}[10]{r}[0pt]{0cm}
\begin{tikzpicture}
   \tkzDefPoint(3,0){B}
   \tkzDefPoint(0,0){A}
   \tkzDefPoint(1.5,1){C}
   \tkzDrawPolygon(A,B,C)
   \tkzText[royalblue](.3,.6){{\scriptsize $ 2x-1 $}}
   \tkzText[royalblue](3.1,.6){{\scriptsize $ 3x+2y-4 $}}
   \end{tikzpicture}
   \end{wrapfigure}\mbox{}\\
Να βρεθούν πραγματικοί οι αριθμοί $ x, y\in\mathbb{R} $ ώστε το διπλανό τρίγωνο να είναι ισοσκελές. Η περίμετρος του τριγώνου είναι 28 μέτρα.
\end{minipage}\\\\
\end{enumerate}
\end{document}