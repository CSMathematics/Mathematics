\documentclass[11pt,a4paper,twocolumn]{article}
\usepackage[english,greek]{babel}
\usepackage[utf8]{inputenc}
\usepackage{nimbusserif}
\usepackage[T1]{fontenc}
\usepackage[left=1.50cm, right=1.50cm, top=2.00cm, bottom=2.00cm]{geometry}
\usepackage{amsmath}
\let\myBbbk\Bbbk
\let\Bbbk\relax
\usepackage[amsbb,subscriptcorrection,zswash,mtpcal,mtphrb,mtpfrak]{mtpro2}
\usepackage{graphicx,multicol,multirow,enumitem,tabularx,mathimatika,gensymb,venndiagram,hhline,longtable,tkz-euclide,fontawesome5,eurosym,tcolorbox,tabularray}
\usepackage[explicit]{titlesec}
\tcbuselibrary{skins,theorems,breakable}
\newlist{rlist}{enumerate}{3}
\setlist[rlist]{itemsep=0mm,label=\roman*.}
\newlist{alist}{enumerate}{3}
\setlist[alist]{itemsep=0mm,label=\alph*.}
\newlist{balist}{enumerate}{3}
\setlist[balist]{itemsep=0mm,label=\bf\alph*.}
\newlist{Alist}{enumerate}{3}
\setlist[Alist]{itemsep=0mm,label=\Alph*.}
\newlist{bAlist}{enumerate}{3}
\setlist[bAlist]{itemsep=0mm,label=\bf\Alph*.}
\newlist{askhseis}{enumerate}{3}
\setlist[askhseis]{label={\Large\thesection}.\arabic*.}
\renewcommand{\textstigma}{\textsigma\texttau}
\newlist{thema}{enumerate}{3}
\setlist[thema]{label=\bf\large{ΘΕΜΑ \textcolor{black}{\Alph*}},itemsep=0mm,leftmargin=0cm,itemindent=18mm}
\newlist{erwthma}{enumerate}{3}
\setlist[erwthma]{label=\bf{\large{\textcolor{black}{\Alph{themai}.\arabic*}}},itemsep=0mm,leftmargin=0.8cm}

\newcommand{\kerkissans}[1]{{\fontfamily{maksf}\selectfont \textbf{#1}}}
\renewcommand{\textdexiakeraia}{}

\usepackage[
backend=biber,
style=alphabetic,
sorting=ynt
]{biblatex}

\DeclareTblrTemplate{caption}{nocaptemplate}{}
\DeclareTblrTemplate{capcont}{nocaptemplate}{}
\DeclareTblrTemplate{contfoot}{nocaptemplate}{}
\NewTblrTheme{mytabletheme}{
\SetTblrTemplate{caption}{nocaptemplate}{}
\SetTblrTemplate{capcont}{nocaptemplate}{}
\SetTblrTemplate{contfoot}{nocaptemplate}{}
}

\NewTblrEnviron{mytblr}
\SetTblrStyle{firsthead}{font=\bfseries}
\SetTblrStyle{firstfoot}{fg=red2}
\SetTblrOuter[mytblr]{theme=mytabletheme}
\SetTblrInner[mytblr]{
rowspec={t{7mm}},columns = {c},
width = 0.85\linewidth,
row{odd} = {bg=red9,fg=black,ht=8mm},
row{even} = {bg=red7,fg=black,ht=8mm},
hlines={white},vlines={white},
row{1} = {bg=red4, fg=white, font=\bfseries\fontfamily{maksf}},rowhead = 1,
hline{2} = {.7mm}, % midrule  
}
\newcounter{askhsh}
\setcounter{askhsh}{1}
\newcommand{\askhsh}{\large\theaskhsh.\ \addtocounter{askhsh}{1}}

\titleformat{\section}{\Large}{\kerkissans{\thesection}}{10pt}{\Large\kerkissans{#1}}

\setlength{\columnsep}{5mm}
\titleformat{\paragraph}
{\large}%
{}{0em}%
{\textcolor{red!80!black}{\faSquare\ \ \kerkissans{\bmath{#1}}}}
\setlength{\parindent}{0pt}

\newcommand{\eng}[1]{\selectlanguage{english}#1\selectlanguage{greek}}

\begin{document}
\twocolumn[{
\centering
\kerkissans{{\huge Γραμμικά συστήματα}\\\vspace{3mm} {\Large ΑΣΚΗΣΕΙΣ}}\vspace{5mm}}]
\paragraph{Γραμμική εξίσωση}
\askhsh Για καθεμιά από τις παρακάτω εξισώσεις να γραφτούν οι συντελεστές καθώς και ο σταθερός όρος.
\begin{multicols}{2}
\begin{alist}
\item $ 2x+y=8 $
\item $ -3x+7y=-1 $
\item $ x=2 $
\item $ y=9 $
\item $ \sqrt{2}x+\frac{y}{4}=1 $
\item $ 2x=y $
\end{alist}
\end{multicols}
\askhsh Να εξεταστεί αν το σημείο $ A(2,1) $ ανήκει σε καθεμία από τις παρακάτω ευθείες.
\begin{multicols}{2}
\begin{alist}
\item $ x-3y=4 $
\item $ 2x+3y=7 $
\item $ 4x+2y=5 $
\item $ 8x-7y=9 $
\item $ y=3 $
\item $ x=2 $
\end{alist}
\end{multicols}
\askhsh Να βρεθεί ποιο ή ποια από τα παρακάτω σημεία ανήκουν στην ευθεία $ x+4y=9 $.
\begin{multicols}{2}
\begin{alist}
\item $ A(2,-3) $
\item $ A(1,2) $
\item $ A(-3,3) $
\item $ A(0,2) $
\end{alist}
\end{multicols}
\askhsh Να βρεθούν τα σημεία τομής των παρακάτω ευθειών με τους άξονες $ x'x $ και $ y'y $.
\begin{multicols}{2}
\begin{alist}
\item $ x-2y=4 $
\item $ 4x-y=8 $
\item $ 2x-3y=-6 $
\item $ 7x-4y=11 $
\end{alist}
\end{multicols}
\askhsh Να βρεθούν τα σημεία τομής των παρακάτω ευθειών με τους άξονες $ x'x $ και $ y'y $.
\begin{multicols}{2}
\begin{alist}
\item $ x=3 $
\item $ y=5 $
\item $ -2x=-7 $
\item $ 2y=4 $
\end{alist}
\end{multicols}
\askhsh Να βρεθεί η μορφή των λύσεων καθεμιάς από τις παρακάτω εξισώσεις.
\begin{multicols}{2}
\begin{alist}
\item $ x-2y=4 $
\item $ 3x+4y=7 $
\item $ x=7 $
\item $ y=-4 $
\end{alist}
\end{multicols}
\askhsh Να σχεδιαστούν οι ακόλουθες ευθείες σε ορθογώνιο σύστημα συντεταγμένων.
\begin{multicols}{2}
\begin{alist}
\item $ x-3y=6 $
\item $ 2x-y=-3 $
\item $ x=5 $
\item $ y=3 $
\end{alist}
\end{multicols}
\paragraph{Γραμμικό Σύστημα}
\askhsh Για καθένα από τα παρακάτω γραμμικά συστήματα, να γράψετε τους συντελεστές των μεταβλητών και τους σταθερούς όρους.
\begin{alist}
\begin{multicols}{2}
\item $ \systeme{3x-y=5,2x+4y=7} $
\item $ \systeme{x-y=4,8x+5y=0} $
\item $ \systeme{\frac{3x}{4}-\frac{y}{2}=\frac{1}{5},\frac{3x}{4}-\frac{y}{2}=\frac{1}{5}} $
\item $ \systeme{{0,1}x+{1.2}y=2,x-{0,4}y=3} $
\end{multicols}
\item $ \systeme{\sqrt{3}x-\sqrt{2}y=4,{\left(\sqrt{5}-1\right)}x-y=9} $
\end{alist}
\askhsh Για καθένα από τα παρακάτω γραμμικά συστήματα, να γράψετε τους συντελεστές των μεταβλητών και τους σταθερούς όρους.
\begin{multicols}{2}
\begin{alist}
\item $ \systeme{x-y=5,y=3} $
\item $ \systeme{x=2,y=-4} $
\item $ \systeme{x=2y,x-y=2} $
\item $ \systeme{x+3y=0,y=0} $
\end{alist}
\end{multicols}
\paragraph{Μεθοδος Αντικαταστασης}
\askhsh Να λυθούν τα παρακάτω γραμμικά συστήματα με τη μέθοδο της αντικατάστασης.
\begin{multicols}{2}
\begin{alist}
\item $ \systeme{x-y=1,x=4} $
\item $ \systeme{2x+4y=8,y=3} $
\item $ \systeme{x=4,x-y=9} $
\item $ \systeme{x-y=2,x+y=8} $
\end{alist}
\end{multicols}
\askhsh Να λυθούν τα παρακάτω γραμμικά συστήματα με τη μέθοδο της αντικατάστασης.
\begin{multicols}{2}
\begin{alist}
\item $ \systeme{3x+2y=5,2x-y=1} $
\item $ \systeme{x+4y=-2,3x-7y=13} $
\item $ \systeme{4x-3y=-1,5x-2y=4} $
\item $ \systeme{7x+2y=29,3x-y=18} $
\end{alist}
\end{multicols}
\askhsh Να λυθούν τα παρακάτω γραμμικά συστήματα με τη μέθοδο της αντικατάστασης.
\begin{multicols}{2}
\begin{alist}
\item $ \systeme{x-2y=4,2x-4y=8} $
\item $ \systeme{3x-4y=1,-6x+8y=-2} $
\item $ \systeme{4x+2y=6,6x+3y=9} $
\end{alist}
\end{multicols}
\askhsh Να λυθούν τα παρακάτω γραμμικά συστήματα με τη μέθοδο της αντικατάστασης.
\begin{multicols}{2}
\begin{alist}
\item $ \systeme{-x+y=2,2x-2y=3} $
\item $ \systeme{x=2y-1,4x-8y=5} $
\item $ \systeme{2x+y=1,y=7-2x} $
\end{alist}
\end{multicols}
\paragraph{Μεθοδος Αντιθετων Συντελεστων}
\askhsh Να λυθούν τα παρακάτω γραμμικά συστήματα με τη μέθοδο των αντίθετων συντελεστών.
\begin{multicols}{2}
\begin{alist}
\item $ \systeme{x-y=3,x+y=7} $
\item $ \systeme{2x-3y=1,4x-5y=1} $
\item $ \systeme{x+y=10,3x+y=16} $
\item $ \systeme{-x-y=4,7x+4y=-19} $
\end{alist}
\end{multicols}
\askhsh Να λυθούν τα παρακάτω γραμμικά συστήματα με τη μέθοδο των αντίθετων συντελεστών.
\begin{multicols}{2}
\begin{alist}
\item $ \systeme{4x-5y=-1,3x+7y=10} $
\item $ \systeme{4x-y=7,x+2y=4} $
\item $ \systeme{11x-8y=27,5x+9y=-13} $
\item $ \systeme{8x+6y=28,7x-5y=4} $
\end{alist}
\end{multicols}
\paragraph{Μεθοδος Οριζουσων}
\askhsh Να λυθούν τα παρακάτω γραμμικά συστήματα με τη μέθοδο των οριζουσών.
\begin{multicols}{2}
\begin{alist}
\item $ \systeme{2x+y=5,x-4y=-2} $
\item $ \systeme{3x+5y=16,4x-y=6} $
\item $ \systeme{x+5y=12,7x+3y=20} $
\item $ \systeme{6x-y=20,4x+9y=-6} $
\end{alist}
\end{multicols}
\askhsh Να λυθούν τα παρακάτω γραμμικά συστήματα με τη μέθοδο των οριζουσών.
\begin{multicols}{2}
\begin{alist}
\item $ \systeme{2x+y=5,4x+2y=3} $
\item $ \systeme{4x+2y=8,6x+3y=12} $
\item $ \systeme{x-y=3,2x-2y=5} $
\item $ \systeme{3x+5y=2,6x+10y=4} $
\end{alist}
\end{multicols}
\paragraph{Γραφικη Επιλυση}
\askhsh Να λυθούν γραφικά τα παρακάτω γραμμικά συστήματα.
\begin{multicols}{2}
\begin{alist}
\item $ \systeme{x-y=3,3x+y=13} $
\item $ \systeme{2x+y=4,x+4y=8} $
\item $ \systeme{3x-y=2,6x-2y=4} $
\item $ \systeme{x-2y=-3,-2x+4y=5} $
\end{alist}
\end{multicols}
\askhsh Να βρεθούν, αν υπάρχουν, τα κοινά σημεία των παρακάτω ευθειών.
\begin{alist}
\item $ x+3y=6 $ και $ 2x+y=8 $
\item $ 3x+4y=5 $ και $ -x+5y=3 $
\item $ 2x-y=10 $ και $ 4x-2y=7 $
\item $ 3x-y=2 $ και $ 6x-2y=4 $
\end{alist}
\paragraph{Συνθετα Συστηματα}
\askhsh Να λυθούν τα παρακάτω γραμμικά συστήματα με οποιαδήποτε μέθοδο επίλυσης.
\begin{alist}
\item $ \ccases{2(x-1)+3(y+2)=11\\x+3-(4-y)=2} $
\item $ \ccases{3(x+y)-2y=1+x\\x-4y+2=3x+4} $
\end{alist}
\askhsh Να λυθούν τα παρακάτω γραμμικά συστήματα με οποιαδήποτε μέθοδο επίλυσης.
\begin{alist}
\item $ \ccases{4(x-3)+3(y+2)=1\\3x-5=2(3-y)+2} $
\item $ \ccases{5(x-y)+3(2x+y)=16\\15-x-y=3x+2y-2} $
\end{alist}
\askhsh Να λυθούν τα παρακάτω γραμμικά συστήματα με οποιαδήποτε μέθοδο επίλυσης.

\begin{alist}[leftmargin=3mm]
\item $ \ccases{
2(x-1)-(y-2)=9\\
-(1-x)+3y=0} $
\item $ \ccases{
2(x-1)+3(y+2)=13\\
x-(2y-1)=2} $
\item $\ccases{
\;2(x-2)+3(y+1)=1\\
\;4x-(2-y)=2}$
\end{alist}
\askhsh Να λυθούν τα παρακάτω γραμμικά συστήματα με οποιαδήποτε μέθοδο.
\begin{enumerate}[label=\roman*.,itemsep=3mm]
\item $\ccases{
\;(2x-1)(y+1)-(x+4)(2y-3)=1\\
\;(1-x)(3y+1)+(x+2)(3y+4)=2}$
\item $\ccases{
\;\dfrac{x+1}{3}-\dfrac{3(y-2)}{4}=1\\[3mm]
\;\dfrac{x}{2}-\dfrac{2-y}{2}=x+y}$
\item $\ccases{
\;\dfrac{x-1}{2}+\dfrac{x-y}{3}=1-2x\\[3mm]
\;\dfrac{3y-x}{4}-\dfrac{3(y-2x)}{2}=\dfrac{1}{8}}$
\item $\ccases{
\;\dfrac{3x^2-x+1}{3}-\dfrac{2x^2-y}{2}=-2\\[3mm]
\;\dfrac{5y^2-x}{5}-\dfrac{y(3y-2)}{3}=\dfrac{1}{8}}$
\end{enumerate}
\paragraph{Γραμμικά Συστήματα $ 3\times3$}
\askhsh Να επιλυθούν τα παρακάτω $ 3\times3 $ γραμμικά συστήματα.
\begin{multicols}{2}
\begin{alist}
\item $\ccases{
3x-2y+z=6\\
x-3y-z=3\\
2x+y-4z=-3}$
\item $\ccases{
x-2y+z=4\\
x-y-z=2\\
2x+3y-3z=0}$
\item $\ccases{
x-2y+3z=6\\
2x-4y+6z=12\\
x+y-z=0}$
\end{alist}\end{multicols}
\askhsh Οι ορίζουσες $ D,D_x,D_y $ ενός $ 2\times2 $ γραμμικού συστήματος με μεταβλητές $ x,y $ ικανοποιούν τις παρακάτω εξισώσεις :
\[ \ccases{
\;D-2D_{x}-2D_{y}=-6\\
\;4D-3D_{x}-2D_{y}=-1\\
\;2D+3D_{x}-D_{y}=-4} \]
Να βρεθεί η λύση $ (x,y) $ του $ 2\times2 $ γραμμικού συστήματος.\\
\askhsh Να βρεθεί η εξίσωση της παραβολής $ y=ax^2+\beta x+\gamma $ η οποία διέρχεται από τα σημεία
\begin{alist}
\item $ A(-2,1) $ , $ B(3,0) $ και $ \varGamma(1,-2) $
\item $ A(-1,1) $ , $ B(1,3) $ και $ \varGamma(0,-2) $
\item $ A(-4,3) $ , $ B(1,2) $ και $ \varGamma(0,1) $
\item $ A(-2,4) $ , $ B(3,9) $ και $ \varGamma(1,1) $
\end{alist}
\paragraph{Προβλήματα}
\askhsh Ένα ξενοδοχείο έχει 45 δωμάτια, άλλα δίκλινα και άλλα τρίκλινα. Συνολικά τα κρεβάτια είναι 110. Πόσα είναι τα δίκλινα και πόσα τα τρίκλινα δωμάτια;\\\\
\askhsh Ένας μαθητής έχει στο πορτοφόλι του 15 χαρτονομίσματα. Κάποια είναι των 5\officialeuro\; και κάποια των 10\officialeuro. Με τα χρήματα αυτά αγοράζει ένα κινητό τηλέφωνο αξίας 112\officialeuro\; και παίρνει ρέστα 8\officialeuro. Πόσα χαρτονομίσματα είναι των 5\officialeuro\; και πόσα των 10\officialeuro;\\\\
\askhsh Μια εταιρία κινητής τηλεφωνίας έχει τις εξής χρεώσεις : 0{,}07\officialeuro/sms και 0{,}09\officialeuro/1' ομιλίας. Ένας συνδρομητής, με μια κάρτα των 10\officialeuro\;ξόδεψε συνολικά 120 λεπτά και μηνύματα. Πόσα ήταν τα λεπτά ομιλίας και πόσα τα μηνύματα;\\\\
\askhsh Ένας πατέρας είναι 32 χρόνια μεγαλύτερος από το γιο του. Σε 8 χρόνια ο πατέρας θα έχει τα 3πλάσια χρόνια από το γιο του. Ποια είναι η ηλικία του πατέρα και του γιου;\\\\
\askhsh Σε ένα κουτί υπάρχουν κόκκινες και πράσινες μπάλες. Αν προσθέσουμε στο κουτί 3 κόκκινες μπάλες, οι πράσινες θα είναι διπλάσιες από τις κόκκινες ενώ αν προσθέσουμε 4 πράσινες τότε, κόκκινες και πράσινες θα είναι ίσες. Πόσες μπάλες από το κάθε χρώμα υπάρχουν;\\\\
\askhsh Σε μια φάρμα ζουν 80 σε πλήθος κότες και αγελάδες. Αν όλα τα ζώα έχουν συνολικά 260 πόδια να βρεθούν πόσες κότες και πόσες αγελάδες ζουν στη φάρμα.\\\\
\askhsh Σε ένα ορθογώνιο, το μήκος είναι διπλάσιο του πλάτους ενώ η περίμετρος είναι ίση με το μήκος αυξημένο κατά 12 μέτρα. Να βρεθούν οι πλευρές του ορθογωνίου.\\\\
\askhsh Η περίμετρος του τριγώνου του παρακάτω σχήματος είναι $ 38 $ εκατοστά. Να βρεθούν πραγματικοί οι αριθμοί $ x, y\in\mathbb{R} $ ώστε το τρίγωνο να είναι ισοσκελές.
\vspace{-5mm}
\begin{center}
\begin{tikzpicture}
\tkzDefPoint(3,0){C}
\tkzDefPoint(0,0){B}
\tkzDefPoint(1.5,1){A}
\tkzDrawPolygon[pl](A,B,C)
\tkzText(.2,.6){{\scriptsize $ 2x+6 $}}
\tkzText(3.2,.6){{\scriptsize $ 3x+2y-4 $}}
\tkzText(1.5,-.24){{\scriptsize $ y+7 $}}
\tkzLabelPoint[above](A){$A$}
\tkzLabelPoint[left](B){$B$}
\tkzLabelPoint[right](C){$\varGamma$}
\tkzDrawPoints(A,B,C)
\end{tikzpicture}
\end{center}
\paragraph{Παραμετρικά Συστήματα - Εύρεση παραμέτρου}
\askhsh Να βρεθούν οι τιμές της παραμέτρου $ \lambda\in\mathbb{R} $ ώστε η ευθεία $ \lambda x+(\lambda-1)y=4 $ να διέρχεται από το σημείο $ A(-2,3) $.\\\\
\askhsh Να βρεθούν οι τιμές της παραμέτρου $ \lambda\in\mathbb{R} $ ώστε η ευθεία $ (\lambda^2-1)x+(1-\lambda)y=2 $ να διέρχεται από το σημείο $ A(1,3) $.\\\\
\askhsh Αν γνωρίζουμε ότι το σημείο $ A(3\lambda-1,4-\lambda) $ ανήκει στην ευθεία $ 2x+3y=1 $ τότε να βρεθούν οι τιμές της παραμέτρου $ \lambda\in\mathbb{R} $.\\\\
\askhsh Δίνεται το παρακάτω παραμετρικό σύστημα με παράμετρο $ \lambda\in\mathbb{R} $.
\[ \ccases{
\lambda x+(\lambda+2)y=\lambda\\
x+ \lambda y=\lambda-1} \]
\begin{alist}
\item Να βρεθούν οι ορίζουσες $ D,D_x,D_y $ του συστήματος με τη βοήθεια της παραμέτρου $ \lambda $.
\item Να εξεταστεί για ποιες τιμές της παραμέτρου το σύστημα έχει μοναδική λύση.
\item Για ποια τιμή της παραμέτρου το σύστημα είναι αόριστο και για ποια αδύνατο;
\end{alist}
\askhsh Να βρεθούν οι λύσεις των παρακάτω συστημάτων για κάθε τιμή της παραμέτρου $ \lambda\in\mathbb{R} $.
\begin{alist}
\item $\ccases{
2\lambda x+(\lambda+3)y=2\\
x+\lambda y=-1}$
\item $\ccases{
x+\lambda y=2-\lambda\\
\lambda x+y=\lambda}$
\end{alist}
\askhsh Να βρεθούν οι λύσεις των παρακάτω συστημάτων για κάθε τιμή της παραμέτρου $ \lambda\in\mathbb{R} $.
\begin{alist}
\item $\ccases{
(\lambda^2+1)x-y=2\\
2\lambda x+y=4}$
\item $\ccases{
(\lambda+2)x-3y=\lambda+2\\
\lambda x+(\lambda-2)y=1}$
\item $\ccases{
\lambda^2x+4y=2\lambda\\
(\lambda-1) x+y=\lambda-1}$
\end{alist}
\askhsh Να βρεθούν οι λύσεις των παρακάτω συστημάτων για κάθε τιμή της παραμέτρου $ \lambda\in\mathbb{R} $.
\begin{alist}
\item $\ccases{
\lambda x+(\lambda-3)y=-1\\
2x+(\lambda-3)y=1}$
\item $\ccases{
(\lambda+1) x-3y=-1\\
x+(\lambda-3)y=1}$
\end{alist}
\end{document}
