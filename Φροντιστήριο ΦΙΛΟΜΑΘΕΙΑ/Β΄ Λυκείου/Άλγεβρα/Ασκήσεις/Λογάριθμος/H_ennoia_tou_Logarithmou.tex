\documentclass[11pt,a4paper,twocolumn]{article}
\usepackage[english,greek]{babel}
\usepackage[utf8]{inputenc}
\usepackage{nimbusserif}
\usepackage[T1]{fontenc}
\usepackage[left=1.50cm, right=1.50cm, top=2.00cm, bottom=2.00cm]{geometry}
\usepackage{amsmath}
\let\myBbbk\Bbbk
\let\Bbbk\relax
\usepackage[amsbb,subscriptcorrection,zswash,mtpcal,mtphrb,mtpfrak]{mtpro2}
\usepackage{graphicx,multicol,multirow,enumitem,tabularx,mathimatika,gensymb,venndiagram,hhline,longtable,tkz-euclide,fontawesome5,eurosym,tcolorbox,tabularray}
\usepackage[explicit]{titlesec}
\tcbuselibrary{skins,theorems,breakable}
\newlist{rlist}{enumerate}{3}
\setlist[rlist]{itemsep=0mm,label=\roman*.}
\newlist{alist}{enumerate}{3}
\setlist[alist]{itemsep=0mm,label=\alph*.}
\newlist{balist}{enumerate}{3}
\setlist[balist]{itemsep=0mm,label=\bf\alph*.}
\newlist{Alist}{enumerate}{3}
\setlist[Alist]{itemsep=0mm,label=\Alph*.}
\newlist{bAlist}{enumerate}{3}
\setlist[bAlist]{itemsep=0mm,label=\bf\Alph*.}
\newlist{askhseis}{enumerate}{3}
\setlist[askhseis]{label={\Large\thesection}.\arabic*.}
\renewcommand{\textstigma}{\textsigma\texttau}
\newlist{thema}{enumerate}{3}
\setlist[thema]{label=\bf\large{ΘΕΜΑ \textcolor{black}{\Alph*}},itemsep=0mm,leftmargin=0cm,itemindent=18mm}
\newlist{erwthma}{enumerate}{3}
\setlist[erwthma]{label=\bf{\large{\textcolor{black}{\Alph{themai}.\arabic*}}},itemsep=0mm,leftmargin=0.8cm}

\newcommand{\kerkissans}[1]{{\fontfamily{maksf}\selectfont \textbf{#1}}}
\renewcommand{\textdexiakeraia}{}

\usepackage[
backend=biber,
style=alphabetic,
sorting=ynt
]{biblatex}

\DeclareTblrTemplate{caption}{nocaptemplate}{}
\DeclareTblrTemplate{capcont}{nocaptemplate}{}
\DeclareTblrTemplate{contfoot}{nocaptemplate}{}
\NewTblrTheme{mytabletheme}{
\SetTblrTemplate{caption}{nocaptemplate}{}
\SetTblrTemplate{capcont}{nocaptemplate}{}
\SetTblrTemplate{contfoot}{nocaptemplate}{}
}

\NewTblrEnviron{mytblr}
\SetTblrStyle{firsthead}{font=\bfseries}
\SetTblrStyle{firstfoot}{fg=red2}
\SetTblrOuter[mytblr]{theme=mytabletheme}
\SetTblrInner[mytblr]{
rowspec={t{7mm}},columns = {c},
width = 0.85\linewidth,
row{odd} = {bg=red9,fg=black,ht=8mm},
row{even} = {bg=red7,fg=black,ht=8mm},
hlines={white},vlines={white},
row{1} = {bg=red4, fg=white, font=\bfseries\fontfamily{maksf}},rowhead = 1,
hline{2} = {.7mm}, % midrule  
}
\newcounter{askhsh}
\setcounter{askhsh}{1}
\newcommand{\askhsh}{{\large\theaskhsh.}\ \addtocounter{askhsh}{1}}

\titleformat{\section}{\Large}{\kerkissans{\thesection}}{10pt}{\Large\kerkissans{#1}}

\setlength{\columnsep}{5mm}
\titleformat{\paragraph}
{\large}%
{}{0em}%
{\textcolor{red!80!black}{\faSquare\ \ \kerkissans{\bmath{#1}}}}
\setlength{\parindent}{0pt}

\newcommand{\eng}[1]{\selectlanguage{english}#1\selectlanguage{greek}}

\begin{document}
\twocolumn[{
\centering
\kerkissans{{\huge Η έννοια του λογαρίθμου}\\\vspace{3mm} {\Large ΑΣΚΗΣΕΙΣ}}\vspace{5mm}}]
\paragraph{Υπολογιστικές}
\askhsh Να υπολογίσετε την τιμή των παρακάτω λογαρίθμων.
\begin{multicols}{4}
\begin{alist}[leftmargin=3mm]
\item $ \log_{2}{4} $
\item $ \log_{3}{9} $
\item $ \log_{5}{125} $
\item $ \log_{2}{16} $
\item $ \log_{3}{27} $
\item $ \log_{4}{16} $
\item $ \log_{2}{32} $
\item $ \log_{2}{64} $
\end{alist}
\end{multicols}
\askhsh Να υπολογίσετε την τιμή των παρακάτω λογαρίθμων.
\begin{multicols}{3}
\begin{alist}
\item $ \log{100} $
\item $ \log{10000} $
\item $ \log{10^7} $
\item $ \log{10^{-19}} $
\item $ \ln{e^2} $
\item $ \ln{e^{-23}} $
\end{alist}
\end{multicols}
\askhsh Να υπολογίσετε την τιμή των παρακάτω λογαρίθμων.
\begin{multicols}{3}
\begin{alist}
\item $ \log_{2}{\dfrac{1}{4}} $
\item $ \log_{2}{\dfrac{1}{32}} $
\item $ \log_{3}{\dfrac{1}{9}} $
\item $ \log_{3}{\dfrac{1}{81}} $
\item $ \log_{4}{\dfrac{1}{64}} $
\item $ \log_{8}{\dfrac{1}{512}} $
\end{alist}
\end{multicols}
\askhsh Να υπολογίσετε την τιμή των παρακάτω λογαρίθμων.
\begin{multicols}{3}
\begin{alist}
\item $ \log{\dfrac{1}{10}} $
\item $ \log{\dfrac{1}{1000}} $
\item $ \log{\dfrac{1}{10^{-3}}} $
\item $ \ln{\dfrac{1}{e}} $
\item $ \ln{\dfrac{1}{e^5}} $
\item $ \ln{\dfrac{1}{e^{-4}}} $
\end{alist}
\end{multicols}
\askhsh Να υπολογίσετε την τιμή των παρακάτω λογαρίθμων.
\begin{multicols}{3}
\begin{alist}
\item $ \log_{2}{0{,}25} $
\item $ \log_{2}{0{,}125} $
\item $ \log_{5}{0{,}04} $
\item $ \log_{8}{0{,}125} $
\item $ \log{0{,}0001} $
\item $ \log_{100}{0{,}01} $
\end{alist}
\end{multicols}
\askhsh Να υπολογίσετε την τιμή των παρακάτω λογαρίθμων.
\begin{multicols}{3}
\begin{alist}[leftmargin=4mm]
\item $ \log_{0{,}1}{0{,}01} $
\item $ \log_{0{,}2}{0{,}008} $
\item $ \log_{0{,}3}{0{,}0081} $
\item $ \log_{1{,}5}{2{,}25} $
\item $ \log_{0{,}4}{6{,}25} $
\item $ \log_{0{,}5}{8} $
\end{alist}
\end{multicols}
\askhsh Να υπολογίσετε την τιμή των παρακάτω λογαρίθμων.
\begin{multicols}{3}
\begin{alist}[leftmargin=4mm]
\item $ \log_{\frac{3}{2}}{\dfrac{9}{4}} $
\item $ \log_{\frac{1}{4}}{\dfrac{1}{64}} $
\item $ \log_{\frac{5}{7}}{\dfrac{125}{343}} $
\item $ \log_{\frac{1}{10}}{\dfrac{1}{10000}} $
\item $ \log_{\frac{4}{e}}{\dfrac{16}{e^2}} $
\item $ \log_{\frac{e}{10}}{\dfrac{e^3}{1000}} $
\end{alist}
\end{multicols}
\askhsh Να υπολογίσετε την τιμή των παρακάτω λογαρίθμων.
\begin{multicols}{2}
\begin{alist}
\item $ \log_{\frac{4}{3}}{\dfrac{9}{16}} $
\item $ \log_{\frac{8}{5}}{\dfrac{125}{512}} $
\item $ \log_{\frac{1}{10}}{1000} $
\item $ \log_{\frac{1}{2}}{16} $
\item $ \log_{\frac{1}{5}}{625} $
\item $ \log_{\frac{1}{4}}{256} $
\end{alist}
\end{multicols}
\askhsh Να υπολογίσετε την τιμή των παρακάτω λογαρίθμων.
\begin{multicols}{2}
\begin{alist}
\item $ \log_{\sqrt{2}}{4} $
\item $ \log_{\sqrt{3}}{3} $
\item $ \log_{\sqrt{5}}{25} $
\item $ \log_{\sqrt{e}}{e^3} $
\item $ \log_{\sqrt{2}}{4\sqrt{2}} $
\item $ \log_{\sqrt[3]{4}}{2} $
\end{alist}
\end{multicols}
\askhsh Να υπολογίσετε τις παρακάτω αριθμητικές παραστάσεις.
\begin{multicols}{2}
\begin{alist}
\item $ \log_{4}{8}+\log_{4}{2} $
\item $ \log_{8}{32}+\log_{8}{16} $
\item $ \log_{6}{12}+\log_{6}{3} $
\item $ \log{20}+\log{50} $
\end{alist}
\end{multicols}
\askhsh Να υπολογίσετε τις παρακάτω αριθμητικές παραστάσεις.
\begin{alist}
\item $ \log_{8}{16}+\log_{8}{32} $
\item $ \log_{9}{27}+\log_{9}{3} $
\item $ \log_{12}{36}+\log_{12}{48} $
\item $ \log{250}+\log{4000} $
\end{alist}

\askhsh Να υπολογίσετε τις παρακάτω αριθμητικές παραστάσεις.
\begin{multicols}{2}
\begin{alist}
\item $ \log_{2}{8}-\log_{2}{2} $
\item $ \log_{3}{54}-\log_{3}{2} $
\item $ \log_{5}{500}-\log_{5}{20} $
\item $ \log{300}-\log{3} $
\end{alist}
\end{multicols}
\askhsh Να υπολογίσετε τις παρακάτω αριθμητικές παραστάσεις.
\begin{multicols}{2}
\begin{alist}
\item $ \ln{e^4}-\ln{e^2} $
\item $ \log{10^7}-\log{1000} $
\item $ \log{7500}-\log{75} $
\item $ \ln{4e^5}-\ln{4} $
\end{alist}
\end{multicols}
\askhsh Να υπολογίσετε τις παρακάτω αριθμητικές παραστάσεις.
\begin{alist}
\item $ \log_{2}{24}+\log_{2}{20}-\log_{2}{15} $
\item $ \log_{4}{12}+\log_{4}{48}-\log_{4}{9} $
\item $ \log_{3}{90}-\log_{3}{2}-\log_{3}{5} $
\item $ \log_{4}{12}+\log_{4}{48}-\log_{4}{9} $
\end{alist}
\askhsh Να υπολογίσετε τις παρακάτω αριθμητικές παραστάσεις.
\begin{alist}
\item $ \log_{3}{36}-2\log_{3}{2} $
\item $ 3\log_{4}{8}+\log_{4}{32} $
\item $ 5\log{2}+2\log{25}+\log{5} $
\item $ 4\log_{5}{10}+3\log_{5}{20}-5\log_{5}{4} $
\end{alist}
\askhsh Να υπολογίσετε τις παρακάτω αριθμητικές παραστάσεις.
\begin{alist}
\item $ \log_{4}{\sqrt{8}}+\dfrac{1}{2}\log_{4}{2} $
\item $ \dfrac{1}{3}\log_{2}{64}-
\dfrac{1}{2}\log_{2}{8} $
\item $ \log_{8}{\sqrt[3]{16}}+\dfrac{2}{3}\log_{8}{4} $
\item $ \log{\sqrt{10}}+\dfrac{3}{2}\log{1000} $
\end{alist}
\askhsh Να υπολογίσετε τις παρακάτω αριθμητικές παραστάσεις.
\begin{alist}
\item $ \log_{2}{\left(2+\sqrt{3}\right) }+\log_{2}{\left(2-\sqrt{3}\right)} $
\item $ \log_{3}{\left(1+\sqrt{10}\right) }+\log_{3}{\left(\sqrt{10}-1\right)} $
\item $ \log_{3}{\left(1+\sqrt[3]{2}\right) }+\log_{3}{\left(1-\sqrt[3]{2}+\sqrt[3]{4}\right)} $
\item $ \log{\left(5-\sqrt[3]{25}\right) }+\log{\left(25-5\sqrt[3]{25}+\sqrt[3]{25}^2\right)} $
\end{alist}
\askhsh Να υπολογίσετε τις παρακάτω αριθμητικές παραστάσεις.
\begin{multicols}{2}
\begin{alist}
\item $ 2^{\log_{2}{12}-\log_{2}{3}} $
\item $ 5^{\log_{5}{25}+\log_{5}{4}} $
\item $ 10^{3\log{5}+\log{8}} $
\item $ e^{\ln{9e^5}-3\ln{3e}} $
\item $ 10^{2\log{\sqrt{1000}}-\log{10}} $
\item $ e^{2\ln{\sqrt{3e}}-\ln{3}} $
\end{alist}
\end{multicols}
\askhsh Να υπολογίσετε τις παρακάτω αριθμητικές παραστάσεις.
\begin{multicols}{2}
\begin{alist}
\item $ \sqrt{\log_4{64}} $
\item $ \sqrt{\log_2{512}} $
\item $ \sqrt{\log{10^{25}}} $
\item $ \sqrt{\ln{e^9}} $
\item $ \sqrt[\log{1000}]{8} $
\item $ \sqrt[\ln{e^4}]{64} $
\item $ \sqrt[\log{10^5}]{4\log_2{256}} $
\item $ \sqrt[\ln{e^3}]{9\log_5{125}} $
\end{alist}
\end{multicols}
\askhsh Να απλοποιήσετε τις παρακάτω παραστάσεις.
\begin{multicols}{2}
\begin{alist}
\item $ \dfrac{\log{27}+\ln{8}}{\log{3}+\ln{2}} $
\item $ \dfrac{\log{100}-\ln{25}}{\log{1000}-\ln{125}} $
\item $ \dfrac{\log_5{48}}{\log_5{3}+\log_5{16}} $
\item $ \dfrac{\log_2{144}-\log_2{3}}{\log_2{6}} $
\end{alist}
\end{multicols}
\paragraph{Επίλυση εξίσωσης}
\askhsh Να υπολογίσετε τον πραγματικό αριθμό $ x>0 $ στις παρακάτω παραστάσεις.
\begin{multicols}{2}
\begin{alist}
\item $ \log_{2}{x}=3 $
\item $ \log_{3}{x}=2 $
\item $ \log_{3}{x}=4 $
\item $ \log_{5}{x}=3 $
\item $ \log_{4}{x}=-3 $
\item $ \log_{2}{x}=-5 $
\end{alist}
\end{multicols}
\askhsh Να υπολογίσετε τον πραγματικό αριθμό $ x>0 $ στις παρακάτω παραστάσεις.
\begin{multicols}{2}
\begin{alist}
\item $ \log{x}=3 $
\item $ \log{x}=-2 $
\item $ \ln{x}=4 $
\item $ \ln{x}=-3 $
\end{alist}
\end{multicols}
\askhsh Να υπολογίσετε τον πραγματικό αριθμό $ x>0 $ στις παρακάτω παραστάσεις.
\begin{multicols}{2}
\begin{alist}
\item $ \log_{0{,}1}{x}=2 $
\item $ \log_{0{,}2}{x}=4 $
\item $ \log_{0{,}5}{x}=3 $
\item $ \log_{0{,}2}{x}=-2 $
\end{alist}
\end{multicols}
\askhsh Να υπολογίσετε τον πραγματικό αριθμό $ x>0 $ στις παρακάτω παραστάσεις.
\begin{multicols}{2}
\begin{alist}
\item $ \log_{\frac{3}{4}}{x}=2 $
\item $ \log_{\frac{2}{5}}{x}=3 $
\item $ \log_{\frac{1}{3}}{x}=4 $
\item $ \log_{\frac{1}{10}}{x}=-2 $
\item $ \log_{\frac{1}{8}}{x}=-3 $
\item $ \log_{\frac{1}{e}}{x}=-4 $
\end{alist}
\end{multicols}
\askhsh Να υπολογίσετε τον πραγματικό αριθμό $ x>0 $ στις παρακάτω παραστάσεις.
\begin{multicols}{2}
\begin{alist}
\item $ \log_{\sqrt{2}}{x}=3 $
\item $ \log_{\sqrt{3}}{x}=2 $
\item $ \log_{\sqrt[3]{2}}{x}=6 $
\item $ \log_{\sqrt[5]{4}}{x}=15 $
\item $ \log_{\sqrt{5}}{x}=-3 $
\item $ \log_{\sqrt{3}}{x}=-4 $
\end{alist}
\end{multicols}
\askhsh Να υπολογίσετε τον πραγματικό αριθμό $ x $ στις παρακάτω παραστάσεις.
\begin{multicols}{2}
\begin{alist}
\item $ \log_{2}{(x-1)}=3 $
\item $ \log_{3}{(2x-3)}=2 $
\item $ \log_{4}{(4-x)}=2 $
\item $ \log{\left( x^2-24\right) }=3 $
\item $ \ln{ex}=2 $
\item $ \log{x^3}=9 $
\end{alist}
\end{multicols}
\askhsh Να υπολογίσετε τον πραγματικό αριθμό $ x $ στις παρακάτω παραστάσεις.
\begin{multicols}{2}
\begin{alist}[leftmargin=4mm]
\item $ \log_{2}{|x-3|}=2 $
\item $ \log_{3}{\sqrt{x-2}}=2 $
\item $ \log_{5}{\left(10x-x^2\right) }=2 $
\item $ \log{\dfrac{1}{x}}=2 $
\item $ \log_{9}{\sqrt[3]{3x-1}}=\dfrac{1}{2} $
\item $ \log{\dfrac{x-2}{x+3}}=1 $
\end{alist}
\end{multicols}
\askhsh Να υπολογίσετε τον πραγματικό αριθμό $ x>0$ με $ x\neq 1 $ στις παρακάτω παραστάσεις.
\begin{multicols}{2}
\begin{alist}
\item $ \log_{x}{8}=3 $
\item $ \log_{x}{4}=2 $
\item $ \log_{x}{27}=3 $
\item $ \log_{x}{64}=3 $
\item $ \log_{x}{625}=4 $
\item $ \log_{x}{343}=3 $
\end{alist}
\end{multicols}
\askhsh Να υπολογίσετε τον πραγματικό αριθμό $ x>0$ με $ x\neq 1 $ στις παρακάτω παραστάσεις.
\begin{multicols}{2}
\begin{alist}
\item $ \log_{x}{100}=2 $
\item $ \log_{x}{10^8}=8 $
\item $ \log_{x}{e^3}=3 $
\item $ \log_{x}{e}=1 $
\end{alist}
\end{multicols}
\askhsh Να υπολογίσετε τον πραγματικό αριθμό $ x>0$ με $ x\neq 1 $ στις παρακάτω παραστάσεις.
\begin{multicols}{2}
\begin{alist}
\item $ \log_{x}{4}=-2 $
\item $ \log_{x}{25}=-2 $
\item $ \log_{x}{\dfrac{1}{64}}=-3 $
\item $ \log_{x}{\dfrac{1}{49}}=-2 $
\item $ \log_{x}{\dfrac{1}{100}}=-2 $
\item $ \log_{x}{e^2}=-2 $
\end{alist}
\end{multicols}
\askhsh Να υπολογίσετε τον πραγματικό αριθμό $ x>0$ με $ x\neq 1 $ στις παρακάτω παραστάσεις.
\begin{multicols}{2}
\begin{alist}
\item $ \log_{x}{\dfrac{4}{25}}=2 $
\item $ \log_{x}{\dfrac{125}{64}}=3 $
\item $ \log_{x}{\dfrac{81}{16}}=-4 $
\item $ \log_{x}{1000}=-3 $
\end{alist}
\end{multicols}
\askhsh Να υπολογίσετε τον πραγματικό αριθμό $ x>0$ με $ x\neq 1 $ στις παρακάτω παραστάσεις.
\begin{multicols}{2}
\begin{alist}
\item $ \log_{x}{0{,}04}=2 $
\item $ \log_{x}{0{,}125}=3 $
\item $ \log_{x}{0{,}0001}=4 $
\item $ \log_{x}{1000}=-3 $
\end{alist}
\end{multicols}
\askhsh Να υπολογίσετε τον πραγματικό αριθμό $ x>0 $ με $ x\neq 1 $ στις παρακάτω παραστάσεις.
\begin{multicols}{2}
\begin{alist}
\item $ \log_{x}{2}=2 $
\item $ \log_{x}{3}=3 $
\item $ \log_{x}{10}=2 $
\item $ \log_{x}{e}=2 $
\end{alist}
\end{multicols}
\askhsh Να υπολογίσετε τον πραγματικό αριθμό $ x $ στις παρακάτω παραστάσεις.
\begin{multicols}{2}
\begin{alist}
\item $ \log_{x-1}{4}=2 $
\item $ \log_{2x-1}{27}=3 $
\item $ \log_{3-x}{16}=2 $
\item $ \log_{x^2}{81}=2 $
\end{alist}
\end{multicols}
\askhsh Να υπολογίσετε τον πραγματικό αριθμό $ x $ στις παρακάτω παραστάσεις.
\begin{multicols}{2}
\begin{alist}
\item $ \log_{\sqrt{x}}{5}=2 $
\item $ \log_{\sqrt[3]{2-x}}{7}=3 $
\item $ \log_{\sqrt[4]{3x-4}}{16}=8 $
\item $ \log_{\sqrt{x^2-3}}{13}=2 $
\end{alist}
\end{multicols}
\askhsh Να υπολογίσετε τον πραγματικό αριθμό $ x $ στις παρακάτω παραστάσεις.
\begin{multicols}{2}
\begin{alist}
\item $ \log_{|x-2|}{25}=2 $
\item $ \log_{x^2-3x+4}{2}=2 $
\item $ \log_{x^3-1}{7}=1 $
\item $ \log_{\sqrt[3]{x^2-2x}}{9}=6 $
\end{alist}
\end{multicols}
\paragraph{Αλγεβρικές παραστάσεις}
\askhsh Να απλοποιήσετε τις παρακάτω παραστάσεις.
\begin{multicols}{2}
\begin{alist}
\item $ \log{x^2}+\log{x} $
\item $ \log{x^3}+\log{x^4} $
\item $ \log{4x^2}+\log{25x^3} $
\item $ \log{8x}+\log{\dfrac{125}{x}} $
\end{alist}
\end{multicols}
\askhsh Να απλοποιήσετε τις παρακάτω παραστάσεις.
\begin{multicols}{2}
\begin{alist}
\item $ \log{x^2y}-\log{xy^2} $
\item $ \log{x^3}-\log{x^4} $
\item $ 3\log{xy}-\log{x^3} $
\item $ 2\log{2y}-2\log{\dfrac{5}{y}} $
\end{alist}
\end{multicols}
\askhsh Να απλοποιήσετε τις παρακάτω παραστάσεις.
\begin{multicols}{2}
\begin{alist}
\item $ \ln{\sqrt{x}}+\ln{x^2} $
\item $ 3\ln{\sqrt[3]{y}}+\ln{y} $
\item $ 2\ln{e\sqrt{x}}-\ln{\dfrac{x}{e^2}} $
\item $ \ln{\dfrac{4x}{e}}-2\ln{\dfrac{2x}{e}} $
\end{alist}
\end{multicols}
\askhsh Να απλοποιήσετε τις παρακάτω παραστάσεις.
\begin{alist}
\item $ \log{(x+y)}+\log{(x-y)} $
\item $ \ln{(2-x)}+\ln{(x+2)} $
\item $ \log{\left( a^3-\beta^3\right) }-\log{(a-\beta)} $
\item $ \ln{(ex-e^2)}+\ln{(x+e)} $
\end{alist}
\end{document}
