\documentclass[11pt,a4paper,twocolumn]{article}
\usepackage[english,greek]{babel}
\usepackage[utf8]{inputenc}
\usepackage{nimbusserif}
\usepackage[T1]{fontenc}
\usepackage[left=1.50cm, right=1.50cm, top=2.00cm, bottom=2.00cm]{geometry}
\usepackage{amsmath}
\let\myBbbk\Bbbk
\let\Bbbk\relax
\usepackage[amsbb,subscriptcorrection,zswash,mtpcal,mtphrb,mtpfrak]{mtpro2}
\usepackage{graphicx,multicol,multirow,enumitem,tabularx,mathimatika,gensymb,venndiagram,hhline,longtable,tkz-euclide,fontawesome5,eurosym,tcolorbox,tabularray}
\usepackage[explicit]{titlesec}
\tcbuselibrary{skins,theorems,breakable}
\newlist{rlist}{enumerate}{3}
\setlist[rlist]{itemsep=0mm,label=\roman*.}
\newlist{alist}{enumerate}{3}
\setlist[alist]{itemsep=0mm,label=\alph*.}
\newlist{balist}{enumerate}{3}
\setlist[balist]{itemsep=0mm,label=\bf\alph*.}
\newlist{Alist}{enumerate}{3}
\setlist[Alist]{itemsep=0mm,label=\Alph*.}
\newlist{bAlist}{enumerate}{3}
\setlist[bAlist]{itemsep=0mm,label=\bf\Alph*.}
\newlist{askhseis}{enumerate}{3}
\setlist[askhseis]{label={\Large\thesection}.\arabic*.}
\renewcommand{\textstigma}{\textsigma\texttau}
\newlist{thema}{enumerate}{3}
\setlist[thema]{label=\bf\large{ΘΕΜΑ \textcolor{black}{\Alph*}},itemsep=0mm,leftmargin=0cm,itemindent=18mm}
\newlist{erwthma}{enumerate}{3}
\setlist[erwthma]{label=\bf{\large{\textcolor{black}{\Alph{themai}.\arabic*}}},itemsep=0mm,leftmargin=0.8cm}

\newcommand{\kerkissans}[1]{{\fontfamily{maksf}\selectfont \textbf{#1}}}
\renewcommand{\textdexiakeraia}{}

\usepackage[
backend=biber,
style=alphabetic,
sorting=ynt
]{biblatex}

\DeclareTblrTemplate{caption}{nocaptemplate}{}
\DeclareTblrTemplate{capcont}{nocaptemplate}{}
\DeclareTblrTemplate{contfoot}{nocaptemplate}{}
\NewTblrTheme{mytabletheme}{
\SetTblrTemplate{caption}{nocaptemplate}{}
\SetTblrTemplate{capcont}{nocaptemplate}{}
\SetTblrTemplate{contfoot}{nocaptemplate}{}
}

\NewTblrEnviron{mytblr}
\SetTblrStyle{firsthead}{font=\bfseries}
\SetTblrStyle{firstfoot}{fg=red2}
\SetTblrOuter[mytblr]{theme=mytabletheme}
\SetTblrInner[mytblr]{
rowspec={t{7mm}},columns = {c},
width = 0.85\linewidth,
row{odd} = {bg=red9,fg=black,ht=8mm},
row{even} = {bg=red7,fg=black,ht=8mm},
hlines={white},vlines={white},
row{1} = {bg=red4, fg=white, font=\bfseries\fontfamily{maksf}},rowhead = 1,
hline{2} = {.7mm}, % midrule  
}
\newcounter{askhsh}
\setcounter{askhsh}{1}
\newcommand{\askhsh}{\large\theaskhsh.\ \addtocounter{askhsh}{1}}

\titleformat{\section}{\Large}{\kerkissans{\thesection}}{10pt}{\Large\kerkissans{#1}}

\setlength{\columnsep}{5mm}
\titleformat{\paragraph}
{\large}%
{}{0em}%
{\textcolor{red!80!black}{\faSquare\ \ \kerkissans{\bmath{#1}}}}
\setlength{\parindent}{0pt}

\newcommand{\eng}[1]{\selectlanguage{english}#1\selectlanguage{greek}}

\begin{document}
\twocolumn[{
\centering
\kerkissans{{\huge Η λογαριθμική συνάρτηση}\\\vspace{3mm} {\Large ΑΣΚΗΣΕΙΣ}}\vspace{5mm}}]
\paragraph{Πεδίο ορισμού}
\askhsh Να βρεθεί το πεδίο ορισμού των παρακάτω λογαριθμικών συναρτήσεων.
\begin{alist}
\item $ f(x)=\log{(x-2)} $
\item $ g(x)=\log{(3-x)} $
\item $ h(x)=\log{(3x-4)} $
\item $ d(x)=\log{\left[ 4(x-2)+5\right] } $
\end{alist}
\askhsh Να βρεθεί το πεδίο ορισμού των παρακάτω λογαριθμικών συναρτήσεων.
\begin{alist}
\item $ f(x)=\log{x^2} $
\item $ g(x)=\log{\left( x^2-4\right) } $
\item $ h(x)=\log{\left( x^2-x+2\right) } $
\item $ d(x)=\log{\left( x^2+6x+9\right) } $
\item $ r(x)=\log{\left(x^2+3x+5\right) } $
\end{alist}
\askhsh Να βρεθεί το πεδίο ορισμού των παρακάτω λογαριθμικών συναρτήσεων.
\begin{alist}
\item $ f(x)=\log{\left(x^3+x-2\right)} $
\item $ g(x)=\log{\left( x^3-3x^2-4x+12\right) } $
\item $ h(x)=\log{\left( x^2+2x^2-7x+4\right) } $
\item $ d(x)=\log{\left( -x^3+4x+3\right) } $
\item $ r(x)=\log{\left(x^4+3x^3+x^2-3x-2\right) } $
\end{alist}
\askhsh Να βρεθεί το πεδίο ορισμού των παρακάτω λογαριθμικών συναρτήσεων.
\begin{alist}
\item $ f(x)=\log{\dfrac{3}{x-1}} $
\item $ g(x)=\log{\dfrac{x-1}{x+2}} $
\item $ h(x)=\log{\dfrac{2x-1}{x^2}} $
\item $ d(x)=\log{\left( 1+\dfrac{1}{x}\right) } $
\end{alist}
\askhsh Να βρεθεί το πεδίο ορισμού των παρακάτω λογαριθμικών συναρτήσεων.
\begin{alist}
\item $ f(x)=\log{\sqrt{x}} $
\item $ g(x)=\log{\left( \sqrt{x}-2\right) } $
\item $ h(x)=\log{\sqrt{x-2}} $
\item $ d(x)=\log{\sqrt{x^2-3x} } $
\end{alist}
\askhsh Να βρεθεί το πεδίο ορισμού των παρακάτω λογαριθμικών συναρτήσεων.
\begin{alist}
\item $ f(x)=\log{|x|} $
\item $ g(x)=\log{|x+3| } $
\item $ h(x)=\log{|2x-1|-3} $
\item $ d(x)=\log{\left| x^2-x-2\right|  } $
\end{alist}
\askhsh Να βρεθεί το πεδίο ορισμού των παρακάτω λογαριθμικών συναρτήσεων.
\begin{alist}
\item $ f(x)=\log{\left(2^x-4\right)} $
\item $ g(x)=\log{\left(27-3^x\right)} $
\item $ h(x)=\log{\left(\dfrac{1}{2^x}-1\right)} $
\item $ d(x)=\log{\left(5^x+2\right)} $
\end{alist}
\paragraph{Γραφική παράσταση}
\askhsh Να χαράξετε στο ίδιο σύστημα συντεταγμένων, τις γραφικές παραστάσεις των συναρτήσεων $f,g$ και $h$.
\begin{alist}[leftmargin=4mm]
\item $f(x)=\log{x},g(x)=\log{x}+2$ και $h(x)=\log{x}-1$
\item $f(x)=\log_2{x},g(x)=\log_2{x}+1$ και $h(x)=\log_2{x}-3$
\item $f(x)=\ln{x},g(x)=\ln{x}-e$ και $h(x)=\ln{x}+\pi$
\end{alist}
\askhsh Να χαράξετε στο ίδιο σύστημα συντεταγμένων, τις γραφικές παραστάσεις των συναρτήσεων $f,g$ και $h$.
\begin{alist}
\item $f(x)=\log{x},g(x)=\log{(x-1)}$ και $h(x)=\log{(x+3)}$
\item $f(x)=\ln{x},g(x)=\ln{(x+3)}$ και $h(x)=\ln{(x-2)}$
\item $f(x)=\log_{\frac{1}{2}}{x},g(x)=\log_{\frac{1}{2}}{\left(x+\dfrac{1}{2}\right)}$ και $h(x)=\log_{\frac{1}{2}}{(x-0{,}3)}$
\end{alist}
\askhsh Να χαράξετε στο ίδιο σύστημα συντεταγμένων, τις γραφικές παραστάσεις των συναρτήσεων $f,g$ και $h$.
\begin{alist}
\item $f(x)=\log{x},g(x)=\log{(x-2)+1}$ και $h(x)=\log{(x+4)-2}$
\item $f(x)=\log_{0{,}5}{x},g(x)=\log_{0{,}5}{(x+3)}-2$ και $h(x)=\log_{0{,}5}{(x-1)+4}$
\end{alist}
\askhsh Να χαράξετε στο ίδιο σύστημα συντεταγμένων, τις γραφικές παραστάσεις των συναρτήσεων $f,g$ και $h$.
\begin{alist}
\item $f(x)=\log{x},g(x)=\log{(10x)}$ και $h(x)=\log{\left(\dfrac{x}{100}\right)}$
\item $f(x)=\log_{\frac{1}{3}}{x},g(x)=\log_{\frac{1}{3}}{\left(\dfrac{x}{27}\right)}$ και $h(x)=\log_{\frac{1}{3}}{(3x)}$
\item $f(x)=\ln{x},g(x)=\ln{(ex)}$ και $h(x)=\ln{(e^2x)}$
\end{alist}
\askhsh Να χαράξετε στο ίδιο σύστημα συντεταγμένων, τις γραφικές παραστάσεις των συναρτήσεων $f,g$ και $h$.
\begin{alist}
\item $f(x)=\log{x},g(x)=\log{\left(\dfrac{x-2}{10}\right)}$ και $h(x)=\log{\left(100x+100\right)}$
\item $f(x)=\log_{\frac{1}{2}}{x},g(x)=\log_{\frac{1}{2}}{\left(\dfrac{x+3}{8}\right)}$ και $h(x)=\log_{\frac{1}{2}}{(8x-16)}$
\end{alist}
\askhsh Να σχεδιάσετε τη γραφική παράσταση της συνάρτησης $f$.
\begin{multicols}{2}
\begin{alist}
\item $f(x)=\log{|x|}$
\item $f(x)=|\log{x}|$
\item $f(x)=\log{(-x)}$
\item $f(x)=-\log{x}$
\item $f(x)=\log{x^2}$
\item $f(x)=\log{\sqrt{x}}$
\end{alist}
\end{multicols}
\paragraph{Εξισώσεις}
\askhsh Να λυθούν οι παρακάτω λογαριθμικές εξισώσεις.
\begin{multicols}{2}
\begin{alist}
\item $\log{x}=2$
\item $\log_2{(x-1)}=3$
\item $\log_3{(4-x)}=2$
\item $\log_{\frac{1}{2}}{(2x)}=4$
\item $\ln{x}=3$
\item $\ln{(x+e)}=-1$
\end{alist}
\end{multicols}
\askhsh Να λυθούν οι παρακάτω λογαριθμικές εξισώσεις.
\begin{alist}
\item $\log_2{x}=\log_2{(x-1)}$
\item $\log{(x+3)}=\log{(1-x)}$
\item $\ln{(x^2)}-\ln{2x}=0$
\item $\log_3{x}=\log_3{\left(\dfrac{1}{x}\right)}$
\item $\ln{(x^3)}=\ln{x}$
\item $\log_{\frac{1}{2}}{x}=\log_{\frac{1}{2}}{(3x)}-2$
\end{alist}
\askhsh Να λυθούν οι παρακάτω λογαριθμικές εξισώσεις.
\begin{alist}
\item $\log{x}+\log{(4x)}=2$
\item $\log_2{x}+\log_2{(x-2)}=3$
\item $\log_{\frac{1}{2}}{x}+\log_{\frac{1}{2}}{(1-x)}=2$
\item $\log{(2x^2)}-\log{x}=1$
\item $\ln{x}=\ln{\sqrt{x}}-2$
\item $\log_2{(x^2)}-1=\log_2{(3x)}$
\end{alist}
\askhsh Να λύσετε τις ακόλουθες εξισώσεις.
\begin{alist}
\item $\log^2{x}-2\log{x}-3=0$
\item $\log_2^2{x}+3\log_2{x}-10=0$
\item $\ln^2{x}+(e+1)\ln{x}+e=0$
\item $\log^2{(x-1)}-4\log{(x-1)}+3=0$
\end{alist}
\paragraph{Ανισώσεις}
\askhsh Να λυθούν οι ακόλουθες ανισώσεις.
\begin{multicols}{2}
\begin{alist}
\item $\log{x}>1$
\item $\log{(x+1)}\leq 2$
\item $\log_2{x}\geq 3$
\item $\log_3{(2-x)}<-1$
\item $\ln{(x-e)}>-1$
\item $\ln{(2x-4)}<0$
\end{alist}
\end{multicols}
\paragraph{Άρτιες - Περιττές}
\askhsh Να μελετήσετε τις παρακάτω συναρτήσεις ως προς την συμμετρία.
\begin{multicols}{2}
\begin{alist}[leftmargin=6mm]
\item $f(x)=\log{\left(x^2+1\right)}$
\item $f(x)=\log{\left(3^x\right)}$
\item $f(x)=\ln{(4-x^2)}$
\item $f(x)=\ln\left(\dfrac{1-x}{1+x}\right)$
\item $f(x)=\dfrac{\ln{2^x}}{x}$
\item $f(x)=\syn{x}\cdot\log{4^x}$
\end{alist}
\end{multicols}
\paragraph{Συνδυαστικές}
\askhsh Έστω η συνάρτηση $ f $ με τύπο :
\[ f(x)=\ln\frac{x^2-4x+3}{x^3-9x} \]
\begin{alist}
\item Να βρεθεί το πεδίο ορισμού της συνάρτησης $ f $. Στη συνέχεια να απλοποιήσετε τον τύπο της.
\item Να βρεθεί τα σημεία στο οποίο η γραφική παράσταση της συνάρτησης $ f $ τέμνει τους άξονες $x'x$ και $ y'y $.
\item 
\end{alist}
\end{document}
