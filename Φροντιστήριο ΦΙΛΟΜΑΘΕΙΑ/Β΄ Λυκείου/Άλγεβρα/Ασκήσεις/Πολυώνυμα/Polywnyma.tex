\documentclass[11pt,a4paper,twocolumn]{article}
\usepackage[english,greek]{babel}
\usepackage[utf8]{inputenc}
\usepackage{nimbusserif}
\usepackage[T1]{fontenc}
\usepackage[left=1.50cm, right=1.50cm, top=2.00cm, bottom=2.00cm]{geometry}
\usepackage{amsmath}
\let\myBbbk\Bbbk
\let\Bbbk\relax
\usepackage[amsbb,subscriptcorrection,zswash,mtpcal,mtphrb,mtpfrak]{mtpro2}
\usepackage{graphicx,multicol,multirow,enumitem,tabularx,mathimatika,gensymb,venndiagram,hhline,longtable,tkz-euclide,fontawesome5,eurosym,tcolorbox,tabularray}
\usepackage[explicit]{titlesec}
\tcbuselibrary{skins,theorems,breakable}
\newlist{rlist}{enumerate}{3}
\setlist[rlist]{itemsep=0mm,label=\roman*.}
\newlist{alist}{enumerate}{3}
\setlist[alist]{itemsep=0mm,label=\alph*.}
\newlist{balist}{enumerate}{3}
\setlist[balist]{itemsep=0mm,label=\bf\alph*.}
\newlist{Alist}{enumerate}{3}
\setlist[Alist]{itemsep=0mm,label=\Alph*.}
\newlist{bAlist}{enumerate}{3}
\setlist[bAlist]{itemsep=0mm,label=\bf\Alph*.}
\newlist{askhseis}{enumerate}{3}
\setlist[askhseis]{label={\Large\thesection}.\arabic*.}
\renewcommand{\textstigma}{\textsigma\texttau}
\newlist{thema}{enumerate}{3}
\setlist[thema]{label=\bf\large{ΘΕΜΑ \textcolor{black}{\Alph*}},itemsep=0mm,leftmargin=0cm,itemindent=18mm}
\newlist{erwthma}{enumerate}{3}
\setlist[erwthma]{label=\bf{\large{\textcolor{black}{\Alph{themai}.\arabic*}}},itemsep=0mm,leftmargin=0.8cm}

\newcommand{\kerkissans}[1]{{\fontfamily{maksf}\selectfont \textbf{#1}}}
\renewcommand{\textdexiakeraia}{}

\usepackage[
backend=biber,
style=alphabetic,
sorting=ynt
]{biblatex}

\DeclareTblrTemplate{caption}{nocaptemplate}{}
\DeclareTblrTemplate{capcont}{nocaptemplate}{}
\DeclareTblrTemplate{contfoot}{nocaptemplate}{}
\NewTblrTheme{mytabletheme}{
\SetTblrTemplate{caption}{nocaptemplate}{}
\SetTblrTemplate{capcont}{nocaptemplate}{}
\SetTblrTemplate{contfoot}{nocaptemplate}{}
}

\NewTblrEnviron{mytblr}
\SetTblrStyle{firsthead}{font=\bfseries}
\SetTblrStyle{firstfoot}{fg=red2}
\SetTblrOuter[mytblr]{theme=mytabletheme}
\SetTblrInner[mytblr]{
rowspec={t{7mm}},columns = {c},
width = 0.85\linewidth,
row{odd} = {bg=red9,fg=black,ht=8mm},
row{even} = {bg=red7,fg=black,ht=8mm},
hlines={white},vlines={white},
row{1} = {bg=red4, fg=white, font=\bfseries\fontfamily{maksf}},rowhead = 1,
hline{2} = {.7mm}, % midrule  
}
\newcounter{askhsh}
\setcounter{askhsh}{1}
\newcommand{\askhsh}{{\large\theaskhsh.}\ \addtocounter{askhsh}{1}}

\titleformat{\section}{\Large}{\kerkissans{\thesection}}{10pt}{\Large\kerkissans{#1}}

\setlength{\columnsep}{5mm}
\titleformat{\paragraph}
{\large}%
{}{0em}%
{\textcolor{red!80!black}{\faSquare\ \ \kerkissans{\bmath{#1}}}}
\setlength{\parindent}{0pt}

\newcommand{\eng}[1]{\selectlanguage{english}#1\selectlanguage{greek}}
\newcommand{\tss}[1]{\textsuperscript{#1}}

\begin{document}
\twocolumn[{
\centering
\kerkissans{{\huge Η έννοια του πολυωνύμου}\\\vspace{3mm} {\Large ΑΣΚΗΣΕΙΣ}}\vspace{5mm}}]
\paragraph{Βασικές έννοιες}
\askhsh Για καθένα από τα παρακάτω πολυώνυμα να βρείτε τους όρους, τους συντελεστές, το σταθερό όρο και το βαθμό.
\begin{alist}
\item $P(x)=2x^3-4x^2+5x-7$
\item $Q(x)=x^2+5x-2$
\item $S(x)=-x^4+x^3-5x+8$
\item $K(x)=x^2-x^3+1$
\item $M(x)=\dfrac{x^3}{4}-\sqrt{2}x^2+\dfrac{3x}{5}$
\end{alist}
\askhsh Βρείτε το βαθμό σε καθένα από τα παρακάτω πολυώνυμα.
\begin{alist}
\item $P(x)=3x-x^3+5+x^2$
\item $Q(x)=4x^2+0x^4-x^3+7$
\item $R(x)=x^4-2x^2+4x^3-x^4+7-5x$
\item $S(x)=0x^2+0x+3$
\item $K(x)=-5$
\item $G(x)=0$
\end{alist}
\askhsh Δίνεται το πολυώνυμο
\[ P(x)=\left(\lambda^2-3\lambda\right)x^3-(\lambda-3)x^2+\left(9-\lambda^2\right)x+2\lambda-6 \]
με $\lambda\in\mathbb{R}$. Να βρείτε το βαθμό του $P(x)$ για κάθε τιμή της παραμέτρου $\lambda$.\\\\
\askhsh Δίνεται το πολυώνυμο:
\[ P(x)=(a^2-a)x^3+(a^2-a-2)x^2+(a^2-1)x+|a|-1 \]
Βρείτε τις τιμές τις παραμέτρου $a\in\mathbb{R}$ έτσι ώστε το $P(x)$:
\begin{alist}
\item να είναι 3\tss{ου} βαθμού.
\item να είναι 2\tss{ου} βαθμού.
\item να είναι 1\tss{ου} βαθμού.
\item να είναι μηδενικού βαθμού.
\item  να μην έχει βαθμό.
\end{alist}
\askhsh Δίνεται το πολυώνυμο 1\tss{ου} βαθμού
\[ P(x)=(a-\beta-2)x^3+(2a+3\beta)x^2+2ax-5\beta \]
με $a,\beta\in\mathbb{R}$. 
\begin{alist}
\item Να δείξετε ότι $a=4$ και $\beta=2$.
\item Να βρεθεί ο βαθμός του πολυωνύμου \[Q(x)=P^2(x)-xP(x)\]
\end{alist}
\paragraph{Ισότητα πολυωνύμων}
\askhsh Βρείτε τις τιμές της παραμέτρου $a\in\mathbb{R}$ έτσι ώστε το πολυώνυμο
\[ P(x)=(|a|-1)x^3+(a^2-a)x^2+(2a-2)x+a^2-1 \]
να είναι το μηδενικό πολυώνυμο.\\\\
\askhsh Να υπολογίσετε τις τιμές της παραμέτρου $\lambda\in\mathbb{R}$ έτσι ώστε τα πολυώνυμα
\begin{align*}
A(x)&=(\lambda+1)x^3+(\lambda^2+2)x^2+2\lambda x-3\ \text{και}\\
B(x)&=\left(\lambda^2-1\right)x^3+3\lambda x^2 +4x-1-\lambda
\end{align*}
να είναι ίσα.\\\\
\askhsh Να βρεθούν οι τιμές της παραμέτρου $ a $ ώστε τα παρακάτω πολυώνυμα να είναι ίσα.
\begin{align*}
P(x)&=(a^2-3a)x^3+x^2+a
\;\;\textrm{ και }\\Q(x)&=-2x^3+a^2x^2+(a^3-1) x + 1
\end{align*}
\paragraph{Τιμές - Ρίζες πολυωνύμων}
\askhsh Δίνεται το πολυώνυμο $P(x)=x^3-3x^2+7x-10$.
\begin{alist}
\item Να υπολογίσετε τις τιμές $P(-2),P(1),P(0)$ και $P(3)$.
\item Να βρεθεί η τιμή της παράστασης $3P^2(2)-4P(-1)+P(0)$.
\end{alist}
\askhsh Δίνεται το πολυώνυμο $P(x)=x^3-4x^2+x+6$. Να εξετάσετε ποιοι από τους αριθμούς $\pm1,\pm2,\pm3$ είναι ρίζες του $P(x)$.\\\\
\askhsh Δίνεται το πολυώνυμο \[P(x)=x^3+ax^2-(2a-1)x-3, \text{με }a\in\mathbb{R},\] για το οποίο ισχύει $P(2)=7$.
\begin{alist}
\item Να δείξετε ότι $a=3$.
\item Να γράψετε τους όρους και τους συντελεστές του $P(x)$.
\item Να υπολογίσετε τις τιμές $P(3),P(-1),P(0)$ και $P(-4)$.
\end{alist}
\askhsh Δίνεται το πολυώνυμο 
\[ P(x)=x^3-(3-a)x^2+a^2x-4 \]
με $a\in\mathbb{R}$, το οποίο έχει ρίζα τον αριθμό $1$.
\begin{alist}
\item Να δείξετε ότι $a=2$.
\item Να βρεθούν οι τιμές $P(-1),P(2)$ και $P(0)$.
\end{alist}
\askhsh Δίνεται το πολυώνυμο 3\tss{ου} βαθμού
\[ P(x)=(a^2-1)x^3+\left(a^2-3a+2\right)x^2+(a+2)x-8 \]
με $a\in\mathbb{R}$, για το οποίο ισχύει $P(1)=-1$.
\begin{alist}
\item  Να δείξετε ότι $a=2$.
\item  Να γράψετε τους όρους και τους συντελεστές του $P(x)$.
\item Να βρεθεί η τιμή της παράστασης \[\dfrac{2P(2)-P^2(-1)}{P(0)}\]
\end{alist}
\askhsh Δίνεται το πολυώνυμο \[P(x)=ax^3+\beta x^2+2x+5\]
με $a,\beta\in\mathbb{R}$, για το οποίο ισχύει $P(2)=-1$ και $P(-1)=2$.
\begin{alist}
\item Να δείξετε ότι $a=1$ και $\beta=-2$.
\item Να βρεθεί η τιμή της παράστασης $ P(P(1)) $.
\end{alist}
\askhsh Δίνεται το πολυώνυμο
\[ P(x)=(x-2)^{8}+3(x-1)^7-4(2x-1)^6-(2-x)^5+1 \]
Να βρεθεί για το $P(x)$:
\begin{alist}
\item το άθροισμα των συντελεστών του.
\item ο σταθερός όρος του.
\end{alist}
\askhsh Δίνεται πολυώνυμο $P(x)$ για το οποίο ισχύει
\[ P(2x-3)=3x^3-5x^2+8x-6 \]
\begin{alist}
\item Να βρεθούν οι τιμές $P(3),P(-1)$ και $P(0)$.
\item Να αποδείξετε ότι το $ 1 $ είναι ρίζα του $P(x)$.
\end{alist}
\paragraph{Πράξεις πολυωνύμων}
\askhsh Δίνονται τα ακόλουθα πολυώνυμα 
\begin{align*}
&A(x)=x^3-2x^2-5x+4,\\&B(x)=-x^3+3x^2+8x-10 \text{ και} \\&\varGamma(x)=3x-4
\end{align*}
\begin{alist}
\item Να υπολογίσετε τα πολυώνυμα
\begin{rlist}
\item $P(x)=A(x)+B(x)$
\item $Q(x)=B(x)-A(x)$
\item $R(x)=\varGamma(x)\cdot A(x)$
\end{rlist}
\item Ποιος είναι ο βαθμός αυτών των πολυωνύμων?
\end{alist}
\askhsh Έστω $P(x)$ ένα πολυώνυμο τέτοιο ώστε 
\[ (x-2)\cdot P(x)=x^3-2x^2+x-1 \]
\begin{alist}
\item Να προσδιορίσετε το πολυώνυμο $P(x)$.
\item Υπολογίστε την τιμή της παράστασης $2P(-1)+P^2(0)$.
\end{alist}
\askhsh Δίνονται τα πολυώνυμα $A(x)=x^2-3x+2$ και $B(x)=x^3-x^2+4x-3$.
Να βρεθούν τα πολυώνυμα
\begin{multicols}{2}
\begin{alist}
\item $P^2(x)$
\item $P(x)\cdot Q(x)$
\item $P^2(x)-xQ(x)$
\item $xP(x)-Q(x)$
\end{alist}
\end{multicols}
\askhsh Δίνεται το πολυώνυμο \[ P(x)=x^4-2x^3-3x^2+4x+4 \] Να βρεθεί πολυώνυμο $ Q(x) $ έτσι ώστε να ισχύει $ Q^2(x)=P(x) $.\\\\
\askhsh Δίνεται το πολυώνυμο \[P(x)=x^3+ax^2+\beta x+3\]
με $a,\beta\in\mathbb{R}$, για το οποίο ισχύει $P(-2)=7$ και $P(1)=10$.
\begin{alist}
\item Να δείξετε ότι $a=4$ και $\beta=2$.
\item Να βρεθεί το πολυώνυμο $ (x-1)P(x) $.
\end{alist}
\askhsh Δίνεται πολυώνυμο $P(x)$ για το οποίο ισχύει:
\[ P(x+3)=x^3+x^2+3x-5 \]
\begin{alist}
\item Να δείξετε ότι $P(2)=-8$.
\item Να βρεθεί το $P(x)$.
\end{alist}
\askhsh Δίνονται πολυώνυμα $P(x),Q(x)$ για τα οποία ισχύει η σχέση:
\[ P(x)\cdot Q(x)=x^3-x^2-2x-8 \]
Να βρεθούν τα $P(x)$ και $Q(x)$.
\paragraph{Τράπεζα Θεμάτων - ΙΕΠ}
\askhsh \textbf{Θέμα 2 - 21998}\\
Δίνεται το πολυώνυμο 
\[ P(x) = (x-2)\cdot (x^6+1) \]
\begin{alist}
\item Ποιος είναι ο βαθμός του πολυωνύμου $P(x)$? Να αιτιολογήσετε την απάντησή σας.
\item Να βρείτε όλες τις ρίζες του πολυωνύμου $P(x)$.
\end{alist}
\askhsh \textbf{Θέμα 2 - 20640}\\
Δίνεται το πολυώνυμο
\[ 2x^3-8x^2+7x-1 \]
\begin{alist}
\item Να αποδείξετε ότι έχει ρίζα τον αριθμό 1.
\item Έστω $Q(x)$ πολυώνυμο το οποίο δεν έχει ρίζα τον αριθμό 1.
\begin{rlist}
\item Να αποδείξετε ότι το πολυώνυμο
\[R_1(x)=P(x)+Q(x)\] 
δεν έχει ρίζα τον αριθμό 1.
\item Να αποδείξετε ότι το πολυώνυμο
\[ R_2(x)=P(x)\cdot Q(x) \]
έχει ρίζα τον αριθμό 1.
\end{rlist}
\end{alist}
\askhsh \textbf{Θέμα 2 - 15113}\\
Δίνονται τα πολυώνυμα:
\[ P(x)=-2x^3+4x^2+2(x^3-1)+9\ \text{και}\ Q(x)=ax^2+7,\ a\in\mathbb{R} \]
\begin{alist}
\item Είναι το πολυώνυμο $P(x)$ 3\tss{ου} βαθμού? Να αιτιολογήσετε την απάντησή σας.
\item Να βρείτε την τιμή του $a$, ώστε τα πολυώνυμα $P(x)$ και $Q(x)$ να είναι ίσα. 
\end{alist}
\paragraph{Ερωτήσεις θεωρίας}
\askhsh
\end{document}
