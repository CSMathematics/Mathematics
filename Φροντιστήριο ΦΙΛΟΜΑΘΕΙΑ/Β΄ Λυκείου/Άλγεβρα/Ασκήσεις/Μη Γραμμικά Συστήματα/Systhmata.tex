\PassOptionsToPackage{no-math,cm-default}{fontspec}
\documentclass[twoside,nofonts,internet]{askhseis}
\usepackage{amsmath}
\usepackage{xgreek}
\let\hbar\relax
\defaultfontfeatures{Mapping=tex-text,Scale=MatchLowercase}
\setmainfont[Mapping=tex-text,Numbers=Lining,Scale=1.0,BoldFont={Minion Pro Bold}]{Minion Pro}
\newfontfamily\scfont{GFS Artemisia}
\font\icon = "Webdings"
\usepackage[amsbb]{mtpro2}
\usepackage{multicol}
\usepackage{hhline}
\usepackage{pgfplots}
\xroma{cyan!50!green}
\usepackage{tikz}
\usepackage{tkz-euclide}
\usepackage{wrapfig}
\usetkzobj{all}
\usepackage{calc}
\usepackage[framemethod=TikZ]{mdframed}
\newcommand{\ypogrammisi}[1]{\underline{\smash{#1}}}
\usepackage{wrap-rl}
\renewcommand{\thepart}{\arabic{part}}
\definecolor{steelblue}{cmyk}{.7,.278,0,.294}
\definecolor{royalblue}{RGB}{0,0,0}
\definecolor{doc}{cmyk}{1,0.455,0,0.569}
\definecolor{olivedrab}{cmyk}{0.25,0,0.75,0.44}

\usepackage{titletoc}
\usepackage[explicit]{titlesec}
\usepackage{graphicx}
\usepackage{multicol}
\usepackage{multirow}
\usepackage{enumitem}
\usepackage{tabularx}
\usepackage[decimalsymbol=comma]{siunitx}
\usepackage{eurosym}

\usepackage{sectsty}
\sectionfont{\centering}

\usepackage{systeme,regexpatch}
\makeatletter
% change the definition of \sysdelim not to store `\left` and `\right`
\def\sysdelim#1#2{\def\SYS@delim@left{#1}\def\SYS@delim@right{#2}}
\sysdelim\{. % reinitialize

% patch the internal command to use
% \LEFTRIGHT<left delim><right delim>{<system>}
% instead of \left<left delim<system>\right<right delim>
\regexpatchcmd\SYS@systeme@iii
  {\cB.\c{SYS@delim@left}(.*)\c{SYS@delim@right}\cE.}
  {\c{SYS@MT@LEFTRIGHT}\cB\{\1\cE\}}
  {}{}
\def\SYS@MT@LEFTRIGHT{%
  \expandafter\expandafter\expandafter\LEFTRIGHT
  \expandafter\SYS@delim@left\SYS@delim@right}
\makeatother
\newcommand{\synt}[2]{{\scriptsize \begin{matrix}
\times#1\\\\ \times#2
\end{matrix}}}

\begin{document}
\titlos{ΑΛΓΕΒΡΑ Β΄ ΛΥΚΕΙΟΥ}{Συστήματα}{ΜΗ ΓΡΑΜΜΙΚΑ ΣΥΣΤΗΜΑΤΑ}
\newpage
\noindent
\askhseis
\begin{enumerate}[label=\bf\textcolor{\xrwma}{{\large \arabic*.}},
itemsep=5mm]
\item \textbf{Μη γραμμικά συστήματα}\\
Να λυθούν τα παρακάτω μη γραμμικά συστήματα συστήματα
\begin{multicols}{3}
\begin{enumerate}[label=\roman*.,itemsep=1mm]
\item $\ccases{
\;x^2+y^2=1\\
\;x+y=0}$
\item $\ccases{
\;x^2-y^2=1\\
\;2x-y=0}$
\item $\ccases{
\;x^2+y^2=4\\
\;xy=1}$
\item $\ccases{
\;x^2-y^2=16\\
\;\dfrac{1}{x}+\dfrac{1}{y}=1}$
\item $\ccases{
\;x^2+2y=4\\
\;x-y=-5}$
\item $\ccases{
\;x+y^2=2y-1\\
\;x^2+y=1}$
\end{enumerate}\end{multicols}
\item \textbf{Μη γραμμικά συστήματα}\\
Να λυθούν τα παρακάτω μη γραμμικά συστήματα συστήματα
\begin{multicols}{3}
\begin{enumerate}[label=\roman*.,itemsep=1mm]
\item $\ccases{
\;x^3+y^3=1\\
\;x+y=1}$
\item $\ccases{
\;(x-y)^2+(x+y)^2=13\\
\;(x-y)-2(x+y)=-4}$
\item $\ccases{
\;2x-3y=2\\
\;|x-y|=1}$
\item $\ccases{
\;2\sqrt{x}+\sqrt{y}=5\\
\;x+4y=6}$
\item $\ccases{
\;\dfrac{1}{x-y^2}+\dfrac{1}{y-x^2}=2\\
\;x^2+y^2-x-y=4}$
\item $\ccases{
\;\eta\mu^2x-2\sigma\upsilon\nu^2y=-\dfrac{1}{2}\\
\;2\eta\mu x\cdot\sigma\upsilon\nu y=1}$
\end{enumerate}\end{multicols}
\item \textbf{Μη γραμμικά συστήματα}\\
Να λυθούν τα παρακάτω μη γραμμικά συστήματα συστήματα
\begin{multicols}{3}
\begin{enumerate}[label=\roman*.,itemsep=3mm]
\item $\ccases{
\;x^2+y^2+z^2=1\\
\;x+y+z=0\\
\;x-y+z=0}$
\item $\ccases{
\;x^2-y^2+z^2=1\\
\;2x-y+z=1\\
\;y+z=1}$
\item $\ccases{
\;\dfrac{1}{x}+\dfrac{1}{y}=1\\
\;\dfrac{1}{y}+\dfrac{1}{z}=1\\
\;\dfrac{1}{x}+\dfrac{1}{z}=1}$
\item $\ccases{
\;xyz=4\\
\;x-y=1\\
\;xy-z=0}$
\item $\ccases{
\;x+y+z+\dfrac{1}{x+y+z}=2\\
\;x-y+z=1\\
\;x+y=2}$
\item $\ccases{
\;\dfrac{1}{x}+\dfrac{1}{y}+\dfrac{1}{z}=1\\
\;x+y+z=1\\
\;y-z=1}$
\end{enumerate}\end{multicols}
\item \textbf{Μη γραμμικά συστήματα}\\
Να λυθούν τα παρακάτω μη γραμμικά συστήματα συστήματα
\begin{multicols}{3}
\begin{enumerate}[label=\roman*.,itemsep=3mm]
\item $\ccases{
\;x^3+y^3=3z^3\\
\;x^2y-xy^2=2z^3\\
\;x+2y+z=1}$
\item $\ccases{
\;x^2-y^2+z^2=1\\
\;\sqrt{x}+\sqrt{y}+\sqrt{z}=0\\
\;x+2y+z=1}$
\item $\ccases{
\;y(x^2-y^2)=z\\
\;-y(x^2+y^2)=z\\
\;(1-x)^2+(y-z-1)^2=0}$
\item $\ccases{
\;xy=2\\
\;yz=3\\
\;xz=8}$
\item $\ccases{
\;x^3+y^3=4\\
\;y^3+z^3=2\\
\;x^3+z^3=4}$
\item $\ccases{
\;x+y^2=2y-1\\
\;x^2+y=1}$
\end{enumerate}\end{multicols}
\newpage
\noindent
\begin{center}
\textcolor{royalblue}{\textbf{ΠΡΟΒΛΗΜΑΤΑ ΜΗ ΓΡΑΜΜΙΚΩΝ ΣΥΣΤΗΜΑΤΩΝ}}
\end{center}
\item Ένα ορθογώνιο παραλληλόγραμμο έχει περίμετρο 24 μέτρα και εμβαδόν 32 τ.μέτρα. Να βρεθει το μήκος και το πλάτος του ορθογωνίου.
\item Να βρεθούν τα κοινά σημεία
\begin{enumerate}[label=\roman*.,itemsep=0mm]
\item του κύκλου $ x^2+y^2=1 $ και της ευθείας $ 2x+3y=1 $
\item του κύκλου $ x^2+y^2=4 $ και της παραβολής $ 3x-y^2=0 $
\item της υπερβολής $ xy=4 $ και της ευθείας $ 3x-y=-1 $
\item των κύκλων $ x^2+y^2=12 $ και $ x^2+y^2=9 $
\end{enumerate}
\wrapr{-15mm}{7}{4cm}{0mm}{\begin{tikzpicture}
   \tkzDefPoint(3,0){B}
   \tkzDefPoint(0,0){A}
   \tkzDefPoint(0,1){C}
   \tkzDefPoint(0.33,.9){D}
   \tkzMarkRightAngle(B,A,C)
   \tkzMarkRightAngle[size=.2](A,D,B)
   \tkzDrawPolygon(A,B,C)
   \tkzDrawSegment(A,D)
   \tkzLabelPoint[left](A){{\scriptsize A}}
   \tkzLabelPoint[right](B){{\scriptsize B}}
   \tkzLabelPoint[above](C){{\scriptsize $ \varGamma $}}
   \tkzLabelPoint[above right](D){{\scriptsize $ \varDelta $}}
   \tkzText(-.2,.5){{\scriptsize $ 15 $}}
   \tkzText(1.5,-.2){{\scriptsize $ 20 $}}
   \end{tikzpicture}}{\item 
Αν $ A\varDelta=12 $ είναι το ύψος του ορθογωνίου τριγώνου στην υποτεινουσα και $ B\varDelta=x, \varGamma\varDelta=y $ να βρεθούν οι θετικοί πραγματικοί οι αριθμοί $ x, y\in\mathbb{R^+} $ ώστε να ισχύει $ A\varDelta^2=B\varDelta\cdot\varGamma\varDelta $.
}
\item Δύο τετράγωνα με πλευρές $ x-2 $ και $ 2y+3 $ αντίστοιχα έχουν συνολικό εμβαδόν 90τ.μέτρα. Αν ξέρουμε οτι η περίμετρος του $ 2\textsuperscript{ου} $ είναι 3πλάσια από την περίμετρο του $ 1\textsuperscript{ου} $ τότε να βρεθούν οι αριθμοί $ x, y\in\mathbb{R} $.
\item Δύο αυτοκίνητα Α και Β κινούνται με μέση συνολική ταχύτητα $ 150km/h $. Αν γνωρίζουμε οτι κάθε αυτοκίνητο διένυσε απόσταση $ 56km/h $ και ταξίδευαν συνολικά για 1μιση ώρα τότε να βρεθεί ο χρόνος που ταξίδευε το κάθε αμάξι.
\item Η διαγώνιος μιας τηλεόρασης είναι $ 42'' $ και γνώρίζουμε επίσης ότι οι πλευρές έχουν αναλογία 16:9 (HD Video Standard). Να βρεθούν οι διαστάσεις της τηλεόρασης.\\
{\footnotesize \textit{Υπόδειξη} : Το μέγεθος μιας τηλεόρασης δίνεται από το μήκος της διαγωνίου της οθόνης δοσμένο σε ίντες.\\Μια ίντσα $ 1''=2{,}54cm $}
\end{enumerate}
\end{document}