\documentclass[11pt,a4paper,twocolumn]{article}
\usepackage[english,greek]{babel}
\usepackage[utf8]{inputenc}
\usepackage{nimbusserif}
\usepackage[T1]{fontenc}
\usepackage[left=1.50cm, right=1.50cm, top=2.00cm, bottom=2.00cm]{geometry}
\usepackage{amsmath}
\let\myBbbk\Bbbk
\let\Bbbk\relax
\usepackage[amsbb,subscriptcorrection,zswash,mtpcal,mtphrb,mtpfrak]{mtpro2}
\usepackage{graphicx,multicol,multirow,enumitem,tabularx,mathimatika,gensymb,venndiagram,hhline,longtable,tkz-euclide,fontawesome5,eurosym,tcolorbox,tabularray}
\usepackage[explicit]{titlesec}
\tcbuselibrary{skins,theorems,breakable}
\newlist{rlist}{enumerate}{3}
\setlist[rlist]{itemsep=0mm,label=\roman*.}
\newlist{alist}{enumerate}{3}
\setlist[alist]{itemsep=0mm,label=\alph*.}
\newlist{balist}{enumerate}{3}
\setlist[balist]{itemsep=0mm,label=\bf\alph*.}
\newlist{Alist}{enumerate}{3}
\setlist[Alist]{itemsep=0mm,label=\Alph*.}
\newlist{bAlist}{enumerate}{3}
\setlist[bAlist]{itemsep=0mm,label=\bf\Alph*.}
\newlist{askhseis}{enumerate}{3}
\setlist[askhseis]{label={\Large\thesection}.\arabic*.}
\renewcommand{\textstigma}{\textsigma\texttau}
\newlist{thema}{enumerate}{3}
\setlist[thema]{label=\bf\large{ΘΕΜΑ \textcolor{black}{\Alph*}},itemsep=0mm,leftmargin=0cm,itemindent=18mm}
\newlist{erwthma}{enumerate}{3}
\setlist[erwthma]{label=\bf{\large{\textcolor{black}{\Alph{themai}.\arabic*}}},itemsep=0mm,leftmargin=0.8cm}

\newcommand{\kerkissans}[1]{{\fontfamily{maksf}\selectfont \textbf{#1}}}
\renewcommand{\textdexiakeraia}{}

\usepackage[
backend=biber,
style=alphabetic,
sorting=ynt
]{biblatex}

\DeclareTblrTemplate{caption}{nocaptemplate}{}
\DeclareTblrTemplate{capcont}{nocaptemplate}{}
\DeclareTblrTemplate{contfoot}{nocaptemplate}{}
\NewTblrTheme{mytabletheme}{
\SetTblrTemplate{caption}{nocaptemplate}{}
\SetTblrTemplate{capcont}{nocaptemplate}{}
\SetTblrTemplate{contfoot}{nocaptemplate}{}
}

\NewTblrEnviron{mytblr}
\SetTblrStyle{firsthead}{font=\bfseries}
\SetTblrStyle{firstfoot}{fg=red2}
\SetTblrOuter[mytblr]{theme=mytabletheme}
\SetTblrInner[mytblr]{
rowspec={t{7mm}},columns = {c},
width = 0.85\linewidth,
row{odd} = {bg=red9,fg=black,ht=8mm},
row{even} = {bg=red7,fg=black,ht=8mm},
hlines={white},vlines={white},
row{1} = {bg=red4, fg=white, font=\bfseries\fontfamily{maksf}},rowhead = 1,
hline{2} = {.7mm}, % midrule  
}
\newcounter{askhsh}
\setcounter{askhsh}{1}
\newcommand{\askhsh}{\large\theaskhsh.\ \addtocounter{askhsh}{1}}

\titleformat{\section}{\Large}{\kerkissans{\thesection}}{10pt}{\Large\kerkissans{#1}}

\setlength{\columnsep}{5mm}
\titleformat{\paragraph}
{\large}%
{}{0em}%
{\textcolor{red!80!black}{\faSquare\ \ \kerkissans{\bmath{#1}}}}
\setlength{\parindent}{0pt}

\newcommand{\eng}[1]{\selectlanguage{english}#1\selectlanguage{greek}}

\begin{document}
\twocolumn[{
\centering
\kerkissans{{\huge Ιδιότητες συναρτήσεων}\\\vspace{3mm} {\Large ΑΣΚΗΣΕΙΣ}}\vspace{5mm}}]
\paragraph{Μονοτονία}
\askhsh Να μελετήσετε τις ακόλουθες συναρτήσεις ως προς τη μονοτονία.
\begin{multicols}{2}
\begin{alist}
\item $f(x)=\sqrt{x-1}+3$
\item $f(x)=x^3+4$
\item $f(x)=2+\sqrt{3-x}$
\item $f(x)=1+\dfrac{2}{\sqrt{x}}$
\end{alist}
\end{multicols}
\askhsh Να μελετήσετε τις ακόλουθες συναρτήσεις ως προς τη μονοτονία.
\begin{alist}
\item $f(x)=3x^3+5x+1$
\item $f(x)=3x+\sqrt{x+2}$
\item $f(x)=\sqrt{2-x}-x^5$
\item $f(x)=\dfrac{3}{x}-\sqrt{x-2}$
\end{alist}
\paragraph{Ακρότατα}
\askhsh Να μελετήσετε τις παρακάτω συναρτήσεις ως προς τα ακρότατα.
\begin{alist}
\item $f(x)=(x-2)^2-3$
\item $f(x)=4-(3-x)^2$
\item $f(x)=3x^4-12$
\item $f(x)=4-2(x+1)^2$
\end{alist}
\askhsh Να μελετήσετε τις παρακάτω συναρτήσεις ως προς τα ακρότατα.
\begin{alist}
\item $f(x)=|x+2|-1$
\item $f(x)=2-|x-3|$
\item $f(x)=4|x-1|+3$
\item $f(x)=-2-3|2x-4|$
\end{alist}
\askhsh Να μελετήσετε τις παρακάτω συναρτήσεις ως προς τα ακρότατα.
\begin{alist}
\item $f(x)=\sqrt{x+1}-5$
\item $f(x)=4-\sqrt{2-x}$
\item $f(x)=2\sqrt{3x+1}-4$
\item $f(x)=9-\sqrt{1-x}$
\end{alist}
\end{document}
