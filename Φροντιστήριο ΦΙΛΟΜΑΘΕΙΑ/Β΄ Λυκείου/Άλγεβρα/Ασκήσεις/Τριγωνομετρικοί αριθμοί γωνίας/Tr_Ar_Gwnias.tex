\PassOptionsToPackage{no-math,cm-default}{fontspec}
\documentclass[twoside,nofonts,internet,twocolumn]{askhseis}
\usepackage{amsmath}
\usepackage{xgreek}
\let\hbar\relax
\defaultfontfeatures{Mapping=tex-text,Scale=MatchLowercase}
\setmainfont[Mapping=tex-text,Numbers=Lining,Scale=1.0,BoldFont={Minion Pro Bold}]{Minion Pro}
\newfontfamily\scfont{GFS Artemisia}
\font\icon = "Webdings"
\usepackage[amsbb]{mtpro2}
\usepackage{tikz,pgfplots}
\tkzSetUpPoint[size=7,fill=white]
\xroma{red!70!black}
%------TIKZ - ΣΧΗΜΑΤΑ - ΓΡΑΦΙΚΕΣ ΠΑΡΑΣΤΑΣΕΙΣ ----
\usepackage{tikz}
\usepackage{tkz-euclide}
\usetkzobj{all}
\usepackage[framemethod=TikZ]{mdframed}
\usetikzlibrary{decorations.pathreplacing}
\usepackage{pgfplots}
\usetkzobj{all}
%-----------------------


\usepackage{calc}
\usepackage{hhline}
\usepackage{float}
\usepackage{subfig}
\usepackage[explicit]{titlesec}
\usepackage{graphicx}
\usepackage{multicol}
\usepackage{multirow}
\usepackage{enumitem}
\usepackage{tabularx}
\usepackage[decimalsymbol=comma]{siunitx}
\usetikzlibrary{backgrounds}
\usepackage{sectsty}
\sectionfont{\centering}
\usepackage{enumitem}
\setlist[enumerate]{label=\bf{\large \arabic*.}}
\usepackage{adjustbox,mathimatika,gensymb,wrap-rl}


%-------- ΜΑΘΗΜΑΤΙΚΑ ΕΡΓΑΛΕΙΑ ---------
\usepackage{mathtools}
%----------------------

%-------- ΠΙΝΑΚΕΣ ---------
\usepackage{booktabs}
%----------------------
%----- ΥΠΟΛΟΓΙΣΤΗΣ ----------
\usepackage{calculator}
%----------------------------
%------ ΔΙΑΓΩΝΙΟ ΣΕ ΠΙΝΑΚΑ -------
\usepackage{array}
\newcommand\diag[5]{%
\multicolumn{1}{|m{#2}|}{\hskip-\tabcolsep
$\vcenter{\begin{tikzpicture}[baseline=0,anchor=south west,outer sep=0]
\path[use as bounding box] (0,0) rectangle (#2+2\tabcolsep,\baselineskip);
\node[minimum width={#2+2\tabcolsep-\pgflinewidth},
minimum  height=\baselineskip+#3-\pgflinewidth] (box) {};
\draw[line cap=round] (box.north west) -- (box.south east);
\node[anchor=south west,align=left,inner sep=#1] at (box.south west) {#4};
\node[anchor=north east,align=right,inner sep=#1] at (box.north east) {#5};
\end{tikzpicture}}\rule{0pt}{.71\baselineskip+#3-\pgflinewidth}$\hskip-\tabcolsep}}
%---------------------------------

%---- ΟΡΙΖΟΝΤΙΟ - ΚΑΤΑΚΟΡΥΦΟ - ΠΛΑΓΙΟ ΑΓΚΙΣΤΡΟ ------
\newcommand{\orag}[3]{\node at (#1)
{$ \overcbrace{\rule{#2mm}{0mm}}^{{\scriptsize #3}} $};}

\newcommand{\kag}[3]{\node at (#1)
{$ \undercbrace{\rule{#2mm}{0mm}}_{{\scriptsize #3}} $};}

\newcommand{\Pag}[4]{\node[rotate=#1] at (#2)
{$ \overcbrace{\rule{#3mm}{0mm}}^{{\rotatebox{-#1}{\scriptsize$#4$}}}$};}
%-----------------------------------------

%------ Tikz - Item ----------------
\newcommand{\tikzitem}{\leavevmode\vadjust{\vspace{-\baselineskip}}\newline}
%\begin{tikzpicture}[level/.style={sibling distance=50mm/#1},baseline]
%-----------------------------------




\newcommand{\tss}[1]{\textsuperscript{#1}}
\newcommand{\tssL}[1]{\MakeLowercase{\textsuperscript{#1}}}
%---------- ΛΙΣΤΕΣ ----------------------
\newlist{bhma}{enumerate}{3}
\setlist[bhma]{label=\bf\textit{\arabic*\textsuperscript{o}\;Βήμα :},leftmargin=0cm,itemindent=1.8cm,ref=\bf{\arabic*\textsuperscript{o}\;Βήμα}}
\newlist{tropos}{enumerate}{3}
\setlist[tropos]{label=\bf\textit{\arabic*\textsuperscript{oς}\;Τρόπος :},leftmargin=0cm,itemindent=2.3cm,ref=\bf{\arabic*\textsuperscript{oς}\;Τρόπος}}
% Αν μπει το bhma μεσα σε tropo τότε
%\begin{bhma}[leftmargin=.7cm]
\tkzSetUpPoint[size=7,fill=white]
\tikzstyle{pl}=[line width=0.3mm]
\tikzstyle{plm}=[line width=0.4mm]
\usepackage{amsmath}



\newcommand{\roloi}[4][]{
\draw[line width=.5mm,#1] (0,0) circle(2);
\foreach \n in {1,2,...,12}{
\tkzDefPoint(30*\n-90:2){A_\n}
%\tkzDrawPoint(A_\n)
\node at (-30*\n+90:1.65){\n};}
\draw[plm,,#1] (0,0)--(90-30*#2-0.5*#3:1);
\draw[pl,#1] (0,0)--(90-6*#3-0.1*#4:1.5);
\draw[#1](0,0)--(90-6*#4:1.2);
\tkzDrawPoint[fill=#1,color=#1](0,0)
\foreach \s in {1,2,...,12}{
\draw[#1](90-30*\s:1.85)--(90-30*\s:2);}
\foreach \t in {1,2,...,60}{
\draw[#1](90-6*\t:1.93)--(90-6*\t:2);}}


\begin{document}
\twocolkentro{
\titlos{Άλγεβρα Β΄ Λυκείου}{Τριγωνομετρία}{Τριγωνομετρικοί Αριθμοί}
\thewria}
\begin{enumerate}
\item 
\end{enumerate}
\twocolkentro{\askhseis}
\begin{enumerate}
\item \textbf{Μετατροπή μοιρών σε ακτίνια}\\
Οι παρακάτω γωνίες οι οποίες είναι δοσμένες σε μοίρες να εκφραστούν σε ακτίνια (rad).
\begin{multicols}{4}
\begin{rlist}[leftmargin=4mm]
\item $ 30\degree $
\item $ 60\degree $
\item $ 45\degree $
\item $ 120\degree $
\item $ 150\degree $
\item $ 300\degree $
\item $ 270\degree $
\item $ 240\degree $
\item $ 330\degree $
\item $ 400\degree $
\item $ 480\degree $
\item $ 1200\degree $
\end{rlist}
\end{multicols}
\item \textbf{Μετατροπή ακτινίων σε μοίρες}\\
Οι παρακάτω γωνίες οι οποίες είναι δοσμένες σε ακτίνια να εκφραστούν σε μοίρες.
\begin{multicols}{4}
\begin{rlist}[leftmargin=4mm]
\item $ \frac{\pi}{4} $
\item $ \frac{2\pi}{3} $
\item $ \frac{\pi}{6} $
\item $ \frac{3\pi}{4} $
\item $ \frac{2\pi}{5} $
\item $ \pi $
\item $ \frac{3\pi}{2} $
\item $ \frac{4\pi}{5} $
\item $ 24\pi $
\item $ \frac{35\pi}{3} $
\item $ \frac{105\pi}{4} $
\item $ 400\pi $
\end{rlist}
\end{multicols}
\item \textbf{Τριγωνομετρικοί αριθμοί}\\
Να υπολογίσετε τους τριγωνομετρικούς αριθμούς των παρακάτω γωνιών.
\begin{multicols}{3}
\begin{rlist}
\item $ 390\degree $
\item $ 450\degree $
\item $ 780\degree $
\item $ 1260\degree $
\item $ 1125\degree $
\item $ 1845\degree $
\end{rlist}
\end{multicols}
\item \textbf{Τρ. αριθμοί γωνίας σε καρτεσιανό σύστημα}\\
Να υπολογίσετε τους τριγωνομετρικούς αριθμούς της γωνίας $ x\hat{O}M $ η οποία σχηματίζεται μέσα σε ένα ορθοκανονικό σύστημα συντεταγμένων $ xOy $ για καθένα από τα παρακάτω σημεία $ M $.
\begin{multicols}{3}
\begin{rlist}[leftmargin=4mm]
\item $ M(3,4) $
\item $ M(5,12) $
\item $ M(-8,15) $
\item $ M(6,-8) $
\item $ M(-4,-3) $
\item $ M(12,-9) $
\end{rlist}
\end{multicols}
\item \textbf{Τριγωνομετρικοί αριθμοί σε τρίγωνο}\\
Να υπολογίσετε τους τριγωνομετρικούς αριθμούς της γωνίας $ \omega $ σε καθένα από τα παρακάτω ορθογώνια τρίγωνα.
\begin{multicols}{2}
\begin{rlist}
\item\tikzitem \begin{tikzpicture}
\draw (-1,0.5) -- (-1,-1) -- (1,-1) -- cycle;
\draw (0.5955,-0.7061) arc (143.9987:180:0.5);
\node at (0.3,-0.8) {\footnotesize$\omega$};
\node at (-1.2,-0.2) {\footnotesize$3$};
\node at (0.2,-0.2) {\footnotesize$5$};
\node at (-0.1,-1.2) {\footnotesize$  $};
\end{tikzpicture}
\item\tikzitem \begin{tikzpicture}
\draw (-1,0.5) -- (-1,-1) -- (1,-1) -- cycle;
\draw (0.5955,-0.7061) arc (143.9987:180:0.5);
\node at (0.3,-0.8) {\footnotesize$\omega$};
\node at (-0.1,-1.2) {\footnotesize$8$};
\node at (0.2,-0.2) {\footnotesize$10$};
\end{tikzpicture}
\item\tikzitem \begin{tikzpicture}
\draw (-1,0.5) -- (-1,-1)  -- (1,-1) -- cycle;
\draw (0.5955,-0.7061) arc (143.9987:180:0.5);
\node at (0.3,-0.8) {\footnotesize$\omega$};
\node at (-0.1,-1.2) {\footnotesize$12$};
\draw (-1,0.5) -- (-1.9,-1) -- (1,-1);
\draw (-1.6,-1) arc (0:57.9421:0.3);
\node at (-1.3,-0.7) {\scriptsize$60^o$};
\node at (-1.4,-1.2) {\footnotesize$3\sqrt{3}$};
\end{tikzpicture}
\item\tikzitem \begin{tikzpicture}
\draw (-1,0.5) -- (-1,-1)  -- (1,-1) -- cycle;
\draw (0.5955,-0.7061) arc (143.9987:180:0.5);
\node at (-1.2,-0.1) {\footnotesize$\omega$};
\node at (0.2,-0.1) {\footnotesize$20$};
\draw (-1,0.5) node (v1) {} -- (-2,-1) -- (1,-1);
\node at (-1.7,-0.1) {\footnotesize$15$};
\draw (-1,0.1) arc (-90:-123:0.4);
\draw (-1.1,0.33) -- (-0.93,0.2) -- (-0.8,0.37);
\draw  (-1,-0.8) rectangle (-0.8,-1);
\node at (-1.5,-1.2) {\footnotesize$9$};
\end{tikzpicture}
\end{rlist}
\end{multicols}
\item \textbf{Τρ. αριθμοί βασικών γωνιών}\\
Να υπολογίσετε τις παρακάτω αριθμιτικές παραστάσεις.
\begin{multicols}{2}
\begin{rlist}[leftmargin=4mm]
\item $ \hm{30\degree}\cdot\hm{60\degree} $
\item $ \hm^2{40\degree}-2\syn{60\degree} $
\item $ \ef{45\degree}+2\syn^2{30\degree} $
\item $ \syf^2{60\degree}-\hm^2{60\degree} $
\end{rlist}
\end{multicols}
\item Ένα κτήριο ύψους $ h $ δημιουργεί σκιά στο έδαφος μήκους $ 250m $. Αν γνωρίζουμε ότι η γωνία που σχηματίζουν οι ακτίνες του ήλιου με το έδαφος είναι $ 30\degree $ τότε να βρεθεί το ύψος του κτηρίου.
\item Δίνεται ημικύκλιο με διάμετρο $ AB=10cm $ και ένα τυχαίο σημείο $ \varGamma $ του ημικυκλίου. Αν $ M $ είναι το μέσο του τόξου $ \widearc{B\varGamma} $ και $ \varDelta $ το σημείο τομής των ευθειών $ BM $ και $ A\varGamma $ τότε:
\begin{center}
\begin{tikzpicture}
\tkzDefPoint(0:1.5){A}
\tkzDefPoint(180:1.5){B}
\tkzDefPoint(50:1.5){C}
\tkzDefPoint(100:1.5){D}
\draw[pl] (B)--(A) arc (0:180:1.5);
\tkzDrawLine[add= 0 and 1.5](A,C)
\tkzInterLL(B,D)(A,C)\tkzGetPoint{a}
\tkzMarkAngle[size=.5](A,B,C)
\tkzMarkAngle[size=.59](C,B,a)
\draw[pl](C)--(B)--(a);
\tkzDrawPoints(A,B,C,D,a)
\tkzLabelPoint[right](A){$B$}
\tkzLabelPoint[left](B){$A$}
\tkzLabelPoint[above right](C){$M$}
\tkzLabelPoint[above left](D){$\varGamma$}
\tkzLabelPoint[right](a){$\varDelta$}
\node at (-0.8,0.15) {\footnotesize$x$};
\node at (0,-0.25) {\footnotesize$10$};
\end{tikzpicture}
\end{center}
\begin{rlist}
\item να δείξετε ότι $ BM=\varDelta M $ και $ A\varDelta=AB $.
\item να δείξετε ότι $ B\varDelta=20\hm{x} $.
\item αν $ x=45\degree $ να βρείτε τα $ B\varDelta,AM $ και $ A\varGamma $.
\item για ποιά τιμή του $ x $ το τρίγωνο $ AB\varDelta $ γίνεται ισόπλευρο;
\end{rlist}
\item Σε ένα αναλογικό ρολόι τοίχου δείκτης των ωρών έχει μήκος $ 7cm $, ο δείκτης των λεπτών έχει μήκος $ 12cm $ ενώ ο δείκτης των δευτερολέπτων έχει μήκος $ 10cm $.
\begin{center}
\begin{tikzpicture}[scale=.8]
\roloi[black]{3}{35}{55}
\end{tikzpicture}
\end{center}
\begin{rlist}
\item Να υπολογίσετε το μήκος του τόξου που διαγράφει ο δείκτης των δευτερολέπτων σε $ 20 $ δευτερόλεπτα.
\item Να υπολογίσετε το μέτρο και το μήκος του τόξου που διαγράφει ο ωροδείκτης σε $ 2 $ ώρες και $ 10 $ λεπτά.
\item Να υπολογίσετε το μήκος του τόξου που διαγράφει ο λεπτοδείκτης από τις $ 3:30 $ μέχρι τις $ 5:10 $. Ποιο είναι το μέτρο του τόξου αυτού και ποιοι είναι οι τριγωνομετρικοί αριθμοί του;
\end{rlist}
\end{enumerate}
\end{document}
