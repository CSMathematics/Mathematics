\documentclass[11pt,a4paper,twocolumn]{article}
\usepackage[english,greek]{babel}
\usepackage[utf8]{inputenc}
\usepackage{nimbusserif}
\usepackage[T1]{fontenc}
\usepackage[left=1.50cm, right=1.50cm, top=2.00cm, bottom=2.00cm]{geometry}
\usepackage{amsmath}
\let\myBbbk\Bbbk
\let\Bbbk\relax
\usepackage[amsbb,subscriptcorrection,zswash,mtpcal,mtphrb,mtpfrak]{mtpro2}
\usepackage{graphicx,multicol,multirow,enumitem,tabularx,mathimatika,gensymb,venndiagram,hhline,longtable,tkz-euclide,fontawesome5,eurosym,tcolorbox,tabularray}
\usepackage[explicit]{titlesec}
\tcbuselibrary{skins,theorems,breakable}
\newlist{rlist}{enumerate}{3}
\setlist[rlist]{itemsep=0mm,label=\roman*.}
\newlist{alist}{enumerate}{3}
\setlist[alist]{itemsep=0mm,label=\alph*.}
\newlist{balist}{enumerate}{3}
\setlist[balist]{itemsep=0mm,label=\bf\alph*.}
\newlist{Alist}{enumerate}{3}
\setlist[Alist]{itemsep=0mm,label=\Alph*.}
\newlist{bAlist}{enumerate}{3}
\setlist[bAlist]{itemsep=0mm,label=\bf\Alph*.}
\newlist{askhseis}{enumerate}{3}
\setlist[askhseis]{label={\Large\thesection}.\arabic*.}
\renewcommand{\textstigma}{\textsigma\texttau}
\newlist{thema}{enumerate}{3}
\setlist[thema]{label=\bf\large{ΘΕΜΑ \textcolor{black}{\Alph*}},itemsep=0mm,leftmargin=0cm,itemindent=18mm}
\newlist{erwthma}{enumerate}{3}
\setlist[erwthma]{label=\bf{\large{\textcolor{black}{\Alph{themai}.\arabic*}}},itemsep=0mm,leftmargin=0.8cm}

\newcommand{\kerkissans}[1]{{\fontfamily{maksf}\selectfont \textbf{#1}}}
\renewcommand{\textdexiakeraia}{}

\usepackage[
backend=biber,
style=alphabetic,
sorting=ynt
]{biblatex}

\DeclareTblrTemplate{caption}{nocaptemplate}{}
\DeclareTblrTemplate{capcont}{nocaptemplate}{}
\DeclareTblrTemplate{contfoot}{nocaptemplate}{}
\NewTblrTheme{mytabletheme}{
\SetTblrTemplate{caption}{nocaptemplate}{}
\SetTblrTemplate{capcont}{nocaptemplate}{}
\SetTblrTemplate{contfoot}{nocaptemplate}{}
}

\NewTblrEnviron{mytblr}
\SetTblrStyle{firsthead}{font=\bfseries}
\SetTblrStyle{firstfoot}{fg=red2}
\SetTblrOuter[mytblr]{theme=mytabletheme}
\SetTblrInner[mytblr]{
rowspec={t{7mm}},columns = {c},
width = 0.85\linewidth,
row{odd} = {bg=red9,fg=black,ht=8mm},
row{even} = {bg=red7,fg=black,ht=8mm},
hlines={white},vlines={white},
row{1} = {bg=red4, fg=white, font=\bfseries\fontfamily{maksf}},rowhead = 1,
hline{2} = {.7mm}, % midrule  
}
\newcounter{askhsh}
\setcounter{askhsh}{1}
\newcommand{\askhsh}{\large\theaskhsh.\ \addtocounter{askhsh}{1}}

\titleformat{\section}{\Large}{\kerkissans{\thesection}}{10pt}{\Large\kerkissans{#1}}

\setlength{\columnsep}{5mm}
\titleformat{\paragraph}
{\large}%
{}{0em}%
{\textcolor{red!80!black}{\faSquare\ \ \kerkissans{\bmath{#1}}}}
\setlength{\parindent}{0pt}

\newcommand{\eng}[1]{\selectlanguage{english}#1\selectlanguage{greek}}

\begin{document}
\twocolumn[{
\centering
\kerkissans{{\huge Τριγωνομετρικές συναρτήσεις}\\\vspace{3mm} {\Large ΑΣΚΗΣΕΙΣ}}\vspace{5mm}}]
\paragraph{Χάραξη γραφικής παράστασης}
\askhsh Δίνεται η συνάρτηση $f(x)=\hm{(2x)}$ με $x\in\mathbb{R}$.
\begin{alist}
\item Να βρεθεί η περίοδος καθώς και τα ακρότατα της $f$.
\item Να χαράξετε τη γραφική παράσταση της $f$ σε διάστημα μιας περιόδου.
\end{alist}
\askhsh Δίνεται η συνάρτηση $f(x)=\syn{(3x)}$ με $x\in\mathbb{R}$.
\begin{alist}
\item Να βρεθεί η περίοδος καθώς και τα ακρότατα της $f$.
\item Να χαράξετε τη γραφική παράσταση της $f$ στο διάστημα $[0,2\pi]$.
\end{alist}
\askhsh Δίνεται η συνάρτηση $f(x)=\syn{\left(\dfrac{x}{2}\right)}$ με $x\in\mathbb{R}$.
\begin{alist}
\item Να βρεθούν η περίοδος και τα ακρότατα της $f$.
\item Να χαράξετε τη γραφική παράσταση της $f$ στο διάστημα $[0,4\pi]$.
\item Βρείτε τα σημεία τομής της $C_f$ με τον άξονα $x'x$.
\end{alist}
\askhsh Δίνεται η συνάρτηση $f(x)=2\hm{x}$ με $x\in\mathbb{R}$.
\begin{alist}
\item Να βρεθούν η περίοδος και τα ακρότατα της $f$.
\item Σχεδιάστε τη γραφική παράσταση της $f$ στο διάστημα $[0,2\pi]$.
\item Βρείτε τα σημεία τομής της $C_f$ με τους άξονες $x'x$ και $y'y$.
\end{alist}
\askhsh Δίνεται η συνάρτηση $f(x)=3\syn{\left(2x\right)}$ με $x\in\mathbb{R}$.
\begin{alist}
\item Να βρεθούν η περίοδος και τα ακρότατα της $f$.
\item Να χαράξετε τη γραφική παράσταση της $f$ στο διάστημα $[0,2\pi]$.
\item Βρείτε τα διαστήματα μονοτονίας της $f$.
\end{alist}
\askhsh Δίνεται η συνάρτηση $f(x)=2\syn{\left(\dfrac{x}{3}\right)}+1$ με $x\in\mathbb{R}$.
\begin{alist}
\item Να βρεθούν η περίοδος και τα ακρότατα της $f$.
\item Να χαράξετε τη γραφική παράσταση της $f$ στο διάστημα $[0,\pi]$.
\item Βρείτε τα διαστήματα μονοτονίας της $f$.
\end{alist}
\askhsh Δίνεται η συνάρτηση $f(x)=3\syn{\left(\pi x\right)}-2$ με $x\in\mathbb{R}$.
\begin{alist}
\item Να βρεθούν η περίοδος και τα ακρότατα της $f$.
\item Να χαράξετε τη γραφική παράσταση της $f$ στο διάστημα $[0,2]$.
\item Βρείτε τα σημεία τομής της $C_f$ με τους άξονες $x'x$ και $y'y$.
\end{alist}
\askhsh Για καθεμία από τις παρακάτω συναρτήσεις, να βρεθούν η περίοδος και τα ακρότατα.
\begin{multicols}{2}
\begin{alist}
\item $f(x)=\hm{(3x)}$
\item $f(x)=\syn{(4x)}$
\item $f(x)=\hm{\left(\dfrac{x}{4}\right)}$
\item $f(x)=\syn{\left(\dfrac{x}{3}\right)}$
\item $f(x)=\hm{(\pi x)}$
\item $f(x)=\syn{\left(\dfrac{\pi x}{2}\right)}$
\end{alist}
\end{multicols}
\askhsh Για καθεμία από τις παρακάτω συναρτήσεις, να βρεθούν η περίοδος και τα ακρότατα.
\begin{multicols}{2}
\begin{alist}
\item $f(x)=3\hm{x}$
\item $f(x)=4\syn{x}$
\item $f(x)=\dfrac{\hm{x}}{2}$
\item $f(x)=\dfrac{3\syn{x}}{4}$
\item $f(x)=-3\hm{x}$
\item $f(x)=-2\syn{x}$
\end{alist}
\end{multicols}
\askhsh Για καθεμία από τις παρακάτω συναρτήσεις, να βρεθούν η περίοδος και τα ακρότατα.
\begin{multicols}{2}
\begin{alist}
\item $f(x)=2\hm{(3x)}$
\item $f(x)=2\syn{(4x)}$
\item $f(x)=4\hm{\left(\dfrac{x}{2}\right)}$
\item $f(x)=-2\syn{\left(\dfrac{x}{3}\right)}$
\item $f(x)=-3\hm{(\dfrac{\pi x}{2})}$
\item $f(x)=4\syn{\left(\pi x\right)}$
\end{alist}
\end{multicols}
\askhsh Για καθεμία από τις παρακάτω συναρτήσεις, να βρεθούν η περίοδος και τα ακρότατα.
\begin{alist}
\item $f(x)=2\hm{(2x)}-1$
\item $f(x)=5\syn{(3x)}+3$
\item $f(x)=-3\hm{(3x)}+2$
\item $f(x)=-3\syn{\left(\dfrac{x}{2}\right)}-1$
\item $f(x)=8\hm{(\pi x)}-7$
\item $f(x)=-5\syn{\left(\dfrac{\pi x}{4}\right)}+3$
\end{alist}
\askhsh Για καθεμία από τις παρακάτω συναρτήσεις, να σχεδιάσετε τη γραφική παράσταση της στο διάστημα $[0,2\pi]$.
\begin{multicols}{2}
\begin{alist}
\item $f(x)=\ef{(2x)}$
\item $f(x)=\syf{(3x)}$
\item $f(x)=\ef{\left(\dfrac{x}{3}\right)}$
\item $f(x)=\syf{\left(\dfrac{x}{2}\right)}$
\end{alist}
\end{multicols}
\paragraph{Άρτιες - Περιττές}
\askhsh Να εξετάσετε αν οι παρακάτω συναρτήσεις είναι άρτιες ή περιττές.
\begin{multicols}{2}
\begin{alist}
\item $f(x)=\dfrac{\hm{x}}{x^2+3}$
\item $f(x)=\dfrac{\hm{x}}{x}$
\item $f(x)=\dfrac{\syn{x}}{|x|-1}$
\item $f(x)=\hm{\left(x^3-x\right)}$
\end{alist}
\end{multicols}
\paragraph{Περιοδικότητα}
\askhsh Να αποδείξετε ότι καθεμία από τις παρακάτω συναρτήσεις είναι περιοδική, με περίοδο τον δοσμένο αριθμό $T$.
\begin{alist}
\item $f(x)=\hm{(2x)}+\syn{(4x)}$, με $T=\pi$
\item $f(x)=\hm{(4x)}+\ef{(2x)}$, με $T=\dfrac{\pi}{2}$
\item $f(x)=\syn{(4x)}+\ef{(4x)}$, με $T=\dfrac{\pi}{2}$
\item $f(x)=\syn{(2x)}\cdot\syf{x}$, με $T=\pi$
\item $f(x)=\syf{(2x)}+\ef{(8x)}$, με $T=\dfrac{\pi}{2}$
\item $f(x)=\hm{(2x)}\cdot\ef{x}$, με $T=\pi$
\item $f(x)=\syn{(3x)}+\ef{(4x)}$, με $T=\dfrac{2\pi}{3}$
\end{alist}
\askhsh
\end{document}
