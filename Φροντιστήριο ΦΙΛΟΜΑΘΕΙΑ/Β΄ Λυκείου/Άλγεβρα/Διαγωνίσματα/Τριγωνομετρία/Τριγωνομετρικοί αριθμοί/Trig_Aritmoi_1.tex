\documentclass[twoside,nofonts,ektypwsh]{frontisthrio-diag}
\usepackage[amsbb,subscriptcorrection,zswash,mtpcal,mtphrb,mtpfrak]{mtpro2}
\usepackage[no-math,cm-default]{fontspec}
\usepackage{amsmath}
\usepackage{xunicode}
\usepackage{xgreek}
\let\hbar\relax
\defaultfontfeatures{Mapping=tex-text,Scale=MatchLowercase}
\setmainfont[Mapping=tex-text,Numbers=Lining,Scale=1.0,BoldFont={Minion Pro Bold}]{Minion Pro}
\newfontfamily\scfont{GFS Artemisia}
\font\icon = "Webdings"
\usepackage{fontawesome5}
\newfontfamily{\FA}{fontawesome.otf}
\xroma{cyan!70!black}
%------TIKZ - ΣΧΗΜΑΤΑ - ΓΡΑΦΙΚΕΣ ΠΑΡΑΣΤΑΣΕΙΣ ----
\usepackage{tikz,pgfplots}
\usepackage{tkz-euclide}
\usetkzobj{all}
\usepackage[framemethod=TikZ]{mdframed}
\usetikzlibrary{decorations.pathreplacing}
\tkzSetUpPoint[size=7,fill=white]
%-----------------------
\usepackage{calc,tcolorbox}
\tcbuselibrary{skins,theorems,breakable}
\usepackage{hhline}
\usepackage[explicit]{titlesec}
\usepackage{graphicx}
\usepackage{multicol}
\usepackage{multirow}
\usepackage{tabularx}
\usetikzlibrary{backgrounds}
\usepackage{sectsty}
\sectionfont{\centering}
\usepackage{enumitem}
\usepackage{adjustbox}
\usepackage{mathimatika,gensymb,eurosym,wrap-rl}
\usepackage{systeme,regexpatch}
%-------- ΜΑΘΗΜΑΤΙΚΑ ΕΡΓΑΛΕΙΑ ---------
\usepackage{mathtools}
%----------------------
%-------- ΠΙΝΑΚΕΣ ---------
\usepackage{booktabs}
%----------------------
%----- ΥΠΟΛΟΓΙΣΤΗΣ ----------
\usepackage{calculator}
%----------------------------
%------------------------------------------
\newcommand{\tss}[1]{\textsuperscript{#1}}
\newcommand{\tssL}[1]{\MakeLowercase{\textsuperscript{#1}}}
%---------- ΛΙΣΤΕΣ ----------------------
\newlist{bhma}{enumerate}{3}
\setlist[bhma]{label=\bf\textit{\arabic*\textsuperscript{o}\;Βήμα :},leftmargin=0cm,itemindent=1.8cm,ref=\bf{\arabic*\textsuperscript{o}\;Βήμα}}
\newlist{tropos}{enumerate}{3}
\setlist[tropos]{label=\bf\textit{\arabic*\textsuperscript{oς}\;Τρόπος :},leftmargin=0cm,itemindent=2.3cm,ref=\bf{\arabic*\textsuperscript{oς}\;Τρόπος}}
% Αν μπει το bhma μεσα σε tropo τότε
%\begin{bhma}[leftmargin=.7cm]
\tkzSetUpPoint[size=7,fill=white]
\tikzstyle{pl}=[line width=0.3mm]
\tikzstyle{plm}=[line width=0.4mm]
\usepackage{etoolbox}
\makeatletter
\renewrobustcmd{\anw@true}{\let\ifanw@\iffalse}
\renewrobustcmd{\anw@false}{\let\ifanw@\iffalse}\anw@false
\newrobustcmd{\noanw@true}{\let\ifnoanw@\iffalse}
\newrobustcmd{\noanw@false}{\let\ifnoanw@\iffalse}\noanw@false
\renewrobustcmd{\anw@print}{\ifanw@\ifnoanw@\else\numer@lsign\fi\fi}
\makeatother

\usepackage{path}
\pathALa

\begin{document}
\titlos{Άλγεβρα Β΄ Λυκείου}{Τργωνομετρικοί αριθμοί}{Β}
\begin{thema}
\item\mbox{}\\\vspace{-7mm}
\begin{erwthma}
\item Να δώσετε τον ορισμό του ημιτόνου μιας οξείας γωνίας $ \omega $ ενός ορθογωνίου τριγώνου.\monades{8}
\item Να γράψετε τον τύπο μετατροπής μιας γωνίας από μοίρες σε ακτίνια και αντίστροφα.\monades{2}
\item Δίνεται σημείο $ M(x,y) $ σε σύστημα συντεταγμένων $ xOy $ και έστω γωνία $ \omega=x\hat{O}M $. Να γράψετε τον τύπο του συνημιτόνου και της εφαπτομένης της γωνίας $ \omega $.\monades{5}
\item \swstolathospan
\begin{alist}
\item Η γωνία $ 210\degree $ έχει θετικό συνημίτονο.
\item Ισχύει ότι $ \hm{\frac{3\pi}{2}}=-1 $.
\item Αν για μια γωνία $ \omega $ ισχύει $ \syn{\omega}=\frac{3}{2} $ τότε $ \omega=\frac{\pi}{3} $.
\item Οι συντεταγμένες ενός σημείου $ M $ που βρίσκεται πάνω στον τριγωνομετρικό κύκλο έχουν τη μορφή $ M(\syn{\omega},\hm{\omega}) $ όπου $ \omega=x\hat{O}M $.
\item Για οποιαδήποτε γωνία $ \omega $ ισχύει $ |\hm{\omega}|\leq 1 $.
\end{alist}
\end{erwthma}\monades{10}
\item\mbox{}\\\vspace{-7mm}
\begin{erwthma}
\item Να μετατραπούν οι παρακάτω γωνίες σε ακτίνια
\begin{multicols}{2}
\begin{alist}
\item $ 30\degree $
\item $ 315\degree $
\end{alist}
\end{multicols}\monades{6}
\item Να υπολογίσετε τις παρακάτω παραστάσεις
\begin{alist}
\item $ A=\hm^2{30\degree}-\syn^2{45\degree}+\ef^2{60\degree} $
\item $ B=2\syn{\frac{\pi}{4}}\cdot\hm{\frac{\pi}{4}}+3\syf{\frac{\pi}{6}}\cdot\ef{\frac{\pi}{6}} $
\item $ \varGamma=\syn{\pi}-\hm{\frac{3\pi}{2}}+\syf{\frac{\pi}{2}} $
\end{alist}\monades{12}
\item Να υπολογίσετε τους τριγωνομετρικούς αριθμούς των παρακάτω γωνιών.
\begin{multicols}{2}
\begin{alist}
\item $ 1845\degree $
\item $ \dfrac{49\pi}{6} $
\end{alist}
\end{multicols}
\end{erwthma}\monades{7}
\item Δίνεται τρίγωνο $ AB\varGamma $ με $ \hat{A}=90\degree, AB=15 $ και $ \hm{\hat{\varGamma}}=\frac{3}{5} $.
\begin{erwthma}
\item Να βρεθούν οι πλευρές $ B\varGamma $ και $ A\varGamma $.\monades{8}
\item Να βρεθούν οι τριγωνομετρικοί αριθμοί της γωνίας $ \hat{B} $.\monades{6}
\item Αν $ A\varDelta $ είναι το ύψος του τριγώνου τότε να βρεθούν τα μήκη των πλευρών $ A\varDelta,B\varDelta $ και $ \varGamma\varDelta $.\monades{11}
\end{erwthma}
\item\mbox{}\\\vspace{-7mm}
\begin{erwthma}
\item Να βρεθούν τα πρόσημα των τριγωνομετρικών αριθμών και των παρακάτω παραστάσεων.
\begin{alist}
\item $ \hm{\frac{7\pi}{15}} $\monades{3}
\item $ A=\syf{280\degree}\cdot\hm{110\degree}\cdot\syn{215\degree} $.\monades{5}
\end{alist}
\item Να δείξετε ότι για οποιεσδήποτε γωνίες $ x,y $ ισχύουν οι παρακάτω ανισότητες:
\begin{alist}
\item $ -7\leq 5\hm{x}-2\syn{y}\leq 7 $
\item $ 3\hm^2{x}+2\leq 5 $
\end{alist}\monades{8}
\item Δίνονται γωνίες $ \phi $ και $ \omega $ για τις οποίες ισχύει
\[ \systeme[\phi\omega]{2\phi-5\omega=-60\degree,4\phi+\omega=210\degree} \]
Να υπολογίσετε την παράσταση $ A=\hm^2{\phi}+2\syn^2{\omega}+\ef{\phi}+\hm{\omega} $
\end{erwthma}\monades{9}
\end{thema}
\end{document}
