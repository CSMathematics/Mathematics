\documentclass[twoside,nofonts,ektypwsh]{frontisthrio-diag}
\usepackage[amsbb,subscriptcorrection,zswash,mtpcal,mtphrb,mtpfrak]{mtpro2}
\usepackage[no-math,cm-default]{fontspec}
\usepackage{amsmath}
\usepackage{xunicode}
\usepackage{xgreek}
\let\hbar\relax
\defaultfontfeatures{Mapping=tex-text,Scale=MatchLowercase}
\setmainfont[Mapping=tex-text,Numbers=Lining,Scale=1.0,BoldFont={Minion Pro Bold}]{Minion Pro}
\newfontfamily\scfont{GFS Artemisia}
\font\icon = "Webdings"
\usepackage{fontawesome5}
\newfontfamily{\FA}{fontawesome.otf}
\xroma{cyan!70!black}
%------TIKZ - ΣΧΗΜΑΤΑ - ΓΡΑΦΙΚΕΣ ΠΑΡΑΣΤΑΣΕΙΣ ----
\usepackage{tikz,pgfplots}
\usepackage{tkz-euclide}
\usetkzobj{all}
\usepackage[framemethod=TikZ]{mdframed}
\usetikzlibrary{decorations.pathreplacing}
\tkzSetUpPoint[size=7,fill=white]
%-----------------------
\usepackage{calc,tcolorbox}
\tcbuselibrary{skins,theorems,breakable}
\usepackage{hhline}
\usepackage[explicit]{titlesec}
\usepackage{graphicx}
\usepackage{multicol}
\usepackage{multirow}
\usepackage{tabularx}
\usetikzlibrary{backgrounds}
\usepackage{sectsty}
\sectionfont{\centering}
\usepackage{enumitem}
\usepackage{adjustbox}
\usepackage{mathimatika,gensymb,eurosym,wrap-rl}
\usepackage{systeme,regexpatch}
%-------- ΜΑΘΗΜΑΤΙΚΑ ΕΡΓΑΛΕΙΑ ---------
\usepackage{mathtools}
%----------------------
%-------- ΠΙΝΑΚΕΣ ---------
\usepackage{booktabs}
%----------------------
%----- ΥΠΟΛΟΓΙΣΤΗΣ ----------
\usepackage{calculator}
%----------------------------
%------------------------------------------
\newcommand{\tss}[1]{\textsuperscript{#1}}
\newcommand{\tssL}[1]{\MakeLowercase{\textsuperscript{#1}}}
%---------- ΛΙΣΤΕΣ ----------------------
\newlist{bhma}{enumerate}{3}
\setlist[bhma]{label=\bf\textit{\arabic*\textsuperscript{o}\;Βήμα :},leftmargin=0cm,itemindent=1.8cm,ref=\bf{\arabic*\textsuperscript{o}\;Βήμα}}
\newlist{tropos}{enumerate}{3}
\setlist[tropos]{label=\bf\textit{\arabic*\textsuperscript{oς}\;Τρόπος :},leftmargin=0cm,itemindent=2.3cm,ref=\bf{\arabic*\textsuperscript{oς}\;Τρόπος}}
% Αν μπει το bhma μεσα σε tropo τότε
%\begin{bhma}[leftmargin=.7cm]
\tkzSetUpPoint[size=7,fill=white]
\tikzstyle{pl}=[line width=0.3mm]
\tikzstyle{plm}=[line width=0.4mm]
\usepackage{etoolbox}
\makeatletter
\renewrobustcmd{\anw@true}{\let\ifanw@\iffalse}
\renewrobustcmd{\anw@false}{\let\ifanw@\iffalse}\anw@false
\newrobustcmd{\noanw@true}{\let\ifnoanw@\iffalse}
\newrobustcmd{\noanw@false}{\let\ifnoanw@\iffalse}\noanw@false
\renewrobustcmd{\anw@print}{\ifanw@\ifnoanw@\else\numer@lsign\fi\fi}
\makeatother


\begin{document}
\titlos{Άλγεβρα Β΄ Λυκείου}{Τριγωνομετρία}{Β}
\begin{thema}
\item\mbox{}\\
\vspace{-7mm}
\begin{erwthma}
\item Να απαντήσετε στις παρακάτω ερωτήσεις.
\begin{alist}
\item Τι ονομάζεται τριγωνομετρική ταυτότητα;
\item Τι σχέση έχουν μεταξύ τους τα ημίτονα δύο αντίθετων γωνιών;
\item Από ποιόν τύπο δίνονται οι λύσεις της εξίσωσης $\ef{x}=\ef{\theta}$;
\item Ποια είναι η τριγωνομετρική ταυτότητα που συνδέει την εφαπτομένη και τη συνεφαπτομένη μιας γωνίας $\omega$;
\item Για ποιες τιμές του αριθμού $a\in\mathbb{R}$ η εξίσωση $\hm{x}=a$ είναι αδύνατη;
\item Τι ονομάζεται τριγωνομετρική εξίσωση;
\end{alist}\monades{3}
\item\swstolathos
\begin{alist}
\item Ισχύει η σχέση $\hm{40\degree}=-\hm{140\degree}$.
\item Υπάρχει γωνία $x$ για την οποία ισχύει συγχρόνως $\hm{x}=0$ και $\syn{x}=0$.
\item Η γωνία $\theta=\frac{\pi}{4}$ είναι μια λύση της εξίσωσης $2\syn{x}-\sqrt{2}=0$.
\item Η εξίσωση $\hm{x}=\hm{\frac{\pi}{2}}$ έχει λύσεις $x=2\kappa \pi+\frac{\pi}{2}$ όπου $\kappa\in\mathbb{Z}$.
\item Η εξίσωση $\hm{x}=\frac{3}{2}$ έχει λύσεις τις γωνίες $x=2\kappa\pi+\frac{\pi}{3}$ όπου $\kappa\in\mathbb{Z}$.
\end{alist}\monades{2}
\end{erwthma}
\item\mbox{}\\
\vspace{-7mm}
\begin{erwthma}
\item Να υπολογίσετε τους τριγωνομετρικούς αριθμούς της γωνίας
\begin{multicols}{2}
\begin{alist}
\item $210\degree$
\item $\frac{7\pi}{4}$
\end{alist}
\end{multicols}\monades{1.5}
\item Να λυθούν οι τριγωνομετρικές εξισώσεις
\begin{multicols}{2}
\begin{alist}
\item $\hm{x}=-\frac{1}{2}$
\item $3\ef{x}-\sqrt{3}=0$
\end{alist}
\end{multicols}\monades{1.5}
\item Δίνεται γωνία $\omega$ για την οποία ισχύει $\omega\in\left(0,\frac{\pi}{2}\right)$ και $\hm{x}=\frac{3}{5}$. Να βρεθούν οι υπόλοιποι τριγωνομετρικοί αριθμοί της γωνίας.\monades{2}
\end{erwthma}
\item\mbox{}\\
\vspace{-7mm}
\begin{erwthma}
\item Να λυθεί η τριγωνομετρική εξίσωση $2\hm^2{x}+3\syn{x}-3=0$.\monades{2}
\item Να αποδείξετε την τριγωνομετρική ταυτότητα $\dfrac{\hm{x}}{1-\syn{x}}+\dfrac{\hm{x}}{1+\syn{x}}=\dfrac{2}{\hm{x}}$.\monades{1.5}
\item Να υπολογίζετε την τιμή της παράστασης $A=\dfrac{\hm{\frac{31\pi}{4}}\cdot\syn{\frac{41\pi}{6}}}{\ef{\frac{28\pi}{3}}}$.\monades{1.5}
\end{erwthma}
\item\mbox{}\\
\vspace{-7mm}
\begin{erwthma}
\item Να λυθεί η εξίσωση
\[\hm{\left(2x-\frac{\pi}{4}\right)}=\hm{\left(\frac{\pi}{3}-x\right)}\]\monades{2}
\item Να λυθεί η εξίσωση
\[2\hm^2{\left(\frac{\pi}{2}-3x\right)}=2-\syn{3x}\]\monades{2}
\item Δίνεται ορθογώνιο τρίγωνο $AB\varGamma$ με $\hat{A}=90\degree$.Για τις οξείες γωνίες $\hat{B}$ και $\hat{\varGamma}$ ισχύει η παρακάτω σχέση
\[\hm{B}+\syn{\varGamma}=1\]
Να υπολογιστούν οι γωνίες $\hat{B}$ και $\hat{\varGamma}$.\monades{1}
\end{erwthma}

\end{thema}
\end{document}
