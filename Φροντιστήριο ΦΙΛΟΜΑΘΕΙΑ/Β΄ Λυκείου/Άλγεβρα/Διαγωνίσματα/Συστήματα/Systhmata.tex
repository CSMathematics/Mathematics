\documentclass[ektypwsh]{frontisthrio-diag}
\usepackage[amsbb,subscriptcorrection,zswash,mtpcal,mtphrb,mtpfrak]{mtpro2}
\usepackage[no-math,cm-default]{fontspec}
\usepackage{amsmath}
\usepackage{xunicode}
\usepackage{xgreek}
\let\hbar\relax
\defaultfontfeatures{Mapping=tex-text,Scale=MatchLowercase}
\setmainfont[Mapping=tex-text,Numbers=Lining,Scale=1.0,BoldFont={Minion Pro Bold}]{Minion Pro}
\newfontfamily\scfont{GFS Artemisia}
\font\icon = "Webdings"
\usepackage{fontawesome5}
\newfontfamily{\FA}{fontawesome.otf}
\xroma{red!70!black}
%------TIKZ - ΣΧΗΜΑΤΑ - ΓΡΑΦΙΚΕΣ ΠΑΡΑΣΤΑΣΕΙΣ ----
\usepackage{tikz,pgfplots}
\usepackage{tkz-euclide}
\usetkzobj{all}
\usepackage[framemethod=TikZ]{mdframed}
\usetikzlibrary{decorations.pathreplacing}
\tkzSetUpPoint[size=7,fill=white]
%-----------------------
\usepackage{calc,tcolorbox}
\tcbuselibrary{skins,theorems,breakable}
\usepackage{hhline}
\usepackage[explicit]{titlesec}
\usepackage{graphicx}
\usepackage{multicol}
\usepackage{multirow}
\usepackage{tabularx}
\usetikzlibrary{backgrounds}
\usepackage{sectsty}
\sectionfont{\centering}
\usepackage{enumitem}
\usepackage{adjustbox}
\usepackage{mathimatika,gensymb,eurosym,wrap-rl}
\usepackage{systeme,regexpatch}
%-------- ΜΑΘΗΜΑΤΙΚΑ ΕΡΓΑΛΕΙΑ ---------
\usepackage{mathtools}
%----------------------
%-------- ΠΙΝΑΚΕΣ ---------
\usepackage{booktabs}
%----------------------
%----- ΥΠΟΛΟΓΙΣΤΗΣ ----------
\usepackage{calculator}
%----------------------------

%------------------------------------------
\newcommand{\tss}[1]{\textsuperscript{#1}}
\newcommand{\tssL}[1]{\MakeLowercase{\textsuperscript{#1}}}
%---------- ΛΙΣΤΕΣ ----------------------
\newlist{bhma}{enumerate}{3}
\setlist[bhma]{label=\bf\textit{\arabic*\textsuperscript{o}\;Βήμα :},leftmargin=0cm,itemindent=1.8cm,ref=\bf{\arabic*\textsuperscript{o}\;Βήμα}}
\newlist{tropos}{enumerate}{3}
\setlist[tropos]{label=\bf\textit{\arabic*\textsuperscript{oς}\;Τρόπος :},leftmargin=0cm,itemindent=2.3cm,ref=\bf{\arabic*\textsuperscript{oς}\;Τρόπος}}
% Αν μπει το bhma μεσα σε tropo τότε
%\begin{bhma}[leftmargin=.7cm]
\tkzSetUpPoint[size=7,fill=white]
\tikzstyle{pl}=[line width=0.3mm]
\tikzstyle{plm}=[line width=0.4mm]
\usepackage{etoolbox}
\makeatletter
\renewrobustcmd{\anw@true}{\let\ifanw@\iffalse}
\renewrobustcmd{\anw@false}{\let\ifanw@\iffalse}\anw@false
\newrobustcmd{\noanw@true}{\let\ifnoanw@\iffalse}
\newrobustcmd{\noanw@false}{\let\ifnoanw@\iffalse}\noanw@false
\renewrobustcmd{\anw@print}{\ifanw@\ifnoanw@\else\numer@lsign\fi\fi}
\makeatother
\ekthetesdeiktes
\usepackage{path}


\begin{document}
\titlos{Άλγεβρα Β΄ Λυκείου}{Συστήματα}{Β}
\vspace{-15mm}
\begin{thema}
\item\mbox{}\\\vspace{-7mm}
\begin{erwthma}
\item Ποια συνθήκη πρέπει να ισχύει ώστε ένα $ 2\times2 $ γραμμικό σύστημα να έχει μοναδική λύση;\monades{1}
\item Τι ονομάζεται λύση ενός γραμμικού συστήματος;\monades{1}
\item Για ποιες τιμές των συντελεστών $ a,\beta $ παριστάνει ευθεία γραμμή η εξίσωση $ ax+\beta y=\gamma $;\monades{1}
\item \swstolathos
\begin{alist}[leftmargin=3mm]
\item Το σημείο $ A(3,-2) $ ανήκει στην ευθεία $ 2x-y=8 $.
\item Το σύστημα $ \systeme{x+3y=2,2x+6y=1} $ \ είναι αόριστο.
\item Η εξίσωση $ 0x+0y=3 $ παριστάνει ευθεία γραμμή.
\item Το ζεύγος $ (x,y)=(2,1) $ είναι λύση του μη γραμμικού συστήματος $ \ccases{x^2-y^2=4\\3x+y=7} $.
\item Αν για ένα γραμμικό σύστημα ισχύει $ D\neq0 $ τότε έχει μοναδική λύση.\monades{2}
\end{alist}
\end{erwthma}
\item\mbox{}\\
Να λυθούν τα παρακάτω συστήματα.
\begin{erwthma}
\item $ \systeme{4x-3y=9,2x+y=7}\ \  $ με τη μέθοδο της αντικατάστασης.\monades{1,7}
\item $ \systeme{x-7y=-9,3x+4y=23}\ \  $ με τη μέθοδο των αντίθετων συντελεστών.\monades{1,6}
\item $ \systeme{5x-y=7,2x+3y=-4}\ \  $ με τη μέθοδο των οριζουσών.\monades{1,7}
\end{erwthma}
\item\mbox{}\\\vspace{-7mm}
\begin{erwthma}
\item Δίνεται το ακόλουθο παραμετρικό σύστημα.
\[ \systeme[xy]{\lambda x+3y=\lambda,x+{(\lambda-2)}y=4-\lambda} \]
\begin{alist}
\item Να βρεθούν οι τιμές της παραμέτρου $ \lambda $ για τις οποίες το σύστημα έχει μοναδική λύση, καθώς και να βρεθεί η λύση αυτή.\monades{2}
\item Για ποιες τιμές της παραμέτρου το σύστημα είναι αόριστο και για ποιες αδύνατο;\monades{1}
\end{alist}
\item Να λυθεί το ακόλουθο μη γραμμικό σύστημα.
\[ \ccases{x^2+(y-2)^2=13\\3x+y=10} \]\monades{2}
\end{erwthma}
\item\mbox{}\\\vspace{-7mm}
\begin{erwthma}
\item Έστω $ D,D_x,D_y $ είναι οι ορίζουσες ενός $ 2\times2 $ γραμμικού συστήματος με μοναδική λύση για τις οποίες ισχύει η σχέση :
\[ \left( 3D_x-4D_y-D\right) ^2+\left( 2D_x+D_y+D\right) ^2=0\]
\begin{alist}
\item Χρησιμοποιώντας την παραπάνω σχέση να σχηματίσετε το $ 2\times2 $ γραμμικό σύστημα ως προς τις μεταβλητές $ (x,y) $.\\\monades{1}
\item Να βρεθεί η λύση $ (x,y) $ του γραμμικού συστήματος.\monades{2}
\end{alist}
\item Το μέγεθος μιας τηλεόρασης είναι $ 42'' $ και γνωρίζουμε επίσης ότι οι πλευρές έχουν λόγο 16:9 (HD Video Standard). Να βρεθούν οι διαστάσεις της τηλεόρασης.\\
{\footnotesize \textit{Υπόδειξη} : Το μέγεθος μιας τηλεόρασης δίνεται από το μήκος της διαγωνίου της οθόνης δοσμένο σε ίντσες.\\(Μια ίντσα $ 1''=2{,}54cm $)}\monades{2}
\end{erwthma}
\end{thema}
\end{document}
