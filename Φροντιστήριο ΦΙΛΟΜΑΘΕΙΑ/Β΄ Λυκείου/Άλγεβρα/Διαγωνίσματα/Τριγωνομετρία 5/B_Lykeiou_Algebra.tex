\documentclass[11pt,a4paper]{article}
\usepackage[english,greek]{babel}
\usepackage[utf8]{inputenc}
\usepackage{nimbusserif}
\usepackage[T1]{fontenc}
\usepackage[left=2.00cm, right=2.00cm, top=2.00cm, bottom=2.00cm]{geometry}
\usepackage{amsmath}
\let\myBbbk\Bbbk
\let\Bbbk\relax
\usepackage[amsbb,subscriptcorrection,zswash,mtpcal,mtphrb,mtpfrak]{mtpro2}
\usepackage{siunitx, graphicx,multicol,multirow,enumitem,tabularx,mathimatika,gensymb,venndiagram,hhline,longtable,tkz-euclide,fontawesome5,eurosym,tcolorbox,wrap-rl}
\tcbuselibrary{skins,theorems,breakable}
\newlist{rlist}{enumerate}{3}
\setlist[rlist]{itemsep=0mm,label=\roman*.}
\newlist{alist}{enumerate}{3}
\setlist[alist]{itemsep=0mm,label=\alph*.}
\newlist{balist}{enumerate}{3}
\setlist[balist]{itemsep=0mm,label=\bf\alph*.}
\newlist{Alist}{enumerate}{3}
\setlist[Alist]{itemsep=0mm,label=\Alph*.}
\newlist{bAlist}{enumerate}{3}
\setlist[bAlist]{itemsep=0mm,label=\bf\Alph*.}
\renewcommand{\textstigma}{\textsigma\texttau}
\newlist{thema}{enumerate}{3}
\setlist[thema]{label=\bf\large{ΘΕΜΑ \textcolor{black}{\Alph*}},itemsep=0mm,leftmargin=0cm,itemindent=18mm}
\newlist{erwthma}{enumerate}{3}
\setlist[erwthma]{label=\bf{\large{\textcolor{black}{\Alph{themai}.\arabic*}}},itemsep=0mm,leftmargin=0.8cm}

\newcommand{\lysh}{\textcolor{black}{\textbf{\faCheck\ \ ΛΥΣΗ}}}
\renewcommand{\textstigma}{\textsigma\texttau}
%----------- ΟΡΙΣΜΟΣ------------------
\newcounter{orismos}[section]
\renewcommand{\theorismos}{\thesection.\arabic{orismos}}   
\newcommand{\Orismos}{\refstepcounter{orismos}{\textbf{\textcolor{black}{\kerkissans{Ορισμός\hspace{2mm}\theorismos}}\;:\;}{}}}

\newenvironment{orismos}[1]
{\begin{tcolorbox}[title=\Orismos {\textcolor{black}{\kerkissans{#1}}},breakable,bottomtitle=-1.5mm,
enhanced standard,titlerule=-.2pt,toprule=0pt, rightrule=0pt, bottomrule=0pt,
colback=white,left=2mm,top=1mm,bottom=0mm,
boxrule=0pt,
colframe=white,borderline west={1.5mm}{0pt}{black},leftrule=2mm,sharp corners,coltitle=black]}
{\end{tcolorbox}}

\newcommand{\kerkissans}[1]{{\fontfamily{maksf}\selectfont \textbf{#1}}}
\renewcommand{\textdexiakeraia}{}

\usepackage[
backend=biber,
style=alphabetic,
sorting=ynt
]{biblatex}

\begin{document}
\begin{center}
{\LARGE \kerkissans{Άλγεβρα Β' Λυκείου}}\\
{\large \kerkissans{Επαναληπτικό διαγώνισμα - Τριγωνομετρία\\\today}}
\end{center}
\begin{thema}
\item\mbox{}\\\vspace{-5mm}
\begin{erwthma}
\item Να απαντήσετε στις παρακάτω ερωτήσεις.
\begin{alist}
\item Τι ονομάζεται τριγωνομετρική ταυτότητα?
\item Τι σχέση έχουν μεταξύ τους τα ημίτονα δύο αντίθετων γωνιών?
\item Από ποιόν τύπο δίνονται οι λύσεις της εξίσωσης $\ef{x}=\ef{\theta}$?
\item Ποια είναι η περίοδος και το πεδίο ορισμού της συνάρτησης $f(x)=\ef{x}$?
\item Για ποιες τιμές του αριθμού $a\in\mathbb{R}$ η εξίσωση $\hm{x}=a$ είναι αδύνατη?
\item Τι ονομάζεται τριγωνομετρική εξίσωση?
\end{alist}
\item Να χαρακτηρίσετε καθεμία από τις παρακάτω προτάσεις ως \textbf{Σωστή} ή \textbf{Λάθος}.
\begin{alist}
\item Ισχύει η σχέση $\hm{40\degree}=-\hm{140\degree}$.
\item Υπάρχει γωνία $x$ για την οποία ισχύει συγχρόνως $\hm{x}=0$ και $\syn{x}=0$.
\item Η γωνία $\theta=\frac{\pi}{4}$ είναι μια λύση της εξίσωσης $2\syn{x}-\sqrt{2}=0$.
\item Η εξίσωση $\hm{x}=\hm{\frac{\pi}{2}}$ έχει λύσεις $x=2\kappa \pi+\frac{\pi}{2}$ όπου $\kappa\in\mathbb{Z}$.
\item Η εξίσωση $\hm{x}=\frac{3}{2}$ έχει λύσεις τις γωνίες $x=2\kappa\pi+\frac{\pi}{3}$ όπου $\kappa\in\mathbb{Z}$.
\end{alist}
\end{erwthma}
\item\mbox{}\\
\vspace{-7mm}
\begin{erwthma}
\item Να υπολογίσετε τους τριγωνομετρικούς αριθμούς της γωνίας
\begin{multicols}{2}
\begin{alist}
\item $210\degree$
\item $\frac{7\pi}{4}$
\end{alist}
\end{multicols}
\item Να αποδείξετε την τριγωνομετρική ταυτότητα $\dfrac{\hm{x}}{1-\syn{x}}+\dfrac{\hm{x}}{1+\syn{x}}=\dfrac{2}{\hm{x}}$.
\item Δίνεται γωνία $\omega$ για την οποία ισχύει $\omega\in\left(0,\frac{\pi}{2}\right)$ και $\hm{x}=\frac{3}{5}$. Να βρεθούν οι υπόλοιποι τριγωνομετρικοί αριθμοί της γωνίας.
\end{erwthma}
\item\mbox{}\\\vspace{-5mm}
\begin{erwthma}
\item Να λυθεί η τριγωνομετρική εξίσωση $2\hm^2{x}+3\syn{x}-3=0$.
\item Να βρεθούν οι λύσεις της εξίσωσης $ \syn{\left(2x-\frac{\pi}{3}\right)}=\frac{\sqrt{2}}{2} $ που ανήκουν στο διάστημα $[0,\pi]$.
\item Να υπολογίσετε την τιμή της παράστασης
\[ A=\hm^2{x_1}-3\ef{(\pi-x_1)}\cdot\syn{(\pi+x_1)} \]
όπου $x_1=\frac{2\pi}{3}$ η μικρότερη από τις λύσεις της προηγούμενης εξίσωσης.
\end{erwthma}
\item\mbox{}\\Η θερμοκρασία μιας περιοχής σε βαθμούς κελσίου (°$C$ ) κατά τη διάρκεια ενός εικοσιτετράωρου δίνεται κατά προσέγγιση από τη συνάρτηση:
\[ f(t)=-8\syn{\frac{\pi t}{12}}+4\ ,\ 0\leq t\leq 24 \]
όπου $t$ ο χρόνος σε ώρες.
\begin{erwthma}
\item Να βρείτε τη μέγιστη και την ελάχιστη θερμοκρασία κατά τη διάρκεια του εικοσιτετράωρου, καθώς και την περίοδο της συνάρτησης.
\item Να παραστήσετε γραφικά την $f$ για $t\in[0,24]$
\item Να βρείτε με τη βοήθεια της γραφικής παράστασης, σε ποια διαστήματα μέσα στη μέρα η θερμοκρασία
αυξάνεται και σε ποια μειώνεται.
\item Να βρείτε τις ώρες στις οποίες η θερμοκρασία ισούται με $ 8\degree C $.
\end{erwthma}
\end{thema}
\end{document}
