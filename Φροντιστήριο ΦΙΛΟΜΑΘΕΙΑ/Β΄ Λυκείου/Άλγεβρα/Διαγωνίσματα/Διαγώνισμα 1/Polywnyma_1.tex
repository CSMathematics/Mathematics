\documentclass[ektypwsh]{article}
\usepackage[amsbb]{mtpro2}
\usepackage[no-math,cm-default]{fontspec}
\usepackage{xunicode}
\usepackage{xltxtra}
\usepackage{xgreek}
\usepackage{amsmath,xcolor}
\usepackage{multicol,mathimatika,fontawesome5}
\defaultfontfeatures{Mapping=tex-text,Scale=MatchLowercase}
\setmainfont[Mapping=tex-text,Numbers=Lining,Scale=1.0,BoldFont={Nimbus Roman Bold}]{Nimbus Roman}
%\newfontfamily\scfont{GFS Artemisia}
%\font\icon = "Webdings"
\usepackage[amsbb]{mtpro2}
\usepackage[left=2.00cm, right=2.00cm, top=2.00cm, bottom=3.00cm]{geometry}
%\xroma{red!80!black}
%\newcommand{\tss}[1]{\textsuperscript{#1}}
%\newcommand{\tssL}[1]{\MakeLowercase\textsuperscript{#1}}
\usepackage{enumitem}
\newlist{rlist}{enumerate}{3}
\setlist[rlist]{itemsep=0mm,label=\roman*.}

\newlist{alist}{enumerate}{3}
\setlist[alist]{itemsep=0mm,label=\alph*.}
\newlist{balist}{enumerate}{3}
\setlist[balist]{itemsep=0mm,label=\bf\alph*.}
\newlist{Alist}{enumerate}{3}
\setlist[Alist]{itemsep=0mm,label=\Alph*.}
\newlist{bAlist}{enumerate}{3}
\setlist[bAlist]{itemsep=0mm,label=\bf\Alph*.}
\newlist{thema}{enumerate}{3}
\setlist[thema]{label=\bf\large{ΘΕΜΑ \textcolor{black}{\Alph*}},itemsep=0mm,leftmargin=0cm,itemindent=18mm}
\newlist{erwthma}{enumerate}{3}
\setlist[erwthma]{label=\bf{\large{\textcolor{black}{\Alph{themai}.\arabic*}}},itemsep=0mm,leftmargin=0.8cm}
\newcommand{\monades}[1]{
\hspace*{\fill}
\textbf{\textit{\textcolor{black}{Μονάδες #1}}}}
\newcommand{\kaliepityxia}{\vfill
\begin{flushright}
Καλή Επιτυχία!
\end{flushright}}

\begin{document}
\begin{center}
\includegraphics[width=0.4\linewidth]{/usr/local/texlive/texmf-local/tex/latex/local/Logotypo-Filomatheia_1}\\
\vspace{-1mm}
{\faIcon{map-marker-alt}} : Ιακώβου Πολυλά 24 - \ Πεζόδρομος\,\,|\,\,{\faIcon{phone-alt}} : 26610 20144\,\,|\,\, {\faIcon{mobile-alt}} : 6932327283 - 6955058444\\
\vspace{-1mm}
\rule{14.7cm}{.1mm}\\
\vspace{2mm}
{\textbf{\today}}\\
\vspace{3mm}
{\Large \textbf{ΔΙΑΓΩΝΙΣΜΑ ΤΥΠΟΥ : B}}
\end{center}
\begin{center}
{\Large\MakeUppercase{\textbf{Άλγεβρα}}}
\vspace{-5mm}
\section*{\huge \textcolor{black}{Πολυώνυμα}}
\vspace{5mm}
\end{center}

%\titlos{Άλγεβρα Β΄ Λυκείου}{Πολυώνυμα}
\begin{thema}
\item\mbox{}\\
Να απαντήσετε στις παρακάτω ερωτήσεις.
\begin{erwthma}
\item Τι ονομάζουμε βαθμό ενός πολυωνύμου $ P(x) $;
\item Ποια συνθήκη πρέπει να ισχύει ώστε δύο μη μηδενικά πολυώνυμα $ A(x),B(x) $ να είναι ίσα;
\item Αν $ A(x),B(x) $ είναι δύο πολυώνυμα με βαθμούς $ \nu,\mu $ αντίστοιχα με $ \nu\geq\mu $, τότε ποιος είναι ο βαθμός του αθροίσματος $ A(x)+B(x) $ και ποιος του γινομένου $ A(x)\cdot B(x) $;
\item Γράψτε 3 προτάσεις οι οποίες να έχουν την ίδια ισχύ με την πρόταση "Το πολυώνυμο $ x-\rho $ είναι παράγοντας του $ A(x) $".
\end{erwthma}\monades{5}
\item \mbox{}\\
Δίνεται το πολυώνυμο $ A(x)=x^3-3x^2+7x-5 $.
\begin{erwthma}
\item Να βρεθεί με τη χρήση σχήματος Horner, το πηλίκο και το υπόλοιπο της διαίρεσης του $ A(x) $ με το πολυώνυμο :
\begin{alist}
\item $ x+3 $\monades{1}
\item $ x-1 $\monades{1}
\end{alist}
\item Να λυθεί η εξίσωση $ A(x)=0 $.\monades{3}
\end{erwthma}
\item\mbox{}\\
Να λυθούν οι παρακάτω εξισώσεις και ανισώσεις.
\begin{erwthma}
\item $ x^3-4x^2+x+6\geq0 $\monades{2}
\item $ \sqrt{x-2}+2=\sqrt{12-x} $\monades{3}
\end{erwthma}
\item \textbf{Σύνθετο θέμα}\\
Δίνεται το πολυώνυμο $ P(x)=(\lambda-1)x^3-5x^2+(\lambda^2-2)x+8 $ όπου $ \lambda\in\mathbb{R} $ είναι μια παράμετρος. 
\begin{erwthma}
\item Να βρεθεί η τιμή της πραγματικής παραμέτρου $ \lambda $ για την οποία το πολυώνυμο έχει παράγοντα το $ x-2 $.\monades{1}
\item Να βρεθούν οι τιμές της μεταβλητής $ x $ για τις οποίες η γραφική παράσταση της συνάρτησης $ P(x) $ βρίσκεται πάνω από τον άξονα $ x'x $.\\\monades{2}
\item Να βρεθούν οι τιμές της παραμέτρου $ \lambda $ για τις οποίες το πολυώνυμο $ P(x) $ αν διαιρεθεί με το $ x-1 $ δίνει υπόλοιπο $ 2 $.\monades{2}
\end{erwthma}
\end{thema}
\kaliepityxia
\end{document}

