\documentclass[ektypwsh]{diag-xelatex}
\usepackage[amsbb]{mtpro2}
\usepackage[no-math,cm-default]{fontspec}
\usepackage{xunicode}
\usepackage{xltxtra}
\usepackage{xgreek}
\usepackage{amsmath}
\defaultfontfeatures{Mapping=tex-text,Scale=MatchLowercase}
\setmainfont[Mapping=tex-text,Numbers=Lining,Scale=1.0,BoldFont={Minion Pro Bold}]{Minion Pro}
\newfontfamily\scfont{GFS Artemisia}
\font\icon = "Webdings"
\usepackage[amsbb]{mtpro2}
\usepackage[left=2.00cm, right=2.00cm, top=2.00cm, bottom=3.00cm]{geometry}
\xroma{red!80!black}
\newcommand{\tss}[1]{\textsuperscript{#1}}
\newcommand{\tssL}[1]{\MakeLowercase\textsuperscript{#1}}
\newlist{rlist}{enumerate}{3}
\setlist[rlist]{itemsep=0mm,label=\roman*.}
\usepackage{multicol}
\usepackage{amsmath}
\usepackage{tikz,pgfplots,tkz-euclide,mathimatika,gensymb}

\begin{document}
\titlos{Άλγεβρα Β΄ Λυκείου}{ΤΡΙΓΩΝΟΜΕΤΡΙΑ}
\begin{thema}
\item \textbf{Θεωρία}\\
Να απαντήσετε στις παρακάτω ερωτήσεις.
\begin{rlist}
\item Πως ορίζεται το συνημίτονο μιας οξείας γωνίας ενός ορθογωνίου τριγώνου;
\item Ποιές είναι οι σχέσεις μεταξύ των τριγωνομετρικών αριθμών δύο γωνιών με διαφορά $ 180^o $;
\item Ποιά τριγωνομετρική ταυτότητα συνδέει άμεσα το ημίτονο και την εφαπτομένη μιας γωνίας $ \omega $;
\item Ποιά είναι τα πρόσημα των τριγωνομετρικών αριθμών μιας γωνίας $ \omega $ σε κάθε τεταρτημόριο; Να απαντήσετε κατασκευάζοντας έναν κατάλληλο πίνακα.
\end{rlist}\monades{5}
\item \textbf{Τριγωνομετρικές εξισώσεις}\\
Να λυθούν οι παρακάτω τριγωνομετρικές εξισώσεις.
\begin{rlist}
\item $ 2\hm^2{x}+3\hm{x}+1=0 $\monades{2}
\item $ \hm{\left( 2x+\frac{\pi}{3}\right) }=\frac{1}{2}\ ,\ x\in[0,\frac{\pi}{2}] $\monades{3}
\end{rlist}
\item \textbf{Τριγωνομετρικές ταυτότητες}\\
Να αποδειχθούν οι παρακάτω τριγωνομετρικές ταυτότητες.
\begin{multicols}{2}
\begin{rlist}
\item $ \dfrac{\textrm{ημ}{x}}{1+\textrm{συν}{x}}-\dfrac{\textrm{ημ}{x}}{1-\textrm{συν}{x}}=\dfrac{2}{\textrm{ημ}{x}} $
\item $ \dfrac{\left( \textrm{εφ}{x}+\textrm{σφ}{x}\right)^2 }{1+\textrm{εφ}^2{x}}=\frac{1+\textrm{εφ}^2{x}}{\textrm{εφ}^2{x}} $
\end{rlist}\monades{5}
\end{multicols}
\item \textbf{Σύνθετο θέμα}\\
Δίνεται μια οξεία γωνία $ x\in\left[0,\frac{\pi}{2}\right]  $.
\begin{rlist}
\item Να δειχθεί οτι ισχύει η σχέση :
\[ \dfrac{\textrm{ημ}^3{\left( \pi-x\right) }-\textrm{συν}^3{\left( -x\right) }}{\textrm{συν}{\left( \frac{\pi}{2}-x\right) }+\textrm{συν}{\left( \pi+x\right) }}-\dfrac{\textrm{ημ}^3{\left( \pi+x\right) }-\textrm{ημ}^3{\left(\frac{\pi}{2}+x\right) }}{\textrm{ημ}{\left( \frac{\pi}{2}+x\right) }+\textrm{ημ}{x}}=2 \]\monades{4}
\item Αν ισχύει $ \textrm{ημ}{x}=\frac{\textrm{συν}{x}}{2} $ να υπολογίσετε τους τριγωνομετρικούς αριθμούς της γωνίας $ x $.\monades{1}
\end{rlist}
\end{thema}
\kaliepityxia
\end{document}

