\documentclass[ektypwsh]{diag-xelatex}
\usepackage[amsbb]{mtpro2}
\usepackage[no-math,cm-default]{fontspec}
\usepackage{xunicode}
\usepackage{xgreek}
\usepackage{amsmath,mathimatika,gensymb,wrap-rl}
\defaultfontfeatures{Mapping=tex-text,Scale=MatchLowercase}
\setmainfont[Mapping=tex-text,Numbers=Lining,Scale=1.0,BoldFont={Minion Pro Bold}]{Minion Pro}
\newfontfamily\scfont{GFS Artemisia}
\font\icon = "Webdings"
\usepackage[amsbb]{mtpro2}
\xroma{red!80!black}
\newcommand{\tss}[1]{\textsuperscript{#1}}
\newcommand{\tssL}[1]{\MakeLowercase\textsuperscript{#1}}
\newlist{rlist}{enumerate}{3}
\setlist[rlist]{itemsep=0mm,label=\roman*.}


\begin{document}
\titlos{Άλγεβρα Β΄ Λυκείου}{Επαναληπτικό}
\begin{thema}
\item \mbox{}\\
\vspace{-5mm}
\begin{enumerate}[label=\Roman*.]
\item Να αποδείξετε ότι το υπόλοιπο της διαίρεσης ενός πολυωνύμου $ P(x) $ με διαιρέτη της μορφής $ x-\rho $ ισούται με την τιμή του πολυωνύμου για $ x=\rho $.\monades{3}
\item \swstolathos
\begin{rlist}
\item Η εκθετική συνάρτηση $ f(x)=a^x\ ,\ 1\neq a>0 $ παίρνει μόνο θετικές τιμές για κάθε $ x\in\mathbb{R} $.
\item Αν για ένα γραμμικό σύστημα ισχύει $ D=0 $ τότε το σύστημα είναι αόριστο.
\item Το μηδενικό πολυώνυμο έχει άπειρες ρίζες.
\item $ \ln{a}\cdot\ln{\beta}=\ln{\left( a+\beta\right) } $ όπου $ a,\beta>0 $.
\item Το υπόλοιπο της διαίρεσης δύο πολυωνύμων πρέπει να είναι μικρότερο από το διαιρέτη.
\end{rlist}\monades{2}
\end{enumerate}
\item \mbox{}\\
Δίνεται το παρακάτω παραμετρικό σύστημα με $ \lambda\in\mathbb{R} $.
\[ \ccases{(\lambda-2)x+\lambda y=4\\3x-(\lambda+2)y=-2} \]
\begin{rlist}
\item Να βρεθούν οι τιμές της παραμέτρου $ \lambda $ για τις οποίες το σύστημα έχει μοναδική λύση η οποία να βρεθεί.\monades{2}
\item Αν $ \lambda=3 $ να βρεθεί η μοναδική λύση του συστήματος.\monades{2}
\item Να βρεθούν οι τιμές της παραμέτρου $ \lambda $ για τις οποίες το σύστημα είναι αδύνατο ή αόριστο και να επίλυθεί το σύστημα σε κάθε περίπτωση.\monades{2}
\end{rlist}
\item \mbox{}\\
Δίνεται το πολυώνυμο $ P(x)=ax^3-2x^2-5x+\beta $. Αν το υπόλοιπο της διαίρεσης του $ P(x) $ με το $ x-2 $ είναι $ -4 $ και το $ x-1 $ είναι παράγοντας του $ P(x) $ τότε
\begin{rlist}
\item Να αποδείξετε ότι $ a=1 $ και $ \beta=6 $.\monades{1}
\item Να λυθει η εξίσωση $ P(x)=0 $.\monades{2}
\item Να λυθεί η ανίσωση $ P(x)\leq0 $.\monades{2}
\end{rlist}
\item \mbox{}\\
Έστω η συνάρτηση $ f $ με τύπο :
\[ f(x)=\ln\frac{x^2-4x+3}{x^3-9x} \]
\begin{rlist}
\item Να βρεθεί το πεδίο ορισμού της συνάρτησης $ f $.\monades{2}
\item Να βρεθεί το σημείο στο οποίο η γραφική παράσταση της συνάρτησης $ f $ τέμνει τον οριζόντιο άξονα $ y'y $.\monades{1}
\item Να βρεθεί το σημείο στο οποίο η γραφική παράσταση της συνάρτησης $ f $ τέμνει τον οριζόντιο άξονα $ x'x $.\monades{2}
\end{rlist}
\end{thema}
\end{document}

