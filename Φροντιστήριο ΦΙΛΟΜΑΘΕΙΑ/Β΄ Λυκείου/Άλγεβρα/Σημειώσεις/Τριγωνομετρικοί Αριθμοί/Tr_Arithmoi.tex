\documentclass[twoside,nofonts,internet,shmeiwseis]{thewria}
\usepackage[amsbb,subscriptcorrection,zswash,mtpcal,mtphrb,mtpfrak]{mtpro2}
\usepackage[no-math,cm-default]{fontspec}
\usepackage{amsmath}
\usepackage{xgreek}
\let\hbar\relax
\defaultfontfeatures{Mapping=tex-text,Scale=MatchLowercase}
\setmainfont[Mapping=tex-text,Numbers=Lining,Scale=1.0,BoldFont={Minion Pro Bold}]{Minion Pro}
\newfontfamily\scfont{GFS Artemisia}
\font\icon = "Webdings"
\usepackage[amsbb]{mtpro2}
\usepackage{tikz,pgfplots}
\usepackage{tkz-euclide}
\usetkzobj{all}
\tkzSetUpPoint[size=7,fill=white]
\xroma{red!70!black}
%------- ΣΥΣΤΗΜΑ -------------------
\usepackage{systeme,regexpatch}
\makeatletter
% change the definition of \sysdelim not to store `\left` and `\right`
\def\sysdelim#1#2{\def\SYS@delim@left{#1}\def\SYS@delim@right{#2}}
\sysdelim\{. % reinitialize

% patch the internal command to use
% \LEFTRIGHT<left delim><right delim>{<system>}
% instead of \left<left delim<system>\right<right delim>
\regexpatchcmd\SYS@systeme@iii
{\cB.\c{SYS@delim@left}(.*)\c{SYS@delim@right}\cE.}
{\c{SYS@MT@LEFTRIGHT}\cB\{\1\cE\}}
{}{}
\def\SYS@MT@LEFTRIGHT{%
\expandafter\expandafter\expandafter\LEFTRIGHT
\expandafter\SYS@delim@left\SYS@delim@right}
\makeatother
\newcommand{\synt}[2]{{\scriptsize \begin{matrix}
\times#1\\\\ \times#2
\end{matrix}}}
%----------------------------------------
%------ ΜΗΚΟΣ ΓΡΑΜΜΗΣ ΚΛΑΣΜΑΤΟΣ ---------
\DeclareRobustCommand{\frac}[3][0pt]{%
{\begingroup\hspace{#1}#2\hspace{#1}\endgroup\over\hspace{#1}#3\hspace{#1}}}
%----------------------------------------
\newcommand{\hm}[1]{\textrm{ημ}#1}
\newcommand{\syn}[1]{\textrm{συν}#1}
\newcommand{\ef}[1]{\textrm{εφ}#1}
\newcommand{\syf}[1]{\textrm{σφ}#1}
\newlist{rlist}{enumerate}{3}
\setlist[rlist]{itemsep=0mm,label=\roman*.}
\newlist{brlist}{enumerate}{3}
\setlist[brlist]{itemsep=0mm,label=\bf\roman*.}
\newlist{tropos}{enumerate}{3}
\setlist[tropos]{label=\bf\textit{\arabic*\textsuperscript{oς}\;Τρόπος :},leftmargin=0cm,itemindent=2.3cm,ref=\bf{\arabic*\textsuperscript{oς}\;Τρόπος}}
\newcommand{\tss}[1]{\textsuperscript{#1}}
\newcommand{\tssL}[1]{\MakeLowercase{\textsuperscript{#1}}}

\usepackage{hhline}
%----------- ΓΡΑΦΙΚΕΣ ΠΑΡΑΣΤΑΣΕΙΣ ---------
\pgfkeys{/pgfplots/aks_on/.style={axis lines=center,
xlabel style={at={(current axis.right of origin)},xshift=1.5ex, anchor=center},
ylabel style={at={(current axis.above origin)},yshift=1.5ex, anchor=center}}}
\pgfkeys{/pgfplots/grafikh parastash/.style={\xrwma,line width=.4mm,samples=200}}
\pgfkeys{/pgfplots/belh ar/.style={tick label style={font=\scriptsize},axis line style={-latex}}}
%-----------------------------------------
\usepackage{multicol}
\usepackage{wrap-rl}
\usepackage{gensymb}
\usepackage{caption}
\tikzstyle{pl}=[line width=0.3mm]
\tikzstyle{plm}=[line width=0.4mm]
\usepackage{array}
\newcommand\diag[5]{%
\multicolumn{1}{|m{#2}|}{\hskip-\tabcolsep
$\vcenter{\begin{tikzpicture}[baseline=0,anchor=south west,outer sep=0]
\path[use as bounding box] (0,0) rectangle (#2+2\tabcolsep,\baselineskip);
\node[minimum width={#2+2\tabcolsep-\pgflinewidth},
minimum  height=\baselineskip+#3-\pgflinewidth] (box) {};
\draw[line cap=round] (box.north west) -- (box.south east);
\node[anchor=south west,align=left,inner sep=#1] at (box.south west) {#4};
\node[anchor=north east,align=right,inner sep=#1] at (box.north east) {#5};
\end{tikzpicture}}\rule{0pt}{.71\baselineskip+#3-\pgflinewidth}$\hskip-\tabcolsep}}
%---------------------------------

\ekthetesdeiktes

\begin{document}
\titlos{Άλγεβρα Β΄ Λυκείου}{Τριγωνομετρία}{Τριγωνομετρικοί αριθμοί}
\orismoi
\Orismos{Τριγωνομετρικοί αριθμοί}
Έστω $ AB\varGamma $ ένα ορθογώνιο τρίγωνο, με $ A=90\degree $ τότε οι τριγωνομετρικοί αριθμοί των οξείων γωνιών του τριγώνου ορίζονται ως εξής :\\
\begin{minipage}{\linewidth}\mbox{}\\
\vspace{-1cm}
\begin{WrapText1}{12}{3.3cm}
\vspace{0mm}
\begin{tikzpicture}[scale=.8]
\tkzDefPoint(0,0){A}
\tkzDefPoint(3,0){B}
\tkzDefPoint(0,4){C}
\tkzMarkAngle[fill=\xrwma!50,size=.5](C,B,A)
\tkzMarkRightAngle[size=.3](B,A,C)
\tkzDrawPolygon[pl](A,B,C)
\tkzText(2.2,.2){$ \omega $}
\tkzLabelPoint[left](A){$ A $}
\tkzLabelPoint[right](B){$ B $}
\tkzLabelPoint[left](C){$ \varGamma $}
\tkzDrawPoints[size=7,fill=white](A,B,C)
\end{tikzpicture}\captionof{figure}{Τριγωνομετρικοί αριθμοι οξείας γωνίας}
\end{WrapText1}
\begin{enumerate}[itemsep=0mm,label=\bf\arabic*.]
\item \textbf{Ημίτονο}\\
Ημίτονο μιας οξέιας γωνίας ενός ορθογωνίου τριγώνου ονομάζεται ο λόγος της απέναντι κάθετης πλευράς προς την υποτείνουσα.
\[ \textrm{Ημίτονο}=\frac{\textrm{Απέναντι Κάθετη}}{\textrm{Υποτείνουσα}}\;\;,\;\;\hm{\omega}=\frac{A\varGamma}{B\varGamma} \]
\item \textbf{Συνημίτονο}\\
Συνημίτονο μιας οξέιας γωνίας ενός ορθογωνίου τριγώνου ονομάζεται ο λόγος της προσκείμενης κάθετης πλευράς προς την υποτείνουσα.
\[ \textrm{Συνημίτονο}=\frac{\textrm{Προσκείμενη Κάθετη}}{\textrm{Υποτείνουσα}}\;\;,\;\;\syn{\omega}=\frac{AB}{B\varGamma} \]
\end{enumerate}

\begin{enumerate}[itemsep=0mm,label=\bf\arabic*.,start=3]
\item \textbf{Εφαπτομένη}\\
Εφαπτομένη μιας οξέιας γωνίας ενός ορθογωνίου τριγώνου ονομάζεται ο λόγος της απέναντι κάθετης πλευράς προς την προσκείμενη κάθετη.
\[ \textrm{Εφαπτομένη}=\frac{\textrm{Απέναντι Κάθετη}}{\textrm{Προσκείμενη Κάθετη}}\;\;,\;\;\ef{\omega}=\frac{A\varGamma}{AB} \]
\item \textbf{Συνεφαπτομένη}\\
Συνεφαπτομένη μιας οξέιας γωνίας ενός ορθογωνίου τριγώνου ονομάζεται ο λόγος της προσκείμενης κάθετης πλευράς προς την απέναντι κάθετη.
\[ \textrm{Συνεφαπτομένη}=\frac{\textrm{Προσκείμενη Κάθετη}}{\textrm{Απέναντι Κάθετη}}\;\;,\;\;\syf{\omega}=\frac{AB}{A\varGamma} \]
Υπάρχουν και επιπλέον δύο τριγωνομετρικοί αριθμοί τους οποίους συναντούμε σπανιότερα από τους άλλους και τους βλέπουμε κυρίως σε εφαρμογές της τριγωνομετρίας στη μηχανική στη ναυσιπλοοία και άλλες επιστήμες.
\end{enumerate}
\end{minipage}\mbox{}\\
\Orismos{τριγ. αρ. γωνιασ σε συστημα συντεταγμενων}
Έστω $ Oxy $ ένα ορθογώνιο σύστημα συντεταγμένων και $ M(x,y) $ ένα σημείο του. Ενώνοντας το σημείο $ M $ με την αρχή των αξόνων, το ευθύγραμμο τμήμα που προκύπτει δημιουργεί μια γωνία $ \omega $ με το θετικό οριζόντιο ημιάξονα $ Ox $.
Το μήκος του ευθύγραμμου τμήματος $ OM $ είναι :
\[ OM=\rho=\sqrt{x^2+y^2} \]
Οι τριγωνομετρικοί αριθμοί της γωνίας $ x\hat{O}y $ ορίζονται με τη βοήθεια των συντεταγμένων του σημείου και είναι\\
\begin{minipage}{\linewidth}\mbox{}\\
\vspace{-1cm}
\begin{WrapText1}{14}{4.7cm}
\vspace{.5cm}
\begin{tikzpicture}[y=.8cm,x=.9cm]
\draw[draw=black,fill=\xrwma!50] (0,0) -- (.5,0) arc (0:40:.5) -- cycle;
\draw[-latex]  (-.4,0)  -- coordinate (x axis mid) (4,0) node[right,fill=white] {{\footnotesize $ x $}};
\draw[-latex] (0,-.4) -- (0,3.5) node[above,fill=white] {{\footnotesize $ y $}};
\draw (3,.1) -- (3,-.1) node[anchor=north] {\scriptsize $ x $};
\draw (.1,2.5) -- (-.1,2.5) node[anchor=east] {\scriptsize $ y $};
\draw[dashed] (3,0) -- (3,2.5) -- (0,2.5);
\tkzDefPoint(0,0){O}
\tkzDefPoint(3,2.5){M}
\tkzDefPoint(3,0){A}
\tkzDefPoint(0,2.5){B}
\tkzDrawSegment(O,M)
\tkzDrawPoint[size=7,fill=white](M)
\tkzDrawPoint[size=7,fill=white](A)
\tkzDrawPoint[size=7,fill=white](B)
\tkzLabelPoint[below left](O){$ O $}
\tkzLabelPoint[above](M){{\footnotesize $ M(x,y) $}}
\tkzLabelPoint[above right](A){{\footnotesize $ A(x,0) $}}
\tkzLabelPoint[above right](B){{\footnotesize $ B(0,y) $}}
\tkzText(1.5,-.4){$ \undercbrace{\rule{23mm}{0mm}}_{{\scriptsize x}} $}
\tkzText(-.3,1.25){{{\scriptsize $ y $}}$\LEFTRIGHT\{.{ \rule{0pt}{20mm} } $}
\tkzText[fill=white,inner sep=.2mm](2.7,1){{\footnotesize $ \rho=\sqrt{x^2+y^2} $}}
\tkzText(.7,.2){{\footnotesize $ \omega $}}
\end{tikzpicture}\captionof{figure}{Τριγωνομετρικοί αριθμοί σε σύστημα συντεταγμένων.}\end{WrapText1}
\begin{enumerate}[itemsep=0mm,label=\bf\arabic*.]
\item \textbf{Ημίτονο}\\
Ημίτονο της γωνίας  ονομάζεται ο λόγος της τεταγμένης του σημείου προς την απόσταση του από την αρχή των αξόνων.
\[ \hm{\omega}=\frac{AM}{OM}=\frac{y}{\rho} \]
\item \textbf{Συνημίτονο}\\
Συνημίτονο της γωνίας  ονομάζεται ο λόγος της τετμημένης του σημείου προς την απόσταση του από την αρχή των αξόνων.
\[ \syn{\omega}=\frac{BM}{OM}=\frac{x}{\rho} \]
\end{enumerate}\end{minipage}

\begin{enumerate}[itemsep=0mm,label=\bf\arabic*.,start=3]
\item \textbf{Εφαπτομένη}\\
Εφαπτομένη της γωνίας ονομάζεται ο λόγος της τεταγμένης του σημείου προς την τετμημένη του.
\[ \ef{\omega}=\frac{AM}{BM}=\frac{y}{x}\;\;,\;\;x\neq0 \]
\item \textbf{Συνεφαπτομένη}\\
Συνεφαπτομένη της γωνίας  ονομάζεται ο λόγος της τετμημένης του σημείου προς την τεταγμένη του.
\[ \syf{\omega}=\frac{BM}{AM}=\frac{x}{y}\;\;.\;\;y\neq0 \]
\end{enumerate}
\Orismos{μοναδεσ μετρησησ γωνιων - τοξων}
Μονάδες μέτρησης γωνιών - τόξων λέγονται οι γωνίες ή τα τόξα αντίστοιχα με τα οποία μετράμε το μέτρο (άνοιγμα) των πλευρών μιας γωνίας ή αντίστοιχα το μέτρο ενός τόξου.
Οι βασικές μονάδες μέτρησης για τη μέτρηση γωνιών ή τόξων είναι :
\begin{enumerate}[itemsep=0mm,label=\bf\arabic*.]
\item \textbf{Μοίρα}\\
Μοίρα ονομάζεται το τόξο το οποίο είναι ίσο με το $ \frac{1}{360} $ του τόξου ενός κύκλου.
Εναλλακτικά μπορούμε να ορίσουμε τη μοίρα ως τη γωνία η οποία αν γίνει επίκεντρη σε κύκλο, βαίνει σε τόξο ίσο με το $ \frac{1}{360} $ του τόξου του κύκλου.
\begin{itemize}[itemsep=0mm]
\item Συμβολίζεται με $ 1\degree $.
\item Μια μοίρα υποδιαιρείται σε 60 πρώτα λεπτά $ (60') $ και κάθε λεπτό σε 60 δεύτερα λεπτά $ (60'') $.
\end{itemize}
\item \textbf{Ακτίνιο}\\
Ακτίνιο ονομάζεται το τόξο ενός κύκλου του οποίου το μήκος είναι ίσο με την ακτίνα του κύκλου. Ορίζεται και ως η γωνία που αν γίνει επίκεντρη, βαίνει σε τόξο με μήκος ίσο με την ακτίνα του κύκλου.
Συμβολίζεται με $ 1rad $.
\end{enumerate}
%Θεωρηματα----Αν $ \mu $ είναι το μέτρο μιας γωνίας σε μοίρες και $ a $ το μέτρο της ίδιας γωνίας σε ακτίνια, η σχέση που τα συνδέει και με την οποία μπορούμε να μετατρέψουμε το μέτρο μιας γωνίας από μοίρες σε ακτίνια και αντίστροφα είναι :
%\[ \frac{\mu}{180\degree}=\frac{a}{\pi} \]
Στον παρακάτω πίνακα βλέπουμε το μέτρο μερικών βασικών γωνιών δοσμένο σε μοίρες και ακτίνια αλλά και τους τριγωνομετρικούς αριθμούς των γωνιών αυτών.
\begin{center}
\begin{tabular}{c||>{\centering\arraybackslash}m{.8cm}>{\centering\arraybackslash}m{.8cm}>{\centering\arraybackslash}m{.8cm}>{\centering\arraybackslash}m{.8cm}>{\centering\arraybackslash}m{.8cm}>{\centering\arraybackslash}m{.8cm}>{\centering\arraybackslash}m{.8cm}>{\centering\arraybackslash}m{.8cm}>{\centering\arraybackslash}m{.8cm}}
\hline  \multicolumn{10}{c}{\textbf{Βασικές Γωνίες}} \rule[-2ex]{0pt}{5.5ex}  \\ 
\hhline{==========} \rule[-2ex]{0pt}{5.5ex} \textbf{Μοίρες} & $ 0\degree $ & $ 30\degree $ & $ 45\degree $ & $ 60\degree $ & $ 90\degree $ & $ 120\degree $ & $ 135\degree $ & $ 150\degree $ & $ 180\degree $ \\ 
\rule[-2ex]{0pt}{4ex} \textbf{Ακτίνια} & $ 0 $ & $ \frac{\pi}{6} $ & $ \frac{\pi}{4} $ & $ \frac{\pi}{3} $ & $ \frac{\pi}{2} $ & $ \frac{2\pi}{3} $ & $ \frac{3\pi}{4} $ & $ \frac{5\pi}{6} $ & $ \pi $ \\ 
\hline \rule[-2ex]{0pt}{5.5ex} \textbf{Σχήμα} & \begin{tikzpicture}
\fill[fill=\xrwma!50] (0,0) -- (.3,0) arc (0:0:.3) -- cycle;
\draw (-.35,0) -- (.35,0);
\draw (0,-.35) -- (0,.35);
\draw (0,0) circle (.3);
\coordinate (A) at (0:.3);
\draw (0,0) -- (A);
\end{tikzpicture} & \begin{tikzpicture}
\fill[fill=\xrwma!50] (0,0) -- (.3,0) arc (0:30:.3) -- cycle;
\draw (-.35,0) -- (.35,0);
\draw (0,-.35) -- (0,.35);
\draw (0,0) circle (.3);
\coordinate (A) at (30:.3);
\draw (0,0) -- (A);
\end{tikzpicture} & \begin{tikzpicture}
\fill[fill=\xrwma!50] (0,0) -- (.3,0) arc (0:45:.3) -- cycle;
\draw (-.35,0) -- (.35,0);
\draw (0,-.35) -- (0,.35);
\draw (0,0) circle (.3);
\coordinate (A) at (45:.3);
\draw (0,0) -- (A);
\end{tikzpicture} & \begin{tikzpicture}
\fill[fill=\xrwma!50] (0,0) -- (.3,0) arc (0:60:.3) -- cycle;
\draw (-.35,0) -- (.35,0);
\draw (0,-.35) -- (0,.35);
\draw (0,0) circle (.3);
\coordinate (A) at (60:.3);
\draw (0,0) -- (A);
\end{tikzpicture} & \begin{tikzpicture}
\fill[fill=\xrwma!50] (0,0) -- (.3,0) arc (0:90:.3) -- cycle;
\draw (-.35,0) -- (.35,0);
\draw (0,-.35) -- (0,.35);
\draw (0,0) circle (.3);
\coordinate (A) at (90:.3);
\draw (0,0) -- (A);
\end{tikzpicture} & \begin{tikzpicture}
\fill[fill=\xrwma!50] (0,0) -- (.3,0) arc (0:120:.3) -- cycle;
\draw (-.35,0) -- (.35,0);
\draw (0,-.35) -- (0,.35);
\draw (0,0) circle (.3);
\coordinate (A) at (120:.3);
\draw (0,0) -- (A);
\end{tikzpicture} & \begin{tikzpicture}
\fill[fill=\xrwma!50] (0,0) -- (.3,0) arc (0:135:.3) -- cycle;
\draw (-.35,0) -- (.35,0);
\draw (0,-.35) -- (0,.35);
\draw (0,0) circle (.3);
\coordinate (A) at (135:.3);
\draw (0,0) -- (A);
\end{tikzpicture} & \begin{tikzpicture}
\fill[fill=\xrwma!50] (0,0) -- (.3,0) arc (0:150:.3) -- cycle;
\draw (-.35,0) -- (.35,0);
\draw (0,-.35) -- (0,.35);
\draw (0,0) circle (.3);
\coordinate (A) at (150:.3);
\draw (0,0) -- (A);
\end{tikzpicture} & \begin{tikzpicture}
\fill[fill=\xrwma!50] (0,0) -- (.3,0) arc (0:180:.3) -- cycle;
\draw (-.35,0) -- (.35,0);
\draw (0,-.35) -- (0,.35);
\draw (0,0) circle (.3);
\coordinate (A) at (180:.3);
\draw (0,0) -- (A);
\end{tikzpicture} \\ 
\hline \rule[-2ex]{0pt}{5ex} $ \hm{\omega} $ & $ 0 $ & $ \frac{1}{2} $ & $ \frac{\sqrt{2}}{2} $ & $ \frac{\sqrt{3}}{2} $ & $ 1 $ & $ \frac{\sqrt{3}}{2} $ & $ \frac{\sqrt{2}}{2} $ & $ \frac{1}{2} $ & $ 0 $ \\ 
\rule[-2ex]{0pt}{4ex} $ \syn{\omega} $ & $ 1 $ & $ \frac{\sqrt{3}}{2} $ & $ \frac{\sqrt{2}}{2} $ & $ \frac{1}{2} $ & $ 0 $ & $ -\frac{1}{2} $ & $ -\frac{\sqrt{2}}{2} $ & $ -\frac{\sqrt{3}}{2} $ & $ -1 $ \\ 
\rule[-2ex]{0pt}{4ex} $ \ef{\omega} $ & $ 0 $ & $ \frac{\sqrt{3}}{3} $ & $ 1 $ & $ \sqrt{3} $ & \begin{minipage}{.8cm}
\begin{center}
{\scriptsize Δεν\\\vspace{-1mm}ορίζεται}
\end{center}
\end{minipage} & $ -\sqrt{3} $ & $ -1 $ & $ -\frac{\sqrt{3}}{3} $ & $ 0 $ \\
\rule[-2ex]{0pt}{4ex} $ \syf{\omega} $ & \begin{minipage}{.8cm}
\begin{center}
{\scriptsize Δεν\\\vspace{-1mm}ορίζεται}
\end{center}
\end{minipage} & $ \sqrt{3} $ & $ 1 $ & $ \frac{\sqrt{3}}{3} $ & $ 0 $ & $ -\frac{\sqrt{3}}{3} $ & $ -1 $ & $ -\sqrt{3} $ & \begin{minipage}{.8cm}
\begin{center}
{\scriptsize Δεν\\\vspace{-1mm}ορίζεται}
\end{center}
\end{minipage} \\ 
\hline 
\end{tabular}\captionof{table}{Τριγωνομετρικοί αριθμοί βασικών γωνιών}
\end{center}
\Orismos{τριγωνομετρικοσ κυκλοσ}
Τριγωνομετρικός κύκλος ονομάζεται ο κύκλος με ακτίνα ίση με τη μονάδα και κέντρο την αρχή των αξόνων ενός ορθογωνίου συστήματος συντεταγμένων, στους άξονες του οποίου παίρνουν τιμές οι τριγωνομετρικοί αριθμοί των γωνιών.
\begin{center}
\begin{tabular}{p{6.5cm}p{6.5cm}}
\multicolumn{2}{c}{{\Large \textbf{Τριγωνομετρικός Κύκλος}}}\\
\begin{tikzpicture}[>=latex,scale=2]
\fill[fill=\xrwma!50] (0,0) -- (.2,0) arc (0:60:.2) -- cycle;
%axis
\draw[->] (-1.2,0) -- coordinate (x axis mid) (1.5,0) node[right,fill=white] {{\footnotesize $ x $}};
\foreach \x in {-1,-0.8,-0.6,-0.4,-0.2,0,0.2,0.4,0.6,0.8,1}
\draw (\x,.5pt) -- (\x,-.5pt)
node[anchor=north] {{\tiny \x}};
\foreach \y in {-1,-0.8,-0.6,-0.4,-0.2,0,0.2,0.4,0.6,0.8,1}
\draw (.5pt,\y) -- (-.5pt,\y)
node[anchor=east] {{\tiny \y}};
\draw[->] (0,-1.2) -- (0,1.5) node[above,fill=white] {{\footnotesize $ y $}};
\draw[-] (1,-1.2) -- (1,1.8);
\draw[-] (-1.2,1) -- (1.2,1);
\draw[-,thick] (0,1) -- (1.732/3,1);
\draw[-,thick] (1,0) -- (1,1.732);
\draw[-,dashed] (-.7,-1.732*0.7) -- (1,1.732);
\draw circle (1);
\coordinate (A) at (60:1);
\tkzDefPoint(0,0){O}
\tkzDefPoint(cos(pi/3),0){B}
\tkzDefPoint(0,sin(pi/3)){C}
\tkzDefPoint(1,tan(pi/3)){D}
\tkzDefPoint(cot(pi/3),1){E}
\tkzDefPoint(1,0){F}
\tkzDefPoint(0,1){G}
\tkzDrawSegment(O,A)
\tkzDrawSegments[thin,dashed](A,B A,C)
\tkzText(0,1.75){{\scriptsize Άξονας Ημιτόνων}}
\tkzText(1.6,-.12){{\scriptsize Άξονας}}
\tkzText(1.6,-.23){{\scriptsize Συνημιτόνων}}
\tkzText(-1,1.22){{\scriptsize Άξονας}}
\tkzText(-.75,1.1){{\scriptsize Συνεφαπτομένων}}
\tkzText(1.23,-.87){{\scriptsize Άξονας}}
\tkzText(1.42,-1){{\scriptsize Εφαπτομένων}}
\tkzText(-.5,-1.1){{\scriptsize $ \delta $}}
\tkzDrawSegment[thick](O,B)
\tkzDrawSegment[thick](O,C)
\tkzDrawPoints[size=7,fill=white](O,A,B,C,D,E,F,G)
\tkzText(-.4,.43){{{\scriptsize \textrm{ημ}$ \omega $}}$\LEFTRIGHT\{.{ \rule{0pt}{18mm} } $}
\tkzText(.25,-.25){$ \undercbrace{\rule{9mm}{0mm}}_{{\scriptsize \textrm{συν}\omega}} $}
\tkzText(1.2,.87){$\LEFTRIGHT.\}{ \rule{0pt}{33mm} } ${{\scriptsize \textrm{εφ}$ \omega $}}}
\tkzText(.3,1.12){$ \overcbrace{\rule{10mm}{0mm}}^{{\scriptsize \textrm{σφ}\omega}} $}
\tkzText(.25,.15){$ \omega $}
\tkzLabelPoint[below left](O){{\tiny $ O $}}
\tkzLabelPoint[above=1mm,right](A){{\tiny $ M $}}
\tkzLabelPoint[above right](B){{\tiny $ M_1 $}}
\tkzLabelPoint[above=1mm, left](C){{\tiny $ M_2 $}}
\tkzLabelPoint[left](D){{\tiny $ K $}}
\tkzLabelPoint[above](E){{\tiny $ \varLambda $}}
\tkzLabelPoint[below right](F){{\tiny $ A $}}
\tkzLabelPoint[above left](G){{\tiny $ B $}}
\draw [->] (.984*.9,.173*.9) arc (10:45:.9);
\draw [->] (.984*.9,-.173*.9) arc (-10:-45:.9);
\tkzText(.72,.35){$ + $}
\tkzText(.72,-.35){$ - $}
\tkzText(.35,.45){$ \rho $}
\tkzText(-1,.9){{\scriptsize $ \varepsilon_2 $}}
\tkzText(.9,-1){{\scriptsize $ \varepsilon_1 $}}
\end{tikzpicture}\captionof{figure}{Τριγωνομετρικός κύκλος} & \begin{tikzpicture}[>=latex,scale=2]
\fill[fill=\xrwma!50] (0,0) -- (.2,0) arc (0:45:.2) -- cycle;
%axis
\draw[->] (-1.2,0) -- (1.5,0) node[right,fill=white] {{\footnotesize $ x $}};
\draw[->] (0,-1.2) -- (0,1.5) node[above,fill=white] {{\footnotesize $ y $}};

\foreach \gwnia/\xtext in {
30/\frac{\pi}{6},
45/\frac{\pi}{4},
60/\frac{\pi}{3},
90/\frac{\pi}{2},
120/\frac{2\pi}{3},
135/\frac{3\pi}{4},
150/\frac{5\pi}{6},
180/\pi,
210/\frac{7\pi}{6},
240/\frac{4\pi}{3},
270/\frac{3\pi}{2},
300/\frac{5\pi}{3},
330/\frac{11\pi}{6},
360/2\pi}
\draw (\gwnia:0.85cm) node {{\scriptsize $\xtext$}};
\foreach \gwnia/\xtext in {
90/\frac{\pi}{2},
180/\pi,
270/\frac{3\pi}{2},
360/2\pi}
\draw (\gwnia:0.85cm) node[fill=white] {{\scriptsize $\xtext$}};
\tkzDefPoint(0,0){O}
\coordinate (A) at (45:1);
\tkzDrawSegment(O,A)
\draw circle (1);
\foreach \gwnia in {0,30,45,60,90,120,135,150,180,210,240,270,300,330}{
\coordinate (P) at (\gwnia:1);
\draw (\gwnia:1.22cm) node[fill=white,inner sep=0.1mm] {{\scriptsize $\gwnia^\circ$}};
\draw[draw=black,fill=white] (P) circle (.7pt);};
\tkzText(.25,.1){$ \omega $}
\end{tikzpicture}\captionof{figure}{Βασικές γωνίες}
\end{tabular}
\end{center}
\begin{itemize}[itemsep=0mm]
\item Κάθε γωνία $ \omega $ έχει πλευρές, τον θετικό ημιάξονα $ Ox $ και την ακτίνα $ \rho $ του κύκλου, μετρώντας τη γωνία αυτή αριστερόστροφα, φορά που ορίζεται ως \textbf{θετική}.
\item Ο οριζόντιος άξονας $ x'x $ είναι ο άξονας συνημιτόνων ενώ ο κατακόρυφος $ y'y $ ο άξονας ημιτόνων.
\item Κάθε σημείο $ M $ του κύκλου έχει συντεταγμένες $ M(\syn{\omega},\hm{\omega}) $.
\item Η τετμημέμη του σημείου είναι ίση με το συνημίτονο της γωνίας, ενώ η τεταγμένη ίση με το ημίτονο της.
\[ x=\syn{\omega}\;\;,\;\;y=\hm{\omega} \]
\item Η εφαπτόμενη ευθεία στον κύκλο στο σημείο $ A(1,0) $ είναι ο \textbf{άξονας των εφαπτομένων}. Η εφαπτομένη της γωνίας $ \omega $ είναι η τεταγμένη του σημείου τομής της ευθείας $ \varepsilon_1 $ με το φορέα $ \delta $ της ακτίνας.
\[ y_{\!_K}=\ef{\omega} \]
\item Η εφαπτόμενη ευθεία στον κύκλο στο σημείο $ B(0,1) $ είναι ο \textbf{άξονας των συνεφαπτομένων}. Η συνεφαπτομένη της γωνίας $ \omega $ είναι η τετμημένη του σημείου τομής της ευθείας $ \varepsilon_2 $ με το φορέα $ \delta $ της ακτίνας.
\[ x_{\!_K}=\syf{\omega} \]	
\end{itemize}
Πιο κάτω βλέπουμε τα τέσσερα τεταρτημόρια στα οποία χωρίσουν οι άξονες το επίπεδο και τον τριγωνομετρικό κύκλο καθώς και το πρόσημο των τριγωνομετρικών αριθμών των γωνιών σε κάθε τεταρτημόριο.
\begin{center}
\begin{minipage}[m]{6cm}
\centering
\begin{tabular}{c|cccc}
\hline \textbf{Τεταρτημ./Τρ. Αριθμός} & \textbf{{$ \mathbold{\hm{\omega}} $}} & \textbf{{$ \mathbold{\syn{\omega}} $}} & \textbf{{$ \mathbold{\ef{\omega}} $}} & \textbf{{$ \mathbold{\syf{\omega}} $}} \rule[-2ex]{0pt}{5ex}\\ 
\hhline{=====} \rule[-2ex]{0pt}{5ex} \textbf{1\tss{o} Τεταρτημόριο} & $ + $ & $ + $ & $ + $ & $ + $ \\ 
\rule[-2ex]{0pt}{5ex} \textbf{2\tss{o} Τεταρτημόριο} & $ + $ & $ - $ & $ - $ & $ - $ \\ 
\rule[-2ex]{0pt}{5ex} \textbf{3\tss{o} Τεταρτημόριο} & $ - $ & $ - $ & $ + $ & $ + $ \\ 
\rule[-2ex]{0pt}{5ex} \textbf{4\tss{o} Τεταρτημόριο} & $ - $ & $ + $ & $ - $ & $ - $ \\ 
\hline 
\end{tabular}\captionof{table}{Πρόσημα τριγωνομετρικών αριθμών}
\end{minipage}\hspace{4cm}
\begin{minipage}[m]{5cm}
\centering
\begin{tikzpicture}[scale=1.8]
\draw[-latex] (-1.2,0) -- (1.2,0) node[right,fill=white] {{\footnotesize $ x $}};
\draw[-latex] (0,-1.2) -- (0,1.2) node[above,fill=white] {{\footnotesize $ y $}};
\tkzDefPoint(0,0){O}
\draw circle (1);
\tkzLabelPoint[below left,xshift=.5mm,yshift=.5mm](O){$ O $}
\node at (0.4,0.5) {{\scriptsize 1\textsuperscript{o} Τετ.}};
\node at (0.4,-0.5) {{\scriptsize 4\textsuperscript{o} Τετ.}};
\node at (-0.4,-0.5) {{\scriptsize 3\textsuperscript{o} Τετ.}};
\node at (-0.4,0.5) {{\scriptsize 2\textsuperscript{o} Τετ.}};
\node at (0.4,0.3) {{\scriptsize $ (+,+) $}};
\node at (-0.4,0.3) {{\scriptsize $ (-,+) $}};
\node at (-0.4,-0.3) {{\scriptsize $ (-,-) $}};
\node at (0.4,-0.3) {{\scriptsize $ (+,-) $}};
\end{tikzpicture}\captionof{figure}{Τεταρτημόρια και πρόσημα τεταρτημορίων τριγωνομετρικού κύκλου.}
\end{minipage}
\end{center}
\thewrhmata
\Thewrhma{Όρια τριγωνομετρικών αριθμών}
To ημίτονο και το συνημίτονο οποιασδήποτε γωνίας $ \omega $ παίρνει τιμές από $-1$ μέχρι $ 1 $. Οι παρακάτω σχέσεις είναι ισοδύναμες :
\begin{multicols}{2}
\begin{rlist}
\item  $ -1\leq\hm{\omega}\leq1\;\;,\;\;-1\leq\syn{\omega}\leq1 $
\item $ |\hm{\omega}|\leq1\ ,\ |\syn{\omega}|\leq1 $
\end{rlist}
\end{multicols}
\Thewrhma{Τρ. Αριθμοί γωνιών μεγαλύτερων του κύκλου}
Οι τριγωνομετρικοί αριθμοί μιας γωνίας $ \omega $ της οποίας το μέτρο είναι μικρότερο του ενός κύκλου είναι ίσοι με τους τριγωνομετρικούς αριθμούς της γωνίας που θα προκύψει εαν στρέψουμε την $ \omega $ κατά πολλαπλάσια του κύκλου.
 \begin{center}
$ \begin{array}{cc}
\hm{\left( 360\degree\cdot \kappa+\omega\right) }=\hm{\omega} & \syn{\left( 360\degree\cdot \kappa+\omega\right) }=\syn{\omega} \\ 
\ef{\left( 360\degree\cdot \kappa+\omega\right) }=\ef{\omega} & \syf{\left( 360\degree\cdot \kappa+\omega\right) }=\syf{\omega}
\end{array} $
\end{center}
ή ισοδύναμα με τη βοήθεια ακτινίων
\begin{center}
$ \begin{array}{cc}
\hm{\left( 2\kappa\pi+\omega\right) }=\hm{\omega} & \syn{\left( 2\kappa\pi+\omega\right) }=\syn{\omega} \\ 
\ef{\left( 2\kappa\pi+\omega\right) }=\ef{\omega} & \syf{\left( 2\kappa\pi+\omega\right) }=\syf{\omega}
\end{array} $
\end{center}
\Thewrhma{Μετατροπή μοιρών σε ακτίνια}
Αν $ \mu $ είναι το μέτρο μιας γωνίας σε μοίρες και $ a $ το μέτρο της ίδιας γωνίας σε ακτίνια, η σχέση που τα συνδέει και με την οποία μπορούμε να μετατρέψουμε το μέτρο μιας γωνίας από μοίρες σε ακτίνια και αντίστροφα είναι :
\[ \frac{\mu}{180\degree}=\frac{a}{\pi} \]
\end{document}
