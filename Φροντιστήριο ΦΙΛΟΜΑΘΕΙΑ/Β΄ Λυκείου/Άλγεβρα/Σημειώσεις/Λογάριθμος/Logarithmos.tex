\documentclass[twoside,nofonts,internet,shmeiwseis]{thewria}
\usepackage[amsbb,subscriptcorrection,zswash,mtpcal,mtphrb,mtpfrak]{mtpro2}
\usepackage[no-math,cm-default]{fontspec}
\usepackage{amsmath}
\usepackage{xunicode}
\usepackage{xgreek}
\let\hbar\relax
\defaultfontfeatures{Mapping=tex-text,Scale=MatchLowercase}
\setmainfont[Mapping=tex-text,Numbers=Lining,Scale=1.0,BoldFont={Minion Pro Bold}]{Minion Pro}
\newfontfamily\scfont{GFS Artemisia}
\font\icon = "Webdings"
\usepackage{fontawesome}
\newfontfamily{\FA}{fontawesome.otf}
\usepackage[amsbb]{mtpro2}
\usepackage{tikz,pgfplots,tkz-euclide}
\usetkzobj{all}
\tkzSetUpPoint[size=7,fill=white]
\xroma{red!70!black}

\newlist{rlist}{enumerate}{3}
\setlist[rlist]{itemsep=0mm,label=\roman*.}
\newlist{brlist}{enumerate}{3}
\setlist[brlist]{itemsep=0mm,label=\bf\roman*.}
\newlist{tropos}{enumerate}{3}
\setlist[tropos]{label=\bf\textit{\arabic*\textsuperscript{oς}\;Τρόπος :},leftmargin=0cm,itemindent=2.3cm,ref=\bf{\arabic*\textsuperscript{oς}\;Τρόπος}}
\newcommand{\tss}[1]{\textsuperscript{#1}}
\newcommand{\tssL}[1]{\MakeLowercase{\textsuperscript{#1}}}

\usepackage{hhline}

\usepackage{multicol}
\usepackage{wrap-rl}
\usepackage{gensymb,mathimatika,longtable}



\begin{document}
\titlos{Άλγεβρα Β΄ Λυκείου}{Εκθετική και Λογαριθμική Συνάρτηση}{Λογάριθμοι}
\orismoi
\Orismos{Λογάριθμοσ}
Λογάριθμος με βάση ένα θετικό αριθμό $ a\neq1 $ ενός θετικού αριθμού $ \beta $ ονομάζεται ο εκθέτης στον οποίο θα υψωθεί ο αριθμός $ a $ ώστε να δώσει τον αριθμό $ \beta $. Συμβολίζεται :
\[ \log_{a}{\beta} \]
με $ 0<a\neq1\;\textrm{και}\; \beta>0 $.
\begin{itemize}[itemsep=0mm]
\item Ο αριθμός $ a $ ονομάζεται \textbf{βάση του λογαρίθμου}.
\item Ο αριθμός $ \beta $ έχει το ρόλο του αποτελέσματος της δύναμης με βάση $ a $, ενώ ολόκληρος ο λογάριθμος, το ρόλο του εκθέτη.
\item Αν ο λογάριθμος (εκθέτης) με βάση $ a $ του $ \beta $ είναι ίσος με $ x $ τότε θα ισχύει :
\[ \log_{a}{\beta}=x\Leftrightarrow a^x=\beta \]
\item Εαν η βάση ενός λογαρίθμου είναι ο αριθμός $ 10 $ τότε ο λογάριθμος ονομάζεται \textbf{δεκαδικός λογάριθμος} και συμβολίζεται : $ \log{x} $.
\item Εαν η βάση του λογαρίθμου είναι ο αριθμός $ e $ τότε ο λογάριθμος ονομάζεται \textbf{φυσικός λογάριθμος} και συμβολίζεται : $ \ln{x} $.
\end{itemize}
\thewrhmata
\Thewrhma{Ιδιότητεσ Λογαρίθμων}
Για οπουσδήποτε θετικούς πραγματικούς αριθμούς $ x,y\in\mathbb{R}^+ $ έχουμε τις ακόλουθες ιδιότητες που αφορούν το λογάριθμο τους με βάση έναν θετικό πραγματικό αριθμό $ a $.
\begin{center}
\begin{longtable}{cc}
\hline \rule[-2ex]{0pt}{5.5ex} \textbf{Ιδιότητα} & \textbf{Συνθήκη} \\
\hhline{==}\rule[-2ex]{0pt}{5.5ex} Λογάριθμος γινομένου & $ \log_{a}(x\cdot y)=\log_{a}x+\log_{a}y $ \\
\rule[-2ex]{0pt}{5.5ex}  Λογάριθμος πηλίκου & $ \log_{a}\left( \dfrac{x}{y}\right) =\log_{a}x-\log_{a}y $ \\
\rule[-2ex]{0pt}{5.5ex}  Λογάριθμος δύναμης & $ \log_{a}x^\kappa=\kappa\cdot\log_{a}x\;\;,\;\;\kappa\in\mathbb{Z} $ \\
\rule[-2ex]{0pt}{5.5ex}  Λογάριθμος ρίζας & $ \log_{a}\!\sqrt[\nu]{x}=\dfrac{1}{\nu}\log_{a}x\;\;,\;\;\nu\in\mathbb{N} $ \\
\rule[-2ex]{0pt}{5.5ex}  Λογάριθμος ως εκθέτης & $ a^{\log_{a}x}=x $ \\
\rule[-2ex]{0pt}{5.5ex}  Λογάριθμος δύναμης με κοινή βάση & $ \log_{a}a^x=x $ \\
\rule[-2ex]{0pt}{5.5ex}  Αλλαγή βάσης & $ \log_{a}x=\dfrac{\log_{\beta}{x}}{\log_{\beta}{a}} $ \\
\hline
\end{longtable}
\end{center}
Επίσης για κάθε λογάριθμο με οποιαδήποτε βάση $ a\in\mathbb{R}^+ $ έχουμε :
\begin{multicols}{2}
\begin{rlist}
\item $ \log_{a}1=0 $
\item $ \log_{a}a=1 $
\end{rlist}
\end{multicols}
\end{document}
