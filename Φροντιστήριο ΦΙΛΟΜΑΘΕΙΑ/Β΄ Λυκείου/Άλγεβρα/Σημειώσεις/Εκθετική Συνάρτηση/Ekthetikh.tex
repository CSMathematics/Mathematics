\documentclass[twoside,nofonts,internet,shmeiwseis]{thewria}
\usepackage[amsbb,subscriptcorrection,zswash,mtpcal,mtphrb,mtpfrak]{mtpro2}
\usepackage[no-math,cm-default]{fontspec}
\usepackage{amsmath}
\usepackage{xunicode}
\usepackage{xgreek}
\let\hbar\relax
\defaultfontfeatures{Mapping=tex-text,Scale=MatchLowercase}
\setmainfont[Mapping=tex-text,Numbers=Lining,Scale=1.0,BoldFont={Minion Pro Bold}]{Minion Pro}
\newfontfamily\scfont{GFS Artemisia}
\font\icon = "Webdings"
\usepackage{fontawesome}
\newfontfamily{\FA}{fontawesome.otf}
\usepackage[amsbb]{mtpro2}
\usepackage{tikz,pgfplots,tkz-euclide}
\usetkzobj{all}
\tkzSetUpPoint[size=7,fill=white]
\xroma{red!70!black}

\newlist{rlist}{enumerate}{3}
\setlist[rlist]{itemsep=0mm,label=\roman*.}
\newlist{brlist}{enumerate}{3}
\setlist[brlist]{itemsep=0mm,label=\bf\roman*.}
\newlist{tropos}{enumerate}{3}
\setlist[tropos]{label=\bf\textit{\arabic*\textsuperscript{oς}\;Τρόπος :},leftmargin=0cm,itemindent=2.3cm,ref=\bf{\arabic*\textsuperscript{oς}\;Τρόπος}}
\newcommand{\tss}[1]{\textsuperscript{#1}}
\newcommand{\tssL}[1]{\MakeLowercase{\textsuperscript{#1}}}

\usepackage{hhline}

\usepackage{multicol}
\usepackage{wrap-rl}
\usepackage{gensymb,mathimatika}



\begin{document}
\titlos{Άλγεβρα Β΄ Λυκείου}{Εκθετική και Λογαριθμική Συνάρτηση}{Εκθετική Συνάρτηση}
\orismoi
\Orismos{Εκθετική συνάρτηση}
Εκθετική ονομάζεται κάθε συνάρτηση $ f $ της οποίας ο τύπος αποτελεί δύναμη με θετική βάση, διάφορη της μονάδας και εκθέτη που περιέχει την ανεξάρτητη μετβλητή. Η απλή εκθετική συνάρτηση θα είναι της μορφής :
\[ f(x)=a^x\;\;,\;\;0<a\neq1 \]
\thewrhmata
\Thewrhma{Ιδιότητεσ εκθετικών συναρτήσεων}
Οι ιδιότητες των εκθετικών συναρτήσεων της μορφής $ f(x)=a^x $, με $ 0<a\neq1 $, είναι οι εξής. Σε ορισμένες ιδιότητες διακρίνουμε δύο περιπτώσεις για τη βάση $ a $ της συνάρτησης.
\begin{rlist}
\item Η συνάρτηση $ f $ έχει πεδίο ορισμού το σύνολο $ \mathbb{R} $.
\item Το σύνολο τιμών της είναι το σύνολο $ (0,+\infty) $ των θετικών πραγματικών αριθμών.
\item Η συνάρτηση δεν έχει ακρότατες τιμές.
\begin{enumerate}[itemsep=0mm,label=\bf\arabic*.,leftmargin=0cm]
\item[\textbf{A.}] \textbf{Για {\boldmath$ a>1 $}}
\begin{itemize}
\item Αν η βάση $ a $ της εκθετικής συνάρτησης είναι μεγαλύτερη της μονάδας τότε η συνάρτηση $ f(x)=a^x $ είναι γνησίως αυξουσα στο $ \mathbb{R} $.
\item Η συνάρτηση δεν έχει ρίζες στο $ \mathbb{R} $.
\item Η γραφική παράστασή της έχει οριζόντια ασύμπτωτη τον άξονα $ x'x $ στη μεριά του $ -\infty $ ενώ τέμνει τον κατακόρυφο άξονα $ y'y $ στο σημείο $ A(0,1) $.
\item Για κάθε ζεύγος αριθμών $ x_1,x_2\in\mathbb{R} $ ισχύει 
\begin{gather*}
\textrm{Αν }x_1<x_2\Leftrightarrow a^{x_1}<a^{x_2} \\
\textrm{Αν }x_1=x_2\Leftrightarrow a^{x_1}=a^{x_2}
\end{gather*}
\end{itemize}
\begin{tabular}{p{6cm}p{6cm}}
\begin{tikzpicture}
\begin{axis}[x=.7cm,y=.7cm,aks_on,xmin=-3,xmax=3,
ymin=-.5,ymax=4,ticks=none,xlabel={\footnotesize $ x $},
ylabel={\footnotesize $ y $},belh ar]
\begin{scope}
\clip (axis cs:-3,0) rectangle (axis cs:3,3.7);
\addplot[grafikh parastash,domain=-2.7:2.7]{1.8^x};
\end{scope}
\node at (axis cs:-.3,-0.3) {\footnotesize$O$};
\end{axis}
\tkzDefPoint(-.5,1){B}
\tkzDefPoint(2.1,1.05){A}
\tkzDrawPoint[fill=black](A)
\tkzLabelPoint[above left,yshift=-1mm](A){$ (0,1) $}
\node at (3,0.7) {\footnotesize$a>1$};
\node at (3,2.5) {\footnotesize$C_f$};
\end{tikzpicture}	& \begin{tikzpicture}
\begin{axis}[x=.7cm,y=.7cm,aks_on,xmin=-3,xmax=3,
ymin=-.5,ymax=4,ticks=none,xlabel={\footnotesize $ x $},
ylabel={\footnotesize $ y $},belh ar]
\begin{scope}
\clip (axis cs:-3,0) rectangle (axis cs:3,3.7);
\addplot[grafikh parastash,domain=-2.7:2.7]{0.55^x};
\end{scope}
\node at (axis cs:-.3,-0.3) {\footnotesize$O$};
\end{axis}
\tkzDefPoint(-.8,1){B}
\tkzDefPoint(2.1,1.05){A}
\tkzDrawPoint[fill=black](A)
\tkzLabelPoint[above right,yshift=-1mm](A){$ (0,1) $}
\node at (1.2,0.7) {\footnotesize$0<a<1$};
\node at (1.2,2.5) {\footnotesize$C_f$};
\end{tikzpicture} \\ 
\end{tabular} 
\end{enumerate}
\begin{enumerate}[itemsep=0mm,label=\bf\arabic*.,leftmargin=0cm,start=2]
\item[\textbf{B.}] \textbf{Για {\boldmath$ 0<a<1 $}}
\begin{itemize}
\item Αν η βάση $ a $ της εκθετικής συνάρτησης είναι μικρότερη της μονάδας τότε η συνάρτηση $ f(x)=a^x $ είναι γνησίως φθίνουσα στο $ \mathbb{R} $.
\item Η συνάρτηση δεν έχει ρίζες στο $ \mathbb{R} $.
\item Η γραφική παράστασή της έχει οριζόντια ασύμπτωτη τον άξονα $ x'x $ στη μεριά του $ +\infty $ ενώ τέμνει τον κατακόρυφο άξονα $ y'y $ στο σημείο $ A(0,1) $.
\item Για κάθε ζεύγος αριθμών $ x_1,x_2\in\mathbb{R} $ ισχύει 
\begin{gather*}
\textrm{Αν }x_1<x_2\Leftrightarrow a^{x_1}>a^{x_2} \\
\textrm{Αν }x_1=x_2\Leftrightarrow a^{x_1}=a^{x_2}
\end{gather*}
\end{itemize}
\end{enumerate}
\item Οι γραφικές παραστάσεις των εκθετικών συναρτήσεων με αντίστροφες βάσεις $ f(x)=a^x $ και $ g(x)=\left(\frac{1}{a}\right)^x  $, με $ 0<a\neq1 $, είναι συμμετρικές ως προς τον άξονα $ y'y $.
\end{rlist}
\begin{center}
\begin{tikzpicture}
\begin{axis}[x=.7cm,y=.7cm,aks_on,xmin=-3,xmax=3,
ymin=-.5,ymax=4,ticks=none,xlabel={\footnotesize $ x $},
ylabel={\footnotesize $ y $},belh ar]
\begin{scope}
\clip (axis cs:-3,0) rectangle (axis cs:3,3.7);
\addplot[grafikh parastash,domain=-2.7:2.7]{1.8^x};
\addplot[grafikh parastash,domain=-2.7:2.7]{0.55^x};
\end{scope}
\node at (axis cs:-.3,-0.3) {\footnotesize$O$};
\end{axis}
\tkzDrawPoint[fill=black](2.1,1.05)
\node at (3,2.5) {\footnotesize$C_f$};
\node at (1.2,2.5) {\footnotesize$C_g$};
\node at (.8,.9) {\footnotesize$f(x)=a^x$};
\node at (3.7,.9) {\footnotesize$g(x)=\left(\frac{1}{a}\right)^x$};
\end{tikzpicture}
\end{center}
\end{document}
