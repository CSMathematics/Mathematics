\documentclass[twoside,nofonts,internet,shmeiwseis]{thewria}
\usepackage[amsbb,subscriptcorrection,zswash,mtpcal,mtphrb,mtpfrak]{mtpro2}
\usepackage[no-math,cm-default]{fontspec}
\usepackage{amsmath}
\usepackage{xgreek}
\let\hbar\relax
\defaultfontfeatures{Mapping=tex-text,Scale=MatchLowercase}
\setmainfont[Mapping=tex-text,Numbers=Lining,Scale=1.0,BoldFont={Minion Pro Bold}]{Minion Pro}
\newfontfamily\scfont{GFS Artemisia}
\usepackage{fontawesome}
\newfontfamily{\FA}{fontawesome.otf}
\font\icon = "Webdings"
\usepackage[amsbb]{mtpro2}
\usepackage{tikz,pgfplots}
\tkzSetUpPoint[size=7,fill=white]
\xroma{red!70!black}
%------TIKZ - ΣΧΗΜΑΤΑ - ΓΡΑΦΙΚΕΣ ΠΑΡΑΣΤΑΣΕΙΣ ----
\usepackage{tkz-euclide}
\usetkzobj{all}
\usepackage[framemethod=TikZ]{mdframed}
\usetikzlibrary{decorations.pathreplacing}
\usepackage{pgfplots}
\usetkzobj{all}
%-----------------------

%-----ΕΙΚΟΝΑ ΔΙΠΛΑ ΑΠΟ ΚΕΙΜΕΝΟ-------
\usepackage{wrapfig}
\newenvironment{WrapText1}[3][r]
{\wrapfigure[#2]{#1}{#3}}
{\endwrapfigure}

\newenvironment{WrapText2}[3][l]
{\wrapfigure[#2]{#1}{#3}}
{\endwrapfigure}

\newcommand{\wrapr}[6]{
\begin{minipage}{\linewidth}\mbox{}\\
\vspace{#1}
\begin{WrapText1}{#2}{#3}
\vspace{#4}#5\end{WrapText1}#6
\end{minipage}}

\newcommand{\wrapl}[6]{
\begin{minipage}{\linewidth}\mbox{}\\
\vspace{#1}
\begin{WrapText2}{#2}{#3}
\vspace{#4}#5\end{WrapText2}#6
\end{minipage}}
%-------------------------------------------

\usepackage{calc}

\renewcommand{\thepart}{\arabic{part}}

\usepackage[explicit]{titlesec}
\usepackage{graphicx}
\usepackage{multicol}
\usepackage{multirow}
\usepackage{enumitem}
\usepackage{tabularx}
\usetikzlibrary{backgrounds}
\usepackage{sectsty}
\sectionfont{\centering}
\usepackage{enumitem}
\setlist[enumerate]{label=\bf{\large \arabic*.}}
\usepackage{adjustbox}
%--------- ΑΓΓΛΙΚΟ ΚΕΙΜΕΝΟ --------------
\newcommand{\eng}[1]{\selectlanguage{english}#1\selectlanguage{greek}}
%----------------------------------------
%------- ΣΥΣΤΗΜΑ -------------------
\usepackage{systeme,regexpatch}
\makeatletter
% change the definition of \sysdelim not to store `\left` and `\right`
\def\sysdelim#1#2{\def\SYS@delim@left{#1}\def\SYS@delim@right{#2}}
\sysdelim\{. % reinitialize

% patch the internal command to use
% \LEFTRIGHT<left delim><right delim>{<system>}
% instead of \left<left delim<system>\right<right delim>
\regexpatchcmd\SYS@systeme@iii
{\cB.\c{SYS@delim@left}(.*)\c{SYS@delim@right}\cE.}
{\c{SYS@MT@LEFTRIGHT}\cB\{\1\cE\}}
{}{}
\def\SYS@MT@LEFTRIGHT{%
\expandafter\expandafter\expandafter\LEFTRIGHT
\expandafter\SYS@delim@left\SYS@delim@right}
\makeatother
\newcommand{\synt}[2]{{\scriptsize \begin{matrix}
\times#1\\\\ \times#2
\end{matrix}}}
%----------------------------------------
%------ ΜΗΚΟΣ ΓΡΑΜΜΗΣ ΚΛΑΣΜΑΤΟΣ ---------
\DeclareRobustCommand{\frac}[3][0pt]{%
{\begingroup\hspace{#1}#2\hspace{#1}\endgroup\over\hspace{#1}#3\hspace{#1}}}
%----------------------------------------
%-------- ΜΑΘΗΜΑΤΙΚΑ ΕΡΓΑΛΕΙΑ ---------
\usepackage{mathtools}
%----------------------

%-------- ΠΙΝΑΚΕΣ ---------
\usepackage{booktabs}
%----------------------
%----- ΥΠΟΛΟΓΙΣΤΗΣ ----------
\usepackage{calculator}
%----------------------------
%------ ΔΙΑΓΩΝΙΟ ΣΕ ΠΙΝΑΚΑ -------
\usepackage{array}
\newcommand\diag[5]{%
\multicolumn{1}{|m{#2}|}{\hskip-\tabcolsep
$\vcenter{\begin{tikzpicture}[baseline=0,anchor=south west,outer sep=0]
\path[use as bounding box] (0,0) rectangle (#2+2\tabcolsep,\baselineskip);
\node[minimum width={#2+2\tabcolsep-\pgflinewidth},
minimum  height=\baselineskip+#3-\pgflinewidth] (box) {};
\draw[line cap=round] (box.north west) -- (box.south east);
\node[anchor=south west,align=left,inner sep=#1] at (box.south west) {#4};
\node[anchor=north east,align=right,inner sep=#1] at (box.north east) {#5};
\end{tikzpicture}}\rule{0pt}{.71\baselineskip+#3-\pgflinewidth}$\hskip-\tabcolsep}}
%---------------------------------

%---- ΟΡΙΖΟΝΤΙΟ - ΚΑΤΑΚΟΡΥΦΟ - ΠΛΑΓΙΟ ΑΓΚΙΣΤΡΟ ------
\newcommand{\orag}[3]{\node at (#1)
{$ \overcbrace{\rule{#2mm}{0mm}}^{{\scriptsize #3}} $};}

\newcommand{\kag}[3]{\node at (#1)
{$ \undercbrace{\rule{#2mm}{0mm}}_{{\scriptsize #3}} $};}

\newcommand{\Pag}[4]{\node[rotate=#1] at (#2)
{$ \overcbrace{\rule{#3mm}{0mm}}^{{\rotatebox{-#1}{\scriptsize$#4$}}}$};}
%-----------------------------------------

%-------- ΤΡΙΓΩΝΟΜΕΤΡΙΚΟΙ ΑΡΙΘΜΟΙ -----------
\newcommand{\hm}[1]{\textrm{ημ}#1}
\newcommand{\syn}[1]{\textrm{συν}#1}
\newcommand{\ef}[1]{\textrm{εφ}#1}
\newcommand{\syf}[1]{\textrm{σφ}#1}
%--------------------------------------------

%--------- ΠΟΣΟΣΤΟ ΤΟΙΣ ΧΙΛΙΟΙΣ ------------
\DeclareRobustCommand{\perthousand}{%
\ifmmode
\text{\textperthousand}%
\else
\textperthousand
\fi}
%------------------------------------------

%------------------------------------------
\usepackage{extarrows}
\newcommand{\eq}[1]{\xlongequal{#1}}
%------------------------------------------
%------ ΌΡΙΣΜΑ ----------
\newcommand{\Arg}[8]{
\draw[-latex] (#7,#8)-- ++(#1:#2) node[right=#5]{\footnotesize$#4$};
\draw[fill=black!#6] (#7+0.3+#3,#8) arc (0:#1:0.3+#3) -- (#7,#8);}
%------------------------


\newcommand{\pinakasdyo}[8]{
\begin{tikzpicture}
\foreach \x in {#6,#7}{
\draw (-3,0) -- (#8,0);
\draw (\x,-.5)--(\x,.0);
\node[fill=white,inner sep=1pt] at (\x,-0.25) {$0$};}
\draw (-.5,0.5) -- (-.5,-0.5);
\node at (-.15,0.25) {$-\infty$};
\node at (#8-.3,0.25) {$+\infty$};
\node at (-1.75,0.25) {$x$};
\node at (-1.75,-0.3) {$#1$};
\node[fill=white,inner sep=1pt] at (#6,0.25) {$#2$};
\node[fill=white,inner sep=1pt] at (#7,0.25) {$#3$};
\node at (0.5*#6-0.25,-0.3) {$#4$};
\node at (0.5*#7+0.5*#6,-0.3) {$#5$};
\node at (0.5*#7+0.5*#8,-0.3) {$#4$};
\end{tikzpicture}}

\newcommand{\pinakasmia}[5]{
\begin{tikzpicture}
\draw (-3,0) -- (#5,0);
\draw (#4,-.5)--(#4,.0);
\node[fill=white,inner sep=1pt] at (#4,-0.25) {$0$};
\draw (-.5,0.5) -- (-.5,-0.5);
\node at (-.15,0.25) {$-\infty$};
\node at (#5-0.3,0.25) {$+\infty$};
\node at (-1.75,0.25) {$x$};
\node at (-1.75,-0.3) {$#1$};
\node[fill=white,inner sep=1pt] at (#4,0.25) {$#2$};
\node at (0.5*#4-0.25,-.3) {$#3$};
\node at (0.5*#4+0.5*#5,-0.3) {$#3$};
\end{tikzpicture}}

\newcommand{\pinakaskamia}[2]{
\begin{tikzpicture}
\draw (-3,0) -- (5,0);
\draw (-.5,0.5) -- (-.5,-0.5);
\node at (-.15,0.25) {$-\infty$};
\node at (4.7,0.25) {$+\infty$};
\node at (-1.75,0.25) {$x$};
\node at (-1.75,-0.3) {$#1$};
\node[fill=white,inner sep=1pt] at (2.25,-0.3) {$#2$};
\end{tikzpicture}}



\newenvironment{meth}[1][]{%
\refstepcounter{meth}
\begin{mdframed}[%
frametitle={m \themeth\quad\ \MakeUppercase{#1}},
skipabove=\baselineskip plus 2pt minus 1pt,
skipbelow=\baselineskip plus 2pt minus 1pt,
linewidth=0pt,
frametitlerule=false,
linecolor=black,
outerlinewidth=0pt,
rightline=false,
bottomline=false,
topline=false,
frametitlebackgroundcolor=black!40,
frametitlerulewidth=0pt,
innertopmargin=\baselineskip,
innerbottommargin=\baselineskip,
innerrightmargin=20pt,
innerleftmargin=20pt,
backgroundcolor=black!10,
frametitleaboveskip=7pt
]%
}{%
\end{mdframed}}
%-------------------------------
\newcommand{\tss}[1]{\textsuperscript{#1}}
\newcommand{\tssL}[1]{\MakeLowercase{\textsuperscript{#1}}}
\newlist{rlist}{enumerate}{3}
\setlist[rlist]{itemsep=0mm,label=\roman*.}
\newlist{tropos}{enumerate}{3}
\setlist[tropos]{label=\bf\textit{\arabic*\textsuperscript{oς}\;Τρόπος :},leftmargin=2cm}

\ekthetesdeiktes



\begin{document}
\titlos{Αλγεβρα Β΄ Λυκείου}{Συστήματα}{Γραμμικά Συστήματα}
\orismoi
\Orismos{Γραμμική Εξίσωση}
\wrapr{-4mm}{10}{4.5cm}{-4mm}{
\begin{tikzpicture}[domain=-.2:4,y=1cm,scale=.8]
\draw[-latex] (-.5,0) -- coordinate (x axis mid) (4.4,0) node[right,fill=white] {{\footnotesize $ x $}};
\draw[-latex] (0,-.5) -- (0,3.5) node[above,fill=white] {{\footnotesize $ y $}};
\draw[domain=-.2:3.4,samples=100,line width=.4mm,\xrwma] plot function{-0.8*x+2.5};
\tkzText(2.5,2.7){$ ax+\beta y=\gamma $}
\tkzText(2.5,2.2){{\footnotesize $ a,\beta,\gamma\in\mathbb{R} $}}
\tkzText(2.5,1.7){{\footnotesize $ a\neq0 $ ή $ \beta\neq0 $}}
\tkzDefPoint(0,0){O}
\tkzLabelPoint[below left](O){$ O $}
\end{tikzpicture}}{
Γραμμική εξίσωση δύο μεταβλητών, ονομάζεται κάθε πολυωνυμική εξίσωση στην οποία κάθε όρος της είναι μονώνυμο 1\textsuperscript{ου} βαθμού μιας μεταβλητής. Έχει τη μορφή \[ ax+\beta y=\gamma \]
όπου οι συντελεστές και ο σταθερός όρος είναι πραγματικοί αριθμοί $ a,\beta,\gamma\in\mathbb{R} $.
Η καμπύλη της εξίσωσης είναι ευθεία γραμμή αν οι συντελεστές $ a,\beta $ των μεταβλητών $ x,y $ αντίστοιχα, δεν μηδενίζονται συγχρόνως δηλ. $ a\neq0 $ ή $ \beta\neq0 $.}
\begin{itemize}[itemsep=0mm]
\item Οι ευθείες της μορφής $ x=\kappa $ ονομάζονται \textbf{κατακόρυφες} ευθείες ενώ οι ευθείες της μορφής $ y=\kappa $ οριζόντιες ευθείες.
\item Ο πραγματικός αριθμός $ \lambda=-\frac{a}{\beta} $ ονομάζεται \textbf{συντελεστής διεύθυνσης} της ευθείας $ ax+\beta y=\gamma $.
\end{itemize}
\Orismos{Λύση γραμμικήσ εξίσωσησ}
Λύση μιας γραμμικής εξίσωσης της μορφής \[ ax+\beta y=\gamma \] ονομάζεται κάθε διατεταγμένο ζεύγος αριθμών $ \left( x_0,y_0\right)  $ το οποίο επαληθεύει την εξίσωση.\\\\
\Orismos{Γραμμικό σύστημα $ \mathbold{2\times2} $}
Γραμμικό σύστημα δύο εξισώσεων με δύο άγνωστους ονομάζεται ο συνδυασμός - σύζευξη δύο γραμμικών εξισώσεων. Είναι της μορφής :
\[ \ccases{{a}x+{\beta} y={\gamma}\\{a'}x+{\beta'} y={\gamma'} } \]
\begin{itemize}[itemsep=0mm]
\item Οι συντελεστές του συστήματος $ a,a',\beta,\beta' $ και οι σταθεροί όροι $ \gamma,\gamma' $ είναι πραγματικοί αριθμοί.
\item Κάθε διατεταγμένο ζεύγος αριθμών $ \left(x_0,y_0\right)  $ το οποίο επαληθεύει και τις δύο εξισώσεις ονομάζεται \textbf{λύση} του γραμμικού συστήματος.
\item Τα συστήματα τα οποία έχουν ακριβώς τις ίδιες λύσεις ονομάζονται \textbf{ισοδύναμα}.
\item Ένα σύστημα που έχει λύση λέγεται \textbf{συμβιβαστό}. Εαν δεν έχει λύση ονομάζεται \textbf{αδύνατο} ενώ αν έχει άπειρες λύσεις \textbf{αόριστο}.
\end{itemize}
\Orismos{Επαλήθευση Συστήματοσ}
Επαλήθευση ενός συστήματος εξισώσεων ονομάζεται η διαδικασία με την οποία εξετάζουμε εαν ένα ζεύγος αριθμών $ \left(x_0,y_0\right)  $ είναι λύση του, αντικαθιστώντας τους αριθμούς στη θέση των μεταβλητών.\\\\
\Orismos{Ορίζουσα Συστήματοσ {$ \mathbold{2\mathbold{\times}2} $}}
Ορίζουσα των συντελεστών ενός συστήματος $ 2\times2 $ ονομάζεται ο αριθμός $ a\beta'-a'\beta $ η οποία συμβολίζεται
\[ D=\left|\begin{array}{cc}
a & \beta \\ 
a' & \beta'
\end{array}  \right|  \]
$ D_x,D_y $ είναι οι ορίζουσες των μεταβλητών που προκύπτουν αν αντικαταστίσουμε στην ορίζουσα $ D $ τη στήλη των συντελεστών των μεταβλητών $ x,y $ αντίστοιχα με τους σταθερούς όρους $ \gamma,\gamma' $.
\[ D_x=\left|\begin{array}{cc}
\gamma & \beta \\ 
\gamma' & \beta'
\end{array}  \right|\quad,\quad D_y=\left|\begin{array}{cc}
a & \gamma \\ 
a' & \gamma'
\end{array}  \right| \]
\Orismos{Γραμμικό Σύστημα Εξισώσεων $ \mathbold{3\times3} $}
Γραμμικό σύστημα τριών εξισώσεων με τρεις άγνωστους ονομάζεται ένας συνδυασμός από τρεις γραμμικές εξισώσεις της μορφής
\[ \ccases{{a_1}x+{\beta_1} y+{\gamma_1} z=\delta_1\\{a_2}x+{\beta_2} y+{\gamma_2} z=\delta_2\\{a_3}x+{\beta_3} y+{\gamma_3} z=\delta_3} \]
με $ a_i,\beta_i,\gamma_i,\delta_i\in\mathbb{R}\;,\;i=1,2,3 $. Κάθε διατεταγμένη τριάδα αριθμών $ \left( x_0,y_0,z_0\right)  $ η οποία επαληθεύει και τις τρεις εξισώσεις ονομάζεται \textbf{λύση} του γραμμικού συστήματος $ 3\times3 $.\\\\
\Orismos{Παραμετρικό σύστημα}
Παραμετρικό ονομάζεται το γραμμικό σύστημα του οποίου οι συντελεστές ή και οι σταθεροί όροι δίνονται με τη βοήθεια μιας ή περισσότερων παραμέτρων. Η διαδικασία επίλυσης ενός παραμετρικού συστήματος ονομάζεται \textbf{διερεύνηση}.
\\\\
\Orismos{Μη γραμμικό σύστημα}
Ένα σύστημα εξισώσεων θα ονομάζεται μη γραμμικό όταν τουλάχιστον μια εξίσωσή του δεν αποτελεί γραμμική εξίσωση.\\\\
\thewrhmata
\Thewrhma{Είδος ευθείασ}
Η γραμμική εξίσωση $ ax+\beta y=\gamma $ παριστάνει
\begin{rlist}
\item πλάγια ευθεία αν $ a\neq0 $ ή $ \beta\neq0 $.
\item οριζόντια ευθεία αν $ a=0 $ και $ \beta\neq0 $.
\item κατακόρυφη ευθεία αν $ a\neq0 $ και $ \beta=0 $.
\end{rlist}
ενώ αν μηδενίζονται συγχρόνως οι συντελεστές $ a $ και $ \beta $ τότε δεν παριστάνει ευθεία γραμμή.\\\\
\Thewrhma{Σημείο σε ευθεία}
Ένα σημείο $ A(x_0,y_0) $ ανήκει σε μια ευθεία με εξίσωση $ ax+\beta y=\gamma $ αν και μόνο αν οι συντεταγμένες του επαληθεύουν την εξίσωση της.\\\\
\Thewrhma{Λύση συστήματοσ {$ \mathbold{2\mathbold{\times}2} $} με χρήση οριζουσων}
Έστω το γραμμικό σύστημα 
\[ \ccases{{a}x+{\beta} y={\gamma}\\{a'}x+{\beta'} y={\gamma'} } \]
με πραγματικούς συντελεστές και ορίζουσα συντελεστών $ D $.
\begin{rlist}
\item Αν η ορίζουσα των συντελεστών του συστήματος είναι διάφορη του μηδενος δηλαδή $ D\neq0 $ τότε το σύστημα έχει μοναδική λύση. Οι τιμές των μεταβλητών δίνονται από τις σχέσεις
\[ x=\frac{D_x}{D}\;\;,\;\;y=\frac{D_y}{D} \]
ενώ η λύση του συστήματος θα είναι $ (x,y)=\left(\frac{D_x}{D},\frac{D_y}{D} \right)  $.
\item Αν η ορίζουσα των συντελεστών του συστήματος είναι μηδενική δηλαδή $ D=0 $ τότε το σύστημα είναι είτε αόριστο είτε αδύνατο.
\end{rlist}
\end{document}