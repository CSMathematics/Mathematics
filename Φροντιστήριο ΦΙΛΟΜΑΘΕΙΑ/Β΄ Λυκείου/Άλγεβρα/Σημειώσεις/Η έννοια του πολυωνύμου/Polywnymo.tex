\documentclass[twoside,nofonts,internet,shmeiwseis]{thewria}
\usepackage[amsbb,subscriptcorrection,zswash,mtpcal,mtphrb,mtpfrak]{mtpro2}
\usepackage[no-math,cm-default]{fontspec}
\usepackage{amsmath}
\usepackage{xgreek}
\let\hbar\relax
\defaultfontfeatures{Mapping=tex-text,Scale=MatchLowercase}
\setmainfont[Mapping=tex-text,Numbers=Lining,Scale=1.0,BoldFont={Minion Pro Bold}]{Minion Pro}
\newfontfamily\scfont{GFS Artemisia}
\font\icon = "Webdings"
\usepackage[amsbb]{mtpro2}
\usepackage{tikz,pgfplots,tkz-euclide}
\usetkzobj{all}
\tkzSetUpPoint[size=7,fill=white]
\xroma{red!70!black}

\newlist{rlist}{enumerate}{3}
\setlist[rlist]{itemsep=0mm,label=\roman*.}
\newlist{brlist}{enumerate}{3}
\setlist[brlist]{itemsep=0mm,label=\bf\roman*.}
\newlist{tropos}{enumerate}{3}
\setlist[tropos]{label=\bf\textit{\arabic*\textsuperscript{oς}\;Τρόπος :},leftmargin=0cm,itemindent=2.3cm,ref=\bf{\arabic*\textsuperscript{oς}\;Τρόπος}}
\newcommand{\tss}[1]{\textsuperscript{#1}}
\newcommand{\tssL}[1]{\MakeLowercase{\textsuperscript{#1}}}

\usepackage{hhline}

\usepackage{multicol}
\usepackage{wrap-rl}
\usepackage{gensymb,mathimatika}

\ekthetesdeiktes

\begin{document}
\titlos{Άλγεβρα Β΄ Λυκείου}{Πολυώνυμα}{Η έννοια του πολυωνύμου}
\orismoi
\Orismos{Μεταβλητή}
Μεταβλητή ονομάζεται το σύμβολο το οποίο χρησιμοποιούμε για εκφράσουμε έναν άγνωστο αριθμό. Η μεταβλητή μπορεί να βρίσκεται μέσα σε μια εξίσωση και γενικά σε μια αλγεβική παράσταση.
Συμβολίζεται με ένα γράμμα όπως $ a,\beta,x,y,\ldots $ κ.τ.λ.\\\\
\Orismos{ΜΟΝΏΝΥΜΟ}
Μονώνυμο ονομάζεται η ακέραια αλγεβρική παράσταση η οποία έχει μεταξύ των μεταβλητών μόνο την πράξη του πολλαπλασιασμού.
\[ \textrm{{\scriptsize Συντελεστής} }\longrightarrow a\cdot \undercbrace{x^{\nu_1}y^{\nu_2}\cdot \ldots\cdot z^{\nu_\kappa}}_{\textrm{κύριο μέρος}}\;\;,\;\;\nu_1,\nu_2,\ldots,\nu_\kappa\in\mathbb{N} \]
\begin{itemize}[itemsep=0mm]
\item Το γινόμενο των μεταβλητών ενός μονωνύμου ονομάζεται \textbf{κύριο μέρος}.
\item  Ο σταθερός αριθμός με τον οποίο πολλαπλασιάζουμε το κύριο μέρος ενός μονωνύμου ονομάζεται \textbf{συντελεστής}.
\item Τα μονώνυμα μιας μεταβλητής είναι της μορφής $ ax^\nu $, όπου $ a\in\mathbb{R} $ και $ \nu\in\mathbb{N} $.
\end{itemize}
\Orismos{ΠΟΛΥΏΝΥΜΟ}	Πολυώνυμο ονομάζεται η ακέραια αλγεβρική παράσταση η οποία είναι άθροισμα
ανόμοιων μονωνύμων.
\begin{itemize}[itemsep=0mm]
\item Κάθε μονώνυμο μέσα σ' ένα πολυώνυμο ονομάζεται \textbf{όρος} του πολυωνύμου.
\item Το πολυώνυμο με 3 όρους ονομάζεται \textbf{τριώνυμο}.
\item Οι αριθμοί ονομάζονται \textbf{σταθερά πολυώνυμα} ενώ το 0 \textbf{μηδενικό πολυώνυμο}.
\item  Κάθε πολυώνυμο συμβολίζεται με ένα κεφαλαίο γράμμα όπως : $ P, Q, A, B\ldots $ τοποθετώντας δίπλα από το όνομα μια παρένθεση η οποία περιέχει τις μεταβλητές του δηλαδή :  \[ P(x), Q(x,y), A(z,w), B\left( x_1,x_2,\ldots,x_\nu\right) \]
\item \textbf{Βαθμός} ενός πολυωνύμου ορίζεται ο μεγαλύτερος εκθέτης της κάθε μεταβλητής. Ο όρος που περιέχει τη μεταβλητή με το μεγαλύτερο εκθέτη ονομάζεται \textbf{μεγιστοβάθμιος}.
\item Τα πολυώνυμα μιας μεταβλητής τα γράφουμε κατά φθίνουσες δυνάμεις της μεταβλητής δηλαδή από τη μεγαλύτερη στη μικρότερη. Έχουν τη μορφή :
\end{itemize}
\[ P(x)=a_\nu x^\nu+a_{\nu-1}x^{\nu-1}+\ldots+a_1x+a_0 \]
\Orismos{ΤΙΜΉ ΠΟΛΥΩΝΎΜΟΥ}
Τιμή ενός πολυωνύμου $ P(x)=a_\nu x^\nu+a_{\nu-1}x^{\nu-1}+\ldots+a_1x+a_0 $ ονομάζεται ο πραγματικός αριθμός που προκύπτει ύστερα από πράξεις αν αντικαταστίσουμε τη μεταβλητή του πολυωνύμου με έναν αριθμό $ x_0 $. Συμβολίζεται με $ P(x_0) $ και είναι ίση με :
\[ P(x_0)=a_\nu x_0^\nu+a_{\nu-1}x_0^{\nu-1}+\ldots+a_1x_0+a_0 \]
\Orismos{ΡΊΖΑ ΠΟΛΥΩΝΎΜΟΥ}
Ρίζα ενός πολυωνύμου $ P(x)=a_\nu x^\nu+a_{\nu-1}x^{\nu-1}+\ldots+a_1x+a_0 $ ονομάζεται κάθε πραγματικός αριθμός $ \rho\in\mathbb{R} $ ο οποίος μηδενίζει το πολυώνυμο.
\[ P(\rho)=0 \]
\thewrhmata
\Thewrhma{Βαθμός πολυωνύμου}
Έστω δύο πολυώνυμα $ A(x)=a_\nu x^\nu+a_{\nu-1}x^{\nu-1}+\ldots+a_1x+a_0 $ και $ B(x)=\beta_\mu x^\mu+\beta_{\mu-1}x^{\mu-1}+\ldots+\beta_1x+\beta_0 $ βαθμών $ \nu $ και $ \mu $ αντίστοιχα με $ \nu\geq\mu $. Τότε ισχύουν οι παρακάτω προτάσεις :
\begin{rlist}
\item Ο βαθμός του αθροίσματος ή της διαφοράς $ A(x)\pm B(x) $ είναι μικρότερος ίσος του μέγιστου των βαθμών των πολυωνύμων $ A(x) $ και $ B(x) $ : $ \textrm{βαθμός}(A(x)+B(x))\leq\max\{\nu,\mu\} $.
\item Ο βαθμός του γινομένου $ A(x)\cdot B(x) $ ισούται με το άθροισμα των βαθμών των πολυωνύμων $ A(x) $ και $ B(x) $ : $ \textrm{βαθμός}(A(x)\cdot B(x))=\nu+\mu $.
\item Ο βαθμός του πηλίκου $ \pi(x) $ της διαίρεσης $ A(x):B(x) $ ισούται με τη διαφορά των βαθμών των πολυωνύμων $ A(x) $ και $ B(x) $ : $ \textrm{βαθμός}(A(x): B(x))=\nu-\mu $.
\item Ο βαθμός της δύναμης $ [A(x)]^\kappa $ του πολυωνύμου $ A(x) $ ισούται με το γινόμενο του εκθέτη $ \kappa $ με το βαθμό του $ A(x) $ : $ \textrm{βαθμός}([A(x)]^\kappa)=\nu\cdot\kappa $.
\end{rlist}
\Thewrhma{Ίσα πολυώνυμα}
Δύο πολυώνυμα $ A(x)=a_\nu x^\nu+a_{\nu-1}x^{\nu-1}+\ldots+a_1x+a_0 $ και $ B(x)=\beta_\mu x^\mu+\beta_{\mu-1}x^{\mu-1}+\ldots+\beta_1x+\beta_0 $ βαθμών $ \nu $ και $ \mu $ αντίστοιχα με $ \nu\geq\mu $ θα είναι μεταξύ τους ίσα αν και μόνο αν οι συντελεστές των ομοβάθμιων όρων τους είναι ίσοι.
\begin{gather*}
A(x)=B(x)\Leftrightarrow a_i=\beta_i\ ,\ \textrm{για κάθε }i=0,1,2,\ldots,\mu\\
\textrm{και }a_i=0\ ,\ \textrm{για κάθε }i=\mu+1,\mu+2,\ldots,\nu
\end{gather*}
Ένα πολυώνυμο $ A(x)=a_\nu x^\nu+a_{\nu-1}x^{\nu-1}+\ldots+a_1x+a_0 $ ισούται με το μηδενικό πολυώνυμο αν και μόνο αν όλοι του οι συντελεστές είναι μηδενικοί.
\[ A(x)=0\Leftrightarrow a_i=0 \ ,\ \textrm{για κάθε }i=0,1,2,\ldots,\nu\]
\end{document}
