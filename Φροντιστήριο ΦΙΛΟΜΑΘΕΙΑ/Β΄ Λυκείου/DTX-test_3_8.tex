%# Database Document : DTX-test_3_8-----------------
%@ Document type: Β΄ Λυκείου
%#--------------------------------------------------

%# Database File : Alg-Syst-GrSys-GrExParam-SectEx2
%@ Database source: b_lykeiou
Να βρεθούν οι τιμές της παραμέτρου $ \lambda\in\mathbb{R} $ ώστε η ευθεία $ (\lambda^2-1)x+(1-\lambda)y=2 $ να διέρχεται από το σημείο $ A(1,3) $.
%# End of file Alg-Syst-GrSys-GrExParam-SectEx2

%# Database File : Alg-Syst-GrSys-MethAnt-SectEx1
%@ Database source: b_lykeiou
Να λυθούν τα παρακάτω γραμμικά συστήματα με τη μέθοδο της αντικατάστασης.
\begin{multicols}{2}
\begin{rlist}[leftmargin=5mm]
\item $ \systeme{-x+y=2,2x-2y=3} $
\item $ \systeme{x=2y-1,4x-8y=5} $
\item $ \systeme{2x+y=1,y=7-2x} $
\end{rlist}
\end{multicols}
%# End of file Alg-Syst-GrSys-MethAnt-SectEx1

%# Database File : Alg-Syst-GrSys-Def1
%@ Database source: b_lykeiou
Γραμμική εξίσωση με δύο μεταβλητές $x,y$ ονομάζεται κάθε εξίσωση της μορφής
\[ ax+\beta y=\gamma \]
όπου $ a,\beta,\gamma\in\mathbb{R} $.
%# End of file Alg-Syst-GrSys-Def1

%# Database File : Alg-Syst-GrSys-GrExParam-SectEx3
%@ Database source: b_lykeiou
Να βρεθούν οι τιμές της παραμέτρου $ \lambda\in\mathbb{R} $ ώστε η ευθεία $ \lambda x+(\lambda-1)y=4 $ να διέρχεται από το σημείο $ A(-2,3) $.
%# End of file Alg-Syst-GrSys-GrExParam-SectEx3

%# Database File : Alg-Syst-GrSys-GrExParam-SolSE3-1
%@ Database source: b_lykeiou
Η εξίσωση παριστάνει ευθεία για κάθε $ \lambda\in\mathbb{R} $ αφού
\[ \lambda=0\ \ \text{και}\ \ \lambda-1=0\Rightarrow\lambda=1 \]
δηλαδή οι συντελεστές των $ x,y $ δεν μηδενίζονται συγχρόνως. Στη συνέχεια έχουμε ότι το σημείο $ A(-2,3) $ ανήκει στην ευθεία αυτή αν και μόνο αν για $ x=-2 $ και $ y=3 $
\[ \lambda\cdot(-2)+(\lambda-1)\cdot 3=4\Rightarrow -2\lambda+3\lambda-3=4\Rightarrow \lambda=7 \]
%# End of file Alg-Syst-GrSys-GrExParam-SolSE3-1

%# Database File : Alg-Syst-GrSys-MethAnt-SectEx2
%@ Database source: b_lykeiou
Να λυθούν τα παρακάτω γραμμικά συστήματα με τη μέθοδο της αντικατάστασης.
\begin{multicols}{2}
\begin{rlist}[leftmargin=5mm]
\item $ \systeme{x-2y=4,2x-4y=8} $
\item $ \systeme{3x-4y=1,-6x+8y=-2} $
\item $ \systeme{4x+2y=6,6x+3y=9} $
\end{rlist}
\end{multicols}
%# End of file Alg-Syst-GrSys-MethAnt-SectEx2

%# Database File : Alg-EkthLog-Logar-YpolLog-SectEx2
%@ Database source: b_lykeiou
Να υπολογίσετε την τιμή των παρακάτω λογαρίθμων.
\begin{multicols}{3}
\begin{rlist}[leftmargin=5mm]
\item $ \log{100} $
\item $ \log{10000} $
\item $ \log{10^7} $
\item $ \log{10^{-19}} $
\item $ \ln{e^2} $
\item $ \ln{e^{-23}} $
\end{rlist}
\end{multicols}
%# End of file Alg-EkthLog-Logar-YpolLog-SectEx2

