\documentclass[ektypwsh]{diag-xelatex}
\usepackage[amsbb]{mtpro2}
\usepackage[no-math,cm-default]{fontspec}
\usepackage{xunicode}
\usepackage{xltxtra}
\usepackage{xgreek}
\usepackage{amsmath}
\defaultfontfeatures{Mapping=tex-text,Scale=MatchLowercase}
\setmainfont[Mapping=tex-text,Numbers=Lining,Scale=1.0,BoldFont={Minion Pro Bold}]{Minion Pro}
\newfontfamily\scfont{GFS Artemisia}
\font\icon = "Webdings"
\usepackage[amsbb]{mtpro2}
\usepackage[left=2.00cm, right=2.00cm, top=2.00cm, bottom=3.00cm]{geometry}
\xroma{red!80!black}
\newcommand{\tss}[1]{\textsuperscript{#1}}
\newcommand{\tssL}[1]{\MakeLowercase\textsuperscript{#1}}
\newlist{rlist}{enumerate}{3}
\setlist[rlist]{itemsep=0mm,label=\textcolor{\xrwma}{\roman*.}}


\begin{document}
\titlos{Μαθηματικά Κατεύθυνσης Β΄ Λυκείου}{ΔΙΑΝΥΣΜΑΤΑ}
\begin{thema}
\item \textbf{Ερωτήσεις Θεωρίας}\\
Να απαντήσετε στις παρακάτω ερωτήσεις.
\begin{rlist}
\item Έστω δύο μη μηδενικά διανύσματα $ \vec{a} $ και $ \vec{\beta} $ με συντελεστές διεύθυνσης $ \lambda_{\vec{a}} $ και $ \lambda_{\vec{\beta}} $ αντίστοιχα. Ποιά συνθήκη πρέπει να ισχύει ώστε τα διανύσματα να είναι κάθετα;
\item Ποιές συνθήκες γνωρίζετε ώστε δύο μη μηδενικά διανύσματα $ \vec{a} $ και $ \vec{\beta} $ να είναι παράλληλα;
\item Πως ορίζεται ο συντελεστής διεύθυνσης ενός διανύσματος;
\item Από ποιά σχέση δίνεται η απόσταση μεταξύ δύο σημείων $ A(x_1,y_1) $ και $ B(x_2,y_2) $ του επιπέδου;
\item Από ποιά σχέση μπορούμε να υπολογίσουμε τη γωνία $ \varphi $ δύο διανυσμάτων $ a,\beta $;\monades{5}
\end{rlist}
\item \textbf{Πράξεις με διανύσματα}\\
Δίνονται τρία σημεία $ A,B,\varGamma $ του επιπέδου και δύο τυχαία σημεία $ M,N $ ώστε να ισχύει :
\[ \overrightarrow{AM}+2\overrightarrow{BK}-\overrightarrow{N\varGamma}=2\overrightarrow{AK}+\overrightarrow{BN}+\overrightarrow{\varGamma M} \]
Να δειχθεί οτι τα σημεία $ A,B,\varGamma $ είναι συνευθειακά.\monades{4}
\item \textbf{Εύρεση παραμέτρου}\\
Δίνονται τα διανύσματα $ \vec{a}=(2,3), \vec{\beta}=(\lambda^2-2,4) $ και $ \vec{\gamma}=(-1,3-\lambda) $ όπου $ \lambda\in\mathbb{R} $ μια παράμετρος. Να βρεθούν οι τιμές της παραμέτρου $ \lambda $ ώστε :
\begin{rlist}
\item τα διανύσματα $ \vec{a},\vec{\beta} $ να είναι κάθετα : $ \vec{a}\perp\vec{\beta} $\monades{2}
\item τα διανύσματα $ \vec{a},\vec{\gamma} $ να είναι παράλληλα : $ \vec{a}\parallel\vec{\gamma} $\monades{1}
\item τα διανύσματα $ \vec{\beta},\vec{\gamma} $ να είναι  ίσα : $ \vec{\beta}=\vec{\gamma} $\monades{2}
\end{rlist}
\item \textbf{Σύνθετο θέμα}\\
Δίνονται τα σημεία $ A(2,0), B(-3,4) $ και $ \varGamma(3,5) $ του επιπέδου τα οποία σχηματίζουν τρίγωνο $ AB\varGamma $. Να βρεθεί
\begin{rlist}
\item η διάμεσος $ \overrightarrow{BM} $ του τριγώνου. \monades{2}
\item η προβολή $ K(x,y) $ του σημείου $ A $ στην πλευρά $ B\varGamma $.\monades{4}
\end{rlist}
\end{thema}

\end{document}
