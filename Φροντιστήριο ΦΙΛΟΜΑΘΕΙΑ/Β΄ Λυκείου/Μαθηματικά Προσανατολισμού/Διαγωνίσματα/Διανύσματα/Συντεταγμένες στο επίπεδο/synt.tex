\documentclass[twoside,nofonts,ektypwsh]{frontisthrio-diag}
\usepackage[amsbb,subscriptcorrection,zswash,mtpcal,mtphrb,mtpfrak]{mtpro2}
\usepackage[no-math,cm-default]{fontspec}
\usepackage{amsmath}
\usepackage{xunicode}
\usepackage{xgreek}
\let\hbar\relax
\defaultfontfeatures{Mapping=tex-text,Scale=MatchLowercase}
\setmainfont[Mapping=tex-text,Numbers=Lining,Scale=1.0,BoldFont={Minion Pro Bold}]{Minion Pro}
\newfontfamily\scfont{GFS Artemisia}
\font\icon = "Webdings"
\usepackage{fontawesome5}
\newfontfamily{\FA}{fontawesome.otf}
\xroma{cyan!70!black}
%------TIKZ - ΣΧΗΜΑΤΑ - ΓΡΑΦΙΚΕΣ ΠΑΡΑΣΤΑΣΕΙΣ ----
\usepackage{tikz,pgfplots}
\usepackage{tkz-euclide}
\usetkzobj{all}
\usepackage[framemethod=TikZ]{mdframed}
\usetikzlibrary{decorations.pathreplacing}
\tkzSetUpPoint[size=7,fill=white]
%-----------------------
\usepackage{calc,tcolorbox}
\tcbuselibrary{skins,theorems,breakable}
\usepackage{hhline}
\usepackage[explicit]{titlesec}
\usepackage{graphicx}
\usepackage{multicol}
\usepackage{multirow}
\usepackage{tabularx}
\usetikzlibrary{backgrounds}
\usepackage{sectsty}
\sectionfont{\centering}
\usepackage{enumitem}
\usepackage{adjustbox}
\usepackage{mathimatika,gensymb,eurosym,wrap-rl}
\usepackage{systeme,regexpatch}
%-------- ΜΑΘΗΜΑΤΙΚΑ ΕΡΓΑΛΕΙΑ ---------
\usepackage{mathtools}
%----------------------
%-------- ΠΙΝΑΚΕΣ ---------
\usepackage{booktabs}
%----------------------
%----- ΥΠΟΛΟΓΙΣΤΗΣ ----------
\usepackage{calculator}
%----------------------------
%------------------------------------------
\newcommand{\tss}[1]{\textsuperscript{#1}}
\newcommand{\tssL}[1]{\MakeLowercase{\textsuperscript{#1}}}
%---------- ΛΙΣΤΕΣ ----------------------
\newlist{bhma}{enumerate}{3}
\setlist[bhma]{label=\bf\textit{\arabic*\textsuperscript{o}\;Βήμα :},leftmargin=0cm,itemindent=1.8cm,ref=\bf{\arabic*\textsuperscript{o}\;Βήμα}}
\newlist{tropos}{enumerate}{3}
\setlist[tropos]{label=\bf\textit{\arabic*\textsuperscript{oς}\;Τρόπος :},leftmargin=0cm,itemindent=2.3cm,ref=\bf{\arabic*\textsuperscript{oς}\;Τρόπος}}
% Αν μπει το bhma μεσα σε tropo τότε
%\begin{bhma}[leftmargin=.7cm]
\tkzSetUpPoint[size=7,fill=white]
\tikzstyle{pl}=[line width=0.3mm]
\tikzstyle{plm}=[line width=0.4mm]
\usepackage{etoolbox}
\makeatletter
\renewrobustcmd{\anw@true}{\let\ifanw@\iffalse}
\renewrobustcmd{\anw@false}{\let\ifanw@\iffalse}\anw@false
\newrobustcmd{\noanw@true}{\let\ifnoanw@\iffalse}
\newrobustcmd{\noanw@false}{\let\ifnoanw@\iffalse}\noanw@false
\renewrobustcmd{\anw@print}{\ifanw@\ifnoanw@\else\numer@lsign\fi\fi}
\makeatother

\usepackage{path}
\pathALa

\begin{document}
\titlos{Β΄ Λυκείου - Μαθηματικά Προσανατολισμού}{Συντεταγμένες διανύσματος}{Β}
\begin{thema}
\item\mbox{}\\\vspace{-7mm}
\begin{erwthma}
\item Δίνονται τα σημεία $ A(x_1,y_1) $ και $ B(x_2,y_2) $ με $ x_1\neq x_2 $. Να γράψετε με τη βοήθεια των $ x_1,x_2,y_1,y_2 $ τους τύπους από τους οποίους δίνονται
\begin{alist}
\item οι συντεταγμένες του $ \dsx{AB} $.
\item το μέτρο $ |\dsx{AB}| $.
\item οι συντεταγμένες του μέσου $ M $ του τμήματος $ AB $.
\item ο συντελεστής διεύθυνσης του $ \dsx{AB} $.
\end{alist}\monades{10}
\item Να δείξετε ότι $ \dsx{a}+\dsx{\beta}=(x_1+x_2,y_1+y_2) $.\monades{5}
\item \swstolathos
\begin{alist}
\item Το διάνυσμα $ \dsx{AB} $ με άκρα $ A(-2,3) $ και $ B(-2,-1) $ είναι κατακόρυφο.
\item Τα διανύσματα $ \vec{a}=(3,-1) $ και $ \vec{\beta}=(-9,3) $ είναι παράλληλα.
\item Ο συντελεστής διεύθυνσης του διανύσματος $ \vec{a}=(2,4) $ είναι $ \lambda=\frac{1}{2} $.
\item Αν για δύο διανύσματα $ \vec{a},\vec{\beta} $ ισχύει η σχέση $ \vec{a}=-2\vec{\beta} $ τότε $ \det{(\vec{a},\vec{\beta})}=0 $.
\item Το διάνυσμα $ \vec{a}=\left(\frac{1}{2},\frac{\sqrt{3}}{2}\right) $ είναι μοναδιαίο.
\end{alist}
\end{erwthma}\monades{10}
\item Δίνεται τρίγωνο $ AB\varGamma $ με κορυφές $ A(-2,1),B(4,-5) $ και $ \varGamma(6,3) $.
\begin{erwthma}
\item Να βρεθούν οι συντεταγμένες των μέσων $ K,\varLambda,M $ των πλευρών $ AB,A\varGamma $ και $ B\varGamma $ αντίστοιχα.\\\monades{8}
\item Να δείξετε ότι το τρίγωνο είναι ισοσκελές.\monades{9}
\item Να βρεθεί ο συντελεστής διεύθυνσης της διαμέσου $ \dsx{AM} $ και των πλευρών $ \dsx{AB} $ και $ \dsx{B\varGamma} $.\monades{8}
\end{erwthma}
\item Δίνονται τα διανύσματα $ \vec{a}=(\lambda-2,-5),\vec{\beta}=(3,\lambda^2+\lambda) $ και $ \vec{\gamma}=(7,-8) $ με $ \lambda\in\mathbb{R} $, για τα οποία ισχύει $ \vec{a}+\vec{\beta}=(4,7) $.
\begin{erwthma}
\item Να δείξετε ότι $ \lambda=3 $.\monades{7}
\item Να γραφτεί το διάνυσμα $ \vec{\gamma} $ ως γραμμικός συνδυασμός των $ \vec{a},\vec{\beta} $.\monades{9}
\item Να βρεθεί ο συντελεστής διεύθυνσης και το μέτρο του διανύσματος $ \vec{\delta}=2\vec{a}-\vec{\gamma} $.\monades{9}
\end{erwthma}
\item Δίνεται παραλληλόγραμμο $ AB\varGamma\varDelta $ με κορυφές $ A(3,-1), B(-2,4) $ και $ \varGamma(5,8) $.
\begin{erwthma}
\item Να βρεθούν οι συντεταγμένες της κορυφής $ \varDelta $.\monades{7}
\item Αν $ N(x+7,2x+2) $ είναι ένα σημείο για το οποίο ισχύει $ \dsx{BN}\parallel\dsx{A\varDelta} $ τότε
\begin{alist}
\item να δείξετε ότι $ x=5 $.\monades{7}
\item να δείξετε ότι $ |\dsx{B\varGamma}|=|\dsx{\varGamma N}| $.\monades{4}
\end{alist}
\item Να βρείτε το μέτρο του $ \dsx{KN} $ όπου $ K $ είναι το κέντρο του παραλληλογράμμου.\monades{7}
\end{erwthma}
\end{thema}
\kaliepityxia
\diarkeia{2}
\end{document}
