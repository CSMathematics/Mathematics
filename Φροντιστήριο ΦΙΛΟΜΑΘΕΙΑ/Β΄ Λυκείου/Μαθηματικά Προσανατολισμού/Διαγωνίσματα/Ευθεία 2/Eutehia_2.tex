\documentclass[ektypwsh]{diag-xelatex}
\usepackage[amsbb]{mtpro2}
\usepackage[no-math,cm-default]{fontspec}
\usepackage{xunicode}
\usepackage{xltxtra}
\usepackage{xgreek}
\usepackage{amsmath}
\usepackage{tikz,tkz-euclide,pgfplots}
\defaultfontfeatures{Mapping=tex-text,Scale=MatchLowercase}
\setmainfont[Mapping=tex-text,Numbers=Lining,Scale=1.0,BoldFont={Minion Pro Bold}]{Minion Pro}
\newfontfamily\scfont{GFS Artemisia}
\font\icon = "Webdings"
\usepackage[amsbb]{mtpro2}
\usepackage[left=2.00cm, right=2.00cm, top=2.00cm, bottom=3.00cm]{geometry}
\xroma{cyan!80!black}
\newcommand{\tss}[1]{\textsuperscript{#1}}
\newcommand{\tssL}[1]{\MakeLowercase\textsuperscript{#1}}
\newlist{rlist}{enumerate}{3}
\setlist[rlist]{itemsep=0mm,label=\roman*.}
\usepackage{mathimatika,multicol,gensymb}


\begin{document}
\titlos{Μαθηματικά προσανατολισμού Θετικών Σπουδών}{ΕΞΙΣΩΣΗ ΕΥΘΕΙΑΣ}
\begin{thema}
\item \textbf{Θεωρία}\\
Να απαντήσετε στις παρακάτω ερωτήσεις.
\begin{rlist}
\item Από ποιόν τύπο δίνεται η απόσταση ενός σημείου $ A(x_0,y_0) $ από μια ευθεία $(\varepsilon)\ :\  Ax+By+\varGamma=0 $;
\item Ποιός είναι ο συντελεστής διεύθυνσης (αν ορίζεται), το παράλληλο και το κάθετο διάνυσμα της ευθείας $(\varepsilon)\ :\  Ax+By+\varGamma=0 $;
\item Ποιός είναι ο συντελεστής διεύθυνσης της ευθείας που διέρχεται από τα σημεία $ A(x_1,y_1) $ και $ B(x_2,y_2) $;
\item Πως ορίζεται ο συντελεστής διεύθυνσης μιας ευθείας;
\end{rlist}
\item \textbf{Εξίσωση ευθείας}\\
Δίνεται τρίγωνο $ AB\varGamma $ με κορυφές $ A(1,1),B(3,-1) $ και $ \varGamma(5,3) $. Να βρεθεί η εξίσωση της ευθείας
\begin{rlist}
\item της διαμέσου $ AM $\monades{1}
\item του ύψους $ B\varDelta $\monades{2}
\item της μεσοκαθέτου της πλευράς $ AB $\monades{2}
\end{rlist}
\item \textbf{Απόσταση}\\
Δίνεται τρίγωνο $ AB\varGamma $ με κορυφές $ A(2,-1),B(3,4) $ και $ \varGamma(0,2) $. Να βρεθεί 
\begin{rlist}
\item το μήκος του ύψους $ BE $\monades{2}
\item η εξίσωση της διχοτόμου της γωνίας $ \hat{A} $\monades{3}
\end{rlist}
\item \textbf{Σύνθετο θέμα}\\
Δίνονται $ (\varepsilon): x-y+1=0 $ και $ (\eta): ax+(1-a)y+1=0 $.
\begin{rlist}
\item Να αποδείξετε ότι οι ευθείες τέμνονται για κάθε $ a\in\mathbb{R} $.\monades{1}
\item Να γράψετε δύο διανύσματα $ \vec{a},\vec{\beta} $ που να είναι παράλληλα με τις ευθείες $ (\epsilon) $ και $ (\eta) $ αντίστοιχα.\\\monades{1}
\item Να βρεθεί η τιμή του $ a $ ώστε η οξεία γωνία των διανυσμάτων να είναι ίση με $ 45\degree $.\monades{1}
\item Να βρείτε το κοινό σημείο $ M $ των δύο ευθειών.\monades{1}
\item Να βρεθεί ο γεωμετρικός τόπος των σημείων $ M $.\monades{1}
\end{rlist}
\end{thema}
\end{document}

