\documentclass[11pt,a4paper,twocolumn]{article}
\usepackage[english,greek]{babel}
\usepackage[utf8]{inputenc}
\usepackage{nimbusserif}
\usepackage[T1]{fontenc}
\usepackage[left=1.50cm, right=1.50cm, top=2.00cm, bottom=2.00cm]{geometry}
\usepackage{amsmath}
\let\myBbbk\Bbbk
\let\Bbbk\relax
\usepackage[amsbb,subscriptcorrection,zswash,mtpcal,mtphrb,mtpfrak]{mtpro2}
\usepackage{graphicx,multicol,multirow,enumitem,tabularx,mathimatika,gensymb,venndiagram,hhline,longtable,tkz-euclide,fontawesome5,eurosym,tcolorbox,tabularray}
\usepackage[explicit]{titlesec}
\tcbuselibrary{skins,theorems,breakable}
\newlist{rlist}{enumerate}{3}
\setlist[rlist]{itemsep=0mm,label=\roman*.}
\newlist{alist}{enumerate}{3}
\setlist[alist]{itemsep=0mm,label=\alph*.}
\newlist{balist}{enumerate}{3}
\setlist[balist]{itemsep=0mm,label=\bf\alph*.}
\newlist{Alist}{enumerate}{3}
\setlist[Alist]{itemsep=0mm,label=\Alph*.}
\newlist{bAlist}{enumerate}{3}
\setlist[bAlist]{itemsep=0mm,label=\bf\Alph*.}
\newlist{askhseis}{enumerate}{3}
\setlist[askhseis]{label={\Large\thesection}.\arabic*.}
\renewcommand{\textstigma}{\textsigma\texttau}
\newlist{thema}{enumerate}{3}
\setlist[thema]{label=\bf\large{ΘΕΜΑ \textcolor{black}{\Alph*}},itemsep=0mm,leftmargin=0cm,itemindent=18mm}
\newlist{erwthma}{enumerate}{3}
\setlist[erwthma]{label=\bf{\large{\textcolor{black}{\Alph{themai}.\arabic*}}},itemsep=0mm,leftmargin=0.8cm}

%%--------- BOLD Μαθηματικά
%\newcommand{\bmath}[1]{\textbf{\boldmath{#1}}}
%
%%-------- ΤΡΙΓΩΝΟΜΕΤΡΙΚΟΙ ΑΡΙΘΜΟΙ -----------
%\newcommand{\hm}[1]{\textrm{ημ}#1}
%\newcommand{\syn}[1]{\textrm{συν}#1}
%\newcommand{\ef}[1]{\textrm{εφ}#1}
%\newcommand{\syf}[1]{\textrm{σφ}#1}
%%--------------------------------------------

\newcommand{\kerkissans}[1]{{\fontfamily{maksf}\selectfont \textbf{#1}}}
\renewcommand{\textdexiakeraia}{}

\usepackage[
backend=biber,
style=alphabetic,
sorting=ynt
]{biblatex}

\DeclareTblrTemplate{caption}{nocaptemplate}{}
\DeclareTblrTemplate{capcont}{nocaptemplate}{}
\DeclareTblrTemplate{contfoot}{nocaptemplate}{}
\NewTblrTheme{mytabletheme}{
  \SetTblrTemplate{caption}{nocaptemplate}{}
  \SetTblrTemplate{capcont}{nocaptemplate}{}
  \SetTblrTemplate{contfoot}{nocaptemplate}{}
}

\NewTblrEnviron{mytblr}
\SetTblrStyle{firsthead}{font=\bfseries}
\SetTblrStyle{firstfoot}{fg=red2}
\SetTblrOuter[mytblr]{theme=mytabletheme}
\SetTblrInner[mytblr]{
rowspec={t{7mm}},columns = {c},
  width = 0.85\linewidth,
  row{odd} = {bg=red9,fg=black,ht=8mm},
 row{even} = {bg=red7,fg=black,ht=8mm},
hlines={white},vlines={white},
row{1} = {bg=red4, fg=white, font=\bfseries\fontfamily{maksf}},rowhead = 1,
  hline{2} = {.7mm}, % midrule  
}
\newcounter{askhsh}
\setcounter{askhsh}{1}
\newcommand{\askhsh}{\large\theaskhsh.\ \addtocounter{askhsh}{1}}

\titleformat{\section}{\Large}{\kerkissans{\thesection}}{10pt}{\Large\kerkissans{#1}}

\setlength{\columnsep}{5mm}
\titleformat{\paragraph}
{\large}%
{}{0em}%
{\textcolor{red!80!black}{\faSquare\ \ \kerkissans{\bmath{#1}}}}
\setlength{\parindent}{0pt}

\begin{document}
\twocolumn[{
\centering
\kerkissans{{\huge Απόσταση σημείου από ευθεία - Εμβαδόν τριγώνου}\\\vspace{3mm} {\Large ΑΣΚΗΣΕΙΣ}}\vspace{5mm}}]
\paragraph{Απόσταση σημείου από ευθεία}
\askhsh Να βρεθεί η απόσταση του σημείου $A(1,-3)$ από τις ακόλουθες ευθείες.
\begin{multicols}{2}
\begin{alist}
\item $2x+y-1=0$
\item $3x-4y+7=0$
\item $y=5x+4$
\item $4x+y-1=0$
\item $x=4$
\item $y=-2$
\end{alist}
\end{multicols}
\askhsh Να βρείτε την απόσταση του καθενός από τα παρακάτω σημεία, από την ευθεία $5x+12y+7=0$.
\begin{multicols}{3}
\begin{alist}
\item $A(-1,2)$
\item $B(5,-2)$
\item $\varGamma(3,4)$
\item $\varDelta(0,-7)$
\item $E(6,0)$
\item $Z(1,-1)$
\end{alist}
\end{multicols}
\askhsh Δίνεται τρίγωνο $AB\varGamma$ με κορυφές $A(2,1), B(-1,4)$ και $\varGamma(5,-2)$. Να βρεθεί
\begin{alist}
\item η εξίσωση της ευθείας $AB$.
\item το μήκος του ύψους $\varGamma Z$.
\item το μήκος του ύψους $BE$.
\end{alist}
\askhsh Δίνονται οι ευθείες $\varepsilon_1:3x+4y-2=0$ και $\varepsilon_2:5x-12y+10=0$.
\begin{alist}
\item Να υπολογίσετε τις εξισώσεις των διχοτόμων των γωνιών που σχηματίζουν οι δύο ευθείες.
\item Ποια από τις παραπάνω είναι η διχοτόμος της οξείας γωνίας που σχηματίζουν οι $\varepsilon_1,\varepsilon_2$?
\end{alist}
\paragraph{Απόσταση μεταξύ παράλληλων}
\askhsh Σε καθένα από τα παρακάτω ερωτήματα, να υπολογίσετε την απόσταση μεταξύ των παράλληλων ευθειών $\varepsilon_1,\varepsilon_2$.
\begin{alist}
\item $\varepsilon_1:y=3x-1$ και $\varepsilon_2:y=3x+5$
\item $\varepsilon_1:3x-4y+12=0$ και $\varepsilon_2:3x-4y-7=0$
\item $\varepsilon_1:x-2y+1=0$ και $\varepsilon_2:-x+2y-4=0$
\item $\varepsilon_1:y=3$ και $\varepsilon_2:y=-4$
\item $\varepsilon_1:x=-2$ και $\varepsilon_2:x=5$
\end{alist}
\askhsh Δίνονται οι ευθείες $\varepsilon_1:\lambda x-(\lambda-2)y+4=0$ και $\varepsilon_2:(\lambda+2)+3y-(2\lambda-1)=0$ με $\lambda\in\mathbb{R}$.
\begin{alist}
\item Να βρεθούν οι τιμές του $\lambda$ ώστε οι ευθείες να είναι παράλληλες.
\item Για αυτές τις τιμές του $\lambda$, να υπολογίσετε την απόσταση μεταξύ των δύο ευθειών.
\item Για $\lambda=1$, να βρεθεί η εξίσωση της μεσοπαράλληλης $\varepsilon$ των ευθειών $\varepsilon_1,\varepsilon_2$.
\end{alist}
\askhsh Έστω παραλληλόγραμμο $AB\varGamma\varDelta$ με κορυφές $A(2,3),B(7,1),\varGamma(8,-3)$.
\begin{alist}
\item Να βρεθούν οι συντεταγμένες της κορυφής $\varDelta$.
\item Να υπολογίσετε το μήκος της πλευράς $AB$ καθώς και την εξίσωση της ευθείας $AB$.
\item Υπολογίστε το μήκος του ύψους $\varGamma K$, καθώς και το εμβαδόν του παραλληλογράμμου.
\end{alist}
\paragraph{Εμβαδόν τριγώνου}
\askhsh Να βρεθεί το εμβαδόν του τριγώνου $AB\varGamma$ με κορυφές:
\begin{alist}
\item $A(1,3),B(-2,-1)$ και $\varGamma(4,4)$.
\item $A(0,2),B(5,1)$ και $\varGamma(-3,4)$.
\item $A(-4,7),B(2,-3)$ και $\varGamma(3,5)$.
\item $A(-4,0),B(8,5)$ και $\varGamma(7,3)$.
\end{alist}
\askhsh Δίνεται τρίγωνο $AB\varGamma$ με κορυφές $A(3,-1), B(4,-2)$ και $\varGamma(1,7)$. Να βρεθεί
\begin{alist}
\item η εξίσωση της ευθείας $B\varGamma$.
\item το μήκος του ύψους $A\varDelta$.
\item το μήκος της πλευράς $B\varGamma$.
\item το εμβαδόν του τριγώνου, κάνοντας χρήση του αντίστοιχου τύπου. Επαληθεύστε το αποτέλεσμα χρησιμοποιώντας τον κλασσικό γεωμετρικό τύπο του εμβαδού τριγώνου.
\end{alist}
\askhsh Δίνεται κύκλος με κέντρο το σημείο $K(2,-1)$ και ακτίνα $\rho=5$. Να εξετάσετε ποιες από τις ακόλουθες ευθείες εφάπτονται στον κύκλο.
\begin{multicols}{2}
\begin{alist}
\item $3x+4y-27=0$
\item $y=\dfrac{4}{3}x+\dfrac{14}{3}$
\item $4x+3y-10=0$
\item $x=-3$
\end{alist}
\end{multicols}
\askhsh Δίνεται κύκλος $(K,4)$ με κέντρο $K(3,2)$. Εξετάστε τη σχετική θέση του κύκλου με καθεμία από τις παρακάτω ευθείες.
\begin{multicols}{2}
\begin{alist}
\item $2x+y-5=0$
\item $x-4y-13=0$
\item $x=-1$
\item $y=4$
\end{alist}
\end{multicols}
\askhsh Ένα παραλληλόγραμμο $AB\varGamma\varDelta$ έχει κορυφές $A(2,-1),B(3,4)$ και $\varGamma(-4,2)$. Να υπολογίσετε το εμβαδόν του παραλληλογράμμου.\\\\
\askhsh
\end{document}
