\documentclass[internet]{frontisthrio}
\usepackage[amsbb,subscriptcorrection,zswash,mtpcal,mtphrb,mtpfrak]{mtpro2}
\usepackage[no-math,cm-default]{fontspec}
\usepackage{amsmath}
\usepackage{xunicode}
\usepackage{xgreek}
\let\hbar\relax
\defaultfontfeatures{Mapping=tex-text,Scale=MatchLowercase}
\setmainfont[Mapping=tex-text,Numbers=Lining,Scale=1.0,BoldFont={Tinos Bold}]{Tinos}
\newfontfamily\scfont{GFS Artemisia}
\font\OnPar="Century Gothic Bold" at 10pt
\usepackage{fontawesome5}
\newfontfamily{\FA}{fontawesome.otf}
\usepackage[most]{tcolorbox}
\xroma{red!70!black}
%------TIKZ - ΣΧΗΜΑΤΑ - ΓΡΑΦΙΚΕΣ ΠΑΡΑΣΤΑΣΕΙΣ ----
\usepackage{tikz,pgfplots}
\usepackage{tkz-euclide}
\usetkzobj{all}
\usepackage[framemethod=TikZ]{mdframed}
\usetikzlibrary{decorations.pathreplacing}
\tkzSetUpPoint[size=7,fill=white]
%-----------------------
\usepackage{calc,tcolorbox}
\tcbuselibrary{skins,theorems,breakable}
\usepackage{hhline}
\usepackage[explicit]{titlesec}
\usepackage{graphicx}
\usepackage{multicol}
\usepackage{multirow}
\usepackage{tabularx}
\usetikzlibrary{backgrounds}
\usepackage{sectsty}
\sectionfont{\centering}
\usepackage{enumitem,longtable}
\usepackage{adjustbox}
\usepackage{mathimatika,gensymb,eurosym,wrap-rl}
\usepackage{systeme,regexpatch}
%-------- ΜΑΘΗΜΑΤΙΚΑ ΕΡΓΑΛΕΙΑ ---------
\usepackage{mathtools}
%----------------------
%-------- ΠΙΝΑΚΕΣ ---------
\usepackage{booktabs}
%----------------------
%----- ΥΠΟΛΟΓΙΣΤΗΣ ----------
%\usepackage{calculator}
%----------------------------

%------------------------------------------
\newcommand{\tss}[1]{\textsuperscript{#1}}
\newcommand{\tssL}[1]{\MakeLowercase{\textsuperscript{#1}}}
%---------- ΛΙΣΤΕΣ ----------------------
\newlist{bhma}{enumerate}{3}
\setlist[bhma]{label=\bf\textit{\arabic*\textsuperscript{o}\;Βήμα :},leftmargin=0cm,itemindent=1.8cm,ref=\bf{\arabic*\textsuperscript{o}\;Βήμα}}
\newlist{rlist}{enumerate}{3}
\setlist[rlist]{itemsep=0mm,label=\roman*.}
\newlist{brlist}{enumerate}{3}
\setlist[brlist]{itemsep=0mm,label=\bf\roman*.}
\newlist{tropos}{enumerate}{3}
\setlist[tropos]{label=\bf\textit{\arabic*\textsuperscript{oς}\;Τρόπος :},leftmargin=0cm,itemindent=2.3cm,ref=\bf{\arabic*\textsuperscript{oς}\;Τρόπος}}
% Αν μπει το bhma μεσα σε tropo τότε
%\begin{bhma}[leftmargin=.7cm]
\tkzSetUpPoint[size=7,fill=white]
\tikzstyle{pl}=[line width=0.3mm]
\tikzstyle{plm}=[line width=0.4mm]
\usepackage{etoolbox}
\makeatletter
\renewrobustcmd{\anw@true}{\let\ifanw@\iffalse}
\renewrobustcmd{\anw@false}{\let\ifanw@\iffalse}\anw@false
\newrobustcmd{\noanw@true}{\let\ifnoanw@\iffalse}
\newrobustcmd{\noanw@false}{\let\ifnoanw@\iffalse}\noanw@false
\renewrobustcmd{\anw@print}{\ifanw@\ifnoanw@\else\numer@lsign\fi\fi}
\makeatother
\let\Bbbk\relax
\usepackage{enumitem}
\usepackage{lipsum}
\let\Bbbk\relax
\newlist{todolist}{itemize}{2}
\setlist[todolist]{label=\Large$\square$}

\tikzset{>=latex}
\newtcolorbox{mybox}[2][]{colback=white,
colframe=red!75!black,fonttitle=\Large\bfseries,
colbacktitle=red!20!white,coltitle=black,enhanced,breakable,sharp corners,boxrule=0.3mm,center title,boxsep=3mm,top=1mm,subtitle style={fonttitle=\normalsize\bfseries},
title=#2,#1}
\setlength{\columnsep}{1cm}
\setlength{\columnseprule}{0.2pt}
\tcbset{mysubtitle/.style={subtitle style={fontupper={\OnPar\color{black}},top=0pt,colback={white},boxrule=1pt},top=0pt}}

\newtcolorbox{myleftbox}[2][]{nobeforeafter, title=#2,boxrule=0pt,colframe=black,coltitle=black,right=-3mm,left=5mm,left skip=0mm,colbacktitle=white,colback=white,#1,sharp corners,grow to left by=0.68cm,titlerule=0.2mm,fonttitle=\OnPar}

\newtcolorbox{myrightbox}[2][]{nobeforeafter, title=#2,boxrule=0pt,colframe=black,coltitle=black,right=-2mm,left skip=6mm,colbacktitle=white,colback=white,#1,leftrule=0.2mm,sharp corners,right skip=0mm,titlerule=0.2mm,fonttitle=\OnPar}
\usepackage{fancyhdr}
\pagestyle{fancy}

\pagestyle{fancy}
\fancyhf{}
\fancyheadoffset{0cm}
\renewcommand{\headrulewidth}{0pt} 
\renewcommand{\footrulewidth}{0pt}
\fancyhead[R]{
  \color{lightgray}{}
  }
\fancyhead[R]{
 \color{gray} Μάθημα - Τάξη\hspace{1em}\color{lightgray}{\vline}\hspace{1em}\color{gray}\thepage
  }
\fancyhead[L]{
 \color{gray}\leftmark
  }
\fancypagestyle{plain}{%
  \fancyhf{}%
  \fancyhead[R]{\leftmark\hspace{1em}\color{lightgray}{\vline}\hspace{1em}\color{gray}\thepage}%
  }
\renewcommand{\sectionmark}[1]{\markboth{#1}{#1}}
\newcommand{\myitem}{\stepcounter{enumi}\item[\raisebox{0.5mm}{\faExclamationTriangle}\ \Large$\square$]}

\newlist{arithmisi}{enumerate}{2}
\setlist[arithmisi]{itemsep=0mm,label=\textcolor{\xrwma}{\textbf{\textit{{\Large{\thesection}}.\arabic*}}}}



\ekthetesdeiktes

\begin{document}
\section{Η έννοια του διανύσματος}
\begin{flushright}
\faCalendar* Ημερομηνία: .......................
\end{flushright}
\begin{mybox}[mysubtitle]{Πίνακας Ύλης}
\begin{tcbraster}[raster columns=1,raster equal height]
\begin{myleftbox}{Ορισμοί - Βασικές έννοιες\ \ \faBook}
\begin{enumerate}[itemsep=0mm]
\item Διάνυσμα
\item Αρχή και πέρας
\item Στοιχεία διανύσματος: Μέτρο - διεύθυνση - φορά
\item Φορέας διανύσματος
\item Μηδενικό  - Μοναδιαίο διάνυσμα
\item Παράλληλα διανύσματα
\item Ομόρροπα διανύσματα
\item Αντίρροπα διανύσματα
\item Ίσα διανύσματα
\item Αντίθετα διανύσματα
\end{enumerate}
\end{myleftbox}
\end{tcbraster}
\tcbsubtitle{Είδη ασκήσεων - Τι πρέπει να γνωρίζω\ \ \faPen}
\begin{multicols}{2}
\begin{todolist}[itemsep=0mm]
\item Εύρεση παράλληλων διανυσμάτων
\item Εύρεση ομόρροπων και αντίρροπων διανυσμάτων
\item Εύρεση ίσων και αντίθετων διανυσμάτων
\item Υπολογισμός γωνίας διανυσμάτων
\end{todolist}
\end{multicols}
\tcbsubtitle{Τυπολόγιο - Συμβολισμοί\ \ \faFile*}
\begin{multicols}{2}
\begin{enumerate}[itemsep=0mm]
\item Διάνυσμα $\vec{a}$ ή $\DianysmaBelos{AB}$
\item Μέτρο διανύσματος $|\vec{a}|, |\DianysmaBelos{AB}|$
\item Μηδενικό διάνυσμα : $\vec{a}=\vec{0}$
\item Μοναδιαίο διάνυσμα : $|\vec{a}|=1$
\item Ομόρροπα διανύσματα : $\DianysmaBelos{a}\upuparrows \DianysmaBelos{\beta}$
\item Αντίρροπα διανύσματα : $\DianysmaBelos{a}\updownarrows \DianysmaBelos{\beta}$
\item Γωνία διανυσμάτων :\\
$\theta=\GwniaDianysmatwn{a}{\beta},\ \theta\in[0,\pi]$
\end{enumerate}
\end{multicols}
\end{mybox}
\newpage
\section{Πρόσθεση διανυσμάτων}
\begin{flushright}
\faCalendar* Ημερομηνία: .......................
\end{flushright}
\begin{mybox}[mysubtitle]{Πίνακας Ύλης}
\begin{tcbraster}[raster columns=2,raster equal height]
\begin{myleftbox}{Ορισμοί - Βασικές έννοιες\ \ \faBook}
\begin{enumerate}[itemsep=0mm]
\item Πρόσθεση διαδοχικών διανυσμάτων
\item Κανόνας παραλληλογράμμου
\item Αφαίρεση διανυσμάτων
\item Διάνυσμα θέσης
\item Σημείο αναφοράς
\end{enumerate}
\end{myleftbox}
\begin{myrightbox}{Θεωρήματα - Ιδιότητες\ \ \faTools}
\begin{enumerate}[itemsep=0mm]
\item Ιδιότητες πρόσθεσης
\item Διαφορά διανυσματικών ακτίνων
\item Μέτρο αθροίσματος - Τριγωνική ανισότητα
\item Κριτήριο ομόρροπων και αντίρροπων διανυσμάτων
\end{enumerate}
\end{myrightbox}
\end{tcbraster}
\tcbsubtitle{Είδη ασκήσεων - Τι πρέπει να γνωρίζω\ \ \faPen}
\begin{multicols}{2}
\begin{todolist}[itemsep=0mm]
\myitem Πρόσθεση και αφαίρεση διαδοχικών διανυσμάτων
\myitem Πρόσθεση και αφαίρεση διανυσμάτων με κανόνα παραλληλογράμμου
\myitem Απόδειξη διανυσματικής ισότητας
\item Απόδειξη ότι δύο σημεία ταυτίζονται
\item Μέσο ευθύγραμμου τμήματος
\myitem Απόδειξη ότι ένα τετράπλευρο είναι παραλληλόγραμμο
\item Προσδιορισμός σημείου
\item Τριγωνική ανισότητα
\item Κριτήριο ομόρροπων και αντίρροπων διανυσμάτων
\item Γεωμετρικοί τόποι
\end{todolist}
\end{multicols}
\tcbsubtitle{Τυπολόγιο - Συμβολισμοί\ \ \faFile*}
\begin{multicols}{2}
\begin{enumerate}[itemsep=0mm]
\item Πρόσθεση διανυσμάτων $\vec{a}+\vec{\beta}$
\item Αφαίρεση διανυσμάτων : $\vec{a}-\vec{\beta}$
\item Διάνυσμα θέσης σημείου $ M $: $ \DianysmaBelos{OM} $
όπου $O$ σημείο αναφοράς.
\item Τριγωνική ανισότητα:\\$\left||\vec{a}|-|\vec{\beta}|\right|\leq\left|\vec{a}+\vec{\beta}\right|\leq |\vec{a}|+|\vec{\beta}|$
\end{enumerate}
\end{multicols}
\tcbsubtitle{Πίνακες - Σχήματα}
\begin{center}
\begin{tikzpicture}
\Dianysma[color=black]{0,0}{2.7,.5}{E}{Z}
\Dianysma{0,0}{1.5,1}{A}{B}
\Dianysma{1.5,1}{2.7,.5}{C}{D}
\tkzLabelPoint[above](A){$O$}
\tkzLabelPoint[above](B){$A$}
\tkzLabelPoint[right](D){$B$}
\node at (.7,.77){$\vec{a}$};
\node at (2.2,1.04){$\vec{\beta}$};
\node at (1.4,-.05){$\vec{a}+\vec{\beta}$};
\node at (1,1.8){$\vec{a}+\vec{\beta}=\overrightarrow{OA}+\overrightarrow{AB}=\overrightarrow{OB}$};
\node at (1,2.3){\textbf{Κανόνας διαδοχικών διανυσμάτων}};
\end{tikzpicture}\qquad
\begin{tikzpicture}
\Dianysma[color=black]{0,0}{3.4,1.5}{E}{Z}
\Dianysma{0,0}{1,1.5}{A}{B}
\Dianysma{0,0}{2.4,0}{C}{D}
\draw[dashed] (2.4,0)--(3.4,1.5);
\draw[dashed] (1,1.5)--(3.4,1.5);
\tkzLabelPoint[left](A){$O$}
\tkzLabelPoint[above](B){$A$}
\tkzLabelPoint[right](D){$B$}
\tkzLabelPoint[right,yshift=1mm](Z){$M$}
\node at (.2,.77){$\vec{a}$};
\node at (1.2,-.33){$\vec{\beta}$};
\node[rotate=23.8] at (1.4,.9){$\vec{a}+\vec{\beta}$};
\node at (1.9,2.1){$\vec{a}+\vec{\beta}=\overrightarrow{OA}+\overrightarrow{OB}=\overrightarrow{OM}$};
  \node at (1.9,2.6){\textbf{Κανόνας παραλληλογράμμου}};
\end{tikzpicture}\end{center}
\begin{center}
\begin{tikzpicture}
\Dianysma[color=black]{0,0}{2.7,.5}{E}{Z}
\Dianysma{0,0}{1.5,1}{A}{B}
\Dianysma{1.5,1}{2.7,.5}{C}{D}
\tkzLabelPoint[above](A){$O$}
\tkzLabelPoint[above](B){$B$}
\tkzLabelPoint[right](D){$A$}
\node at (.7,.77){$\vec{a}$};
\node at (2.2,1.04){$-\vec{\beta}$};
\node at (1.4,-.05){$\vec{a}-\vec{\beta}$};
\node at (4,-1){$\vec{a}-\vec{\beta}=\vec{a}+\left( -\vec{\beta}\right)$};
\Dianysma[color=black]{7.4,0}{6,1.5}{O}{P}
\Dianysma{5,0}{6,1.5}{K}{L}
\Dianysma{5,0}{7.4,0}{M}{N}
\draw[dashed] (7.4,0)--(8.4,1.5);
\draw[dashed] (6,1.5)--(8.4,1.5)node[xshift=2mm,yshift=2mm](S){$M$};
\tkzLabelPoint[left](K){$O$}
\tkzLabelPoint[above](L){$A$}
\tkzLabelPoint[right](N){$B$}
\node at (5.2,.77){$\vec{a}$};
\node at (6.2,-.33){$\vec{\beta}$};
\node[rotate=-49] at (7.1,.77){$\vec{a}-\vec{\beta}$};
\node at (7,2.4){\textbf{Κανόνας παραλληλογράμμου}};
\node at (1.7,1.9){\textbf{Κανόνας διαδοχικών διανυσμάτων}};
\end{tikzpicture}
\end{center}
\begin{center}
\textbf{Ιδιότητες πρόσθεσης διανυσμάτων}\\
\begin{tabular}{cc}
\hline \rule[-2ex]{0pt}{5.5ex} \textbf{Ιδιότητα} & \textbf{Συνθήκη}  \\ 
\hhline{==} \rule[-2ex]{0pt}{5.5ex} Αντιμεταθετική & $ \vec{a}+\vec{\beta}=\vec{\beta}+\vec{a} $  \\ 
 \rule[-2ex]{0pt}{5.5ex} Προσεταιριστική & $ \vec{a}+\left( \vec{\beta}+\vec{\gamma}\right) =\left( \vec{a}+\vec{\beta}\right) +\vec{\gamma} $  \\ 
 \rule[-2ex]{0pt}{5.5ex} Ουδέτερο στοιχείο & $ \vec{a}+\vec{0}=\vec{a} $  \\ 
  \rule[-2ex]{0pt}{5.5ex} Αντίθετα διανύσματα & $ \vec{a}+(-\vec{a})=\vec{0} $  \\
\hline 
\end{tabular}\end{center}
\end{mybox}
\newpage
\section{Γινόμενο αριθμού με διάνυσμα}
\begin{flushright}
\faCalendar* Ημερομηνία: .......................
\end{flushright}
\begin{mybox}[mysubtitle]{Πίνακας Ύλης}
\begin{tcbraster}[raster columns=2,raster equal height]
\begin{myleftbox}{Ορισμοί - Βασικές έννοιες\ \ \faBook}
\begin{enumerate}[itemsep=0mm]
\item Γινόμενο αριθμού με διάνυσμα
\item Γραμμικός συνδυασμός διανυσμάτων
\end{enumerate}
\end{myleftbox}
\begin{myrightbox}{Θεωρήματα - Ιδιότητες\ \ \faTools}
\begin{enumerate}[itemsep=0mm]
\item Ιδιότητες γινομένου
\item Συνθήκη παραλληλίας
\item Διανυσματική ακτίνα μέσου
\end{enumerate}
\end{myrightbox}
\end{tcbraster}
\tcbsubtitle{Είδη ασκήσεων - Τι πρέπει να γνωρίζω\ \ \faPen}
\begin{multicols}{2}
\begin{todolist}[itemsep=0mm]
\myitem Απόδειξη - Έλεγχος παραλληλίας διανυσμάτων
\item Απόδειξη διανυσματικής ισότητας
\item 
\end{todolist}
\end{multicols}
\tcbsubtitle{Τυπολόγιο - Συμβολισμοί\ \ \faFile*}
\begin{multicols}{2}
\begin{enumerate}[itemsep=0mm]
\item Γινόμενο αριθμού με διάνυσμα: $\lambda\cdot \vec{a}$
\item Συνθήκη παραλληλίας:\\$\vec{a}\parallel\vec{\beta}\Leftrightarrow \vec{a}=\lambda\cdot\vec{\beta}$
\item $\vec{a}\upuparrows\vec{\beta}\Leftrightarrow \vec{a}=\lambda\vec{\beta}\ \ \textrm{και}\ \ \lambda>0$
\item $\vec{a}\updownarrows\vec{\beta}\Leftrightarrow \vec{a}=\lambda\vec{\beta}\ \ \textrm{και}\ \ \lambda<0$
\item Διανυσματική ακτίνα μέσου:\\$\DianysmaBelos{OM}=\frac{\DianysmaBelos{OA}+\DianysmaBelos{OB}}{2}$
\item Γραμμικός συνδυασμός: $\vec{\gamma}=\lambda\vec{a}+\mu\vec{\beta}$
\end{enumerate}
\end{multicols}
\tcbsubtitle{Πίνακες - Διαγράμματα}
\begin{center}
\textbf{Ιδιότητες γινομένου αριθμού με διάνυσμα}\\
\begin{longtable}{cc}
\hline \rule[-2ex]{0pt}{5.5ex} \textbf{Ιδιότητα} & \textbf{Συνθήκη} \\ 
\hhline{==} \rule[-2ex]{0pt}{5.5ex} Επιμεριστική (ως προς αριθμό) & $ \lambda\left( \vec{a}\pm\vec{\beta}\right)=\lambda\cdot\vec{a}\pm\lambda\cdot\vec{\beta} $ \\ 
\rule[-2ex]{0pt}{5.5ex} Επιμεριστική (ως προς διάνυσμα) & $ \left( \lambda\pm\mu\right)\cdot\vec{a}=\lambda\cdot\vec{a}\pm\mu\cdot\vec{a} $ \\
\rule[-2ex]{0pt}{5.5ex} Προσεταιριστική & $ \lambda\left( \mu\vec{a}\right)=\left( \lambda\cdot\mu\right)\cdot\vec{a} $ \\ 
\rule[-2ex]{0pt}{5.5ex} Μηδενικό γινόμενο & $ \lambda\cdot\vec{a}=\vec{0}\Leftrightarrow \lambda=0 \ \textrm{ή}\ \vec{a}=\vec{0} $ \\ 
\rule[-2ex]{0pt}{5.5ex} Πρόσημο γινομένου & $ \left( -\lambda\cdot\vec{a}\right)=(-\lambda)\cdot\vec{a}=-\left( \lambda\cdot\vec{a}\right)  $ \\ 
\rule[-2ex]{0pt}{5.5ex} Νόμος διαγραφής (ως προς διάνυσμα) & Αν $ \lambda\cdot\vec{a}=\mu\cdot\vec{a} $ και $ \vec{a}\neq0 $ τότε $ \lambda=\mu $ \\ 
\rule[-2ex]{0pt}{5.5ex} Νόμος διαγραφής (ως προς αριθμό) & Αν $ \lambda\cdot\vec{a}=\lambda\cdot\vec{\beta} $ και $ \lambda\neq0 $ τότε $ \vec{a}=\vec{\beta} $\\ 
\hline 
\end{longtable}
\end{center} 
\end{mybox}
\newpage
\section{Συντεταγμένες διανύσματος}
\begin{flushright}
\faCalendar* Ημερομηνία: .......................
\end{flushright}
\begin{mybox}[mysubtitle]{Πίνακας Ύλης}
\begin{tcbraster}[raster columns=2,raster equal height]
\begin{myleftbox}{Ορισμοί - Βασικές έννοιες\ \ \faBook}
\begin{enumerate}[itemsep=0mm]
\item Συντεταγμένες διανύσματος
\item Συντελεστής διεύθυνσης διανύσματος
\item Ορίζουσα διανυσμάτων
\end{enumerate}
\end{myleftbox}
\begin{myrightbox}{Θεωρήματα - Ιδιότητες\ \ \faTools}
\begin{enumerate}[itemsep=0mm]
\item Ίσα διανύσματα
\item Οριζόντια - Κατακόρυφα διανύσματα
\item Συντεταγμένες γραμμικού συνδυασμού
\item Συντεταγμένες μέσου τμήματος
\item Συντεταγμένες διανύσματος με γνωστά άκρα
\item Συνθήκες παραλληλίας διανυσμάτων
\item Μέτρο διανύσματος
\item Απόσταση σημείων
\end{enumerate}
\end{myrightbox}
\end{tcbraster}
\tcbsubtitle{Είδη ασκήσεων - Τι πρέπει να γνωρίζω\ \ \faPen}
\begin{multicols}{2}
\begin{todolist}[itemsep=0mm]
\myitem 
\end{todolist}
\end{multicols}
\tcbsubtitle{Τυπολόγιο - Συμβολισμοί\ \ \faFile*}
\begin{multicols}{2}
\begin{enumerate}[itemsep=0mm,leftmargin=5mm]
\item Συντεταγμένες διανύσματος: $\vec{a}=(x,y)$
\item $\lambda=\frac{y}{x}$
\item $\vec{a}=\vec{\beta}\Rightarrow x_1=x_2\ \ \textrm{και}\ \ y_1=y_2$
\item $\det(\vec{a},\vec{\beta})=\begin{vmatrix}
 x_1 & y_1 \\
x_2 & y_2
\end{vmatrix}=x_1y_2-x_2y_1$
\item $x_M=\frac{x_A+x_B}{2}$ και $y_M=\frac{y_A+y_B}{2}$
\item $\DianysmaBelos{AB}=(x_B-x_A,y_B-y_A)$
\item $\vec{a}\parallel\vec{\beta}\Leftrightarrow \lambda_{\vec{a}}=\lambda_{\vec{\beta}}$
\item $\vec{a}\parallel\vec{\beta}\Leftrightarrow \det(\vec{a},\vec{\beta})=0$
\item $|\vec{a}|\sqrt{x^2+y^2}$
\item $AB=|\DianysmaBelos{AB}|=\sqrt{(x_B-x_A)^2+(y_B-y_A)^2}$
\end{enumerate}
\end{multicols}
\tcbsubtitle{Πίνακες - Διαγράμματα}
\begin{center}
\textbf{Συντεταγμένες γραμμικού συνδυασμού}\\
\begin{tabular}{cc}
\hline \rule[-2ex]{0pt}{5.5ex} \textbf{Πράξη} & \textbf{Συντεταγμένες} \\ 
\hhline{==} \rule[-2ex]{0pt}{5.5ex} Άθροισμα & $ \vec{a}+\vec{\beta}=(x_1,y_1)+(x_2,y_2)=(x_1+x_2,y_1+y_2) $ \\ 
 \rule[-2ex]{0pt}{5.5ex} Πολλαπλασιασμός & $ \lambda\cdot\vec{a}=\lambda(x_1,y_1)=(\lambda x_1,\lambda y_1) $ \\ 
 \rule[-2ex]{0pt}{5.5ex} Γραμμικός συνδυασμός & $ \lambda\vec{a}+\mu\vec{\beta}=\lambda(x_1,y_1)+\mu(x_2,y_2)=(\lambda x_1+\mu x_2,\lambda y_1+\mu y_2) $ \\ 
\hline 
\end{tabular}
\end{center} 
\end{mybox}
\newpage
\section{Εσωτερικό γινόμενο}
\begin{flushright}
\faCalendar* Ημερομηνία: .......................
\end{flushright}
\begin{mybox}[mysubtitle]{Πίνακας Ύλης}
\begin{tcbraster}[raster columns=2,raster equal height]
\begin{myleftbox}{Ορισμοί - Βασικές έννοιες\ \ \faBook}
\begin{enumerate}[itemsep=0mm]
\item Εσωτερικό γινόμενο διανυσμάτων
\item Εσωτερικό γινόμενο - Αναλυτικός τύπος
\end{enumerate}
\end{myleftbox}
\begin{myrightbox}{Θεωρήματα - Ιδιότητες\ \ \faTools}
\begin{enumerate}[itemsep=0mm]
\item Ιδιότητες εσωτερικού γινομένου
\item Συνθήκη καθετότητας διανυσμάτων
\item Συνημίτονο γωνίας διανυσμάτων
\end{enumerate}
\end{myrightbox}
\end{tcbraster}
\tcbsubtitle{Είδη ασκήσεων - Τι πρέπει να γνωρίζω\ \ \faPen}
\begin{multicols}{2}
\begin{todolist}[itemsep=0mm]
\myitem 
\end{todolist}
\end{multicols}
\tcbsubtitle{Τυπολόγιο - Συμβολισμοί\ \ \faFile*}
\begin{multicols}{2}
\begin{enumerate}[itemsep=0mm,leftmargin=5mm]
\item Εσωτερικό γινόμενο: \\$\vec{a}\cdot\vec{\beta}=|\vec{a}||\vec{\beta}|\syn{\GwniaDianysmatwn{a}{\beta}}$
\item Αναλυτικός τύπος γινομένου:\\$\vec{a}\cdot\vec{\beta}=x_1x_2+y_1y_2$
\end{enumerate}
\end{multicols}
\tcbsubtitle{Πίνακες - Διαγράμματα}
\begin{center}
\textbf{Ιδιότητες εσωτερικού γινομένου}\\
\begin{longtable}{cc}
\hline \rule[-2ex]{0pt}{5.5ex} \textbf{Ιδιότητα} & \textbf{Συνθήκη} \\ 
\hhline{==}  \rule[-2ex]{0pt}{5.5ex} \textbf{Κάθετα διανύσματα} & Αν $ \vec{a}\bot\vec{\beta}\Leftrightarrow \vec{a}\cdot\vec{\beta}=0 $ και $ \lambda_{\vec{a}}\cdot\lambda_{\vec{\beta}}=-1 $ \\ 
 \rule[-2ex]{0pt}{5.5ex} \textbf{Ομόρροπα διανύσματα} & Αν $ \vec{a}\upuparrows\vec{\beta}\Leftrightarrow \vec{a}\cdot\vec{\beta}=|\vec{a}|\cdot|\vec{\beta}| $ \\ 
 \rule[-2ex]{0pt}{5.5ex} \textbf{Αντίρροπα διανύσματα} & Αν $ \vec{a}\updownarrows\vec{\beta}\Leftrightarrow \vec{a}\cdot\vec{\beta}=-|\vec{a}|\cdot|\vec{\beta}| $ \\ 
 \rule[-2ex]{0pt}{5.5ex} \textbf{Τετράγωνο διανύσματος} & $ \vec{a}^2=|\vec{a}|^2 $ \\ 
\rule[-2ex]{0pt}{5.5ex} \textbf{Αντιμεταθετική} & $ \vec{a}\cdot\vec{\beta}=\vec{\beta}\cdot\vec{a} $ \\
 \rule[-2ex]{0pt}{5.5ex} \textbf{Προσεταιριστική} & $ \mu(\vec{a}\cdot\vec{\beta})=(\mu\vec{\beta})\cdot\vec{a} $ \\
\rule[-2ex]{0pt}{5.5ex} \textbf{Επιμεριστική} & $ \vec{a}\cdot\left( \vec{\beta}+\vec{\gamma}\right) =\vec{a}\cdot\vec{\beta}+\vec{a}\cdot\vec{\gamma} $ \\
\hline 
\end{longtable} 
\end{center} 
\end{mybox}
\end{document}
