\PassOptionsToPackage{no-math,cm-default}{fontspec}
\documentclass[twoside,nofonts,internet,shmeiwseis]{thewria}
\usepackage{amsmath}
\usepackage{xgreek}
\let\hbar\relax
\defaultfontfeatures{Mapping=tex-text,Scale=MatchLowercase}
\setmainfont[Mapping=tex-text,Numbers=Lining,Scale=1.0,BoldFont={Minion Pro Bold}]{Minion Pro}
\newfontfamily\scfont{GFS Artemisia}
\font\icon = "Webdings"
\usepackage[amsbb]{mtpro2}
\usepackage{tikz,pgfplots}
\tkzSetUpPoint[size=7,fill=white]
\xroma{red!70!black}
%------- ΣΥΣΤΗΜΑ -------------------
\usepackage{systeme,regexpatch}
\makeatletter
% change the definition of \sysdelim not to store `\left` and `\right`
\def\sysdelim#1#2{\def\SYS@delim@left{#1}\def\SYS@delim@right{#2}}
\sysdelim\{. % reinitialize

% patch the internal command to use
% \LEFTRIGHT<left delim><right delim>{<system>}
% instead of \left<left delim<system>\right<right delim>
\regexpatchcmd\SYS@systeme@iii
{\cB.\c{SYS@delim@left}(.*)\c{SYS@delim@right}\cE.}
{\c{SYS@MT@LEFTRIGHT}\cB\{\1\cE\}}
{}{}
\def\SYS@MT@LEFTRIGHT{%
\expandafter\expandafter\expandafter\LEFTRIGHT
\expandafter\SYS@delim@left\SYS@delim@right}
\makeatother
\newcommand{\synt}[2]{{\scriptsize \begin{matrix}
\times#1\\\\ \times#2
\end{matrix}}}
%----------------------------------------
%------ ΜΗΚΟΣ ΓΡΑΜΜΗΣ ΚΛΑΣΜΑΤΟΣ ---------
\DeclareRobustCommand{\frac}[3][0pt]{%
{\begingroup\hspace{#1}#2\hspace{#1}\endgroup\over\hspace{#1}#3\hspace{#1}}}
%----------------------------------------

\newlist{rlist}{enumerate}{3}
\setlist[rlist]{itemsep=0mm,label=\roman*.}
\newlist{brlist}{enumerate}{3}
\setlist[brlist]{itemsep=0mm,label=\bf\roman*.}
\newlist{tropos}{enumerate}{3}
\setlist[tropos]{label=\bf\textit{\arabic*\textsuperscript{oς}\;Τρόπος :},leftmargin=0cm,itemindent=2.3cm,ref=\bf{\arabic*\textsuperscript{oς}\;Τρόπος}}
\newcommand{\tss}[1]{\textsuperscript{#1}}
\newcommand{\tssL}[1]{\MakeLowercase{\textsuperscript{#1}}}

\usepackage{hhline}
%----------- ΓΡΑΦΙΚΕΣ ΠΑΡΑΣΤΑΣΕΙΣ ---------
\pgfkeys{/pgfplots/aks_on/.style={axis lines=center,
xlabel style={at={(current axis.right of origin)},xshift=1.5ex, anchor=center},
ylabel style={at={(current axis.above origin)},yshift=1.5ex, anchor=center}}}
\pgfkeys{/pgfplots/grafikh parastash/.style={line width=.4mm,samples=200}}
\pgfkeys{/pgfplots/belh ar/.style={tick label style={font=\scriptsize},axis line style={-latex}}}
%-----------------------------------------
\usepackage{multicol}
\usepackage{wrap-rl}
\tkzSetUpPoint[size=7,fill=white]
\tikzstyle{pl}=[line width=0.3mm]
\tikzstyle{plm}=[line width=0.4mm]
\newcolumntype{C}{>{\centering\arraybackslash} X}
\setlist[itemize]{itemsep=0mm}



\begin{document}
\titlos{ΜΑΘΗΜΑΤΙΚΑ ΚΑΤΕΥΘΥΝΣΗΣ Β΄ ΛΥΚΕΙΟΥ}{Κωνικές Τομές}{Παραβολή}
\orismoi
\Orismos{Παραβολή}
Παραβολή ονομάζεται ο γεωμετρικός τόπος των σημείων του επιπέδου τα οποία έχουν ίσες αποστάσεις από ένα σταθερό σημείο και μια ευθεία.
\[ ME=MP \]
\begin{itemize}
\item Το σταθερό σημείο $ E $ ονομάζεται \textbf{εστία} της παραβολής.
\item Η ευθεία $ \delta $ ονομάζεται \textbf{διευθετούσα}.
\item Το σημείο το οποίο βρίσκεται στο μέσο της απόστασης της εστίας από τη διευθετούσα ονομάζεται \textbf{κορυφή} της παραβολής.
\end{itemize}
\begin{center}
\begin{tabular}{p{4.5cm}cp{4.5cm}}
\begin{tikzpicture}
\begin{axis}[
xmin=-2.2,xmax=2.5,ymin=-1.,ymax=3.5,x=1cm,y=1cm,ticks=none,xlabel={$ x $},
ylabel={$ y $},aks_on,belh ar,
%scale only axis,unit vector ratio={2 1},
]
\addplot [grafikh parastash,domain=-1.7:1.7,\xrwma] {x^2};
\addplot [domain=-2:2] {-0.25};
\coordinate (M) at (axis cs:1.2, 1.44);
\coordinate (E) at (axis cs:0, .25);
\coordinate (P) at (axis cs:1.2, -.25);
\coordinate (O) at (axis cs:0, 0);
\draw[black!50,plm] (E) -- (M) -- (P);
\tkzLabelPoint[above left, xshift=-.7ex,fill=white,inner sep=.2mm](E){$E\left(0, \frac{p}{2}\right)$}
\tkzLabelPoint[right](M){$M(x,y)$}
\tkzLabelPoint[below](P){$P$}
\tkzLabelPoint[below left=1mm,fill=white,inner sep=.2mm](O){$O$}
\end{axis}
\node at (0,.75){\footnotesize$\delta$};
\tkzDrawPoints(E,M,O,P)
\node at (1,4.5){$x^2=2py$};
\end{tikzpicture} & \hspace{1cm} & \begin{tikzpicture}
\begin{axis}[
xmin=-1,xmax=3.5,ymin=-2.,ymax=2.5,x=1cm,y=1cm,ticks=none,xlabel={$ x $},
ylabel={$ y $},aks_on,belh ar,
%scale only axis,unit vector ratio={2 1},
]
\addplot [grafikh parastash,domain=0:2.9,\xrwma] {sqrt(x)};
\addplot [grafikh parastash,domain=0:2.9,\xrwma] {-sqrt(x)};
\coordinate (M) at (axis cs:2, 1.4142);
\coordinate (E) at (axis cs:.25,0);
\coordinate (P) at (axis cs:-.25, 1.4142);
\coordinate (O) at (axis cs:0, 0);
\draw[black!50,plm] (E) -- (M) -- (P);
\tkzLabelPoint[below right, yshift=-1mm,xshift=1.5mm,fill=white,inner sep=.1mm](E)
{$E\left(\frac{p}{2},0\right)$}
\tkzLabelPoint[above left=.1mm](M){$M(x,y)$}
\tkzLabelPoint[left](P){$P$}
\end{axis}
\node at (0.5,.4){\footnotesize$\delta$};
\draw (0.75,4.2) -- (0.75,0.3);
\tkzDrawPoints(E,M,O,P)
\tkzLabelPoint[below left=1mm,fill=white,inner sep=.2mm](O){$O$}
\node at (2.2,4.5){$y^2=2px$};
\end{tikzpicture} \\ 
\end{tabular}
\end{center}
\begin{itemize}
\item Η απόσταση της εστίας από τη διευθετούσα συμβολίζεται με $ |p| $, όπου $ p $ είναι η \textbf{παράμετρος} της παραβολής, με $ p\in\mathbb{R} $.
\item Κάθε παραβολή με κορυφή την αρχή των αξόνων περιγράφεται από εξισώσεις της μορφής \[ x^2=2py\ \ \textrm{και}\ \  y^2=2px \]
\item Η εστία της παραβολής $ x^2=2py $ βρίσκεται στον κατακόρυφο άξονα $ y'y $ ενώ της $ y^2=2px $ στον ορίζόντιο άξονα $ x'x $.
\item Η παραβολή με εξίσωση $ x^2=2py $ έχει άξονα συμμετρίας τον $ y'y $ και εφάπτεται στον οριζόντιο άξονα $ x'x $ στο σημείο $ O $. Αντίστοιχα η παραβολή με εξίσωση $ y^2=2px $ έχει άξονα συμμετρίας τον $ x'x $ και εφάπτεται στον οριζόντιο άξονα $ y'y $ στο ίδιο σημείο.
\item Η ευθεία που είναι κάθετη στη διευθετούσα και διέρχεται από την εστία της παραβολής ονομάζεται \textbf{άξονας} της παραβολής.
\end{itemize}
\Orismos{Εφαπτομένη παραβολήσ}
Εφαπτομένη μιας παραβολής ονομάζεται η ευθεία γραμμή η οποία έχει ένα κοινό σημείο με την παραβολή. Λέμε οτι εφάπτεται αυτής. Το σημείο αυτό ονομάζεται \textbf{σημείο επαφής}.\\
\wrapr{-4mm}{10}{4.5cm}{-11mm}{
\begin{tikzpicture}
\begin{axis}[
xmin=-2,xmax=2.2,ymin=-1,ymax=3,x=1cm,y=1cm,ticks=none,xlabel={$ x $},
ylabel={$ y $},aks_on,belh ar,
%scale only axis,unit vector ratio={2 1},
]
\addplot [grafikh parastash,domain=-1.5:1.5] {x^2};
\addplot [domain=-1.7:1.7,\xrwma,pl] {2.4*x-1.44};
\coordinate (M) at (axis cs:1.2, 1.44);
\coordinate (E) at (axis cs:0, .25);
\coordinate (O) at (axis cs:0, 0);
\tkzLabelPoint[above left, xshift=-.7ex,fill=white,inner sep=.2mm](E){$E$}
\tkzLabelPoint[left,fill=white,inner sep=.1mm,xshift=-1mm](M){$A(x_1,y_1)$}
\tkzLabelPoint[below left=1mm,fill=white,inner sep=.2mm](O){$O$}
\node at (axis cs:1.8,2.3){$\varepsilon$};
\end{axis}
\tkzDrawPoints(E,M)
\end{tikzpicture}}{
Έστω $ A(x_1,y_1) $ το σημείο επαφής της εφαπτομένης με την παραβολή. Τότε η εξίσωση της εφαπτομένης για κάθε μορφή παραβολής από της παραπάνω θα είναι :
\begin{itemize}
\item Για την παραβολή με εστίες στον άξονα $ x'x $ :  $ (\varepsilon) : xx_1=p(y+y_1) $
\item Για την παραβολή με εστίες στον άξονα $ y'y $ :  $ (\varepsilon) : yy_1=p(x+x_1) $
\end{itemize}}\mbox{}\\\\\\

\thewrhmata
\Thewrhma{Ιδιότητες παραβολής}
Για τα σημεία μιας παραβολής και τη γραφική της παράσταση ισχύουν οι παρακάτω ιδιότητες.
\begin{enumerate}
\item Η παραβολή βρίσκεται στο ημιεπίπεδο που ορίζει η διευθετούσα και η εστία της παραβολής.
\item Παραβολή $ x^2=2py $.
\begin{rlist}
\item Για την παραβολή $ x^2=2py $ οι αριθμοί $ p $ και $ y $ είναι ομόσημοι.
\item Αν $ M(x,y) $ είναι ένα σημείο της παραβολής τότε και το σημείο $ M_2(-x,y) $ θα είναι επίσης σημείο της παραβολής.
\end{rlist}
\item  Παραβολή $ y^2=2px $.
\begin{rlist}
\item Για την παραβολή $ y^2=2px $ οι αριθμοί $ p $ και $ x $ είναι ομόσημοι.
\item Αν $ M(x,y) $ είναι ένα σημείο της παραβολής τότε και το σημείο $ M_1(x,-y) $ θα είναι επίσης σημείο της παραβολής.
\end{rlist}
\end{enumerate}
\Thewrhma{Ανακλαστική Ιδιότητα}
Η ευθεία που διέρχεται από ένα τυχαίο σημείο μιας παραβολής και είναι κάθετη στην εφαπτόμενη ευθεία στο σημείο αυτό, διχοτομεί τη γωνία που σχηματίζουν
η ημιευθεία $ Mx $ που είναι παράλληλη με τον άξονα της παραβολής και το ευθύγραμμο τμήμα $ ME $ που ενώνει το σημείο με την εστία της παραβολής.
\begin{center}
\begin{tikzpicture}
\begin{axis}[
xmin=-1.2,xmax=3.5,ymin=-2.,ymax=2.5,x=1cm,y=1cm,ticks=none,xlabel={$ x $},
ylabel={$ y $},aks_on,belh ar,
%scale only axis,unit vector ratio={2 1},
]
\addplot [grafikh parastash,domain=0:2.9] {sqrt(x)};
\addplot [grafikh parastash,domain=0:2.9] {-sqrt(x)};
\addplot[domain=-1.2:3] {0.5*x+0.5};
\addplot[domain=.3:2.4,grafikh parastash,\xrwma] {-2*x+3};
\coordinate (M) at (axis cs:1,1);
\coordinate (E) at (axis cs:.25,0);
\coordinate (P) at (axis cs:3,1);
\coordinate (O) at (axis cs:0, 0);
\draw[pl] (E) -- (M) -- (P);
\tkzLabelPoint[below right, yshift=-1mm,xshift=1.5mm,fill=white,inner sep=.1mm](E)
{$E\left(\frac{p}{2},0\right)$}
\tkzLabelPoint[left,xshift=-2mm,yshift=1mm](M){$M(x,y)$}
\end{axis}
\tkzDrawPoints(E,M,O)
\tkzLabelPoint[below left=1mm,fill=white,inner sep=.2mm](O){$O$}
\node at (3.8,4){$\varepsilon$};
\node at (4.4,3){$x$};
\node at (2,4){$\zeta$};
\draw (M)+(233:.33cm) arc (233:297:.33cm);
\draw (M)+(297:.39cm) arc (297:360:.39cm);
\node at (2.2,2.4){$\varphi_1$};
\node at (2.9,2.7){$\varphi_2$};
\end{tikzpicture} 
\end{center}
Η ιδιότητα αυτή της έλλειψης ονομάζεται \textbf{ανακλαστική} και δείχνει ότι κάθε ευθεία γραμμή που διέρχεται από την εστία της παραβολής, ανακλάται πάνω στην παραβολή με τέτοιο τρόπο ώστε η γωνία πρόσπτωσης να είναι ίση με τη γωνία ανάκλασης της με αποτέλεσμα να γίνει παράλληλη με τον άξονα συμμετρίας. 
\end{document}
