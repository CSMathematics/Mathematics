\PassOptionsToPackage{no-math,cm-default}{fontspec}
\documentclass[twoside,nofonts,internet,shmeiwseis]{thewria}
\usepackage{amsmath}
\usepackage{xgreek}
\let\hbar\relax
\defaultfontfeatures{Mapping=tex-text,Scale=MatchLowercase}
\setmainfont[Mapping=tex-text,Numbers=Lining,Scale=1.0,BoldFont={Minion Pro Bold}]{Minion Pro}
\newfontfamily\scfont{GFS Artemisia}
\font\icon = "Webdings"
\usepackage[amsbb]{mtpro2}
\usepackage{tikz,pgfplots,tkz-euclide}
\tkzSetUpPoint[size=7,fill=white]
\xroma{red!70!black}
%------- ΣΥΣΤΗΜΑ -------------------
\usepackage{systeme,regexpatch}
\makeatletter
% change the definition of \sysdelim not to store `\left` and `\right`
\def\sysdelim#1#2{\def\SYS@delim@left{#1}\def\SYS@delim@right{#2}}
\sysdelim\{. % reinitialize

% patch the internal command to use
% \LEFTRIGHT<left delim><right delim>{<system>}
% instead of \left<left delim<system>\right<right delim>
\regexpatchcmd\SYS@systeme@iii
{\cB.\c{SYS@delim@left}(.*)\c{SYS@delim@right}\cE.}
{\c{SYS@MT@LEFTRIGHT}\cB\{\1\cE\}}
{}{}
\def\SYS@MT@LEFTRIGHT{%
\expandafter\expandafter\expandafter\LEFTRIGHT
\expandafter\SYS@delim@left\SYS@delim@right}
\makeatother
\newcommand{\synt}[2]{{\scriptsize \begin{matrix}
\times#1\\\\ \times#2
\end{matrix}}}
%----------------------------------------
%------ ΜΗΚΟΣ ΓΡΑΜΜΗΣ ΚΛΑΣΜΑΤΟΣ ---------
\DeclareRobustCommand{\frac}[3][0pt]{%
{\begingroup\hspace{#1}#2\hspace{#1}\endgroup\over\hspace{#1}#3\hspace{#1}}}
%----------------------------------------

\newlist{rlist}{enumerate}{3}
\setlist[rlist]{itemsep=0mm,label=\roman*.}
\newlist{brlist}{enumerate}{3}
\setlist[brlist]{itemsep=0mm,label=\bf\roman*.}
\newlist{tropos}{enumerate}{3}
\setlist[tropos]{label=\bf\textit{\arabic*\textsuperscript{oς}\;Τρόπος :},leftmargin=0cm,itemindent=2.3cm,ref=\bf{\arabic*\textsuperscript{oς}\;Τρόπος}}
\newcommand{\tss}[1]{\textsuperscript{#1}}
\newcommand{\tssL}[1]{\MakeLowercase{\textsuperscript{#1}}}

\usepackage{hhline}
%----------- ΓΡΑΦΙΚΕΣ ΠΑΡΑΣΤΑΣΕΙΣ ---------
\pgfkeys{/pgfplots/aks_on/.style={axis lines=center,
xlabel style={at={(current axis.right of origin)},xshift=1.5ex, anchor=center},
ylabel style={at={(current axis.above origin)},yshift=1.5ex, anchor=center}}}
\pgfkeys{/pgfplots/grafikh parastash/.style={\xrwma,line width=.4mm,samples=200}}
\pgfkeys{/pgfplots/belh ar/.style={tick label style={font=\scriptsize},axis line style={-latex}}}
%-----------------------------------------
\usepackage{multicol}
\usepackage{wrap-rl}
\usetkzobj{all}
\tikzstyle{pl}=[line width=0.3mm]
\tikzstyle{plm}=[line width=0.4mm]


\begin{document}
\titlos{Μαθηματικά Κατεύθυνσης Β΄ Λυκείου}{Κωνικές Τομές}{Υπερβολή}
\orismoi
\Orismos{Υπερβολή} Υπερβολή ονομάζεται ο γεωμετρικός τόπος των σημείων του επιπέδου των οποίων η απόλυτη τιμή της διαφοράς των αποστάσεων τους από δύο σταθερά σημεία παραμένει σταθερή.
\[ |ME-ME'|=2a \]
Η καμπύλη της υπερβολής αποτελείται από δύο κλάδους, χαρακτηριστικό το οποίο εξηγεί την ύπαρξη της απόλυτης τιμής στην παραπάνω σχέση.
\begin{itemize}
\item Τα σταθερά σημεία που ορίζουν την υπερβολή ονομάζονται \textbf{εστίες} της υπερβολής.
\item Η σταθερή διαφορά των αποστάσεων του τυχαίου σημείου $ M $ από τις εστίες συμβολίζεται με $ 2a $.
\item Η απόσταση $ EE' $ μεταξύ των εστιών ονομάζεται \textbf{εστιακή απόσταση} και συμβολίζεται με $ 2\gamma $.
\end{itemize}
\begin{center}
\begin{tabular}{p{4.5cm}cp{4.5cm}}
\begin{tikzpicture}
\begin{axis}[
xmin=-2.2,xmax=2.5,ymin=-2.4,ymax=2.5,x=1cm,y=1cm,ticks=none,xlabel={$ x $},
ylabel={$ y $},aks_on,belh ar,
%scale only axis,unit vector ratio={2 1},
]
\pgfmathsetmacro{\a}{.7}
\pgfmathsetmacro{\b}{.7}
\pgfmathsetmacro{\c}{sqrt(\a^2 + \b^2)}
\addplot [grafikh parastash,domain=-1.7:1.7] ({.7*cosh(x)}, {.7*sinh(x)});
\addplot [grafikh parastash,domain=-1.7:1.7] ({-.7*cosh(x)}, {.7*sinh(x)});
%\addplot [grafikh parastash,domain=-1.7:1.7] ({.7*sinh(x)}, {.7*cosh(x)});
%\addplot [grafikh parastash,domain=-1.7:1.7] ({-.7*sinh(x)}, {-.7*cosh(x)});
\coordinate (M) at (axis cs:1.08,0.82);
\coordinate (E) at (axis cs:\c,0);
\coordinate (E') at (axis cs:-\c,0);
\coordinate (O) at (axis cs:0, 0);
\coordinate (A) at (axis cs:.7, 0);
\coordinate (A') at (axis cs:-.7, 0);
\draw[plm,black] (E) -- (M) -- (E');
\tkzLabelPoint[below right](E){\footnotesize$E\left(\gamma,0\right)$}
\tkzLabelPoint[right](M){$M(x,y)$}
\tkzLabelPoint[below left,xshift=2mm](E'){\footnotesize$E'\left(-\gamma,0\right)$}
\tkzLabelPoint[below left=1mm,fill=white,inner sep=.2mm](O){$O$}
\tkzLabelPoint[above left](A){$A$}
\tkzLabelPoint[above right=1mm,fill=white,inner sep=.4mm](A'){$A'$}
\end{axis}
\tkzDrawPoints(E,M,E',A,A')
\node[fill=white,inner sep=.2mm] at (2.2,4){$\frac{x^2}{a^2}-\frac{y^2}{\beta^2}=1$};
\end{tikzpicture} & \hspace{1cm} & \begin{tikzpicture}
\begin{axis}[
xmin=-2.2,xmax=2.5,ymin=-2.4,ymax=2.5,x=1cm,y=1cm,ticks=none,xlabel={$ x $},
ylabel={$ y $},aks_on,belh ar,
%scale only axis,unit vector ratio={2 1},
]
\pgfmathsetmacro{\a}{.7}
\pgfmathsetmacro{\b}{.7}
\pgfmathsetmacro{\c}{sqrt(\a^2 + \b^2)}
%\addplot [grafikh parastash,domain=-1.7:1.7] ({.7*cosh(x)}, {.7*sinh(x)});
%\addplot [grafikh parastash,domain=-1.7:1.7] ({-.7*cosh(x)}, {.7*sinh(x)});
\addplot [grafikh parastash,domain=-1.7:1.7] ({.7*sinh(x)}, {.7*cosh(x)});
\addplot [grafikh parastash,domain=-1.7:1.7] ({-.7*sinh(x)}, {-.7*cosh(x)});
\coordinate (M) at (axis cs:0.82,1.08);
\coordinate (E) at (axis cs:0,\c);
\coordinate (E') at (axis cs:0,-\c);
\coordinate (O) at (axis cs:0, 0);
\coordinate (A) at (axis cs:0, 0.7);
\coordinate (A') at (axis cs:0, -0.7);
\draw[plm] (E) -- (M) -- (E');
\tkzLabelPoint[above,fill=white,yshift=2mm,inner sep=.1mm](E){\footnotesize$E\left(\gamma,0\right)$}
\tkzLabelPoint[below right,xshift=-.5mm,yshift=.2mm](M){$M(x,y)$}
\tkzLabelPoint[below,fill=white,yshift=-2mm,inner sep=.1mm](E'){\footnotesize$E'\left(-\gamma,0\right)$}
\tkzLabelPoint[below left=1mm,fill=white,inner sep=.2mm](O){$O$}
\tkzLabelPoint[left=1mm,fill=white,inner sep=.2mm](A){$A$}
\tkzLabelPoint[left=1mm,fill=white,inner sep=.2mm](A'){$A'$}
\end{axis}
\tkzDrawPoints(E,M,E',A,A')
\node[fill=white,inner sep=.2mm] at (0.5,2.8){$\frac{y^2}{a^2}-\frac{x^2}{\beta^2}=1$};
\end{tikzpicture} \\ 
\end{tabular}
\end{center}
\begin{itemize}
\item Τα σημεία στα οποία η υπερβολή τέμνει τους άξονες ονομάζονται \textbf{κορυφές} της.
\item Η καμπύλη της υπερβολής περιγράφεται αλγεβρικά από μια εξίσωση της μορφής \[\frac{x^2}{a^2}-\frac{y^2}{\beta^2}=1 \ \ \textrm{ ή }\ \  \frac{y^2}{a^2}-\frac{x^2}{\beta^2}=1 \] όπου $ \beta=\sqrt{\gamma^2-a^2} $ και $ x,y $ οι συντεταγμένες των σημείων της.
\item Η υπερβολή με εξίσωση $\frac{x^2}{a^2}-\frac{y^2}{\beta^2}=1$ έχει τις εστίες της στον οριζόντιο άξονα $ x'x $ ενώ η υπερβολή $\frac{y^2}{a^2}-\frac{x^2}{\beta^2}=1$ έχει της εστίες της στον κατακόρυφο άξονα $ y'y $.
\newpage
\noindent
\item Οι δύο άξονες είναι άξονες συμμετρίας της υπερβολής και ονομάζονται \textbf{άξονες} της υπερβολής, ενώ το σημείο τομής τους δηλαδή η αρχή $ O $ των αξόνων, \textbf{κέντρο} της υπερβολής το οποίο είναι κέντρο συμμετρίας της.
\end{itemize}
\Orismos{Ασύμπτωτεσ υπερβολήσ}
Ασύμπτωτες μιας υπερβολής ονομάζονται οι ευθείες γραμμές οι οποίες βρίσκονται απειροελάχιστα κοντά στην καμπύλη της υπερβολής, χωρίς να τέμνονται ή να εφάπτονται μ' αυτή.
\begin{center}
\begin{tabular}{p{4.5cm}cp{4.5cm}}
\begin{tikzpicture}
\begin{axis}[
xmin=-2.2,xmax=2.5,ymin=-2.4,ymax=2.5,x=1cm,y=1cm,ticks=none,xlabel={$ x $},
ylabel={$ y $},aks_on,belh ar,
%scale only axis,unit vector ratio={2 1},
]
\pgfmathsetmacro{\a}{.7}
\pgfmathsetmacro{\b}{.7}
\pgfmathsetmacro{\c}{sqrt(\a^2 + \b^2)}
\addplot [grafikh parastash,domain=-1.7:1.7] ({.7*cosh(x)}, {.7*sinh(x)});
\addplot [grafikh parastash,domain=-1.7:1.7] ({-.7*cosh(x)}, {.7*sinh(x)});
\addplot [domain=-2:2]{-\x};
\addplot [domain=-2:2]{\x};
\coordinate (E) at (axis cs:\c,0);
\coordinate (E') at (axis cs:-\c,0);
\coordinate (O) at (axis cs:0, 0);
\coordinate (B) at (axis cs:\a, \b);
\coordinate (A) at (axis cs:-\a, \b);
\coordinate (D) at (axis cs:-\a, -\b);
\coordinate (C) at (axis cs:\a, -\b);
\draw (A)--(B)--(C)--(D)--cycle;
\tkzLabelPoint[below right](E){\footnotesize$E$}
\tkzLabelPoint[below left](E'){\footnotesize$E'$}
\tkzLabelPoint[below left=1mm,fill=white,inner sep=.2mm](O){$O$}
\tkzLabelPoint[above](A){$A$}
\tkzLabelPoint[above](B){$B$}
\tkzLabelPoint[below](C){$\varGamma$}
\tkzLabelPoint[below](D){$\varDelta$}
\node at (axis cs:-1,2.1){$ y=-\frac{\beta}{a}x $};
\node at (axis cs:1,2.1){$ y=\frac{\beta}{a}x $};
\end{axis}
\tkzDrawPoints(E,E')
\end{tikzpicture} & \hspace{.5cm} & \begin{tikzpicture}
\begin{axis}[
xmin=-2.2,xmax=2.5,ymin=-2.4,ymax=2.5,x=1cm,y=1cm,ticks=none,xlabel={$ x $},
ylabel={$ y $},aks_on,belh ar,
%scale only axis,unit vector ratio={2 1},
]
\pgfmathsetmacro{\a}{.7}
\pgfmathsetmacro{\b}{.7}
\pgfmathsetmacro{\c}{sqrt(\a^2 + \b^2)}
\addplot [domain=-2:2]{-\x};
\addplot [domain=-2:2]{\x};
\addplot [grafikh parastash,domain=-1.7:1.7] ({.7*sinh(x)}, {.7*cosh(x)});
\addplot [grafikh parastash,domain=-1.7:1.7] ({-.7*sinh(x)}, {-.7*cosh(x)});
\coordinate (E) at (axis cs:0,\c);
\coordinate (E') at (axis cs:0,-\c);
\coordinate (O) at (axis cs:0, 0);
\coordinate (B) at (axis cs:\a, \b);
\coordinate (A) at (axis cs:-\a, \b);
\coordinate (D) at (axis cs:-\a, -\b);
\coordinate (C) at (axis cs:\a, -\b);
\draw (A)--(B)--(C)--(D)--cycle;
\tkzLabelPoint[above,fill=white,yshift=2mm,inner sep=.1mm](E){\footnotesize$E$}
\tkzLabelPoint[below,fill=white,yshift=-2mm,inner sep=.1mm](E'){\footnotesize$E'$}
\tkzLabelPoint[below left=1mm,fill=white,inner sep=.2mm](O){$O$}
\tkzLabelPoint[left](A){$A$}
\tkzLabelPoint[right](B){$B$}
\tkzLabelPoint[left](D){$\varDelta$}
\tkzLabelPoint[right](C){$\varGamma$}
\end{axis}
\tkzDrawPoints(E,E')
\node at (4,3.3){$ y=\frac{a}{\beta}x $};
\node at (.3,3.3){$ y=-\frac{a}{\beta}x $};
\end{tikzpicture} \\ 
\end{tabular}
\end{center}
Καθώς οι κλάδοι της υπερβολής επεκτείνονται απεριόριστα, οι ασύμπτωτες πλησιάζουν όλο και περισσότερο την καμπύλη με αποτέλεσμα η απόστασή τους απ' αυτήν να τείνει στο μηδέν.
\begin{itemize}
\item Οι ασύμπτωτες ευθείες κάθε υπερβολής είναι δύο.
\item Οι εξισώσεις των ασύμπτωτων ευθειών της υπερβολής με εξίσωση της μορφής $ \frac{x^2}{a^2}-\frac{y^2}{\beta^2}=1 $ είναι $ y=\frac{\beta}{a}x $ και $ y=-\frac{\beta}{a}x $, ενώ οι ασύμπτωτες της υπερβολής $ \frac{y^2}{a^2}-\frac{x^2}{\beta^2}=1 $ έχουν εξισώσεις $ y=\frac{a}{\beta}x $ και $ y=-\frac{a}{\beta}x $.
\item Τα σημεία $ A(-a,\beta), B(a,\beta), \varGamma(a,-\beta) $ και $ \varDelta(-a,-\beta) $ είναι σημεία των ασύμπτωτων ευθειών και ορίζουν ένα ορθογώνιο παραλληλόγραμμο το οποίο ονομάζεται \textbf{ορθογώνιο βάσης} της υπερβολής.
\item Δύο από τις απέναντι πλευρές του ορθογωνίου βάσης εφάπτονται της υπερβολής στις κορυφές της.
\end{itemize}
\Orismos{Εφαπτομένη Υπερβολήσ}
Εφαπτομένη μιας υπερβολής ονομάζεται η ευθεία γραμμή η οποία εφάπτεται στην υπερβολή σε ένα σημείο της.\\
\wrapr{-4mm}{10}{4.7cm}{-7mm}{
\begin{tikzpicture}
\begin{axis}[
xmin=-2,xmax=2.2,ymin=-1.8,ymax=2.2,x=.9cm,y=.9cm,ticks=none,xlabel={$ x $},
ylabel={$ y $},aks_on,belh ar,
%scale only axis,unit vector ratio={2 1},
]
\pgfmathsetmacro{\a}{.7}
\pgfmathsetmacro{\b}{.7}
\pgfmathsetmacro{\c}{sqrt(\a^2 + \b^2)}
\addplot [grafikh parastash,black,domain=-1.5:1.5] ({.7*cosh(x)}, {.7*sinh(x)});
\addplot [grafikh parastash,black,domain=-1.5:1.5] ({-.7*cosh(x)}, {.7*sinh(x)});
\addplot [domain=-.5:1.8,\xrwma,pl] {1.317*x-0.6};
\coordinate (E) at (axis cs:\c,0);
\coordinate (E') at (axis cs:-\c,0);
\coordinate (O) at (axis cs:0, 0);
\coordinate (A) at (axis cs:1.08,.82);
\tkzLabelPoint[below right](E){\footnotesize$E$}
\tkzLabelPoint[below left](E'){\footnotesize$E'$}
\tkzLabelPoint[below left=1mm,fill=white,inner sep=.2mm](O){$O$}
\tkzLabelPoint[left,fill=white,inner sep=.2mm,xshift=-1.8mm](A){$A(x_1,y_1)$}
\node at (axis cs: 1.7,2){$\varepsilon$};
\end{axis}
\tkzDrawPoints(E,E',A)
\end{tikzpicture}}{
Έστω $ A(x_1,y_1) $ το κοινό σημείο της υπερβολής με την εφαπτόμενη ευθεία. Τότε η εξίσωση της εφαπτομένης για κάθε μορφή υπερβολής από της παραπάνω θα είναι :
\begin{itemize}
\item Για την υπερβολή με εστίες στον άξονα $ x'x $ :  $ (\varepsilon) : \frac{xx_1}{a^2}-\frac{yy_1}{\beta^2}=1 $
\item Για την υπερβολή με εστίες στον άξονα $ y'y $ :  $ (\varepsilon) : \frac{yy_1}{a^2}-\frac{xx_1}{\beta^2}=1 $
\end{itemize}}\mbox{}\\\\\\
\Orismos{Εκκεντρότητα Υπερβολήσ}
\wrapr{-4mm}{10}{5cm}{-8mm}{\begin{tikzpicture}
\begin{axis}[
xmin=-2.2,xmax=2.4,ymin=-1.8,ymax=2,x=1cm,y=1cm,ticks=none,xlabel={$ x $},
ylabel={$ y $},aks_on,belh ar]
\pgfmathsetmacro{\a}{.7}
\pgfmathsetmacro{\b}{.7}
\pgfmathsetmacro{\c}{sqrt(\a^2 + \b^2)}
\addplot [grafikh parastash,domain=-1.7:1.7,\xrwma!30] ({.7*cosh(x)}, {.7*sinh(x)});
\addplot [grafikh parastash,domain=-1.7:1.7,\xrwma!30] ({-.7*cosh(x)}, {.7*sinh(x)});
\addplot [grafikh parastash,domain=-1.7:1.7,\xrwma!50] ({.7*cosh(x)}, {.5*sinh(x)});
\addplot [grafikh parastash,domain=-1.7:1.7,\xrwma!50] ({-.7*cosh(x)}, {.5*sinh(x)});
\addplot [grafikh parastash,domain=-1.7:1.7,\xrwma!70] ({.7*cosh(x)}, {.3*sinh(x)});
\addplot [grafikh parastash,domain=-1.7:1.7,\xrwma!70] ({-.7*cosh(x)}, {.3*sinh(x)});
\coordinate (E) at (axis cs:\c,0);
\coordinate (E') at (axis cs:-\c,0);
\coordinate (O) at (axis cs:0, 0);
\tkzLabelPoint[below right](E){\footnotesize$E$}
\tkzLabelPoint[below left](E'){\footnotesize$E'$}
\tkzLabelPoint[below left=1mm,fill=white,inner sep=.2mm](O){$O$}
\end{axis}
\draw[fill=\xrwma!30] (.3,3.4) circle (.07) node[right,fill=white,inner sep=.2mm,xshift=1mm]{\scriptsize{$\varepsilon=\sqrt{2}$}};
\draw[fill=\xrwma!50] (1.8,3.4) circle (.07) node[right,fill=white,inner sep=.2mm,xshift=1mm]{\scriptsize{$\varepsilon=1.23$}};
\draw[fill=\xrwma!70] (3.3,3.4) circle (.07) node[right,fill=white,inner sep=.2mm,xshift=1mm]{\scriptsize{$\varepsilon=1.09$}};
\tkzDrawPoints(E,E')
\end{tikzpicture}}{
Εκκεντρότητα μιας υπερβολής ονομάζεται ο πραγματικός αριθμός $ \varepsilon\in\mathbb{R} $ που ορίζεται ως ο λόγος της εστιακής απόστασης προς την απόσταση των κορυφών της.
\[ \varepsilon=\dfrac{\gamma}{a} \]
Το μέγεθος της εκκεντρότητας μιας υπερβολής καθορίζει το σχήμα της. Καθώς μειώνεται η εκκεντρότητα, η υπερβολή γίνεται όλο και πιο επιμήκης κατα μήκος του άξονα στον οποίον βρίσκονται οι εστίες.}\mbox{}\\\\
\Orismos{Συζυγείσ υπερβολέσ}
Συζυγείς ονομάζονται δύο υπερβολές οι οποίες έχουν τις εστίες τους σε κάθετους μεταξύ τους άξονες και κοινές ασύμπτωτες ευθείες. Έχουν τη μορφή \[ \frac{x^2}{a^2}-\frac{y^2}{\beta^2}=1\;\;,\;\;\frac{y^2}{a^2}-\frac{x^2}{\beta^2}=1 \]
Οι συζυγείς υπερβολές έχουν το ίδιο ορθογώνιο βάσης.
\begin{center}
\begin{tabular}{p{5cm}cp{5cm}}
\begin{tikzpicture}
\begin{axis}[
xmin=-2,xmax=2.2,ymin=-2.,ymax=2.2,x=1cm,y=1cm,ticks=none,xlabel={$ x $},
ylabel={$ y $},aks_on,belh ar,
%scale only axis,unit vector ratio={2 1},
]
\pgfmathsetmacro{\a}{.7}
\pgfmathsetmacro{\b}{.7}
\pgfmathsetmacro{\c}{sqrt(\a^2 + \b^2)}
\addplot [grafikh parastash,domain=-1.7:1.7] ({.7*cosh(x)}, {.7*sinh(x)});
\addplot [grafikh parastash,domain=-1.7:1.7] ({-.7*cosh(x)}, {.7*sinh(x)});
\addplot [grafikh parastash,domain=-1.7:1.7] ({.7*sinh(x)}, {.7*cosh(x)});
\addplot [grafikh parastash,domain=-1.7:1.7] ({-.7*sinh(x)}, {-.7*cosh(x)});
\addplot [domain=-2:2]{-\x};
\addplot [domain=-2:2]{\x};
\coordinate (E) at (axis cs:\c,0);
\coordinate (E') at (axis cs:-\c,0);
\coordinate (F) at (axis cs:0,\c);
\coordinate (F') at (axis cs:0,-\c);
\coordinate (O) at (axis cs:0, 0);
\coordinate (B) at (axis cs:\a, \b);
\coordinate (A) at (axis cs:-\a, \b);
\coordinate (D) at (axis cs:-\a, -\b);
\coordinate (C) at (axis cs:\a, -\b);
\draw (A)--(B)--(C)--(D)--cycle;
\tkzLabelPoint[below right](E){\footnotesize$E_1$}
\tkzLabelPoint[below left](E'){\footnotesize$E_1'$}
\tkzLabelPoint[above,fill=white,inner sep=.2mm,yshift=1mm](F){\footnotesize$E_2$}
\tkzLabelPoint[below,fill=white,inner sep=.2mm,yshift=-1mm](F'){\footnotesize$E_2'$}
\tkzLabelPoint[below left=1mm,fill=white,inner sep=.2mm](O){$O$}
\tkzLabelPoint[above left,fill=white,inner sep=.1mm](A){\footnotesize$A$}
\tkzLabelPoint[above right,fill=white,inner sep=.1mm](B){\footnotesize$B$}
\tkzLabelPoint[below right,fill=white,inner sep=.1mm](C){\footnotesize$\varGamma$}
\tkzLabelPoint[below left,fill=white,inner sep=.1mm](D){\footnotesize$\varDelta$}
\end{axis}
\tkzDrawPoints(E,E',F,F')
\end{tikzpicture}	&  & \begin{tikzpicture}
\begin{axis}[
xmin=-2,xmax=2.2,ymin=-2.,ymax=2.2,x=1cm,y=1cm,ticks=none,xlabel={$ x $},
ylabel={$ y $},aks_on,belh ar,
]
\pgfmathsetmacro{\a}{.7}
\pgfmathsetmacro{\b}{.7}
\pgfmathsetmacro{\c}{sqrt(\a^2 + \b^2)}
\addplot [grafikh parastash,domain=-1.7:1.7] ({.7*cosh(x)}, {.7*sinh(x)});
\addplot [grafikh parastash,domain=-1.7:1.7] ({-.7*cosh(x)}, {.7*sinh(x)});
\addplot [domain=-2:2]{-\x};
\addplot [domain=-2:2]{\x};
\coordinate (E) at (axis cs:\c,0);
\coordinate (E') at (axis cs:-\c,0);
\coordinate (O) at (axis cs:0, 0);
\coordinate (B) at (axis cs:\a, \b);
\coordinate (C) at (axis cs:\a, -\b);
\tkzLabelPoint[below right](E){\footnotesize$E$}
\tkzLabelPoint[below left](E'){\footnotesize$E'$}
\tkzLabelPoint[below left=1mm,fill=white,inner sep=.2mm](O){$O$}
\node[fill=white,inner sep=.2mm] at (axis cs:0,1.7){$ x^2-y^2=a^2 $};
\end{axis}
\tkzMarkRightAngle(C,O,B)
\tkzDrawPoints(E,E')
\end{tikzpicture} \\ 
\end{tabular} 
\end{center}
\Orismos{Ισοσκελήσ Υπερβολή}
Ισοσκελής ονομάζεται η υπερβολή για την οποία οι παράμετροι $ a $ και $ \beta $ είναι ίσες. Με $ a=\beta $ η εξίσωση μιας ισοσκελούς υπερβολής θα έχει τη μορφή \[ x^2-y^2=a^2\;\;\textrm{ή}\;\; y^2-x^2=a^2 \]

\thewrhmata
\Thewrhma{Συμμετρικά σημεία υπερβολής}
Αν $ \frac{x^2}{a^2}-\frac{y^2}{\beta^2}=1\ ,\ \frac{y^2}{a^2}-\frac{x^2}{\beta^2}=1 $ είναι μια υπερβολή του επιπέδου και $ M(x,y) $ ένα σημείο αυτής, τότε τα σημεία $ M_1(-x,y),M_2(-x,-y) $ και $ M_3(x,-y) $ ανήκουν και αυτά στην υπερβολή.
\begin{center}
\begin{tabular}{p{5cm}cp{5cm}} 
\begin{tikzpicture}
\begin{axis}[
xmin=-2,xmax=2.2,ymin=-2,ymax=2.2,x=1cm,y=1cm,ticks=none,xlabel={$ x $},
ylabel={$ y $},aks_on,belh ar,
%scale only axis,unit vector ratio={2 1},
]
\pgfmathsetmacro{\a}{.7}
\pgfmathsetmacro{\b}{.7}
\pgfmathsetmacro{\c}{sqrt(\a^2 + \b^2)}
\addplot [grafikh parastash,black,domain=-1.5:1.5] ({.7*cosh(x)}, {.7*sinh(x)});
\addplot [grafikh parastash,black,domain=-1.5:1.5] ({-.7*cosh(x)}, {.7*sinh(x)});
\addplot [domain=-.5:1.8,\xrwma,pl] {1.317*x-0.6};
\coordinate (E) at (axis cs:\c,0);
\coordinate (E') at (axis cs:-\c,0);
\coordinate (O) at (axis cs:0, 0);
\coordinate (A) at (axis cs:1.08,.82);
\tkzLabelPoint[below right](E){\footnotesize$E$}
\tkzLabelPoint[below left](E'){\footnotesize$E'$}
\tkzLabelPoint[below left=1mm,fill=white,inner sep=.2mm](O){$O$}
\tkzLabelPoint[left,fill=white,inner sep=.2mm,xshift=-1.7mm,yshift=2mm](A){$A(x_1,y_1)$}
\node at (axis cs: .5,.33){{\footnotesize $\varphi_1$}};
\node[fill=white,inner sep=.2mm] at (axis cs: .8,.07){{\footnotesize $\varphi_2$}};
\node at (axis cs: 1.7,2){$\varepsilon$};
\end{axis}
\draw[pl] (E) -- (A)--(E');
\draw (A)+(201.6:.5) arc (201.6:232:.5);
\draw (A)+(232:.6) arc (232:263.67:.6);
\tkzDrawPoints(E,E',A)
\end{tikzpicture}
& & \begin{tikzpicture}
\begin{axis}[
xmin=-2,xmax=2.2,ymin=-2.,ymax=2.2,x=1cm,y=1cm,ticks=none,xlabel={$ x $},
ylabel={$ y $},aks_on,belh ar,
%scale only axis,unit vector ratio={2 1},
]
\pgfmathsetmacro{\a}{.7}
\pgfmathsetmacro{\b}{.7}
\pgfmathsetmacro{\c}{sqrt(\a^2 + \b^2)}
\addplot [grafikh parastash,domain=-1.7:1.7] ({.7*cosh(x)}, {.7*sinh(x)});
\addplot [grafikh parastash,domain=-1.7:1.7] ({-.7*cosh(x)}, {.7*sinh(x)});
\coordinate (E) at (axis cs:\c,0);
\coordinate (E') at (axis cs:-\c,0);
\coordinate (O) at (axis cs:0, 0);
\coordinate (B) at (axis cs:-1,.714);
\coordinate (A) at (axis cs:1,.714);
\coordinate (D) at (axis cs:1,-.714);
\coordinate (C) at (axis cs:-1,-.714);
\draw (A)--(B)--(C)--(D)--cycle;
\tkzLabelPoint[below right](E){\footnotesize$E$}
\tkzLabelPoint[below left](E'){\footnotesize$E'$}
\tkzLabelPoint[below left=1mm,fill=white,inner sep=.2mm](O){$O$}
\tkzLabelPoint[above,fill=white,inner sep=.3mm,yshift=1mm](A){\footnotesize$M$}
\tkzLabelPoint[above,fill=white,inner sep=.3mm,yshift=1mm](B){\footnotesize$M_1$}
\tkzLabelPoint[below,fill=white,inner sep=.3mm,yshift=-1mm](C){\footnotesize$M_2$}
\tkzLabelPoint[below,fill=white,inner sep=.3mm,yshift=-1mm](D){\footnotesize$M_3$}
\end{axis}
\tkzDrawPoints(E,E',F,F',A,B,C,D)
\end{tikzpicture} \\ 
\end{tabular}
\end{center}
\Thewrhma{Ανακλαστική ιδιότητα υπερβολής}
Η εφαπτόμενη ευθεία μιας υπερβολής με εστίες $ E,E' $ σε ένα τυχαίο σημείο της $ M $, διχοτομεί τη γωνία που σχηματίζουν τα ευθύγραμμα τμήματα $ ME $ και $ ME' $ που ενώνουν το σημείο με τις εστίες της υπερβολής. Σύμφωνα με την ιδιότητα αυτή, μια ευθεία με κατεύθυνση μια εστία της υπερβολής, ανακλάται στην καμπύλη και παίρνει κατεύθυνση προς την άλλη εστία της.
\end{document}
