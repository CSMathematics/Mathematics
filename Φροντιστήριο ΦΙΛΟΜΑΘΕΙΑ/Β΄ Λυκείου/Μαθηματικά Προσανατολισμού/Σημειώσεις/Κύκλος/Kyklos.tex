\PassOptionsToPackage{no-math,cm-default}{fontspec}
\documentclass[twoside,nofonts,internet,shmeiwseis]{thewria}
\usepackage{amsmath}
\usepackage{xgreek}
\let\hbar\relax
\defaultfontfeatures{Mapping=tex-text,Scale=MatchLowercase}
\setmainfont[Mapping=tex-text,Numbers=Lining,Scale=1.0,BoldFont={Minion Pro Bold}]{Minion Pro}
\newfontfamily\scfont{GFS Artemisia}
\font\icon = "Webdings"
\usepackage[amsbb]{mtpro2}
\usepackage{tikz,pgfplots,gensymb,tkz-euclide,tkz-fct,varwidth}
\tkzSetUpPoint[size=7,fill=white]
\xroma{red!70!black}
\newlist{rlist}{enumerate}{3}
\setlist[rlist]{itemsep=0mm,label=\roman*.}
\newlist{brlist}{enumerate}{3}
\setlist[brlist]{itemsep=0mm,label=\bf\roman*.}
\newlist{tropos}{enumerate}{3}
\setlist[tropos]{label=\bf\textit{\arabic*\textsuperscript{oς}\;Τρόπος :},leftmargin=0cm,itemindent=2.3cm,ref=\bf{\arabic*\textsuperscript{oς}\;Τρόπος}}
\newcommand{\tss}[1]{\textsuperscript{#1}}
\newcommand{\tssL}[1]{\MakeLowercase{\textsuperscript{#1}}}
\usepackage{hhline}
\usepackage{multicol}
\usepackage{mathimatika,wrap-rl}
\setlist[itemize]{itemsep=0mm}




\begin{document}
\titlos{Μαθηματικά Κατεύθυνσης Β' Λυκείου}{Κωνικές Τομές}{Κύκλος}
\orismoi
\Orismos{Κύκλος}
Κύκλος ονομάζεται το σύνολο όλων των σημείων του επιπέδου που έχουν σταθερή απόσταση από ένα σταθερό σημείο του ίδιου επιπέδου.
\begin{itemize}
\item Το σταθερό σημείο ονομάζεται \textbf{κέντρο} του κύκλου.
\item Η σταθερή απόσταση των σημείων του κύκλου από το κέντρο ονομάζεται \textbf{ακτίνα} του κύκλου : $  KM=\rho $.
\item Ένας κύκλος συμβολίζεται ως $ (K,\rho) $ όπου $ K $ είναι το κέντρο και $ \rho $ η ακτίνα του.
\begin{center}
\begin{tabular}{p{5cm}cp{5cm}}
\begin{tikzpicture}
\begin{axis}[xmin=-2.2,xmax=2.2,ymin=-2.2,ymax=2.2,x=1cm,y=1cm,
ticks=none,xlabel={\footnotesize $ x $},ylabel={\footnotesize $ y $},
aks_on,belh ar]
\coordinate (O) at (axis cs:0, 0);
\coordinate (A) at (axis cs:1,1.24);
\coordinate (B) at (axis cs:0,1.6);
\coordinate (C) at (axis cs:1.6,0);
\coordinate (D) at (axis cs:0,-1.6);
\coordinate (E) at (axis cs:-1.6,0);
\node at(axis cs:.7,.6){\footnotesize$\rho$};
\end{axis}
\draw[pl,\xrwma] (O) circle (1.6);
\draw[pl] (A)--(O);
\tkzDrawPoints(A,O,B,C,D,E)
\tkzLabelPoint[right,yshift=2mm](A){\footnotesize$M(x,y)$}
\tkzLabelPoint[below left](O){\footnotesize$O$}
\tkzLabelPoint[above,xshift=-1.4mm](B){\footnotesize$B(0,\rho)$}
\tkzLabelPoint[below,fill=white,inner sep=.1mm,yshift=-1mm](C){\footnotesize$A(\rho,0)$}
\tkzLabelPoint[below,xshift=-.7mm](D){\footnotesize$\varDelta(0,-\rho)$}
\tkzLabelPoint[below,fill=white,inner sep=.1mm,yshift=-1mm](E){\footnotesize$\varGamma(-\rho,0)$}
\node at (2.2,5){\footnotesize$x^2+y^2=\rho^2$};
\end{tikzpicture}
 &  & \begin{tikzpicture}
 \begin{axis}[xmin=-.7,xmax=3.4,ymin=-.7,ymax=3.7,x=1cm,y=1cm, ticks=none,xlabel={\footnotesize $ x $},ylabel={\footnotesize $ y $},
 aks_on,belh ar]
 \coordinate (O) at (axis cs:1,1);
 \coordinate (A) at (axis cs:2,2.24);
 \node at(axis cs:1.2,1.7){\footnotesize$\rho$};
 \end{axis}
 \draw[pl,\xrwma] (O) circle (1.6);
 \draw[pl] (A)--(O);
 \tkzDrawPoints(A,O)
 \tkzLabelPoint[above,xshift=4mm](A){\footnotesize$M(x,y)$}
 \tkzLabelPoint[below](O){\footnotesize$K(x_0,y_0)$}
 \node[fill=white,inner sep=.1mm] at (2.2,3.8){\footnotesize$(x-x_0)^2+(y-y_0)^2=\rho^2$};
 \tkzLabelPoint[below left](0.7,.7){\footnotesize$O$}
 \end{tikzpicture} \\ 
\end{tabular} 
\end{center}
\item Η καμπύλη του κύκλου με κέντρο το σημείο $ K(x_0,y_0) $ και ακτίνα $ \rho $, παριστάνεται αλγεβρικά από την εξίσωση
\[ (x-x_0)^2+(y-y_0)^2=\rho^2 \]
όπου $ x,y $ είναι οι συντεταγμένες των σημείων $ M(x,y) $ του κύκλου.
\item Αν ο κύκλος έχει κέντρο την αρχή των αξόνων τότε θα έχει εξίσωσή της μορφής $ x^2+y^2=\rho^2 $. Αν η ακτίνα του κύκλου αυτού είναι ίση με τη μονάδα τότε ο κύκλος ονομάζεται \textbf{μοναδιαίος} και έχει εξίσωση $ x^2+y^2=1 $.
\end{itemize}
\Orismos{Εφαπτομένη κύκλου}
\wrapr{-5mm}{7}{4.5cm}{-4mm}{\begin{tikzpicture}
\begin{axis}[xmin=-2.2,xmax=2.2,ymin=-1.8,ymax=2.2,x=.8cm,y=.8cm,
ticks=none,xlabel={\footnotesize $ x $},ylabel={\footnotesize $ y $},
aks_on,belh ar]
\coordinate (O) at (axis cs:0, 0);
\coordinate (A) at (axis cs:-.97,1);
\node at(axis cs:-.3,.6){\footnotesize$\rho$};
\addplot[domain=-2.2:2,pl,\xrwma] {.97*x+1.96};
\end{axis}
\draw[pl] (O) circle (1.12);
\draw[pl] (A)--(O);
\tkzDrawPoints(A,O)
\tkzLabelPoint[left,yshift=2mm](A){\footnotesize$A(x_1,y_1)$}
\tkzLabelPoint[below left](O){\footnotesize$O$}
\end{tikzpicture}}{
Εφαπτομένη ενός κύκλου $ (K,\rho) $ σε ένα σημείο $ A(x_1,y_1) $ ονομάζεται η ευθεία η οποία εφάπτεται στον κύκλο στο σημείο αυτό, έχει δηλαδή ένα μόνο κοινό σημείο με τον κύκλο.
\begin{itemize}
\item Η εφαπτόμενη ευθεία για τον κύκλο $ x^2+y^2=\rho^2 $ με κέντρο την αρχή των αξόνων έχει εξίσωση \[ xx_1+yy_1=\rho^2 \]
\end{itemize}

\begin{itemize}
\item Η εφαπτόμενη ευθεία του κύκλου με κέντρο $ K(x_0,y_0) $ και ακτίνα $ \rho $ έχει εξίσωση  $ \overrightarrow{KA}\cdot\overrightarrow{AM}=0 $ όπου $ M $ είναι ένα τυχαίο σημείο της ευθείας.
\end{itemize}}\mbox{}\\\\
\thewrhmata
\Thewrhma{Η εξίσωση {\MakeLowercase{$\mathbold{ x^2+y^2+\MakeUppercase{A}x+\MakeUppercase{B}y+\varGamma=0} $}}}
Κάθε εξίσωση της μορφής $ x^2+y^2+Ax+By+\varGamma=0 $ παριστάνει κύκλο με κέντρο το σημείο $ K\left(-\frac{A}{2},-\frac{B}{2} \right) $ και ακτίνα $ \rho=\frac{\sqrt{A^2+B^2-4\varGamma}}{2} $ αν και μόνο αν ισχύει η σχέση $ A^2+B^2-4\varGamma>0 $. Αντιστρόφως, κάθε κύκλος με κέντρο $ K(x_0,y_0) $ και ακτίνα $ \rho $ έχει εξίσωση την μορφής $ x^2+y^2+Ax+By+\varGamma=0 $.
\begin{rlist}
\item Αν ισχύει $ A^2+B^2-4\varGamma=0 $ τότε η παραπάνω εξίσωση παριστάνει ένα σημείο, το $ K\left(-\frac{A}{2},-\frac{B}{2} \right) $.
\item Αν ισχύει $ A^2+B^2-4\varGamma<0 $ τότε η παραπάνω εξίσωση δεν έχει λύσεις και κατά συνέπεια κανενός σημείου οι συντεταγμένες δεν την επαληθεύουν.
\end{rlist}
\end{document}

