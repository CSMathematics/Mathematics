\PassOptionsToPackage{no-math,cm-default}{fontspec}
\documentclass[twoside,nofonts,internet,shmeiwseis]{thewria}
\usepackage{amsmath}
\usepackage{xgreek}
\let\hbar\relax
\defaultfontfeatures{Mapping=tex-text,Scale=MatchLowercase}
\setmainfont[Mapping=tex-text,Numbers=Lining,Scale=1.0,BoldFont={Minion Pro Bold}]{Minion Pro}
\newfontfamily\scfont{GFS Artemisia}
\font\icon = "Webdings"
\usepackage[amsbb]{mtpro2}
\usepackage{tikz,pgfplots,gensymb,tkz-euclide}
\tkzSetUpPoint[size=7,fill=white]
\xroma{red!70!black}
\newlist{rlist}{enumerate}{3}
\setlist[rlist]{itemsep=0mm,label=\roman*.}
\newlist{brlist}{enumerate}{3}
\setlist[brlist]{itemsep=0mm,label=\bf\roman*.}
\newlist{tropos}{enumerate}{3}
\setlist[tropos]{label=\bf\textit{\arabic*\textsuperscript{oς}\;Τρόπος :},leftmargin=0cm,itemindent=2.3cm,ref=\bf{\arabic*\textsuperscript{oς}\;Τρόπος}}
\newcommand{\tss}[1]{\textsuperscript{#1}}
\newcommand{\tssL}[1]{\MakeLowercase{\textsuperscript{#1}}}
\usepackage{hhline}
\usepackage{multicol}
\usepackage{mathimatika,longtable,wrap-rl}
\setlist[itemize]{itemsep=0mm}





\begin{document}
\titlos{Μαθηματικά Κατεύθυνσης Β' Λυκείου}{Ευθεία}{Η ευθεία στο επίπεδο}
\orismoi
\Orismos{Εξίσωση γραμμής}
Εξίσωση μιας γραμμής $ C $ του επιπέδου, ονομάζεται μια εξίσωση με δύο άγνωστους $ x,y $ η οποία επαληθεύεται μόνο από τις συντεταγμένες των σημείων της γραμμής.\\\\
\Orismos{Συντελεστής διεύθυνσης ευθείας}
Συντελεστής διεύθυνσης $ \lambda $ μιας ευθείας $ \varepsilon $ ονομάζεται η εφαπτομένη της γωνίας $ \hat{\omega} $ που σχηματίζει η ευθεία με τον οριζόντιο άξονα $ x'x $.
\[ \lambda=\ef{\omega} \]
\wrapr{-11mm}{8}{4cm}{0mm}{
\begin{tikzpicture}
\begin{axis}[aks_on,belh ar,xlabel={\footnotesize $x$},ylabel={\footnotesize $ y $},xmin=-.5,xmax=4,ymin=-.5,ymax=3,x=.8cm,y=.8cm,ticks=none]
\addplot[domain=-1:3.5,pl] {0.8*x-.7};
\end{axis}
\tkzDefPoint(1.08,0.4){A}
\tkzDrawPoint(A)
\tkzLabelPoint[above](A){$A$}
\draw (1.08,0.4) ++(0:.7) arc (0:38.6:.7);
\node at (2,.7) {$ \omega $};
\node at (2.7,2) {$ \varepsilon $};
\end{tikzpicture}
}{
\begin{itemize}
\item Η γωνία $ \hat{\omega} $ παίρνει τιμές από $ 0\degree $ μέχρι $ 180\degree $ : $ 0\degree\leq\omega\leq 180\degree $.
\item Κορυφή της γωνίας είναι το σημείο τομής της ευθείας με τον άξονα $ x'x $.
\item Αν $ \hat{\omega}=0\textrm{ ή }\hat{\omega}=180\degree\Leftrightarrow \varepsilon\parallel x'x $ και $ \lambda=0 $.
\item Αν $ \hat{\omega}=90\degree\Leftrightarrow \varepsilon\parallel y'y $ και δεν ορίζεται συντελεστής διεύθυνσης.
\item Αν $ \vec{\delta} $ είναι ένα διάνυσμα παράλληλο στην ευθεία $ \varepsilon $ τότε έχουν τον ίδιο συντελεστή διεύθυνσης.
\end{itemize}}
\thewrhmata
\Thewrhma{Συντελεστής διεύθυνσης ευθείας με γνωστά άκρα}
Αν $ A(x_1,y_1) $ και $ B(x_2,y_2) $ είναι δύο τυχαία σημεία του επιπέδου τότε ο συντελεστής διεύθυνσης της ευθείας που διέρχεται από τα σημεία αυτά ισούται με το πηλίκο της διαφοράς των τεταγμένων προς τη διαφορά των τετμημένων του σημείου $ A $ από το σημείο $ B $.
\[ \lambda_{AB}=\frac{y_2-y_1}{x_2-x_1} \]
\Thewrhma{Συνθήκες παραλληλίας και καθετότητας ευθειών}
Αν $ \varepsilon_1 $ και $ \varepsilon_2 $ είναι δύο ευθείες του επιπέδου και $ \vec{\delta_1},\vec{\delta_2} $ τα παράλληλα διανύσματα των ευθειών αντίστοιχα, τότε ισχύουν οι παρακάτω συνθήκες :
\begin{rlist}
\item Οι ευθείες είναι παράλληλες αν και μόνο αν έχουν ίσους συντελεστές διεύθυνσης :  \[ \varepsilon_1\parallel\varepsilon_2\Leftrightarrow\vec{\delta_1}\parallel\vec{\delta_2}\Leftrightarrow\lambda_1=\lambda_2 \]
\item Οι ευθείες είναι κάθετες αν και μόνο αν το γινόμενο των συντελεστών διεύθυνσής τους ισούται με $ -1 $.
\[ \varepsilon_1\bot\varepsilon_2\Leftrightarrow\vec{\delta_1}\bot\vec{\delta_2}\Leftrightarrow\lambda_1\cdot\lambda_2=-1 \]
\end{rlist}
\Thewrhma{Εξίσωση ευθείας}
Η εξίσωση της ευθείας που διέρχεται από ένα σταθερό σημείο $ A(x_0,y_0) $ του επιπέδου και έχει συντελεστή διεύθυνσης $ \lambda $ δίνεται από τον παρακάτω τύπο
\[ y-y_0=\lambda(x-x_0) \]
\begin{rlist}
\item Αν το σημείο $ A $ ανήκει στον κατακόρυφο άξονα τότε η ευθεία γράφεται στη μορφή $ y=\lambda x+\beta $.
\item Αν η ευθεία διέρχεται από την αρχή των αξόνων θα είναι της μορφής $ y=\lambda x $.
\item Οι οριζόντιες ευθείες έχουν εξίσωση της μορφής $ y=y_0 $.
\item Οι κατακόρυφες ευθείες έχουν εξίσωση της μορφής $ x=x_0 $.
\end{rlist}
\end{document}

