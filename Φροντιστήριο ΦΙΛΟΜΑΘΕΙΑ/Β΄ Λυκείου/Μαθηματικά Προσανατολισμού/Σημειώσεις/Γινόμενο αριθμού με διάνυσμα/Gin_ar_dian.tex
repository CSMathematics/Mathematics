\PassOptionsToPackage{no-math,cm-default}{fontspec}
\documentclass[twoside,nofonts,internet,shmeiwseis]{thewria}
\usepackage{amsmath}
\usepackage{xgreek}
\let\hbar\relax
\defaultfontfeatures{Mapping=tex-text,Scale=MatchLowercase}
\setmainfont[Mapping=tex-text,Numbers=Lining,Scale=1.0,BoldFont={Minion Pro Bold}]{Minion Pro}
\newfontfamily\scfont{GFS Artemisia}
\font\icon = "Webdings"
\usepackage[amsbb]{mtpro2}
\usepackage{tikz,pgfplots}
\tkzSetUpPoint[size=7,fill=white]
\xroma{red!70!black}

\usepackage{longtable}
\newlist{rlist}{enumerate}{3}
\setlist[rlist]{itemsep=0mm,label=\textcolor{\xrwma}{\roman*.}}
\newlist{brlist}{enumerate}{3}
\setlist[brlist]{itemsep=0mm,label=\bf\roman*.}
\newlist{tropos}{enumerate}{3}
\setlist[tropos]{label=\bf\textit{\arabic*\textsuperscript{oς}\;Τρόπος :},leftmargin=0cm,itemindent=2.3cm,ref=\bf{\arabic*\textsuperscript{oς}\;Τρόπος}}
\newcommand{\tss}[1]{\textsuperscript{#1}}
\newcommand{\tssL}[1]{\MakeLowercase{\textsuperscript{#1}}}

\usepackage{hhline}
\usepackage{gensymb}
\setlist[enumerate]{label=\bf{\large{\textcolor{\xrwma}{\arabic*.}}}}
\usepackage{multicol}
\usepackage{wrap-rl}
\usepackage{mathimatika}

\begin{document}
\titlos{ΜΑΘΗΜΑΤΙΚΑ ΚΑΤΕΥΘΥΝΣΗΣ Β΄ ΛΥΚΕΙΟΥ}{ΔΙΑΝΥΣΜΑΤΑ}{Γινόμενο αριθμού με διάνυσμα}
\orismoi
\Orismos{Γινόμενο αριθμού με διάνυσμα}
Γινόμενο ενός πραγματικού αριθμού $ \lambda $ με διάνυσμα $ \vec{a} $ ονομάζεται το διάνυσμα $ \lambda\cdot\vec{a} $ το οποίο είναι 
\begin{itemize}[itemsep=0mm]
\item παράλληλο με το διάνυσμα $ \vec{a} $ και
\item έχει μέτρο πολλαπλάσιο του μέτρου του $ \vec{a} $ ίσο με $ |\lambda|\cdot|\vec{a}| $.
\end{itemize}
Αν $ \lambda>0 $ τότε το διάνυσμα $ \lambda\cdot\vec{a} $ είναι ομόρροπο με το $ \vec{a} $, ενώ αν $ \lambda<0 $ τότε το διάνυσμα $ \lambda\cdot\vec{a} $ είναι αντίρροπο με το $ \vec{a} $.\\\\
\Orismos{Γραμμικός συνδυασμός}
Γραμμικός συνδυασμός δύο διανυσμάτων $ \vec{a} $ και $ \vec{\beta} $ ονομάζεται κάθε διάνυσμα $ \vec{\delta} $ το οποίο μπορεί να γραφτεί με τη βοήθεια των $ \vec{a} $ και $ \vec{\beta} $ στη μορφή :
\[ \vec{\delta}=\lambda\cdot\vec{a}+\mu\cdot\vec{\beta} \] όπου $ \lambda,\mu $ είναι πραγματικοί αριθμοί. Γενικότερα ο γραμμικός συνδυασμός $ \nu $ σε πλήθος διανυσμάτων $ \vec{a_1},\vec{a_2},\ldots,\vec{a_2} $ θα έχει ομοίως τη μορφή
\[ \vec{\nu}=\lambda_1\cdot\vec{a_1}+\lambda_2\cdot\vec{a_2}+\ldots+\lambda_\nu\cdot\vec{a_\nu} \]
\thewrhmata
\Thewrhma{Ιδιότητες πολλαπλασιασμού}
Για οποιαδήποτε διανύσματα $ \vec{a},\vec{\beta} $ και πραγματικούς αριθμούς $ \lambda,\mu $ ισχύουν οι ακόλουθες ιδιότητες για την πράξη του πολλαπλασιασμού διανύσματος με αριθμό.
\begin{center}
\begin{longtable}{cc}
\hline \rule[-2ex]{0pt}{5.5ex} \textbf{Ιδιότητα} & \textbf{Συνθήκη} \\ 
\hhline{==} \rule[-2ex]{0pt}{5.5ex} Επιμεριστική (ως προς αριθμό) & $ \lambda\left( \vec{a}\pm\vec{\beta}\right)=\lambda\cdot\vec{a}\pm\lambda\cdot\vec{\beta} $ \\ 
\rule[-2ex]{0pt}{5.5ex} Επιμεριστική (ως προς διάνυσμα) & $ \left( \lambda\pm\mu\right)\cdot\vec{a}=\lambda\cdot\vec{a}\pm\mu\cdot\vec{a} $ \\
\rule[-2ex]{0pt}{5.5ex} Προσεταιριστική & $ \lambda\left( \mu\vec{a}\right)=\left( \lambda\cdot\mu\right)\cdot\vec{a} $ \\ 
\rule[-2ex]{0pt}{5.5ex} Μηδενικό γινόμενο & $ \lambda\cdot\vec{a}=\vec{0}\Leftrightarrow \lambda=0 \ \textrm{ή}\ \vec{a}=\vec{0} $ \\ 
\rule[-2ex]{0pt}{5.5ex} Πρόσημο γινομένου & $ \left( -\lambda\cdot\vec{a}\right)=(-\lambda)\cdot\vec{a}=-\left( \lambda\cdot\vec{a}\right)  $ \\ 
\rule[-2ex]{0pt}{5.5ex} Νόμος διαγραφής (ως προς διάνυσμα) & Αν $ \lambda\cdot\vec{a}=\mu\cdot\vec{a} $ και $ \vec{a}\neq0 $ τότε $ \lambda=\mu $ \\ 
\rule[-2ex]{0pt}{5.5ex} Νόμος διαγραφής (ως προς αριθμό) & Αν $ \lambda\cdot\vec{a}=\lambda\cdot\vec{\beta} $ και $ \lambda\neq0 $ τότε $ \vec{a}=\vec{\beta} $\\ 
\hline 
\end{longtable}
\end{center} 
\Thewrhma{Συνθήκη παραλληλίας διανυσμάτων}
Δύο μη μηδενικά διανύσματα $ \vec{a} $ και $ \vec{\beta} $ είναι παράλληλα αν και μόνο αν υπάρχει πραγματικός αριθμός $ \lambda\in\mathbb{R} $ ώστε το ένα διάνυσμα να είναι πολλαπλάσιο του άλλου.
\[ \vec{a}\parallel\vec{\beta}\Leftrightarrow \vec{a}=\lambda\cdot\vec{\beta}\ ,\ \lambda\in\mathbb{R} \]
\begin{rlist}
\item Αν $ \lambda>0 $ τότε τα διανύσματα είναι ομόρροπα : $ \vec{a}\upuparrows\vec{\beta} $
\item Αν $ \lambda<0 $ τότε τα διανύσματα είναι αντίρροπα : $ \vec{a}\updownarrows\vec{\beta} $
\end{rlist}
\Thewrhma{Διανυσματική ακτίνα μέσου τμήματος}
Η διανυσματική ακτίνα του μέσου $ M $ ενός ευθύγραμμου τμήματος $ AB $ ισούται με το ημιάθροισμα των διανυσματικών ακτίνων των άκρων $ A $ και $ B $.
\begin{center}
\begin{tikzpicture}
\dianysma[\xrwma]{0,0.5}{2,2}{A}{D}
\dianysma{0,0.5}{1,3}{A}{B}
\dianysma{0,0.5}{3,1}{A}{C}
\draw[pl](B)--(C);
\tkzLabelPoint[above](B){$A$}
\tkzLabelPoint[above right](C){$B$}
\tkzLabelPoint[above right](D){$M$}
\tkzLabelPoint[left](A){$O$}
\tkzDrawPoints(B,C,D)
\node at (6,2){$ \overrightarrow{OM}=\dfrac{\overrightarrow{OA}+\overrightarrow{OB}}{2} $};
\end{tikzpicture}
\end{center}
\Thewrhma{Διανυσματική ακτίνα βαρύκεντρου}
Η διανυσματική ακτίνα $ \overrightarrow{OG} $ του βαρύκεντρου $ G $ ενός τριγώνου $ AB\varGamma $ ισούται με το ένα τρίτο του αθροίσματος των διανυσματικών ακτίνων $ \overrightarrow{OA},\overrightarrow{OB} $ και $ \overrightarrow{O\varGamma} $ των κορυφών του τριγώνου.
\begin{center}
\begin{tikzpicture}
\tkzDefPoint(3,0){C}
\tkzDefPoint(0,0){B}
\tkzDefPoint(1,2){A}
\tkzDefPoint(1.33,.66){G}
\tkzDefPoint(1.5,0){M}
\tkzDefPoint(.5,1){N}
\tkzDefPoint(2,1){P}
\draw[pl](A)--(B)--(C)-- cycle;
\draw(A)--(M);
\draw(B)--(P);
\draw(C)--(N);
\tkzLabelPoint[left](B){$B$}
\tkzLabelPoint[below](M){$\varDelta$}
\tkzLabelPoint[right](P){$E$}
\tkzLabelPoint[left](N){$Z$}
\tkzLabelPoint[right](C){$\varGamma$}
\tkzLabelPoint[above](A){$A$}
\tkzLabelPoint[left,yshift=-3mm,xshift=1.7mm](G){$G$}
\tkzDrawPoints(A,B,C,M,N,P,G)
\node at (6,1.5){$ \overrightarrow{OG}=\dfrac{\overrightarrow{OA}+\overrightarrow{OB}+\overrightarrow{O\varGamma}}{3} $};
\node at (6,.5){$ \overrightarrow{GA}+\overrightarrow{GB}+\overrightarrow{G\varGamma}=\vec{0} $};
\end{tikzpicture}
\end{center}
Επιπλέον το άθροισμα των διανυσματικών ακτίνων των κορυφών $ A,B $ και $ \varGamma $ του τριγώνου με σημείο αναφοράς το βαρύκεντρο του, ισούται με το μηδενικό διάνυσμα.
\end{document}





