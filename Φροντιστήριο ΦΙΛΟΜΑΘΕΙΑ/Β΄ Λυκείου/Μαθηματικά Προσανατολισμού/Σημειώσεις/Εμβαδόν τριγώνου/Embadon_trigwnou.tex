\PassOptionsToPackage{no-math,cm-default}{fontspec}
\documentclass[twoside,nofonts,internet,shmeiwseis]{thewria}
\usepackage{amsmath}
\usepackage{xgreek}
\let\hbar\relax
\defaultfontfeatures{Mapping=tex-text,Scale=MatchLowercase}
\setmainfont[Mapping=tex-text,Numbers=Lining,Scale=1.0,BoldFont={Minion Pro Bold}]{Minion Pro}
\newfontfamily\scfont{GFS Artemisia}
\font\icon = "Webdings"
\usepackage[amsbb]{mtpro2}
\usepackage{tikz,pgfplots,gensymb,tkz-euclide}
\tkzSetUpPoint[size=7,fill=white]
\xroma{red!70!black}
\newlist{rlist}{enumerate}{3}
\setlist[rlist]{itemsep=0mm,label=\roman*.}
\newlist{brlist}{enumerate}{3}
\setlist[brlist]{itemsep=0mm,label=\bf\roman*.}
\newlist{tropos}{enumerate}{3}
\setlist[tropos]{label=\bf\textit{\arabic*\textsuperscript{oς}\;Τρόπος :},leftmargin=0cm,itemindent=2.3cm,ref=\bf{\arabic*\textsuperscript{oς}\;Τρόπος}}
\newcommand{\tss}[1]{\textsuperscript{#1}}
\newcommand{\tssL}[1]{\MakeLowercase{\textsuperscript{#1}}}
\usepackage{hhline}
\usepackage{multicol}
\usepackage{mathimatika,longtable}
\setlist[itemize]{itemsep=0mm}






\begin{document}
\titlos{Μαθηματικά Κατεύθυνσης Β' Λυκείου}{Ευθεία}{Εμβαδόν τριγώνου}
\thewrhmata
\Thewrhma{Απόσταση σημείου από ευθεία}
Η απόσταση ενός σημείου $ A(x_0,y_0) $ του επιπέδου από μια ευθεία $(\varepsilon) : Ax+By+\varGamma=0 $ συμβολίζεται με $ d(A,\varepsilon) $ και δίνεται από τον παρακάτω τύπο.
\[ d(A,\varepsilon)=\frac{|Ax_0+By_0+\varGamma|}{\sqrt{A^2+B^2}} \]
\Thewrhma{Εμβαδόν τριγώνου}
Το εμβαδόν ενός τριγώνου $ AB\varGamma $ με κορυφές $ A(x_1,y_1),B(x_2,y_2) $ και $ \varGamma(x_3,y_3) $ ισούται με τη μισή απόλυτη τιμή της ορίζουσας των διανυσμάτων δύο πλευρών του τριγώνου.
\[ (AB\varGamma)=\frac{1}{2}\left|\det{(\overrightarrow{AB},\overrightarrow{A\varGamma})} \right|=\frac{1}{2}\left|\det{(\overrightarrow{AB},\overrightarrow{B\varGamma})} \right|=\frac{1}{2}\left|\det{(\overrightarrow{B\varGamma},\overrightarrow{A\varGamma})} \right| \]
\end{document}

