\documentclass[ektypwsh]{diag-xelatex}
\usepackage[amsbb,subscriptcorrection,zswash,mtpcal,mtphrb]{mtpro2}
\usepackage[no-math,cm-default]{fontspec}
\usepackage{xunicode}
\usepackage{xgreek}
\usepackage{amsmath}
\defaultfontfeatures{Mapping=tex-text,Scale=MatchLowercase}
\setmainfont[Mapping=tex-text,Numbers=Lining,Scale=1.0,BoldFont={Minion Pro Bold}]{Minion Pro}
\newfontfamily\scfont{GFS Artemisia}
\font\icon = "Webdings"
\usepackage[amsbb,subscriptcorrection,zswash,mtpcal,mtphrb]{mtpro2}
\xroma{red!80!black}
%------TIKZ - ΣΧΗΜΑΤΑ - ΓΡΑΦΙΚΕΣ ΠΑΡΑΣΤΑΣΕΙΣ ----
\usepackage{tikz}
\usepackage{tkz-euclide}
\usetkzobj{all}
\usepackage[framemethod=TikZ]{mdframed}
\usetikzlibrary{decorations.pathreplacing}
\usepackage{pgfplots}
\usetkzobj{all}
%-----------------------
\usepackage{calc}
\usepackage{hhline}
\usepackage[explicit]{titlesec}
\usepackage{graphicx}
\usepackage{multicol}
\usepackage{multirow}
\usepackage{enumitem}
\usepackage{tabularx}
\usepackage[decimalsymbol=comma]{siunitx}
\usetikzlibrary{backgrounds}
\usepackage{sectsty}
\sectionfont{\centering}
\usepackage{enumitem}
\setlist[enumerate]{label=\bf{\large \arabic*.}}
\usepackage{adjustbox}
\usepackage{mathimatika,gensymb,eurosym,wrap-rl}
\usepackage{systeme,regexpatch}
%-------- ΜΑΘΗΜΑΤΙΚΑ ΕΡΓΑΛΕΙΑ ---------
\usepackage{mathtools}
%----------------------
%-------- ΠΙΝΑΚΕΣ ---------
\usepackage{booktabs}
%----------------------
%----- ΥΠΟΛΟΓΙΣΤΗΣ ----------
\usepackage{calculator}
%----------------------------
%------ ΔΙΑΓΩΝΙΟ ΣΕ ΠΙΝΑΚΑ -------
\usepackage{array}
\newcommand\diag[5]{%
\multicolumn{1}{|m{#2}|}{\hskip-\tabcolsep
$\vcenter{\begin{tikzpicture}[baseline=0,anchor=south west,outer sep=0]
\path[use as bounding box] (0,0) rectangle (#2+2\tabcolsep,\baselineskip);
\node[minimum width={#2+2\tabcolsep-\pgflinewidth},
minimum  height=\baselineskip+#3-\pgflinewidth] (box) {};
\draw[line cap=round] (box.north west) -- (box.south east);
\node[anchor=south west,align=left,inner sep=#1] at (box.south west) {#4};
\node[anchor=north east,align=right,inner sep=#1] at (box.north east) {#5};
\end{tikzpicture}}\rule{0pt}{.71\baselineskip+#3-\pgflinewidth}$\hskip-\tabcolsep}}
%---------------------------------
%---- ΟΡΙΖΟΝΤΙΟ - ΚΑΤΑΚΟΡΥΦΟ - ΠΛΑΓΙΟ ΑΓΚΙΣΤΡΟ ------
\newcommand{\orag}[3]{\node at (#1)
{$ \overcbrace{\rule{#2mm}{0mm}}^{{\scriptsize #3}} $};}
\newcommand{\kag}[3]{\node at (#1)
{$ \undercbrace{\rule{#2mm}{0mm}}_{{\scriptsize #3}} $};}
\newcommand{\Pag}[4]{\node[rotate=#1] at (#2)
{$ \overcbrace{\rule{#3mm}{0mm}}^{{\rotatebox{-#1}{\scriptsize$#4$}}}$};}
%-----------------------------------------


%------------------------------------------
\newcommand{\tss}[1]{\textsuperscript{#1}}
\newcommand{\tssL}[1]{\MakeLowercase{\textsuperscript{#1}}}
%---------- ΛΙΣΤΕΣ ----------------------
\newlist{bhma}{enumerate}{3}
\setlist[bhma]{label=\bf\textit{\arabic*\textsuperscript{o}\;Βήμα :},leftmargin=0cm,itemindent=1.8cm,ref=\bf{\arabic*\textsuperscript{o}\;Βήμα}}
\newlist{rlist}{enumerate}{3}
\setlist[rlist]{itemsep=0mm,label=\roman*.}
\newlist{brlist}{enumerate}{3}
\setlist[brlist]{itemsep=0mm,label=\bf\roman*.}
\newlist{tropos}{enumerate}{3}
\setlist[tropos]{label=\bf\textit{\arabic*\textsuperscript{oς}\;Τρόπος :},leftmargin=0cm,itemindent=2.3cm,ref=\bf{\arabic*\textsuperscript{oς}\;Τρόπος}}
% Αν μπει το bhma μεσα σε tropo τότε
%\begin{bhma}[leftmargin=.7cm]
\tkzSetUpPoint[size=7,fill=white]
\tikzstyle{pl}=[line width=0.3mm]
\tikzstyle{plm}=[line width=0.4mm]

\begin{document}
\titlos{Άλγεβρα Α΄ Λυκείου}{ΕΠΑΝΑΛΗΠΤΙΚΟ}
\begin{thema}
\item \mbox{}\\
Α. Να αποδείξετε ότι το άθροισμα των λύσεων $ x_1,x_2 $ μιας εξίσωσης δευτέρου βαθμού $ ax^2+\beta x+\gamma=0\ ,\ a\neq0 $ δίνονται από τον τύπο $ S=-\frac{\beta}{a} $.\monades{3}\\\\
Β. \swstolathos
\begin{rlist}
\item Για οποιουσδήποτε αριθμούς $ a,\beta\in\mathbb{R} $ ισχύει $ |a+ \beta|=|a|+|\beta| $.
\item Η πιθανότητα οποιουδήποτε ενδεχομένου $ A $ είναι θετικός αριθμός.
\item Μια εξίσωση δευτέρου βαθμού έχει πάντα δύο λύσεις.
\item Ισχύει $ |x-x_0|<\rho\Rightarrow x_0-\rho<x<x_0+\rho $.
\item Αν για μια εξίσωση δευτέρου βαθμού ισχύει $ P<0 $ τότε οι λύσεις της είναι ετερόσημες.\monades{2}
\end{rlist}
\item \mbox{}\\
Να λυθούν οι παρακάτω ανισώσεις.
\begin{multicols}{2}
\begin{rlist}
\item $ |3x-5|<7 $
\item $ |2-4x|>3 $\monades{2{,}5+2{,}5}
\end{rlist}
\end{multicols}
\item \mbox{}\\
Για δύο ενδεχόμενα $ A $ και $ B $ έχουμε ότι $ P(A)=0{,}3,P(B')=0{,}6 $ και $ P(A\cup B)=0{,}5 $.
\begin{rlist}
\item Να βρεθεί η πιθανότητα $ P(A\cap B) $.\monades{3}
\item Να βρεθούν οι πιθανότητες $ P(A-B) $ και $ P(B-A) $.\monades{2}
\end{rlist}
\item Δίνεται η εξίσωση $ x^2+(2\lambda+1)x+\lambda^2-4=0 $ με $ \lambda\in\mathbb{R} $.
\begin{rlist}
\item Να βρεθεί η διακρίνουσα του τριωνύμου, το άθροισμα $ S $ και το γινόμενο $ P $ των λύσεων της εξίσωσης.\monades{2}
\item Να βρεθεί η τιμή της παραμέτρου $ \lambda $ ώστε η εξίσωση να έχει μια λύση.\monades{1}
\item Να βρεθεί η τιμή της παραμέτρου $ \lambda $ ώστε η εξίσωση να έχει δύο λύσεις ετερόσημες.\monades{2}
\end{rlist}
\end{thema}
\end{document}

