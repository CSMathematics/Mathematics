\documentclass[ektypwsh]{diag-xelatex}
\usepackage[amsbb,subscriptcorrection,zswash,mtpcal,mtphrb]{mtpro2}
\usepackage[no-math,cm-default]{fontspec}
\usepackage{xunicode}
\usepackage{xgreek}
\usepackage{amsmath}
\defaultfontfeatures{Mapping=tex-text,Scale=MatchLowercase}
\setmainfont[Mapping=tex-text,Numbers=Lining,Scale=1.0,BoldFont={Minion Pro Bold}]{Minion Pro}
\newfontfamily\scfont{GFS Artemisia}
\font\icon = "Webdings"
\usepackage[amsbb,subscriptcorrection,zswash,mtpcal,mtphrb]{mtpro2}
\xroma{red!80!black}
%------TIKZ - ΣΧΗΜΑΤΑ - ΓΡΑΦΙΚΕΣ ΠΑΡΑΣΤΑΣΕΙΣ ----
\usepackage{tikz}
\usepackage{tkz-euclide}
\usetkzobj{all}
\usepackage[framemethod=TikZ]{mdframed}
\usetikzlibrary{decorations.pathreplacing}
\usepackage{pgfplots}
\usetkzobj{all}
%-----------------------
\usepackage{calc}
\usepackage{hhline}
\usepackage[explicit]{titlesec}
\usepackage{graphicx}
\usepackage{multicol}
\usepackage{multirow}
\usepackage{enumitem}
\usepackage{tabularx}
\usepackage[decimalsymbol=comma]{siunitx}
\usetikzlibrary{backgrounds}
\usepackage{sectsty}
\sectionfont{\centering}
\usepackage{enumitem}
\setlist[enumerate]{label=\bf{\large \arabic*.}}
\usepackage{adjustbox}
\usepackage{mathimatika,gensymb,eurosym,wrap-rl}
\usepackage{systeme,regexpatch}
%-------- ΜΑΘΗΜΑΤΙΚΑ ΕΡΓΑΛΕΙΑ ---------
\usepackage{mathtools}
%----------------------
%-------- ΠΙΝΑΚΕΣ ---------
\usepackage{booktabs}
%----------------------
%----- ΥΠΟΛΟΓΙΣΤΗΣ ----------
\usepackage{calculator}
%----------------------------
%------ ΔΙΑΓΩΝΙΟ ΣΕ ΠΙΝΑΚΑ -------
\usepackage{array}
\newcommand\diag[5]{%
\multicolumn{1}{|m{#2}|}{\hskip-\tabcolsep
$\vcenter{\begin{tikzpicture}[baseline=0,anchor=south west,outer sep=0]
\path[use as bounding box] (0,0) rectangle (#2+2\tabcolsep,\baselineskip);
\node[minimum width={#2+2\tabcolsep-\pgflinewidth},
minimum  height=\baselineskip+#3-\pgflinewidth] (box) {};
\draw[line cap=round] (box.north west) -- (box.south east);
\node[anchor=south west,align=left,inner sep=#1] at (box.south west) {#4};
\node[anchor=north east,align=right,inner sep=#1] at (box.north east) {#5};
\end{tikzpicture}}\rule{0pt}{.71\baselineskip+#3-\pgflinewidth}$\hskip-\tabcolsep}}
%---------------------------------
%---- ΟΡΙΖΟΝΤΙΟ - ΚΑΤΑΚΟΡΥΦΟ - ΠΛΑΓΙΟ ΑΓΚΙΣΤΡΟ ------
\newcommand{\orag}[3]{\node at (#1)
{$ \overcbrace{\rule{#2mm}{0mm}}^{{\scriptsize #3}} $};}
\newcommand{\kag}[3]{\node at (#1)
{$ \undercbrace{\rule{#2mm}{0mm}}_{{\scriptsize #3}} $};}
\newcommand{\Pag}[4]{\node[rotate=#1] at (#2)
{$ \overcbrace{\rule{#3mm}{0mm}}^{{\rotatebox{-#1}{\scriptsize$#4$}}}$};}
%-----------------------------------------


%------------------------------------------
\newcommand{\tss}[1]{\textsuperscript{#1}}
\newcommand{\tssL}[1]{\MakeLowercase{\textsuperscript{#1}}}
%---------- ΛΙΣΤΕΣ ----------------------
\newlist{bhma}{enumerate}{3}
\setlist[bhma]{label=\bf\textit{\arabic*\textsuperscript{o}\;Βήμα :},leftmargin=0cm,itemindent=1.8cm,ref=\bf{\arabic*\textsuperscript{o}\;Βήμα}}
\newlist{rlist}{enumerate}{3}
\setlist[rlist]{itemsep=0mm,label=\roman*.}
\newlist{brlist}{enumerate}{3}
\setlist[brlist]{itemsep=0mm,label=\bf\roman*.}
\newlist{tropos}{enumerate}{3}
\setlist[tropos]{label=\bf\textit{\arabic*\textsuperscript{oς}\;Τρόπος :},leftmargin=0cm,itemindent=2.3cm,ref=\bf{\arabic*\textsuperscript{oς}\;Τρόπος}}
% Αν μπει το bhma μεσα σε tropo τότε
%\begin{bhma}[leftmargin=.7cm]
\tkzSetUpPoint[size=7,fill=white]
\tikzstyle{pl}=[line width=0.3mm]
\tikzstyle{plm}=[line width=0.4mm]

\begin{document}
\titlos{Γεωμετρία Α΄ Λυκείου}{ΕΠΑΝΑΛΗΠΤΙΚΟ}
\begin{thema}
\item Α. Να αποδείξετε ότι σε κάθε ισοσκελές τρίγωνο οι γωνίες πάνω στη βάση του είναι ίσες.\monades{3}\\\\
Β. \swstolathos
\begin{rlist}
\item Σε κάθε τρίγωνο με πλευρές $ a,\beta,\gamma $ ισχύει η σχέση $ \beta-\gamma<a<\beta+\gamma $.
\item Δύο εντός εναλλάξ γωνίες είναι μεταξύ τους ίσες.
\item Τα εφαπτόμενα τμήματα προς ένα κύκλο που άγονται από ένα εξωτερικό σημείο του είναι ίσα.
\item Αν δύο ορθογώνια τρίγωνα έχουν δύο γωνίες ίσες μια προς μια τότε είναι ίσα.
\item Τα σημεία της διχοτόμου μιας γωνίας ισαπέχουν από τις πλευρές της.\monades{2}
\end{rlist}
\item Έστω $ AB\varGamma $ ένα ισοσκελές τρίγωνο με $ AB=A\varGamma $ και $ B\varDelta,\varGamma E $ διάμεσοι του τριγώνου. Αν $ M $ είναι το σημείο τομής των δύο διαμέσων τότε να αποδείξετε ότι
\begin{rlist}
\item $ B\varDelta=\varGamma E $.\monades{2}
\item $ M\varDelta=ME $.\monades{1}
\item το τρίγωνο $ MB\varGamma $ είναι ισοσκέλές.\monades{2}
\end{rlist}
\item Δίνεται ισοσκελές τρίγωνο $ AB\varGamma (AB=A\varGamma)$ και η διάμεσός του $ AM $. Φέρουμε $ \varGamma x\bot B\varGamma $ προς το ημιεπίπεδο που δεν ανήκει το $ A $ και παίρνουμε σε αυτή τμήμα $ \varGamma\varDelta=AB $. 
\begin{rlist}
\item Να αποδείξετε ότι η $ A\varDelta $ είναι διχοτόμος της γωνίας $ M\hat{A} \varGamma$.\monades{3}
\item Να αποδείξετε ότι $ A\hat{\varDelta}\varGamma=45\degree-\frac{\hat{\varGamma}}{2} $.\monades{2}
\end{rlist}
\item Έστω δύο κύκλοι $ (K, R) $ και $ (\varLambda,\rho) $ με $ R>\rho $, που δεν τέμνονται. Φέρουμε τις κοινές εξωτερικές εφαπτόμενες τους. Να δείξετε ότι:
\begin{center}
\begin{tikzpicture}
\tkzDefPoint[label=left:$K$](0,0){K}
\tkzDefPoint[label=right:$O$](4,0){O}
\tkzDefPoint[label=left:$\varLambda$](2,0){L}
\tkzDefPoint[label=above:$A$](75:1){A}
\tkzDefPoint[label=above:$B$,shift={(2,0)}](75:.5){B}
\tkzDefPoint[label=below:$\varGamma$](-75:1){C}
\tkzDefPoint[label=below:$\varDelta$,shift={(2,0)}](-75:.5){D}
\draw[pl](K) circle (1);
\draw[pl](L) circle (.5);
\draw[pl] (A)--(K)--(C);
\draw[pl] (B)--(L)--(D);
\tkzDrawLine[add=.4 and 1.](A,B)
\tkzDrawLine[add=.4 and 1](C,D)
\tkzDrawPoints(K,L,A,B,C,D,O)
\end{tikzpicture}
\end{center}
\begin{rlist}
\item τέμνονται σε σημείο της διακέντρου.\monades{2}
\item οι μεσοκάθετοι των κοινών εξωτερικών εφαπτόμενων τμημάτων τέμνονται σε σημείο της διακέντρου.\\\monades{3}
\end{rlist}
\end{thema}
\end{document}

