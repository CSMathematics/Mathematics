\documentclass[twoside,nofonts,ektypwsh,math,spyros]{frontisthrio-diag}
\usepackage[amsbb,subscriptcorrection,zswash,mtpcal,mtphrb,mtpfrak]{mtpro2}
\usepackage[no-math,cm-default]{fontspec}
\usepackage{amsmath}
\usepackage{xunicode}
\usepackage{xgreek}
\let\hbar\relax
\defaultfontfeatures{Mapping=tex-text,Scale=MatchLowercase}
\setmainfont[Mapping=tex-text,Numbers=Lining,Scale=1.0,BoldFont={Minion Pro Bold}]{Minion Pro}
\newfontfamily\scfont{GFS Artemisia}
\font\icon = "Webdings"
\usepackage{fontawesome5}
\newfontfamily{\FA}{fontawesome.otf}
\xroma{red!70!black}
%------TIKZ - ΣΧΗΜΑΤΑ - ΓΡΑΦΙΚΕΣ ΠΑΡΑΣΤΑΣΕΙΣ ----
\usepackage{tikz,pgfplots}
\usepackage{tkz-euclide}
\usetkzobj{all}
\usepackage[framemethod=TikZ]{mdframed}
\usetikzlibrary{decorations.pathreplacing}
\tkzSetUpPoint[size=7,fill=white]
%-----------------------
\usepackage{calc,tcolorbox}
\tcbuselibrary{skins,theorems,breakable}
\usepackage{hhline}
\usepackage[explicit]{titlesec}
\usepackage{graphicx}
\usepackage{multicol}
\usepackage{multirow}
\usepackage{tabularx}
\usetikzlibrary{backgrounds}
\usepackage{sectsty}
\sectionfont{\centering}
\usepackage{enumitem}
\usepackage{adjustbox}
\usepackage{mathimatika,gensymb,eurosym,wrap-rl}
\usepackage{systeme,regexpatch}
%-------- ΜΑΘΗΜΑΤΙΚΑ ΕΡΓΑΛΕΙΑ ---------
\usepackage{mathtools}
%----------------------
%-------- ΠΙΝΑΚΕΣ ---------
\usepackage{booktabs}
%----------------------
%----- ΥΠΟΛΟΓΙΣΤΗΣ ----------
\usepackage{calculator}
%----------------------------
%------ ΔΙΑΓΩΝΙΟ ΣΕ ΠΙΝΑΚΑ -------
\usepackage{array}
\newcommand\diag[5]{%
\multicolumn{1}{|m{#2}|}{\hskip-\tabcolsep
$\vcenter{\begin{tikzpicture}[baseline=0,anchor=south west,outer sep=0]
\path[use as bounding box] (0,0) rectangle (#2+2\tabcolsep,\baselineskip);
\node[minimum width={#2+2\tabcolsep-\pgflinewidth},
minimum  height=\baselineskip+#3-\pgflinewidth] (box) {};
\draw[line cap=round] (box.north west) -- (box.south east);
\node[anchor=south west,align=left,inner sep=#1] at (box.south west) {#4};
\node[anchor=north east,align=right,inner sep=#1] at (box.north east) {#5};
\end{tikzpicture}}\rule{0pt}{.71\baselineskip+#3-\pgflinewidth}$\hskip-\tabcolsep}}
%---------------------------------
%---- ΟΡΙΖΟΝΤΙΟ - ΚΑΤΑΚΟΡΥΦΟ - ΠΛΑΓΙΟ ΑΓΚΙΣΤΡΟ ------
\newcommand{\orag}[3]{\node at (#1)
{$ \overcbrace{\rule{#2mm}{0mm}}^{{\scriptsize #3}} $};}
\newcommand{\kag}[3]{\node at (#1)
{$ \undercbrace{\rule{#2mm}{0mm}}_{{\scriptsize #3}} $};}
\newcommand{\Pag}[4]{\node[rotate=#1] at (#2)
{$ \overcbrace{\rule{#3mm}{0mm}}^{{\rotatebox{-#1}{\scriptsize$#4$}}}$};}
%-----------------------------------------
%------------------------------------------
\newcommand{\tss}[1]{\textsuperscript{#1}}
\newcommand{\tssL}[1]{\MakeLowercase{\textsuperscript{#1}}}
%---------- ΛΙΣΤΕΣ ----------------------
\newlist{bhma}{enumerate}{3}
\setlist[bhma]{label=\bf\textit{\arabic*\textsuperscript{o}\;Βήμα :},leftmargin=0cm,itemindent=1.8cm,ref=\bf{\arabic*\textsuperscript{o}\;Βήμα}}
\newlist{rlist}{enumerate}{3}
\setlist[rlist]{itemsep=0mm,label=\roman*.}
\newlist{brlist}{enumerate}{3}
\setlist[brlist]{itemsep=0mm,label=\bf\roman*.}
\newlist{tropos}{enumerate}{3}
\setlist[tropos]{label=\bf\textit{\arabic*\textsuperscript{oς}\;Τρόπος :},leftmargin=0cm,itemindent=2.3cm,ref=\bf{\arabic*\textsuperscript{oς}\;Τρόπος}}
% Αν μπει το bhma μεσα σε tropo τότε
%\begin{bhma}[leftmargin=.7cm]
\tkzSetUpPoint[size=7,fill=white]
\tikzstyle{pl}=[line width=0.3mm]
\tikzstyle{plm}=[line width=0.4mm]
\usepackage{etoolbox}
\makeatletter
\renewrobustcmd{\anw@true}{\let\ifanw@\iffalse}
\renewrobustcmd{\anw@false}{\let\ifanw@\iffalse}\anw@false
\newrobustcmd{\noanw@true}{\let\ifnoanw@\iffalse}
\newrobustcmd{\noanw@false}{\let\ifnoanw@\iffalse}\noanw@false
\renewrobustcmd{\anw@print}{\ifanw@\ifnoanw@\else\numer@lsign\fi\fi}
\makeatother

\usepackage{path}
\pathal

\begin{document}
\titlos{Διαγώνισμα}{Άλγεβρα Α΄ Λυκείου}{Ανισώσεις}
\begin{thema}
\item\mbox{}\\\vspace{-7mm}
\begin{erwthma}
\item Να απαντήσετε στις παρακάτω ερωτήσεις.
\begin{alist}
\item Τι ονομάζεται ανίσωση;
\item Τι ονομάζεται λύση μιας ανίσωσης;
\item Να γράψετε τον τύπο με τον οποίο παραγοντοποιείται το τριώνυμο $ ax^2+\beta x+\gamma $, όταν $ \Delta>0 $.
\item Ποια ανίσωση ονομάζεται αδύνατη;
\item Ποια ανίσωση ονομάζεται αόριστη;
\item Να γράψετε τον τύπο με τον οποίο παραγοντοποιείται το τριώνυμο $ ax^2+\beta x+\gamma $, όταν $ \Delta=0 $. 
\end{alist}\monades{3}
\item \swstolathos
\begin{alist}
\item Οι λύσεις της ανίσωσης $ x-2<0 $ είναι το διάστημα $ (2,+\infty) $.
\item H ανίσωση $ ax^2+\beta x+\gamma>0 $ είναι 2\tss{ου} βαθμού για κάθε τιμή του αριθμού $ a $.
\item Η ανίσωση $ |x-1|<-3 $ είναι αδύνατη.
\item Η ανίσωση $ |2x+1|>-1 $ είναι αόριστη.
\item Οι κοινές λύσεις των ανισώσεων $ x\leq 1 $ και $ x<0 $ είναι $ x\in(-\infty,0) $.\monades{2}
\end{alist}
\end{erwthma}
\item\mbox{}\\\vspace{-7mm}
\begin{erwthma}
\item Να λυθούν οι παρακάτω ανισώσεις.
\begin{multicols}{2}
\begin{alist}
\item $ 2(x-3)+7\leq 4-(5-x) $
\item $ \dfrac{1-2x}{3}+\dfrac{x}{4}>1-\dfrac{x+1}{12} $
\end{alist}
\end{multicols}\monades{2}
\item Να λυθούν οι παρακάτω ανισώσεις.
\begin{multicols}{2}
\begin{alist}
\item $ |3x-1|<7 $
\item $ |5-2x|\geq 9 $
\end{alist}
\end{multicols}\monades{2}
\end{erwthma}
\item\mbox{}\\\vspace{-7mm}
\begin{erwthma}
\item Να παραγοντοποιηθούν τα παρακάτω τριώνυμα.
\begin{multicols}{2}
\begin{alist}
\item $ x^2-x-12 $
\item $ 4y^2+4y+1 $
\end{alist}
\end{multicols}\monades{2}
\item Να βρεθούν τα πρόσημα των τιμών καθενός από τα παρακάτω τριώνυμα.
\begin{multicols}{2}
\begin{alist}
\item $ x^2-5x+6 $
\item $ -9x^2+6x-1 $
\end{alist}
\end{multicols}\monades{2}
\item Να λυθούν οι παρακάτω ανισώσεις.
\begin{multicols}{2}
\begin{alist}
\item $ x^2-7x+10>0 $
\item $ -x^2+x+6\leq 0 $
\end{alist}
\end{multicols}\monades{2}
\end{erwthma}
\item\mbox{}\\\vspace{-7mm}
\begin{erwthma}
\item Να βρεθούν οι κοινές λύσεις των παρακάτω ανισώσεων.
\[ |x-7|<3\ \ \textrm{και}\ \ x^2-12x+35\geq 0 \]\monades{2}
\item Δίνεται η παραμετρική εξίσωση
\[ x^2+(\lambda-3)x+3-\lambda=0 \]
όπου $ \lambda\in\mathbb{R} $.
\begin{alist}
\item Να γραφτούν οι συντελεστές $ a,\beta,\gamma $ της εξίσωσης και να δείξετε ότι η διακρίνουσα της είναι \[ \Delta=\lambda^2-2\lambda-3 \]
\monades{1}
\item Να βρείτε τις τιμές του $ \lambda $ ώστε $ \Delta>0 $.\monades{1,5}
\item Να βρείτε τις τιμές του $ \lambda $ ώστε η εξίσωση να είναι αδύνατη. \monades{0,5}
\end{alist}
\end{erwthma}
\end{thema}
\kaliepityxia
\end{document}
