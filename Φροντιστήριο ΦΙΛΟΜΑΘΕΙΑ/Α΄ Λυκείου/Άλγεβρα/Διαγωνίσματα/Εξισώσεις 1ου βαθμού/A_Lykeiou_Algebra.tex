\documentclass[11pt,a4paper]{article}
\usepackage[english,greek]{babel}
\usepackage[utf8]{inputenc}
\usepackage{nimbusserif}
\usepackage[T1]{fontenc}
\usepackage[left=2.00cm, right=2.00cm, top=2.00cm, bottom=2.00cm]{geometry}
\usepackage{amsmath}
\let\myBbbk\Bbbk
\let\Bbbk\relax
\usepackage[amsbb,subscriptcorrection,zswash,mtpcal,mtphrb,mtpfrak]{mtpro2}
\usepackage{siunitx, graphicx,multicol,multirow,enumitem,tabularx,mathimatika,gensymb,venndiagram,hhline,longtable,tkz-euclide,fontawesome5,eurosym,tcolorbox,wrap-rl}
\tcbuselibrary{skins,theorems,breakable}
\newlist{rlist}{enumerate}{3}
\setlist[rlist]{itemsep=0mm,label=\roman*.}
\newlist{alist}{enumerate}{3}
\setlist[alist]{itemsep=0mm,label=\alph*.}
\newlist{balist}{enumerate}{3}
\setlist[balist]{itemsep=0mm,label=\bf\alph*.}
\newlist{Alist}{enumerate}{3}
\setlist[Alist]{itemsep=0mm,label=\Alph*.}
\newlist{bAlist}{enumerate}{3}
\setlist[bAlist]{itemsep=0mm,label=\bf\Alph*.}
\renewcommand{\textstigma}{\textsigma\texttau}
\newlist{thema}{enumerate}{3}
\setlist[thema]{label=\bf\large{ΘΕΜΑ \textcolor{black}{\Alph*}},itemsep=0mm,leftmargin=0cm,itemindent=18mm}
\newlist{erwthma}{enumerate}{3}
\setlist[erwthma]{label=\bf{\large{\textcolor{black}{\Alph{themai}.\arabic*}}},itemsep=0mm,leftmargin=0.8cm}

\newcommand{\lysh}{\textcolor{black}{\textbf{\faCheck\ \ ΛΥΣΗ}}}
\renewcommand{\textstigma}{\textsigma\texttau}
%----------- ΟΡΙΣΜΟΣ------------------
\newcounter{orismos}[section]
\renewcommand{\theorismos}{\thesection.\arabic{orismos}}   
\newcommand{\Orismos}{\refstepcounter{orismos}{\textbf{\textcolor{black}{\kerkissans{Ορισμός\hspace{2mm}\theorismos}}\;:\;}{}}}

\newenvironment{orismos}[1]
{\begin{tcolorbox}[title=\Orismos {\textcolor{black}{\kerkissans{#1}}},breakable,bottomtitle=-1.5mm,
enhanced standard,titlerule=-.2pt,toprule=0pt, rightrule=0pt, bottomrule=0pt,
colback=white,left=2mm,top=1mm,bottom=0mm,
boxrule=0pt,
colframe=white,borderline west={1.5mm}{0pt}{black},leftrule=2mm,sharp corners,coltitle=black]}
{\end{tcolorbox}}

\newcommand{\kerkissans}[1]{{\fontfamily{maksf}\selectfont \textbf{#1}}}
\renewcommand{\textdexiakeraia}{}

\usepackage[
backend=biber,
style=alphabetic,
sorting=ynt
]{biblatex}

\begin{document}
\begin{center}
{\LARGE \kerkissans{Άλγεβρα Α' Λυκείου}}\\
{\large \kerkissans{Επαναληπτικό διαγώνισμα - Εξισώσεις 1ου βαθμού\\\today}}
\end{center}
\begin{thema}
\item\mbox{}\\\vspace{-5mm}
\begin{erwthma}
\item Να απαντήσετε στις παρακάτω ερωτήσεις.
\begin{alist}
\item Τι ονομάζεται εξίσωση?
\item Ποια εξίσωση ονομάζεται αόριστη (ταυτότητα) και ποια αδύνατη?
\item Τι ονομάζουμε λύση μιας εξίσωσης?
\end{alist}
\item Να χαρακτηρίσετε καθεμία από τις παρακάτω προτάσεις ως \textbf{Σωστή} ή \textbf{Λάθος}.
\begin{alist}
\item Αν μια εξίσωση 1ου βαθμού επαληθεύεται για $x=2$ και $x=3$ τότε είναι αόριστη.
\item Η εξίσωση $0x=1$ είναι αδύνατη.
\item Η εξίσωση $2x=0$ είναι αδύνατη.
\item Ο αριθμός $x=2$ είναι λύση της εξίσωσης $ 0x=0 $.
\item Η εξίσωση $(\lambda-3)x=1-\lambda$ έχει μοναδική λύση για κάθε $\lambda\neq 3$.
\end{alist}
\end{erwthma}
\item\mbox{}\\
\vspace{-7mm}
\begin{erwthma}
\item Να λύσετε τις εξισώσεις
\begin{alist}
\item $ 2(x-1)+3=7-(x-3) $
\item $ -5(x+1)+2(3-x)=8-(3x-1) $
\item $ \dfrac{2x-1}{5}-\dfrac{4+x}{3}=1-\dfrac{x+18}{15} $
\end{alist}
\end{erwthma}
\item Δίνεται η παραμετρική εξίσωση
\[ \lambda^2x-4(x+\lambda)=\lambda^2-2\lambda \]
\begin{erwthma}
\item Να βρείτε τις λύσεις της εξίσωσης για τις διάφορες τιμές της παραμέτρου $\lambda\in\mathbb{R}$.
\item Να βρείτε τις τιμές του $\lambda$ ώστε η εξίσωση να έχει λύση την $x=2$.
\end{erwthma}
\item\mbox{}\\\vspace{-7mm}
\begin{erwthma}
\item Να λύσετε την κλασματική εξίσωση
\[ \frac{x}{x^2-4}-\frac{2}{x^2-2x}=\frac{1}{x^2+2x} \]
\item Να βρείτε τις τιμές της παραμέτρου $\mu\in\mathbb{R}$ ώστε η εξίσωση
\[ \frac{x-1}{\mu^2-x^2}+\frac{1}{\mu+x}=\frac{2}{\mu-x} \]
να έχει μοναδική λύση την $x=-2$.
\end{erwthma}
\end{thema}
\end{document}
