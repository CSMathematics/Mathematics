\documentclass[internet]{diag-xelatex}
\usepackage[amsbb]{mtpro2}
\usepackage[no-math,cm-default]{fontspec}
\usepackage{xunicode}
\usepackage{xltxtra}
\usepackage{xgreek}
\usepackage{amsmath}
\defaultfontfeatures{Mapping=tex-text,Scale=MatchLowercase}
\setmainfont[Mapping=tex-text,Numbers=Lining,Scale=1.0,BoldFont={Minion Pro Bold}]{Minion Pro}
\newfontfamily\scfont{GFS Artemisia}
\font\icon = "Webdings"
\usepackage[amsbb]{mtpro2}
\usepackage[left=2.00cm, right=2.00cm, top=2.00cm, bottom=3.00cm]{geometry}
\xroma{green!50!blue}
\newcommand{\tss}[1]{\textsuperscript{#1}}
\newcommand{\tssL}[1]{\MakeLowercase\textsuperscript{#1}}
\newlist{rlist}{enumerate}{3}
\setlist[rlist]{itemsep=0mm,label=\roman*.}
\usepackage{multicol}

\begin{document}
\titlos{ΑΛΓΕΒΡΑ Α΄ ΛΥΚΕΙΟΥ}{ΕΞΙΣΩΣΕΙΣ 2\MakeUppercase{ου} ΒΑΘΜΟΥ}
\begin{thema}
\item \textbf{ΘΕΩΡΙΑ}
\begin{rlist}
\item Αν $ x_1,x_2 $ είναι οι λύσεις της εξίσωσης 2\tss{ου} βαθμού $ ax^2+\beta x+\gamma=0 $ να αποδειχτούν οι τύποι του Vieta :
\[ S=-\frac{\beta}{a}\ \ \textrm{και}\ \ P=\frac{\gamma}{a} \tag*{\monades{3}}\]
\item Να χαρακτηριστούν οι παρακάτω εξισώσεις ως σωστές (Σ) ή λανθασμένες (Λ).
\begin{enumerate}[label=\alph*.]
\item Αν για μια εξίσωση 2\textsuperscript{ου} βαθμού έχουμε $ \varDelta>0 $ τότε έχει 2 άνισες λύσεις.
\item Αν για μια εξίσωση 2\textsuperscript{ου} βαθμού έχουμε $ \varDelta<0 $ τότε έχει μια διπλή λύση.
\item Η εξίσωση $ ax^2+\beta x+\gamma=0 $ παριστάνει μια εξίσωση 2\textsuperscript{ου} βαθμού για κάθε τιμή του $ a $.
\item Αν $ x_1, x_2 $ είναι οι λύσεις μιας εξίσωσης 2\textsuperscript{ου} βαθμού τότε : $ x_1+x_2=\frac{\beta}{a} $ και $ x_1\cdot x_2=\frac{\gamma}{a} $.
\item Αν $ x_1, x_2 $ είναι οι λύσεις μιας εξίσωσης 2\textsuperscript{ου} βαθμού με $ x_1=-x_2 $ τότε $ \beta=0 $.
\end{enumerate}\monades{2}
\end{rlist}
\item \textbf{ΕΞΙΣΩΣΕΙΣ 2\tss{ου} ΒΑΘΜΟΎ}\\
Να λυθεί η παρακάτω εξίσωση για την οποία ισχύει $ x\neq0 $.\[ \left(1+\frac{1}{x}\right)^2+3\cdot\frac{x+1}{x}-2=0  \]\monades{5}
\item \textbf{ΠΑΡΑΜΕΤΡΙΚΗ ΕΞΙΣΩΣΗ}\\
Να δειχθεί οτι η εξίσωση
\[ x^2+x-\lambda^2=0 \]
έχει 2 άνισες λύσεις για κάθε τιμή του $ \lambda\in\mathbb{R} $.\monades{5}
\item \textbf{ΠΑΡΑΜΕΤΡΙΚΗ ΕΞΙΣΩΣΗ}\\
Αν $ x_1,x_2 $ είναι οι λύσεις της παρακάτω παραμετρικής εξίσωσης
\[ x^2-(\lambda-2)+\lambda+2=0 \]
με $ \lambda\in\mathbb{R} $, τότε να βρεθεί η τιμή της παραμέτρου $ \lambda $ ώστε
\begin{rlist}
\item Η εξίσωση να έχει μια διπλή λύση.\monades{3}
\item $ x_1^2x_2+x_1x_2^2=-3 $\monades{2}
\end{rlist}
\end{thema}
\end{document}
