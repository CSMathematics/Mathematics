\documentclass[ektypwsh]{diag-xelatex}
\usepackage[amsbb]{mtpro2}
\usepackage[no-math,cm-default]{fontspec}
\usepackage{xunicode}
\usepackage{xltxtra}
\usepackage{xgreek}
\usepackage{amsmath}
\defaultfontfeatures{Mapping=tex-text,Scale=MatchLowercase}
\setmainfont[Mapping=tex-text,Numbers=Lining,Scale=1.0,BoldFont={Minion Pro Bold}]{Minion Pro}
\newfontfamily\scfont{GFS Artemisia}
\font\icon = "Webdings"
\usepackage[amsbb]{mtpro2}
\usepackage[left=2.00cm, right=2.00cm, top=2.00cm, bottom=3.00cm]{geometry}
\xroma{red!80!black}
\newcommand{\tss}[1]{\textsuperscript{#1}}
\newcommand{\tssL}[1]{\MakeLowercase\textsuperscript{#1}}
\newlist{rlist}{enumerate}{3}
\setlist[rlist]{itemsep=0mm,label=\roman*.}
\usepackage{mathimatika,multicol,gensymb}


\begin{document}
\titlos{Άλγεβρα Α΄ Λυκείου}{ΕΞΙΣΩΣΕΙΣ}
\askhseis
\begin{thema}
\item \textbf{Θεωρία}\\
Να απαντήσετε στις παρακάτω ερωτήσεις.
\begin{rlist}
\item Ποιά συνθήκη πρέπει να ισχύει ώστε η εξίσωση $ ax^2+\beta x+\gamma=0,\ a\neq 0 $ να έχει πραγματικές ρίζες;
\item Πόσες και ποίες ρίζες έχει η εξίσωση $ x^\nu=a $ αν γνωρίζουμε ότι ο $ \nu $ είναι άρτιος και $ a>0 $.
\item Γράψτε τους τύπους του Vieta για τιν εξίσωση $ ax^2+\beta x+\gamma=0,\ a\neq 0 $ με λύσεις $ x_1,x_2 $.
\item Ποιές συνθήκες πρέπει να ισχύουν ώστε η εξίσωση $ ax^2+\beta x+\gamma=0,\ a\neq 0 $ να έχει θετικές ρίζες;
\end{rlist}\monades{5}
\item \textbf{Παραμετρική εξίσωση 1\tss{ου} βαθμού}\\
Να λυθούν οι παρακάτω εξισώσεις.
\begin{multicols}{2}
\begin{rlist}
\item $ (\lambda -2)x-4=\lambda^2\cdot(x-1) $
\item $ \dfrac{1-\mu}{x-1}+\dfrac{\mu}{x}=\dfrac{1}{x^2-x} $\monades{5}
\end{rlist}
\end{multicols}
\item \textbf{Εξίσώσεις 2\tss{ου} βαθμού}\\
Να λυθούν οι παρακάτω εξισώσεις.
\begin{multicols}{2}
\begin{rlist}
\item $ x^4-5x^2=-4 $
\item $ (x-2)^2-8|x-2|+12=0 $\monades{5}
\end{rlist}
\end{multicols}
\item \textbf{Σύνθετο θέμα}\\
Δίνεται η εξίσωση 2\tss{ου} βαθμού $ x^2-(2\lambda-1)x+\lambda^2-\lambda-2=0 $ όπου $ \lambda\in\mathbb{R} $ είναι μια πραγματική παράμετρος και $ x_1,x_2 $ είναι οι ρίζες της εξίσωσης.
\begin{rlist}
\item Να βρεθούν οι τιμές της παραμέτρου $ \lambda $ ώστε η εξίσωση να έχει δύο ρίζες ίσες.\monades{2}
\item Να βρεθούν οι τιμές του $ \lambda $ ώστε η εξίσωση να έχει μια μηδενική και μια θετική ρίζα.\monades{3}
\end{rlist}
\end{thema}
\end{document}

