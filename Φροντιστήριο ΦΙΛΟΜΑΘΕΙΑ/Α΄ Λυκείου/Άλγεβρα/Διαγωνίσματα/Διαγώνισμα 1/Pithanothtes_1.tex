\documentclass[internet]{diag-xelatex}
\usepackage[amsbb]{mtpro2}
\usepackage[no-math,cm-default]{fontspec}
\usepackage{xunicode}
\usepackage{xltxtra}
\usepackage{xgreek}
\usepackage{amsmath}
\usepackage{enumitem}
\defaultfontfeatures{Mapping=tex-text,Scale=MatchLowercase}
\setmainfont[Mapping=tex-text,Numbers=Lining,Scale=1.0,BoldFont={Minion Pro Bold}]{Minion Pro}
\newfontfamily\scfont{GFS Artemisia}
\font\icon = "Webdings"
\usepackage[amsbb]{mtpro2}
\xroma{red!80!black}
\newcommand{\tss}[1]{\textsuperscript{#1}}
\newcommand{\tssL}[1]{\MakeLowercase\textsuperscript{#1}}
\newlist{rlist}{enumerate}{3}
\setlist[rlist]{itemsep=0mm,label=\textcolor{\xrwma}{\roman*.}}


\begin{document}
\titlos{ΑΛΓΕΒΡΑ Α΄ ΛΥΚΕΙΟΥ}{ΠΙΘΑΝΟΤΗΤΕΣ}
\begin{thema}
\item \textbf{Θεωρία}\\
Να απαντήσετε στις παρακάτω ερωτήσεις.
\begin{rlist}
\item Πως ορίζεται στον κλασικό ορισμό η πιθανότητα ενός ενδεχομένου $ A $ ενός δειγματικού χώρου $ \varOmega $;
\item Να γράψετε τη σχέση που δίνει την πιθανότητα της ένωσης δύο ενδεχομένων $ A $ και $ B $.
\item Να αποδείξετε οτι για κάθε ενδεχόμενο $ A $ ισχύει $ P(A')=1-P(A) $.
\item Συμπληρώστε το κενό στην πρόταση : $ P(A-B)=\ldots\ldots\ldots\ldots $
\end{rlist}\monades{5}
\item \textbf{Κανόνες λογισμού πιθανοτήτων}\\
Ο καθηγητής των Μαθηματικών επιλέγει τυχαία μαθητές από όλο το σχολείο για να συμμετάσχουν στη Διεθνή Μαθηματική Ολυμπιάδα του 2015. Σε κάθε μαθητή εξετάζει τα εξής στοιχεία :
\begin{itemize}[itemsep=0mm]
\item την τάξη στην οποία ανήκει ο μαθητής (Α, Β, ή Γ, Λυκείου)
\item το φύλο του μαθητή (αγόρι (α) ή κορίστι (κ))
\item αν έχει συμμετάσχει ξανά ο μαθητής σε διαγωνισμό (ν) ή όχι (ο).
\end{itemize}
Αν ο καθηγητής επιλέξει τυχαία έναν μαθητή από το σχολείο τότε
\begin{rlist}
\item Να βρεθεί ο δειγματικός χώρος του πειράματος.\monades{2}
\item Να βρεθούν τα παρακάτω ενδεχόμενα :\\
Α : ``Ο μαθητής να είναι αγόρι''.\\
Β : ``Ο μαθητής να ανήκει σε κάποια ομάδα προσανατολισμού".\\
Γ : ``Ο μαθητής να είναι κορίτσι και να έχει συμμετάσχει ξανά σε διαγωνισμό Μαθηματικών.\\\monades{2}
\item Να υπολογιστούν οι πιθανότητες των παραπάνω ενδεχομένων.\monades{1}
\end{rlist}
\item \mbox{}\\
Για δύο ενδεχόμενα $ A,B $ ενός δειγματικού χώρου $ \varOmega $ ισχύουν οι σχέσεις $ P(A-B)=\frac{1}{4} $, $ P(B-A)=\frac{3}{8}$ και $ P(A)+P(B)=\frac{11}{12} $
\begin{rlist}
\item Να βρεθούν οι πιθανότητες $ P(A\cap B) $ και $ P(A\cup B) $.\monades{3}
\item Να βρεθούν οι πιθανότητες $ P(A), P(B) $ των ενδεχομένων $ A,B $.\monades{2}
\end{rlist}
\item \mbox{}\\
Έστω $ A,B $ δύο ενδεχόμενα ενός δειγματικού χώρου $ \varOmega $ για τα οποία ισχύει $ P(A)=0{,}5 $ και $ P(B-A)=0{,}4 $. Να αποδειχθεί οτι
\begin{rlist}
\item $ P(A)\leq 0{,}9 $\monades{2}
\item Αν $ P(B)=0{,}7 $ τότε $ 0{,}2\leq P(A\cap B)\leq 0{,}5 $\monades{3}
\end{rlist}
\end{thema}
\end{document}
