\documentclass[a4paper,11pt]{article}
\usepackage[utf8]{inputenc}
\usepackage[english,greek]{babel}
\usepackage{alphabeta}
\usepackage[T1]{fontenc}
\RequirePackage[left=2.00cm, right=2.00cm, top=2.00cm, bottom=2.00cm]{geometry}
\usepackage{nimbusserif,multicol,longtable,multirow,hhline,enumitem,tikz,pgfplots,tkz-euclide,tkz-tab,capt-of,fontawesome5,gensymb,mathimatika}
\let\myBbbk\Bbbk
\let\Bbbk\relax
\usepackage[amsbb]{mtpro2}
\usepackage{xcolor}
\usepackage[explicit]{titlesec}
\usepackage{soul}
\newcommand{\eng}{\selectlanguage{english}}
\newcommand{\gr}{\selectlanguage{greek}}

\newcommand{\ekthetesdeiktes}{\DeclareMathSizes{10.95}{10.95}{7}{5}
\DeclareMathSizes{6}{6}{3.8}{2.7}
\DeclareMathSizes{8}{8}{5.1}{3.6}
\DeclareMathSizes{9}{9}{5.8}{4.1}
\DeclareMathSizes{10}{10}{6.4}{4.5}
\DeclareMathSizes{12}{12}{7.7}{5.5}
\DeclareMathSizes{14.4}{14.4}{9.2}{6.5}
\DeclareMathSizes{17.28}{17.28}{11}{7.9}
\DeclareMathSizes{20.74}{20.74}{13.3}{9.4}
\DeclareMathSizes{24.88}{24.88}{16}{11.3}


\makeatletter
\newcommand{\subsup}{
\AtBeginDocument{
\check@mathfonts
\fontdimen16\textfont2=2.5pt
\fontdimen17\textfont2=2.5pt
\fontdimen14\textfont2=4.5pt
\fontdimen13\textfont2=4.5pt}
}
\makeatother}
\usepackage{wrapfig}
\newenvironment{WrapText1}[3][r]
{\wrapfigure[#2]{#1}{#3}}
{\endwrapfigure}

\newenvironment{WrapText2}[3][l]
{\wrapfigure[#2]{#1}{#3}}
{\endwrapfigure}
\newcommand{\wrapr}[6]{
\begin{minipage}{\linewidth}\mbox{}\\
\vspace{#1}
\begin{WrapText1}{#2}{#3}
\vspace{#4}#5\end{WrapText1}#6
\end{minipage}}

\newcommand{\wrapl}[6]{
\begin{minipage}{\linewidth}\mbox{}\\
\vspace{#1}
\begin{WrapText2}{#2}{#3}
\vspace{#4}#5\end{WrapText2}#6
\end{minipage}}
\usepackage{etoolbox,hhline,moreenum}
\usepackage{caption} 
\captionsetup[table]{skip=5pt}
\makeatletter
\newif\ifLT@nocaption
\preto\longtable{\LT@nocaptiontrue}
\appto\endlongtable{%
\ifLT@nocaption
\addtocounter{table}{\m@ne}%
\fi}
\preto\LT@caption{%
\noalign{\global\LT@nocaptionfalse}}
\makeatother

\newlist{alist}{enumerate}{1}
\setlist[alist]{label=\let\textdexiakeraia\relax\alph*.}

\pgfmathdeclarefunction{gauss}{2}{%
  \pgfmathparse{1/(#2*sqrt(2*pi))*exp(-((x-#1)^2)/(2*#2^2))}%
}
\pgfkeys{/pgfplots/aks_on/.style={axis lines=center,
xlabel style={at={(current axis.right of origin)},xshift=1.5ex,anchor=center},
ylabel style={at={(current axis.above origin)},yshift=1.5ex, anchor=center}}}
\pgfkeys{/pgfplots/grafikh parastash/.style={red,line width=.4mm,samples=200}}
\pgfkeys{/pgfplots/belh ar/.style={tick label style={font=\scriptsize},axis line style={-latex}}}
\tikzstyle{pl}=[line width=0.3mm]
\tikzstyle{plm}=[line width=0.4mm]

\newcommand{\kerkissans}[1]{{\fontfamily{maksf}\selectfont {#1}}}
\AtBeginDocument{\renewcommand{\textstigma}{\textsigma\texttau}}

\definecolor{titlecolor}{HTML}{cd0f00}

\newbox\TitleUnderlineTestBox
\newcommand*\TitleUnderline[1]
  {%
    \bgroup
    \setbox\TitleUnderlineTestBox\hbox{\colorbox{titleblue}\strut}%
    \setul{\dimexpr\dp\TitleUnderlineTestBox-.3ex\relax}{.3ex}%
    \ul{#1}%
    \egroup
  }
\newcommand*\SectionNumberBox[1]
  {%
    \colorbox{red!80!black}
      {%
        \makebox[2em][c]
          {%
            \color{white}%
            \strut
            \csname the#1\endcsname
          }%
      }%
    \TitleUnderline{\ \ \ }%
  }
\titleformat{\section}[hang]
    {\Large\bfseries\fontfamily{maksf}\selectfont}%
    {\colorbox{red!80!black}{%
      \raisebox{0pt}[13pt][3pt]{\makebox[80pt]{% height, width
          \color{white}{\kerkissans{\thesection ο Κεφάλαιο}}}%
      }}}%
    {0pt}%
    {\colorbox{black}{\raisebox{0pt}[13pt][3pt]{\color{white}\ #1\ }}}

\titleformat{\subsection}[hang]
    {\large\bfseries\fontfamily{maksf}\selectfont}%
    {\colorbox{red!80!black}{%
      \raisebox{0pt}[13pt][3pt]{\makebox[30pt]{% height, width
          \color{white}{\kerkissans{\thesubsection}}}%
      }}}%
    {0pt}%
    {\colorbox{black}{\raisebox{0pt}[13pt][3pt]{\color{white}\ #1\ }}}

\makeatletter
\@addtoreset{section}{part}
\makeatother

\titleformat{\part}[display]
  {\normalfont\huge\filcenter\bfseries}{}{-30pt}{\Huge \textcolor{red!80!black}{ \kerkissans{ #1}}}
\titlespacing*{\part} 
  {0pt}{0pt}{0pt}

\setlist[enumerate]{itemsep=0mm,label=\thesection.\arabic*}
\definecolor{bblue}{HTML}{4F81BD}
\definecolor{rred}{HTML}{C0504D}
\definecolor{ggreen}{HTML}{9BBB59}
\definecolor{ppurple}{HTML}{9F4C7C}

\makeatletter
\usetikzlibrary{patterns}
\tikzstyle{chart}=[
legend label/.style={font={\scriptsize},anchor=west,align=left},
legend box/.style={rectangle, draw, minimum size=5pt},
axis/.style={black,semithick,->},
axis label/.style={anchor=east,font={\tiny}},
]

\tikzstyle{bar chart}=[
chart,
bar width/.code={
\pgfmathparse{##1/2}
\global\let\bar@w\pgfmathresult
},
bar/.style={very thick, draw=white},
bar label/.style={font={\bf\small},anchor=north},
bar value/.style={font={\footnotesize}},
bar width=.75,
]

\tikzstyle{pie chart}=[
chart,
slice/.style={line cap=round, line join=round,thick,draw=white},
pie title/.style={font={\bf}},
slice type/.style 2 args={
##1/.style={fill=##2},
values of ##1/.style={}
}
]

\pgfdeclarelayer{background}
\pgfdeclarelayer{foreground}
\pgfsetlayers{background,main,foreground}


\newcommand{\pie}[3][]{
\begin{scope}[#1]
\pgfmathsetmacro{\curA}{90}
\pgfmathsetmacro{\r}{1}
\def\c{(0,0)}
\node[pie title] at (90:1.3) {#2};
\foreach \v/\s/\l in{#3}{
\pgfmathsetmacro{\deltaA}{\v/100*360}
\pgfmathsetmacro{\nextA}{\curA + \deltaA}
\pgfmathsetmacro{\midA}{(\curA+\nextA)/2}

\path[slice,\s] \c
-- +(\curA:\r)
arc (\curA:\nextA:\r)
-- cycle;
\pgfmathsetmacro{\d}{max((\deltaA * -(.5/50) + 1) , .5)}

\begin{pgfonlayer}{foreground}
\path \c -- node[pos=\d,pie values,values of \s]{$\l$} +(\midA:\r);
\end{pgfonlayer}

\global\let\curA\nextA
}
\end{scope}
}

\newcommand{\legend}[2][]{
\begin{scope}[#1]
\path
\foreach \n/\s in {#2}
{
++(0,-10pt) node[\s,legend box] {} +(5pt,0) node[legend label] {\n}
}
;
\end{scope}
}
\definecolor{a}{cmyk}{0,1,1,0.05}
\definecolor{b}{cmyk}{0,.8,.8,.15}
\definecolor{c}{cmyk}{0,.8,.8,.0}
\definecolor{d}{cmyk}{0,.7,.7,0}
\definecolor{e}{cmyk}{0,.5,.5,0}


\pgfplotsset{every axis/.append style={
x tick label style={/pgf/number format/.cd, 1000 sep={.}}}}

\begin{document}
\begin{center}
\includegraphics[width=0.4\linewidth]{/home/spyros/texmf/tex/latex/local/frontisthrio/Logotypo-Filomatheia_1}\\
\vspace{-1mm}
{\faIcon{map-marker-alt}} : Ιακώβου Πολυλά 24 - \ Πεζόδρομος\,\,|\,\,{\faIcon{phone-alt}} : 26610 20144\,\,|\,\, {\faIcon{mobile-alt}} : 6932327283 - 6955058444\\
\rule{14.7cm}{.1mm}\\
\vspace{2mm}
{\bf\today}\\
\vspace{3cm}
{\Huge \textbf{Άλγεβρα Α' Λυκείου}}\\
\vspace*{3cm}
{{\LARGE }\textbf{ΤΥΠΟΛΟΓΙΟ ΚΑΙ ΜΕΘΟΔΟΛΟΓΙΑ}\\[2mm]\textbf{ΒΑΣΙΚΩΝ ΑΣΚΗΣΕΩΝ}}
\vspace*{\fill}
\vfil
Φρόνιμος Σπύρος
\end{center}
\pagenumbering{gobble}
\newpage
\null
\newpage
\pagenumbering{arabic}
\begin{center}
\part{Τυπολόγιο}
\end{center}
\section{Σύνολα}
\subsection{Η έννοια του συνόλου}
\begin{enumerate}[label=\thesection.\arabic*]
\item Σύνολο : Ομάδα όμοιων αντικειμένων.
\item Το $ x $ ανήκει στο σύνολο $ A $: $ x\in A $.
\item Κενό σύνολο : Το σύνολο χωρίς στοιχεία : $ \varnothing $.
\item Βασικά σύνολα αριθμών
\begin{alist}
\item Φυσικοί αριθμοί : $\mathbb{N}=\{0,1,2,\ldots\}$
\item Ακέραιοι αριθμοί : $\mathbb{Z}=\{\ldots,-2,-1,0,1,2,\ldots\}$
\item Ρητοί Αριθμοί : $ \mathbb{Q}=\left\lbrace \left. \frac{a}{\beta}\right|a,\beta\in\mathbb{Z},\beta\neq0\;\right\rbrace  $.
\item Άρρητοι Αριθμοί : Κάθε αριθμός που δεν είναι ρητός.
\item Πραγματικοί Αριθμοί : $\mathbb{R}=\{ \textrm{όλοι οι αριθμοί} \} $.
\end{alist}
\item Ίσα σύνολα : $ A=B $ αν έχουν τα ίδια στοιχεία. 
\item Υποσύνολο : $ A\subseteq B $.
\end{enumerate}
\subsection{Πράξεις συνόλων}
\begin{enumerate}[resume]
\item Πράξεις μεταξύ συνόλων
\begin{alist}
\item Ένωση : $ A\cup B=\left\lbrace x\in\varOmega\left| x\in A \textrm{ ή } x\in B\right.\right\rbrace $
\item Τομή : $ A\cap B=\left\lbrace x\in\varOmega\left| x\in A \textrm{ και } x\in B\right.\right\rbrace $
\item Συμπλήρωμα : $ A'=\left\lbrace x\in\varOmega\left| x\notin A\right.\right\rbrace $
\item Διαφορά : $ A-B=\left\lbrace x\in\varOmega\left| x\in A\textrm{ και }x\notin B\right. \right\rbrace $
\end{alist}
\end{enumerate}
\section{Πραγματικοί αριθμοί}
\subsection{Πράξεις πραγματικών αριθμών}
\begin{enumerate}
\item Δύναμη πραγματικού αριθμού: $ a\cdot a\cdot\ldots a=a^\nu $. Ο $ a $ λέγεται \textbf{βάση} και ο $ \nu $ \textbf{εκθέτης}.
\item Ιδιότητες δυνάμεων : 
\begin{multicols}{2}
\begin{alist}
\item $a^{\nu}\cdot a^{\mu}=a^{\nu+\mu}$
\item $a^{\nu}:a^{\mu}=a^{\nu-\mu}$
\item $(a\cdot \beta)^{\nu}=a^{\nu}\cdot \beta^{\nu}$
\item $\left(\dfrac{a}{\beta}\right)^{\nu}=\dfrac{a^{\nu}}{\beta^{\nu}}$
\item $(a^{\nu})^{\mu}=a^{\nu\cdot\mu}$
\end{alist}
\end{multicols}
\item \textbf{Ταυτότητα:} Μια ισότητα που περιέχει μεταβλητές και επαληθεύεται για κάθε τιμή των μεταβλητών.
\begin{multicols}{2}
\begin{enumerate}[itemsep=0mm,label=\bf\arabic*.]
\item \parbox[t]{7cm}{\textbf{Άθροισμα στο τετράγωνο}\\$ (a+\beta)^2=a^2+2a\beta+\beta^2 $}
\item \parbox[t]{7cm}{\textbf{Διαφορά στο τετράγωνο}\\$ (a-\beta)^2=a^2-2a\beta+\beta^2 $}
\item \parbox[t]{7cm}{\textbf{Άθροισμα στον κύβο}\\$ (a+\beta)^3=a^3+3a^2\beta+3a\beta^2+\beta^3 $}
\item \parbox[t]{7cm}{\textbf{Διαφορά στον κύβο}\\$ (a-\beta)^3=a^3-3a^2\beta+3a\beta^2-\beta^3 $}
\item \parbox[t]{7cm}{\textbf{Γινόμενο αθροίσματος επί διαφορά}\\$ (a+\beta)(a-\beta)=a^2-\beta^2 $}
\item \parbox[t]{7cm}{\textbf{Άθροισμα κύβων}\\$ (a+\beta)\left(a^2-a\beta+\beta^2 \right)=a^3+\beta^3 $}
\item \parbox[t]{7cm}{\textbf{Διαφορά κύβων}\\$ (a-\beta)\left(a^2+a\beta+\beta^2 \right)=a^3-\beta^3 $}
\end{enumerate}
\end{multicols}
\end{enumerate}
\subsection{Διάταξη}
\subsection{Απόλυτη τιμή}
\begin{enumerate}
\item Απόλυτη τιμή πραγματικού αριθμού $a$ : $|a|$.
\end{enumerate}
\subsection{Ρίζες}
\begin{enumerate}[resume]
\item Τετραγωνική ρίζα : $\sqrt{x}$ με $x\geq 0$ και $a\geq 0$
\item Ν-οστή ρίζα : $\sqrt[\nu]{x}$ με $x\geq 0,\nu\in\mathbb{N}^*$ και $a\geq 0$
\begin{itemize}[itemsep=0mm]
\item Ο αριθμός $ x $ ονομάζεται \textbf{υπόριζο}.
\item Δεν ορίζεται ρίζα αρνητικού αριθμού.
\end{itemize}
\end{enumerate}
\section{Εξισώσεις}
\subsection{Εξισώσεις 1ου βαθμού}
\begin{enumerate}
\item Εξίσωση 1ου βαθμού : $ax+\beta$
\end{enumerate}
\subsection{Εξισώσεις της μορφής \bmath{$x^{\nu}=a$}}
\subsection{Εξισώσεις 2ου βαθμού}
\begin{enumerate}
\item Εξίσωση 2ου βαθμού : $ax^2+\beta x+\gamma=0$ με $a\neq 0$
\item Τύποι \eng{Vieta} : $S=x_1+x_2=\dfrac{\beta}{a}$ και $P=x_1\cdot x_2=\dfrac{\gamma}{a}$ όπου $x_1,x_2$ οι ρίζες της εξίσωσης.
\end{enumerate}
\section{Ανισώσεις}
\subsection{Ανισώσεις 1ου βαθμού}
\begin{enumerate}
\item Ανίσωση 1ου βαθμού : $ax+\beta\gtrless0\ $.
\end{enumerate}
\subsection{Ανισώσεις 2ου βαθμού}
\begin{enumerate}
\item Ανίσωση 2ου βαθμού : $ax^2+\beta x+\gamma\gtrless0$ με $a\neq 0$.
\end{enumerate}
\section{Ακολουθίες}
\subsection{Η έννοια της ακολουθίας}
\subsection{Αριθμητική πρόοδος}
\begin{enumerate}
\item Αριθμητική πρόοδος : $a_{\nu+1}=a_{\nu}+\omega\ ,\ \nu\in\mathbb{N}^*$.
\item Γενικός τύπος αριθμητικής προόδου : $a_{\nu}=a_1+(\nu-1)\cdot\omega$
\item Διαφορά αριθμητικής προόδου : $\omega$
\end{enumerate}
\subsection{Γεωμετρική πρόοδος}
\begin{enumerate}[resume]
\item Γεωμετρική πρόοδος : $a_{\nu+1}=\lambda\cdot a_{\nu}\ ,\ \nu\in\mathbb{N}^*$.
\item Γενικός τύπος γεωμετρικής προόδου : $a_{\nu}=a_1\cdot\lambda^{\nu+1}$
\item Λόγος γεωμετρικής προόδου : $\lambda\neq 0$.
\end{enumerate}
\section{Συναρτήσεις}
\subsection{Η έννοια της συνάρτησης}
\subsection{Γραφική παράσταση}
\subsection{Η συνάρτηση \bmath{$f(x)=ax+\beta$}}
\end{document}


