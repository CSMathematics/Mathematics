\PassOptionsToPackage{no-math,cm-default}{fontspec}
\documentclass[twoside,nofonts,internet,shmeiwseis]{thewria}
\usepackage{amsmath}
\usepackage{xgreek}
\let\hbar\relax
\defaultfontfeatures{Mapping=tex-text,Scale=MatchLowercase}
\setmainfont[Mapping=tex-text,Numbers=Lining,Scale=1.0,BoldFont={Minion Pro Bold}]{Minion Pro}
\newfontfamily\scfont{GFS Artemisia}
\font\icon = "Webdings"
\usepackage[amsbb]{mtpro2}
\usepackage{tikz,pgfplots}
\tkzSetUpPoint[size=7,fill=white]
\xroma{red!70!black}
%------- ΣΥΣΤΗΜΑ -------------------
\usepackage{systeme,regexpatch}
\makeatletter
% change the definition of \sysdelim not to store `\left` and `\right`
\def\sysdelim#1#2{\def\SYS@delim@left{#1}\def\SYS@delim@right{#2}}
\sysdelim\{. % reinitialize

% patch the internal command to use
% \LEFTRIGHT<left delim><right delim>{<system>}
% instead of \left<left delim<system>\right<right delim>
\regexpatchcmd\SYS@systeme@iii
{\cB.\c{SYS@delim@left}(.*)\c{SYS@delim@right}\cE.}
{\c{SYS@MT@LEFTRIGHT}\cB\{\1\cE\}}
{}{}
\def\SYS@MT@LEFTRIGHT{%
\expandafter\expandafter\expandafter\LEFTRIGHT
\expandafter\SYS@delim@left\SYS@delim@right}
\makeatother
\newcommand{\synt}[2]{{\scriptsize \begin{matrix}
\times#1\\\\ \times#2
\end{matrix}}}
%----------------------------------------
%------ ΜΗΚΟΣ ΓΡΑΜΜΗΣ ΚΛΑΣΜΑΤΟΣ ---------
\DeclareRobustCommand{\frac}[3][0pt]{%
{\begingroup\hspace{#1}#2\hspace{#1}\endgroup\over\hspace{#1}#3\hspace{#1}}}
%----------------------------------------

\newlist{rlist}{enumerate}{3}
\setlist[rlist]{itemsep=0mm,label=\roman*.}
\newlist{brlist}{enumerate}{3}
\setlist[brlist]{itemsep=0mm,label=\bf\roman*.}
\newlist{tropos}{enumerate}{3}
\setlist[tropos]{label=\bf\textit{\arabic*\textsuperscript{oς}\;Τρόπος :},leftmargin=0cm,itemindent=2.3cm,ref=\bf{\arabic*\textsuperscript{oς}\;Τρόπος}}
\newcommand{\tss}[1]{\textsuperscript{#1}}
\newcommand{\tssL}[1]{\MakeLowercase{\textsuperscript{#1}}}

\usepackage{hhline}
%----------- ΓΡΑΦΙΚΕΣ ΠΑΡΑΣΤΑΣΕΙΣ ---------
\pgfkeys{/pgfplots/aks_on/.style={axis lines=center,
xlabel style={at={(current axis.right of origin)},xshift=1.5ex, anchor=center},
ylabel style={at={(current axis.above origin)},yshift=1.5ex, anchor=center}}}
\pgfkeys{/pgfplots/grafikh parastash/.style={\xrwma,line width=.4mm,samples=200}}
\pgfkeys{/pgfplots/belh ar/.style={tick label style={font=\scriptsize},axis line style={-latex}}}
%-----------------------------------------
\usepackage{multicol}
\usepackage{wrapfig}
\usepackage{venndiagram}
%------ ΕΙΚΟΝΑ ΓΥΡΩ ΑΠΟ ΚΕΙΜΕΝΟ ------------
\newenvironment{WrapText1}[3][r]
{\wrapfigure[#2]{#1}{#3}}
{\endwrapfigure}

\newenvironment{WrapText2}[3][l]
{\wrapfigure[#2]{#1}{#3}}
{\endwrapfigure}

\newcommand{\wrapr}[6]{
\begin{minipage}{\linewidth}\mbox{}\\
\vspace{#1}
\begin{WrapText1}{#2}{#3}
\vspace{#4}#5\end{WrapText1}#6
\end{minipage}}

\newcommand{\wrapl}[6]{
\begin{minipage}{\linewidth}\mbox{}\\
\vspace{#1}
\begin{WrapText2}{#2}{#3}
\vspace{#4}#5\end{WrapText2}#6
\end{minipage}}
%-------------------------------------------


\begin{document}
\titlos{ΑΛΓΕΒΡΑ Α΄ ΛΥΚΕΙΟΥ}{Σύνολα - Πιθανότητες}{Η έννοια του συνόλου}
\orismoi
\Orismos{Σύνολο} Σύνολο ονομάζεται μια συλλογή όμοιων αντικειμένων, τα οποία είναι καλά ορισμένα και διακριτά μεταξύ τους.
\begin{itemize}[itemsep=0mm]
\item Τα αντικείμενα ενός συνόλου ονομάζονται \textbf{στοιχεία}.
\item Τα σύνολα τα συμβολίζουμε με ένα κεφαλαίο γράμμα.
\end{itemize}
\textbf{ΒΑΣΙΚΑ ΣΥΝΟΛΑ ΑΡΙΘΜΩΝ}
\begin{enumerate}[itemsep=0mm,label=\bf\arabic*.]
\item \textbf{Φυσικοί Αριθμοί} : Το σύνολο των αριθμών από το 0 έως το άπειρο όπου κάθε αριθμός έχει διαφορά μιας μονάδας από τον προηγούμενο. Συμβολίζεται με $ \mathbb{N} $ και είναι : $ \mathbb{N}=\{0,1,2,\ldots\} $.
\item \textbf{Ακέραιοι Αριθμοί} : Το σύνολο των φυσικών αριθμών μαζί με τους αντίθετους τους. Συμβολίζεται με $ \mathbb{Z} $ και είναι : $ \mathbb{Z}=\{\ldots,-2,-1,0,1,2,\ldots\} $.
\item \textbf{Ρητοί Αριθμοί} : Όλοι οι αριθμοί που μπορούν να γραφτούν με τη μορφή κλάσματος με ακέραιους όρους. Συμβολίζεται με $ \mathbb{Q} $ και είναι : $ \mathbb{Q}=\LEFTRIGHT\{\}{\left. \frac{a}{\beta}\right|a,\beta\in\mathbb{Z},\beta\neq0\;} $.
\item \textbf{Άρρητοι Αριθμοί} : Κάθε αριθμός ο οποίος δεν είναι ρητός. Κατά κύριο λόγο, άρρητοι αριθμοί είναι οι ρίζες που δεν έχουν ρητό αποτέλεσμα, ο αριθμός $ \pi $ κ.τ.λ.
\item \textbf{Πραγματικοί Αριθμοί} : Οι ρητοί μαζί με το σύνολο των άρρητων μας δίνουν τους πραγματικούς αριθμούς, όλους τους αριθμούς που γνωρίζουμε. Συμβολίζεται με $ \mathbb{R} $ και είναι : $ \mathbb{R}= $ $ \{ $όλοι οι αριθμοί$ \} $.
\end{enumerate}
Τα παραπάνω σύνολα χωρίς το μηδενικό τους στοιχείο συμβολίζονται αντίστοιχα με $ \mathbb{N}^*,\mathbb{Z}^*,\mathbb{Q}^*,\mathbb{R}^*$.\\\\
\Orismos{Ίσα σύνολα} Ίσα ονομάζονται δύο σύνολα $ A,B $ τα οποία έχουν ακριβώς τα ίδια στοιχεία. Ισοδύναμα, τα σύνολα $ Α,Β $ λέγονται ίσα εαν ισχύουν οι σχέσεις :
\begin{enumerate}[itemsep=0mm]
\item Κάθε στοιχείο του $ A $ είναι και στοιχείο του $ B $
\item Κάθε στοιχείο του $ B $ είναι και στοιχείο του $ A $.
\end{enumerate}
\Orismos{Υποσύνολο} Ένα σύνολο $ A $ λέγεται υποσύνολο ενός συνόλου $ B $ όταν κάθε στοιχείο του $ A $ είναι και στοιχείο του $ B $. Συμβολίζεται με τη χρήση του συμβόλου $ \subseteq $ ως εξής :
\[ A\subseteq B \]
\Orismos{Κενό σύνολο} Κενό ονομάζεται το σύνολο που δεν έχει κανένα στοιχείο. Συμβολίζεται με $ \varnothing $ ή $ \left\lbrace \right\rbrace  $.\\\\
\Orismos{Βασικό σύνολο} Βασικό ονομάζεται το σύνολο το οποίο περιέχει όλα τα στοιχεία που μπορούμε να επιλέξουμε, από τα οποία φτιάχνουμε άλλα σύνολα. Συμβολίζεται με $ \varOmega $.\\\\
\Orismos{Πράξεισ μεταξύ συνόλων}
\vspace{-4mm}
\begin{enumerate}[label=\bf\arabic*.,itemsep=0mm]
\item \textbf{Ένωση}\\
\begin{minipage}{\linewidth}
\begin{WrapText1}{8}{3.5cm}
\vspace{-5mm}
\begin{venndiagram2sets}[tikzoptions={scale=.7,samples=100},shade=\xrwma!50,labelNotAB={$ \varOmega $}]
\fillA \fillB
\end{venndiagram2sets}
\end{WrapText1}
Ένωση δύο υποσυνόλων $ A,B $ ενός βασικού συνόλου $ \varOmega $ ονομάζεται το σύνολο των στοιχείων του $ \varOmega $ τα οποία ανήκουν σε τουλάχιστον ένα από τα σύνολα $ A $ και $ B $. Συμβολίζεται με $ A\cup B $.  \[ A\cup B=\left\lbrace x\in\varOmega\left| x\in A \textrm{ ή } x\in B\right.\right\rbrace \]
Η ένωση των συνόλων $ A $ και $ B $ περιέχει τα κοινά και μή κοινά στοιχεία των δύο συνόλων. Τα κοινά στοιχεία αναγράφονται μια φορά.\end{minipage}
\item \textbf{Τομή}\\
\begin{minipage}{\linewidth}
\begin{WrapText1}{7}{3.5cm}
\vspace{-5mm}
\begin{venndiagram2sets}[tikzoptions={scale=.7},shade=\xrwma!50,labelNotAB={$ \varOmega $}]
\fillACapB
\end{venndiagram2sets}
\end{WrapText1}
Τομή δύο υποσυνόλων $ A,B $ ενός βασικού συνόλου $ \varOmega $ ονομάζεται το σύνολο των στοιχείων του $ \varOmega $ τα οποία ανήκουν και στα δύο σύνολα $ A $ και $ B $. Συμβολίζεται με $ A\cap B $. \[ A\cap B=\left\lbrace x\in\varOmega\left| x\in A \textrm{ και } x\in B\right.\right\rbrace \]
Η τομή των συνόλων $ A $ και $ B $ περιέχει μόνο τα κοινά στοιχεία των δύο συνόλων.\end{minipage}
\item \textbf{Συμπλήρωμα}\\
\begin{minipage}{\linewidth}
\begin{WrapText1}{8}{3.5cm}
\vspace{-5mm}
\begin{tikzpicture}[scale=.77]
\filldraw[fill=\xrwma!50] (-2,-2) rectangle (2.6,1);
\scope % A \cap B
\fill[white] (-.45,-.5) circle (1.1);
\draw[black] (-.45,-.5) circle (1.1);
\endscope
\tkzText(-1.6,-1.6){$ \varOmega $}
\tkzText(-.45,.3){$ A $}
\end{tikzpicture}
\end{WrapText1}
Συμπλήρωμα ενός συνόλου $ A $ ονομάζεται το σύνολο των στοιχείων του βασικού συνόλου $ \varOmega $ τα οποία \textbf{δεν} ανήκουν στο σύνολο $ A $. Συμβολίζεται με $ A' $. \[ A'=\left\lbrace x\in\varOmega\left| x\notin A\right.\right\rbrace \] Ονομάζεται συμπλήρωμα του $ Α $ γιατί η ένωσή του με το σύνολο αυτό μας δίνει το βασικό σύνολο $ \varOmega $.\end{minipage}
\item \textbf{Διαφορά}\\
\begin{minipage}{\linewidth}
\begin{WrapText1}{8}{3.5cm}
\vspace{-5mm}
\begin{venndiagram2sets}[tikzoptions={scale=.7},shade=\xrwma!50,labelNotAB={$ \varOmega $}]
\fillANotB
\end{venndiagram2sets}
\end{WrapText1}
Διαφορά ενός συνόλου $ B $ από ένα σύνολο $ A $ ονομάζεται το σύνολο των στοιχείων του βασικού συνόλου $ \varOmega $ τα οποία ανήκουν μόνο στο σύνολο $ A $, το πρώτο σύνολο της διαφοράς. Συμβολίζεται με $ A-B $. \[ A-B=\left\lbrace x\in\varOmega\left| x\in A\textrm{ και }x\notin B\right. \right\rbrace  \]
\end{minipage}
\end{enumerate}\mbox{}\\\\
\thewrhmata
\Thewrhma{Ιδιότητεσ Υποσυνολου}
Για οποιαδήποτε σύνολα $ A,B,\varGamma $ ισχύουν οι ακόλουθες ιδιότητες που αφορούν τη σχέση του υποσυνόλου :
\begin{rlist}
\item Για κάθε σύνολο $ A $ ισχύει : $ A\subseteq A $.
\item Αν $ A\subseteq B $ και $ B\subseteq \varGamma $ τότε $ A\subseteq \varGamma $.
\item Αν $ A\subseteq B $ και $ B\subseteq A $ τότε $ A=B $.
\end{rlist}
\end{document}
