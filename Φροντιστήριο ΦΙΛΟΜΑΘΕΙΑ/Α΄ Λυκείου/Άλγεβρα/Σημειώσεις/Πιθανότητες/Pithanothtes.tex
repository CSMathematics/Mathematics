\PassOptionsToPackage{no-math,cm-default}{fontspec}
\documentclass[twoside,nofonts,internet,shmeiwseis]{thewria}
\usepackage{amsmath}
\usepackage{xgreek}
\let\hbar\relax
\defaultfontfeatures{Mapping=tex-text,Scale=MatchLowercase}
\setmainfont[Mapping=tex-text,Numbers=Lining,Scale=1.0,BoldFont={Minion Pro Bold}]{Minion Pro}
\newfontfamily\scfont{GFS Artemisia}
\font\icon = "Webdings"
\usepackage[amsbb]{mtpro2}
\usepackage{tikz,pgfplots}
\tkzSetUpPoint[size=7,fill=white]
\xroma{blue!50!green}
%------- ΣΥΣΤΗΜΑ -------------------
\usepackage{systeme,regexpatch}
\makeatletter
% change the definition of \sysdelim not to store `\left` and `\right`
\def\sysdelim#1#2{\def\SYS@delim@left{#1}\def\SYS@delim@right{#2}}
\sysdelim\{. % reinitialize

% patch the internal command to use
% \LEFTRIGHT<left delim><right delim>{<system>}
% instead of \left<left delim<system>\right<right delim>
\regexpatchcmd\SYS@systeme@iii
{\cB.\c{SYS@delim@left}(.*)\c{SYS@delim@right}\cE.}
{\c{SYS@MT@LEFTRIGHT}\cB\{\1\cE\}}
{}{}
\def\SYS@MT@LEFTRIGHT{%
\expandafter\expandafter\expandafter\LEFTRIGHT
\expandafter\SYS@delim@left\SYS@delim@right}
\makeatother
\newcommand{\synt}[2]{{\scriptsize \begin{matrix}
\times#1\\\\ \times#2
\end{matrix}}}
%----------------------------------------
%------ ΜΗΚΟΣ ΓΡΑΜΜΗΣ ΚΛΑΣΜΑΤΟΣ ---------
\DeclareRobustCommand{\frac}[3][0pt]{%
{\begingroup\hspace{#1}#2\hspace{#1}\endgroup\over\hspace{#1}#3\hspace{#1}}}
%----------------------------------------

\newlist{rlist}{enumerate}{3}
\setlist[rlist]{itemsep=0mm,label=\roman*.}
\newlist{brlist}{enumerate}{3}
\setlist[brlist]{itemsep=0mm,label=\bf\roman*.}
\newlist{tropos}{enumerate}{3}
\setlist[tropos]{label=\bf\textit{\arabic*\textsuperscript{oς}\;Τρόπος :},leftmargin=0cm,itemindent=2.3cm,ref=\bf{\arabic*\textsuperscript{oς}\;Τρόπος}}
\newcommand{\tss}[1]{\textsuperscript{#1}}
\newcommand{\tssL}[1]{\MakeLowercase{\textsuperscript{#1}}}

\usepackage{hhline}
%----------- ΓΡΑΦΙΚΕΣ ΠΑΡΑΣΤΑΣΕΙΣ ---------
\pgfkeys{/pgfplots/aks_on/.style={axis lines=center,
xlabel style={at={(current axis.right of origin)},xshift=1.5ex, anchor=center},
ylabel style={at={(current axis.above origin)},yshift=1.5ex, anchor=center}}}
\pgfkeys{/pgfplots/grafikh parastash/.style={\xrwma,line width=.4mm,samples=200}}
\pgfkeys{/pgfplots/belh ar/.style={tick label style={font=\scriptsize},axis line style={-latex}}}
%-----------------------------------------
\usepackage{multicol}
\usepackage{wrap-rl}
\usepackage{multirow}

\begin{document}
\titlos{Άλγεβρα Α΄ Λυκειου}{Σύνολα - Πιθανότητες}{Πιθανότητες}
\orismoi
\Orismos{Κλασικόσ Ορισμόσ Πιθανότητασ}
Πιθανότητα ενός ενδεχομένου $ A=\{a_1,a_2,\ldots,a_\kappa\} $ ενός δειγματικού χώρου $ \varOmega $ ονομάζεται ο λόγος του πλήθους των ευνοϊκών περιπτώσεων του $ A $ προς το πλήθος όλων των δυνατών περιπτώσεων.
\[ P(A)=\frac{N(A)}{N(\varOmega)} \]
\begin{itemize}[itemsep=0mm]
\item Ο παραπάνω ορισμός ονομάζεται \textbf{κλασικός ορισμός} της πιθανότητας και εφαρμόζεται όταν το ενδεχόμενο $ A $ αποτελείται από ισοπίθανα απλά ενδεχόμενα $ \{a_i\}\ ,\ i=1,2,\ldots,\kappa $.
\item Το πλήθος των στοιχείων ενός ενδεχομένου $ A $ συμβολίζεται με $ N(A) $.
\end{itemize}
\Orismos{Αξιωματικός Ορισμός Πιθανότητας}
Η πιθανότητα ενός ενδεχομένου $ A=\{a_1,a_2,\ldots,a_\kappa\} $ ενός δειγματικού χώρου $ \varOmega=\{\omega_1,\omega_2,\ldots,\omega_\nu\} $ ορίζεται ώς το άθροισμα των πιθανοτήτων $ P(a_i)\ ,\ i=1,2,\ldots,\nu $ των απλών ενδεχομένων του.
\[ P(A)=P(a_1)+P(a_2)+\ldots+P(a_\kappa) \]
\begin{itemize}[itemsep=0mm]
\item Για κάθε στοιχείο $ \omega_i\ ,\ i=1,2,\ldots,\nu $ του δειγματικού χώρου $ \varOmega $ ονομάζουμε τον αριθμό $ P(\omega_i) $ πιθανότητα του ενδεχομένου $ \{\omega_i\} $.
\item Ο παραπάνω ορισμός ονομάζεται \textbf{αξιοματικός ορισμός} της πιθανότητας και εφαρμόζεται όταν το ενδεχόμενο $ A $ δεν αποτελείται από ισοπίθανα απλά ενδεχόμενα $ \{a_i\}\ ,\ i=1,2,\ldots,\kappa $.
\end{itemize}
\thewrhmata
\Thewrhma{Ιδιότητες Πιθανοτήτων}
Από τον κλασικό ορισμό της πιθανότητας προκύπτουν οι παρακάτω ιδιότητες :
\begin{rlist}
\item Πιθανότητα κενού συνόλου : $ P(\varnothing)=0 $.
\item Πιθανότητα δειγματικού χώρου : $ P(\varOmega)=1 $.
\item Για κάθε ενδεχόμενο $ A $ ισχύει : $ 0\leq P(A)\leq1 $.
\end{rlist}
\Thewrhma{Κανόνες λογισμού πιθανοτήτων}
Οι παρακάτω ιδιότητες μας δείχνουν τις σχέσεις με τις οποίες συνδέονται οι πιθανότητες οποιονδήποτε ενδεχομένων $ A,B $ με τις πιθανότητες των ενδεχομένων των πράξεων που περιέχουν τα ενδεχόμενα αυτά.
\begin{center}
\begin{tabular}{cc}
\hline \rule[-2ex]{0pt}{5.5ex} \textbf{Ενδεχόμενο} & \textbf{Πιθανότητα} \\ 
\hhline{==} \rule[-2ex]{0pt}{7.5ex} Ένωση & $ P(A\cup B)=\ccases{P(A)+P(B)-P(A\cap B)\ \ ,\ \ \textrm{αν }A\cap B\neq\varnothing\\
P(A)+P(B)\ \ ,\ \ \textrm{αν }A\cap B=\varnothing} $ \\ 
 \rule[-2ex]{0pt}{5.5ex} Συμπλήρωμα & $ P(A')=1-P(A) $ \\ 
 \hhline{~-}\rule[-2ex]{0pt}{5.5ex} \multirow{3}{*}{Διαφορά} & $ P(A-B)=P(A)-P(A\cap B) $ \\ 
\rule[-2ex]{0pt}{5.5ex}  & $ P(B-A)=P(B)-P(A\cap B) $ \\ 
   \hhline{~-}\rule[-2ex]{0pt}{5.5ex} Υποσύνολο & $ A\subseteq B\Rightarrow P(A)\leq P(B) $ \\ 
\hline 
\end{tabular} 
\end{center}\mbox{}\\
\Thewrhma{Ανισότητες μεtαξύ πιθανοτήτων}
Μεταξύ των πιθανοτήτων δύο οποιονδήποτε ενδεχομένων $ A,B $ καθώς και των ενδεχομένων που προκύπτουν από πράξεις που τα περιέχουν, ισχύουν οι ακόλουθες ανισότητες.
\begin{multicols}{3}
\begin{rlist}
\item $ P(A)\leq P(A\cup B) $
\item $ P(B)\leq P(A\cup B) $
\item $ P(A\cap B)\leq P(A) $
\item $ P(A\cap B)\leq P(B) $
\item $ P(A\cap B)\leq P(A\cup B) $
\item $ P(A-B)\leq P(A) $
\item $ P(B-A)\leq P(B) $
\item $ P(A-B)\leq P(A\cup B) $
\item $ P(B-A)\leq P(A\cup B) $
\end{rlist}
\end{multicols}
\end{document}
