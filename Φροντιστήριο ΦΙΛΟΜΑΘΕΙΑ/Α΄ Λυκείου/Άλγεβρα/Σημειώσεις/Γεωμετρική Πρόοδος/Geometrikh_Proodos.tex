\PassOptionsToPackage{no-math,cm-default}{fontspec}
\documentclass[twoside,nofonts,internet,shmeiwseis]{thewria}
\usepackage{amsmath}
\usepackage{xgreek}
\let\hbar\relax
\defaultfontfeatures{Mapping=tex-text,Scale=MatchLowercase}
\setmainfont[Mapping=tex-text,Numbers=Lining,Scale=1.0,BoldFont={Minion Pro Bold}]{Minion Pro}
\usepackage[amsbb]{mtpro2}
\usepackage{tikz,pgfplots}
\tkzSetUpPoint[size=7,fill=white]
\xroma{red!70!black}
%------- ΣΥΣΤΗΜΑ -------------------
\usepackage{systeme,regexpatch}
\makeatletter
% change the definition of \sysdelim not to store `\left` and `\right`
\def\sysdelim#1#2{\def\SYS@delim@left{#1}\def\SYS@delim@right{#2}}
\sysdelim\{. % reinitialize

% patch the internal command to use
% \LEFTRIGHT<left delim><right delim>{<system>}
% instead of \left<left delim<system>\right<right delim>
\regexpatchcmd\SYS@systeme@iii
{\cB.\c{SYS@delim@left}(.*)\c{SYS@delim@right}\cE.}
{\c{SYS@MT@LEFTRIGHT}\cB\{\1\cE\}}
{}{}
\def\SYS@MT@LEFTRIGHT{%
\expandafter\expandafter\expandafter\LEFTRIGHT
\expandafter\SYS@delim@left\SYS@delim@right}
\makeatother
\newcommand{\synt}[2]{{\scriptsize \begin{matrix}
\times#1\\\\ \times#2
\end{matrix}}}
%----------------------------------------
%------ ΜΗΚΟΣ ΓΡΑΜΜΗΣ ΚΛΑΣΜΑΤΟΣ ---------
\DeclareRobustCommand{\frac}[3][0pt]{%
{\begingroup\hspace{#1}#2\hspace{#1}\endgroup\over\hspace{#1}#3\hspace{#1}}}
%----------------------------------------

\newlist{rlist}{enumerate}{3}
\setlist[rlist]{itemsep=0mm,label=\roman*.}
\newlist{brlist}{enumerate}{3}
\setlist[brlist]{itemsep=0mm,label=\bf\roman*.}
\newlist{tropos}{enumerate}{3}
\setlist[tropos]{label=\bf\textit{\arabic*\textsuperscript{oς}\;Τρόπος :},leftmargin=0cm,itemindent=2.3cm,ref=\bf{\arabic*\textsuperscript{oς}\;Τρόπος}}
\newcommand{\tss}[1]{\textsuperscript{#1}}
\newcommand{\tssL}[1]{\MakeLowercase{\textsuperscript{#1}}}

\usepackage{hhline}
%----------- ΓΡΑΦΙΚΕΣ ΠΑΡΑΣΤΑΣΕΙΣ ---------
\pgfkeys{/pgfplots/aks_on/.style={axis lines=center,
xlabel style={at={(current axis.right of origin)},xshift=1.5ex, anchor=center},
ylabel style={at={(current axis.above origin)},yshift=1.5ex, anchor=center}}}
\pgfkeys{/pgfplots/grafikh parastash/.style={\xrwma,line width=.4mm,samples=200}}
\pgfkeys{/pgfplots/belh ar/.style={tick label style={font=\scriptsize},axis line style={-latex}}}
%-----------------------------------------
\usepackage{multicol}
\usepackage{wrap-rl}


\begin{document}
\titlos{ΑΛΓΕΒΡΑ Α΄ ΛΥΚΕΙΟΥ}{Ακολουθίες - Πρόοδοι}{Γεωμετρική Πρόοδος}
\orismoi
\Orismos{Γεωμετρική πρόοδοσ}
Γεωμετρική πρόοδος ονομάζεται κάθε ακολουθία $ (a_\nu),\nu\in\mathbb{N}^* $ πραγματικών αριθμών στην οποία κάθε όρος της προκύπτει πολλαπλασιάζοντας κάθε φορά τον προηγούμενο όρο με τον ίδιο σταθερό αριθμό. Θα ισχύει
\[ a_{\nu+1}=\lambda\cdot a_\nu \]
Ο αριθμός $ \lambda=\frac{a_{\nu+1}}{a_\nu} $ ονομάζεται \textbf{λόγος} της γεωμετρικής προόδου.\\\\
\Orismos{Γεωμετρικόσ μέσοσ}
Γεωμετρικός μέσος τριών διαδοχικών όρων $ a,\beta,\gamma $ μιας γεωμετρικής προόδου $ (a_\nu) $ ονομάζεται ο μεσαίος όρος $ \beta $ για τον οποίο ισχύει \[ \beta^2=a\cdot\gamma \]
Πιο γενικά, ο γεωμετρικός μέσος $ \nu $ διαδοχικών όρων $ a_1,a_2,\ldots,a_\nu $ γεωμετρικής προόδου ονομάζεται ο πραγματικός αριθμός $ \mu $ για τον οποίο ισχύει \[ \mu^\nu=a_1\cdot a_2\cdot\ldots\cdot a_\nu \]
\Orismos{Παρεμβολή γεωμετρικών ενδιαμέσων}
Γεωμετρικοί ενδιάμεσοι δύο αριθμών $ a $ και $ \beta $ ονομάζονται $\nu$ σε πλήθος πραγματικοί αριθμοί $ x_1,x_2,\ldots,x_\nu $ όταν αυτοί μπορούν να παρεμβληθούν μεταξύ των $ a $ και $ \beta $ ώστε οι πραγματικοί αριθμοί \[ a,x_1,x_2,\ldots x_\nu,\beta \] να αποτελούν, $ \nu+2 $ σε πλήθος, διαδοχικούς όρους γεωμετρικής προόδου.\\\\
\thewrhmata
\Thewrhma{Γενικόσ όροσ γεωμετρικήσ προόδου}
Εαν $ (a_\nu) $ είναι μια γεωμετρική πρόοδος με λόγο $ \lambda $ τότε ο γενικός όρος της $ a_\nu $ θα δίνεται από τον τύπο \[ a_\nu=a_1\cdot\lambda^{\nu-1} \]
\Thewrhma{Αθροισμα όρων γεωμετρικήσ προόδου}
Εαν $ (a_\nu) $ είναι μια γεωμετρική πρόοδος με λόγο $ \lambda\neq1 $, τότε το άθροισμα των $ \nu $ πρώτων όρων της δίνεται από τους τύπους
\[ S_\nu=\dfrac{a_\nu\cdot\lambda-a_1}{\lambda-1}\;\;,\;\;S_\nu=a_1\frac{\lambda^\nu-1}{\lambda-1} \]
Εαν ο λόγος είναι $ \lambda=1 $ τότε το άθροισμα θα δίνεται από τον τύπο $ S_\nu=\nu a_1 $.\\\\
\Thewrhma{Γεωμετρικόσ μέσοσ}
Τρεις πραγματικοί αριθμοί $ a,\beta,\gamma $ αποτελούν διαδοχικούς όρους γεωμετρικής προόδου αν και μόνο αν ισχύει \[ \beta^2=a\cdot\gamma \]
Εαν οι τρεις όροι $ a,\beta,\gamma $ είναι θετικοί έχουμε ισοδύναμα $ \beta=\!\sqrt{a\cdot\gamma} $.\\
Γενικά έχουμε οτι μια ακολουθία πραγματικών αριθμών $ (a_\nu) $ αποτελεί γεωμετρική πρόοδο αν και μόνο αν γιια κάθε $ \nu\in\mathbb{N}^* $ ισχύει \[ a_\nu^2=a_{\nu+1}\cdot a_{\nu-1} \]
\Thewrhma{Λόγοσ γεωμετρικήσ παρεμβολήσ}
Εαν οι πραγματικοί αριθμοί $ x_1,x_2,\ldots,x_\nu $ είναι γεωμετρικοί ενδιάμεσοι δύο αριθμών $ a,\beta\in\mathbb{R}^*$ τότε για το λόγο της γεωμετρικής προόδου στην οποία ανήκουν ισχύει :
\begin{enumerate}[itemsep=0mm]
\item Αν ο εκθέτης $ \nu+1 $ είναι άρτιος και $ a,\beta $ ομόσημοι τότε $ \lambda=\pm\!\sqrt[\nu+1]{\frac{\beta}{a}} $.
\item Αν ο εκθέτης $ \nu+1 $ είναι περιττός έχουμε
\begin{enumerate}[itemsep=0mm,label=\roman*.]
\item Αν $ a,\beta $ ομόσημοι τότε $ \lambda=\!\sqrt[\nu+1]{\frac{\beta}{a}} $
\item Αν $ a,\beta $ ετερόσημοι τότε $ \lambda=-\!\sqrt[\nu+1]{\left| \frac{\beta}{a}\right| } $
\end{enumerate}
\end{enumerate}
Στην περίπτωση όπου ο εκθέτης $ \nu+1 $ είναι άρτιος και $ a,\beta $ ετερόσημοι τότε δεν ορίζεται λόγος $ \lambda $ και κατά συνέπεια δε σχηματίζεται γεωμετρική πρόοδος.\\\\
\Thewrhma{Παράσταση όρων αριθμητικήσ προόδου}
Εαν $ (a_\nu) $ είναι μια αριθμητική πρόοδος με διαφορά $ \omega $ τότε ισχύουν οι παρακάτω ιδιότητες για τους όρους της :
\begin{enumerate}[itemsep=0mm,label=\roman*.]
\item Εαν $ a_1,a_2,\ldots,a_\nu $ είναι $ \nu $ σε πλήθος διαδοχικοί όροι γεωμετρικής προόδου τότε ο $ \mu -$οστός όρος από το τέλος βρίσκεται στη θέση $ \nu-\mu+1 $ και δίνεται από τον τύπο \[ a_{\nu-\mu+1}=a_\nu\cdot\lambda^{1-\mu} \]
\item Το άθροισμα $ S $ των $ \mu $ τελευταίων όρων μιας γεωμετρικής προόδου $ (a_\nu) $ είναι \[ S=S_\nu-S_{\nu-\mu} \]
\end{enumerate}
\end{document}
