\PassOptionsToPackage{no-math,cm-default}{fontspec}
\documentclass[twoside,nofonts,internet,shmeiwseis]{thewria}
\usepackage{amsmath}
\usepackage{xgreek}
\let\hbar\relax
\defaultfontfeatures{Mapping=tex-text,Scale=MatchLowercase}
\setmainfont[Mapping=tex-text,Numbers=Lining,Scale=1.0,BoldFont={Minion Pro Bold}]{Minion Pro}
\newfontfamily\scfont{GFS Artemisia}
\font\icon = "Webdings"
\usepackage[amsbb]{mtpro2}
\usepackage{tikz,pgfplots}
\tkzSetUpPoint[size=7,fill=white]
\xroma{red!70!black}
%------- ΣΥΣΤΗΜΑ -------------------
\usepackage{systeme,regexpatch}
\makeatletter
% change the definition of \sysdelim not to store `\left` and `\right`
\def\sysdelim#1#2{\def\SYS@delim@left{#1}\def\SYS@delim@right{#2}}
\sysdelim\{. % reinitialize

% patch the internal command to use
% \LEFTRIGHT<left delim><right delim>{<system>}
% instead of \left<left delim<system>\right<right delim>
\regexpatchcmd\SYS@systeme@iii
{\cB.\c{SYS@delim@left}(.*)\c{SYS@delim@right}\cE.}
{\c{SYS@MT@LEFTRIGHT}\cB\{\1\cE\}}
{}{}
\def\SYS@MT@LEFTRIGHT{%
\expandafter\expandafter\expandafter\LEFTRIGHT
\expandafter\SYS@delim@left\SYS@delim@right}
\makeatother
\newcommand{\synt}[2]{{\scriptsize \begin{matrix}
\times#1\\\\ \times#2
\end{matrix}}}
%----------------------------------------
%------ ΜΗΚΟΣ ΓΡΑΜΜΗΣ ΚΛΑΣΜΑΤΟΣ ---------
\DeclareRobustCommand{\frac}[3][0pt]{%
{\begingroup\hspace{#1}#2\hspace{#1}\endgroup\over\hspace{#1}#3\hspace{#1}}}
%----------------------------------------

\newlist{rlist}{enumerate}{3}
\setlist[rlist]{itemsep=0mm,label=\roman*.}
\newlist{brlist}{enumerate}{3}
\setlist[brlist]{itemsep=0mm,label=\bf\roman*.}
\newlist{tropos}{enumerate}{3}
\setlist[tropos]{label=\bf\textit{\arabic*\textsuperscript{oς}\;Τρόπος :},leftmargin=0cm,itemindent=2.3cm,ref=\bf{\arabic*\textsuperscript{oς}\;Τρόπος}}
\newcommand{\tss}[1]{\textsuperscript{#1}}
\newcommand{\tssL}[1]{\MakeLowercase{\textsuperscript{#1}}}
\usepackage{longtable}
\usepackage{hhline}
%----------- ΓΡΑΦΙΚΕΣ ΠΑΡΑΣΤΑΣΕΙΣ ---------
\pgfkeys{/pgfplots/aks_on/.style={axis lines=center,
xlabel style={at={(current axis.right of origin)},xshift=1.5ex, anchor=center},
ylabel style={at={(current axis.above origin)},yshift=1.5ex, anchor=center}}}
\pgfkeys{/pgfplots/grafikh parastash/.style={\xrwma,line width=.4mm,samples=200}}
\pgfkeys{/pgfplots/belh ar/.style={tick label style={font=\scriptsize},axis line style={-latex}}}
%-----------------------------------------
\usepackage{multicol}
\usepackage{wrapfig}
\usepackage{venndiagram}
%------ ΕΙΚΟΝΑ ΓΥΡΩ ΑΠΟ ΚΕΙΜΕΝΟ ------------
\newenvironment{WrapText1}[3][r]
{\wrapfigure[#2]{#1}{#3}}
{\endwrapfigure}

\newenvironment{WrapText2}[3][l]
{\wrapfigure[#2]{#1}{#3}}
{\endwrapfigure}

\newcommand{\wrapr}[6]{
\begin{minipage}{\linewidth}\mbox{}\\
\vspace{#1}
\begin{WrapText1}{#2}{#3}
\vspace{#4}#5\end{WrapText1}#6
\end{minipage}}

\newcommand{\wrapl}[6]{
\begin{minipage}{\linewidth}\mbox{}\\
\vspace{#1}
\begin{WrapText2}{#2}{#3}
\vspace{#4}#5\end{WrapText2}#6
\end{minipage}}
%-------------------------------------------
\usepackage{multicol}


\begin{document}
\titlos{ΑΛΓΕΒΡΑ Α΄ ΛΥΚΕΙΟΥ}{Σύνολα - Πιθανότητες}{Δειγματικός Χώρος - Ενδεχόμενα}
\orismoi
\Orismos{Πείραμα τύχησ} Πείραμα τύχης ονομάζεται κάθε πείραμα του οποίου το αποτέλεσμα δεν μπορεί να προβλευθεί με απόλυτη βεβαιότητα όσες φορές κι αν αυτό επαναληφθεί, κάτω από τις ίδιες συνθήκες.\\\\
\Orismos{Δειγματικόσ Χώροσ} Δειγματικός χώρος ονομάζεται το σύνολο το οποίο περιέχει όλα τα πιθανά αποτελέσματα ενός πειράματος τύχης. Ο δειγματικός αποτελέι βασικό σύνολο. \[ \varOmega=\left\lbrace \omega_1,\omega_2,\ldots,\omega_\nu \right\rbrace \]
\Orismos{Ενδεχόμενο} Ενδεχόμενο ονομάζεται το σύνολο το οποίο περιέχει ένα ή περισσότερα στοιχεία του δειγματικού χώρου ενός πειράματος.
\begin{itemize}[itemsep=0mm]
\item Κάθε ενδεχόμενο είναι υποσύνολο του δειγματικού του χώρου.
\item Συμβολίζεται με κεφαλαίο γράμμα π.χ. : $ A,B,\ldots $
\item Τα ενδεχόμενα που έχουν ένα στοιχείο ονομάζονται \textbf{απλά} ενδεχόμενα, ενώ αν περιέχουν περισσότερα στοιχεία ονομάζονται \textbf{σύνθετα}.
\item Εαν το αποτέλεσμα ενός πειράματος είναι στοιχείο ενός ενδεχομένου τότε το ενδεχόμενο \textbf{πραγματοποιείται}.
\item Τα στοιχεία ενός ενδεχομένου ονομάζονται ευνοϊκές περιπτώσεις.
\item Ο δειγματικός χώρος $ \varOmega $ ονομάζεται \textbf{βέβαιο} ενδεχόμενο, ενώ το κενό σύνολο ονομάζεται \textbf{αδύνατο} ενδεχόμενο.
\item Εαν δύο ενδεχόμενα $ A,B $ δεν έχουν κοινά στοιχεία τότε ονομάζονται \textbf{ασυμβίβαστα} ή ξένα μεταξύ τους δηλαδή : \[ A,B \textrm{ ασυμβίβαστα }\Leftrightarrow A\cap B=\varnothing \]
\end{itemize}\mbox{}\\
\Orismos{πράξεισ με ενδεχόμενα}
Οι πράξεις μεταξύ ενδεχομένων ορίζονται ακριβώς όπως και οι πράξεις μεταξύ συνόλων. Κάθε ορισμός προσαρμόζεται ώστε να περιγράψει την ισχύ του ενδεχομένου σε κάθε περίπτωση.
\begin{enumerate}[label=\bf\arabic*.,itemsep=0mm]
\item \textbf{Ένωση}\\
Ένωση δύο ενδεχομένων $ A,B $ ονομάζεται το ενδεχόμενο το οποίο περιέχει τα κοινά και μη κοινά στοιχεία των δύο ενδεχομένων. Η ένωση πραγματοποιείται όταν πραγματοποιείται τουλάχιστον ένα από τα ενδεχόμενα $ A $ ή $ B $. \[ x\in A\cup B\Leftrightarrow x\in A \textrm{ ή }x\in B \]
\item \textbf{Τομή}\\
Τομή δύο ενδεχομένων $ A,B $ ονομάζεται το ενδεχόμενο το οποίο περιέχει τα κοινά στοιχεία των δύο ενδεχομένων. Η τομή πραγματοποιείται όταν πραγματοποιούνται συγχρόνως και τα δύο ενδεχόμενα $ A $ και $ B $. \[ x\in A\cap B\Leftrightarrow x\in A \textrm{ και }x\in B \]
\item \textbf{Συμπλήρωμα}\\Συμπλήρωμα ενός ενδεχομένου $ A $ ονομάζεται το ενδεχόμενο το οποίο περιέχει τα στοιχεία εκείνα τα οποία \textbf{δεν} ανήκουν στο σύνολο $ A $. Το συμπλήρωμα πραγματοποιείται όταν δεν πραγματοποιείται το $ A $. \[ x\in A'\Leftrightarrow x\notin A\]
\item \textbf{Διαφορά}\\
Διαφορά ενός ενδεχομένου $ A $ από ένα ενδεχόμενο $ B $ ονομάζεται το ενδεχόμενο που περιέχει τα στοιχεία που ανήκουν μόνο στο ενδεχόμενο $ A $. Η διαφορά πραγματοποιείται όταν πραγματοποιείται μόνο το ενδεχόμενο $ A $. \[ x\in A-B\Leftrightarrow x\in A \textrm{ και }x\notin B \]
\end{enumerate}
Στον παρακάτω πίνακα φαίνονται τα ενδεχόμενα, οι πράξεις μεταξύ δύο ενδεχομένων $ A,B $, οι συμβολισμοί τους, λεκτική περιγραφή καθώς και διάγραμμα για κάθε περίπτωση.
\begin{center}
\begin{longtable}{c>{\centering}m{2.5cm}>{\centering}m{5cm} c}
\hline \rule[-2ex]{0pt}{5.5ex} \textbf{Συμβολισμός} & \textbf{Ενδεχόμενο} & \textbf{Περιγραφή} & \textbf{Διάγραμμα} \\ 
\hhline{====} \rule[-2ex]{0pt}{8.5ex} $ x\in A $ & Ενδεχόμενο Α & Το ενδεχόμενο $ Α $ πραγματοποιείται. & \parbox[c]{22mm}{\mbox{}\\\begin{tikzpicture}[scale=.438]
\draw (-2,-2) rectangle (2.6,1);
\scope % A \cap B
\fill[\xrwma!50] (-.45,-.5) circle (1.1);
\draw[black] (-.45,-.5) circle (1.1);
\endscope
\tkzText(-1.6,-1.6){{\scriptsize $ \varOmega $}}
\tkzText(-.45,.1){{\scriptsize $ A $}}
\end{tikzpicture}} \\ 
\rule[-2ex]{0pt}{8.5ex} $ x\in A' $ & Συμπλήρωμα του $ A $ & Το ενδεχόμενο $ A $ \textbf{δεν} πραγματοποιείται. & \parbox[c]{22mm}{\mbox{}\\\begin{tikzpicture}[scale=.438]
\filldraw[fill=\xrwma!50] (-2,-2) rectangle (2.6,1);
\scope % A \cap B
\fill[white] (-.45,-.5) circle (1.1);
\draw[black] (-.45,-.5) circle (1.1);
\endscope
\tkzText(-1.6,-1.6){{\scriptsize $ \varOmega $}}
\tkzText(-.45,.1){{\scriptsize $ A $}}
\end{tikzpicture}} \\ 
\rule[-2ex]{0pt}{8.5ex} $ x\in A\cup B $ & Ένωση του $ A $ με το $ B $ & Πραγματοποιείται ένα \textbf{τουλάχιστον} από τα ενδεχόμενα $ A $ και $ B $. & \parbox[c]{22mm}{\mbox{}\\\begin{venndiagram2sets}[tikzoptions={scale=.4},shade=\xrwma!50,labelA={{\scriptsize $ A $}},labelB={{\scriptsize $ B $}},labelNotAB={{\scriptsize $ \varOmega $}}]
\fillA \fillB
\end{venndiagram2sets}} \\ 
\rule[-2ex]{0pt}{8.5ex} $ x\in A\cap B $ & Τομή του $ A $ με το $ B $ & Πραγματοποιούνται \textbf{συγχρόνως} τα ενδ. $ A $ και $ B $. & \parbox[c]{22mm}{\mbox{}\\\begin{venndiagram2sets}[tikzoptions={scale=.4},shade=\xrwma!50,labelA={{\scriptsize $ A $}},labelB={{\scriptsize $ B $}},labelNotAB={{\scriptsize $ \varOmega $}}]
\fillACapB
\end{venndiagram2sets}} \\ 
\rule[-2ex]{0pt}{8.5ex} $ x\in A-B $ & Διαφορά του $ B $ απ' το $ A $ & Πραγματοποιείται \textbf{μόνο} το ενδεχόμενο $ A $. & \parbox[c]{22mm}{\mbox{}\\\begin{venndiagram2sets}[tikzoptions={scale=.4},shade=\xrwma!50,labelA={{\scriptsize $ A $}},labelB={{\scriptsize $ B $}},labelNotAB={{\scriptsize $ \varOmega $}}]
\fillANotB
\end{venndiagram2sets}} \\ 
\rule[-2ex]{0pt}{8.5ex} $ x\in B-A $ & Διαφορά του $ A $ απ' το $ B $ & Πραγματοποιείται \textbf{μόνο} το ενδεχόμενο $ B $. & \parbox[c]{22mm}{\mbox{}\\\begin{venndiagram2sets}[tikzoptions={scale=.4},shade=\xrwma!50,labelA={{\scriptsize $ A $}},labelB={{\scriptsize $ B $}},labelNotAB={{\scriptsize $ \varOmega $}}]
\fillBNotA
\end{venndiagram2sets}} \\ 
\rule[-2ex]{0pt}{8.5ex} $ x\in\left(A-B\right)\cup\left(B-A\right) $ & Ένωση διαφορών & Πραγματοποιείται \textbf{μόνο} ένα από τα δύο σύνολα (ή μόνο το $ A $ ή μόνο το $ B $). & \parbox[c]{22mm}{\mbox{}\\\begin{venndiagram2sets}[tikzoptions={scale=.4},shade=\xrwma!50,labelA={{\scriptsize $ A $}},labelB={{\scriptsize $ B $}},labelNotAB={{\scriptsize $ \varOmega $}}]
\fillANotB \fillBNotA
\end{venndiagram2sets}} \\ 
\rule[-2ex]{0pt}{8.5ex} \begin{minipage}{2.8cm}
\begin{center}
$ A\subseteq B $\\
$ x\in A\Rightarrow x\in B $
\end{center}
\end{minipage} & $ A $ υποσύνολο του $ Β $ & Η πραγματοποίηση του $ A $ συνεπάγεται πραγμ/ση του $ B $. & \parbox[c]{22mm}{\mbox{}\\\begin{tikzpicture}[scale=.438]
\draw (-2,-2) rectangle (2.6,1);
\scope % A \cap B
\filldraw[fill=\xrwma!50] (-.45,-.5) circle (1.1);
\draw[fill=\xrwma!50] (-.5,-.5) circle (.7);
\endscope
\tkzText(-1.6,-1.6){{\scriptsize $ \varOmega $}}
\tkzText(.9,.1){{\scriptsize $ B $}}
\tkzText(-.45,-.2){{\scriptsize $ A $}}
\end{tikzpicture}} \\ 
\rule[-2ex]{0pt}{8.5ex} $ x\in\left(A\cap B\right)' $ & Συμπλήρωμα τομής & \textbf{Δεν} πραγματοποιούνται \textbf{συγχρονως} τα ενδ. $ A $ και $ B $. & \parbox[c]{22mm}{\mbox{}\\\begin{venndiagram2sets}[tikzoptions={scale=.4},shade=\xrwma!50,labelA={{\scriptsize $ A $}},labelB={{\scriptsize $ B $}},labelNotAB={{\scriptsize $ \varOmega $}}]
\fillNotAorNotB
\end{venndiagram2sets}}\\
\rule[-2ex]{0pt}{8.5ex} $ x\in\left(A\cup B\right)' $ & Συμπλήρωμα ένωσης & Δεν πραγματοποιείται \textbf{κανένα} από τα ενδ. $ A $ και $ B $. & \parbox[c]{22mm}{\mbox{}\\\begin{venndiagram2sets}[tikzoptions={scale=.4},shade=\xrwma!50,labelA={{\scriptsize $ A $}},labelB={{\scriptsize $ B $}},labelNotAB={{\scriptsize $ \varOmega $}}]
\fillNotAorB
\end{venndiagram2sets}}\\
\rule[-2ex]{0pt}{8.5ex} $ x\in\left( A-B\right)'  $ & Συμπλήρωμα διαφοράς & \textbf{Δεν} πραγματοποιείται μόνο το ενδεχόμενο $ A $. & \parbox[c]{22mm}{\mbox{}\\\begin{venndiagram2sets}[tikzoptions={scale=.4},shade=\xrwma!50,labelA={{\scriptsize $ A $}},labelB={{\scriptsize $ B $}},labelNotAB={{\scriptsize $ \varOmega $}}]
\fillNotAorB \fillB
\end{venndiagram2sets}} \\
\rule[-2ex]{0pt}{8.5ex} $ x\in \left(B-A\right)'  $ & Συμπλήρωμα διαφοράς & \textbf{Δεν} πραγματοποιείται μόνο το ενδεχόμενο $ B $. & \parbox[c]{22mm}{\mbox{}\\\begin{venndiagram2sets}[tikzoptions={scale=.4},shade=\xrwma!50,labelA={{\scriptsize $ A $}},labelB={{\scriptsize $ B $}},labelNotAB={{\scriptsize $ \varOmega $}}]
\fillNotAorB \fillA
\end{venndiagram2sets}} \\
\rule[-2ex]{0pt}{8.5ex} $ x\in\left( \left(A-B\right)\cup\left(B-A\right)\right)'  $ & Συμπλήρωμα ένωσης διαφορών & \textbf{Δεν} πραγματοποιείται μόνο ένα από τα δύο σύνολα (ή κανένα από τα δύο ή και τα δύο). & \parbox[c]{22mm}{\mbox{}\\\begin{venndiagram2sets}[tikzoptions={scale=.4},shade=\xrwma!50,labelA={{\scriptsize $ A $}},labelB={{\scriptsize $ B $}},labelNotAB={{\scriptsize $ \varOmega $}}]
\fillNotAorB \fillACapB
\end{venndiagram2sets}} \\
\rule[-1ex]{0pt}{0ex} &&&\\
\hline
\end{longtable}
\end{center}
\end{document}
