\documentclass[11pt,a4paper,modern]{FFExercises}
\usepackage[english,greek]{babel}
\usepackage[utf8]{inputenc}
\usepackage{nimbusserif}
\usepackage[T1]{fontenc}
\usepackage{amsmath}
\let\myBbbk\Bbbk
\let\Bbbk\relax
\usepackage[amsbb,subscriptcorrection,zswash,mtpcal,mtphrb,mtpfrak]{mtpro2}
\usepackage{graphicx,multicol,multirow,enumitem,tabularx,mathimatika,gensymb,venndiagram,hhline,longtable,tkz-euclide,fontawesome5,eurosym,tcolorbox,tabularray,tikzpagenodes,relsize,siunitx}
\definecolor{xrwma}{HTML}{aa1212}
\usetikzlibrary{calc}
\usetikzlibrary{positioning}
\tcbuselibrary{skins,theorems,breakable}
\renewcommand{\textstigma}{\textsigma\texttau}
\renewcommand{\textdexiakeraia}{}

\ekthetesdeiktes
\begin{document}

\titlos{Άλγεβρα}{Α' Λυκείου}{Τύποι \eng{Vieta}}
\paragraph{Υπολογισμός $S$ και $P$}
%\askhsh Για καθεμία από τις παρακάτω εξισώσεις να υπολογίσετε το άθροισμα $S$ και το γινόμενο $P$ των λύσεων, εφόσον υπάρχουν.
%\begin{multicols}{2}
%\begin{alist}
%\item $x^2-3x+1=0$
%\item $x^2-4x+4=0$
%\item $2x^2+x-4=0$
%\item $x^2+x+\dfrac{1}{4}=0$
%\item $x^2-2x+3=0$
%\item $9x^2+6x+1=0$
%\item $\dfrac{x^2}{2}-x-3=0$
%\end{alist}
%\end{multicols}
%\askhsh Για καθεμία από τις παρακάτω εξισώσεις να υπολογίσετε το άθροισμα $S$ και το γινόμενο $P$ των λύσεων, εφόσον υπάρχουν.
%\begin{alist}
%\item $x^2-\left(\sqrt{2}+1\right)x+\sqrt{2}=0$
%\item $x^2-3\sqrt{2}x+4=0$
%\item $\sqrt{2}x^2+\sqrt{18}x-\sqrt{8}=0$
%\end{alist}
\paragraph{Κατασκευή εξίσωσης 2ου βαθμού}
%\askhsh Να βρεθεί η εξίσωση 2ου βαθμού η οποία έχει λύσεις τους αριθμούς $x_1,x_2$
%\begin{alist}
%\begin{multicols}{2}
%\item $x_1=3$ και $x_2=4$
%\item $x_1=-2$ και $x_2=4$
%\item $x_1=1$ και $x_2=-5$
%\item $x_1=-4$ και $x_2=-1$
%\item $x_1=\sqrt{8}$ και $x_2=\sqrt{2}$
%\item $x=4$ διπλή λύση
%\end{multicols}
%\item $x_1=1-\sqrt{3}$ και $x_2=1+\sqrt{3}$
%\item $x_1=\dfrac{1}{2}$ και $x_2=\dfrac{3}{4}$
%\item $x_1=\dfrac{3+\sqrt{2}}{4}$ και $x_2=\dfrac{3-\sqrt{2}}{4}$
%\end{alist}
%\askhsh Να βρεθούν, εάν υπάρχουν, αριθμοί $x_1,x_2$ οι οποίοι έχουν
%\begin{alist}
%\item άθροισμα $4$ και γινόμενο $-5$
%\item άθροισμα $-3$ και γινόμενο $-10$
%\item άθροισμα $7$ και γινόμενο $6$
%\item άθροισμα $4$ και γινόμενο $4$
%\item άθροισμα $1$ και γινόμενο $3$
%\item άθροισμα $-2$ και γινόμενο $-8$
%\item άθροισμα $\dfrac{3}{2}$ και γινόμενο $\dfrac{1}{2}$
%\item άθροισμα $3$ και γινόμενο $-5$
%\end{alist}
%\askhsh Δίνεται ορθογώνιο παραλληλόγραμμο με περίμετρο $24\si{cm}$ και εμβαδόν $32\si{cm}^2$. Να υπολογίσετε τις διαστάσεις του ορθογωνίου.
\paragraph{Παραστάσεις των $x_1,x_2$}
%\askhsh Δίνεται η εξίσωση $x^2-5x+3=0$. Αν $x_1,x_2$ είναι οι λύσεις της, τότε να βρείτε τις τιμές των παραστάσεων.
%\begin{multicols}{2}
%\begin{alist}
%\item $x_1+x_2$
%\item $x_1\cdot x_2$
%\item $x_1^2+x_2^2$
%\item $x_1^3+x_2^3$
%\item $\dfrac{1}{x_1}+\dfrac{1}{x_2}$
%\end{alist}
%\end{multicols}
%\askhsh Δίνεται η εξίσωση $x^2+2x-5=0$. Αν $x_1,x_2$ είναι οι λύσεις της, τότε να βρείτε τις τιμές των παραστάσεων.
%\begin{multicols}{2}
%\begin{alist}
%\item $x_1+x_2$
%\item $x_1\cdot x_2$
%\item $x_1^2x_2+x_2x_2^2$
%\item $ x_1^2+2x_1x_2+x_2^2 $
%\item $\dfrac{x_2}{x_1}+\dfrac{x_1}{x_2}$
%\end{alist}
%\end{multicols}
\paragraph{Παραμετρικές εξισώσεις - Είδη ριζών}
\askhsh
\end{document}
