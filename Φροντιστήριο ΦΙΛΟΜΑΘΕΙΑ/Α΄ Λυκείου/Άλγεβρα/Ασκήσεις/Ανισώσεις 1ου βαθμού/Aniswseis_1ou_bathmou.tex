\PassOptionsToPackage{no-math,cm-default}{fontspec}
\documentclass[twoside,nofonts,internet]{askhseis}
\usepackage{amsmath}
\usepackage{xgreek}
\let\hbar\relax
\defaultfontfeatures{Mapping=tex-text,Scale=MatchLowercase}
\setmainfont[Mapping=tex-text,Numbers=Lining,Scale=1.0,BoldFont={Minion Pro Bold}]{Minion Pro}
\newfontfamily\scfont{GFS Artemisia}
\font\icon = "Webdings"
\usepackage[amsbb,subscriptcorrection,zswash,mtpcal,mtphrb]{mtpro2}
\xroma{red!70!black}
%------TIKZ - ΣΧΗΜΑΤΑ - ΓΡΑΦΙΚΕΣ ΠΑΡΑΣΤΑΣΕΙΣ ----
\usepackage{tikz}
\usepackage{tkz-euclide}
\usetkzobj{all}
\usepackage[framemethod=TikZ]{mdframed}
\usetikzlibrary{decorations.pathreplacing}
\usepackage{pgfplots}
\usetkzobj{all}
%-----------------------
\usepackage{calc}
\usepackage{hhline}
\usepackage[explicit]{titlesec}
\usepackage{graphicx}
\usepackage{multicol}
\usepackage{multirow}
\usepackage{enumitem}
\usepackage{tabularx}
\usepackage[decimalsymbol=comma]{siunitx}
\usetikzlibrary{backgrounds}
\usepackage{sectsty}
\sectionfont{\centering}
\usepackage{enumitem}
\setlist[enumerate]{label=\bf{\large \arabic*.}}
\usepackage{adjustbox}
\usepackage{mathimatika,gensymb,eurosym,wrap-rl}
\usepackage{systeme,regexpatch}
%-------- ΜΑΘΗΜΑΤΙΚΑ ΕΡΓΑΛΕΙΑ ---------
\usepackage{mathtools}
%----------------------
%-------- ΠΙΝΑΚΕΣ ---------
\usepackage{booktabs}
%----------------------
%----- ΥΠΟΛΟΓΙΣΤΗΣ ----------
\usepackage{calculator}
%----------------------------
%------ ΔΙΑΓΩΝΙΟ ΣΕ ΠΙΝΑΚΑ -------
\usepackage{array}
\newcommand\diag[5]{%
\multicolumn{1}{|m{#2}|}{\hskip-\tabcolsep
$\vcenter{\begin{tikzpicture}[baseline=0,anchor=south west,outer sep=0]
\path[use as bounding box] (0,0) rectangle (#2+2\tabcolsep,\baselineskip);
\node[minimum width={#2+2\tabcolsep-\pgflinewidth},
minimum  height=\baselineskip+#3-\pgflinewidth] (box) {};
\draw[line cap=round] (box.north west) -- (box.south east);
\node[anchor=south west,align=left,inner sep=#1] at (box.south west) {#4};
\node[anchor=north east,align=right,inner sep=#1] at (box.north east) {#5};
\end{tikzpicture}}\rule{0pt}{.71\baselineskip+#3-\pgflinewidth}$\hskip-\tabcolsep}}
%---------------------------------
%---- ΟΡΙΖΟΝΤΙΟ - ΚΑΤΑΚΟΡΥΦΟ - ΠΛΑΓΙΟ ΑΓΚΙΣΤΡΟ ------
\newcommand{\orag}[3]{\node at (#1)
{$ \overcbrace{\rule{#2mm}{0mm}}^{{\scriptsize #3}} $};}
\newcommand{\kag}[3]{\node at (#1)
{$ \undercbrace{\rule{#2mm}{0mm}}_{{\scriptsize #3}} $};}
\newcommand{\Pag}[4]{\node[rotate=#1] at (#2)
{$ \overcbrace{\rule{#3mm}{0mm}}^{{\rotatebox{-#1}{\scriptsize$#4$}}}$};}
%-----------------------------------------


%------------------------------------------
\newcommand{\tss}[1]{\textsuperscript{#1}}
\newcommand{\tssL}[1]{\MakeLowercase{\textsuperscript{#1}}}
%---------- ΛΙΣΤΕΣ ----------------------
\newlist{bhma}{enumerate}{3}
\setlist[bhma]{label=\bf\textit{\arabic*\textsuperscript{o}\;Βήμα :},leftmargin=0cm,itemindent=1.8cm,ref=\bf{\arabic*\textsuperscript{o}\;Βήμα}}
\newlist{brlist}{enumerate}{3}
\setlist[brlist]{itemsep=0mm,label=\bf\roman*.}
\newlist{tropos}{enumerate}{3}
\setlist[tropos]{label=\bf\textit{\arabic*\textsuperscript{oς}\;Τρόπος :},leftmargin=0cm,itemindent=2.3cm,ref=\bf{\arabic*\textsuperscript{oς}\;Τρόπος}}
% Αν μπει το bhma μεσα σε tropo τότε
%\begin{bhma}[leftmargin=.7cm]
\tkzSetUpPoint[size=7,fill=white]
\tikzstyle{pl}=[line width=0.3mm]
\tikzstyle{plm}=[line width=0.4mm]
%---------------------------------
\makeatletter
\renewcommand*{\@alph}[1]{%
  \ifcase#1\or α\or β\or γ\or
    δ\or ε\or ζ\or η\or θ\or ι\or κ\or
    λ\or μ\or ν\or ξ\or ο\or π\or ρ\or σ\or
    τ\or υ\or φ\or χ\or ψ\or
    ω\else\@ctrerr\fi
}
\renewcommand*{\@Alph}[1]{%
  \ifcase#1\or Α\or Β\or Γ\or
    Δ\or Ε\or Ζ\or Η\or Θ\or Ι\or Κ\or
    Λ\or Μ\or Ν\or Ξ\or Ο\or Π\or Ρ\or Σ\or
    Τ\or Υ\or Φ\or Χ\or Ψ\or
    Ω\else\@ctrerr\fi
}
\makeatother
%--------------------------------



\begin{document}
\titlos{Άλγεβρα Α΄ Λυκείου}{Ανισώσεις}{Ανισώσεις 1\tssL{ου} Βαθμού}
\thewria
\begin{enumerate}
\item 
\end{enumerate}
\askhseis
\begin{enumerate}[label=\bf\textcolor{black}{{\large \arabic*.}},
itemsep=5mm]
\item
\begin{enumerate}[label=\roman*.]
\item Τι ονομάζουμε ανίσωση 1\textsuperscript{ου} βαθμού;
\item Πότε μια ανίσωση ονομάζεται αδύνατη;
\item Πότε μια ανίσωση ονομάζεται αόριστη;
\end{enumerate}
\item \textbf{Να χαρακτηριστούν οι παρακάτω εξισώσεις ως σωστές (Σ) ή λανθασμένες (Λ).}
\begin{enumerate}[label=\roman*.]
\item Η ανίσωση $ ax+\beta>0 $ με $ a>0 $ έχει μια λύση την $ x>-\dfrac{\beta}{a} $.
\item Η ανίσωση $ ax+\beta>0 $ με $ a=0 $ και $ \beta>0 $ είναι αόριστη.
\begin{multicols}{2}
\item Η αν. $ 0x<\beta $ με $ \beta>0 $ είναι αδύνατη.
\item Η αν. $ 0x<\beta $ με $ \beta<0 $ είναι αδύνατη.
\item Η αν. $ 0x>\beta $ με $ \beta=0 $ είναι αόριστη.
\item Η αν. $ 0x\geq\beta $ με $ \beta=0 $ είναι αόριστη.
\item Η αν. $ 0x>\beta $ με $ \beta>0 $ είναι αδύνατη.
\item Η αν. $ 0x\leq\beta $ με $ \beta=0 $ είναι αδύνατη.
\item Η αν. $ 0x<-\beta $ με $ \beta>0 $ είναι αόριστη.
\item Η αν. $ 0x\geq-\beta $ με $ \beta=0 $ είναι αόριστη.
\end{multicols}
\end{enumerate}
\item \textbf{Να λυθούν οι ανισώσεις και να παρασταθούν γραφικά οι λύσεις.}
\begin{multicols}{3}
\begin{enumerate}[label=\roman*.]
\item $ 2x-3>7-3x $
\item $ 4x+5<2-x+8 $
\item $ 3x-2\leq4-2x+8 $
\item $ -x-4\geq7-3x+2 $
\item $ 7x-3+x<2x+9+5x $
\item $ -3x+8>4-5x+12 $
\end{enumerate}
\end{multicols}
\item \textbf{Να λυθούν οι ανισώσεις και να παρασταθούν γραφικά οι λύσεις.}
\begin{multicols}{2}
\begin{enumerate}[label=\roman*.]
\item $ 2(x-1)+3>4-x $
\item $ 2x-3(4-x)<9+4x $
\item $ 4(3-x)+2(3x-1)<3x+2-(x-1) $
\item $ 3(2x+3)-5>5(x-4)+12 $
\item $ -2-3(4-3x)+5x\leq3-(7-2x) $
\item $ 2-(3x-4)+x\geq3(2x+3)-12-(x-2) $
\end{enumerate}
\end{multicols}
\item \textbf{Να λυθούν οι ανισώσεις και να παρασταθούν γραφικά οι λύσεις.}
\begin{multicols}{2}
\begin{enumerate}[label=\roman*.]
\item $ \dfrac{x}{2}+\dfrac{x+1}{3}>1 $
\item $ \dfrac{2x-1}{3}-\dfrac{x-2}{4}<\dfrac{1}{6} $
\item $ \dfrac{x}{5}+\dfrac{3x-2}{3}\leq\dfrac{x-1}{15} $
\item $ \dfrac{4x-3}{3}-\dfrac{3-2x}{4}\geq1+\dfrac{5x}{12} $
\vfill
\columnbreak
\vfill
\item $ 2x-\dfrac{3x-2}{5}+\dfrac{x-1}{15}\leq\dfrac{1}{3}-\dfrac{2-3x}{15} $
\item $ \dfrac{-2-x}{4}+\dfrac{4x-5}{8}<3x-1-\dfrac{7-4x}{4} $
\item $ \dfrac{1-\dfrac{x}{2}}{3}>2 $
\item $ \dfrac{\dfrac{x-1}{3}+\dfrac{x-2}{4}}{2}-\dfrac{2x-1}{6}>\dfrac{x}{12} $
\end{enumerate}
\end{multicols}
\item \textbf{Να λυθούν οι ανισώσεις.}
\begin{multicols}{2}
\begin{enumerate}[label=\roman*.]
\item $ 3x-2<x+4+2x $
\item $ 4x-3+x\geq2x-3+2x $
\item $ 2(x-3)+1<-3x+5(x-2) $
\item $ 4x-(3+2x)>5(x-2)+3(2-x)+1 $
\item $ 5-(x-2)+3x\leq3(2+x)-x-1 $
\item $ \dfrac{2x-3}{4}-\dfrac{x}{2}>1 $
\item $ \dfrac{3x-4}{5}-\dfrac{x-3}{3}\geq\dfrac{x-1}{15}+\dfrac{x-4}{5} $
\end{enumerate}
\end{multicols}
\item \textbf{Να λυθούν οι ανισώσεις και να παρασταθούν γραφικά οι λύσεις και να γραφτούν με τη μορφή διαστήματος.}
\begin{multicols}{2}
\begin{enumerate}[label=\roman*.]
\item $ 4x-3<3x<2-5x $
\item $ 3-2x\leq x+1<4x-5 $
\item $ 3(1-x+2)<4x\leq2(x+2)+3 $
\item $ 5(2x-1)-4\leq 7(3-x)\leq 4(2-x)+3x $
\item $ 3-(3x-4)<2(x-2)+4(3-x)<7-(x-3) $
\item $ \dfrac{x-1}{2}<\dfrac{x}{3}+1\leq\dfrac{2x-1}{2}-\dfrac{1}{3} $
\item $ \dfrac{3x-4}{5}\leq\dfrac{2-x}{3}<\dfrac{x}{15}+1 $
\end{enumerate}
\end{multicols}
\item \textbf{Να βρεθούν οι κοινές λύσεις των ανισώσεων και να γραφτούν με τη μορφή διαστήματος.}
\begin{multicols}{2}
\begin{enumerate}[label=\roman*.]
\item $ 3x-1>5 $ και $ 4x-3<9 $
\item $ 4-3x<2 $ και $ 2x+5\leq7 $
\item $ 2(x-3)+5>x-1 $ και $ 3-(x-4)\leq5-2x $
\vfill
\columnbreak
\vfill
\item $ 4(x-2)+3(5-x)\geq4x-3+2(x-1) $ και $ 5(2-x)+3(x+1)<4-(x-7) $
\item $ \dfrac{x+5}{12}+1\geq\dfrac{x}{4} $ και $ \dfrac{2x+3}{4}+\dfrac{x-1}{3}>1 $
\end{enumerate}
\end{multicols}
\item \textbf{Να λυθούν οι ανισώσεις.}
\begin{multicols}{4}
\begin{enumerate}[label=\roman*.]
\item $ \left|x\right|<4 $
\item $ \left|x\right|>5 $
\item $ \left|x-1\right|<2 $
\item $ \left|x+2\right|>3 $
\item $ \left|2x-1\right|\leq5 $
\item $ \left|3x+4\right|\geq8 $
\item $ \left|1-x\right|<2 $
\item $ \left|3-4x\right|\geq5 $
\end{enumerate}
\end{multicols}
\item \textbf{Να λυθούν οι ανισώσεις.}
\begin{multicols}{4}
\begin{enumerate}[label=\roman*.]
\item $ \left|x\right|<-2 $
\item $ \left|4x\right|>-1 $
\item $ \left|x-3\right|\leq0 $
\item $ \left|2x-4\right|\geq0 $
\end{enumerate}
\end{multicols}
\item \textbf{Να λυθούν οι εξισώσεις.}
\begin{multicols}{4}
\begin{enumerate}[label=\roman*.]
\item $ \left|x-3\right|=x+2 $
\item $ \left|4x-1\right|=2x-5 $
\item $ \left|2x-3\right|=4-7x $
\item $ \left|\dfrac{x}{2}-1\right|=\dfrac{x+3}{4} $
\end{enumerate}
\end{multicols}
\item \textbf{Να βρεθούν οι κοινές λύσεις των ανισώσεων και να γραφτούν με τη μορφή διαστήματος.}
\begin{multicols}{3}
\begin{enumerate}[label=\roman*.]
\item $ 2\leq\left|x\right|\leq3 $
\item $ 3\leq\left|x-1\right|\leq7 $
\item $4\leq\left|2x-4\right|\leq8 $
\end{enumerate}
\end{multicols}
\item \textbf{Να βρεθούν οι κοινές λύσεις των ανισώσεων και να γραφτούν με τη μορφή διαστήματος.}
\begin{multicols}{3}
\begin{enumerate}[label=\roman*.]
\item $ 2\left( \left|x\right|+2\right) $
\end{enumerate}
\end{multicols}
\end{enumerate}
\end{document}

