\documentclass[11pt,a4paper,twocolumn]{article}
\usepackage[english,greek]{babel}
\usepackage[utf8]{inputenc}
\usepackage{nimbusserif}
\usepackage[T1]{fontenc}
\usepackage[left=1.50cm, right=1.50cm, top=2.00cm, bottom=2.00cm]{geometry}
\usepackage{amsmath}
\let\myBbbk\Bbbk
\let\Bbbk\relax
\usepackage[amsbb,subscriptcorrection,zswash,mtpcal,mtphrb,mtpfrak]{mtpro2}
\usepackage{graphicx,multicol,multirow,enumitem,tabularx,mathimatika,gensymb,venndiagram,hhline,longtable,tkz-euclide,fontawesome5,eurosym,tcolorbox,tabularray,tikzpagenodes,relsize}
\definecolor{xrwma}{HTML}{3d93cd}
\usetikzlibrary{calc}
\usetikzlibrary{positioning}
\usepackage[explicit]{titlesec}
\tcbuselibrary{skins,theorems,breakable}
\newlist{rlist}{enumerate}{3}
\setlist[rlist]{itemsep=0mm,label=\roman*.}
\newlist{alist}{enumerate}{3}
\setlist[alist]{itemsep=0mm,label=\alph*.}
\newlist{balist}{enumerate}{3}
\setlist[balist]{itemsep=0mm,label=\bf\alph*.}
\newlist{Alist}{enumerate}{3}
\setlist[Alist]{itemsep=0mm,label=\Alph*.}
\newlist{bAlist}{enumerate}{3}
\setlist[bAlist]{itemsep=0mm,label=\bf\Alph*.}
\newlist{askhseis}{enumerate}{3}
\setlist[askhseis]{label={\Large\thesection}.\arabic*.}
\renewcommand{\textstigma}{\textsigma\texttau}
\newlist{thema}{enumerate}{3}
\setlist[thema]{label=\bf\large{ΘΕΜΑ \textcolor{black}{\Alph*}},itemsep=0mm,leftmargin=0cm,itemindent=18mm}
\newlist{erwthma}{enumerate}{3}
\setlist[erwthma]{label=\bf{\large{\textcolor{black}{\Alph{themai}.\arabic*}}},itemsep=0mm,leftmargin=0.8cm}

\newcommand{\kerkissans}[1]{{\fontfamily{maksf}\selectfont \textbf{#1}}}
\renewcommand{\textdexiakeraia}{}

\usepackage[
backend=biber,
style=alphabetic,
sorting=ynt
]{biblatex}

\DeclareTblrTemplate{caption}{nocaptemplate}{}
\DeclareTblrTemplate{capcont}{nocaptemplate}{}
\DeclareTblrTemplate{contfoot}{nocaptemplate}{}
\NewTblrTheme{mytabletheme}{
\SetTblrTemplate{caption}{nocaptemplate}{}
\SetTblrTemplate{capcont}{nocaptemplate}{}
\SetTblrTemplate{contfoot}{nocaptemplate}{}
}

\NewTblrEnviron{mytblr}
\SetTblrStyle{firsthead}{font=\bfseries}
\SetTblrStyle{firstfoot}{fg=red2}
\SetTblrOuter[mytblr]{theme=mytabletheme}
\SetTblrInner[mytblr]{
rowspec={t{7mm}},columns = {c},
width = 0.85\linewidth,
row{odd} = {bg=red9,fg=black,ht=8mm},
row{even} = {bg=red7,fg=black,ht=8mm},
hlines={white},vlines={white},
row{1} = {bg=xrwma, fg=white, font=\bfseries\fontfamily{maksf}},rowhead = 1,
hline{2} = {.7mm}, % midrule  
}
\newcounter{askhsh}
\setcounter{askhsh}{1}
\newcommand{\askhsh}{\large\theaskhsh.\ \addtocounter{askhsh}{1}}

\titleformat{\section}{\Large}{\kerkissans{\thesection}}{10pt}{\Large\kerkissans{#1}}

\setlength{\columnsep}{5mm}
\titleformat{\paragraph}
{\large}%
{}{0em}%
{\textcolor{xrwma}{\faSquare\ \ \kerkissans{\bmath{#1}}}}
\setlength{\parindent}{0pt}
\newcommand{\eng}[1]{\selectlanguage{english}#1\selectlanguage{greek}}

\begin{document}
\twocolumn[{
\begin{tikzpicture}[overlay,remember picture]
\fill[xrwma]($(current page.north west)$)--($(current page.north west)+(2cm,0)$)--($(current page.north west)+(2cm,-2.4cm)$)--($(current page.north west)+(-2cm,-2.4cm)$)--cycle;
\node (fig1) at ($(current page.north west)+(1cm,-1.2cm)$)
{\includegraphics[width=0.07\linewidth]{/home/spyros/Μαθηματικά/Φροντιστήριο ΦΙΛΟΜΑΘΕΙΑ/Βαθμοί - Φυλλάδιο ύλης - Διάφορα/Λογότυπα/logo.png}};
\node[gray2] at ($(0,0)+(-2.4cm,1.5cm)$) {\kerkissans{Φ\,ΡΟΝΤΙΣΤΗΡΙΟ ΜΕΣΗΣ ΕΚΠΑΙΔΕΥΣΗΣ}};
\node[xrwma] at ($(0,0)+(-2.4cm,.9cm)$) {\kerkissans{\huge {\fontsize{28}{33.6}\selectfont Φ\,ΙΛΟΜΑΘΕΙΑ}}};
\draw[gray4] (-5.2,.4)--(.4,.4);
\node at ($(0,0)+(-2.4cm,0cm)$) {\textcolor{xrwma}{\faMapMarker*} Ιακώβου Πολυλά 24, Πεζόδρομος};
\node (title) at ($(current page.north east)+(-3.5cm,-.7cm)$){\textcolor{gray2}{ \kerkissans{\LARGE Φ\,ΥΛΛΑΔΙΟ ΑΣΚΗΣΕΩΝ}}};
\node[below=of title.east,anchor=east,yshift=3mm] (mob) {\textcolor{xrwma}{\faPhone*} 26610 20144 - \textcolor{xrwma}{\faMobile*\ \faTelegram\ \faViber} 693 232 7283};
\node[below=of mob.east,anchor=east,yshift=4mm] (fb) {\textcolor{xrwma}{\faFacebook} Φροντιστήριο Φιλομάθεια - \textcolor{xrwma}{\faInstagram}\ {\eng{front\_filomatheia}}};
\end{tikzpicture}
\vspace{10mm}\mbox{}\\
\centering
\kerkissans{{\huge Αλγεβρα - Α' Λυκείου}\\\vspace{2mm} {\LARGE Ανισωσεις 1ου βαθμού}}\\\vspace{4mm}{\kerkissans{\today}}\\\vspace{3mm}}]
\paragraph{Ερωτήσεις Θεωρίας}
\askhsh 
\begin{alist}
\item Τι ονομάζουμε ανίσωση 1\textsuperscript{ου} βαθμού;
\item Πότε μια ανίσωση ονομάζεται αδύνατη;
\item Πότε μια ανίσωση ονομάζεται αόριστη;
\end{alist}
\askhsh Να χαρακτηριστούν οι παρακάτω εξισώσεις ως σωστές (Σ) ή λανθασμένες (Λ).
\begin{alist}[label=\roman*.]
\item Η ανίσωση $ ax+\beta>0 $ με $ a>0 $ έχει μια λύση την $ x>-\dfrac{\beta}{a} $.
\item Η ανίσωση $ ax+\beta>0 $ με $ a=0 $ και $ \beta>0 $ είναι αόριστη.
\item Η αν. $ 0x<\beta $ με $ \beta>0 $ είναι αδύνατη.
\item Η αν. $ 0x<\beta $ με $ \beta<0 $ είναι αδύνατη.
\item Η αν. $ 0x>\beta $ με $ \beta=0 $ είναι αόριστη.
\item Η αν. $ 0x\geq\beta $ με $ \beta=0 $ είναι αόριστη.
\item Η αν. $ 0x>\beta $ με $ \beta>0 $ είναι αδύνατη.
\item Η αν. $ 0x\leq\beta $ με $ \beta=0 $ είναι αδύνατη.
\item Η αν. $ 0x<-\beta $ με $ \beta>0 $ είναι αόριστη.
\item Η αν. $ 0x\geq-\beta $ με $ \beta=0 $ είναι αόριστη.
\end{alist}
\paragraph{Απλές ανισώσεις}
\askhsh Να λυθούν οι ανισώσεις και να παρασταθούν γραφικά οι λύσεις.
\begin{alist}
\item $ 2x-3>7-3x $
\item $ 4x+5<2-x+8 $
\item $ 3x-2\leq4-2x+8 $
\item $ -x-4\geq7-3x+2 $
\item $ 7x-3+x<2x+9+5x $
\item $ -3x+8>4-5x+12 $
\end{alist}
\askhsh Να λυθούν οι ανισώσεις και να παρασταθούν γραφικά οι λύσεις.
\begin{alist}
\item $ 2(x-1)+3>4-x $
\item $ 2x-3(4-x)<9+4x $
\item $ 4(3-x)+2(3x-1)<3x+2-(x-1) $
\item $ 3(2x+3)-5>5(x-4)+12 $
\item $ -2-3(4-3x)+5x\leq3-(7-2x) $
\item $ 2-(3x-4)+x\geq3(2x+3)-12-(x-2) $
\end{alist}
\askhsh Να λυθούν οι ανισώσεις και να παρασταθούν γραφικά οι λύσεις.
\begin{alist}
\item $ \dfrac{x}{2}+\dfrac{x+1}{3}>1 $
\item $ \dfrac{2x-1}{3}-\dfrac{x-2}{4}<\dfrac{1}{6} $
\item $ \dfrac{x}{5}+\dfrac{3x-2}{3}\leq\dfrac{x-1}{15} $
\item $ \dfrac{4x-3}{3}-\dfrac{3-2x}{4}\geq1+\dfrac{5x}{12} $
\item $ 2x-\dfrac{3x-2}{5}+\dfrac{x-1}{15}\leq\dfrac{1}{3}-\dfrac{2-3x}{15} $
\item $ \dfrac{-2-x}{4}+\dfrac{4x-5}{8}<3x-1-\dfrac{7-4x}{4} $
\item $ \dfrac{1-\dfrac{x}{2}}{3}>2 $
\item $ \dfrac{\dfrac{x-1}{3}+\dfrac{x-2}{4}}{2}-\dfrac{2x-1}{6}>\dfrac{x}{12} $
\end{alist}
\askhsh Να λυθούν οι ανισώσεις.
\begin{alist}
\item $ 3x-2<x+4+2x $
\item $ 4x-3+x\geq2x-3+2x $
\item $ 2(x-3)+1<-3x+5(x-2) $
\item $ 4x-(3+2x)>5(x-2)+3(2-x)+1 $
\item $ 5-(x-2)+3x\leq3(2+x)-x-1 $
\item $ \dfrac{2x-3}{4}-\dfrac{x}{2}>1 $
\item $ \dfrac{3x-4}{5}-\dfrac{x-3}{3}\geq\dfrac{x-1}{15}+\dfrac{x-4}{5} $
\end{alist}
\paragraph{Κοινές λύσεις ανισώσεων}
\askhsh Να λυθούν οι ανισώσεις και να παρασταθούν γραφικά οι λύσεις και να γραφτούν με τη μορφή διαστήματος.
\begin{alist}
\item $ 4x-3<3x<2-5x $
\item $ 3-2x\leq x+1<4x-5 $
\item $ 3(1-x+2)<4x\leq2(x+2)+3 $
\item $ 5(2x-1)-4\leq 7(3-x)\leq 4(2-x)+3x $
\item $ 3-(3x-4)<2(x-2)+4(3-x)<7-(x-3) $
\item $ \dfrac{x-1}{2}<\dfrac{x}{3}+1\leq\dfrac{2x-1}{2}-\dfrac{1}{3} $
\item $ \dfrac{3x-4}{5}\leq\dfrac{2-x}{3}<\dfrac{x}{15}+1 $
\end{alist}
\askhsh Να βρεθούν οι κοινές λύσεις των ανισώσεων και να γραφτούν με τη μορφή διαστήματος.
\begin{alist}
\item $ 3x-1>5 $ και $ 4x-3<9 $
\item $ 4-3x<2 $ και $ 2x+5\leq7 $
\item $ 2(x-3)+5>x-1 $ και $ 3-(x-4)\leq5-2x $
\item $ 4(x-2)+3(5-x)\geq4x-3+2(x-1) $ και $ 5(2-x)+3(x+1)<4-(x-7) $
\item $ \dfrac{x+5}{12}+1\geq\dfrac{x}{4} $ και $ \dfrac{2x+3}{4}+\dfrac{x-1}{3}>1 $
\end{alist}
\paragraph{Ανισώσεις με απόλυτες τιμές}
\askhsh Να λυθούν οι ανισώσεις.
\begin{multicols}{2}
\begin{alist}
\item $ \left|x\right|<4 $
\item $ \left|x\right|>5 $
\item $ \left|x-1\right|<2 $
\item $ \left|x+2\right|>3 $
\item $ \left|2x-1\right|\leq5 $
\item $ \left|3x+4\right|\geq8 $
\item $ \left|1-x\right|<2 $
\item $ \left|3-4x\right|\geq5 $
\end{alist}
\end{multicols}
\askhsh Να λυθούν οι ανισώσεις.
\begin{multicols}{2}
\begin{alist}
\item $ \left|2x+1\right|-3<0 $
\item $ \left|1-3x\right|+2>4 $
\item $ \left|3x+4\right|-5\leq0 $
\item $ 7-\left|2x-5\right|\geq0 $
\end{alist}
\end{multicols}
\askhsh Να λυθούν οι ανισώσεις.
\begin{multicols}{2}
\begin{alist}
\item $ \left|x\right|<-2 $
\item $ \left|4x\right|>-1 $
\item $ \left|x-3\right|\leq0 $
\item $ \left|2x-4\right|\geq0 $
\end{alist}
\end{multicols}
\askhsh Να λυθούν οι εξισώσεις.
\begin{multicols}{2}
\begin{alist}
\item $ \left|x-3\right|=x+2 $
\item $ \left|4x-1\right|=2x-5 $
\item $ \left|2x-3\right|=4-7x $
\item $ \left|\dfrac{x}{2}-1\right|=\dfrac{x+3}{4} $
\end{alist}
\end{multicols}
\askhsh Να βρεθούν οι κοινές λύσεις των ανισώσεων και να γραφτούν με τη μορφή διαστήματος.
\begin{multicols}{2}
\begin{alist}
\item $ 2\leq\left|x\right|\leq3 $
\item $ 3\leq\left|x-1\right|\leq7 $
\item $4\leq\left|2x-4\right|\leq8 $
\item $1<\left|x+5\right|\leq5 $
\item $2\leq\left|2-3x\right|<7 $
\end{alist}
\end{multicols}
\askhsh Να βρεθούν οι κοινές λύσεις των ανισώσεων και να γραφτούν με τη μορφή διαστήματος.
\begin{multicols}{2}
\begin{alist}
\item $ 2\left( \left|x\right|+2\right) $
\end{alist}
\end{multicols}
\end{document}
