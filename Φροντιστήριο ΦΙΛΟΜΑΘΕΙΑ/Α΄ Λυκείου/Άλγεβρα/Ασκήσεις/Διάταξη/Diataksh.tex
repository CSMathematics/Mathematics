\documentclass[11pt,a4paper,twocolumn]{article}
\usepackage[english,greek]{babel}
\usepackage[utf8]{inputenc}
\usepackage{nimbusserif}
\usepackage[T1]{fontenc}
\usepackage[left=1.50cm, right=1.50cm, top=2.00cm, bottom=2.00cm]{geometry}
\usepackage{amsmath}
\let\myBbbk\Bbbk
\let\Bbbk\relax
\usepackage[amsbb,subscriptcorrection,zswash,mtpcal,mtphrb,mtpfrak]{mtpro2}
\usepackage{graphicx,multicol,multirow,enumitem,tabularx,mathimatika,gensymb,venndiagram,hhline,longtable,tkz-euclide,fontawesome5,eurosym,tcolorbox,tabularray,tikzpagenodes,relsize}
\definecolor{xrwma}{HTML}{3d93cd}
\usetikzlibrary{calc}
\usetikzlibrary{positioning}
\usepackage[explicit]{titlesec}
\tcbuselibrary{skins,theorems,breakable}
\newlist{rlist}{enumerate}{3}
\setlist[rlist]{itemsep=0mm,label=\roman*.}
\newlist{alist}{enumerate}{3}
\setlist[alist]{itemsep=0mm,label=\alph*.}
\newlist{balist}{enumerate}{3}
\setlist[balist]{itemsep=0mm,label=\bf\alph*.}
\newlist{Alist}{enumerate}{3}
\setlist[Alist]{itemsep=0mm,label=\Alph*.}
\newlist{bAlist}{enumerate}{3}
\setlist[bAlist]{itemsep=0mm,label=\bf\Alph*.}
\newlist{askhseis}{enumerate}{3}
\setlist[askhseis]{label={\Large\thesection}.\arabic*.}
\renewcommand{\textstigma}{\textsigma\texttau}
\newlist{thema}{enumerate}{3}
\setlist[thema]{label=\bf\large{ΘΕΜΑ \textcolor{black}{\Alph*}},itemsep=0mm,leftmargin=0cm,itemindent=18mm}
\newlist{erwthma}{enumerate}{3}
\setlist[erwthma]{label=\bf{\large{\textcolor{black}{\Alph{themai}.\arabic*}}},itemsep=0mm,leftmargin=0.8cm}

\newcommand{\kerkissans}[1]{{\fontfamily{maksf}\selectfont \textbf{#1}}}
\renewcommand{\textdexiakeraia}{}

\usepackage[
backend=biber,
style=alphabetic,
sorting=ynt
]{biblatex}

\DeclareTblrTemplate{caption}{nocaptemplate}{}
\DeclareTblrTemplate{capcont}{nocaptemplate}{}
\DeclareTblrTemplate{contfoot}{nocaptemplate}{}
\NewTblrTheme{mytabletheme}{
\SetTblrTemplate{caption}{nocaptemplate}{}
\SetTblrTemplate{capcont}{nocaptemplate}{}
\SetTblrTemplate{contfoot}{nocaptemplate}{}
}

\NewTblrEnviron{mytblr}
\SetTblrStyle{firsthead}{font=\bfseries}
\SetTblrStyle{firstfoot}{fg=red2}
\SetTblrOuter[mytblr]{theme=mytabletheme}
\SetTblrInner[mytblr]{
rowspec={t{7mm}},columns = {c},
width = 0.85\linewidth,
row{odd} = {bg=red9,fg=black,ht=8mm},
row{even} = {bg=red7,fg=black,ht=8mm},
hlines={white},vlines={white},
row{1} = {bg=xrwma, fg=white, font=\bfseries\fontfamily{maksf}},rowhead = 1,
hline{2} = {.7mm}, % midrule  
}
\newcounter{askhsh}
\setcounter{askhsh}{1}
\newcommand{\askhsh}{\large\theaskhsh.\ \addtocounter{askhsh}{1}}

\titleformat{\section}{\Large}{\kerkissans{\thesection}}{10pt}{\Large\kerkissans{#1}}

\setlength{\columnsep}{5mm}
\titleformat{\paragraph}
{\large}%
{}{0em}%
{\textcolor{xrwma}{\faSquare\ \ \kerkissans{\bmath{#1}}}}
\setlength{\parindent}{0pt}
\newcommand{\eng}[1]{\selectlanguage{english}#1\selectlanguage{greek}}

\begin{document}
\twocolumn[{
\begin{tikzpicture}[overlay,remember picture]
\fill[xrwma]($(current page.north west)$)--($(current page.north west)+(2cm,0)$)--($(current page.north west)+(2cm,-2.4cm)$)--($(current page.north west)+(-2cm,-2.4cm)$)--cycle;
\node (fig1) at ($(current page.north west)+(1cm,-1.2cm)$)
{\includegraphics[width=0.07\linewidth]{/home/spyros/Μαθηματικά/Φροντιστήριο ΦΙΛΟΜΑΘΕΙΑ/Βαθμοί - Φυλλάδιο ύλης - Διάφορα/Λογότυπα/logo.png}};
\node[gray2] at ($(0,0)+(-2.4cm,1.5cm)$) {\kerkissans{Φ\,ΡΟΝΤΙΣΤΗΡΙΟ ΜΕΣΗΣ ΕΚΠΑΙΔΕΥΣΗΣ}};
\node[xrwma] at ($(0,0)+(-2.4cm,.9cm)$) {\kerkissans{\huge {\fontsize{28}{33.6}\selectfont Φ\,ΙΛΟΜΑΘΕΙΑ}}};
\draw[gray4] (-5.2,.4)--(.4,.4);
\node at ($(0,0)+(-2.4cm,0cm)$) {\textcolor{xrwma}{\faMapMarker*} Ιακώβου Πολυλά 24, Πεζόδρομος};
\node (title) at ($(current page.north east)+(-3.5cm,-.7cm)$){\textcolor{gray2}{ \kerkissans{\LARGE Φ\,ΥΛΛΑΔΙΟ ΑΣΚΗΣΕΩΝ}}};
\node[below=of title.east,anchor=east,yshift=3mm] (mob) {\textcolor{xrwma}{\faPhone*} 26610 20144 - \textcolor{xrwma}{\faMobile*\ \faTelegram\ \faViber} 693 232 7283};
\node[below=of mob.east,anchor=east,yshift=4mm] (fb) {\textcolor{xrwma}{\faFacebook} Φροντιστήριο Φιλομάθεια - \textcolor{xrwma}{\faInstagram}\ {\eng{front\_filomatheia}}};
\end{tikzpicture}
\vspace{10mm}\mbox{}\\
\centering
\kerkissans{{\huge Άλγεβρα - Α' Λυκείου}\\\vspace{2mm} {\LARGE Διάταξη}}\\\vspace{4mm}{\kerkissans{\today}}\\\vspace{3mm}}]
\paragraph{Πρόσημο παράστασης}
\askhsh \textbf{Πρόσημο παράστασης}\\
Δίνεται πραγματικός αριθμός $ a\in\mathbb{R} $ τέτοιος ώστε να ισχύει $ 1<a<4 $. Να βρεθούν τα πρόσημα των παρακάτω παραστάσεων.
\begin{multicols}{3}
\begin{rlist}
\item $ a-4 $
\item $ a-1 $
\item $ 1-a $
\item $ 8-a $
\item $ a-5 $
\item $ 7-a $
\end{rlist}
\end{multicols}
\askhsh \textbf{Πρόσημο παράστασης}\\
Έστω ένας πραγματικός αριθμός $ a\in\mathbb{R} $ ο οποίος ικανοποιεί τη σχέση $ 2<a<7 $. Να βρεθούν τα πρόσημα των παρακάτω παραστάσεων.
\begin{rlist}
\item $ (a-2)(a-7) $
\item $ (2a-3)(7-a)(1-a) $
\item $ (3a-5)(14-2a) $
\end{rlist}
\askhsh \textbf{Πρόσημο παράστασης}\\
Δίνονται τρεις πραγματικοί αριθμοί $ x,y,z $ τέτοιοι ώστε $ x<y<z $. Να βρεθεί το πρόσημο των παρακάτω παραστάσεων.
\begin{rlist}
\item $ (x-y)(z-x) $
\item $ (x-z)(y-x) $
\item $ (y-z)(x-y)(x-z) $
\end{rlist}
\askhsh \textbf{Πρόσημο παράστασης}\\
Έστω ένας πραγματικός αριθμός $ a $ για τον οποίο γνωρίζουμε ότι ισχύει $ 2<a<4 $. Να βρεθούν οι αριθμοί μεταξύ των οποίων βρίσκονται οι παρακάτω παραστάσεις.
\begin{multicols}{3}
\begin{rlist}
\item $ a^2-4 $
\item $ 2a^2+3 $
\item $ a^3 $
\item $ 2a-a^2 $
\item $ 3a^2-a^3 $
\item $ a^4-3a^2 $
\end{rlist}
\end{multicols}
\askhsh \textbf{Σύγκριση αριθμών}\\
Να συγκρίνετε τους πραγματικούς αριθμούς $ A $ και $ B $ στις παρακάτω περιπτώσεις.
\begin{rlist}
\item $ A=3\sqrt{3}-1 $ και $ B=2\sqrt{3}-2 $
\item $ A=4\sqrt{5}+\sqrt{2} $ και $ B=3\sqrt{5}+2\sqrt{2} $
\item $ A=\frac{2\sqrt{2}-1}{2} $ και $ B=\frac{3}{2} $
\item $ A=\left(\!\sqrt{2}\right)^3+1 $ και $ B=\sqrt{2}+2 $
\end{rlist}
\askhsh \textbf{Σύγκριση αριθμών}\\
Να συγκρίνετε τους πραγματικούς αριθμούς $ A $ και $ B $ στις παρακάτω περιπτώσεις.
\begin{rlist}
\item $ A=2^{70} $ και $ B=4^{30} $
\item $ A=9^{120} $ και $ B=8^{150} $
\item $ A= $ και $ B= $
\item $ A= $ και $ B= $
\end{rlist}
\askhsh \textbf{Σύγκριση αριθμών}\\
Να συγκριθούν οι παρακάτω αλγεβρικές παραστάσεις $ A $ και $ B $ στις παρακάτω περιπτώσεις.
\begin{rlist}
\item $ A=(x+y)^2 $ και $ B=1-(x-y)^2 $
\item 
\item 
\item 
\end{rlist}
\askhsh \textbf{Σύγκριση αριθμών}\\
Να συγκριθούν οι παρακάτω αλγεβρικές παραστάσεις $ A $ και $ B $ στις παρακάτω περιπτώσεις.
\begin{rlist}
\item $ A=\frac{a-\beta}{a+\beta} $ και $ B=\frac{a\beta}{a^2-\beta^2} $
\item 
\item 
\item 
\end{rlist}
\askhsh \textbf{Τετράγωνο αριθμού}\\
Να αποδειχθεί ότι ισχύουν οι παρακάτω ανισότητες για κάθε τιμή της μεταβλητής $ x $.
\begin{multicols}{2}
\begin{rlist}
\item $ x^2+1\geq 2x $
\item $ 4x\leq 1+4x^2 $
\item $ 6x-9\leq x^2 $
\item $ (x+1)^2\geq 4x $
\end{rlist}
\end{multicols}
\askhsh \textbf{Τετράγωνο αριθμού}\\
Να αποδειχθεί ότι ισχύουν οι παρακάτω ανισότητες για κάθε τιμή των μεταβλητών $ x,y $.
\begin{multicols}{2}
\begin{rlist}
\item $ x^2+y^2\geq 2xy $
\item $ x^2+1\geq 2y-y^2 $
\item $ (x+y)^2\geq 4xy $
\item $ 4x^2\geq x^2-4y^2 $
\end{rlist}
\end{multicols}
\vspace{-8mm}
\begin{rlist}[start=5]
\item $ 3x^2+1\geq 4x-x^2-y^2 $
\end{rlist}
\askhsh \textbf{Τετράγωνο αριθμού}\\
Να βρεθούν οι πραγματικοί αριθμοί $ x,y $ σε καθεμία από τις παρακάτω περιπτώσεις.
\begin{rlist}
\item $ (x-1)^2+(y+2)^2=0 $
\item $ (3x-6)^2+(2y-8)^2=0 $
\item $ x^2+y^2-2y+1=0 $
\item $ x^2+y^2-4x+6y+13=0 $
\end{rlist}
\askhsh \textbf{Ιδιότητες ανισοτήτων}\\
Έστω $ a\in\mathbb{R} $ πραγματικός αριθμός για τον οποίο ισχύει $ 2<a<5 $. Να βρεθεί μεταξύ ποιών αριθμών βρίσκονται οι παρακάτω παραστάσεις.
\begin{multicols}{2}
\begin{rlist}
\item $ a-7 $
\item $ 3a+4 $
\item $ 3-2a $
\item $ 7a+13 $
\end{rlist}
\end{multicols}
\askhsh \textbf{Ιδιότητες ανισοτήτων}\\
Δίνονται δύο πραγματικοί αριθμοί $ a,\beta\in\mathbb{R} $ για τους οποίους ισχύουν οι σχέσεις $ 3<a<7 $ και $ 2<\beta<4 $. Να αποδειχθούν οι παρακάτω ανισότητες.
\begin{multicols}{2}
\begin{rlist}
\item $ 8<2a+\beta<18 $
\item $ -1<a-\beta<5 $
\item $ 24<4a\beta<112 $
\item $ \frac{3}{4}<\frac{a}{\beta}<\frac{7}{2} $
\end{rlist}
\end{multicols}
\askhsh \textbf{Ιδιότητες ανισοτήτων}\\
Δίνονται δύο πραγματικοί αριθμοί $ a,\beta\in\mathbb{R} $ για τους οποίους ισχύουν οι σχέσεις $ 1<a<5 $ και $ 2<\beta<3 $. Να βρεθεί μεταξύ ποιών αριθμών βρίσκονται οι παρακάτω παραστάσεις.
\begin{multicols}{2}
\begin{rlist}
\item $ 3a+2\beta $
\item $ 4a-3\beta $
\item $ 4a\beta-5 $
\item $ \frac{3a}{2\beta} $
\end{rlist}
\end{multicols}
\askhsh \textbf{Διαστήματα}\\
Να γραφτούν τα παρακάτω διαστήματα με τη μορφή ανισοτήτων.
\begin{multicols}{3}
\begin{rlist}
\item $ [3,7] $
\item $ [-2,5) $
\item $ (0,3] $
\item $ (-\infty,4] $
\item $ (0,+\infty) $
\item $ (-4,4) $
\end{rlist}
\end{multicols}
\askhsh \textbf{Διαστήματα}\\
Να γραφτούν οι παρακάτω ανισότητες με τη μορφή διαστημάτων.
\begin{multicols}{2}
\begin{rlist}
\item $ 2\leq x\leq 8 $
\item $ -3<x<10 $
\item $ x>-2 $
\item $ x\leq 4 $
\item $ -1\leq x<3 $
\item $ x\geq 0 $
\end{rlist}
\end{multicols}
\askhsh \textbf{Διαστήματα}\\
Να παρασταθούν τα παρακάτω διαστήματα γραφικά, πάνω στην ευθεία των πραγματικών αριθμών.
\begin{multicols}{3}
\begin{rlist}[leftmargin=5mm]
\item $ [-2,2] $
\item $ [4,9) $
\item $ (-3,0] $
\item $ (4,+\infty) $
\item $ (-\infty,3] $
\item $ (-4,5) $
\end{rlist}
\end{multicols}
\askhsh \textbf{Διαστήματα}\\
Να παραστήσετε γραφικά σε άξονα τα σύνολα αριθμών που ορίζονται από τις παρακάτω ανισότητες.
\begin{multicols}{3}
\begin{rlist}[leftmargin=2mm]
\item $ -3\leq x\leq 2 $
\item $ 1<x\leq 5 $
\item $ 0<x<10 $
\item $ x>-4 $
\item $ x<0 $
\item $ x\geq 9 $
\end{rlist}
\end{multicols}
\askhsh \textbf{Διαστήματα}\\
Να γράψετε σε μορφή διαστήματος κάθε σύνολο αριθμών από τα παρακάτω, όπως αυτά φαίνονται στους άξονες.
\begin{multicols}{2}
\begin{rlist}
\item \tikzitem\begin{tikzpicture}
\Diasthma{1}{4}{0}{2}{.3}{xrwma}
\Axonas{-.3}{2.5}
\Akro{k}{0}
\Akro{k}{2}
\end{tikzpicture}
\item \tikzitem\begin{tikzpicture}
\Diasthma{-2}{5}{0}{2}{.3}{xrwma}
\Axonas{-.3}{2.5}
\Akro{a}{0}
\Akro{k}{2}
\end{tikzpicture}
\item \tikzitem\begin{tikzpicture}
\Diasthma{0}{1}{0}{2}{.3}{xrwma}
\Axonas{-.3}{2.5}
\Akro{k}{0}
\Akro{a}{2}
\end{tikzpicture}
\item \tikzitem\begin{tikzpicture}
\Diasthma{-4}{5}{0}{2}{.3}{xrwma}
\Axonas{-.3}{2.5}
\Akro{a}{0}
\Akro{a}{2}
\end{tikzpicture}
\end{rlist}
\end{multicols}
\vspace{-5mm}
\askhsh \textbf{Διαστήματα}\\
Να γράψετε σε μορφή διαστήματος κάθε σύνολο αριθμών από τα παρακάτω, όπως αυτά φαίνονται στους άξονες.
\begin{multicols}{2}
\begin{rlist}
\item \tikzitem\begin{tikzpicture}
\Xapeiro{2}{0}{2.3}{.3}{xrwma}
\Axonas{-.3}{2.5}
\Akro{a}{0}
\end{tikzpicture}
\item \tikzitem\begin{tikzpicture}
\ApeiroX{4}{2}{-.3}{.3}{xrwma}
\Axonas{-.3}{2.5}
\Akro{k}{2}
\end{tikzpicture}
\item \tikzitem\begin{tikzpicture}
\Xapeiro{-5}{0}{2.3}{.3}{xrwma}
\Axonas{-.3}{2.5}
\Akro{k}{0}
\end{tikzpicture}
\item \tikzitem\begin{tikzpicture}
\ApeiroX{0}{2}{-.3}{.3}{xrwma}
\Axonas{-.3}{2.5}
\Akro{a}{2}
\end{tikzpicture}
\end{rlist}
\end{multicols}
\vspace{-5mm}
\askhsh \textbf{Διαστήματα}\\
Να βρεθούν τα κοινά στοιχεία των παρακάτω διαστημάτων.
\begin{multicols}{2}
\begin{rlist}
\item $ [2,4]\ ,\ [3,7] $
\item $ (-\infty,5)\ ,\ (-3,7) $
\item $ [0,3)\ ,\ (1,3] $
\item $ (-\infty,4)\ ,\ [5,+\infty) $
\item $ (-3,4]\ ,\ (-\infty,4) $
\item $ [2,+\infty)\ ,\ (3,+\infty) $
\end{rlist}
\end{multicols}
\askhsh \textbf{Διαστήματα}\\
Αν $ \varDelta_1,\varDelta_2 $ είναι δύο διαστήματα πραγματικών αριθμών τότε να βρεθεί η ένωση $ \varDelta_1\cup\varDelta_2 $ και η τομή τους $ \varDelta_1\cap\varDelta_2 $ σε καθεμία από τις παρακάτω περιπτώσεις.
\begin{rlist}
\item $ \varDelta_1=[-3,3] $ και $ \varDelta_2=(-2,4] $.
\item $ \varDelta_1=[1,4) $ και $ \varDelta_2=(0,+\infty) $.
\item $ \varDelta_1=(-4,0) $ και $ \varDelta_2=(-\infty,-2] $.
\item $ \varDelta_1=(-\infty,10) $ και $ \varDelta_2=(11,+\infty) $.
\end{rlist}
\askhsh \textbf{Διαστήματα}\\
Δίνονται τα διαστήματα πραγματικών αρθμών $ \varDelta_1=[1,4],\varDelta_2=(2,+\infty) $. Να βρεθούν τα παρακάτω σύνολα.
\begin{multicols}{3}
\begin{rlist}
\item $ \varDelta_1\cup\varDelta_2 $
\item $ \varDelta_1\cap\varDelta_2 $
\item $ \varDelta_1' $
\item $ \varDelta_2' $
\item $ \varDelta_1-\varDelta_2 $
\item $ \varDelta_2-\varDelta_1 $
\end{rlist}
\end{multicols}
\askhsh \textbf{Μετατροπή σχήματος σε σύνολο}\\
Να βρείτε και να γράψετε στη μορφή διαστήματος ή ένωσης διαστημάτων τα κοινά στοιχεία των συνόλων που φαίνονται στα παρακάτω σχήματα.
\begin{multicols}{2}
\begin{rlist}[leftmargin=4mm]
\item \tikzitem\begin{tikzpicture}
\ApeiroX{5}{2}{-.4}{.35}{yellow}
\Xapeiro{1}{0}{2.4}{.3}{xrwma}
\Axonas{-.3}{2.5}
\Akro{a}{0}
\Akro{k}{2}
\end{tikzpicture}
\item \tikzitem\begin{tikzpicture}
\Xapeiro{3}{.5}{2.3}{.35}{yellow}
\Xapeiro{-2}{0}{2.5}{.3}{xrwma}
\Axonas{-.3}{2.5}
\Akro{a}{0}
\Akro{k}{0.5}
\end{tikzpicture}
\item \tikzitem\begin{tikzpicture}
\ApeiroX{-1}{2.1}{-.3}{.3}{yellow}
\ApeiroX{-4}{1.5}{-.1}{.35}{xrwma}
\Axonas{-.3}{2.5}
\Akro{a}{1.5}
\Akro{k}{2.1}
\end{tikzpicture}
\item \tikzitem\begin{tikzpicture}
\ApeiroX{5}{1.5}{-.3}{.3}{yellow}
\Diasthma{2}{7}{0.1}{2}{.35}{xrwma}
\Axonas{-.3}{2.5}
\Akro{k}{.1}
\Akro{a}{1.5}
\Akro{k}{2}
\end{tikzpicture}
\item \tikzitem\begin{tikzpicture}
\Xapeiro{-3}{.55}{2.4}{.35}{yellow}
\Diasthma{-5}{4}{0}{2}{.3}{xrwma}
\Axonas{-.3}{2.5}
\Akro{a}{0}
\Akro{k}{.55}
\Akro{k}{2}
\end{tikzpicture}
\item \tikzitem\begin{tikzpicture}
\Diasthma{0}{7}{0}{1.5}{.3}{yellow}
\Diasthma{2}{9}{0.5}{2}{.35}{xrwma}
\Axonas{-.3}{2.5}
\Akro{a}{0}
\Akro{k}{.5}
\Akro{a}{2}
\Akro{k}{1.5}
\end{tikzpicture}
\end{rlist}
\end{multicols}
\end{document}
