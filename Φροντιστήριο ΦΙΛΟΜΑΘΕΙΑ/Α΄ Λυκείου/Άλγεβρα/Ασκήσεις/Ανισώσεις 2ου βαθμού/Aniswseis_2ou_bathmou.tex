\documentclass[11pt,a4paper,modern]{FFExercises}
\usepackage[english,greek]{babel}
\usepackage[utf8]{inputenc}
\usepackage{nimbusserif}
\usepackage[T1]{fontenc}
\usepackage{amsmath}
\let\myBbbk\Bbbk
\let\Bbbk\relax
\usepackage[amsbb,subscriptcorrection,zswash,mtpcal,mtphrb,mtpfrak]{mtpro2}
\usepackage{graphicx,multicol,multirow,enumitem,tabularx,mathimatika,gensymb,venndiagram,hhline,longtable,tkz-euclide,fontawesome5,eurosym,tcolorbox,tabularray,tikzpagenodes,relsize}
\definecolor{xrwma}{HTML}{aa1212}
\usetikzlibrary{calc}
\usetikzlibrary{positioning}
\tcbuselibrary{skins,theorems,breakable}
\renewcommand{\textstigma}{\textsigma\texttau}
\renewcommand{\textdexiakeraia}{}

\ekthetesdeiktes
\begin{document}

\titlos{Άλγεβρα}{Α' Λυκείου}{Ανισώσεις 2ου βαθμού}
\paragraph{Παραγοντοποίηση τριωνύμου}
\askhsh Να παραγοντοποιηθούν τα παρακάτω τριώνυμα
\begin{multicols}{2}
\begin{alist}
\item $ x^2-5x+6 $
\item $ x^2-3x+2 $
\item $ x^2+4x+4 $
\item $ 25x^2-10x+1 $
\item $ 2x^2-5x+3 $
\item $ x^2+x+2 $
\end{alist}
\end{multicols}
\askhsh Να απλοποιηθούν οι παρακάτω ρητές παραστάσεις.
\begin{multicols}{2}
\begin{alist}
\item $ \dfrac{x^2-3x+2}{x^2-5x+6} $
\item $ \dfrac{x^2-2x+1}{x^2-1} $
\item $ \dfrac{2x^2-5x+3}{4x^2-4x+1} $
\end{alist}
\end{multicols}
\paragraph{Ανισώσεις - Πρόσημο τριωνύμου}
\askhsh Να βρεθούν τα πρόσημα των παρακάτω τριωνύμων.
\begin{multicols}{2}
\begin{alist}
\item $ x^2-3x+2 $
\item $ -x^2+8x-7 $
\item $ 3x^2-7x+2 $
\item $ x^2+6x+9 $
\item $ -x^2+10x-25 $
\item $ x^2+x+1 $
\end{alist}
\end{multicols}
\askhsh Να λυθούν οι παρακάτω ανισώσεις.
\begin{multicols}{2}
\begin{alist}
\item $ x^2-4x+3>0 $
\item $ 4x^2-5x+1<0 $
\item $ -2x^2-7x-6\leq0 $
\item $ -x^2+9x-10\geq0 $
\item $ 9x^2-6x+1>0 $
\item $ x^2+10x+25<0 $
\item $ -4x^2+4x-1<0 $
\item $ x^2+3x+5\geq0 $
\item $ -x^2-x+4>0 $
\end{alist}
\end{multicols}
\askhsh Να λυθούν οι παρακάτω ανισώσεις.
\begin{multicols}{2}
\begin{alist}
\item $ x^2-8x\leq -7 $
\item $ 4-x^2\geq 3x $
\item $ (x-2)^2>2x-5 $
\item $ 2(3-x)<(1-x)^2+4 $
\end{alist}
\end{multicols}
\askhsh Να βρεθούν οι κοινές λύσεις των παρακάτω ανισώσεων.
\begin{alist}
\item $ x^2-7x+6<0 \;$ και $\; -x^2+5x-6>0 $
\item $ x^2-6x+9\geq0 \;$ και $\; x^2+4x-3>0 $
\item $ 3-(x-1)^2<2x-5 \;$ και $\; (x+2)^2\geq(2x+3)^2 $
\end{alist}
\paragraph{Παραμετρικές ανισώσεις}
\askhsh Να δειχθεί ότι η εξίσωση \[ (\lambda+1)x^2-2\lambda x+\lambda-1=0 \] με $ \lambda\neq-1 $ έχει δύο πραγματικές λύσεις για κάθε $ \lambda\in\mathbb{R} $.
\askhsh Δίνεται η εξίσωση $ (1-\lambda)x^2+2\lambda x-4=0 $ με $ \lambda\neq1 $.
\begin{alist}
\item Να γραφτεί η διακρίνουσα της παραπάνω εξίσωσης σαν συνάρτηση του $ \lambda $.
\item Να υπολογιστούν οι τιμές της παραμέτρου $ \lambda $ για τις οποίες η εξίσωση
\begin{rlist}
\item έχει δύο ρίζες άνισες.
\item έχει μια ρίζα.
\item είναι αδύνατη.
\end{rlist}
\end{alist}
\askhsh Δίνεται η εξίσωση $ x^2-(\lambda-3)x+4=0 $.
\begin{alist}
\item Να βρεθούν οι τιμές της παραμέτρου $ \lambda $ ώστε η εξίσωση να έχει δύο πραγματικές και άνισες λύσεις.
\item Αν $ x_1, x_2 $ είναι οι λύσεις της εξίσωσης τότε να υπολογιστούν το άθροισμα τους $ S $ και το γινόμενό τους $ P $.
\item Να λυθεί η ανίσωση $ -(x_1+x_2)^2+6x_1x_2+1\geq0 $
\end{alist}
\askhsh Δίνεται η εξίσωση $ (\lambda-2) x^2-2\lambda x-1=0 $ με $ 1<\lambda\neq2 $.
\begin{alist}
\item Να δειχθεί οτι η εξίσωση έχει πάντα πραγματικές λύσεις για κάθε τιμή του $ \lambda\in(1,+\infty)-\{2\} $.
\item Αν $ x_1, x_2 $ είναι οι λύσεις της εξίσωσης να εκφραστούν το άθροισμα $ S $ και το γινόμενο $ P $ των λύσεων με τη βοήθεια του $ \lambda $.
\item Να βρεθούν οι τιμές του $ \lambda $ για τις οποίες ισχύει $ x_1+x_2+\dfrac{18x_1x_2}{\lambda}=0 $
\end{alist}
\askhsh Να βρεθούν οι τιμές της παραμέτρου $ \lambda\in\mathbb{R} $ ώστε η εξίσωση 
\[ x^2+\left( \lambda^2-3\lambda+2\right) x+1=0 \]
\begin{alist}
\item να έχει δύο λύσεις άνισες.
\item οι λύσεις της εξίσωσης να είναι θετικές για κάθε τιμή της παραμέτρου $ \lambda $.
\end{alist}
\askhsh Να βρεθούν οι τιμές της παραμέτρου $ \lambda\in(4,+\infty) $ ώστε οι λύσεις της εξίσωσης  \[ x^2-\left( \lambda^2-5\lambda+6\right) x+\lambda-3=0 \] να είναι θετικές για κάθε τιμή της παραμέτρου $ \lambda $. Η διακρίνουσα του τριωνύμου είναι θετική.\\\\
\askhsh Δίνεται η εξίσωση \[ x^2-\left(\lambda^2-4\lambda+3\right)x+4-3\lambda-\lambda^2=0 \] με $\lambda\in\mathbb{R}$. Να βρεθούν οι τιμές της παραμέτρου $ \lambda $ ώστε
\begin{alist}
\item η εξίσωση να έχει δύο λύσεις άνισες.
\item η εξίσωση να έχει μια διπλή λύση.
\item οι ρίζες τις εξίσωσης να είναι
\begin{rlist}
\item ομόσημες
\item ετερόσημες
\item θετικές
\item αρνητικές
\end{rlist}
\end{alist}

\end{document}
