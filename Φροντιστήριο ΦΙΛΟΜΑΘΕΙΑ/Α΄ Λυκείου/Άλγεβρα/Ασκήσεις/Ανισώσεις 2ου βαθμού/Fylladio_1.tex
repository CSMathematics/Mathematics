\documentclass[11pt]{article}
\usepackage[utf8]{inputenc}
\usepackage{nimbusserif}
\usepackage[T1]{fontenc}
\usepackage[english,greek]{babel}
\usepackage{amsmath} 
\let\myBbbk\Bbbk 
\let\Bbbk\relax 
\usepackage[amsbb,subscriptcorrection,zswash,mtpcal,mtphrb,mtpfrak]{mtpro2} 
%------TIKZ - ΣΧΗΜΑΤΑ - ΓΡΑΦΙΚΕΣ ΠΑΡΑΣΤΑΣΕΙΣ ---- 
\usepackage{tikz,pgfplots,tkz-tab} 
\usepackage{tkz-euclide} 
\usepackage[framemethod=TikZ]{mdframed} 
\usetikzlibrary{decorations.pathreplacing} 
\tkzSetUpPoint[size=2.9,fill=white]
%----------------------- 
\usepackage{calc,tcolorbox} 
\tcbuselibrary{skins,theorems,breakable} 
\usepackage{hhline} 
\usepackage[explicit]{titlesec} 
\usepackage{graphicx} 
\usepackage{multicol} 
\usepackage{multirow} 
\usepackage{tabularx} 
\usetikzlibrary{backgrounds} 
\usepackage{sectsty} 
\sectionfont{\centering} 
\usepackage{enumitem} 
\usepackage{adjustbox} 
\usepackage{mathimatika,gensymb,eurosym,wrap-rl} 
\usepackage{systeme,regexpatch} 
%-------- ΜΑΘΗΜΑΤΙΚΑ ΕΡΓΑΛΕΙΑ --------- 
\usepackage{mathtools} 
%---------------------- 
%-------- ΠΙΝΑΚΕΣ --------- 
\usepackage{booktabs} 
%---------------------- 
%----- ΥΠΟΛΟΓΙΣΤΗΣ ---------- 
\usepackage{calculator} 
%---------------------------- 
%------------------------------------------ 
\newcommand{\tss}[1]{\textsuperscript{#1}} 
\newcommand{\tssL}[1]{\MakeLowercase{\textsuperscript{#1}}} 
\tikzstyle{pl}=[line width=0.3mm] 
\tikzstyle{plm}=[line width=0.4mm] 
\usepackage{etoolbox} 
\makeatletter 
\renewrobustcmd{\anw@true}{\let\ifanw@\iffalse} 
\renewrobustcmd{\anw@false}{\let\ifanw@\iffalse}\anw@false 
\newrobustcmd{\noanw@true}{\let\ifnoanw@\iffalse} 
\newrobustcmd{\noanw@false}{\let\ifnoanw@\iffalse}\noanw@false 
\renewrobustcmd{\anw@print}{\ifanw@\ifnoanw@\else\numer@lsign\fi\fi} 
\makeatother
\newlist{alist}{enumerate}{3}
\setlist[alist]{itemsep=0mm,label=\alph*.}
\newlist{rlist}{enumerate}{3}
\setlist[rlist]{itemsep=0mm,label=\roman*.}
\newlist{balist}{enumerate}{3}
\setlist[balist]{itemsep=0mm,label=\bf\alph*.}
\newlist{Alist}{enumerate}{3}
\setlist[Alist]{itemsep=0mm,label=\Alph*.}
\newlist{bAlist}{enumerate}{3}
\setlist[bAlist]{itemsep=0mm,label=\bf\Alph*.}
\renewcommand{\textstigma}{\textsigma\texttau}
\makeatletter
\xpatchcmd{\tkzTabLine}
{\node at (Z\thetkz@cnt@impair\thetkz@cnt@lg){$0$};} % search
{\node[fill=white,inner sep=.5mm] at (Z\thetkz@cnt@impair\thetkz@cnt@lg){$0$};} % replace
{}{}
\makeatother
\newcommand{\en}[1]{\selectlanguage{english}{#1}\selectlanguage{greek}}
\newcommand{\roloi}[4][]{
\draw[line width=.5mm,#1] (0,0) circle(2);
\foreach \n in {1,2,...,12}{
\tkzDefPoint(30*\n-90:2){A_\n}
%\tkzDrawPoint(A_\n)
\node at (-30*\n+90:1.65){\n};}
\draw[plm,,#1] (0,0)--(90-30*#2-0.5*#3:1);
\draw[pl,#1] (0,0)--(90-6*#3-0.1*#4:1.5);
\draw[#1](0,0)--(90-6*#4:1.2);
\tkzDrawPoint[fill=#1,color=#1](0,0)
\foreach \s in {1,2,...,12}{
\draw[#1](90-30*\s:1.85)--(90-30*\s:2);}
\foreach \t in {1,2,...,60}{
\draw[#1](90-6*\t:1.93)--(90-6*\t:2);}}


\begin{document}
\begin{enumerate}


\item
%@ Όνομα: Ακολουθία
%@ Βαθμός: 2
Ακολουθία πραγματικών αριθμών ονομάζεται κάθε συνάρτηση της μορφής $ a:\mathbb{N}^*\rightarrow\mathbb{R} $ όπου κάθε φυσικός αριθμός $ \nu\in\mathbb{N}^* $, εκτός του μηδενός, αντιστοιχεί σε ένα πραγματικό αριθμό $ a(\nu)\in\mathbb{R} $ ή πιο απλά $ a_\nu $.
\begin{itemize}[itemsep=0mm]
\item Η ακολουθία των πραγματικών αριθμών συμβολίζεται $ \left( a_\nu\right)  $.
\item Οι πραγματικοί αριθμοί $ a_1, a_2,\ldots,a_\nu $ ονομάζονται \textbf{όροι} της ακολουθίας.
\item Ο όρος $ a_\nu $ ονομάζεται \textbf{ν-οστός} ή \textbf{γενικός} όρος της ακολουθίας.
\item Οι όροι μιας ακολουθίας μπορούν να δίνονται είτε από 
\begin{itemize}[itemsep=0mm]
\item έναν \textbf{γενικό τύπο} της μορφής $ a_\nu=f(\nu) $, όπου δίνεται κατευθείαν ο γενικός όρος της
\item είτε από \textbf{αναδρομικό τύπο} όπου κάθε όρος δίνεται με τη βοήθεια ενός ή περισσότερων προηγούμενων όρων. Θα είναι της μορφής \[ a_{\nu+i}=f(a_{\nu+i-1},\ldots,a_{\nu+1},a_\nu)\;\;,\;\;a_1,a_2,\ldots,a_i\textrm{ γνωστοί όροι.} \] Στον αναδρομικό τύπο, ο αριθμός $ i\in\mathbb{N} $ είναι το πλήθος των προηγούμενων όρων από τους οποίους εξαρτάται ο όρος $ a_{\nu+i} $. Είναι επίσης αναγκαίο να γνωρίζουμε τις τιμές των $ i $ πρώτων όρων της προκειμένου να υπολογίσουμε τους υπόλοιπους.
\end{itemize}
\item Μια ακολουθία της οποίας όλοι οι όροι είναι ίσοι ονομάζεται \textbf{σταθερή}.
\end{itemize}



\item
%@ Κωδικός: Alg-GrammikaSys-GrafEpil-AA1
%@ Κεφάλαιο: Γραμμικά Συστήματα
%@ Είδος: Γραφική Επίλυση Συστήματος
%@ Δυσκολία: 1
Να λυθούν γραφικά τα παρακάτω γραμμικά συστήματα.
\begin{multicols}{2}
\begin{rlist}[leftmargin=5mm]
\item $ \systeme{x-y=3,3x+y=13} $
\item $ \systeme{2x+y=4,x+4y=8} $
\item $ \systeme{3x-y=2,6x-2y=4} $
\item $ \systeme{x-2y=-3,-2x+4y=5} $
\end{rlist}
\end{multicols}



\item
%@ Κωδικός: Alg-Anis2ou-EpilAnis-AA1
%@ Κεφάλαιο: Ανισώσεις 2ου βαθμού
%@ Είδος: Επίλυση απλής πολυωνυμικής ανίσωσης
%@ Δυσκολία: 1
Να βρεθούν οι λύσεις των παρακάτω ανισώσεων
\begin{multicols}{2}
\begin{alist}
\item $ x^2-3x+2>0 $
\item $ x^2-4x+3<0 $
\item $ 2x^2-5x+3\geq 0 $
\item $ x^2-x-2\leq 0 $
\item $ x^2-6x+5<0 $
\item $ 2x^2-x-1>0 $
\end{alist}
\end{multicols}



\item
%@ Κωδικός: Alg-GrammikaSys-GrafEpil-AA2
%@ Κεφάλαιο: Γραμμικά Συστήματα
%@ Είδος: Γραφική Επίλυση Συστήματος
%@ Δυσκολία: 1
Να βρεθούν, αν υπάρχουν, τα κοινά σημεία των παρακάτω ευθειών.
\begin{rlist}
\item $ x+3y=6 $ και $ 2x+y=8 $
\item $ 3x+4y=5 $ και $ -x+5y=3 $
\item $ 2x-y=10 $ και $ 4x-2y=7 $
\item $ 3x-y=2 $ και $ 6x-2y=4 $
\end{rlist}



\item
%@ Κωδικός: Alg-Anis2ou-EpilAnis-AA2
%@ Κεφάλαιο: Ανισώσεις 2ου βαθμού
%@ Είδος: Επίλυση απλής πολυωνυμικής ανίσωσης
%@ Δυσκολία: 1
Να βρεθούν οι λύσεις των παρακάτω ανισώσεων
\begin{multicols}{2}
\begin{alist}
\item $ -x^2+7x-12>0 $
\item $ -2x^2+3x+2<0 $
\item $ -x^2+2x+15\geq 0 $
\item $ -3x^2-2x+1\leq 0 $
\item $ -x^2-3x>0 $
\item $ -4x^2+x<0 $
\end{alist}
\end{multicols}


\end{enumerate}
\end{document}