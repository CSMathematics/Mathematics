\documentclass[11pt,a4paper,twocolumn]{article}
\usepackage[english,greek]{babel}
\usepackage[utf8]{inputenc}
\usepackage{nimbusserif}
\usepackage[T1]{fontenc}
\usepackage[left=1.50cm, right=1.50cm, top=2.00cm, bottom=2.00cm]{geometry}
\usepackage{amsmath}
\let\myBbbk\Bbbk
\let\Bbbk\relax
\usepackage[amsbb,subscriptcorrection,zswash,mtpcal,mtphrb,mtpfrak]{mtpro2}
\usepackage{graphicx,multicol,multirow,enumitem,tabularx,mathimatika,gensymb,venndiagram,hhline,longtable,tkz-euclide,fontawesome5,eurosym,tcolorbox,tabularray,tikzpagenodes,relsize,siunitx}
\definecolor{ag}{HTML}{A102D1}
\definecolor{bg}{HTML}{BE4267}
\definecolor{gg}{HTML}{E16024}
\definecolor{al}{HTML}{64AA00}
\definecolor{bl}{HTML}{C80000}
\definecolor{gl}{HTML}{3D93CD}
\definecolor{ep}{HTML}{BBC320}
\colorlet{xrwma}{al}
%\definecolor{xrwma}{HTML}{3d93cd}
\usetikzlibrary{calc}
\usetikzlibrary{positioning}
\usepackage[explicit]{titlesec}
\tcbuselibrary{skins,theorems,breakable}
\newlist{rlist}{enumerate}{3}
\setlist[rlist]{itemsep=0mm,label=\roman*.}
\newlist{alist}{enumerate}{3}
\setlist[alist]{itemsep=0mm,label=\alph*.}
\newlist{balist}{enumerate}{3}
\setlist[balist]{itemsep=0mm,label=\bf\alph*.}
\newlist{Alist}{enumerate}{3}
\setlist[Alist]{itemsep=0mm,label=\Alph*.}
\newlist{bAlist}{enumerate}{3}
\setlist[bAlist]{itemsep=0mm,label=\bf\Alph*.}
\newlist{askhseis}{enumerate}{3}
\setlist[askhseis]{label={\Large\thesection}.\arabic*.}
\renewcommand{\textstigma}{\textsigma\texttau}
\newlist{thema}{enumerate}{3}
\setlist[thema]{label=\bf\large{ΘΕΜΑ \textcolor{black}{\Alph*}},itemsep=0mm,leftmargin=0cm,itemindent=18mm}
\newlist{erwthma}{enumerate}{3}
\setlist[erwthma]{label=\bf{\large{\textcolor{black}{\Alph{themai}.\arabic*}}},itemsep=0mm,leftmargin=0.8cm}

\newcommand{\kerkissans}[1]{{\fontfamily{maksf}\selectfont \textbf{#1}}}
\renewcommand{\textdexiakeraia}{}

\usepackage[
backend=biber,
style=alphabetic,
sorting=ynt
]{biblatex}

\DeclareTblrTemplate{caption}{nocaptemplate}{}
\DeclareTblrTemplate{capcont}{nocaptemplate}{}
\DeclareTblrTemplate{contfoot}{nocaptemplate}{}
\NewTblrTheme{mytabletheme}{
\SetTblrTemplate{caption}{nocaptemplate}{}
\SetTblrTemplate{capcont}{nocaptemplate}{}
\SetTblrTemplate{contfoot}{nocaptemplate}{}
}

\NewTblrEnviron{mytblr}
\SetTblrStyle{firsthead}{font=\bfseries}
\SetTblrStyle{firstfoot}{fg=red2}
\SetTblrOuter[mytblr]{theme=mytabletheme}
\SetTblrInner[mytblr]{
rowspec={t{7mm}},columns = {c},
width = 0.85\linewidth,
row{odd} = {bg=red9,fg=black,ht=8mm},
row{even} = {bg=red7,fg=black,ht=8mm},
hlines={white},vlines={white},
row{1} = {bg=xrwma, fg=white, font=\bfseries\fontfamily{maksf}},rowhead = 1,
hline{2} = {.7mm}, % midrule  
}
\newcounter{askhsh}
\setcounter{askhsh}{1}
\newcommand{\askhsh}{{\large\theaskhsh.}\ \addtocounter{askhsh}{1}}

\titleformat{\section}{\Large}{\kerkissans{\thesection}}{10pt}{\Large\kerkissans{#1}}

\setlength{\columnsep}{5mm}
\titleformat{\paragraph}
{\large}%
{}{0em}%
{\textcolor{xrwma}{\faSquare\ \ \kerkissans{\bmath{#1}}}}
\setlength{\parindent}{0pt}


\newcommand{\eng}[1]{\selectlanguage{english}#1\selectlanguage{greek}}



\begin{document}
\twocolumn[{
\begin{tikzpicture}[overlay,remember picture]
\fill[xrwma]($(current page.north west)$)--($(current page.north west)+(2cm,0)$)--($(current page.north west)+(2cm,-2.4cm)$)--($(current page.north west)+(-2cm,-2.4cm)$)--cycle;
\node (fig1) at ($(current page.north west)+(1cm,-1.2cm)$)
       {\includegraphics[width=0.07\linewidth]{/home/spyros/Μαθηματικά/Φροντιστήριο ΦΙΛΟΜΑΘΕΙΑ/Βαθμοί - Φυλλάδιο ύλης - Διάφορα/Λογότυπα/logo.png}};
\node[gray2] at ($(0,0)+(-2.4cm,1.5cm)$) {\kerkissans{Φ\,ΡΟΝΤΙΣΤΗΡΙΟ ΜΕΣΗΣ ΕΚΠΑΙΔΕΥΣΗΣ}};
\node[xrwma] at ($(0,0)+(-2.4cm,.9cm)$) {\kerkissans{\huge {\fontsize{28}{33.6}\selectfont Φ\,ΙΛΟΜΑΘΕΙΑ}}};
\draw[gray4] (-5.2,.4)--(.4,.4);
\node at ($(0,0)+(-2.4cm,0cm)$) {\textcolor{xrwma}{\faMapMarker*} Ιακώβου Πολυλά 24, Πεζόδρομος};
\node (title) at ($(current page.north east)+(-3.5cm,-.7cm)$){\textcolor{gray2}{ \kerkissans{\LARGE Φ\,ΥΛΛΑΔΙΟ ΑΣΚΗΣΕΩΝ}}};
\node[below=of title.east,anchor=east,yshift=3mm] (mob) {\textcolor{xrwma}{\faPhone*} 26610 20144 - \textcolor{xrwma}{\faMobile*\ \faTelegram\ \faViber} 693 232 7283};
\node[below=of mob.east,anchor=east,yshift=4mm] (fb) {\textcolor{xrwma}{\faFacebook} Φροντιστήριο Φιλομάθεια - \textcolor{xrwma}{\faInstagram}\ {\eng{front\_filomatheia}}};
\end{tikzpicture}
\vspace{10mm}\mbox{}\\
\centering
\kerkissans{{\huge Άλγεβρα - Α' Λυκίου}\\\vspace{2mm} {\LARGE Εξισώσεις 2ου βαθμού}}\\\vspace{4mm}{\kerkissans{\today}}\\\vspace{3mm}}]
\paragraph{Ερωτήσεις θεωρίας}
\askhsh 
\begin{alist}
\item Τι ονομάζουμε εξίσωση 2\textsuperscript{ου} βαθμού;
\item Ποιος αριθμός μας δείχνει το πλήθος των ριζών μιας εξίσωσης 2\textsuperscript{ου} βαθμού;
\item Πότε μια εξίσωση έχει 2 ρίζες, πότε μια και πότε είναι αδύνατη στο $ \mathbb{R} $;
\item Να γραφούν οι τύποι του Vieta.
\end{alist}
\askhsh \textbf{Σωστό - Λάθος}\\
Να χαρακτηριστούν οι παρακάτω εξισώσεις ως σωστές (Σ) ή λανθασμένες (Λ).
\begin{alist}[leftmargin=4mm]
\item Αν για μια εξίσωση 2\textsuperscript{ου} βαθμού έχουμε $ \varDelta>0 $ τότε έχει 2 άνισες λύσεις.
\item Αν για μια εξίσωση 2\textsuperscript{ου} βαθμού έχουμε $ \varDelta<0 $ τότε έχει μια διπλή λύση.
\item Η εξίσωση $ ax^2+\beta x+\gamma=0 $ παριστάνει μια εξίσωση 2\textsuperscript{ου} βαθμού για κάθε τιμή του $ a $.
\item Αν $ x_1, x_2 $ είναι οι λύσεις μιας εξίσωσης 2\textsuperscript{ου} βαθμού τότε : $ x_1+x_2=\frac{\beta}{a} $ και $ x_1\cdot x_2=\frac{\gamma}{a} $.
\item Αν $ x_1, x_2 $ είναι οι λύσεις μιας εξίσωσης 2\textsuperscript{ου} βαθμού με $ x_1=-x_2 $ τότε $ \beta=0 $.
\end{alist}
\paragraph{Επίλυση εξισώσεων}
\askhsh
Να λυθούν οι παρακάτω εξισώσεις.
\begin{multicols}{2}
\begin{alist}[leftmargin=5mm]
\item $ x^2-5x+6=0 $
\item $ x^2-3x+2=0 $
\item $ x^2-7x+12=0 $
\item $ x^2+3x-4=0 $
\item $ x^2-6x+8=0 $
\item $ x^2-6x+5=0 $
\item $ 2x^2-5x+3=0 $
\item $ 2x^2-9x+10=0 $
\item $ 3x^2-x-4=0 $
\end{alist}
\end{multicols}
\askhsh
Να λυθούν οι παρακάτω εξισώσεις.
\begin{multicols}{2}
\begin{alist}[leftmargin=5mm]
\item $ x^2-4x+4=0 $
\item $ x^2-6x+9=0 $
\item $ x^2-10x+25=0 $
\item $ 4x^2+4x+1=0 $
\item $ 36x^2+12x+1=0 $
\item $ 4x^2+12x+9=0 $
\end{alist}
\end{multicols}
\askhsh
Να λυθούν οι παρακάτω εξισώσεις.
\begin{multicols}{2}
\begin{alist}
\item $ x^2+x+4=0 $
\item $ x^2+3x+12=0 $
\item $ 2x^2-3x+8=0 $
\item $ 2x^2+x+5=0 $
\end{alist}
\end{multicols}
\askhsh
Να λυθούν οι παρακάτω εξισώσεις.
\begin{alist}[itemsep=0mm]
\item $ x^2-\left(\sqrt{2}-1\right)x-\sqrt{2}=0 $
\item $ x^2-\left(\sqrt{3}+1\right)x+\sqrt{3}=0 $
\item $ x^2+\left(\sqrt{3}+\sqrt{5}\right)x+\sqrt{15}=0 $
\item $ x^2-\left(\sqrt{8}-\sqrt{2}\right)x-4=0 $
\end{alist}
\paragraph{Τύποι \eng{Vieta}}
\askhsh Για καθεμία από τις παρακάτω εξισώσεις να υπολογίσετε το άθροισμα $S$ και το γινόμενο $P$ των λύσεων, εφόσον υπάρχουν.
\begin{multicols}{2}
\begin{alist}
\item $x^2-3x+1=0$
\item $x^2-4x+4=0$
\item $2x^2+x-4=0$
\item $x^2+x+\dfrac{1}{4}=0$
\item $x^2-2x+3=0$
\item $9x^2+6x+1=0$
\item $\dfrac{x^2}{2}-x-3=0$
\end{alist}
\end{multicols}
\askhsh Για καθεμία από τις παρακάτω εξισώσεις να υπολογίσετε το άθροισμα $S$ και το γινόμενο $P$ των λύσεων, εφόσον υπάρχουν.
\begin{alist}
\item $x^2-\left(\sqrt{2}+1\right)x+\sqrt{2}=0$
\item $x^2-3\sqrt{2}x+4=0$
\item $\sqrt{2}x^2+\sqrt{18}x-\sqrt{8}=0$
\end{alist}
\askhsh Να βρεθεί η εξίσωση 2ου βαθμού η οποία έχει λύσεις τους αριθμούς $x_1,x_2$
\begin{alist}
\begin{multicols}{2}
\item $x_1=3$ και $x_2=4$
\item $x_1=-2$ και $x_2=4$
\item $x_1=1$ και $x_2=-5$
\item $x_1=-4$ και $x_2=-1$
\item $x_1=\sqrt{8}$ και $x_2=\sqrt{2}$
\item $x=4$ διπλή λύση
\item $ x_1=3 $ και $ x_2=5 $
\item $ x_1=-2 $ και $ x_2=-4 $
\item $ x_1=\frac{1}{2} $ και $ x_2=-\frac{3}{4} $
\item $ x_1=\sqrt{2} $ και $ x_2=3 $
\end{multicols}
\item $x_1=1-\sqrt{3}$ και $x_2=1+\sqrt{3}$
\item $x_1=\dfrac{3+\sqrt{2}}{4}$ και $x_2=\dfrac{3-\sqrt{2}}{4}$
\end{alist}
\askhsh
Να βρεθούν οι λύσεις $ x_1, x_2 $, μιας εξίσωσης 2{ου} βαθμού, οι οποίες έχουν άθροισμα $ S $ και γινόμενο $ P $ με:
\begin{multicols}{2}
\begin{alist}[leftmargin=5mm]
\item $ S=9 $ και $ P=-10 $
\item $ S=-7 $ και $ P=12 $
\item $ S=6 $ και $ P=9 $
\item $ S=0 $ και $ P=4 $
\item $ S=12 $ και $ P=0 $
\item $ S=8 $ και $ P=-8 $
\end{alist}
\end{multicols}
\askhsh Να βρεθούν, εάν υπάρχουν, αριθμοί $x_1,x_2$ οι οποίοι έχουν
\begin{alist}
\item άθροισμα $4$ και γινόμενο $-5$
\item άθροισμα $-3$ και γινόμενο $-10$
\item άθροισμα $7$ και γινόμενο $6$
\item άθροισμα $4$ και γινόμενο $4$
\item άθροισμα $1$ και γινόμενο $3$
\item άθροισμα $-2$ και γινόμενο $-8$
\item άθροισμα $\dfrac{3}{2}$ και γινόμενο $\dfrac{1}{2}$
\item άθροισμα $3$ και γινόμενο $-5$
\end{alist}
\askhsh 
Αν $ x_1, x_2 $ είναι οι λύσεις μιας εξίσωσης 2{ου} βαθμού, να βρεθεί η εξίσωση, αν γι αυτήν ισχύει
\begin{alist}
\item $ x_1+x_2=7 $ και $ x_1\cdot x_2=6 $
\item $ x_1+x_2=8 $ και $ x_1\cdot x_2=12 $
\item $ x_1+x_2=-3 $ και $ x_1\cdot x_2=-28 $
\item $ x_1+x_2=-12 $ και $ x_1\cdot x_2=20 $
\end{alist}

\askhsh Δίνεται η εξίσωση $x^2-5x+3=0$. Αν $x_1,x_2$ είναι οι λύσεις της, τότε να βρείτε τις τιμές των παραστάσεων.
\begin{multicols}{2}
\begin{alist}
\item $x_1+x_2$
\item $x_1\cdot x_2$
\item $x_1^2+x_2^2$
\item $x_1^3+x_2^3$
\item $\dfrac{1}{x_1}+\dfrac{1}{x_2}$
\end{alist}
\end{multicols}
\askhsh Δίνεται η εξίσωση $x^2+2x-5=0$. Αν $x_1,x_2$ είναι οι λύσεις της, τότε να βρείτε τις τιμές των παραστάσεων.
\begin{multicols}{2}
\begin{alist}
\item $x_1+x_2$
\item $x_1\cdot x_2$
\item $x_1^2x_2+x_2x_2^2$
\item $ x_1^2+2x_1x_2+x_2^2 $
\item $\dfrac{x_2}{x_1}+\dfrac{x_1}{x_2}$
\end{alist}
\end{multicols}

\askhsh
Να βρεθεί εξίσωση 2ου βαθμού, με λύσεις $ x_1, x_2 $ για τις οποίες ισχύουν οι παρακάτω σχέσεις:
\begin{alist}
\item $ x_1+x_2=4 $ και $ x_1\cdot x_2=3 $
\item $ x_1^2+x_2^2=-8 $ και $ x_1\cdot x_2=2 $
\item $ x_1+x_2=3 $ και $ x_1^2x_2+x_1x_2^2=6 $
\item $ x_1^2+x_2^2=29 $ και $ (x_1+x_2)^2=49 $
\end{alist}
\paragraph{Εξισώσεις που ανάγονται σε 2ου βαθμού}
\askhsh
Να λυθούν οι παρακάτω εξισώσεις.
\begin{multicols}{2}
\begin{alist}[leftmargin=5mm]
\item $ x^2-5\left|x\right|+6=0 $
\item $ x^2-4\left|x\right|+3=0 $
\item $ x^2-2\left|x\right|-3=0 $
\item $ x^2+7\left|x\right|+10=0 $
\item $ 2x^2-\left|x\right|-10=0 $
\item $ x^2-10\left|x\right|+25=0 $
\end{alist}
\end{multicols}
\askhsh
Να λυθούν οι παρακάτω εξισώσεις.
\begin{multicols}{2}
\begin{alist}[leftmargin=5mm]
\item $ x^4-5x^2+6=0 $
\item $ x^4-4x^2+3=0 $
\item $ x^4-6x^2+9=0 $
\item $ x^6-2x^3-15=0 $
\item $ 2x^4-x^2-10=0 $
\item $ x^8-10x^4+9=0 $
\end{alist}
\end{multicols}
\askhsh
Να λυθούν οι παρακάτω εξισώσεις.
\begin{alist}
\item $ \dfrac{x-3}{x}+\dfrac{x}{x-1}=\dfrac{3-x}{x^2-x} $
\item $ \dfrac{2 x-1}{x-2}+\dfrac{x-1}{x-1}=\dfrac{3-2 x}{x^2-3 x+2} $
\item $ \dfrac{x+4}{x^2-4}+\dfrac{2 x+1}{x-2}=\dfrac{x-3}{x+2} $
\item $ \dfrac{x+4}{x^2-2 x}+2=\dfrac{x-2}{x} $
\end{alist}
\askhsh
Να λυθούν οι παρακάτω εξισώσεις.
\begin{alist}
\item $ \left(x-1\right)^2-5\left|x-1\right|+6=0 $
\item $ \left(2x-3\right)^2-7\left|2x-3\right|+12=0 $
\item $ \left(x-2\right)^4-13\left(x-2\right)^2+36=0 $
\item $ \left(x+3\right)^6+19\left(x+3\right)^3-216=0 $
\end{alist}
\askhsh
Να λυθούν οι παρακάτω εξισώσεις.
\begin{alist}
\item $ \left(x^2-x\right)^2+4\left|x^2-x\right|-12=0 $
\item $ \left(x+\frac{1}{x}\right)^2-7\left(x+\frac{1}{x}\right)+10=0 $
\item $ \left(x^3-2\right)^2+19\left(x^3-2\right)-150=0 $
\item $ \left(|x|-3\right)^2+8\left(|x|-3\right)-12=0 $
\end{alist}
\paragraph{Παραμετρικές}
\askhsh
Δίνεται η παρακάτω εξίσωση 2{ου} βαθμού
\[ x^2+(\lambda-2) x+2\lambda^2=0 \]
όπου $ \lambda\in\mathbb{R} $ είναι μια τυχαία παράμετρος.
\begin{alist}
\item Να βρεθεί η διακρίνουσα της εξίσωσης.
\item Να βρεθούν οι τιμές τις παραμέτρου $ \lambda $ ώστε η εξίσωση να έχει δύο άνισες λύσεις.
\item Να βρεθούν οι τιμές τις παραμέτρου $ \lambda $ ώστε η εξίσωση να έχει μια διπλή λύση.
\item Για ποίες τιμές τις παραμέτρου $ \lambda $ είναι αδύνατη η εξίσωση;
\end{alist}
\askhsh Δίνεται η παρακάτω εξίσωση 2{ου} βαθμού
\[ x^2+3\lambda x+2\lambda^2-\lambda=0 \]
όπου $ \lambda\in\mathbb{R} $ είναι μια τυχαία παράμετρος.
\begin{alist}
\item Να βρεθεί η διακρίνουσα της εξίσωσης.
\item Να βρεθούν οι τιμές τις παραμέτρου $ \lambda $ ώστε η εξίσωση να έχει μια διπλή λύση.
\end{alist}
\askhsh
Να δειχθεί ότι η εξίσωση
\[ x^2+x-\lambda^2=0 \]
έχει 2 άνισες λύσεις για κάθε τιμή του $ \lambda\in\mathbb{R} $.\\\\
\askhsh
Να δειχτεί ότι η εξίσωση
\[ ax^2+(a-1)x-1=0 \]
\begin{alist}
\item έχει λύσεις για κάθε $ a\in\mathbb{R}^* $.
\item έχει μια διπλή λύση για $ a=-1 $.
\end{alist}
\askhsh
Να δειχτεί ότι η εξίσωση
\[ x^2+(a-3) x+a^2+4=0 \]
δεν έχει λύσεις για καμία τιμή του $ a $.\\\\
\askhsh
Να βρεθεί η τιμή της παραμέτρου $ a\in\mathbb{R}^* $ έτσι ώστε η εξίσωση
\[ 2ax^2+(a-4)x+a+2=0 \]
να έχει μια διπλή ρίζα.
\askhsh
Να βρεθούν οι σταθερές $ a,\beta\in\mathbb{R}^* $ έτσι ώστε η εξίσωση
\[ ax^2+(2a-3\beta)x+(a-\beta+2)=0 \] 
να έχει λύσεις τις $ x_1=-2, x_2=1 $.\\\\
\askhsh
Να βρεθούν οι σταθερές $ a, \beta\in\mathbb{R} $ έτσι ώστε η εξίσωση
 \[ x^2+2(\beta-1)x+a+\beta^2-7=0 \] 
 να έχει μια διπλή λύση τη $ x=-2$.\\\\
\askhsh
Να βρεθούν οι σταθερές $ a, \beta\in\mathbb{R} $ έτσι ώστε η εξίσωση
 \[ x^2+(a+3\beta-2)x+4a-2a\beta-2=0 \]
να έχει λύσεις τις $ x_1=4-2a $ και $ x_2=\beta-3 $.\\\\
\askhsh
Αν $ x_1,x_2 $ είναι οι λύσεις της εξίσωσης
\[ x^2-7x+8=0 \]
τότε χωρίς αυτή να λυθεί, να υπολογίσετε τις παρακάτω παραστάσεις.
\begin{multicols}{2}
\begin{alist}
\item $ x_1+x_2 $
\item $ x_1x_2 $
\item $ x_1^2+x_2^2 $
\item $ x_1^2x_2+x_1x_2^2 $
\item $ x_1^3+x_2^3 $
\item $ \frac{1}{x_1}+\frac{1}{x_2} $
\end{alist}
\end{multicols}
\askhsh
Αν $ x_1,x_2 $ είναι οι λύσεις της εξίσωσης \[ x^2-4x+3=0 \] τότε να βρεθεί να βρεθεί η εξίσωση η οποία έχει λύσεις τις $ y_1=2x_1+x_2 $ και $ y_2=x_1-3x_2 $.
\paragraph{Προβλήματα}
\askhsh
Να βρεθεί η τιμή της μεταβλητής $ x $ για την οποία το εμβαδόν του παρακάτω σχήματος ισούται με $ E=84m^2 $.
\begin{center}
\begin{tikzpicture}
\draw  (-2.5,1.5) rectangle (1,-0.5);
\node at (-.75,1.75) {$x$};
\node at (-3.2,0.5) {$x-5$};
\node at (1.5,0.5) {};
\node at (-.75,0.5) {$E=84m^2$};
\end{tikzpicture}
\end{center}
\askhsh Δίνεται ορθογώνιο παραλληλόγραμμο με περίμετρο $24\si{cm}$ και εμβαδόν $32\si{cm}^2$. Να υπολογίσετε τις διαστάσεις του ορθογωνίου.
\end{document}

