\documentclass[11pt,a4paper,twocolumn]{article}
\usepackage[english,greek]{babel}
\usepackage[utf8]{inputenc}
\usepackage{nimbusserif}
\usepackage[T1]{fontenc}
\usepackage[left=1.50cm, right=1.50cm, top=2.00cm, bottom=2.00cm]{geometry}
\usepackage{amsmath}
\let\myBbbk\Bbbk
\let\Bbbk\relax
\usepackage[amsbb,subscriptcorrection,zswash,mtpcal,mtphrb,mtpfrak]{mtpro2}
\usepackage{graphicx,multicol,multirow,enumitem,tabularx,mathimatika,gensymb,venndiagram,hhline,longtable,tkz-euclide,fontawesome5,eurosym,tcolorbox,tabularray,changepage,lipsum}
\definecolor{xrwma}{HTML}{C90000}
\usepackage[explicit]{titlesec}
\tcbuselibrary{skins,theorems,breakable}
\newlist{rlist}{enumerate}{3}
\setlist[rlist]{itemsep=0mm,label=\roman*.}
\newlist{alist}{enumerate}{3}
\setlist[alist]{itemsep=0mm,label=\alph*.}
\newlist{balist}{enumerate}{3}
\setlist[balist]{itemsep=0mm,label=\bf\alph*.}
\newlist{Alist}{enumerate}{3}
\setlist[Alist]{itemsep=0mm,label=\Alph*.}
\newlist{bAlist}{enumerate}{3}
\setlist[bAlist]{itemsep=0mm,label=\bf\Alph*.}
\newlist{askhseis}{enumerate}{3}
\setlist[askhseis]{label={\Large\thesection}.\arabic*.}
\renewcommand{\textstigma}{\textsigma\texttau}
\newlist{thema}{enumerate}{3}
\setlist[thema]{label=\bf\large{ΘΕΜΑ \textcolor{black}{\Alph*}},itemsep=0mm,leftmargin=0cm,itemindent=18mm}
\newlist{erwthma}{enumerate}{3}
\setlist[erwthma]{label=\bf{\large{\textcolor{black}{\Alph{themai}.\arabic*}}},itemsep=0mm,leftmargin=0.8cm}

\newcommand{\kerkissans}[1]{{\fontfamily{maksf}\selectfont \textbf{#1}}}
\renewcommand{\textdexiakeraia}{}

\usepackage[
backend=biber,
style=alphabetic,
sorting=ynt
]{biblatex}

\DeclareTblrTemplate{caption}{nocaptemplate}{}
\DeclareTblrTemplate{capcont}{nocaptemplate}{}
\DeclareTblrTemplate{contfoot}{nocaptemplate}{}
\NewTblrTheme{mytabletheme}{
\SetTblrTemplate{caption}{nocaptemplate}{}
\SetTblrTemplate{capcont}{nocaptemplate}{}
\SetTblrTemplate{contfoot}{nocaptemplate}{}
}

\NewTblrEnviron{mytblr}
\SetTblrStyle{firsthead}{font=\bfseries}
\SetTblrStyle{firstfoot}{fg=red2}
\SetTblrOuter[mytblr]{theme=mytabletheme}
\SetTblrInner[mytblr]{
rowspec={t{7mm}},columns = {c},
width = 0.85\linewidth,
row{odd} = {bg=red9,fg=black,ht=8mm},
row{even} = {bg=red7,fg=black,ht=8mm},
hlines={white},vlines={white},
row{1} = {bg=red4, fg=white, font=\bfseries\fontfamily{maksf}},rowhead = 1,
hline{2} = {.7mm}, % midrule  
}
\newcounter{askhsh}
\setcounter{askhsh}{1}
\newcommand{\askhsh}{{\large\theaskhsh.}\ \addtocounter{askhsh}{1}}

\titleformat{\section}{\Large}{\kerkissans{\thesection}}{10pt}{\Large\kerkissans{#1}}

\setlength{\columnsep}{5mm}
\titleformat{\paragraph}
{\large}%
{}{0em}%
{\textcolor{xrwma}{\faSquare\ \ \kerkissans{\bmath{#1}}}}
\setlength{\parindent}{0pt}

\newcommand{\eng}[1]{\selectlanguage{english}#1\selectlanguage{greek}}
\newcommand{\scfont}{\fontfamily{artemisia}\selectfont}
\newcommand{\titlefont}{\fontfamily{txr}\selectfont}

\begin{document}
\twocolumn[{
\centering
{\scfont \textsc{\textcolor{xrwma}{Σπυρος Φρονιμος }- Μαθηματικος}}\\
{\faEnvelope[regular] : \eng{spyrosfronimos@gmail.com\ \  \raisebox{-.4ex}{\rule{0.2mm}{5mm}} \ \faMobile*\ \faTelegramPlane\ \faViber\ : 693 232 7283}}\\
\rule{12cm}{.2mm}\\\vspace{2mm}
{ΑΣΚΗΣΕΙΣ - ΠΡΟΒΛΗΜΑΤΑ}\\
\textbf{\today}\\\vspace{4mm}
{\Large \textbf{\titlefont Άλγεβρα - Α' Λυκείου}}\\\vspace{2mm}
{\huge \textcolor{xrwma}{\textbf{\titlefont Πραγματικοί αριθμοί}}}\\\vspace{2mm}
{\Large \textbf{\titlefont Απόλυτη τιμή}}\vspace{3mm}
}]
\paragraph{Αριθμητικές παραστάσεις}
\askhsh Υπολογίστε τις τιμές των παρακάτω αριθμιτικών παραστάσεων.
\begin{alist}
\item $Α=|-2|-|4|+|-5|$
\item $Β=|-7|-|-3|+|8|$
\item $\varGamma=|-3|(|-4+9|-1)+|10|$
\item $\varDelta=\left(|15|:|-3|+7\right)\cdot\left(|-11+5|-3\right)$
\end{alist}
\askhsh Βρείτε τις τιμές των παρακάτω αριθμητικών παραστάσεων.
\begin{alist}
\begin{multicols}{2}
\item $A=|\sqrt{2}-1|$
\item $B=|2-\sqrt{5}|$
\item $\varGamma=|\sqrt{3}-\sqrt{2}|$
\item $\varDelta=|\pi-3|$
\item $E=|\pi-4|$
\item $Z=|5-\pi|$
\end{multicols}
\end{alist}
\paragraph{Απλοποίηση αλγεβρικών παραστάσεων}
\askhsh Απλοποιήστε τις παρακάτω παραστάσεις.
\begin{alist}
\item $A=\left|x^2+3\right|$
\item $B=\left|x^2-2x+2\right|$
\item $\varGamma=\left|-2-x^4\right|$
\item $\varDelta=\left|x^2+1\right|+\left|-4-x^2\right|$
\end{alist}
\askhsh Δίνεται ο πραγματικός αριθμός $a$ με $a>1$. Να απλοποιήσετε τις επόμενες παραστάσεις.
\begin{alist}
\item $A=|a-1|-|a|$
\item $B=|1-a|+2a+3$
\item $\varGamma=|2a-2|+3|a|-4$
\end{alist}
\askhsh Αν ισχύει $1<a<4$ να απλοποιήσετε τις ακόλουθες παραστάσεις.
\begin{alist}
\item $A=|a-1|+|4-a|-3$
\item $B=|a-4|+|a-5|+a$
\item $\varGamma=|2a-8|-|2-2a|$
\item $\varDelta=|2a|-|1-a|+|a+1|$
\end{alist}
\askhsh Δίνονται οι πραγματικοί αριθμοί $a,\beta$ για τους οποίους ισχύει $a<3<\beta$.
\begin{alist}
\item Να γράψετε χωρίς απόλυτες τιμές την παράσταση
\[ A=|a-3|+|\beta-3|+a\beta-1 \]
\item Να αποδείξετε ότι $A>0$.
\end{alist}
\askhsh Να γράψετε καθεμία από τις παρακάτω παραστάσεις χωρίς απόλυτη τιμή.
\begin{alist}
\item $A=|x-3|+4$
\item $B=|x-2|+2x-1$
\item $\varGamma=x+|4-x|$
\item $\varDelta=3-|2x-1|$
\item $E=3x-2-|x+2|$
\item $Z=|2x+2|-3x+4$
\end{alist}
\askhsh Γράψτε καθεμία από τις παρακάτω παραστάσεις χωρίς απόλυτη τιμή.
\begin{alist}
\item $Α=|x-1|+|3-x|$
\item $B=|x+2|-|x-4|$
\item $\varGamma=|x+1|+|4-x|$
\item $\varDelta=|3-x|+|1-x|$
\item $E=|2x-1|+|x|-|9-3x|$
\end{alist}
\askhsh Απλοποιήστε τις επόμενες αλγεβρικές παραστάσεις.
\begin{alist}
\item $A=\sqrt{x^2+4x+4}$
\item $B=\dfrac{\sqrt{x^2+2x+1}}{x+1}$
\item 
\end{alist}
\paragraph{Ιδιότητες απόλυτων τιμών}
\askhsh 
\begin{alist}
\item 
\end{alist}
\askhsh Σε καθεμιά από τις παρακάτω σχέσεις, να προσδιορίσετε τους πραγματικούς αριθμούς $x,y$.
\begin{alist}
\item $|x-2|+|y+3|=0$
\item $|x^2-4x|+|x^2-16|=0$
\item $|y^2+3y-4|+|y^2-1|=0$
\item $|x^2-4x+4|+|y^2+2y+1|=0$
\item $\sqrt{x^2-2x+1}+\sqrt{y^2+6y+9}=0$
\end{alist}
\paragraph{Μήκος - κέντρο - ακτίνα διαστήματος}
\askhsh Να βρεθούν το μήκος, το κέντρο και η ακτίνα των παρακάτω διαστημάτων.
\begin{multicols}{3}
\begin{alist}
\item $ [1,5] $
\item $ (-2,4) $
\item $ [-10,-1) $
\item $ (0,8] $
\item $ \left(\frac{1}{2},\frac{5}{4}\right) $
\item $ \left[\frac{3}{8},2\right] $
\end{alist}
\end{multicols}
\askhsh Το κέντρο του διαστήματος $ [1,\lambda] $ είναι το $ 4 $. Να βρεθεί 
\begin{alist}
\item ο πραγματικός αριθμός $ \lambda\in\mathbb{R} $. 
\item το μήκος και η ακτίνα του διαστήματος.
\end{alist}
\askhsh Το μήκος του διαστήματος $ [\lambda-1,\lambda^2] $ ισούται με $ 3 $, όπου $ \lambda\in\mathbb{R} $.
\begin{alist}
\item Να βρεθεί η τιμή του $ \lambda $.
\item Για κάθε τιμή της παραμέτρου $\lambda$ που βρήκατε στο προηγούμενο ερώτημσ, να βρείτε το κέντρο και την ακτίνα του διαστήματος.
\end{alist}
\askhsh Η ακτίνα του διαστήματος $ [\lambda-1,3] $ είναι $ 4 $.
\begin{alist}
\item Να βρεθεί η τιμή του $ \lambda $.
\item Να βρεθεί το μήκος και το κέντρο του διαστήματος.
\end{alist}
\askhsh Το διάστημα $ [2\lambda+3,2-\lambda] $, όπου $ \lambda\in\mathbb{R} $, έχει αντίθετα άκρα. Να βρεθούν 
\begin{alist}
\item η τιμή της παραμέτρου $ \lambda $.
\item το κέντρο, το μήκος και η ακτίνα του διαστήματος.
\end{alist}
\askhsh Συμπληρώστε τον ακόλουθο πίνακα.
\begin{center}
\begin{mytblr}{}
Απόλυτη τιμή & Απόσταση & Διάστημα\\
$|x-2|<3$ & & \\
 & $d(x,4)\leq 5$ &  \\
$|x+1|\geq 3$ & & \\
 & $d(x,-2)>1$ & \\
 &  & $[-3,5]$\\
 &  & $(-\infty,2)\cup(4,+\infty)$ \\
 &  & $(-2,8)$ \\
 &  & $(-\infty,-1]\cup[5,+\infty)$
\end{mytblr}
\end{center}
\end{document}
