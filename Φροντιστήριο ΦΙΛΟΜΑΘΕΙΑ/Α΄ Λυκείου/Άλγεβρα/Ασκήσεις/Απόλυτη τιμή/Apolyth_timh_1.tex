\documentclass[11pt,a4paper]{article}
\usepackage[utf8]{inputenc}
\usepackage{nimbusserif}
\usepackage[T1]{fontenc}
\usepackage[english,greek]{babel}
\usepackage{amsmath} 
\let\myBbbk\Bbbk 
\let\Bbbk\relax 
\usepackage[amsbb,subscriptcorrection,zswash,mtpcal,mtphrb,mtpfrak]{mtpro2}
\usepackage[left=2.00cm, right=2.00cm, top=2.00cm, bottom=2.00cm]{geometry}
%------TIKZ - ΣΧΗΜΑΤΑ - ΓΡΑΦΙΚΕΣ ΠΑΡΑΣΤΑΣΕΙΣ ---- 
\usepackage{tikz,pgfplots,tkz-tab} 
\usepackage{tkz-euclide} 
\usepackage[framemethod=TikZ]{mdframed} 
\usetikzlibrary{decorations.pathreplacing} 
\tkzSetUpPoint[size=2.9,fill=white]
%----------------------- 
\usepackage{calc,tcolorbox} 
\tcbuselibrary{skins,theorems,breakable} 
\usepackage{hhline} 
\usepackage[explicit]{titlesec} 
\usepackage{graphicx} 
\usepackage{multicol} 
\usepackage{multirow} 
\usepackage{tabularx} 
\usetikzlibrary{backgrounds} 
\usepackage{sectsty} 
\sectionfont{\centering} 
\usepackage{enumitem} 
\usepackage{adjustbox} 
\usepackage{mathimatika,gensymb,eurosym,wrap-rl} 
\usepackage{systeme,regexpatch} 
%-------- ΜΑΘΗΜΑΤΙΚΑ ΕΡΓΑΛΕΙΑ --------- 
\usepackage{mathtools} 
%---------------------- 
%-------- ΠΙΝΑΚΕΣ --------- 
\usepackage{booktabs} 
%---------------------- 
%----- ΥΠΟΛΟΓΙΣΤΗΣ ---------- 
\usepackage{calculator} 
%---------------------------- 
%------------------------------------------ 
\newcommand{\tss}[1]{\textsuperscript{#1}} 
\newcommand{\tssL}[1]{\MakeLowercase{\textsuperscript{#1}}} 
\tikzstyle{pl}=[line width=0.3mm] 
\tikzstyle{plm}=[line width=0.4mm] 
\usepackage{etoolbox} 
\makeatletter 
\renewrobustcmd{\anw@true}{\let\ifanw@\iffalse} 
\renewrobustcmd{\anw@false}{\let\ifanw@\iffalse}\anw@false 
\newrobustcmd{\noanw@true}{\let\ifnoanw@\iffalse} 
\newrobustcmd{\noanw@false}{\let\ifnoanw@\iffalse}\noanw@false 
\renewrobustcmd{\anw@print}{\ifanw@\ifnoanw@\else\numer@lsign\fi\fi} 
\makeatother
\newlist{alist}{enumerate}{3}
\setlist[alist]{itemsep=0mm,label=\alph*.}
\newlist{rlist}{enumerate}{3}
\setlist[rlist]{itemsep=0mm,label=\roman*.}
\newlist{balist}{enumerate}{3}
\setlist[balist]{itemsep=0mm,label=\bf\alph*.}
\newlist{Alist}{enumerate}{3}
\setlist[Alist]{itemsep=0mm,label=\Alph*.}
\newlist{bAlist}{enumerate}{3}
\setlist[bAlist]{itemsep=0mm,label=\bf\Alph*.}
\renewcommand{\textstigma}{\textsigma\texttau}
\makeatletter
\xpatchcmd{\tkzTabLine}
{\node at (Z\thetkz@cnt@impair\thetkz@cnt@lg){$0$};} % search
{\node[fill=white,inner sep=.5mm] at (Z\thetkz@cnt@impair\thetkz@cnt@lg){$0$};} % replace
{}{}
\makeatother
\newcommand{\en}[1]{\selectlanguage{english}{#1}\selectlanguage{greek}}
\newcommand{\roloi}[4][]{
\draw[line width=.5mm,#1] (0,0) circle(2);
\foreach \n in {1,2,...,12}{
\tkzDefPoint(30*\n-90:2){A_\n}
%\tkzDrawPoint(A_\n)
\node at (-30*\n+90:1.65){\n};}
\draw[plm,,#1] (0,0)--(90-30*#2-0.5*#3:1);
\draw[pl,#1] (0,0)--(90-6*#3-0.1*#4:1.5);
\draw[#1](0,0)--(90-6*#4:1.2);
\tkzDrawPoint[fill=#1,color=#1](0,0)
\foreach \s in {1,2,...,12}{
\draw[#1](90-30*\s:1.85)--(90-30*\s:2);}
\foreach \t in {1,2,...,60}{
\draw[#1](90-6*\t:1.93)--(90-6*\t:2);}}


\begin{document}




%@ Κωδικός: Alg-ApolTimh-MhkKenAkt-AA1
%@ Κεφάλαιο: Απόλυτη τιμή
%@ Είδος: Μήκος - κέντρο και ακτίνα διαστήματος
%@ Δυσκολία: 1
Να βρεθούν το μήκος, το κέντρο και η ακτίνα των παρακάτω διαστημάτων.
\begin{multicols}{2}
\begin{alist}
\item $ [1,5] $
\item $ (-2,4) $
\item $ [-10,-1) $
\item $ (0,8] $
\item $ \left(\frac{1}{2},\frac{5}{4}\right) $
\item $ \left[\frac{3}{8},2\right] $
\end{alist}
\end{multicols}



%@ Κωδικός: Alg-ApolTimh-MhkKenAkt-AA2
%@ Κεφάλαιο: Απόλυτη τιμή
%@ Είδος: Μήκος - κέντρο και ακτίνα διαστήματος
%@ Δυσκολία: 1
Το κέντρο του διαστήματος $ [1,\lambda] $ είναι το $ 4 $. Να βρεθεί ο πραγματικός αριθμός $ \lambda $ καθώς και το μήκος και η ακτίνα του διαστήματος.




%@ Κωδικός: Alg-ApolTimh-MhkKenAkt-AA3
%@ Κεφάλαιο: Απόλυτη τιμή
%@ Είδος: Μήκος - κέντρο και ακτίνα διαστήματος
%@ Δυσκολία: 1
Το μήκος του διαστήματος $ [\lambda-1,\lambda^2] $ είναι $ 3 $, όπου $ \lambda\in\mathbb{R} $.
\begin{alist}
\item Να βρεθεί η τιμή του $ \lambda $.
\item Να βρεθεί το κέντρο και η ακτίνα του διαστήματος.
\end{alist}



%@ Κωδικός: Alg-ApolTimh-MhkKenAkt-AA4
%@ Κεφάλαιο: Απόλυτη τιμή
%@ Είδος: Μήκος - κέντρο και ακτίνα διαστήματος
%@ Δυσκολία: 1
Η ακτίνα του διαστήματος $ [\lambda-1,3] $ είναι $ 4 $.
\begin{alist}
\item Να βρεθεί η τιμή του $ \lambda $.
\item Να βρεθεί το μήκος και το κέντρο του διαστήματος.
\end{alist}



%@ Κωδικός: Alg-ApolTimh-MhkKenAkt-AA5
%@ Κεφάλαιο: Απόλυτη τιμή
%@ Είδος: Μήκος - κέντρο και ακτίνα διαστήματος
%@ Δυσκολία: 1
Το διάστημα $ [2\lambda+3,2-\lambda] $, όπου $ \lambda\in\mathbb{R} $, έχει αντίθετα άκρα. Να βρεθούν 
\begin{alist}
\item η τιμή της παραμέτρου $ \lambda $.
\item το κέντρο, το μήκος και η ακτίνα του διαστήματος.
\end{alist}



\end{document}