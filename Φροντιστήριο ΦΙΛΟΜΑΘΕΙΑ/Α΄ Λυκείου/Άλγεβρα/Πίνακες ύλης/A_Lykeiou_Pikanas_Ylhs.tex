\documentclass[twoside,nofonts,internet,math,spyros]{frontisthrio}
\usepackage[amsbb,subscriptcorrection,zswash,mtpcal,mtphrb,mtpfrak]{mtpro2}
\usepackage[no-math,cm-default]{fontspec}
\usepackage{amsmath}
\usepackage{xunicode}
\usepackage{xgreek}
\let\hbar\relax
\defaultfontfeatures{Mapping=tex-text,Scale=MatchLowercase}
\setmainfont[Mapping=tex-text,Numbers=Lining,Scale=1.0,BoldFont={Minion Pro Bold}]{Minion Pro}
\newfontfamily\scfont{GFS Artemisia}
\font\OnPar="Century Gothic Bold" at 10pt
\usepackage{fontawesome5}
\newfontfamily{\FA}{fontawesome.otf}
\usepackage[most]{tcolorbox}
\xroma{red!70!black}
%------TIKZ - ΣΧΗΜΑΤΑ - ΓΡΑΦΙΚΕΣ ΠΑΡΑΣΤΑΣΕΙΣ ----
\usepackage{tikz,pgfplots}
\usepackage{tkz-euclide}
\usetkzobj{all}
\usepackage[framemethod=TikZ]{mdframed}
\usetikzlibrary{decorations.pathreplacing}
\tkzSetUpPoint[size=7,fill=white]
%-----------------------
\usepackage{calc,tcolorbox}
\tcbuselibrary{skins,theorems,breakable}
\usepackage{hhline}
\usepackage[explicit]{titlesec}
\usepackage{graphicx}
\usepackage{multicol}
\usepackage{multirow}
\usepackage{tabularx}
\usetikzlibrary{backgrounds}
\usepackage{sectsty}
\sectionfont{\centering}
\usepackage{enumitem}
\usepackage{adjustbox}
\usepackage{mathimatika,gensymb,eurosym,wrap-rl}
\usepackage{systeme,regexpatch}
%-------- ΜΑΘΗΜΑΤΙΚΑ ΕΡΓΑΛΕΙΑ ---------
\usepackage{mathtools}
%----------------------
%-------- ΠΙΝΑΚΕΣ ---------
\usepackage{booktabs}
%----------------------
%----- ΥΠΟΛΟΓΙΣΤΗΣ ----------
\usepackage{calculator}
%----------------------------

%------------------------------------------
\newcommand{\tss}[1]{\textsuperscript{#1}}
\newcommand{\tssL}[1]{\MakeLowercase{\textsuperscript{#1}}}
%---------- ΛΙΣΤΕΣ ----------------------
\newlist{bhma}{enumerate}{3}
\setlist[bhma]{label=\bf\textit{\arabic*\textsuperscript{o}\;Βήμα :},leftmargin=0cm,itemindent=1.8cm,ref=\bf{\arabic*\textsuperscript{o}\;Βήμα}}
\newlist{rlist}{enumerate}{3}
\setlist[rlist]{itemsep=0mm,label=\roman*.}
\newlist{brlist}{enumerate}{3}
\setlist[brlist]{itemsep=0mm,label=\bf\roman*.}
\newlist{tropos}{enumerate}{3}
\setlist[tropos]{label=\bf\textit{\arabic*\textsuperscript{oς}\;Τρόπος :},leftmargin=0cm,itemindent=2.3cm,ref=\bf{\arabic*\textsuperscript{oς}\;Τρόπος}}
% Αν μπει το bhma μεσα σε tropo τότε
%\begin{bhma}[leftmargin=.7cm]
\tkzSetUpPoint[size=7,fill=white]
\tikzstyle{pl}=[line width=0.3mm]
\tikzstyle{plm}=[line width=0.4mm]
\usepackage{etoolbox}
\makeatletter
\renewrobustcmd{\anw@true}{\let\ifanw@\iffalse}
\renewrobustcmd{\anw@false}{\let\ifanw@\iffalse}\anw@false
\newrobustcmd{\noanw@true}{\let\ifnoanw@\iffalse}
\newrobustcmd{\noanw@false}{\let\ifnoanw@\iffalse}\noanw@false
\renewrobustcmd{\anw@print}{\ifanw@\ifnoanw@\else\numer@lsign\fi\fi}
\makeatother
\let\Bbbk\relax
\usepackage{enumitem,amssymb}
\usepackage{lipsum,venndiagram}
\let\Bbbk\relax
\newlist{todolist}{itemize}{2}
\setlist[todolist]{label=\Large$\square$}


\newtcolorbox{mybox}[2][]{colback=white,
colframe=red!75!black,fonttitle=\Large\bfseries,
colbacktitle=red!20!white,coltitle=black,enhanced,breakable,sharp corners,boxrule=0.3mm,center title,boxsep=3mm,top=1mm,subtitle style={fonttitle=\normalsize\bfseries},
title=#2,#1}
\setlength{\columnsep}{1cm}
\setlength{\columnseprule}{0.2pt}
\tcbset{mysubtitle/.style={subtitle style={fontupper={\OnPar\color{black}},top=0pt,colback={white},boxrule=1pt},top=0pt}}

\newtcolorbox{myleftbox}[2][]{nobeforeafter, title=#2,boxrule=0pt,colframe=black,coltitle=black,right=-3mm,left=5mm,left skip=0mm,colbacktitle=white,colback=white,#1,sharp corners,grow to left by=0.68cm,titlerule=0.2mm,fonttitle=\OnPar}

\newtcolorbox{myrightbox}[2][]{nobeforeafter, title=#2,boxrule=0pt,colframe=black,coltitle=black,right=-2mm,left skip=6mm,colbacktitle=white,colback=white,#1,leftrule=0.2mm,sharp corners,right skip=0mm,titlerule=0.2mm,fonttitle=\OnPar}
\usepackage{fancyhdr}
\pagestyle{fancy}

\pagestyle{fancy}
\fancyhf{}
\fancyheadoffset{0cm}
\renewcommand{\headrulewidth}{0pt} 
\renewcommand{\footrulewidth}{0pt}
\fancyhead[R]{
  \color{lightgray}{}
  }
\fancyhead[R]{
 \color{gray} Άλγεβρα Α΄ Λυκείου\hspace{1em}\color{lightgray}{\vline}\hspace{1em}\color{gray}\thepage
  }
\fancypagestyle{plain}{%
  \fancyhf{}%
  \fancyhead[R]{\leftmark\hspace{1em}\color{lightgray}{\vline}\hspace{1em}\color{gray}\thepage}%
  }
\renewcommand{\sectionmark}[1]{\markboth{#1}{#1}}
\newcommand{\myitem}{\stepcounter{enumi}\item[\raisebox{0.5mm}{\faExclamationTriangle}\ \Large$\square$]}

\newlist{arithmisi}{enumerate}{2}
\setlist[arithmisi]{itemsep=0mm,label=\textcolor{\xrwma}{\textbf{\textit{{\Large{\thesection}}.\arabic*}}}}

\begin{document}
\begin{flushright}
\faCalendar* Ημερομηνία: .......................
\end{flushright}
\begin{mybox}[mysubtitle]{\section{Σύνολα}}
\begin{tcbraster}[raster columns=2,raster equal height]
\begin{myleftbox}{Ορισμοί - Βασικές έννοιες\ \ \faBook}
\begin{enumerate}[itemsep=0mm]
\item Σύνολο
\item Τρόποι παράστασης συνόλου
\item Βασικό σύνολο
\item Κενό σύνολο
\item Διάγραμμα Venn
\item Ίσα σύνολα
\item Υποσύνολο
\item Πράξεις συνόλων
\end{enumerate}
\end{myleftbox}
\begin{myrightbox}{Θεωρήματα - Ιδιότητες\ \ \faTools}
\begin{enumerate}[itemsep=0mm]
\item Ιδιότητες υποσυνόλου
\item Ιδιότητες ένωσης
\item Ιδιότητες τομής
\item Ιδιότητες συμπληρώματος
\item Ιδιότητες διαφοράς συνόλων
\end{enumerate}
\end{myrightbox}
\end{tcbraster}
\tcbsubtitle{Είδη ασκήσεων - Τι πρέπει να γνωρίζω\ \ \faPen}
\begin{multicols}{2}
\begin{todolist}[itemsep=0mm]
\item Μετατροπή αναγραφής σε περιγραφή.
\myitem Πράξεις μεταξύ συνόλων
\end{todolist}
\end{multicols}
\tcbsubtitle{Τυπολόγιο - Συμβολισμοί\ \ \faFile*}
\begin{multicols}{2}
\begin{enumerate}[itemsep=0mm]
\item Ανήκει: $ \in $
\item Δεν ανήκει: $ \notin $
\item Φυσικοί αριθμοί: $ \mathbb{N} $
\item Ακέραιοι αριθμοί: $ \mathbb{Z} $
\item Ρητοί αριθμοί: $ \mathbb{Q} $
\item Πραγματικοί αριθμοί: $ \mathbb{R} $
\item Κενό σύνολο: $ \varnothing $
\item Βασικό σύνολο: $ \Omega $
\item Υποσύνολο: $ \subseteq $
\item Ένωση:\\$ A\cup B=\{x\in\Omega:x\in A\textrm{ και }x\in B \} $
\item Τομή:\\
$ A\cap B=\left\lbrace x\in\varOmega\left| x\in A \textrm{ και } x\in B\right.\right\rbrace $
\item Συμπλήρωμα:\\
$ A'=\{x\in\Omega:x\notin A\} $
\item Διαφορά:\\
$ A-B=\{x\in\Omega:x\in A\textrm{ και }x\notin B \} $
\end{enumerate}
\end{multicols}
\end{mybox}
\newpage
\orismoi
\begin{arithmisi}
\item\textbf{Σύνολο}\\
Σύνολο ονομάζεται μια συλλογή όμοιων αντικειμένων, που είναι καλά ορισμένα και διακριτά μεταξύ τους.
\begin{itemize}[itemsep=0mm]
\item Τα αντικείμενα ενός συνόλου ονομάζονται \textbf{στοιχεία}.
\item Τα σύνολα τα συμβολίζουμε με ένα κεφαλαίο γράμμα.
\item Για να δηλώσουμε ότι ένα στοιχείο $ x $ \textbf{ανήκει} σε ένα σύνολο $ A $ γράφουμε $ x\in A $. Ενώ αν το $ x $ \textbf{δεν ανήκει} στο σύνολο $ A $ γράφουμε $ x\notin A $.
\end{itemize}
\item\textbf{Βασικό σύνολο}\\
Βασικό ονομάζεται το σύνολο που περιέχει όλα τα στοιχεία στο χώρο στον οποίο εργαζόμαστε. Συμβολίζεται με $ \varOmega $.
\item\textbf{Κενό σύνολο}\\
Κενό ονομάζεται το σύνολο που δεν έχει στοιχεία. Συμβολίζεται με $ \varnothing $ ή $ \left\lbrace \right\rbrace  $.
\item\textbf{Τρόποι παράστασης συνόλου}\\
Οι τρόποι με τους οποίους μπορούμε να παραστήσουμε ένα σύνολο είναι οι εξής:
\vspace{-2mm}
\begin{enumerate}[label=\bf\arabic*.]
\item \textbf{Αναγραφή}\\
Γράφουμε τα στοιχεία ενός συνόλου μέσα σε άγκιστρα: $ \{\,\,\} $ όπου κάθε στοιχείο αναγράφεται μια φορά.
\[ A=\{a_1,a_2,\ldots,a_\nu\} \]
Τα στοιχεία του συνόλου χωρίζονται με κόμμα (,).
\item \textbf{Περιγραφή}\\
Γράφουμε που ανήκουν τα στοιχεία και ποια ιδιότητα έχουν. Έχει τη μορφή: $ A=\{x\in\varOmega\;|\;\textrm{Ιδιότητα }I\} $.
\item \textbf{Διάγραμμα Venn}\\
\wrapr{-4mm}{5}{2.5cm}{-11mm}{\begin{tikzpicture}[scale=.5]
\draw(-2,-2) rectangle (2.6,1);
\scope % A \cap B
\fill[\xrwma!30] (-.3,-.5) circle (1.1);
\draw[black] (-.3,-.5) circle (1.1);
\endscope
\tkzText(-1.5,-1.6){$ \varOmega $}
\tkzText(-.3,.2){$ A $}
\end{tikzpicture}}{
Σχεδιάζουμε με ορθογώνιο το βασικό σύνολο και με κύκλους τα υποσύνολά του.}\mbox{}\\
\end{enumerate}
\item\textbf{Ίσα σύνολα}\\
Δύο σύνολα $ A,B $ ονομάζονται αν έχουν ακριβώς τα ίδια στοιχεία. Συμβολίζεται $ A=B $. Ισοδύναμα, τα σύνολα $ Α,Β $ λέγονται ίσα εάν ισχύουν συγχρόνως οι σχέσεις :
\begin{enumerate}[itemsep=0mm]
\item Κάθε στοιχείο του $ A $ είναι και στοιχείο του $ B $
\item Κάθε στοιχείο του $ B $ είναι και στοιχείο του $ A $.
\end{enumerate}
\item\textbf{Υποσύνολο}\\
Ένα σύνολο $ A $ λέγεται υποσύνολο ενός συνόλου $ B $ όταν κάθε στοιχείο του $ A $ είναι και στοιχείο του $ B $. Συμβολίζεται $ A\subseteq B $.
\item\textbf{Ξένα σύνολα}\\
Δύο σύνολα $ A,B $ ονομάζονται ξένα μεταξύ τους αν δεν έχουν κοινά στοιχεία.
\item\textbf{Πράξεις μεταξύ συνόλων}\\
\vspace{-5mm}
\begin{enumerate}[label=\bf\arabic*.,itemsep=3mm]
\item \textbf{Ένωση}\\
\begin{minipage}{\linewidth}
\begin{WrapText1}{8}{3.5cm}
\vspace{-5mm}
\begin{venndiagram2sets}[tikzoptions={scale=.7,samples=100},shade=\xrwma!30,labelNotAB={$ \varOmega $}]
\fillA \fillB
\end{venndiagram2sets}
\end{WrapText1}
Ένωση δύο υποσυνόλων $ A,B $ ενός βασικού συνόλου $ \varOmega $ ονομάζεται το σύνολο των στοιχείων του $ \varOmega $ τα οποία ανήκουν σε \textbf{τουλάχιστον ένα} από τα σύνολα $ A $ και $ B $. Συμβολίζεται με $ A\cup B $.  \[ A\cup B=\left\lbrace x\in\varOmega\left| x\in A \textrm{ ή } x\in B\right.\right\rbrace \]
Η ένωση των συνόλων $ A $ και $ B $ περιέχει \textbf{όλα} τα στοιχεία των δύο συνόλων. Τα κοινά στοιχεία αναγράφονται μια φορά.\end{minipage}
\item \textbf{Τομή}\\
\begin{minipage}{\linewidth}
\begin{WrapText1}{7}{3.5cm}
\vspace{-5mm}
\begin{venndiagram2sets}[tikzoptions={scale=.7},shade=\xrwma!30,labelNotAB={$ \varOmega $}]
\fillACapB
\end{venndiagram2sets}
\end{WrapText1}
Τομή δύο υποσυνόλων $ A,B $ ενός βασικού συνόλου $ \varOmega $ ονομάζεται το σύνολο των στοιχείων του $ \varOmega $ τα οποία ανήκουν \textbf{και στα δύο} σύνολα $ A $ και $ B $. Συμβολίζεται με $ A\cap B $. \[ A\cap B=\left\lbrace x\in\varOmega\left| x\in A \textrm{ και } x\in B\right.\right\rbrace \]
Η τομή των συνόλων $ A $ και $ B $ περιέχει μόνο τα \textbf{κοινά} στοιχεία των δύο συνόλων.\end{minipage}
\item \textbf{Συμπλήρωμα}\\
\begin{minipage}{\linewidth}
\begin{WrapText1}{8}{3.5cm}
\vspace{-9mm}
\begin{tikzpicture}[scale=.77]
\filldraw[fill=\xrwma!30] (-2,-2) rectangle (2.6,1);
\scope % A \cap B
\fill[white] (-.45,-.5) circle (1.1);
\draw[black] (-.45,-.5) circle (1.1);
\endscope
\tkzText(-1.6,-1.6){$ \varOmega $}
\tkzText(-.45,.3){$ A $}
\end{tikzpicture}
\end{WrapText1}
Συμπλήρωμα ενός συνόλου $ A $ ονομάζεται το σύνολο των στοιχείων του βασικού συνόλου $ \varOmega $ τα οποία \textbf{δεν} ανήκουν στο $ A $. Συμβολίζεται με $ A' $. \[ A'=\left\lbrace x\in\varOmega\left| x\notin A\right.\right\rbrace \]\end{minipage}
\item \textbf{Διαφορά}\\
\begin{minipage}{\linewidth}
\begin{WrapText1}{8}{3.5cm}
\vspace{-5mm}
\begin{venndiagram2sets}[tikzoptions={scale=.7},shade=\xrwma!30,labelNotAB={$ \varOmega $}]
\fillANotB
\end{venndiagram2sets}
\end{WrapText1}
Διαφορά ενός συνόλου $ B $ από ένα σύνολο $ A $ ονομάζεται το σύνολο των στοιχείων του βασικού συνόλου $ \varOmega $ τα οποία ανήκουν \textbf{μόνο} στο σύνολο $ A $, το πρώτο σύνολο της διαφοράς. Συμβολίζεται με $ A-B $. \[ A-B=\left\lbrace x\in\varOmega\left| x\in A\textrm{ και }x\notin B\right. \right\rbrace  \]
\end{minipage}
\end{enumerate}\mbox{}\\
\end{arithmisi}
\thewrhmata
\begin{arithmisi}
\item\textbf{Ιδιότητες υποσυνόλου}\\
Για οποιαδήποτε σύνολα $ A,B,\varGamma $ ισχύουν οι παρακάτω ιδιότητες που αφορούν τη σχέση του υποσυνόλου:
\begin{rlist}
\item $ A\subseteq A $.
\item Αν $ A\subseteq B $ και $ B\subseteq \varGamma $ τότε $ A\subseteq \varGamma $.
\item Αν $ A\subseteq B $ και $ B\subseteq A $ τότε $ A=B $.
\end{rlist}
\item\textbf{Ιδιότητες ένωσης συνόλων}\\
Για οποιαδήποτε σύνολα $ A,B,\varGamma $ ισχύουν οι παρακάτω ιδιότητες για την πράξη της ένωσης.
\begin{multicols}{2}
\begin{rlist}
\item $ A\cup\varnothing=A $
\item $ A\cup A=A $
\item $ A\cup B=B\cup A $
\item $ (A\cup B)\cup\varGamma=A\cup(B\cup\varGamma) $
\item $ A\subseteq A\cup B $ και $ B\subseteq A\cup B $
\item Αν $ A\subseteq B $ τότε $ A\cup B=B $
\end{rlist}
\end{multicols}
\item\textbf{Ιδιότητες τομής συνόλων}\\
Για οποιαδήποτε σύνολα $ A,B,\varGamma $ ισχύουν οι παρακάτω ιδιότητες για την πράξη της τομής.
\begin{multicols}{2}
\begin{rlist}
\item $ A\cap\varnothing=\varnothing $
\item $ A\cap A=A $
\item $ A\cap B=B\cap A $
\item $ (A\cap B)\cap\varGamma=A\cap(B\cap\varGamma) $
\item $ A\cap B\subseteq A $ και $ A\cap B\subseteq B $
\item Αν $ A\subseteq B $ τότε $ A\cap B=A $
\item Αν $ A\cap B=\varnothing $ τα $ A,B $ είναι ξένα μεταξύ τους.
\end{rlist}
\end{multicols}
\item\textbf{Επιμεριστική ιδιότητα ως προς ένωση και τομή}\\
Για οποιαδήποτε σύνολα $ A,B,\varGamma $ ισχύουν οι σχέσεις
\begin{center}
\begin{tikzpicture}
\draw[-latex] (-3.2,0) arc (130.7501:45.0195:0.55);
\node at (-0.5,-0.25) {$ A\cap(B\cup \varGamma)=(A\cap B)\cup(A\cap\varGamma)$};
\draw[-latex] (-3.2,0) arc (150.402:26.8204:0.85);
\draw[-latex] (-1.695,-0.45) arc (-121.4289:-55.6842:2.2);

\draw[-latex] (-3.2,-1.5) arc (130.7501:45.0195:0.55);
\draw[-latex] (-3.2,-1.5) arc (150.402:26.8204:0.85);
\draw[-latex] (-1.695,-1.95) arc (-121.4289:-55.6842:2.2);
\node at (-0.5,-1.75) {$ A\cup(B\cap \varGamma)=(A\cup B)\cap(A\cup\varGamma)$};
\end{tikzpicture}
\end{center}
\item\textbf{Ιδιότητες συμπληρώματος}\\
Για οποιοδήποτε σύνολο $ A\subseteq\Omega $ ισχύουν οι παρακάτω ιδιότητες για την πράξη του συμπληρώματος.
\begin{multicols}{2}
\begin{rlist}
\item $ \Omega'=\varnothing $
\item $ \varnothing'=\Omega $
\item $ A\cap A'=\varnothing $
\item $ A\cup A'=\Omega $
\item $ (A')'=A $
\end{rlist}
\end{multicols}
\item\textbf{Ιδιότητες διαφοράς}\\
Για οποιαδήποτε σύνολα $ A,B $ ισχύουν οι παρακάτω ιδιότητες για την πράξη της διαφοράς.
\begin{multicols}{2}
\begin{rlist}
\item $ A-B=A\cap B' $
\item $ B-A=B\cap A' $
\item $ A-B\subseteq A $
\item $ B-A\subseteq B $
\end{rlist}
\end{multicols}
\end{arithmisi}
\methodologia
\begin{arithmisi}
\item\textbf{Τα σύμβολα ανήκει $ \in $ και δεν ανήκει $ \notin $}\\
Θέλουμε να εξετάσουμε αν ένας αριθμός $ x $ ανήκει ή όχι σε ένα σύνολο $ A $.
\begin{itemize}
\item Εκτελούμε αρχικά τυχόν πράξεις.
\item Αν το σύνολο είναι γραμμένο με αναγραφή, εξετάζουμε αν ο $ x $ είναι ένα από τα στοιχεία του $ A $.
\item Αν το σύνολο είναι γραμμένο με περιγραφή, εξετάζουμε αν ο $ x $ ικανοποιεί την ιδιότητα που ζητάει το σύνολο. 
\end{itemize}
\item\textbf{Μετατροπή μεταξύ αναγραφής, περιγραφής και διαγράμματος Venn}
\begin{rlist}
\item Αναγραφή $ \leftrightarrow $ Περιγραφή
\begin{bhma}
\item 
\end{bhma}
\end{rlist}
\end{arithmisi}
\end{document}
