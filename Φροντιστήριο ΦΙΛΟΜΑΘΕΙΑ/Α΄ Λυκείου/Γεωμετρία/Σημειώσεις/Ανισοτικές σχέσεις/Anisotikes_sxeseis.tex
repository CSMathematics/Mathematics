\PassOptionsToPackage{no-math,cm-default}{fontspec}
\documentclass[twoside,nofonts,internet,shmeiwseis]{thewria}
\usepackage{amsmath}
\usepackage{xgreek}
\let\hbar\relax
\defaultfontfeatures{Mapping=tex-text,Scale=MatchLowercase}
\setmainfont[Mapping=tex-text,Numbers=Lining,Scale=1.0,BoldFont={Minion Pro Bold}]{Minion Pro}
\newfontfamily\scfont{GFS Artemisia}
\font\icon = "Webdings"
\usepackage[amsbb]{mtpro2}
\usepackage{tikz,pgfplots,tkz-euclide}
\tkzSetUpPoint[size=7,fill=white]
\xroma{red!70!black}
%------- ΣΥΣΤΗΜΑ -------------------
\usepackage{systeme,regexpatch}
\makeatletter
% change the definition of \sysdelim not to store `\left` and `\right`
\def\sysdelim#1#2{\def\SYS@delim@left{#1}\def\SYS@delim@right{#2}}
\sysdelim\{. % reinitialize

% patch the internal command to use
% \LEFTRIGHT<left delim><right delim>{<system>}
% instead of \left<left delim<system>\right<right delim>
\regexpatchcmd\SYS@systeme@iii
{\cB.\c{SYS@delim@left}(.*)\c{SYS@delim@right}\cE.}
{\c{SYS@MT@LEFTRIGHT}\cB\{\1\cE\}}
{}{}
\def\SYS@MT@LEFTRIGHT{%
\expandafter\expandafter\expandafter\LEFTRIGHT
\expandafter\SYS@delim@left\SYS@delim@right}
\makeatother
\newcommand{\synt}[2]{{\scriptsize \begin{matrix}
\times#1\\\\ \times#2
\end{matrix}}}
%----------------------------------------
%------ ΜΗΚΟΣ ΓΡΑΜΜΗΣ ΚΛΑΣΜΑΤΟΣ ---------
\DeclareRobustCommand{\frac}[3][0pt]{%
{\begingroup\hspace{#1}#2\hspace{#1}\endgroup\over\hspace{#1}#3\hspace{#1}}}
%----------------------------------------

\newlist{rlist}{enumerate}{3}
\setlist[rlist]{itemsep=0mm,label=\roman*.}
\newlist{brlist}{enumerate}{3}
\setlist[brlist]{itemsep=0mm,label=\bf\roman*.}
\newlist{tropos}{enumerate}{3}
\setlist[tropos]{label=\bf\textit{\arabic*\textsuperscript{oς}\;Τρόπος :},leftmargin=0cm,itemindent=2.3cm,ref=\bf{\arabic*\textsuperscript{oς}\;Τρόπος}}
\newcommand{\tss}[1]{\textsuperscript{#1}}
\newcommand{\tssL}[1]{\MakeLowercase{\textsuperscript{#1}}}
\usetkzobj{all}
\usepackage{hhline}
%----------- ΓΡΑΦΙΚΕΣ ΠΑΡΑΣΤΑΣΕΙΣ ---------
\pgfkeys{/pgfplots/aks_on/.style={axis lines=center,
xlabel style={at={(current axis.right of origin)},xshift=1.5ex, anchor=center},
ylabel style={at={(current axis.above origin)},yshift=1.5ex, anchor=center}}}
\pgfkeys{/pgfplots/grafikh parastash/.style={\xrwma,line width=.4mm,samples=200}}
\pgfkeys{/pgfplots/belh ar/.style={tick label style={font=\scriptsize},axis line style={-latex}}}
%-----------------------------------------
\usepackage{multicol}
\usepackage{wrap-rl}
\tkzSetUpPoint[size=7,fill=white]
\tikzstyle{pl}=[line width=0.3mm]
\tikzstyle{plm}=[line width=0.4mm]
\usepackage{gensymb}


\begin{document}
\titlos{Γεωμετρία Α΄ Λυκείου}{Τρίγωνα}{Ανισοτικές σχέσεις}
\thewrhmata
\Thewrhma{Σχέση εξωτερικής και εξωτερικής γωνίας}
Σε κάθε τρίγωνο, οποιαδήποτε εξωτερική γωνία του είναι μεγαλύτερη από κάθε απέναντι εσωτερική.
\begin{center}
\begin{tikzpicture}
\tkzDefPoint[label=above left:$A$](1,1.7){A}
\tkzDefPoint[label=below:$B$](0,0){B}
\tkzDefPoint[label=below:$\varGamma$](3,0){C}
\tkzDefPoint(4,0){d}
\tkzDefPoint(-.7,0){e}
\tkzDefPoint(1.3,2.21){f}
\tkzMarkAngle[size=.3,fill=\xrwma](d,C,A)
\tkzMarkAngle[size=.3,fill=\xrwma](A,B,e)
\tkzMarkAngle[size=.3,fill=\xrwma](C,A,f)
\tkzMarkAngle[size=.4,fill=\xrwma!70](B,A,C)
\tkzMarkAngle[size=.4,fill=\xrwma!70](C,B,A)
\tkzMarkAngle[size=.4,fill=\xrwma!70](A,C,B)
\draw[pl](A)--(B)--(C)--cycle;
\draw(e)--(d);
\draw(B)--(f);
\tkzDrawPoints(A,B,C)
\node at (3.2,0.5) {\footnotesize$ \hat{\varGamma}_{\textrm{εξ}} $};
\node at (-0.2,0.5) {\footnotesize$ \hat{B}_{\textrm{εξ}} $};
\node at (1.6,1.8) {\footnotesize$ \hat{A}_{\textrm{εξ}} $};
\node at(5.5,1.6){$ \hat{A}_{\textrm{εξ}}>\hat{B} \textrm{ και }\hat{A}_{\textrm{εξ}}>\hat{\varGamma} $};
\node at(5.5,1){$ \hat{B}_{\textrm{εξ}}>\hat{A} \textrm{ και }\hat{B}_{\textrm{εξ}}>\hat{\varGamma} $};
\node at(5.5,.4){$ \hat{\varGamma}_{\textrm{εξ}}>\hat{A} \textrm{ και }\hat{\varGamma}_{\textrm{εξ}}>\hat{B} $};
\end{tikzpicture}
\end{center}
\Thewrhma{Σχέσεις μεταξύ πλευρών και γωνιών}
Σε κάθε τρίγωνο απέναντι από δύο άνισες πλευρές βρίσκονται δύο όμια άνισες γωνίες. Αντίστροφα απέναντι από δύο άνισες γωνίες βρίσκονται δύο όμοια άνισες πλευρές.
\begin{center}
\begin{tikzpicture}
\tkzDefPoint[label=above:$A$](1,1.7){A}
\tkzDefPoint[label=left:$B$](0,0){B}
\tkzDefPoint[label=right:$\varGamma$](3,0){C}
\tkzMarkAngle[size=.4,fill=\xrwma](C,B,A)
\tkzMarkAngle[size=.4,fill=\xrwma](A,C,B)
\draw[pl](A)--(B)--(C)--cycle;
\draw[pl,\xrwma] (B)--(A)--(C);
\tkzDrawPoints(A,B,C)
\node at (5,1) {$AB<A\varGamma\Leftrightarrow\hat{\varGamma}<\hat{B}$};
\end{tikzpicture}
\end{center}
\Thewrhma{Πορίσματα για σχέσεις γωνιών και πλευρών}
Ισχύουν οι εξής προτάσεις για τις σχέσεις μεταξύ πλευρών και γωνιών ενός τριγώνου :
\begin{rlist}
\item Απέναντι από την ορθή γωνία σε ένα ορθογώνιο τρίγωνο και απέναντι από την αμβλεία γωνία σε ένα αμβλυγώνιο τρίγωνο βρίσκεται η μεγαλύτερη πλευρά του τριγώνου.
\item Αν ένα τρίγωνο έχει δύο γωνίες ίσες τότε είναι ισοσκελές.
\item Αν ένα τρίγωνο έχει και τις τρεις γωνίες του ίσες τότε είναι ισόπλευρο.
\end{rlist}
\Thewrhma{Τριγωνική ανισότητα}
\wrapr{-4mm}{8}{4cm}{-7mm}{\begin{tikzpicture}
\tkzDefPoint[label=above:$A$](1,1.7){A}
\tkzDefPoint[label=left:$B$](0,0){B}
\tkzDefPoint[label=right:$\varGamma$](3,0){C}
\draw[pl](A)--(B)--(C)--cycle;
\tkzDrawPoints(A,B,C)
\node at (1.5,-0.2) {\footnotesize$a$};
\node at (2.2,1) {\footnotesize$\beta$};
\node at (0.3,1) {\footnotesize$\gamma$};
\end{tikzpicture}}{
Σε κάθε τρίγωνο, οποιαδήποτε πλευρά είναι μικρότερη από το άθροισμα των άλλων δύο πλευρών και μεγαλύτερη από τη διαφορά τους.
\[ \beta-\gamma<a<\beta+\gamma\quad,\quad\textrm{με }\beta\geq\gamma \]}\mbox{}\\\\\\
\Thewrhma{Κριτήριο για ισοσκελές τρίγωνο}
Αν σε ένα τρίγωνο, το ευθύγραμμο τμήμα που ενώνει μια κορυφή με ένα σημείο της απέναντι πλευράς είναι δύο από τα τρία δευτερεύοντα στοιχεία
\begin{multicols}{3}
\begin{rlist}
\item διάμεσος
\item διχοτόμος
\item ύψος
\end{rlist}
\end{multicols}
τότε το τρίγωνο είναι ισοσκελές με βάση την πλευρά αυτή.\\\\
\Thewrhma{Ανισότητα πλευρών}
Αν δύο τρίγωνα έχουν δύο πλευρές ίσες και τις περιεχόμενες γωνίες τους άνισες, τότε οι απέναντι πλευρές θα είναι όμοια άνισες.
\[ AB=A\varGamma\textrm{ και }\hat{A}>\hat{\varDelta}\Rightarrow B\varGamma>EZ \]
\begin{center}
\begin{tikzpicture}
\tkzDefPoint[label=above:$A$](1,1.7){A}
\tkzDefPoint[label=left:$B$](0,0){B}
\tkzDefPoint[label=right:$\varGamma$](3,0){C}
\draw[pl](A)--(B)--(C)--cycle;
\tkzDefPoint[label=above:$\varDelta$](5,1.7){D}
\tkzDefPoint[label=left:$E$](4.2,0){E}
\tkzDefPoint[label=right:$Z$](6.8,0){Z}
\tkzMarkSegments[mark=|](A,B D,E)
\tkzMarkSegments[mark=||](A,C D,Z)
\tkzMarkAngle[size=.3,fill=\xrwma](B,A,C)
\tkzMarkAngle[size=.3,fill=\xrwma](E,D,Z)
\draw[pl,\xrwma](B)--(C);
\draw[pl](D)--(E)--(Z)--cycle;
\draw[pl,\xrwma](E)--(Z);
\tkzDrawPoints(A,B,C,D,E,Z)
\end{tikzpicture}
\end{center}
Αντίστροφα, αν δύο τρίγωνα έχουν δύο πλευρές ίσες και τις τρίτες πλευρές τους άνισες τότε οι απέναντι γωνίες θα είναι όμοια άνισες.
\[ AB=A\varGamma\textrm{ και }B\varGamma>EZ \Rightarrow \hat{A}>\hat{\varDelta} \]
\end{document}
