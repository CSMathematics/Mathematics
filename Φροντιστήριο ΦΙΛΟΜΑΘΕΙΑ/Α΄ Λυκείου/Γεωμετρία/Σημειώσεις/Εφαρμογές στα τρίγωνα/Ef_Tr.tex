\PassOptionsToPackage{no-math,cm-default}{fontspec}
\documentclass[twoside,nofonts,internet,shmeiwseis]{thewria}
\usepackage{amsmath}
\usepackage{xgreek}
\let\hbar\relax
\defaultfontfeatures{Mapping=tex-text,Scale=MatchLowercase}
\setmainfont[Mapping=tex-text,Numbers=Lining,Scale=1.0,BoldFont={Minion Pro Bold}]{Minion Pro}
\newfontfamily\scfont{GFS Artemisia}
\font\icon = "Webdings"
\usepackage[amsbb]{mtpro2}
\usepackage{tikz,pgfplots,tkz-euclide,enumitem}
\usetkzobj{all}
\tkzSetUpPoint[size=7,fill=white]
\xroma{red!70!black}
\newlist{rlist}{enumerate}{3}
\setlist[rlist]{itemsep=0mm,label=\roman*.}
\newlist{brlist}{enumerate}{3}
\setlist[brlist]{itemsep=0mm,label=\bf\roman*.}
\newlist{tropos}{enumerate}{3}
\setlist[tropos]{label=\bf\textit{\arabic*\textsuperscript{oς}\;Τρόπος :},leftmargin=0cm,itemindent=2.3cm,ref=\bf{\arabic*\textsuperscript{oς}\;Τρόπος}}
\newcommand{\tss}[1]{\textsuperscript{#1}}
\newcommand{\tssL}[1]{\MakeLowercase{\textsuperscript{#1}}}
\usepackage{rotating}
\usepackage{hhline}
\usepackage{multicol,multirow,gensymb,mathimatika}
\usepackage{wrap-rl}


\begin{document}
\titlos{Γεωμετρία Α΄ Λυκείου}{Παραλληλόγραμμα}{Εφαρμογές στα τρίγωνα}
\orismoi
\Orismos{Μεσοπαράλληλος}
\wrapr{-4mm}{7}{4.5cm}{-5mm}{\begin{tikzpicture}
\draw (-2,1.5) -- (2,1.5);
\draw[\xrwma] (-2,0.75) -- (2,0.75);
\draw (-2,0) -- (2,0);
\tkzDefPoint(-1.4,0){A}
\tkzDefPoint(-1,0){B}
\tkzDefPoint(-1.4,1.5){C}
\node at (2.4,1.5) {\footnotesize$\varepsilon_1$};
\node at (2.4,0) {\footnotesize$\varepsilon_2$};
\node at (2.4,.75) {\footnotesize$\varepsilon$};
\draw (-1.4,1.5) -- (-1.4,0);
\tkzMarkSegment[mark=|,pos=.25](A,C)
\tkzMarkSegment[mark=|,pos=.75](A,C)
\tkzMarkRightAngle[size=.2](B,A,C)
\tkzLabelPoint[above left](C){$ A $}
\tkzLabelPoint[below left](A){$ B $}
\tkzLabelPoint[above left](-1.4,.75){$ \varGamma $}
\end{tikzpicture}}{
Μεσοπαράλληλος δύο παράλληλων ευθειών $ \varepsilon_1,\varepsilon_2 $ ονομάζεται ο γεωμετρικός τόπος των σημείων του ίδιου επιπέδου τα οποία έχουν ίσες αποστάσεις από τις ευθείες αυτές.
\[ \varepsilon\parallel\varepsilon_1\parallel\varepsilon_2\qquad A\varGamma=B\varGamma \]
Είναι ευθεία γραμμή, παράλληλη με τις $ \varepsilon_1,\varepsilon_2 $ και βρίσκεται στο μέσο της απόστασής τους.}
\thewrhmata
\Thewrhma{Τμήμα από τα μέσα δύο πλευρών}
Το ευθύγραμμο τμήμα που ενώνει τα μέσα των δύο πλευρών ενός τριγώνου είναι παράλληλο με την τρίτη πλευρά και ισούται με το μισό της.
\[ MN\parallel=\frac{B\varGamma}{2} \]
για ένα τρίγωνο $ AB\varGamma $ με $ M,N $ τα μέσα των πλευρών $ AB,A\varGamma $ αντίστοιχα.
\begin{center}
\begin{tikzpicture}
\tkzDefPoint(0,0){B}
\tkzDefPoint(3.5,0){C}
\tkzDefPoint(1.2,2){A}
\tkzDefPoint(.6,1){M}
\tkzDefPoint(2.35,1){N}
\draw[pl](A)--(B)--(C)--cycle;
\draw[dashed] (0,1)--(3.3,1);
\draw[plm,\xrwma] (M)--(N);
\tkzDrawPoints(A,B,C,M,N)
\tkzLabelPoint[above](A){$A$}
\tkzLabelPoint[left](B){$B$}
\tkzLabelPoint[right](C){$\varGamma$}
\tkzLabelPoint[left,yshift=2mm](M){$M$}
\tkzLabelPoint[right,yshift=2mm](N){$N$}
\end{tikzpicture}
\end{center}
\Thewrhma{Τμήμα παράλληλο από μέσο}
Η ευθεία που διέρχεται από το μέσο μιας πλευράς ενός τριγώνου και είναι παράλληλη προς μια δεύτερη πλευρά, θα διέρχεται και από το μέσο της τρίτης πλευράς.
\[ M\textrm{ μέσο }AB\textrm{ και }MN\parallel B\varGamma\Rightarrow N\textrm{ μέσο }A\varGamma \]
\Thewrhma{Ίσα τμήματα από παράλληλες ευθείες}
\wrapr{-4mm}{5}{4.5cm}{-8mm}{\begin{tikzpicture}[scale=1.3]
\tkzDefPoint(0,0){A}
\tkzDefPoint(3,0){B}
\tkzDefPoint(0,.5){C}
\tkzDefPoint(3,0.5){D}
\tkzDefPoint(0,1){E}
\tkzDefPoint(3,1){Z}
\tkzDefPoint(.7,1.3){H}
\tkzDefPoint(.4,-.3){I}
\tkzDefPoint(1.7,1.3){K}
\tkzDefPoint(2.7,-.3){L}
\draw[pl] (A)--(B);
\draw[pl] (C)--(D);
\draw[pl] (E)--(Z);
\draw[pl,\xrwma] (H)--(I);
\draw[pl,\xrwma] (K)--(L);
\tkzInterLL(E,Z)(H,I)\tkzGetPoint{S}
\tkzInterLL(C,D)(H,I)\tkzGetPoint{T}
\tkzInterLL(A,B)(H,I)\tkzGetPoint{Y}
\tkzInterLL(E,Z)(K,L)\tkzGetPoint{O}
\tkzInterLL(C,D)(K,L)\tkzGetPoint{P}
\tkzInterLL(A,B)(K,L)\tkzGetPoint{Q}
\tkzDrawPoints(S,T,Y,O,P,Q)
\tkzLabelPoint[above left](S){$A$}
\tkzLabelPoint[above left](T){$B$}
\tkzLabelPoint[above left](Y){$\varGamma$}
\tkzLabelPoint[above right](O){$\varDelta$}
\tkzLabelPoint[above right](P){$E$}
\tkzLabelPoint[above right](Q){$Z$}
\node at (3.2,1) {\footnotesize$\varepsilon_1$};
\node at (3.2,0.5) {\footnotesize$\varepsilon_2$};
\node at (3.2,0) {\footnotesize$\varepsilon_3$};
\node at (0.6,-0.2) {\footnotesize$\varepsilon$};
\node at (2.4,-0.2) {\footnotesize$\zeta$};
\end{tikzpicture}}{
Αν τρεις ή περισσότερες παράλληλες ευθείες ορίζουν ίσα τμήματα σε μια τέμνουσα, τότε θα ορίζουν ίσα τμήματα και σε οποιαδήποτε άλλη τέμουσα ευθεία.
\[ \varepsilon_1\parallel\varepsilon_2\parallel\varepsilon_3\textrm{ και }AB=B\varGamma\Rightarrow\varDelta E=EZ \]
}
\end{document}
