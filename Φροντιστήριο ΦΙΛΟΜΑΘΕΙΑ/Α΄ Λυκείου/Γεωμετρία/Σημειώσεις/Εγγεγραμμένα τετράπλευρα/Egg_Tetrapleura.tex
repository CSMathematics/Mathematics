\PassOptionsToPackage{no-math,cm-default}{fontspec}
\documentclass[twoside,nofonts,internet,shmeiwseis]{thewria}
\usepackage{amsmath}
\usepackage{xgreek}
\let\hbar\relax
\defaultfontfeatures{Mapping=tex-text,Scale=MatchLowercase}
\setmainfont[Mapping=tex-text,Numbers=Lining,Scale=1.0,BoldFont={Minion Pro Bold}]{Minion Pro}
\newfontfamily\scfont{GFS Artemisia}
\font\icon = "Webdings"
\usepackage[amsbb]{mtpro2}
\usepackage{tikz,pgfplots,tkz-euclide,enumitem}
\usetkzobj{all}
\tkzSetUpPoint[size=7,fill=white]
\xroma{red!70!black}
\setlist[itemize]{itemsep=0mm}
\newlist{rlist}{enumerate}{3}
\setlist[rlist]{itemsep=0mm,label=\roman*.}
\newlist{brlist}{enumerate}{3}
\setlist[brlist]{itemsep=0mm,label=\bf\roman*.}
\newlist{tropos}{enumerate}{3}
\setlist[tropos]{label=\bf\textit{\arabic*\textsuperscript{oς}\;Τρόπος :},leftmargin=0cm,itemindent=2.3cm,ref=\bf{\arabic*\textsuperscript{oς}\;Τρόπος}}
\newcommand{\tss}[1]{\textsuperscript{#1}}
\newcommand{\tssL}[1]{\MakeLowercase{\textsuperscript{#1}}}
\usepackage{rotating}
\usepackage{hhline}
\usepackage{multicol,multirow,gensymb,mathimatika}
\usepackage{wrap-rl}

\begin{document}
\titlos{Γεωμετρία Α΄ Λυκείου}{Εγγεγραμμένα σχήματα}{Εγγεγραμμένα - Περιγεγραμμένα τετράπλευρα}
\orismoi
\Orismos{Εγγεγραμμένο τετράπλευρο}
\wrapr{-4mm}{7}{3.3cm}{-14mm}{\begin{tikzpicture}
\tkzDefPoint[label=above left:$O$](0,0){O}
\tkzDefPoint(120:1.25){A}
\tkzDefPoint(50:1.25){B}
\tkzDefPoint(340:1.25){C}
\tkzDefPoint(210:1.25){D}
\draw[pl] (O) circle (1.25);
\draw[pl,\xrwma](A)--(B)--(C)--(D)--cycle;
\tkzDrawPoints(A,B,C,D,O)
\tkzLabelPoint[above left](A){$A$}
\tkzLabelPoint[above right](B){$B$}
\tkzLabelPoint[right](C){$\varGamma$}
\tkzLabelPoint[left](D){$\varDelta$}
\end{tikzpicture}}{
Εγγεγραμμένο ονομάζεται ένα τετράπλευρο του οποίου οι κορυφές είναι σημεία ενός κύκλου. Ο κύκλος αυτός ονομάζεται \textbf{περιγεγραμμένος}.}\mbox{}\\\\\\
\Orismos{Εγγράψιμο τετράπλευρο}
Εγγράψιμο ονομάζεται ένα τετράπλευρο όταν υπάρχει κύκλος που να διέρχεται από όλες τις κορυφές του.\\\\
\Orismos{Περιγεγραμμένο τετράπλευρο}
\wrapr{-4mm}{5}{4cm}{-14mm}{\begin{tikzpicture}
\tkzDefPoint[label=right:$O$](0,0){O}
\tkzDefPoint(160:1){E}
\tkzDefPoint(80:1){Z}
\tkzDefPoint(20:1){H}
\tkzDefPoint(270:1){J}
\tkzTangent[at=E](O)\tkzGetPoint{e}
\tkzTangent[at=Z](O)\tkzGetPoint{z}
\tkzTangent[at=H](O)\tkzGetPoint{h}
\tkzTangent[at=J](O)\tkzGetPoint{j}
\tkzInterLL(E,e)(Z,z)\tkzGetPoint{A}
\tkzInterLL(H,h)(Z,z)\tkzGetPoint{B}
\tkzInterLL(H,h)(J,j)\tkzGetPoint{C}
\tkzInterLL(J,j)(E,e)\tkzGetPoint{D}
\draw[pl] (O) circle (1);
\draw[pl,\xrwma](A)--(B)--(C)--(D)--cycle;
\tkzDrawPoints(E,Z,H,J,O,A,B,C,D)
\tkzLabelPoint[above left](A){$A$}
\tkzLabelPoint[above right](B){$B$}
\tkzLabelPoint[right](C){$\varGamma$}
\tkzLabelPoint[left](D){$\varDelta$}
\end{tikzpicture}}{
Περιγεγραμμένο τετράπλευρο ονομάζεται το τετράπλευρο του οποίου οι πλευρές είναι εφαπτόμενες στον ίδιο κύκλο. Ο κύκλος αυτός ονομάζεται \textbf{εγγεγραμμένος}.}\mbox{}\\\\\\
\Orismos{Περιγράψιμο τετράπλευρο}
Περιγράψιμο ονομάζεται το τετράπλευρο εκείνο για το οποίο υπάρχει κύκλος που να εφάπτεται σε όλες τις πλευρές του.\\\\
\thewrhmata
\Thewrhma{Ιδιότητες εγγεγραμμένου τετραπλεύρου}
\wrapr{-4mm}{5}{3.6cm}{-4mm}{\begin{tikzpicture}
\clip (-2,-1.3) rectangle (1.7,1.5);
\tkzDefPoint[label=right:$O$](0,0){O}
\tkzDefPoint(120:1.25){A}
\tkzDefPoint(50:1.25){B}
\tkzDefPoint(340:1.25){C}
\tkzDefPoint(210:1.25){D}
\tkzDefPointBy[symmetry= center D](C)\tkzGetPoint{a}
\tkzMarkAngle[fill=\xrwma,size=.3](D,A,C)
\tkzMarkAngle[fill=\xrwma,size=.3](D,B,C)
\tkzMarkAngle[fill=\xrwma,size=.3](A,D,a)
\draw (B)--(D);
\draw (A)--(C);
\tkzDrawLine[add=-.3 and 0](a,C)
\draw[pl] (O) circle (1.25);
\draw[pl,\xrwma](A)--(B)--(C)--(D)--cycle;
\tkzDrawPoints(A,B,C,D,O)
\tkzLabelPoint[above left](A){$A$}
\tkzLabelPoint[above right](B){$B$}
\tkzLabelPoint[right](C){$\varGamma$}
\tkzLabelPoint[below left](D){$\varDelta$}
\end{tikzpicture}}{
Για κάθε εγγεγραμμένο τετράπλευρο $ AB\varGamma\varDelta $ ισχύουν οι ακόλουθες ιδιότητες :
\begin{rlist}
\item Οι απέναντι γωνίες του είναι παραπληρωματικές :  \[ \hat{A}+\hat{\varGamma}=180\degree\ \textrm{ και }\  \hat{B}+\hat{\varDelta}=180\degree \]
\item Κάθε πλευρά του φαίνεται από τις απέναντι κορυφές υπό ίσες εγγεγραμμένες γωνίες.
\end{rlist}}\mbox{}\\
\vspace{-3mm}
\begin{rlist}[start=3]
\item Κάθε εξωτερική γωνία ισούται με την απέναντι εσωτερική :
$ \hat{A}_{\textrm{εξ}}=\hat{\varGamma}\ ,\ \hat{B}_{\textrm{εξ}}=\hat{\varDelta}\ ,\ \hat{\varGamma}_{\textrm{εξ}}=\hat{A}\ ,\ \hat{\varDelta}_{\textrm{εξ}}=\hat{B} $.
\end{rlist}
\Thewrhma{Κριτήρια εγγράψιμου τετραπλεύρου}
Ένα τετράπλευρο είναι εγγράψιμο σε έναν κύκλο αν ισχύει μια από τις παρακάτω προτάσεις :
\begin{rlist}
\item Δύο απέναντι γωνίες είναι παραπληρωματικές.
\item Μια πλευρά φαίνεται από τις απέναντι κορυφές υπό ίσες εγγεγραμμένες γωνίες.
\item Μια εξωτερική γωνία να ισούται με την απέναντι εσωτερική.
\end{rlist}
\Thewrhma{Ιδιότητες περιγεγραμμένου τετραπλεύρου}
\wrapr{-4mm}{7}{3.9cm}{-9mm}{\begin{tikzpicture}
\tkzDefPoint(0,0){O}
\tkzDefPoint(160:1){E}
\tkzDefPoint(80:1){Z}
\tkzDefPoint(20:1){H}
\tkzDefPoint(270:1){J}
\tkzTangent[at=E](O)\tkzGetPoint{e}
\tkzTangent[at=Z](O)\tkzGetPoint{z}
\tkzTangent[at=H](O)\tkzGetPoint{h}
\tkzTangent[at=J](O)\tkzGetPoint{j}
\tkzInterLL(E,e)(Z,z)\tkzGetPoint{A}
\tkzInterLL(H,h)(Z,z)\tkzGetPoint{B}
\tkzInterLL(H,h)(J,j)\tkzGetPoint{C}
\tkzInterLL(J,j)(E,e)\tkzGetPoint{D}
\tkzMarkRightAngle[size=.17](O,E,A)
\tkzMarkRightAngle[size=.17](O,Z,B)
\tkzMarkRightAngle[size=.17](O,H,C)
\tkzMarkRightAngle[size=.17](O,J,D)
\draw[pl] (O) circle (1);
\draw[pl,\xrwma](A)--(B)--(C)--(D)--cycle;
\draw (O)--(E);\draw (O)--(Z);\draw (O)--(H);\draw (O)--(J);
\tkzDrawPoints(E,Z,H,J,O,A,B,C,D)
\tkzLabelPoint[above left,xshift=1mm](O){$O$}
\tkzLabelPoint[above left](A){$A$}
\tkzLabelPoint[above right](B){$B$}
\tkzLabelPoint[right](C){$\varGamma$}
\tkzLabelPoint[left](D){$\varDelta$}
\tkzLabelPoint[left](E){$E$}
\tkzLabelPoint[above](Z){$Z$}
\tkzLabelPoint[right](H){$H$}
\tkzLabelPoint[below](J){$\varTheta$}
\end{tikzpicture}}{
Σε κάθε περιγεγραμμένο τετράπλευρο $ AB\varGamma\varDelta $ ισχύουν οι παρακάτω ιδιότητες :
\begin{rlist}
\item Οι διχοτόμοι των γωνιών του διέρχονται από το ίδιο σημείο. Το σημείο αυτό είναι κέντρο του εγγεγραμμένου κύκλου.
\item Τα αθροίσματα των απέναντι πλευρών είναι ίσα :
\[ AB+\varGamma\varDelta=A\varDelta+B\varGamma \]
\end{rlist}}\mbox{}\\\\\\
\Thewrhma{Κριτήρια περιγράψιμου τετραπλεύρου}
Ένα τετράπλευρο $ AB\varGamma\varDelta $ είναι περιγράψιμο σε κύκλο αν ισχύει μια από τις παρακάτω προτάσεις :
\begin{rlist}
\item Οι διχοτόμοι των γωνιών του διέρχονται από το ίδιο σημείο.
\item Τα αθροίσματα των απέναντι πλευρών είναι ίσα.
\end{rlist}
\end{document}
