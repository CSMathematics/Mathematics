\PassOptionsToPackage{no-math,cm-default}{fontspec}
\documentclass[twoside,nofonts,internet,shmeiwseis]{thewria}
\usepackage{amsmath}
\usepackage{xgreek}
\let\hbar\relax
\defaultfontfeatures{Mapping=tex-text,Scale=MatchLowercase}
\setmainfont[Mapping=tex-text,Numbers=Lining,Scale=1.0,BoldFont={Minion Pro Bold}]{Minion Pro}
\newfontfamily\scfont{GFS Artemisia}
\font\icon = "Webdings"
\usepackage[amsbb]{mtpro2}
\usepackage{tikz,pgfplots,tkz-euclide,enumitem}
\usetkzobj{all}
\tkzSetUpPoint[size=7,fill=white]
\xroma{red!70!black}
\newlist{rlist}{enumerate}{3}
\setlist[rlist]{itemsep=0mm,label=\roman*.}
\newlist{brlist}{enumerate}{3}
\setlist[brlist]{itemsep=0mm,label=\bf\roman*.}
\newlist{tropos}{enumerate}{3}
\setlist[tropos]{label=\bf\textit{\arabic*\textsuperscript{oς}\;Τρόπος :},leftmargin=0cm,itemindent=2.3cm,ref=\bf{\arabic*\textsuperscript{oς}\;Τρόπος}}
\newcommand{\tss}[1]{\textsuperscript{#1}}
\newcommand{\tssL}[1]{\MakeLowercase{\textsuperscript{#1}}}
\usepackage{rotating}
\usepackage{hhline}
\usepackage{multicol,multirow,gensymb,mathimatika}
\usepackage{wrap-rl}


\begin{document}
\titlos{Γεωμετρία Α΄ Λυκείου}{Παραλληλόγραμμα}{Βαρύκετρο - Ορθόκεντρο}
\orismoi
\Orismos{Βαρύκεντρο Τριγώνου}
Βαρύκεντρο ή κέντρο βάρους ενός τριγώνου ονομάζεται το σημείο τομής των τριών διαμέσων του τριγώνου.\\\\
\Orismos{Ορθόκεντρο Τριγώνου}
Ορθόκεντρο ενός τριγώνου ονομάζεται το σημείο τομής των τριών υψών ή των φορέων των υψών του τριγώνου.
\begin{center}
\begin{tabular}{p{4.2cm}cp{4.2cm}}
\begin{tikzpicture}
\tkzDefPoint(0,0){B}
\tkzDefPoint(3.5,0){C}
\tkzDefPoint(1.3,2.1){A}
\tkzDefPoint(.65,1.05){M}
\tkzDefPoint(2.4,1.05){L}
\tkzDefPoint(1.75,0){K}
\tkzDefPoint(1.6,.7){G}
\draw[pl](A)--(B)--(C)--cycle;
\draw[pl,\xrwma] (A)--(K);
\draw[pl,\xrwma] (B)--(L);
\draw[pl,\xrwma] (C)--(M);
\tkzDrawPoints(A,B,C,K,L,M,G)
\tkzLabelPoint[above](A){$A$}
\tkzLabelPoint[left](B){$B$}
\tkzLabelPoint[right](C){$\varGamma$}
\tkzLabelPoint[below](K){$K$}
\tkzLabelPoint[right](L){$\varLambda$}
\tkzLabelPoint[left](M){$M$}
\tkzLabelPoint[above,yshift=.5mm,xshift=-2.5mm](G){$\varTheta$}
\end{tikzpicture} &  & \begin{tikzpicture}
\clip (-.5,-.52) rectangle (4,2.5);
\tkzDefPoint(0,0){B}
\tkzDefPoint(3.5,0){C}
\tkzDefPoint(1.3,2.1){A}
\tkzDefPoint(.96,1.55){M}
\tkzDefPoint(1.67,1.74){L}
\tkzDefPoint(1.3,0){K}
\tkzInterLL(A,K)(B,L)\tkzGetPoint{H}
\draw[pl](A)--(B)--(C)--cycle;
\tkzDrawAltitude[draw=\xrwma](A,B)(C)
\tkzDrawAltitude[draw=\xrwma](A,C)(B)
\tkzDrawAltitude[draw=\xrwma](B,C)(A)
\tkzDrawPoints(A,B,C,K,L,M,H)
\tkzLabelPoint[above](A){$A$}
\tkzLabelPoint[left](B){$B$}
\tkzLabelPoint[right](C){$\varGamma$}
\tkzLabelPoint[below](K){$K$}
\tkzLabelPoint[right,yshift=1mm](L){$\varLambda$}
\tkzLabelPoint[left](M){$M$}
\tkzLabelPoint[right,xshift=.5mm](H){$H$}
\end{tikzpicture} \\ 
\end{tabular} 
\end{center}
\Orismos{Ορθοκεντρική τετράδα}
Ορθοκεντρική τετράδα ονομάζεται ένα σύνολο τεσσάρων σημείων για τα οποία κάθε τρίγωνο με κορυφές τρια απ' αυτά τα σημεία έχει ορθόκεντρο το τέταρτο σημείο. 
\thewrhmata
\Thewrhma{Βαρύκεντρο τριγώνου}
Οι τρεις διάμεσοι ενός τριγώνου διέρχονται από το ίδιο σημείο, το βαρύκεντρο του. Το βαρύκεντρο ισαπέχει από τις κορυφές του τριγώνου και κάθε απόσταση είναι ίση με τα $ \frac{2}{3} $ της αντίστοιχης διαμέσου.\\\\
\Thewrhma{Τρίγωνο παράλληλων ευθειών}
Οι ευθείες που διέρχονται από τις κορυφές ενός τριγώνου και είναι παράλληλες προς τις απέναντι πλευρές του, ορίζουν τρίγωνο του οποίου τα μέσα των πλευρών είναι οι κορυφές του αρχικού τριγώνου.\\\\
\Thewrhma{Ορθόκεντρο τριγώνου}
Σε κάθε τρίγωνο ισχύουν οι εξής προτάσεις :
\begin{rlist}
\item Οι φορείς των υψών ενός τριγώνου τέμνονται στο ίδιο σημείο, το ορθόκεντρο του τριγώνου.
\item Οι κορυφές του τριγώνου μαζί με το ορθόκεντρο αποτελούν ορθοκεντρική τετράδα.
\end{rlist}
\end{document}
