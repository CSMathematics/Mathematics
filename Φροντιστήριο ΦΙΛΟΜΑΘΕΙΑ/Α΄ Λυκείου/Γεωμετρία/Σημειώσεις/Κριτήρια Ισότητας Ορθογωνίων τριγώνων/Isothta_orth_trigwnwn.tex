\PassOptionsToPackage{no-math,cm-default}{fontspec}
\documentclass[twoside,nofonts,internet,shmeiwseis]{thewria}
\usepackage{amsmath}
\usepackage{xgreek}
\let\hbar\relax
\defaultfontfeatures{Mapping=tex-text,Scale=MatchLowercase}
\setmainfont[Mapping=tex-text,Numbers=Lining,Scale=1.0,BoldFont={Minion Pro Bold}]{Minion Pro}
\newfontfamily\scfont{GFS Artemisia}
\font\icon = "Webdings"
\usepackage[amsbb]{mtpro2}
\usepackage{tikz,pgfplots,tkz-euclide}
\tkzSetUpPoint[size=7,fill=white]
\xroma{red!70!black}
%------- ΣΥΣΤΗΜΑ -------------------
\usepackage{systeme,regexpatch}
\makeatletter
% change the definition of \sysdelim not to store `\left` and `\right`
\def\sysdelim#1#2{\def\SYS@delim@left{#1}\def\SYS@delim@right{#2}}
\sysdelim\{. % reinitialize

% patch the internal command to use
% \LEFTRIGHT<left delim><right delim>{<system>}
% instead of \left<left delim<system>\right<right delim>
\regexpatchcmd\SYS@systeme@iii
{\cB.\c{SYS@delim@left}(.*)\c{SYS@delim@right}\cE.}
{\c{SYS@MT@LEFTRIGHT}\cB\{\1\cE\}}
{}{}
\def\SYS@MT@LEFTRIGHT{%
\expandafter\expandafter\expandafter\LEFTRIGHT
\expandafter\SYS@delim@left\SYS@delim@right}
\makeatother
\newcommand{\synt}[2]{{\scriptsize \begin{matrix}
\times#1\\\\ \times#2
\end{matrix}}}
%----------------------------------------
%------ ΜΗΚΟΣ ΓΡΑΜΜΗΣ ΚΛΑΣΜΑΤΟΣ ---------
\DeclareRobustCommand{\frac}[3][0pt]{%
{\begingroup\hspace{#1}#2\hspace{#1}\endgroup\over\hspace{#1}#3\hspace{#1}}}
%----------------------------------------

\newlist{rlist}{enumerate}{3}
\setlist[rlist]{itemsep=0mm,label=\roman*.}
\newlist{brlist}{enumerate}{3}
\setlist[brlist]{itemsep=0mm,label=\bf\roman*.}
\newlist{tropos}{enumerate}{3}
\setlist[tropos]{label=\bf\textit{\arabic*\textsuperscript{oς}\;Τρόπος :},leftmargin=0cm,itemindent=2.3cm,ref=\bf{\arabic*\textsuperscript{oς}\;Τρόπος}}
\newcommand{\tss}[1]{\textsuperscript{#1}}
\newcommand{\tssL}[1]{\MakeLowercase{\textsuperscript{#1}}}
\usetkzobj{all}
\usepackage{hhline}
%----------- ΓΡΑΦΙΚΕΣ ΠΑΡΑΣΤΑΣΕΙΣ ---------
\pgfkeys{/pgfplots/aks_on/.style={axis lines=center,
xlabel style={at={(current axis.right of origin)},xshift=1.5ex, anchor=center},
ylabel style={at={(current axis.above origin)},yshift=1.5ex, anchor=center}}}
\pgfkeys{/pgfplots/grafikh parastash/.style={\xrwma,line width=.4mm,samples=200}}
\pgfkeys{/pgfplots/belh ar/.style={tick label style={font=\scriptsize},axis line style={-latex}}}
%-----------------------------------------
\usepackage{multicol}
\usepackage{wrap-rl}
\tkzSetUpPoint[size=7,fill=white]
\tikzstyle{pl}=[line width=0.3mm]
\tikzstyle{plm}=[line width=0.4mm]
\usepackage{gensymb,hhline}


\begin{document}
\titlos{Γεωμετρία Α΄ Λυκείου}{Τρίγωνα}{ΙΣΟΤΗΤΑ ΟΡΘΟΓΩΝΙΩΝ ΤΡΙΓΩΝΩΝ}
\thewrhmata
\Thewrhma{Μοναδικότητα καθέτου}
Από ένα σημείο που βρίσκεται εκτός μιας ευθείας διέρχεται μοναδική κάθετη προς την ευθεία.\\\\
\Thewrhma{1\tssL{ο} Κριτήριο ισότητας ορθογωνίων τριγώνων}
Δύο ορθογώνια τρίγωνα είναι ίσα αν έχουν τις κάθετες πλευρές τους ίσες μια προς μια.\\\\
\Thewrhma{2\tssL{ο} Κριτήριο ισότητας ορθογωνίων τριγώνων}
Δύο ορθογώνια τρίγωνα είναι ίσα αν έχουν μια κάθετη πλευρά ίση και μια προσκείμενη γωνία ίση.\\\\
\Thewrhma{3\tssL{ο} Κριτήριο ισότητας ορθογωνίων τριγώνων}
Δύο ορθογώνια τρίγωνα είναι ίσα αν έχουν ίσες υποτείνουσες και μια προσκείμενη οξεία γωνία ίση\\\\
\Thewrhma{4\tssL{ο} Κριτήριο ισότητας ορθογωνίων τριγώνων}
Δύο ορθογώνια τρίγωνα είναι ίσα αν έχουν μια κάθετη πλευρά ίση και ίσες υποτείνουσες.\\\\
\begin{center}
\begin{tabular}{>{\centering\arraybackslash}m{7cm}|>{\centering\arraybackslash}m{7cm}}
\hline \rule[-2ex]{0pt}{5.5ex}\textbf{1ο Κριτήριο} & \textbf{2ο Κριτήριο} \\
\hhline{==} \rule[-2ex]{0pt}{15ex}\begin{tikzpicture}
\tkzDefPoint(0,0){A}
\tkzDefPoint(2.5,0){B}
\tkzDefPoint(0,1.5){C}
\tkzDefPoint(3.5,0){D}
\tkzDefPoint(6,0){E}
\tkzDefPoint(3.5,1.5){Z}
\tkzMarkRightAngle[size=.25,fill=\xrwma](B,A,C)
\tkzMarkRightAngle[size=.25,fill=\xrwma](E,D,Z)
\draw[pl](A)--(B)--(C)--cycle;
\draw[pl](D)--(E)--(Z)--cycle;
\tkzMarkSegments[mark=|,size=3pt,color=\xrwma](A,B D,E)
\tkzMarkSegments[mark=||,size=3pt,color=\xrwma](A,C D,Z)
\tkzLabelPoint[left](A){$A$}
\tkzLabelPoint[right](B){$B$}
\tkzLabelPoint[left](C){$\varGamma$}
\tkzLabelPoint[left](D){$\varDelta$}
\tkzLabelPoint[right](E){$E$}
\tkzLabelPoint[left](Z){$Z$}
\tkzDrawPoints(A,B,C,D,E,Z)
\end{tikzpicture} & \begin{tikzpicture}
\tkzDefPoint(0,0){A}
\tkzDefPoint(2.5,0){B}
\tkzDefPoint(0,1.5){C}
\tkzDefPoint(3.5,0){D}
\tkzDefPoint(6,0){E}
\tkzDefPoint(3.5,1.5){Z}
\tkzMarkRightAngle[size=.25,fill=\xrwma](B,A,C)
\tkzMarkRightAngle[size=.25,fill=\xrwma](E,D,Z)
\tkzMarkAngle[size=.5,mark=|,fill=\xrwma](C,B,A)
\tkzMarkAngle[size=.5,mark=|,fill=\xrwma](Z,E,D)
\draw[pl](A)--(B)--(C)--cycle;
\draw[pl](D)--(E)--(Z)--cycle;
\tkzMarkSegments[mark=|,size=3pt,color=\xrwma](A,B D,E)
\tkzLabelPoint[left](A){$A$}
\tkzLabelPoint[right](B){$B$}
\tkzLabelPoint[left](C){$\varGamma$}
\tkzLabelPoint[left](D){$\varDelta$}
\tkzLabelPoint[right](E){$E$}
\tkzLabelPoint[left](Z){$Z$}
\tkzDrawPoints(A,B,C,D,E,Z)
\end{tikzpicture}\\
\hline \rule[-2ex]{0pt}{5.5ex}\textbf{3ο Κριτήριο} & \textbf{4ο Κριτήριο} \\
\hline \rule[-2ex]{0pt}{15ex}\begin{tikzpicture}
\tkzDefPoint(0,0){A}
\tkzDefPoint(2.5,0){B}
\tkzDefPoint(0,1.5){C}
\tkzDefPoint(3.5,0){D}
\tkzDefPoint(6,0){E}
\tkzDefPoint(3.5,1.5){Z}
\tkzMarkRightAngle[size=.25](B,A,C)
\tkzMarkRightAngle[size=.25](E,D,Z)
\tkzMarkAngle[size=.5,mark=|,fill=\xrwma](C,B,A)
\tkzMarkAngle[size=.5,mark=|,fill=\xrwma](Z,E,D)
\draw[pl](A)--(B)--(C)--cycle;
\draw[pl](D)--(E)--(Z)--cycle;
\tkzMarkSegments[mark=|,size=3pt,color=\xrwma](B,C E,Z)
\tkzLabelPoint[left](A){$A$}
\tkzLabelPoint[right](B){$B$}
\tkzLabelPoint[left](C){$\varGamma$}
\tkzLabelPoint[left](D){$\varDelta$}
\tkzLabelPoint[right](E){$E$}
\tkzLabelPoint[left](Z){$Z$}
\tkzDrawPoints(A,B,C,D,E,Z)
\end{tikzpicture} & \begin{tikzpicture}
\tkzDefPoint(0,0){A}
\tkzDefPoint(2.5,0){B}
\tkzDefPoint(0,1.5){C}
\tkzDefPoint(3.5,0){D}
\tkzDefPoint(6,0){E}
\tkzDefPoint(3.5,1.5){Z}
\tkzMarkRightAngle[size=.25](B,A,C)
\tkzMarkRightAngle[size=.25](E,D,Z)
\draw[pl](A)--(B)--(C)--cycle;
\draw[pl](D)--(E)--(Z)--cycle;
\tkzMarkSegments[mark=|,size=3pt,color=\xrwma](A,B D,E)
\tkzMarkSegments[mark=||,size=3pt,color=\xrwma](B,C E,Z)
\tkzLabelPoint[left](A){$A$}
\tkzLabelPoint[right](B){$B$}
\tkzLabelPoint[left](C){$\varGamma$}
\tkzLabelPoint[left](D){$\varDelta$}
\tkzLabelPoint[right](E){$E$}
\tkzLabelPoint[left](Z){$Z$}
\tkzDrawPoints(A,B,C,D,E,Z)
\end{tikzpicture}\\
\hline
\end{tabular} 
\end{center}
\Thewrhma{Συγκετρωτικά κριτήρια ισότητας ορθογωνίων τριγώνων}
Τα παραπάνω τέσσερα κριτήρια συνοψίζονται στα δύο παρακάτω γενικά κριτήρια. Δύο ορθογώνια τρίγωνα είναι ίσα αν έχουν
\begin{rlist}
\item δύο πλευρές ίσες μια προς μια.
\item μια πλευρά και μια προσκείμενη οξεία γωνία ίσες μια προς μια.
\end{rlist}
\Thewrhma{Πόρισμα Ισοσκελές τρίγωνο}
Σε κάθε ισοσκελές τρίγωνο, το ύψος από την κορυφή προς τη βάση είναι και διάμεσος και διχοτόμος.\\\\
\Thewrhma{Πόρισμα για τη χορδή και το τόξο κύκλου}
Η κάθετη από το κέντρο ενός κύκλου προς μια χορδή του, διχοτομεί τη χορδή και το αντίστοιχο τόξο της.\\\\
\Thewrhma{Διχοτόμος γωνίας}
Τα σημεία της διχοτόμου μιας γωνίας ισαπέχουν από τις πλευρές της. Αντίστροφα, κάθε σημείο που ισαπέχει από τις πλευρές μιας γωνίας θα ανήκει στη διχοτόμο της.
\begin{center}
\begin{tikzpicture}[scale=1.2]
\clip (-.5,-.5) rectangle (5.2,2.2);
\tkzDefPoint(0,0){A}
\tkzDefPoint(3,0){B}
\tkzDefPoint(2.5,1.7){C}
\draw (B) -- (A) -- (C);
\tkzDrawBisector[draw=\xrwma](B,A,C)\tkzGetPoint{a}
\tkzDefPointWith[linear,K=0.8](A,a) \tkzGetPoint{D}
\tkzDefPointBy[projection=onto A--C](D)
\tkzGetPoint{h}
\tkzDrawSegment(D,h)
\tkzMarkRightAngle[fill=\xrwma](A,h,D)
\tkzDefPointBy[projection=onto A--B](D)
\tkzGetPoint{f}
\tkzDrawSegment(D,f)
\tkzMarkRightAngle[fill=\xrwma](A,f,D)
\tkzLabelPoint[left](A){$A$}
\tkzLabelPoint[above,xshift=2mm](D){$M$}
\tkzLabelPoint[above](h){$B$}
\tkzLabelPoint[below](f){$\varGamma$}
\tkzLabelPoint[above](C){$y$}
\tkzLabelPoint[right](B){$x$}
\tkzDrawPoints(A,h,f,D)
\node at (4,.8){$ MB=M\varGamma $};
\end{tikzpicture}
\end{center}
Προκύπτει λοιπόν ότι η διχοτόμος μιας γωνίας είναι ο γεωμετρικός τόπος των σημείων του επιπέδου που ισαπέχουν από τις πλευρές της γωνίας.\\\\
\Thewrhma{Χορδή και απόστημα κύκλου}
Δύο χορδές ενός κύκλου είναι ίσες αν και μόνο αν τα αποστήματά τους είναι ίσα.
\begin{center}
\begin{tikzpicture}[scale=.9]
\tkzDefPoint[label=left:$O$](0,0){O}
\draw[pl]  (0,0) circle (2);
\coordinate [label=above right:$A$] (A) at (30:2);
\coordinate [label=above left:$B$] (B) at (120:2);
\coordinate [label=below right:$\varGamma$] (C) at (330:2);
\coordinate [label=below left:$\varDelta$] (D) at (240:2);
\tkzDefPointBy[projection=onto A--B](O)\tkzGetPoint{M}
\tkzDefPointBy[projection=onto C--D](O)\tkzGetPoint{N}
\tkzMarkRightAngle[fill=\xrwma](O,M,A)
\tkzMarkRightAngle[fill=\xrwma](C,N,O)
\tkzMarkSegments[mark=|,color=\xrwma](O,M O,N)
\tkzMarkSegments[mark=||,color=\xrwma,pos=.55](A,B C,D)
\draw[pl](A)--(B);
\draw[pl](C)--(D);
\draw[pl](O)--(M);
\draw[pl](O)--(N);
\tkzDrawPoints(O,A,B,C,D,M,N)
\end{tikzpicture}
\end{center}
\end{document}
