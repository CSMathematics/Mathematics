\documentclass[twoside,nofonts,internet,shmeiwseis]{thewria}
\usepackage[amsbb,subscriptcorrection,zswash,mtpcal,mtphrb,mtpfrak]{mtpro2}
\usepackage[no-math,cm-default]{fontspec}
\usepackage{amsmath}
\usepackage{xgreek}
\usepackage{fontawesome}
\newfontfamily{\FA}{fontawesome.otf}
\let\hbar\relax
\defaultfontfeatures{Mapping=tex-text,Scale=MatchLowercase}
\setmainfont[Mapping=tex-text,Numbers=Lining,Scale=1.0,BoldFont={Minion Pro Bold}]{Minion Pro}
\newfontfamily\scfont{GFS Artemisia}
\font\icon = "Webdings"
\usepackage{tikz,pgfplots}
\tkzSetUpPoint[size=7,fill=white]
\xroma{red!70!black}
\newlist{rlist}{enumerate}{3}
\setlist[rlist]{itemsep=0mm,label=\roman*.}
\newlist{brlist}{enumerate}{3}
\setlist[brlist]{itemsep=0mm,label=\bf\roman*.}
\newlist{tropos}{enumerate}{3}
\setlist[tropos]{label=\bf\textit{\arabic*\textsuperscript{oς}\;Τρόπος :},leftmargin=0cm,itemindent=2.3cm,ref=\bf{\arabic*\textsuperscript{oς}\;Τρόπος}}
\newcommand{\tss}[1]{\textsuperscript{#1}}
\newcommand{\tssL}[1]{\MakeLowercase{\textsuperscript{#1}}}
\usepackage{rotating}
\usepackage{hhline}
\usepackage{multicol,multirow,gensymb,mathimatika}
\usepackage{wrap-rl}
%------TIKZ - ΣΧΗΜΑΤΑ - ΓΡΑΦΙΚΕΣ ΠΑΡΑΣΤΑΣΕΙΣ ----
\usepackage{tikz}
\usepackage{tkz-euclide}
\usetkzobj{all}
\usepackage[framemethod=TikZ]{mdframed}
\usetikzlibrary{decorations.pathreplacing}
\usepackage{pgfplots}
\usetkzobj{all}
%-----------------------
\usepackage{calc}
\usepackage{hhline}
\usepackage[explicit]{titlesec}
\usepackage{graphicx}
\usepackage{multicol}
\usepackage{multirow}
\usepackage{enumitem}
\usepackage{tabularx}
\usetikzlibrary{backgrounds}
\usepackage{sectsty}
\sectionfont{\centering}
\usepackage{adjustbox}
\usepackage{mathimatika,gensymb,eurosym,wrap-rl}
\usepackage{systeme,regexpatch}
%-------- ΜΑΘΗΜΑΤΙΚΑ ΕΡΓΑΛΕΙΑ ---------
\usepackage{mathtools}
%----------------------
%-------- ΠΙΝΑΚΕΣ ---------
\usepackage{booktabs}
%----------------------
%----- ΥΠΟΛΟΓΙΣΤΗΣ ----------
\usepackage{calculator}
%----------------------------

\begin{document}
\pagenumbering{gobble}% Remove page numbers (and reset to 1)
\clearpage
\titlos{Γεωμετρία Α΄ Λυκείου}{Παραλληλόγραμμα}{Ορισμοί και Θεωρήματα}
\vspace{3cm}
\begin{center}
\begin{tikzpicture}[scale=1.7]
\tkzDefPoint(-3,-.5){D}
\tkzDefPoint(-2,1){A}
\tkzDefPoint(.5,1){B}
\tkzDefPoint(-.5,-.5){C}
\tkzMarkAngle[mark=|,size=.4,fill=\xrwma!70](A,B,C)
\tkzMarkAngle[mark=|,size=.4,fill=\xrwma!70](C,D,A)
\tkzMarkAngle[mark=||,size=.3,fill=\xrwma!70](D,A,B)
\tkzMarkAngle[mark=||,size=.3,fill=\xrwma!70](B,C,D)
\draw[pl] (-3,-0.5) -- (-2,1) -- (0.5,1) -- (-0.5,-0.5) -- cycle;
\tkzLabelPoint[above](A){$A$}
\tkzLabelPoint[above](B){$B$}
\tkzLabelPoint[below](C){$\varGamma$}
\tkzLabelPoint[below](D){$\varDelta$}
\draw[pl] (A)--(C);
\draw[pl] (B)--(D);
\tkzInterLL(A,C)(B,D)\tkzGetPoint{O}
\tkzLabelPoint[above,xshift=.3mm](O){$O$}
\tkzDrawPoints(A,B,C,D,O)
\tkzMarkSegments[mark=|,size=3pt](A,B C,D)
\tkzMarkSegments[mark=||,size=3pt](A,D C,B)
\end{tikzpicture}
\end{center}
\clearpage
\pagenumbering{arabic}
\section{Παραλληλόγραμμο - Ορθογώνιο - Ρομβος - Τετράγωνο}
\orismoi
\Orismos{Παραλληλόγραμμο}
\wrapr{-4mm}{3}{3.8cm}{-7mm}{\begin{tikzpicture}
\tkzDefPoint(-3,-.5){D}
\tkzDefPoint(-2,1){A}
\tkzDefPoint(.5,1){B}
\tkzDefPoint(-.5,-.5){C}
\tkzDefPoint(-2,1){E}
\tkzDefPoint(-2,-0.5){Z}
\tkzMarkRightAngle(C,Z,E)
\draw[pl] (-3,-0.5) -- (-2,1) -- (0.5,1) -- (-0.5,-0.5) -- cycle;
\tkzLabelPoint[above](A){$A$}
\tkzLabelPoint[above](B){$B$}
\tkzLabelPoint[below](C){$\varGamma$}
\tkzLabelPoint[below](D){$\varDelta$}
\draw[pl] (A)--(C);
\draw[pl] (B)--(D);
\draw (-2,1) -- (-2,-0.5);
\tkzInterLL(A,C)(B,D)\tkzGetPoint{O}
\tkzLabelPoint[above,xshift=.3mm](O){$O$}
\tkzDrawPoints(A,B,C,D,O)
\end{tikzpicture}}{
Παραλληλόγραμμο ονομάζεται το τετράπλευρο το οποίο έχει τις απέναντι πλευρές του ανα δύο παράλληλες.
\begin{itemize}[leftmargin=5mm]
\item Τα ευθύγραμμα τμήματα που ενώνουν τις απέναντι κορυφές του παραλληλογράμμου ονομάζονται \textbf{διαγώνιοι}.
\item Το σημείο τομής των διαγωνίων του ονομάζεται \textbf{κέντρο} του παραλληλογράμμου.
\end{itemize}}\mbox{}\\
\vspace{-1.5mm}
\begin{itemize}[itemsep=0mm,leftmargin=5mm]
\item Το ευθύγραμμο τμήμα που έχει τα άκρα του στις απέναντι πλευρές ενός παραλληλογράμμου και είναι κάθετο σ' αυτές ονομάζεται \textbf{ύψος}.
\end{itemize}\mbox{}\\
\Orismos{Ορθογώνιο Παραλληλόγραμμο}
\wrapr{-5mm}{7}{3.4cm}{-7mm}{\begin{tikzpicture}
\tkzDefPoint(0,0){D}
\tkzDefPoint(0,1.5){A}
\tkzDefPoint(3,1.5){B}
\tkzDefPoint(3,0){C}
\draw[pl] (0,0) -- (0,1.5) -- (3,1.5) -- (3,0) -- cycle;
\tkzMarkRightAngle(C,D,A)
\tkzMarkRightAngle(B,C,D)
\tkzMarkRightAngle(D,A,B)
\tkzMarkRightAngle(A,B,C)
\tkzLabelPoint[above](A){$A$}
\tkzLabelPoint[above](B){$B$}
\tkzLabelPoint[below](C){$\varGamma$}
\tkzLabelPoint[below](D){$\varDelta$}
\tkzDrawPoints(A,B,C,D)
\end{tikzpicture}}{
Ορθογώνιο ονομάζεται το παραλληλόγραμμο το οποίο έχει όλες τις γωνίες του ορθές. Ισοδύναμα μπορούμε να ορίσουμε το ορθογώνιο ως το παραλληλόγραμμο το οποίο έχει μια ορθή γωνία και κατά συνέπεια από τις ιδιότητες του παραλληλογράμμου, προκύπτουν και οι υπόλοιπες γωνίες του ορθές.}\mbox{}\\\\\\
\Orismos{Ρόμβοσ}
Ρόμβος ονομάζεται το παραλληλόγραμμο το οποίο έχει τις διαδοχικές πλευρές του μεταξύ τους ίσες.\\\\
\Orismos{Τετράγωνο}
Τετράγωνο ονομάζεται το παραλληλόγραμμο το οποίο είναι και ορθογώνιο και ρόμβος συγχρόνως.
\begin{center}
\begin{tabular}{p{5cm}cp{2.9cm}}
\begin{tikzpicture}[scale=1.4]
\tkzDefPoint(0,0.75){D}
\tkzDefPoint(1.5,1.5){A}
\tkzDefPoint(3,.75){B}
\tkzDefPoint(1.5,0){C}
\draw[pl] (0,0.75) -- (1.5,1.5) -- (3,0.75) -- (1.5,0) -- cycle;
\tkzLabelPoint[above](A){$A$}
\tkzLabelPoint[right](B){$B$}
\tkzLabelPoint[below](C){$\varGamma$}
\tkzLabelPoint[left](D){$\varDelta$}
\tkzDrawPoints(A,B,C,D)
\end{tikzpicture} & & \begin{tikzpicture}[scale=.7]
\tkzDefPoint(0,-1.5){D}
\tkzDefPoint(0,1.5){A}
\tkzDefPoint(3,1.5){B}
\tkzDefPoint(3,-1.5){C}
\tkzMarkRightAngle[scale=1.5](C,D,A)
\tkzMarkRightAngle[scale=1.5](B,C,D)
\tkzMarkRightAngle[scale=1.5](D,A,B)
\tkzMarkRightAngle[scale=1.5](A,B,C)
\draw[pl] (0,-1.5) -- (0,1.5) -- (3,1.5) -- (3,-1.5) -- cycle;
\tkzLabelPoint[above](A){$A$}
\tkzLabelPoint[above](B){$B$}
\tkzLabelPoint[below](C){$\varGamma$}
\tkzLabelPoint[below](D){$\varDelta$}
\tkzDrawPoints(A,B,C,D)
\end{tikzpicture} \\ 
\end{tabular} 
\end{center}
\Orismos{Μεσοπαράλληλος}
\wrapr{-4mm}{7}{4.5cm}{-5mm}{\begin{tikzpicture}
\draw (-2,1.5) -- (2,1.5);
\draw[\xrwma] (-2,0.75) -- (2,0.75);
\draw (-2,0) -- (2,0);
\tkzDefPoint(-1.4,0){A}
\tkzDefPoint(-1,0){B}
\tkzDefPoint(-1.4,1.5){C}
\node at (2.4,1.5) {\footnotesize$\varepsilon_1$};
\node at (2.4,0) {\footnotesize$\varepsilon_2$};
\node at (2.4,.75) {\footnotesize$\varepsilon$};
\draw (-1.4,1.5) -- (-1.4,0);
\tkzMarkSegment[mark=|,pos=.25](A,C)
\tkzMarkSegment[mark=|,pos=.75](A,C)
\tkzMarkRightAngle[size=.2](B,A,C)
\tkzLabelPoint[above left](C){$ A $}
\tkzLabelPoint[below left](A){$ B $}
\tkzLabelPoint[above left](-1.4,.75){$ \varGamma $}
\end{tikzpicture}}{
Μεσοπαράλληλος δύο παράλληλων ευθειών $ \varepsilon_1,\varepsilon_2 $ ονομάζεται ο γεωμετρικός τόπος των σημείων του ίδιου επιπέδου τα οποία έχουν ίσες αποστάσεις από τις ευθείες αυτές.
\[ \varepsilon\parallel\varepsilon_1\parallel\varepsilon_2\qquad A\varGamma=B\varGamma \]
Είναι ευθεία γραμμή, παράλληλη με τις $ \varepsilon_1,\varepsilon_2 $ και βρίσκεται στο μέσο της απόστασής τους.}\mbox{}\\\\\\
\Orismos{Βαρύκεντρο Τριγώνου}
Βαρύκεντρο ή κέντρο βάρους ενός τριγώνου ονομάζεται το σημείο τομής των τριών διαμέσων του τριγώνου.\\\\
\Orismos{Ορθόκεντρο Τριγώνου}
Ορθόκεντρο ενός τριγώνου ονομάζεται το σημείο τομής των τριών υψών ή των φορέων των υψών του τριγώνου.
\begin{center}
\begin{tabular}{p{4.2cm}cp{4.2cm}}
\begin{tikzpicture}
\tkzDefPoint(0,0){B}
\tkzDefPoint(3.5,0){C}
\tkzDefPoint(1.3,2.1){A}
\tkzDefPoint(.65,1.05){M}
\tkzDefPoint(2.4,1.05){L}
\tkzDefPoint(1.75,0){K}
\tkzDefPoint(1.6,.7){G}
\draw[pl](A)--(B)--(C)--cycle;
\draw[pl,\xrwma] (A)--(K);
\draw[pl,\xrwma] (B)--(L);
\draw[pl,\xrwma] (C)--(M);
\tkzDrawPoints(A,B,C,K,L,M,G)
\tkzLabelPoint[above](A){$A$}
\tkzLabelPoint[left](B){$B$}
\tkzLabelPoint[right](C){$\varGamma$}
\tkzLabelPoint[below](K){$K$}
\tkzLabelPoint[right](L){$\varLambda$}
\tkzLabelPoint[left](M){$M$}
\tkzLabelPoint[above,yshift=.5mm,xshift=-2.5mm](G){$\varTheta$}
\end{tikzpicture} &  & \begin{tikzpicture}
\clip (-.5,-.52) rectangle (4,2.5);
\tkzDefPoint(0,0){B}
\tkzDefPoint(3.5,0){C}
\tkzDefPoint(1.3,2.1){A}
\tkzDefPoint(.96,1.55){M}
\tkzDefPoint(1.67,1.74){L}
\tkzDefPoint(1.3,0){K}
\tkzInterLL(A,K)(B,L)\tkzGetPoint{H}
\draw[pl](A)--(B)--(C)--cycle;
\tkzDrawAltitude[draw=\xrwma](A,B)(C)
\tkzDrawAltitude[draw=\xrwma](A,C)(B)
\tkzDrawAltitude[draw=\xrwma](B,C)(A)
\tkzDrawPoints(A,B,C,K,L,M,H)
\tkzLabelPoint[above](A){$A$}
\tkzLabelPoint[left](B){$B$}
\tkzLabelPoint[right](C){$\varGamma$}
\tkzLabelPoint[below](K){$K$}
\tkzLabelPoint[right,yshift=1mm](L){$\varLambda$}
\tkzLabelPoint[left](M){$M$}
\tkzLabelPoint[right,xshift=.5mm](H){$H$}
\end{tikzpicture} \\ 
\end{tabular} 
\end{center}
\Orismos{Ορθοκεντρική τετράδα}
Ορθοκεντρική τετράδα ονομάζεται ένα σύνολο τεσσάρων σημείων για τα οποία κάθε τρίγωνο με κορυφές τρια απ' αυτά τα σημεία έχει ορθόκεντρο το τέταρτο σημείο.\\\\
\Orismos{Τραπέζιο - Ισοσκελές τραπέζιο}
Τραπέζιο ονομάζεται το τετράπλευρο το οποίο έχει δύο απέναντι πλευρές του παράλληλες.
\begin{itemize}[itemsep=0mm]
\item Οι παράλληλες πλευρές ενός τραπεζίου ονομάζονται \textbf{βάσεις} του. Οι βάσεις ενός τραπεζίου δεν είναι ίσες. Ονομάζονται \textbf{μικρή} και \textbf{μεγάλη} βάση αντίστοιχα.
\item Το ευθύγραμμο τμήμα που ενώνει τα μέσα των δύο μη παράλληλων πλευρών ενός τραπεζίου ονομάζεται \textbf{διάμεσος} του τραπεζίου.
\end{itemize}
\begin{center}
\begin{tabular}{p{3.9cm}cp{4cm}}
\begin{tikzpicture}
\tkzDefPoint(0,-1.5){D}
\tkzDefPoint(0.5,.5){A}
\tkzDefPoint(2.5,.5){B}
\tkzDefPoint(3.5,-1.5){C}
\tkzDefPoint(.25,-.5){M}
\tkzDefPoint(3,-.5){N}
\tkzDefPoint(0.9,0.5){E}
\tkzDefPoint(0.9,-1.5){Z}
\tkzMarkRightAngle(C,Z,E)
\draw (0.9,0.5) -- (0.9,-1.5);
\draw[pl] (0,-1.5) -- (0.5,0.5) -- (2.5,0.5) -- (3.5,-1.5) -- cycle;
\draw[plm,\xrwma](M)--(N);
\tkzLabelPoint[above](A){$A$}
\tkzLabelPoint[above](B){$B$}
\tkzLabelPoint[below](C){$\varGamma$}
\tkzLabelPoint[below](D){$\varDelta$}
\tkzLabelPoint[left](M){$M$}
\tkzLabelPoint[right](N){$N$}
\tkzDrawPoints(A,B,C,D,M,N)
\node at (1.5,0.7) {\footnotesize$\beta$};
\node at (1.7,-1.8) {\footnotesize$B$};
\node at (.7,-.2) {\footnotesize$ \upsilon $};
\node at (1.75,-.35) {\footnotesize$ \delta $};
\end{tikzpicture} & \hspace{.5cm} & \begin{tikzpicture}
\tkzDefPoint(0,-1.5){D}
\tkzDefPoint(0.75,.5){A}
\tkzDefPoint(2.75,.5){B}
\tkzDefPoint(3.5,-1.5){C}
\tkzDefPoint(.25,-.5){M}
\tkzDefPoint(3,-.5){N}
\tkzDefPoint(0.9,0.5){E}
\tkzDefPoint(0.9,-1.5){Z}
\tkzDrawSegment[pl](A,B)
\tkzDrawSegment[pl](C,D)
\tkzDrawSegment[plm,\xrwma](A,D)
\tkzDrawSegment[plm,\xrwma](B,C)
\tkzMarkSegment[mark=|](A,D)
\tkzMarkSegment[mark=|](B,C)
\tkzLabelPoint[above](A){$A$}
\tkzLabelPoint[above](B){$B$}
\tkzLabelPoint[below](C){$\varGamma$}
\tkzLabelPoint[below](D){$\varDelta$}
\tkzDrawPoints(A,B,C,D)
\node at (1.7,0.7) {\footnotesize$\beta$};
\node at (1.7,-1.8) {\footnotesize$B$};
\end{tikzpicture} \\ 
\end{tabular} 
\end{center}
\begin{itemize}[itemsep=0mm]
\item Το ευθύγραμμο τμήμα που είναι κάθετο στις δύο βάσεις ενός τραπεζίου ονομάζεται \textbf{ύψος} του τραπεζίου.
\item Το τραπέζιο το οποίο έχει τις μη παράλληλες πλευρές του ίσες ονομάζεται \textbf{ισοσκελές τραπέζιο}. Συμβολίζεται με $ \delta $.
\end{itemize}
\thewrhmata
\Thewrhma{Ιδιότητες παραλληλογράμμου}
Σε κάθε παραλληλόγραμμο $ AB\varGamma\varDelta $ ισχύει ότι :\\
\wrapr{-11mm}{7}{5cm}{0mm}{\begin{tikzpicture}[scale=1.3]
\tkzDefPoint(-3,-.5){D}
\tkzDefPoint(-2,1){A}
\tkzDefPoint(.5,1){B}
\tkzDefPoint(-.5,-.5){C}
\tkzMarkAngle[mark=|,size=.4,fill=\xrwma!70](A,B,C)
\tkzMarkAngle[mark=|,size=.4,fill=\xrwma!70](C,D,A)
\tkzMarkAngle[mark=||,size=.3,fill=\xrwma!70](D,A,B)
\tkzMarkAngle[mark=||,size=.3,fill=\xrwma!70](B,C,D)
\draw[pl] (-3,-0.5) -- (-2,1) -- (0.5,1) -- (-0.5,-0.5) -- cycle;
\tkzLabelPoint[above](A){$A$}
\tkzLabelPoint[above](B){$B$}
\tkzLabelPoint[below](C){$\varGamma$}
\tkzLabelPoint[below](D){$\varDelta$}
\draw[pl] (A)--(C);
\draw[pl] (B)--(D);
\tkzInterLL(A,C)(B,D)\tkzGetPoint{O}
\tkzLabelPoint[above,xshift=.3mm](O){$O$}
\tkzDrawPoints(A,B,C,D,O)
\tkzMarkSegments[mark=|,size=3pt](A,B C,D)
\tkzMarkSegments[mark=||,size=3pt](A,D C,B)
\end{tikzpicture}}{
\begin{rlist}
\item Οι απέναντι πλευρές του είναι ίσες : $ AB=\varGamma\varDelta $ και $ A\varDelta=B\varGamma $.
\item Οι απέναντι γωνίες του είναι ίσες : $ \hat{A}=\hat{\varGamma} $ και $ \hat{B}=\hat{\varDelta} $.
\item Δύο διαδοχικές γωνίες του είναι παραπληρωματικές : $ \hat{A}+\hat{B}=180\degree $.
\item Οι διαγώνιοι διχοτομούνται.
\end{rlist}}\mbox{}\\\\\\
\Thewrhma{Κριτήρια Παραλληλογράμμου}
Ένα τετράπλευρο $ AB\varGamma\varDelta $ θα είναι παραλληλόγραμμο αν ισχύει μια από τις παρακάτω προτάσεις :
\begin{rlist}
\item Οι απέναντι πλευρές του είναι παράλληλες.
\item Οι απέναντι πλευρές του είναι ίσες.
\item Δύο απέναντι πλευρές του είναι παράλληλες και ίσες.
\item Οι απέναντι γωνίες του είναι ίσες.
\item Οι διαγώνιοί του διχοτομούνται.
\end{rlist}
\Thewrhma{Πορίσματα για το παραλληλόγραμμο}
\vspace{-5mm}
\begin{rlist}
\item Το κέντρο ενός παραλληλογράμμου $ AB\varGamma\varDelta $ είναι κέντρο συμμετρίας του.
\item Εάν δύο ή περισσότερα παράλληλα τμήματα έχουν τα άκρα τους πάνω σε παράλληλες ευθείες τότε είναι ίσα.
\end{rlist}
\Thewrhma{Ιδιότητες ορθογωνίου}
Σε κάθε ορθογώνιο $ AB\varGamma\varDelta $ ισχύουν οι παρακάτω προτάσεις :\\
\wrapr{-11mm}{7}{4.6cm}{-4mm}{\begin{tikzpicture}[scale=1.2]
\tkzDefPoint(0,0){D}
\tkzDefPoint(0,1.5){A}
\tkzDefPoint(3,1.5){B}
\tkzDefPoint(3,0){C}
\tkzDefPoint(1.5,.75){O}
\draw[pl] (0,0) -- (0,1.5) -- (3,1.5) -- (3,0) -- cycle;
\draw[pl] (A)--(C);
\draw[pl] (B)--(D);
\tkzMarkRightAngle[fill=\xrwma](C,D,A)
\tkzMarkRightAngle[fill=\xrwma](B,C,D)
\tkzMarkRightAngle[fill=\xrwma](D,A,B)
\tkzMarkRightAngle[fill=\xrwma](A,B,C)
\tkzLabelPoint[above left](A){$A$}
\tkzLabelPoint[above right](B){$B$}
\tkzLabelPoint[right](C){$\varGamma$}
\tkzLabelPoint[left](D){$\varDelta$}
\tkzLabelPoint[above](O){$O$}
\tkzDrawPoints(A,B,C,D,O)
\tkzMarkSegments[mark=|,size=3pt](A,B B,C C,D D,A)
\tkzMarkSegments[mark=||,pos=.55,size=3pt](A,C B,D)
\end{tikzpicture}}{
\begin{rlist}
\item Οι διαγώνιοι του είναι ίσες : $ A\varGamma=B\varDelta $.
\item Όλες του οι γωνίες είναι ίσες : $ \hat{A}=\hat{B}=\hat{\varGamma}=\hat{\varDelta}=90\degree $.
\item Έχει όλες τις ιδιότητες ενός παραλληλογράμμου.
\end{rlist}}\mbox{}\\\\
\Thewrhma{Κριτήρια ορθογωνίου}
Ένα τετράπλευρο $ AB\varGamma\varDelta $ είναι ορθογώνιο αν ισχύει μια από τις παρακάτω προτάσεις :
\begin{rlist}
\item Είναι παραλληλόγραμμο και έχει μια ορθή γωνία.
\item Είναι παραλληλόγραμμο και οι διαγώνιοί του είναι ίσες.
\item Έχει 3 ορθές γωνίες.
\item Έχει όλες τις γωνίες του ίσες.
\end{rlist}
\Thewrhma{Ιδιότητες ρόμβου}
Σε κάθε ρόμβο $ AB\varGamma\varDelta $ ισχύουν οι παρακάτω προτάσεις.\\
\wrapr{-11mm}{5}{5.2cm}{-4mm}{\begin{tikzpicture}[scale=.7]
\tkzDefPoint(0,1.5){D}
\tkzDefPoint(3,3){A}
\tkzDefPoint(6,1.5){B}
\tkzDefPoint(3,0){C}
\tkzDefPoint(3,1.5){O}
\tkzMarkRightAngle[size=.4](B,O,A)
\tkzMarkAngle[size=.7,mark=|,fill=\xrwma](B,D,A)
\tkzMarkAngle[size=.7,mark=|,fill=\xrwma](C,D,B)
\tkzMarkAngle[size=.7,mark=|,fill=\xrwma](A,B,D)
\tkzMarkAngle[size=.7,mark=|,fill=\xrwma](D,B,C)
\tkzMarkAngle[size=.5,mark=||,fill=\xrwma](D,A,C)
\tkzMarkAngle[size=.5,mark=||,fill=\xrwma](C,A,B)
\tkzMarkAngle[size=.5,mark=||,fill=\xrwma](B,C,A)
\tkzMarkAngle[size=.5,mark=||,fill=\xrwma](A,C,D)
\draw[pl] (A)--(B)--(C)--(D) -- cycle;
\draw[pl] (A)--(C);
\draw[pl] (B)--(D);
\tkzLabelPoint[above](A){$A$}
\tkzLabelPoint[right](B){$B$}
\tkzLabelPoint[below](C){$\varGamma$}
\tkzLabelPoint[left](D){$\varDelta$}
\tkzLabelPoint[above left](O){$O$}
\tkzDrawPoints(A,B,C,D,O)
\tkzMarkSegments[mark=|,size=3pt](A,B B,C C,D D,A)
\end{tikzpicture}}{
\begin{rlist}
\item Οι διαδοχικές πλευρές του είναι ίσες : $ AB=B\varGamma=\varGamma\varDelta=\varDelta A $.
\item Οι διαγώνιοί του τέμνονται κάθετα : $ A\varGamma\bot B\varDelta $.
\item Οι διαγώνιοί του διχοτομούν τις γωνίες του :
\begin{multicols}{2}
\begin{itemize}[itemsep=0mm]
\item $ A\varGamma $ διχ. των $ \hat{A} $ και $ \hat{\varGamma} $.
\item $ B\varDelta $ διχ. των $ \hat{B} $ και $ \hat{\varDelta} $.
\end{itemize}
\end{multicols}
\vspace{-3mm}
\item Έχει όλες τις ιδιότητες ενός παραλληλογράμμου.
\end{rlist}}\mbox{}\\\\\\
\Thewrhma{Κριτήρια ρόμβου}
Ένα τετράπλευρο $ AB\varGamma\varDelta $ είναι ρόμβος αν ισχύει μια από τις παρακάτω προτάσεις :
\begin{rlist}
\item Όλες οι πλευρές του είναι ίσες.
\item Είναι παραλληλόγραμμο και έχει δύο διαδοχικές πλευρές ίσες.
\item Είναι παραλληλόγραμμο και έχει διαγώνιους κάθετες.
\item Είναι παραλληλόγραμμο και μια διαγώνιος διχοτομεί μια γωνία.
\end{rlist}
\Thewrhma{Ιδιότητες τετραγώνου}
\wrapr{-4mm}{8}{3.5cm}{-7mm}{\begin{tikzpicture}[scale=1]
\tkzDefPoint(0,-1.5){D}
\tkzDefPoint(0,1.5){A}
\tkzDefPoint(3,1.5){B}
\tkzDefPoint(3,-1.5){C}
\tkzDefPoint(1.5,0){O}
\tkzMarkRightAngle[scale=1.5,fill=\xrwma](C,D,A)
\tkzMarkRightAngle[scale=1.5,fill=\xrwma](B,C,D)
\tkzMarkRightAngle[scale=1.5,fill=\xrwma](D,A,B)
\tkzMarkRightAngle[scale=1.5,fill=\xrwma](A,B,C)
\draw[pl] (A) -- (B) -- (C) -- (D) -- cycle;
\draw[pl] (A)--(C);
\draw[pl] (B)--(D);
\tkzLabelPoint[above](A){$A$}
\tkzLabelPoint[above](B){$B$}
\tkzLabelPoint[below](C){$\varGamma$}
\tkzLabelPoint[below](D){$\varDelta$}
\tkzLabelPoint[above](O){$O$}
\tkzDrawPoints(A,B,C,D,O)
\tkzMarkSegments[mark=|,size=3pt](A,B B,C C,D D,A)
\tkzMarkSegments[mark=||,pos=.55,size=3pt](A,C B,D)
\end{tikzpicture}}{
Κάθε τετράγωνο $ AB\varGamma\varDelta $ έχει όλες τις ιδιότητες του παρραληλογράμμου, του ορθογωνίου και του ρόμβου :
\begin{rlist}
\item Όλες οι πλευρές του είναι ίσες : $ AB=B\varGamma=\varGamma\varDelta=A\varDelta $.
\item Όλες οι γωνίες του είναι ίσες : $ \hat{A}=\hat{B}=\hat{\varGamma}=\hat{\varDelta}=90\degree $.
\item Οι απέναντι πλευρές είναι παράλληλες : $ AB\parallel\varGamma\varDelta\ ,\ A\varDelta\parallel B\varGamma $.
\item Οι διαγώνιοί του είναι ίσες,διχοτομούνται , διχοτομούν τις γωνίες του και τέμνονται κάθετα.
\begin{multicols}{2}
\begin{itemize}
\item $ A\varGamma=B\varDelta $ και $ A\varGamma\bot B\varDelta $.
\item $ AO=O\varGamma\ ,\ BO=O\varDelta $.
\item $ A\varGamma $ διχ. των $ \hat{A} $ και $ \hat{\varGamma} $.
\item $ B\varDelta $ διχ. των $ \hat{B} $ και $ \hat{\varDelta} $.
\end{itemize}
\end{multicols}
\end{rlist}}\mbox{}\\\\\\
\Thewrhma{Κριτήρια τετραγώνου}
Ένα τετράπλευρο $ AB\varGamma\varDelta $ είναι τετράγωνο εάν είναι παραλληλόγραμμο και ισχύει και μια από τις παρακάτω προτάσεις :
\begin{rlist}
\item Έχει μια ορθή γωνία και δύο διαδοχικές πλευρές ίσες.
\item Έχει μια ορθή γωνία και διαγώνιους κάθετες.
\item Έχει μια ορθή γωνία και μια διαγώνιος διχοτομεί μια γωνία.
\item Έχει διαγώνιους ίσες και κάθετες.
\item Έχει διαγώνιους ίσες και δύο διαδοχικές πλευρές ίσες.
\item Έχει διαγώνιους ίσες και μια απ' αυτές διχοτομεί μια γωνία.
\end{rlist}
Από τα παραπάνω κριτήρια παρατηρούμε ότι συνδυάζονται δύο ιδιότητες του ορθογωνίου με τρεις ιδιότητες του ρόμβου προκειμένου να οριστούν τα κριτήρια αυτά. Οι συνδυασμοί αυτοί φαίνονται στον παρακάτω πίνακα.
\begin{center}
\begin{tabular}{c|c|c|c}
\hline \multicolumn{4}{c}{\textbf{{\boldmath$ AB\varGamma\varDelta $} Παραλληλόγραμμο και}}  \rule[-2ex]{0pt}{5.5ex}\\ 
\hhline{====} \multicolumn{2}{c|}{} & \multicolumn{2}{c}{\textbf{Ιδιότητες Ορθογωνίου}}  \rule[-2ex]{0pt}{5.5ex}\\ 
\hhline{~~|--}  \multicolumn{2}{c|}{}  & Μια ορθή γωνία & Διαγώνιοι ίσες \rule[-2ex]{0pt}{5.5ex}\\ 
\hline \multirow{5}{*}{\textbf{Ιδιότητες ρόμβου}} & Διαδοχικές πλευρές ίσες & 1ο Κριτήριο & 4ο Κριτήριο \rule[-2ex]{0pt}{5.5ex}\\ 
\hhline{~-|--} \rule[-2ex]{0pt}{5.5ex} & Διαγώνιοι κάθετες & 2ο Κριτήριο & 5ο Κριτήριο \\ 
\hhline{~---} \rule[-2ex]{0pt}{5.5ex} & Διαγώνιος διχοτομεί μια γωνία & 3ο Κριτήριο & 6ο Κριτήριο \\ 
\hline 
\end{tabular} 
\end{center}
\Thewrhma{Τμήμα από τα μέσα δύο πλευρών}
Το ευθύγραμμο τμήμα που ενώνει τα μέσα των δύο πλευρών ενός τριγώνου είναι παράλληλο με την τρίτη πλευρά και ισούται με το μισό της.
\[ MN\parallel=\frac{B\varGamma}{2} \]
για ένα τρίγωνο $ AB\varGamma $ με $ M,N $ τα μέσα των πλευρών $ AB,A\varGamma $ αντίστοιχα.
\begin{center}
\begin{tikzpicture}
\tkzDefPoint(0,0){B}
\tkzDefPoint(3.5,0){C}
\tkzDefPoint(1.2,2){A}
\tkzDefPoint(.6,1){M}
\tkzDefPoint(2.35,1){N}
\draw[pl](A)--(B)--(C)--cycle;
\draw[dashed] (0,1)--(3.3,1);
\draw[plm,\xrwma] (M)--(N);
\tkzDrawPoints(A,B,C,M,N)
\tkzLabelPoint[above](A){$A$}
\tkzLabelPoint[left](B){$B$}
\tkzLabelPoint[right](C){$\varGamma$}
\tkzLabelPoint[left,yshift=2mm](M){$M$}
\tkzLabelPoint[right,yshift=2mm](N){$N$}
\end{tikzpicture}
\end{center}
\Thewrhma{Τμήμα παράλληλο από μέσο}
Η ευθεία που διέρχεται από το μέσο μιας πλευράς ενός τριγώνου και είναι παράλληλη προς μια δεύτερη πλευρά, θα διέρχεται και από το μέσο της τρίτης πλευράς.
\[ M\textrm{ μέσο }AB\textrm{ και }MN\parallel B\varGamma\Rightarrow N\textrm{ μέσο }A\varGamma \]
\Thewrhma{Ίσα τμήματα από παράλληλες ευθείες}
\wrapr{-4mm}{5}{4.5cm}{-8mm}{\begin{tikzpicture}[scale=1.3]
\tkzDefPoint(0,0){A}
\tkzDefPoint(3,0){B}
\tkzDefPoint(0,.5){C}
\tkzDefPoint(3,0.5){D}
\tkzDefPoint(0,1){E}
\tkzDefPoint(3,1){Z}
\tkzDefPoint(.7,1.3){H}
\tkzDefPoint(.4,-.3){I}
\tkzDefPoint(1.7,1.3){K}
\tkzDefPoint(2.7,-.3){L}
\draw[pl] (A)--(B);
\draw[pl] (C)--(D);
\draw[pl] (E)--(Z);
\draw[pl,\xrwma] (H)--(I);
\draw[pl,\xrwma] (K)--(L);
\tkzInterLL(E,Z)(H,I)\tkzGetPoint{S}
\tkzInterLL(C,D)(H,I)\tkzGetPoint{T}
\tkzInterLL(A,B)(H,I)\tkzGetPoint{Y}
\tkzInterLL(E,Z)(K,L)\tkzGetPoint{O}
\tkzInterLL(C,D)(K,L)\tkzGetPoint{P}
\tkzInterLL(A,B)(K,L)\tkzGetPoint{Q}
\tkzDrawPoints(S,T,Y,O,P,Q)
\tkzLabelPoint[above left](S){$A$}
\tkzLabelPoint[above left](T){$B$}
\tkzLabelPoint[above left](Y){$\varGamma$}
\tkzLabelPoint[above right](O){$\varDelta$}
\tkzLabelPoint[above right](P){$E$}
\tkzLabelPoint[above right](Q){$Z$}
\node at (3.2,1) {\footnotesize$\varepsilon_1$};
\node at (3.2,0.5) {\footnotesize$\varepsilon_2$};
\node at (3.2,0) {\footnotesize$\varepsilon_3$};
\node at (0.6,-0.2) {\footnotesize$\varepsilon$};
\node at (2.4,-0.2) {\footnotesize$\zeta$};
\end{tikzpicture}}{
Αν τρεις ή περισσότερες παράλληλες ευθείες ορίζουν ίσα τμήματα σε μια τέμνουσα, τότε θα ορίζουν ίσα τμήματα και σε οποιαδήποτε άλλη τέμνουσα ευθεία.
\[ \varepsilon_1\parallel\varepsilon_2\parallel\varepsilon_3\textrm{ και }AB=B\varGamma\Rightarrow\varDelta E=EZ \]
}\mbox{}\\\\\\
\Thewrhma{Βαρύκεντρο τριγώνου}
Οι τρεις διάμεσοι ενός τριγώνου διέρχονται από το ίδιο σημείο, το βαρύκεντρο του. Το βαρύκεντρο ισαπέχει από τις κορυφές του τριγώνου και κάθε απόσταση είναι ίση με τα $ \frac{2}{3} $ της αντίστοιχης διαμέσου.\\\\
\Thewrhma{Τρίγωνο παράλληλων ευθειών}
Οι ευθείες που διέρχονται από τις κορυφές ενός τριγώνου και είναι παράλληλες προς τις απέναντι πλευρές του, ορίζουν τρίγωνο του οποίου τα μέσα των πλευρών είναι οι κορυφές του αρχικού τριγώνου.\\\\
\Thewrhma{Ορθόκεντρο τριγώνου}
Σε κάθε τρίγωνο ισχύουν οι εξής προτάσεις :
\begin{rlist}
\item Οι φορείς των υψών ενός τριγώνου τέμνονται στο ίδιο σημείο, το ορθόκεντρο του τριγώνου.
\item Οι κορυφές του τριγώνου μαζί με το ορθόκεντρο αποτελούν ορθοκεντρική τετράδα.
\end{rlist}
\Thewrhma{Διάμεσος από ορθή γωνία}
\wrapr{-4mm}{7}{3cm}{-4mm}{\begin{tikzpicture}
\tkzDefPoint(0,0){A}
\tkzDefPoint(2,0){B}
\tkzDefPoint(0,3){C}
\tkzDefPoint(1,1.5){M}
\draw[pl,\xrwma](A)--(M);
\tkzMarkRightAngle[size=.3](B,A,C)
\draw[pl](A)--(B)--(C)--cycle;
\tkzMarkSegments[mark=|](C,M M,B M,A)
\tkzLabelPoint[left](A){$A$}
\tkzLabelPoint[right](B){$B$}
\tkzLabelPoint[left](C){$\varGamma$}
\tkzLabelPoint[right,yshift=1mm](M){$M$}
\tkzDrawPoints(A,B,C,M)
\end{tikzpicture}}{
Σε κάθε ορθογώνιο τρίγωνο ισχύουν οι εξής προτάσεις που αφορούν τη διάμεσο που αντιστοιχεί στην υποτείνουσα.
\begin{rlist}
\item Η διάμεσος που άγεται από την ορθή γωνία προς την υποτείνουσα σε κάθε ορθογώνιο τρίγωνο, ισούται με το μισό της υποτείνουσας.
\item (Αντίστροφο) Αν σε ένα τρίγωνο, μια διάμεσος ισούται με τη μισή πλευρά στην οποία αντιστοιχεί, τότε το τρίγωνο είναι ορθογώνιο με υποτείνουσα την πελυρά αυτή.
\end{rlist}}\mbox{}\\\\\\
\Thewrhma{Ορθογώνιο με γωνία {$ \mathbold{30\degree} $}}
Σε ένα ορθογώνιο τρίγωνο μια οξεία γωνία ισούται με $ 30\degree $ αν και μόνο αν η απέναντι κάθετη πλευρά είναι ίση με τη μισή υποτείνουσα.\\\\
\Thewrhma{Πορίσματα για τη διάμεσο}
Έστω ένα τραπέζιο $ AB\varGamma\varDelta $ με $ AB\parallel\varGamma\varDelta $ και $ \varGamma\varDelta>AB $ ενώ $ M,N $ είναι τα μέσα των μη παράλληλων πλευρών. Επίσης $ E,Z $ ορίζουμε τα μέσα των διαγωνίων $ A\varGamma,B\varDelta $. Ισχύουν οι παρακάτω προτάσεις :\\
\wrapr{-11mm}{8}{4cm}{3mm}{\begin{tikzpicture}
\tkzDefPoint(0,-1.5){D}
\tkzDefPoint(0.5,.5){A}
\tkzDefPoint(2.5,.5){B}
\tkzDefPoint(3.5,-1.5){C}
\tkzDefPoint(.25,-.5){M}
\tkzDefPoint(3,-.5){N}
\draw[pl] (0,-1.5) -- (0.5,0.5) -- (2.5,0.5) -- (3.5,-1.5) -- cycle;
\draw[plm,\xrwma](M)--(N);
\draw[pl] (A)--(C);
\draw[pl] (B)--(D);
\tkzInterLL(M,N)(B,D) \tkzGetPoint{E}
\tkzInterLL(M,N)(A,C) \tkzGetPoint{Z}
\tkzLabelPoint[above](A){$A$}
\tkzLabelPoint[above](B){$B$}
\tkzLabelPoint[below](C){$\varGamma$}
\tkzLabelPoint[below](D){$\varDelta$}
\tkzLabelPoint[left](M){$M$}
\tkzLabelPoint[right](N){$N$}
\tkzLabelPoint[above left](E){$E$}
\tkzLabelPoint[above right](Z){$Z$}
\tkzDrawPoints(A,B,C,D,M,N,E,Z)
\node at (1.75,-.77) {\footnotesize$ \delta $};
\end{tikzpicture}}{
\begin{rlist}
\item Η διάμεσος $ MN $ του τραπεζίου είναι παράλληλη με τις βάσεις $ AB,\varGamma\varDelta $ και ίση με το ημιάθροισμά τους.
\[ \delta=MN=\frac{AB+\varGamma\varDelta}{2} \]
\item Το ευθύγραμμο τμήμα $ EZ $ που ενώνει τα μέσα των διαγωνίων $ A\varGamma,B\varDelta $ είναι παράλληλο με τις βάσεις και ίσο με την ημιδιαφορά τους.
\[ EZ=\frac{\varGamma\varDelta-AB}{2} \]
\end{rlist}}\mbox{}\\\\\\
\Thewrhma{Ιδιότητες ισοσκελούς τραπεζίου}
Σε κάθε ισοσκελές τραπέζιο $ AB\varGamma\varDelta $ με $ AB\parallel\varGamma\varDelta $ ισχύουν οι παρακάτω προτάσεις :
\begin{rlist}
\item Οι προσκείμενες σε κάθε βάση γωνίες είναι ίσες  : $ \hat{A}=\hat{B} $ ή $ \hat{\varGamma}=\hat{\varDelta} $.
\item Οι διαγώνιοί του είναι ίσες : $ A\varGamma=B\varDelta $.
\end{rlist}
\vspace{-7mm}
\begin{center}
\begin{tikzpicture}
\tkzDefPoint(0,-1.5){D}
\tkzDefPoint(0.75,.5){A}
\tkzDefPoint(2.75,.5){B}
\tkzDefPoint(3.5,-1.5){C}
\tkzDefPoint(.25,-.5){M}
\tkzDefPoint(3,-.5){N}
\tkzDefPoint(0.9,0.5){E}
\tkzDefPoint(0.9,-1.5){Z}
\tkzMarkAngle[size=.4,mark=|,fill=\xrwma](D,A,B)
\tkzMarkAngle[size=.4,mark=|,fill=\xrwma](A,B,C)
\tkzMarkAngle[size=.4,mark=||,fill=\xrwma](C,D,A)
\tkzMarkAngle[size=.4,mark=||,fill=\xrwma](B,C,D)
\tkzDrawSegment[pl](A,B)
\tkzDrawSegment[pl](C,D)
\tkzDrawSegment[plm](A,D)
\tkzDrawSegment[plm](B,C)
\draw[pl,\xrwma] (A)--(C);
\draw[pl,\xrwma] (B)--(D);
\tkzMarkSegments[mark=|](A,D B,C)
\tkzMarkSegments[mark=|](A,C B,D)
\tkzLabelPoint[above](A){$A$}
\tkzLabelPoint[above](B){$B$}
\tkzLabelPoint[below](C){$\varGamma$}
\tkzLabelPoint[below](D){$\varDelta$}
\tkzDrawPoints(A,B,C,D)
\node at (1.7,0.7) {\footnotesize$\beta$};
\node at (1.7,-1.8) {\footnotesize$B$};
\end{tikzpicture}
\end{center}
\Thewrhma{Κριτήριο για ισοσκελές τραπέζιο}
Ένα τραπέζιο $ AB\varGamma\varDelta $ με $ AB\parallel\varGamma\varDelta $ θα είναι ισοσκελές αν ισχύει μια από τις προτάσεις :
\begin{rlist}
\item Οι προσκείμενες γωνίες μιας βάσης είναι ίσες.
\item Οι διχοτόμοι είναι ίσες.
\end{rlist}
\newpage
\begin{sidewaysfigure}
\begin{tabular}{c|>{\centering\arraybackslash}m{5.3cm}|>{\centering\arraybackslash}m{4.5cm}|>{\centering\arraybackslash}m{5.5cm}|>{\centering\arraybackslash}m{6.5cm}}
\hline\rule[-2ex]{0pt}{5.5ex}& \textbf{Παραλληλόγραμμο} & \textbf{Ορθογώνιο} & \textbf{Ρόμβος} & \textbf{Τετράγωνο} \\
\hhline{=====}\rule[-2ex]{0pt}{5.5ex}\textbf{Σχήμα}  & \begin{tikzpicture}
\tkzDefPoint(-3,-.5){D}
\tkzDefPoint(-2,1){A}
\tkzDefPoint(.5,1){B}
\tkzDefPoint(-.5,-.5){C}
\tkzDefPoint(-2,1){E}
\draw[pl] (-3,-0.5) -- (-2,1) -- (0.5,1) -- (-0.5,-0.5) -- cycle;
\tkzLabelPoint[above](A){$A$}
\tkzLabelPoint[above](B){$B$}
\tkzLabelPoint[below](C){$\varGamma$}
\tkzLabelPoint[below](D){$\varDelta$}
\draw[pl] (A)--(C);
\draw[pl] (B)--(D);
\tkzInterLL(A,C)(B,D)\tkzGetPoint{O}
\tkzLabelPoint[above,xshift=.3mm](O){$O$}
\tkzDrawPoints(A,B,C,D,O)
\end{tikzpicture} & \begin{tikzpicture}[scale=1]
\tkzDefPoint(0,0){D}
\tkzDefPoint(0,1.8){A}
\tkzDefPoint(3,1.8){B}
\tkzDefPoint(3,0){C}
\tkzDefPoint(1.5,.9){O}
\draw[pl] (0,0) -- (0,1.8) -- (3,1.8) -- (3,0) -- cycle;
\draw[pl] (A)--(C);
\draw[pl] (B)--(D);
\tkzMarkRightAngle(C,D,A)
\tkzMarkRightAngle(B,C,D)
\tkzMarkRightAngle(D,A,B)
\tkzMarkRightAngle(A,B,C)
\tkzLabelPoint[above left](A){$A$}
\tkzLabelPoint[above right](B){$B$}
\tkzLabelPoint[right](C){$\varGamma$}
\tkzLabelPoint[left](D){$\varDelta$}
\tkzLabelPoint[above](O){$O$}
\tkzDrawPoints(A,B,C,D,O)
\end{tikzpicture} & \begin{tikzpicture}[scale=.7]
\tkzDefPoint(0,1.5){D}
\tkzDefPoint(3,3){A}
\tkzDefPoint(6,1.5){B}
\tkzDefPoint(3,0){C}
\tkzDefPoint(3,1.5){O}
\tkzMarkRightAngle[size=.4](B,O,A)
\tkzMarkAngle[size=.7](B,D,A)
\tkzMarkAngle[size=.7](C,D,B)
\tkzMarkAngle[size=.7](A,B,D)
\tkzMarkAngle[size=.7](D,B,C)
\tkzMarkAngle[size=.5](D,A,C)
\tkzMarkAngle[size=.5](C,A,B)
\tkzMarkAngle[size=.5](B,C,A)
\tkzMarkAngle[size=.5](A,C,D)
\draw[pl] (A)--(B)--(C)--(D) -- cycle;
\draw[pl] (A)--(C);
\draw[pl] (B)--(D);
\tkzLabelPoint[above](A){$A$}
\tkzLabelPoint[right](B){$B$}
\tkzLabelPoint[below](C){$\varGamma$}
\tkzLabelPoint[left](D){$\varDelta$}
\tkzLabelPoint[above left](O){$O$}
\tkzDrawPoints(A,B,C,D,O)
\end{tikzpicture} & \begin{tikzpicture}[scale=.7]
\tkzDefPoint(0,-1.5){D}
\tkzDefPoint(0,1.5){A}
\tkzDefPoint(3,1.5){B}
\tkzDefPoint(3,-1.5){C}
\tkzDefPoint(1.5,0){O}
\tkzMarkRightAngle[scale=1.5](C,D,A)
\tkzMarkRightAngle[scale=1.5](B,C,D)
\tkzMarkRightAngle[scale=1.5](D,A,B)
\tkzMarkRightAngle[scale=1.5](A,B,C)
\draw[pl] (A) -- (B) -- (C) -- (D) -- cycle;
\draw[pl] (A)--(C);
\draw[pl] (B)--(D);
\tkzLabelPoint[above](A){$A$}
\tkzLabelPoint[above](B){$B$}
\tkzLabelPoint[below](C){$\varGamma$}
\tkzLabelPoint[below](D){$\varDelta$}
\tkzLabelPoint[above](O){$O$}
\tkzDrawPoints(A,B,C,D,O)
\end{tikzpicture} \\
\hline\rule[-7ex]{0pt}{14ex}\textbf{Ορισμός}  & Παραλληλόγραμμο ονομάζεται το τετράπλευρο το οποίο έχει τις απέναντι πλευρές του ανά δύο παράλληλες. & Ορθογώνιο ονομάζεται το παραλληλόγραμμο το οποίο έχει όλες τις γωνίες του ορθές. & Ρόμβος ονομάζεται το παραλληλόγραμμο το οποίο έχει τις διαδοχικές πλευρές του μεταξύ τους ίσες. & Τετράγωνο ονομάζεται το παραλληλόγραμμο το οποίο είναι και ορθογώνιο και ρόμβος. \\
\hline \textbf{Ιδιότητες} & \begin{rlist}[leftmargin=5mm]
\item Οι απέναντι πλευρές είναι ίσες.
\item Οι απέναντι γωνίες είναι ίσες.
\item Δύο διαδοχικές γωνίες είναι παραπληρωματικές.
\item Οι διαγώνιοι διχοτομούνται.
\end{rlist} & \begin{rlist}[leftmargin=5mm]
\item Οι διαγώνιοι είναι ίσες.
\item Όλες οι γωνίες είναι ίσες.
\item Έχει όλες τις ιδιότητες ενός παραλληλογράμμου.
\end{rlist} & \begin{rlist}[leftmargin=5mm]
\item Οι διαδοχικές πλευρές είναι ίσες.
\item Οι διαγώνιοι τέμνονται κάθετα.
\item Οι διαγώνιοι διχοτομούν τις γωνίες του.
\vspace{-3mm}
\item Έχει όλες τις ιδιότητες ενός παραλληλογράμμου.
\end{rlist} & \begin{rlist}[leftmargin=5mm]
\item Όλες οι πλευρές είναι ίσες.
\item Όλες οι γωνίες είναι ίσες.
\item Οι απέναντι πλευρές είναι παράλληλες.
\item Οι διαγώνιοι είναι ίσες, διχοτομούν τις γωνίες του και τέμνονται κάθετα.
\end{rlist} \\
\hline\rule[-2ex]{0pt}{5.5ex}\textbf{Κριτήρια}  & \begin{rlist}[leftmargin=5mm]
\item Οι απέναντι πλευρές είναι παράλληλες.
\item Οι απέναντι πλευρές είναι ίσες.
\item Δύο απέναντι πλευρές είναι παράλληλες και ίσες.
\item Οι απέναντι γωνίες είναι ίσες.
\item Οι διαγώνιοι διχοτομούνται.
\end{rlist} & \begin{rlist}[leftmargin=5mm]
\item Είναι παραλληλόγραμμο και έχει μια ορθή γωνία.
\item Είναι παραλληλόγραμμο και οι διαγώνιοι είναι ίσες.
\item Έχει 3 ορθές γωνίες.
\item Έχει όλες τις γωνίες ίσες.
\end{rlist} & \begin{rlist}[leftmargin=5mm]
\item Όλες οι πλευρές του είναι ίσες.
\item Είναι παραλληλόγραμμο και έχει δύο διαδοχικές πλευρές ίσες.
\item Είναι παραλληλόγραμμο και έχει διαγώνιους κάθετες.
\item Είναι παραλληλόγραμμο και μια διαγώνιος διχοτομεί μια γωνία.
\end{rlist} & Παραλληλόγραμμο και
\begin{rlist}[leftmargin=5mm]
\item Έχει μια ορθή γωνία και δύο διαδοχικές πλευρές ίσες.
\item Έχει μια ορθή γωνία και διαγώνιους κάθετες.
\item Έχει μια ορθή γωνία και μια διαγώνιος διχοτομεί μια γωνία.
\item Έχει διαγώνιους ίσες και κάθετες.
\item Έχει διαγώνιους ίσες και δύο διαδοχικές πλευρές ίσες.
\item Έχει διαγώνιους ίσες και μια απ' αυτές διχοτομεί μια γωνία.
\end{rlist} \\
\hline
\end{tabular}
\end{sidewaysfigure}
\end{document}
