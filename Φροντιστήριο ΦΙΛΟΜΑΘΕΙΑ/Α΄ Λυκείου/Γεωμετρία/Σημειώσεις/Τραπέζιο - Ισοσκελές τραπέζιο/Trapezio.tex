\PassOptionsToPackage{no-math,cm-default}{fontspec}
\documentclass[twoside,nofonts,internet,shmeiwseis]{thewria}
\usepackage{amsmath}
\usepackage{xgreek}
\let\hbar\relax
\defaultfontfeatures{Mapping=tex-text,Scale=MatchLowercase}
\setmainfont[Mapping=tex-text,Numbers=Lining,Scale=1.0,BoldFont={Minion Pro Bold}]{Minion Pro}
\newfontfamily\scfont{GFS Artemisia}
\font\icon = "Webdings"
\usepackage[amsbb]{mtpro2}
\usepackage{tikz,pgfplots,tkz-euclide,enumitem}
\usetkzobj{all}
\tkzSetUpPoint[size=7,fill=white]
\xroma{red!70!black}

\newlist{rlist}{enumerate}{3}
\setlist[rlist]{itemsep=0mm,label=\roman*.}
\newlist{brlist}{enumerate}{3}
\setlist[brlist]{itemsep=0mm,label=\bf\roman*.}
\newlist{tropos}{enumerate}{3}
\setlist[tropos]{label=\bf\textit{\arabic*\textsuperscript{oς}\;Τρόπος :},leftmargin=0cm,itemindent=2.3cm,ref=\bf{\arabic*\textsuperscript{oς}\;Τρόπος}}
\newcommand{\tss}[1]{\textsuperscript{#1}}
\newcommand{\tssL}[1]{\MakeLowercase{\textsuperscript{#1}}}
\usepackage{rotating}
\usepackage{hhline}
\usepackage{multicol,multirow,gensymb,mathimatika}
\usepackage{wrap-rl}


\begin{document}
\titlos{Γεωμετρία Α΄ Λυκείου}{Παραλληλόγραμμα}{Τραπέζιο - Ισοσκελές Τραπέζιο}
\orismoi
\Orismos{Τραπέζιο - Ισοσκελές τραπέζιο}
Τραπέζιο ονομάζεται το τετράπλευρο το οποίο έχει δύο απέναντι πλευρές του παράλληλες.
\begin{itemize}[itemsep=0mm]
\item Οι παράλληλες πλευρές ενός τραπεζίου ονομάζονται \textbf{βάσεις} του. Οι βάσεις ενός τραπεζίου δεν είναι ίσες. Ονομάζονται \textbf{μικρή} και \textbf{μεγάλη} βάση αντίστοιχα.
\item Το ευθύγραμμο τμήμα που ενώνει τα μέσα των δύο μη παράλληλων πλευρών ενός τραπεζίου ονομάζεται \textbf{διάμεσος} του τραπεζίου. Συμβολίζεται με $ \delta $.
\end{itemize}
\begin{center}
\begin{tabular}{p{3.9cm}cp{4cm}}
\begin{tikzpicture}
\tkzDefPoint(0,-1.5){D}
\tkzDefPoint(0.5,.5){A}
\tkzDefPoint(2.5,.5){B}
\tkzDefPoint(3.5,-1.5){C}
\tkzDefPoint(.25,-.5){M}
\tkzDefPoint(3,-.5){N}
\tkzDefPoint(0.9,0.5){E}
\tkzDefPoint(0.9,-1.5){Z}
\tkzMarkRightAngle(C,Z,E)
\draw (0.9,0.5) -- (0.9,-1.5);
\draw[pl] (0,-1.5) -- (0.5,0.5) -- (2.5,0.5) -- (3.5,-1.5) -- cycle;
\draw[plm,\xrwma](M)--(N);
\tkzLabelPoint[above](A){$A$}
\tkzLabelPoint[above](B){$B$}
\tkzLabelPoint[below](C){$\varGamma$}
\tkzLabelPoint[below](D){$\varDelta$}
\tkzLabelPoint[left](M){$M$}
\tkzLabelPoint[right](N){$N$}
\tkzDrawPoints(A,B,C,D,M,N)
\node at (1.5,0.7) {\footnotesize$\beta$};
\node at (1.7,-1.8) {\footnotesize$B$};
\node at (.7,-.2) {\footnotesize$ \upsilon $};
\node at (1.75,-.35) {\footnotesize$ \delta $};
\end{tikzpicture} & \hspace{.5cm} & \begin{tikzpicture}
\tkzDefPoint(0,-1.5){D}
\tkzDefPoint(0.75,.5){A}
\tkzDefPoint(2.75,.5){B}
\tkzDefPoint(3.5,-1.5){C}
\tkzDefPoint(.25,-.5){M}
\tkzDefPoint(3,-.5){N}
\tkzDefPoint(0.9,0.5){E}
\tkzDefPoint(0.9,-1.5){Z}
\tkzDrawSegment[pl](A,B)
\tkzDrawSegment[pl](C,D)
\tkzDrawSegment[plm,\xrwma](A,D)
\tkzDrawSegment[plm,\xrwma](B,C)
\tkzMarkSegment[mark=|](A,D)
\tkzMarkSegment[mark=|](B,C)
\tkzLabelPoint[above](A){$A$}
\tkzLabelPoint[above](B){$B$}
\tkzLabelPoint[below](C){$\varGamma$}
\tkzLabelPoint[below](D){$\varDelta$}
\tkzDrawPoints(A,B,C,D)
\node at (1.7,0.7) {\footnotesize$\beta$};
\node at (1.7,-1.8) {\footnotesize$B$};
\end{tikzpicture} \\ 
\end{tabular} 
\end{center}
\begin{itemize}[itemsep=0mm]
\item Το ευθύγραμμο τμήμα που είναι κάθετο στις δύο βάσεις ενός τραπεζίου ονομάζεται \textbf{ύψος} του τραπεζίου.
\item Το τραπέζιο το οποίο έχει τις μη παράλληλες πλευρές του ίσες ονομάζεται \textbf{ισοσκελές τραπέζιο}. 
\end{itemize}

\thewrhmata
\Thewrhma{Πορίσματα για τη διάμεσο}
Έστω ένα τραπέζιο $ AB\varGamma\varDelta $ με $ AB\parallel\varGamma\varDelta $ και $ \varGamma\varDelta>AB $ ενώ $ M,N $ είναι τα μέσα των μη παράλληλων πλευρών. Επίσης $ E,Z $ ορίζουμε τα μέσα των διαγωνίων $ A\varGamma,B\varDelta $. Ισχύουν οι παρακάτω προτάσεις :\\
\wrapr{-11mm}{8}{4cm}{3mm}{\begin{tikzpicture}
\tkzDefPoint(0,-1.5){D}
\tkzDefPoint(0.5,.5){A}
\tkzDefPoint(2.5,.5){B}
\tkzDefPoint(3.5,-1.5){C}
\tkzDefPoint(.25,-.5){M}
\tkzDefPoint(3,-.5){N}
\draw[pl] (0,-1.5) -- (0.5,0.5) -- (2.5,0.5) -- (3.5,-1.5) -- cycle;
\draw[plm,\xrwma](M)--(N);
\draw[pl] (A)--(C);
\draw[pl] (B)--(D);
\tkzInterLL(M,N)(B,D) \tkzGetPoint{E}
\tkzInterLL(M,N)(A,C) \tkzGetPoint{Z}
\tkzLabelPoint[above](A){$A$}
\tkzLabelPoint[above](B){$B$}
\tkzLabelPoint[below](C){$\varGamma$}
\tkzLabelPoint[below](D){$\varDelta$}
\tkzLabelPoint[left](M){$M$}
\tkzLabelPoint[right](N){$N$}
\tkzLabelPoint[above left](E){$E$}
\tkzLabelPoint[above right](Z){$Z$}
\tkzDrawPoints(A,B,C,D,M,N,E,Z)
\node at (1.75,-.77) {\footnotesize$ \delta $};
\end{tikzpicture}}{
\begin{rlist}
\item Η διάμεσος $ MN $ του τραπεζίου είναι παράλληλη με τις βάσεις $ AB,\varGamma\varDelta $ και ίση με το ημιάθροισμά τους.
\[ \delta=MN=\frac{AB+\varGamma\varDelta}{2} \]
\item Το ευθύγραμμο τμήμα $ EZ $ που ενώνει τα μέσα των διαγωνίων $ A\varGamma,B\varDelta $ είναι παράλληλο με τις βάσεις και ίσο με την ημιδιαφορά τους.
\[ EZ=\frac{\varGamma\varDelta-AB}{2} \]
\end{rlist}}
\newpage
\noindent
\Thewrhma{Ιδιότητες ισοσκελούς τραπεζίου}
Σε κάθε ισοσκελές τραπέζιο $ AB\varGamma\varDelta $ με $ AB\parallel\varGamma\varDelta $ ισχύουν οι παρακάτω προτάσεις :
\begin{rlist}
\item Οι προσκείμενες σε κάθε βάση γωνίες είναι ίσες  : $ \hat{A}=\hat{B} $ ή $ \hat{\varGamma}=\hat{\varDelta} $.
\item Οι διαγώνιοί του είναι ίσες : $ A\varGamma=B\varDelta $.
\end{rlist}
\vspace{-7mm}
\begin{center}
\begin{tikzpicture}
\tkzDefPoint(0,-1.5){D}
\tkzDefPoint(0.75,.5){A}
\tkzDefPoint(2.75,.5){B}
\tkzDefPoint(3.5,-1.5){C}
\tkzDefPoint(.25,-.5){M}
\tkzDefPoint(3,-.5){N}
\tkzDefPoint(0.9,0.5){E}
\tkzDefPoint(0.9,-1.5){Z}
\tkzMarkAngle[size=.3,mark=|,fill=\xrwma](D,A,B)
\tkzMarkAngle[size=.3,mark=|,fill=\xrwma](A,B,C)
\tkzMarkAngle[size=.4,mark=||,fill=\xrwma](C,D,A)
\tkzMarkAngle[size=.4,mark=||,fill=\xrwma](B,C,D)
\tkzDrawSegment[pl](A,B)
\tkzDrawSegment[pl](C,D)
\tkzDrawSegment[plm](A,D)
\tkzDrawSegment[plm](B,C)
\draw[pl,\xrwma] (A)--(C);
\draw[pl,\xrwma] (B)--(D);
\tkzMarkSegments[mark=|](A,D B,C)
\tkzMarkSegments[mark=|](A,C B,D)
\tkzLabelPoint[above](A){$A$}
\tkzLabelPoint[above](B){$B$}
\tkzLabelPoint[below](C){$\varGamma$}
\tkzLabelPoint[below](D){$\varDelta$}
\tkzDrawPoints(A,B,C,D)
\node at (1.7,0.7) {\footnotesize$\beta$};
\node at (1.7,-1.8) {\footnotesize$B$};
\end{tikzpicture}
\end{center}
\Thewrhma{Κριτήριο για ισοσκελές τραπέζιο}
Ένα τραπέζιο $ AB\varGamma\varDelta $ με $ AB\parallel\varGamma\varDelta $ θα είναι ισοσκελές αν ισχύει μια από τις προτάσεις :
\begin{rlist}
\item Οι προσκείμενες γωνίες μιας βάσης είναι ίσες.
\item Οι διαγώνιοι είναι ίσες.
\end{rlist}
\Thewrhma{Πορίσματα στο ισοσκελές τραπέζιο}
\wrapr{-4mm}{8}{3.2cm}{-17mm}{\begin{tikzpicture}[scale=.8]
\tkzDefPoint(0,-1.5){D}
\tkzDefPoint(0.75,.5){A}
\tkzDefPoint(2.75,.5){B}
\tkzDefPoint(3.5,-1.5){C}
\tkzDefPoint(.25,-.5){M}
\tkzDefPoint(3,-.5){N}
\tkzDefPoint(1.75,.5){E}
\tkzDefPoint(1.75,-1.5){Z}
\tkzDefPoint(1.75,3.15){M}
\tkzMarkRightAngle[size=.3,mark=||,fill=\xrwma](C,Z,M)
\tkzMarkRightAngle[size=.3,mark=||,fill=\xrwma](B,E,M)
\tkzDrawSegment[pl](A,B)
\tkzDrawSegment[pl](C,D)
\tkzDrawSegment[pl](A,D)
\tkzDrawSegment[pl](M,D)
\tkzDrawSegment[pl](M,C)
\tkzDrawSegment[pl](B,C)
\draw[pl,\xrwma] (M)--(Z);
\tkzMarkSegments[mark=|,size=3pt](A,D B,C)
\tkzMarkSegments[mark=||,size=3pt](A,E E,B)
\tkzMarkSegments[mark=|||,size=3pt](D,Z Z,C)
\tkzLabelPoint[left](A){$A$}
\tkzLabelPoint[right](B){$B$}
\tkzLabelPoint[below](C){$\varGamma$}
\tkzLabelPoint[below](D){$\varDelta$}
\tkzLabelPoint[above left](E){$E$}
\tkzLabelPoint[above left](Z){$Z$}
\tkzLabelPoint[above](M){$M$}
\tkzDrawPoints(A,B,C,D,E,Z,M)
\end{tikzpicture}}{
Σε κάθε ισοσκελές τραπέζιο ισχύουν οι παρακάτω προτάσεις :
\begin{rlist}
\item Οι προεκτάσεις των μη παράλληλων πλευρών ορίζουν δύο ισοσκελή τρίγωνα με κοινή κορυφή, το σημείο τομής τους και βάσεις, τις βάσεις του τραπεζίου.
\item Η ευθεία που διέρχεται από τα μέσα των βάσεων είναι μεσοκάθετος και των δύο βάσεων.
\end{rlist}}
\end{document}
