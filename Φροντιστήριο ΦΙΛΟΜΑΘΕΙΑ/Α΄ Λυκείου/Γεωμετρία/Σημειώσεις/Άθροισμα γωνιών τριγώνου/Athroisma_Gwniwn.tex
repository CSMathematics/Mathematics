\PassOptionsToPackage{no-math,cm-default}{fontspec}
\documentclass[twoside,nofonts,internet,shmeiwseis]{thewria}
\usepackage{amsmath}
\usepackage{xgreek}
\let\hbar\relax
\defaultfontfeatures{Mapping=tex-text,Scale=MatchLowercase}
\setmainfont[Mapping=tex-text,Numbers=Lining,Scale=1.0,BoldFont={Minion Pro Bold}]{Minion Pro}
\newfontfamily\scfont{GFS Artemisia}
\font\icon = "Webdings"
\usepackage[amsbb]{mtpro2}
\usepackage{tikz,pgfplots,tkz-euclide,mathtools}
\tkzSetUpPoint[size=7,fill=white]
\xroma{red!70!black}
%------- ΣΥΣΤΗΜΑ -------------------
\usepackage{systeme,regexpatch}
\makeatletter
% change the definition of \sysdelim not to store `\left` and `\right`
\def\sysdelim#1#2{\def\SYS@delim@left{#1}\def\SYS@delim@right{#2}}
\sysdelim\{. % reinitialize

% patch the internal command to use
% \LEFTRIGHT<left delim><right delim>{<system>}
% instead of \left<left delim<system>\right<right delim>
\regexpatchcmd\SYS@systeme@iii
{\cB.\c{SYS@delim@left}(.*)\c{SYS@delim@right}\cE.}
{\c{SYS@MT@LEFTRIGHT}\cB\{\1\cE\}}
{}{}
\def\SYS@MT@LEFTRIGHT{%
\expandafter\expandafter\expandafter\LEFTRIGHT
\expandafter\SYS@delim@left\SYS@delim@right}
\makeatother
\newcommand{\synt}[2]{{\scriptsize \begin{matrix}
\times#1\\\\ \times#2
\end{matrix}}}
%----------------------------------------
%------ ΜΗΚΟΣ ΓΡΑΜΜΗΣ ΚΛΑΣΜΑΤΟΣ ---------
\DeclareRobustCommand{\frac}[3][0pt]{%
{\begingroup\hspace{#1}#2\hspace{#1}\endgroup\over\hspace{#1}#3\hspace{#1}}}
%----------------------------------------

\newlist{rlist}{enumerate}{3}
\setlist[rlist]{itemsep=0mm,label=\roman*.}
\newlist{brlist}{enumerate}{3}
\setlist[brlist]{itemsep=0mm,label=\bf\roman*.}
\newlist{tropos}{enumerate}{3}
\setlist[tropos]{label=\bf\textit{\arabic*\textsuperscript{oς}\;Τρόπος :},leftmargin=0cm,itemindent=2.3cm,ref=\bf{\arabic*\textsuperscript{oς}\;Τρόπος}}
\newcommand{\tss}[1]{\textsuperscript{#1}}
\newcommand{\tssL}[1]{\MakeLowercase{\textsuperscript{#1}}}
\usetkzobj{all}
\usepackage{hhline}
%----------- ΓΡΑΦΙΚΕΣ ΠΑΡΑΣΤΑΣΕΙΣ ---------
\pgfkeys{/pgfplots/aks_on/.style={axis lines=center,
xlabel style={at={(current axis.right of origin)},xshift=1.5ex, anchor=center},
ylabel style={at={(current axis.above origin)},yshift=1.5ex, anchor=center}}}
\pgfkeys{/pgfplots/grafikh parastash/.style={\xrwma,line width=.4mm,samples=200}}
\pgfkeys{/pgfplots/belh ar/.style={tick label style={font=\scriptsize},axis line style={-latex}}}
%-----------------------------------------
\usepackage{multicol}
\usepackage{wrap-rl}
\tkzSetUpPoint[size=7,fill=white]
\tikzstyle{pl}=[line width=0.3mm]
\tikzstyle{plm}=[line width=0.4mm]
\usepackage{gensymb}


\begin{document}

\titlos{Γεωμετρία Α΄ Λυκείου}{Παράλληλες Ευθείες}{Άθροισμα γωνιών τριγώνου /\MakeLowercase{\boldmath{$ \nu- $}}γωνου - Γωνίες με πλευρές κάθετες}
\thewrhmata
\Thewrhma{Άθροισμα γωνιών τριγώνου}
Σε κάθε τρίγωνο το άθροισμα των γωνιών ισούται με $ 180\degree $.\\\\
\Thewrhma{Πορίσματα για τις γωνίες τριγώνου}
Για τις γωνίες ενός τριγώνου ισχύουν οι ακόλουθες προτάσεις :
\begin{rlist}
\item Κάθε εξωτερική γωνία σε ένα τρίγωνο είναι ίση με το άθροισμα των δύο απέναντι εσωτερικών.
\item Αν σε δύο τρίγωνα δύο γωνίες είναι μεταξύ τους ίσες μια προς μια, τότε θα είναι και οι τρίτες γωνίες ίσες.
\item Σε κάθε ορθογώνιο τρίγωνο οι οξείες γωνίες είναι συμπληρωματικές.
\item Οι γωνίες ενός ισόπλευρου τριγώνου ισούνται με $ 60\degree $.
\end{rlist}
\Thewrhma{Γωνίες με κάθετες πλευρές}
\wrapr{-4mm}{7}{5.5cm}{-8mm}{\begin{tikzpicture}
\tkzDefPoint(-1,0){A}
\tkzDefPoint(4,0){B}
\tkzDefPoint(-1,-.5){C}
\tkzDefPoint(2.2,1.2){D}
\tkzDefPoint(3.5,0){E}
\tkzDefPoint(3.5,-2.5){F}
\tkzDefPoint(3.5,-2){G}
\tkzDefPointBy[projection=onto C--D](G) \tkzGetPoint{H}
\tkzInterLL(A,B)(C,D) \tkzGetPoint{K}
\tkzInterLL(A,B)(G,H) \tkzGetPoint{M}
\tkzLabelPoint[above](K){$O$}
\tkzLabelPoint[right](D){$ x $}
\tkzLabelPoint[right](B){$ y $}
\tkzLabelPoint[above right](M){$B$}
\tkzLabelPoint[above](E){$\varGamma$}
\tkzLabelPoint[right](G){$O'$}
\tkzLabelPoint[above left](H){$A$}
\tkzMarkAngle[fill=\xrwma!70,size=4mm](B,K,D)
\tkzMarkAngle[fill=\xrwma!70,size=5mm](E,G,M)
\tkzMarkAngle[fill=\xrwma!90,size=3mm](M,G,F)
\tkzMarkAngle[fill=\xrwma!50,size=3mm](H,M,A)
\tkzMarkAngle[fill=\xrwma!50,size=3mm](G,M,B)
\tkzMarkAngle[fill=\xrwma!90,size=3mm](C,K,B)
\tkzMarkRightAngle[fill=\xrwma](K,H,M)
\tkzMarkRightAngle[fill=\xrwma](A,E,G)
\tkzDrawSegments(A,B C,D E,F G,H)
\tkzLabelAngle[pos=.6](B,K,D){$ \omega $}
\tkzLabelAngle[pos=.7](E,G,M){$ \varphi $}
\tkzLabelAngle[pos=.5](H,M,A){\footnotesize$1$}
\tkzLabelAngle[pos=.5](G,M,B){\footnotesize$2$}
\tkzLabelAngle[pos=-.5](M,G,F){$ \theta' $}
\tkzLabelAngle[pos=.5](C,K,B){$ \theta $}
\tkzText(2.7,-1){$ y' $}
\tkzText(3.7,-1){$ x' $}
\end{tikzpicture}}{
Εαν δύο γωνίες $ x\hat{O}y,\ x'\hat{O'}y' $ έχουν τις πλευρές τους κάθετες τότε
\begin{rlist}
\item αν είναι και οι δύο οξείες ή και οι δύο αμβλείες είναι ίσες.
\item αν είναι μια οξεία και μια αμβλεία τότε είναι παραπληρωματικές.
\end{rlist}}\mbox{}\\\\\\
\Thewrhma{Άθροισμα γωνιών {\MakeLowercase{$ \mathbold{\nu}- $}} γωνου}
Σε κάθε κυρτό $ \nu- $γωνο $ A_1A_2\ldots A_\nu $ ισχύει ότι :
\begin{rlist}
\item Το άθροισμα των εσωτερικών γωνιών ενός κυρτού $ \nu- $γωνου ισούται με $ (2\nu-4)\cdot 90\degree $.
\item Το άθροισμα των εξωτερικών γωνιών ενός κυρτού $ \nu- $γωνου ισούται με $ 360\degree $.
\end{rlist}
\begin{center}
\begin{tabular}{ccc}
\begin{tikzpicture}[scale=1.5]
\tkzDefPoint(0,0.2){B}
\tkzDefPoint(0,-.5){C}
\tkzDefPoint(.7,.7){A}
\tkzDefPoint(1,-1){D}
\tkzDefPoint(2,-.7){E}
\tkzDefPoint(1.7,.7){H}
\tkzDefPoint(.2,.7){K}
\tkzDefPoint(-.35,-.05){L}
\tkzDefPoint(0,-.9){M}
\tkzDefPoint(1,.7){N}
\tkzMarkAngle[fill=\xrwma!70,size=.2](B,A,H)
\tkzMarkAngle[fill=\xrwma!70,size=.25](C,B,A)
\tkzMarkAngle[fill=\xrwma!70,size=.25](D,C,B)
\tkzMarkAngle[fill=\xrwma!70,size=.25](E,D,C)
\tkzDrawSegments(A,B B,C C,D H,A)
\tkzDrawSegment[dashed](D,E)
\tkzLabelPoint[above](A){$ A_{1} $}
\tkzLabelPoint[above left](B){$ A_{2} $}
\tkzLabelPoint[left](C){$ A_{3} $}
\tkzLabelPoint[below](D){$ A_{4} $}
\tkzLabelPoint[below](E){$ A_{5} $}
\tkzLabelPoint[above](H){$ A_{\nu} $}
\node at (1,-1.5) {$\hat{A}_{1}+\hat{A}_{2}+\ldots+ \hat{A}_{\nu}=(2\nu-4)\cdot 90\degree$};
\end{tikzpicture} & & \begin{tikzpicture}[scale=1.5]
\tkzDefPoint(0,0.2){B}
\tkzDefPoint(0,-.5){C}
\tkzDefPoint(.7,.7){A}
\tkzDefPoint(1,-1){D}
\tkzDefPoint(2,-.7){E}
\tkzDefPoint(1.7,.7){H}
\tkzDefPoint(.2,.7){K}
\tkzDefPoint(-.35,-.05){L}
\tkzDefPoint(0,-.9){M}
\tkzDefPoint(1,.7){N}
\tkzMarkAngle[fill=\xrwma!70,%
size=.25](K,A,B)
\tkzMarkAngle[fill=\xrwma!70,%
size=.25](M,C,D)
\tkzMarkAngle[fill=\xrwma!70,%
size=.25](L,B,C)
\tkzDrawSegments(H,K A,L B,M C,D H,A)
\tkzDrawSegment[dashed](D,E)
\tkzLabelPoint[above](A){$ A_{1} $}
\tkzLabelPoint[above left](B){$ A_{2} $}
\tkzLabelPoint[left](C){$ A_{3} $}
\tkzLabelPoint[below](D){$ A_{4} $}
\tkzLabelPoint[below](E){$ A_{5} $}
\tkzLabelPoint[above](H){$ A_{\nu} $}
\node at (1,-1.5) {$\hat{A}_{1\varepsilon\xi}+\hat{A}_{2\varepsilon\xi}+\ldots+ \hat{A}_{\nu\varepsilon\xi}=360\degree$};
\end{tikzpicture} \\ 
\end{tabular} 
\end{center}
\end{document}
