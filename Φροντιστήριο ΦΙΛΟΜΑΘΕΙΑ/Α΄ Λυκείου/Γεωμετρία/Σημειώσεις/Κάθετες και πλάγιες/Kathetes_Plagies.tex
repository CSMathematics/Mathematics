\PassOptionsToPackage{no-math,cm-default}{fontspec}
\documentclass[twoside,nofonts,internet,shmeiwseis]{thewria}
\usepackage{amsmath}
\usepackage{xgreek}
\let\hbar\relax
\defaultfontfeatures{Mapping=tex-text,Scale=MatchLowercase}
\setmainfont[Mapping=tex-text,Numbers=Lining,Scale=1.0,BoldFont={Minion Pro Bold}]{Minion Pro}
\newfontfamily\scfont{GFS Artemisia}
\font\icon = "Webdings"
\usepackage[amsbb]{mtpro2}
\usepackage{tikz,pgfplots,tkz-euclide}
\tkzSetUpPoint[size=7,fill=white]
\xroma{red!70!black}
%------- ΣΥΣΤΗΜΑ -------------------
\usepackage{systeme,regexpatch}
\makeatletter
% change the definition of \sysdelim not to store `\left` and `\right`
\def\sysdelim#1#2{\def\SYS@delim@left{#1}\def\SYS@delim@right{#2}}
\sysdelim\{. % reinitialize

% patch the internal command to use
% \LEFTRIGHT<left delim><right delim>{<system>}
% instead of \left<left delim<system>\right<right delim>
\regexpatchcmd\SYS@systeme@iii
{\cB.\c{SYS@delim@left}(.*)\c{SYS@delim@right}\cE.}
{\c{SYS@MT@LEFTRIGHT}\cB\{\1\cE\}}
{}{}
\def\SYS@MT@LEFTRIGHT{%
\expandafter\expandafter\expandafter\LEFTRIGHT
\expandafter\SYS@delim@left\SYS@delim@right}
\makeatother
\newcommand{\synt}[2]{{\scriptsize \begin{matrix}
\times#1\\\\ \times#2
\end{matrix}}}
%----------------------------------------
%------ ΜΗΚΟΣ ΓΡΑΜΜΗΣ ΚΛΑΣΜΑΤΟΣ ---------
\DeclareRobustCommand{\frac}[3][0pt]{%
{\begingroup\hspace{#1}#2\hspace{#1}\endgroup\over\hspace{#1}#3\hspace{#1}}}
%----------------------------------------

\newlist{rlist}{enumerate}{3}
\setlist[rlist]{itemsep=0mm,label=\roman*.}
\newlist{brlist}{enumerate}{3}
\setlist[brlist]{itemsep=0mm,label=\bf\roman*.}
\newlist{tropos}{enumerate}{3}
\setlist[tropos]{label=\bf\textit{\arabic*\textsuperscript{oς}\;Τρόπος :},leftmargin=0cm,itemindent=2.3cm,ref=\bf{\arabic*\textsuperscript{oς}\;Τρόπος}}
\newcommand{\tss}[1]{\textsuperscript{#1}}
\newcommand{\tssL}[1]{\MakeLowercase{\textsuperscript{#1}}}
\usetkzobj{all}
\usepackage{hhline}
%----------- ΓΡΑΦΙΚΕΣ ΠΑΡΑΣΤΑΣΕΙΣ ---------
\pgfkeys{/pgfplots/aks_on/.style={axis lines=center,
xlabel style={at={(current axis.right of origin)},xshift=1.5ex, anchor=center},
ylabel style={at={(current axis.above origin)},yshift=1.5ex, anchor=center}}}
\pgfkeys{/pgfplots/grafikh parastash/.style={\xrwma,line width=.4mm,samples=200}}
\pgfkeys{/pgfplots/belh ar/.style={tick label style={font=\scriptsize},axis line style={-latex}}}
%-----------------------------------------
\usepackage{multicol}
\usepackage{wrap-rl}
\tkzSetUpPoint[size=7,fill=white]
\tikzstyle{pl}=[line width=0.3mm]
\tikzstyle{plm}=[line width=0.4mm]
\usepackage{gensymb}


\begin{document}
\titlos{Γεωμετρία Α΄ Λυκείου}{Τρίγωνα}{Κάθετες και πλάγιες}
\orismoi
\Orismos{Ίχνος πλάγιας - καθέτου}
\wrapr{-4mm}{8}{4cm}{-7mm}{\begin{tikzpicture}
\draw (-1,0) -- (3.2,0);
\tkzDefPoint[label=above:$A$](-.4,1){A}
\tkzDefPoint[label=above left:$K$](-.4,0){K}
\tkzDefPoint[label=above right:$B$](1.77,0){B}
\tkzMarkRightAngle[fill=\xrwma](B,K,A)
\draw (A)--(-.4,-.2);
\draw (A)--(2.2,-0.2);
\node at (3.4,0) {\footnotesize$\zeta$};
\node at (-0.2,-0.2) {\footnotesize$\kappa$};
\node at (2.4,-0.2) {\footnotesize$\varepsilon$};
\tkzDrawPoints(A,K,B)
\end{tikzpicture}}{
Ίχνος μιας πλάγιας ή κάθετης ευθείας $ \varepsilon $ πάνω σε μια ευθεία $ \zeta $ ονομάζεται το σημείο στο οποίο η ευθεία $ \varepsilon $ τέμνει τη $ \zeta $. Ομοίως, ίχνος ενός ευθυγράμμου τμήματος πάνω σε μια ευθεία ονομάζεται το σημείο τομής τους.}\mbox{}\\\\\\
\thewrhmata
\Thewrhma{Ίσα πλάγια τμήματα}
Αν δύο πλάγια προς μια ευθεία τμήματα είναι ίσα τότε τα ίχνη τους ισαπέχουν από το ίχνος της καθέτου.\\\\
\Thewrhma{Πορίσματα για τα πλάγια και κάθετα τμήματα}
Αν φέρουμε από ένα σημείο εκτός ευθείας το κάθετο και δύο πλάγια ευθύγραμμα τμήματα τότε
\begin{rlist}
\item Το κάθετο τμήμα έχει το μικρότερο μήκος από οποιοδήποτε άλλο πλάγιο.
\item Δύο πλάγια τμήματα είναι άνισα αν και μόνο αν οι αποστάσεις των ιχνών τους από το ίχνος της καθέτου είναι όμοια άνισες.
\end{rlist}
\begin{center}
\begin{tabular}{cc}
\begin{tikzpicture}
\draw (0,0) -- (3,0);
\tkzDefPoint[label=above:$A$](1.5,1.4){A}
\tkzDefPoint[label=above left:$K$](1.5,0){K}
\tkzDefPoint[label=above right:$B$](2.4,0){B}
\tkzDefPoint[label=above left:$\varGamma$](.6,0){C}
\tkzMarkRightAngle[fill=\xrwma](B,K,A)
\draw[pl] (A)--(K);
\draw[pl,\xrwma] (A)--(B);
\draw[pl,\xrwma] (A)--(C);
\tkzMarkSegments[mark=|,size=2pt](A,B A,C)
\tkzMarkSegments[mark=||,size=2pt](K,B K,C)
\tkzDrawPoints(A,K,B,C)
\node at (1.5,-0.4) {$AB=A\varGamma\Leftrightarrow KB=K\varGamma$};
\end{tikzpicture} & \begin{tikzpicture}
\draw (0,0) -- (3.7,0);
\tkzDefPoint[label=above:$A$](1.5,1.4){A}
\tkzDefPoint[label=above left:$K$](1.5,0){K}
\tkzDefPoint[label=above right:$B$](3,0){B}
\tkzDefPoint[label=above left:$\varGamma$](.6,0){C}
\tkzMarkRightAngle[fill=\xrwma](B,K,A)
\draw[pl] (A)--(K);
\draw[pl,\xrwma] (A)--(B);
\draw[pl,\xrwma] (A)--(C);
\tkzDrawPoints(A,K,B,C)
\node at (1.8,-0.4) {$AB>A\varGamma\Leftrightarrow KB>K\varGamma$};
\end{tikzpicture} \\ 
\end{tabular} 
\end{center}
\end{document}
