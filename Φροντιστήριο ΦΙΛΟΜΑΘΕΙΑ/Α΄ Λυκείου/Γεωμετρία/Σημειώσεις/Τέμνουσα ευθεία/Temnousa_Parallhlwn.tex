\PassOptionsToPackage{no-math,cm-default}{fontspec}
\documentclass[twoside,nofonts,internet,shmeiwseis]{thewria}
\usepackage{amsmath}
\usepackage{xgreek}
\let\hbar\relax
\defaultfontfeatures{Mapping=tex-text,Scale=MatchLowercase}
\setmainfont[Mapping=tex-text,Numbers=Lining,Scale=1.0,BoldFont={Minion Pro Bold}]{Minion Pro}
\newfontfamily\scfont{GFS Artemisia}
\font\icon = "Webdings"
\usepackage[amsbb]{mtpro2}
\usepackage{tikz,pgfplots,tkz-euclide,mathtools}
\tkzSetUpPoint[size=7,fill=white]
\xroma{red!70!black}
%------- ΣΥΣΤΗΜΑ -------------------
\usepackage{systeme,regexpatch}
\makeatletter
% change the definition of \sysdelim not to store `\left` and `\right`
\def\sysdelim#1#2{\def\SYS@delim@left{#1}\def\SYS@delim@right{#2}}
\sysdelim\{. % reinitialize

% patch the internal command to use
% \LEFTRIGHT<left delim><right delim>{<system>}
% instead of \left<left delim<system>\right<right delim>
\regexpatchcmd\SYS@systeme@iii
{\cB.\c{SYS@delim@left}(.*)\c{SYS@delim@right}\cE.}
{\c{SYS@MT@LEFTRIGHT}\cB\{\1\cE\}}
{}{}
\def\SYS@MT@LEFTRIGHT{%
\expandafter\expandafter\expandafter\LEFTRIGHT
\expandafter\SYS@delim@left\SYS@delim@right}
\makeatother
\newcommand{\synt}[2]{{\scriptsize \begin{matrix}
\times#1\\\\ \times#2
\end{matrix}}}
%----------------------------------------
%------ ΜΗΚΟΣ ΓΡΑΜΜΗΣ ΚΛΑΣΜΑΤΟΣ ---------
\DeclareRobustCommand{\frac}[3][0pt]{%
{\begingroup\hspace{#1}#2\hspace{#1}\endgroup\over\hspace{#1}#3\hspace{#1}}}
%----------------------------------------

\newlist{rlist}{enumerate}{3}
\setlist[rlist]{itemsep=0mm,label=\roman*.}
\newlist{brlist}{enumerate}{3}
\setlist[brlist]{itemsep=0mm,label=\bf\roman*.}
\newlist{tropos}{enumerate}{3}
\setlist[tropos]{label=\bf\textit{\arabic*\textsuperscript{oς}\;Τρόπος :},leftmargin=0cm,itemindent=2.3cm,ref=\bf{\arabic*\textsuperscript{oς}\;Τρόπος}}
\newcommand{\tss}[1]{\textsuperscript{#1}}
\newcommand{\tssL}[1]{\MakeLowercase{\textsuperscript{#1}}}
\usetkzobj{all}
\usepackage{hhline}
%----------- ΓΡΑΦΙΚΕΣ ΠΑΡΑΣΤΑΣΕΙΣ ---------
\pgfkeys{/pgfplots/aks_on/.style={axis lines=center,
xlabel style={at={(current axis.right of origin)},xshift=1.5ex, anchor=center},
ylabel style={at={(current axis.above origin)},yshift=1.5ex, anchor=center}}}
\pgfkeys{/pgfplots/grafikh parastash/.style={\xrwma,line width=.4mm,samples=200}}
\pgfkeys{/pgfplots/belh ar/.style={tick label style={font=\scriptsize},axis line style={-latex}}}
%-----------------------------------------
\usepackage{multicol}
\usepackage{wrap-rl}
\tkzSetUpPoint[size=7,fill=white]
\tikzstyle{pl}=[line width=0.3mm]
\tikzstyle{plm}=[line width=0.4mm]
\usepackage{gensymb}


\begin{document}
\titlos{Γεωμετρία Α΄ Λυκείου}{Παράλληλες Ευθείες}{Τέμνουσα παράλληλων}
\orismoi
\Orismos{Παράλληλες ευθείες}
\wrapr{-4mm}{5}{4.1cm}{-7mm}{\begin{tikzpicture}
\draw (-1,0) -- (2.5,0);
\draw (-1,-0.5) -- (2.5,-0.5);
\draw (-1,-1.5) -- (2.5,-1.5);
\node at (3,0) {\footnotesize$\varepsilon_1$};
\node at (3,-0.5) {\footnotesize$\varepsilon_2$};
\node at (3,-1.5) {\footnotesize$\varepsilon_\nu$};
\node at (-0.5,-.9) {$\vdots$};
\node at (0.5,-.9) {$\vdots$};
\node at (1.5,-.9) {$\vdots$};
\end{tikzpicture}}{
Παράλληλες ονομάζονται δύο ή περισσότερες ευθείες του ίδιου επιπέδου οι οποίες δεν έχουν κανένα κοινό σημείο. Ανάμεσα σε δύο παράλληλες ευθείες χρησιμοποιούμε το συμβολισμό $ \parallel $.
\[ \varepsilon_1\parallel\varepsilon_2\parallel\ldots\parallel\varepsilon_\nu \]}\mbox{}\\\\\\
\Orismos{Χαρακτηρισμοί γωνιών σε παράλληλες ευθείες}
Δίνονται δύο παράλληλες ευθείες $ \varepsilon_1,\varepsilon_2 $ και μια τέμνουσα $ \varepsilon $ των δύο ευθειών. Η τέμνουσα ευθεία τέμνει τις παράλληλες $ \varepsilon_1,\varepsilon_2 $ στα σημεία $ A,B $ αντίστοιχα, οπότε σχηματίζονται 8 γωνίες με κορυφές τα σημεία $ A $ και $ B $. Οι χαρακτηρισμοί που δίνονται σ' αυτές τις γωνίες είναι οι ακόλουθοι :
\begin{itemize}[itemsep=0mm]
\item Οι γωνίες που βρίσκονται μεταξύ των παράλληλων ευθειών ονομάζονται \textbf{εντός}.
\item Οι γωνίες που βρίσκονται στην περιοχή έξω από τις παράλληλες ευθείες ονομάζονται \textbf{εκτός}.
\item Οι γωνίες που βρίσκονται στο ίδιο ημιεπίπεδο που ορίζει η τέμνουσα ονομάζονται \textbf{επί τα αυτά}.
\item Οι γωνίες που βρίσκονται εκατέρωθεν της τέμνουσας ονομάζονται \textbf{εναλλάξ}.
\end{itemize}
\begin{center}
\begin{tabular}{cccc}
\begin{tikzpicture}[x=.77cm]
\fill[\xrwma!20] (-.9,-1) rectangle (2.4,0);
\draw (-1,0) -- (2.5,0);
\draw (-1,-1) -- (2.5,-1);
\node at (3,0) {\footnotesize$\varepsilon_1$};
\node at (3,-1) {\footnotesize$\varepsilon_2$};
\draw (0,0.5) -- (1.5,-1.5);
\tkzDefPoint(.37,0){A}
\tkzDefPoint(1.12,-1){B}
\tkzLabelPoint[above,xshift=1mm](A){$A$}
\tkzLabelPoint[below,xshift=-1mm](B){$B$}
\tkzDrawPoints(A,B)
\node at (0,-0.5) {εντός};
\end{tikzpicture} & \begin{tikzpicture}[x=.77cm]
\fill[\xrwma!20] (-.9,-1) rectangle (2.4,-1.5);
\fill[\xrwma!20] (-.9,0) rectangle (2.4,.5);
\draw (-1,0) -- (2.5,0);
\draw (-1,-1) -- (2.5,-1);
\node at (3,0) {\footnotesize$\varepsilon_1$};
\node at (3,-1) {\footnotesize$\varepsilon_2$};
\draw (0,0.5) -- (1.5,-1.5);
\tkzDefPoint(.37,0){A}
\tkzDefPoint(1.12,-1){B}
\tkzLabelPoint[above,xshift=1mm](A){$A$}
\tkzLabelPoint[below,xshift=-1mm](B){$B$}
\tkzDrawPoints(A,B)
\node at (1.5,0.25) {εκτός};
\node at (0,-1.25) {εκτός};
\end{tikzpicture} & \begin{tikzpicture}[x=.77cm]
\fill[\xrwma!20] (1.5,-1.5) -- (3,-1.5)--(1.5,0.5)--(0,0.5);
\draw (-1,0) -- (2.5,0);
\draw (-1,-1) -- (2.5,-1);
\node at (3,0) {\footnotesize$\varepsilon_1$};
\node at (3,-1) {\footnotesize$\varepsilon_2$};
\draw (0,0.5) -- (1.5,-1.5);
\tkzDefPoint(.37,0){A}
\tkzDefPoint(1.12,-1){B}
\tkzLabelPoint[above,xshift=1mm](A){$A$}
\tkzLabelPoint[below,xshift=-1mm](B){$B$}
\tkzDrawPoints(A,B)
\node[rotate=-60,fill=\xrwma!20,inner sep=.4mm] at (1.5,-0.5) {επι τα αυτά};
\end{tikzpicture} & \begin{tikzpicture}[x=.77cm]
\fill[\xrwma!20] (1.5,-1.5) -- (3,-1.5)--(1.5,0.5)--(0,0.5);
\fill[\xrwma!50] (0,-1.5) -- (1.5,-1.5)--(0,0.5)--(-1.5,0.5);
\draw (-1,0) -- (2.5,0);
\draw (-1,-1) -- (2.5,-1);
\node at (3,0) {\footnotesize$\varepsilon_1$};
\node at (3,-1) {\footnotesize$\varepsilon_2$};
\draw (0,0.5) -- (1.5,-1.5);
\tkzDefPoint(.37,0){A}
\tkzDefPoint(1.12,-1){B}
\tkzLabelPoint[above,xshift=1mm](A){$A$}
\tkzLabelPoint[below,xshift=-1mm](B){$B$}
\tkzDrawPoints(A,B)
\node at (.8,-0.5) {εναλλάξ};
\end{tikzpicture} \\ 
\end{tabular} 
\end{center}
Επιλέγοντας δύο γωνίες, μια με κορυφή το σημείο $ A $ και μια με κορυφή το $ B $ συνδυάζουμε τους παραπάνω χαρακτηρισμούς οπότε προκύπτουν οι εξής ονομασίες :\\
\wrapr{-11mm}{7}{5cm}{0mm}{\begin{tikzpicture}[scale=1.2]
\tkzDefPoint(.37,0){A}
\tkzDefPoint(1.12,-1){B}
\fill[\xrwma] (A) circle (.2);
\fill[\xrwma] (B) circle (.2);
\draw (-1,0) -- (2.5,0);
\draw (-1,-1) -- (2.5,-1);
\node at (3,0) {\footnotesize$\varepsilon_1$};
\node at (3,-1) {\footnotesize$\varepsilon_2$};
\draw (0,0.5) -- (1.5,-1.5);
\tkzLabelPoint[above,xshift=1mm,yshift=1.2mm](A){$A$}
\tkzLabelPoint[below,xshift=-1mm,yshift=-1.2mm](B){$B$}
\tkzDrawPoints(A,B)
\node at (0.8,-0.2) {\scriptsize$2$};
\node at (1.45,-1.2) {\scriptsize$2$};
\node at (0.8,-1.2) {\scriptsize$3$};
\node at (1.4,-0.8) {\scriptsize$1$};
\node at (0.8,-0.8) {\scriptsize$4$};
\node at (0.8,0.2) {\scriptsize$1$};
\node at (0,0.2) {\scriptsize$4$};
\node at (0,-0.2) {\scriptsize$3$};
\end{tikzpicture}}{
\begin{rlist}
\item εντός εναλλάξ : $ \hat{A}_2,\hat{B}_4 $ και $ \hat{A}_3,\hat{B}_1 $
\item εκτός εναλλάξ : $ \hat{A}_1,\hat{B}_3 $ και $ \hat{A}_4,\hat{B}_2 $
\item εντός εκτός εναλλάξ : $ \hat{A}_1,\hat{B}_4 $,\quad $ \hat{A}_4,\hat{B}_1 $,\quad $ \hat{A}_3,\hat{B}_2 $ και\quad $ \hat{A}_2,\hat{B}_3 $
\item εντός και επί τα αυτά : $ \hat{A}_3,\hat{B}_4 $ και $ \hat{A}_2,\hat{B}_1 $
\item εκτός και επί τα αυτά : $ \hat{A}_4,\hat{B}_3 $ και $ \hat{A}_1,\hat{B}_2 $
\item εντός εκτός και επί τα αυτά : $ \hat{A}_1,\hat{B}_1 $,\quad $ \hat{A}_2,\hat{B}_2 $,\quad $ \hat{A}_3,\hat{B}_3 $ και\quad $ \hat{A}_4,\hat{B}_4 $
\end{rlist}}
\newpage
\noindent
\thewrhmata
\Thewrhma{Συνθήκη παραλληλίας}
Έστω δύο ευθείες $ \varepsilon_1 $ και $ \varepsilon_2 $ και μια τέμνουσα $ \varepsilon $ τέμνει αυτές στα σημεία $ A,B $ αντίστοιχα. Αν ισχύει μια από τις προτάσεις :
\begin{rlist}
\item οι εντός εναλλάξ γωνίες είναι ίσες.
\item οι εντός εκτός και επί τα αυτά γωνίες είναι ίσες.
\item οι εντός και επί τα αυτά γωνίες είναι παραπληρωματικές.
\end{rlist}
τότε οι ευθείες $ \varepsilon_1,\varepsilon_2 $ είναι παράλληλες : $ \varepsilon_1\parallel\varepsilon_2 $.\\\\
\Thewrhma{Σχέσεις γωνιών από παράλληλες ευθείες}
Έστω δύο ευθείες $ \varepsilon_1 $ και $ \varepsilon_2 $ και μια τέμνουσα $ \varepsilon $ τέμνει αυτές στα σημεία $ A,B $ αντίστοιχα. Αν οι ευθείες $ \varepsilon_1,\varepsilon_2 $ είναι παράλληλες τότε :
\begin{rlist}
\item Οι εντός εναλλάξ γωνίες είναι ίσες.
\item Οι εντός εκτός και επί τα αυτά γωνίες είναι ίσες.
\item Οι εντός και επί τα αυτά γωνίες είναι παραπληρωματικές.
\end{rlist}
\Thewrhma{Αίτημα παραλληλίας}
Από ένα σημείο εκτός ευθείας διέρχεται μόνο μια ευθεία παράλληλη προς αυτήν.\\\\
\Thewrhma{Κάθετες στην ίδια ευθεία}
\wrapr{-4mm}{5}{4.1cm}{-7mm}{\begin{tikzpicture}
\tkzDefPoint(0,0){A}
\tkzDefPoint(0,-1){B}
\tkzDefPoint(1,0){C}
\tkzDefPoint(1,-1){D}
\tkzMarkRightAngle[size=.25,fill=\xrwma](B,A,C)
\tkzMarkRightAngle[size=.25,fill=\xrwma](D,B,A)
\draw (-1,0) -- (2.5,0);
\draw (-1,-1) -- (2.5,-1);
\node at (3,0) {\footnotesize$\varepsilon_1$};
\node at (3,-1) {\footnotesize$\varepsilon_2$};
\draw (0,0.25) -- (0,-1.25);
\tkzLabelPoint[above right](A){$A$}
\tkzLabelPoint[below right](B){$B$}
\tkzDrawPoints(A,B)
\node at (-0.2,0.2) {\footnotesize$\varepsilon$};
\end{tikzpicture}}{
Αν δύο ευθείες $ \varepsilon_1,\varepsilon_2 $ είναι κάθετες σε μια τρίτη ευθεία $ \varepsilon $ σε διαφορετικά σημεία της, τότε είναι μεταξύ τους παράλληλες.
\[ \varepsilon_1\bot\varepsilon\ \textrm{ και }\ \varepsilon_2\bot\varepsilon\Rightarrow \varepsilon_1\parallel\varepsilon_2 \]}\mbox{}\\\\\\
\Thewrhma{Ευθείες ανά δύο παράλληλες}
\wrapr{-4mm}{7}{4.1cm}{-7mm}{\begin{tikzpicture}
\draw (-1,0) -- (2.5,0);
\draw (-1,-1) -- (2.5,-1);
\node at (3,0) {\footnotesize$\varepsilon_1$};
\node at (3,-1) {\footnotesize$\varepsilon_2$};
\draw[\xrwma] (-1,-0.6) -- (2.5,-0.6);
\node at (3,-0.6) {\footnotesize$\varepsilon$};
\end{tikzpicture}}{
Αν δύο ευθείες $ \varepsilon_1,\varepsilon_2 $ είναι παράλληλες προς μια τρίτη ευθεία $ \varepsilon $ τότε θα είναι και μεταξύ τους παράλληλες.
\[ \varepsilon_1\parallel\varepsilon\ \textrm{ και }\ \varepsilon_2\parallel\varepsilon\Rightarrow \varepsilon_1\parallel\varepsilon_2 \]}\mbox{}\\\\\\
\Thewrhma{Τέμνουσα ευθεία}
Αν μια ευθεία $ \varepsilon $ είναι τέμνουσα μιας από τις δύο παράλληλες ευθείες $ \varepsilon_1,\varepsilon_2 $ τότε θα είναι τέμνουσα και της άλλης. Προκύπτει παρόμοια ότι άν μια ευθεία είναι κάθετη σε μια από τις δύο παράλληλες τότε θα είναι κάθετη και με την άλλη.\\\\
\Thewrhma{Γωνίες με πλευρές παράλληλες}
Εαν δύο γωνίες $ x\hat{O}y,\ x'\hat{O'}y' $ έχουν τις πλευρές τους παράλληλες τότε
\begin{rlist}
\item αν είναι και οι δύο οξείες ή και οι δύο αμβλείες είναι ίσες.
\item αν είναι μια οξεία και μια αμβλεία τότε είναι παραπληρωματικές.
\end{rlist}
\begin{center}
\begin{tabular}{ccc}
\begin{tikzpicture}
\tkzDefPoint(1.5,0){B}
\tkzDefPoint(0,0){A}
\tkzDefPoint(1,1.5){C}
\tkzDefPoint(-1.5.2,1){D}
\tkzDefPoint(-.2,1){E}
\tkzDefPoint(-1.2,-.5){F}
\tkzMarkAngle[fill=\xrwma,size=.3,mark=|](B,A,C)
\tkzMarkAngle[fill=\xrwma,size=.3,mark=|](D,E,F)
\tkzDrawSegments(A,B A,C)
\tkzDrawSegments(D,E E,F)
\tkzLabelPoint[above](D){$ x' $}
\tkzLabelPoint[left](F){$ y' $}
\tkzLabelPoint[above](E){$ O' $}
\tkzLabelPoint[below](B){$ x $}
\tkzLabelPoint[left](C){$ y $}
\tkzLabelPoint[below](A){$ O $}
\node at (0,-1){$x\hat{O}y=x'\hat{O'}y'$};
\end{tikzpicture} 
&\begin{tikzpicture}
\tkzDefPoint(2,0){B}
\tkzDefPoint(0.2,0){A}
\tkzDefPoint(-.8,1.5){C}
\tkzDefPoint(-2.5,1){D}
\tkzDefPoint(-1,1){E}
\tkzDefPoint(0,-.5){F}
\tkzMarkAngle[fill=\xrwma,size=.3,mark=|](B,A,C)
\tkzMarkAngle[fill=\xrwma,size=.3,mark=|](D,E,F)
\tkzDrawSegments(A,B A,C)
\tkzDrawSegments(D,E E,F)
\tkzLabelPoint[above](D){$ x' $}
\tkzLabelPoint[left](F){$ y' $}
\tkzLabelPoint[above](E){$ O' $}
\tkzLabelPoint[below](B){$ x $}
\tkzLabelPoint[right](C){$ y $}
\tkzLabelPoint[below](A){$ O $}
\node at (0,-1){$x\hat{O}y=x'\hat{O'}y'$};
\end{tikzpicture}
&\begin{tikzpicture}
\tkzDefPoint(2,0){B}
\tkzDefPoint(0,0){A}
\tkzDefPoint(1,1.5){C}
\tkzDefPoint(.5,-.3){D}
\tkzDefPoint(2.5,-.3){E}
\tkzDefPoint(3.5,1.2){F}
\tkzMarkAngle[fill=\xrwma,size=.3](B,A,C)
\tkzMarkAngle[fill=\xrwma,size=.3](F,E,D)
\tkzDrawSegments(A,B A,C)
\tkzDrawSegments(D,E E,F)
\tkzLabelPoint[below](D){$ x' $}
\tkzLabelPoint[left](F){$ y' $}
\tkzLabelPoint[below](E){$ O' $}
\tkzLabelPoint[above](B){$ x $}
\tkzLabelPoint[left](C){$ y $}
\tkzLabelPoint[below](A){$ O $}
\node at (1.5,-1){$x\hat{O}y+x'\hat{O'}y'=180\degree$};
\end{tikzpicture}\\ 
\end{tabular} 
\end{center}
\end{document}
