\PassOptionsToPackage{no-math,cm-default}{fontspec}
\documentclass[twoside,nofonts,internet,shmeiwseis]{thewria}
\usepackage{amsmath}
\usepackage{xgreek}
\let\hbar\relax
\defaultfontfeatures{Mapping=tex-text,Scale=MatchLowercase}
\setmainfont[Mapping=tex-text,Numbers=Lining,Scale=1.0,BoldFont={Minion Pro Bold}]{Minion Pro}
\newfontfamily\scfont{GFS Artemisia}
\font\icon = "Webdings"
\usepackage[amsbb]{mtpro2}
\usepackage{tikz,pgfplots,tkz-euclide,enumitem}
\usetkzobj{all}
\tkzSetUpPoint[size=7,fill=white]
\xroma{red!70!black}

\newlist{rlist}{enumerate}{3}
\setlist[rlist]{itemsep=0mm,label=\roman*.}
\newlist{brlist}{enumerate}{3}
\setlist[brlist]{itemsep=0mm,label=\bf\roman*.}
\newlist{tropos}{enumerate}{3}
\setlist[tropos]{label=\bf\textit{\arabic*\textsuperscript{oς}\;Τρόπος :},leftmargin=0cm,itemindent=2.3cm,ref=\bf{\arabic*\textsuperscript{oς}\;Τρόπος}}
\newcommand{\tss}[1]{\textsuperscript{#1}}
\newcommand{\tssL}[1]{\MakeLowercase{\textsuperscript{#1}}}
\usepackage{rotating}
\usepackage{hhline}
\usepackage{multicol,multirow,gensymb,mathimatika}
\usepackage{wrap-rl}


\begin{document}
\titlos{Γεωμετρία Β΄ Λυκείου}{Παραλληλόγραμμα - Τραπέζια}{Παραλληλόγραμμο - Ορθογώνιο - Ρομβος - Τετράγωνο}
\orismoi
\Orismos{Παραλληλόγραμμο}
\wrapr{-4mm}{3}{3.8cm}{-7mm}{\begin{tikzpicture}
\tkzDefPoint(-3,-.5){D}
\tkzDefPoint(-2,1){A}
\tkzDefPoint(.5,1){B}
\tkzDefPoint(-.5,-.5){C}
\tkzDefPoint(-2,1){E}
\tkzDefPoint(-2,-0.5){Z}
\tkzMarkRightAngle(C,Z,E)
\draw[pl] (-3,-0.5) -- (-2,1) -- (0.5,1) -- (-0.5,-0.5) -- cycle;
\tkzLabelPoint[above](A){$A$}
\tkzLabelPoint[above](B){$B$}
\tkzLabelPoint[below](C){$\varGamma$}
\tkzLabelPoint[below](D){$\varDelta$}
\draw[pl] (A)--(C);
\draw[pl] (B)--(D);
\draw (-2,1) -- (-2,-0.5);
\tkzInterLL(A,C)(B,D)\tkzGetPoint{O}
\tkzLabelPoint[above,xshift=.3mm](O){$O$}
\tkzDrawPoints(A,B,C,D,O)
\end{tikzpicture}}{
Παραλληλόγραμμο ονομάζεται το τετράπλευρο το οποίο έχει τις απέναντι πλευρές του ανα δύο παράλληλες.
\begin{itemize}[leftmargin=5mm]
\item Τα ευθύγραμμα τμήματα που ενώνουν τις απέναντι κορυφές του παραλληλογράμμου ονομάζονται \textbf{διαγώνιοι}.
\item Το σημείο τομής των διαγωνίων του ονομάζεται \textbf{κέντρο} του παραλληλογράμμου.
\end{itemize}}\mbox{}\\
\vspace{-1.5mm}
\begin{itemize}[itemsep=0mm,leftmargin=5mm]
\item Το ευθύγραμμο τμήμα που έχει τα άκρα του στις απέναντι πλευρές ενός παραλληλογράμου και είναι κάθετο σ' αυτές ονομάζεται \textbf{ύψος}.
\end{itemize}\mbox{}\\
\Orismos{Ορθογώνιο Παραλληλόγραμμο}
\wrapr{-5mm}{7}{3.4cm}{-7mm}{\begin{tikzpicture}
\tkzDefPoint(0,0){D}
\tkzDefPoint(0,1.5){A}
\tkzDefPoint(3,1.5){B}
\tkzDefPoint(3,0){C}
\draw[pl] (0,0) -- (0,1.5) -- (3,1.5) -- (3,0) -- cycle;
\tkzMarkRightAngle(C,D,A)
\tkzMarkRightAngle(B,C,D)
\tkzMarkRightAngle(D,A,B)
\tkzMarkRightAngle(A,B,C)
\tkzLabelPoint[above](A){$A$}
\tkzLabelPoint[above](B){$B$}
\tkzLabelPoint[below](C){$\varGamma$}
\tkzLabelPoint[below](D){$\varDelta$}
\tkzDrawPoints(A,B,C,D)
\end{tikzpicture}}{
Ορθογώνιο ονομάζεται το παραλληλόγραμμο το οποίο έχει όλες τις γωνίες του ορθές. Ισοδύναμα μπορούμε να ορίσουμε το ορθογώνιο ως το παραλληλόγραμμο το οποίο έχει μια ορθή γωνία και κατά συνέπεια από τις ιδιότητες του παραλληλογράμμου, προκύπτουν και οι υπόλοιπες γωνίες του ορθές.}\mbox{}\\\\\\
\Orismos{Ρόμβοσ}
Ρόμβος ονομάζεται το παραλληλόγραμμο το οποίο έχει τις διαδοχικές πλευρές του μεταξύ τους ίσες.\\\\
\Orismos{Τετράγωνο}
Τετράγωνο ονομάζεται το παραλληλόγραμμο το οποίο έιναι και ορθογώνιο και ρόμβος συγχρόνως.
\begin{center}
\begin{tabular}{p{5cm}cp{2.9cm}}
\begin{tikzpicture}[scale=1.4]
\tkzDefPoint(0,0.75){D}
\tkzDefPoint(1.5,1.5){A}
\tkzDefPoint(3,.75){B}
\tkzDefPoint(1.5,0){C}
\draw[pl] (0,0.75) -- (1.5,1.5) -- (3,0.75) -- (1.5,0) -- cycle;
\tkzLabelPoint[above](A){$A$}
\tkzLabelPoint[right](B){$B$}
\tkzLabelPoint[below](C){$\varGamma$}
\tkzLabelPoint[left](D){$\varDelta$}
\tkzDrawPoints(A,B,C,D)
\end{tikzpicture} & & \begin{tikzpicture}[scale=.7]
\tkzDefPoint(0,-1.5){D}
\tkzDefPoint(0,1.5){A}
\tkzDefPoint(3,1.5){B}
\tkzDefPoint(3,-1.5){C}
\tkzMarkRightAngle[scale=1.5](C,D,A)
\tkzMarkRightAngle[scale=1.5](B,C,D)
\tkzMarkRightAngle[scale=1.5](D,A,B)
\tkzMarkRightAngle[scale=1.5](A,B,C)
\draw[pl] (0,-1.5) -- (0,1.5) -- (3,1.5) -- (3,-1.5) -- cycle;
\tkzLabelPoint[above](A){$A$}
\tkzLabelPoint[above](B){$B$}
\tkzLabelPoint[below](C){$\varGamma$}
\tkzLabelPoint[below](D){$\varDelta$}
\tkzDrawPoints(A,B,C,D)
\end{tikzpicture} \\ 
\end{tabular} 
\end{center}
\newpage
\thewrhmata
\Thewrhma{Ιδιότητες παραλληλογράμμου}
Σε κάθε παραλληλόγραμμο $ AB\varGamma\varDelta $ ισχύει ότι :\\
\wrapr{-11mm}{7}{5cm}{0mm}{\begin{tikzpicture}[scale=1.3]
\tkzDefPoint(-3,-.5){D}
\tkzDefPoint(-2,1){A}
\tkzDefPoint(.5,1){B}
\tkzDefPoint(-.5,-.5){C}
\tkzMarkAngle[mark=|,size=.4,fill=\xrwma!70](A,B,C)
\tkzMarkAngle[mark=|,size=.4,fill=\xrwma!70](C,D,A)
\tkzMarkAngle[mark=||,size=.3,fill=\xrwma!70](D,A,B)
\tkzMarkAngle[mark=||,size=.3,fill=\xrwma!70](B,C,D)
\draw[pl] (-3,-0.5) -- (-2,1) -- (0.5,1) -- (-0.5,-0.5) -- cycle;
\tkzLabelPoint[above](A){$A$}
\tkzLabelPoint[above](B){$B$}
\tkzLabelPoint[below](C){$\varGamma$}
\tkzLabelPoint[below](D){$\varDelta$}
\draw[pl] (A)--(C);
\draw[pl] (B)--(D);
\tkzInterLL(A,C)(B,D)\tkzGetPoint{O}
\tkzLabelPoint[above,xshift=.3mm](O){$O$}
\tkzDrawPoints(A,B,C,D,O)
\tkzMarkSegments[mark=|,size=3pt](A,B C,D)
\tkzMarkSegments[mark=||,size=3pt](A,D C,B)
\end{tikzpicture}}{
\begin{rlist}
\item Οι απέναντι πλευρές του είναι ίσες : $ AB=\varGamma\varDelta $ και $ A\varDelta=B\varGamma $.
\item Οι απέναντι γωνίες του είναι ίσες : $ \hat{A}=\hat{\varGamma} $ και $ \hat{B}=\hat{\varDelta} $.
\item Δύο διαδοχικές γωνίες του είναι παραπληρωματικές : $ \hat{A}+\hat{B}=180\degree $.
\item Οι διαγώνιοι διχοτομούνται.
\end{rlist}}\mbox{}\\\\\\
\Thewrhma{Κριτήρια Παραλληλογράμμου}
Ένα τετράπλευρο $ AB\varGamma\varDelta $ θα είναι παραλληλόγραμμο αν ισχύει μια από τις παρακάτω προτάσεις :
\begin{rlist}
\item Οι απέναντι πλευρές του είναι παράλληλες.
\item Οι απέναντι πλευρές του είναι ίσες.
\item Δύο απέναντι πλευρές του είναι παράλληλες και ίσες.
\item Οι απέναντι γωνίες του είναι ίσες.
\item Οι διαγώνιοί του διχοτομούνται.
\end{rlist}
\Thewrhma{Πορίσματα για το παραλληλόγραμμο}
\vspace{-5mm}
\begin{rlist}
\item Το κέντρο ενός παραλληλογράμμου $ AB\varGamma\varDelta $ είναι κέντρο συμμετρίας του.
\item Εαν δύο ή περισσότερα παράλληλα τμήματα έχουν τα άκρα τους πάνω σε παράλληλες ευθείες τότε είναι ίσα.
\end{rlist}
\Thewrhma{Ιδιότητες ορθογωνίου}
Σε κάθε ορθογώνιο $ AB\varGamma\varDelta $ ισχύουν οι παρακάτω προτάσεις :\\
\wrapr{-11mm}{7}{4.6cm}{-4mm}{\begin{tikzpicture}[scale=1.2]
\tkzDefPoint(0,0){D}
\tkzDefPoint(0,1.5){A}
\tkzDefPoint(3,1.5){B}
\tkzDefPoint(3,0){C}
\tkzDefPoint(1.5,.75){O}
\draw[pl] (0,0) -- (0,1.5) -- (3,1.5) -- (3,0) -- cycle;
\draw[pl] (A)--(C);
\draw[pl] (B)--(D);
\tkzMarkRightAngle[fill=\xrwma](C,D,A)
\tkzMarkRightAngle[fill=\xrwma](B,C,D)
\tkzMarkRightAngle[fill=\xrwma](D,A,B)
\tkzMarkRightAngle[fill=\xrwma](A,B,C)
\tkzLabelPoint[above left](A){$A$}
\tkzLabelPoint[above right](B){$B$}
\tkzLabelPoint[right](C){$\varGamma$}
\tkzLabelPoint[left](D){$\varDelta$}
\tkzLabelPoint[above](O){$O$}
\tkzDrawPoints(A,B,C,D,O)
\tkzMarkSegments[mark=|,size=3pt](A,B B,C C,D D,A)
\tkzMarkSegments[mark=||,pos=.55,size=3pt](A,C B,D)
\end{tikzpicture}}{
\begin{rlist}
\item Οι διαγώνιοι του είναι ίσες : $ A\varGamma=B\varDelta $.
\item Όλες του οι γωνίες είναι ίσες : $ \hat{A}=\hat{B}=\hat{\varGamma}=\hat{\varDelta}=90\degree $.
\item Έχει όλες τις ιδιότητες ενός παραλληλογράμμου.
\end{rlist}}\mbox{}\\\\
\Thewrhma{Κριτήρια ορθογωνίου}
Ένα τετράπλευρο $ AB\varGamma\varDelta $ είναι ορθογώνιο αν ισχύει μια από τις παρακάτω προτάσεις :
\begin{rlist}
\item Είναι παραλληλόγραμμο και έχει μια ορθή γωνία.
\item Είναι παραλληλόγραμμο και οι διαγώνιοί του είναι ίσες.
\item Έχει 3 ορθές γωνίες.
\item Έχει όλες τις γωνίες του ίσες.
\end{rlist}
\Thewrhma{Ιδιότητες ρόμβου}
Σε κάθε ρόμβο $ AB\varGamma\varDelta $ ισχύουν οι παρακάτω προτάσεις.\\
\wrapr{-11mm}{5}{5.2cm}{-4mm}{\begin{tikzpicture}[scale=.7]
\tkzDefPoint(0,1.5){D}
\tkzDefPoint(3,3){A}
\tkzDefPoint(6,1.5){B}
\tkzDefPoint(3,0){C}
\tkzDefPoint(3,1.5){O}
\tkzMarkRightAngle[size=.4](B,O,A)
\tkzMarkAngle[size=.7,mark=|,fill=\xrwma](B,D,A)
\tkzMarkAngle[size=.7,mark=|,fill=\xrwma](C,D,B)
\tkzMarkAngle[size=.7,mark=|,fill=\xrwma](A,B,D)
\tkzMarkAngle[size=.7,mark=|,fill=\xrwma](D,B,C)
\tkzMarkAngle[size=.5,mark=||,fill=\xrwma](D,A,C)
\tkzMarkAngle[size=.5,mark=||,fill=\xrwma](C,A,B)
\tkzMarkAngle[size=.5,mark=||,fill=\xrwma](B,C,A)
\tkzMarkAngle[size=.5,mark=||,fill=\xrwma](A,C,D)
\draw[pl] (A)--(B)--(C)--(D) -- cycle;
\draw[pl] (A)--(C);
\draw[pl] (B)--(D);
\tkzLabelPoint[above](A){$A$}
\tkzLabelPoint[right](B){$B$}
\tkzLabelPoint[below](C){$\varGamma$}
\tkzLabelPoint[left](D){$\varDelta$}
\tkzLabelPoint[above left](O){$O$}
\tkzDrawPoints(A,B,C,D,O)
\tkzMarkSegments[mark=|,size=3pt](A,B B,C C,D D,A)
\end{tikzpicture}}{
\begin{rlist}
\item Οι διαδοχικές πλευρές του είναι ίσες : $ AB=B\varGamma=\varGamma\varDelta=\varDelta A $.
\item Οι διαγώνιοί του τέμνονται κάθετα : $ A\varGamma\bot B\varDelta $.
\item Οι διαγώνιοί του διχοτομούν τις γωνίες του :
\begin{multicols}{2}
\begin{itemize}[itemsep=0mm]
\item $ A\varGamma $ διχ. των $ \hat{A} $ και $ \hat{\varGamma} $.
\item $ B\varDelta $ διχ. των $ \hat{B} $ και $ \hat{\varDelta} $.
\end{itemize}
\end{multicols}
\vspace{-3mm}
\item Έχει όλες τις ιδιότητες ενός παραλληλογράμμου.
\end{rlist}}\mbox{}\\\\\\
\Thewrhma{Κριτήρια ρόμβου}
Ένα τετράπλευρο $ AB\varGamma\varDelta $ είναι ρόμβος αν ισχύει μια από τις παρακάτω προτάσεις :
\begin{rlist}
\item Όλες οι πλευρές του είναι ίσες.
\item Είναι παραλληλόγραμμο και έχει δύο διαδοχικές πλευρές ίσες.
\item Είναι παραλληλόγραμμο και έχει διαγώνιους κάθετες.
\item Είναι παραλληλόγραμμο και μια διαγώνιος διχοτομεί μια γωνία.
\end{rlist}
\Thewrhma{Ιδιότητες τετραγώνου}
\wrapr{-4mm}{8}{3.5cm}{-7mm}{\begin{tikzpicture}[scale=1]
\tkzDefPoint(0,-1.5){D}
\tkzDefPoint(0,1.5){A}
\tkzDefPoint(3,1.5){B}
\tkzDefPoint(3,-1.5){C}
\tkzDefPoint(1.5,0){O}
\tkzMarkRightAngle[scale=1.5,fill=\xrwma](C,D,A)
\tkzMarkRightAngle[scale=1.5,fill=\xrwma](B,C,D)
\tkzMarkRightAngle[scale=1.5,fill=\xrwma](D,A,B)
\tkzMarkRightAngle[scale=1.5,fill=\xrwma](A,B,C)
\draw[pl] (A) -- (B) -- (C) -- (D) -- cycle;
\draw[pl] (A)--(C);
\draw[pl] (B)--(D);
\tkzLabelPoint[above](A){$A$}
\tkzLabelPoint[above](B){$B$}
\tkzLabelPoint[below](C){$\varGamma$}
\tkzLabelPoint[below](D){$\varDelta$}
\tkzLabelPoint[above](O){$O$}
\tkzDrawPoints(A,B,C,D,O)
\tkzMarkSegments[mark=|,size=3pt](A,B B,C C,D D,A)
\tkzMarkSegments[mark=||,pos=.55,size=3pt](A,C B,D)
\end{tikzpicture}}{
Κάθε τετράγωνο $ AB\varGamma\varDelta $ έχει όλες τις ιδιότητες του παρραληλογράμμου, του ορθογωνίου και του ρόμβου :
\begin{rlist}
\item Όλες οι πλευρές του είναι ίσες : $ AB=B\varGamma=\varGamma\varDelta=A\varDelta $.
\item Όλες οι γωνίες του είναι ίσες : $ \hat{A}=\hat{B}=\hat{\varGamma}=\hat{\varDelta}=90\degree $.
\item Οι απέναντι πλευρές είναι παράλληλες : $ AB\parallel\varGamma\varDelta\ ,\ A\varDelta\parallel B\varGamma $.
\item Οι διαγώνιοί του είναι ίσες,διχοτομούνται , διχοτομούν τις γωνίες του και τέμνονται κάθετα.
\begin{multicols}{2}
\begin{itemize}
\item $ A\varGamma=B\varDelta $ και $ A\varGamma\bot B\varDelta $.
\item $ AO=O\varGamma\ ,\ BO=O\varDelta $.
\item $ A\varGamma $ διχ. των $ \hat{A} $ και $ \hat{\varGamma} $.
\item $ B\varDelta $ διχ. των $ \hat{B} $ και $ \hat{\varDelta} $.
\end{itemize}
\end{multicols}
\end{rlist}}\mbox{}\\\\\\
\Thewrhma{Κριτήρια τετραγώνου}
Ένα τετράπλευρο $ AB\varGamma\varDelta $ είναι τετράγωνο εαν είναι παραλληλόγραμμο και ισχύει και μια από τις παρακάτω προτάσεις :
\begin{rlist}
\item Έχει μια ορθή γωνία και δύο διαδοχικές πλευρές ίσες.
\item Έχει μια ορθή γωνία και διαγώνιους κάθετες.
\item Έχει μια ορθή γωνία και μια διαγώνιος διχοτομεί μια γωνία.
\item Έχει διαγώνιους ίσες και κάθετες.
\item Έχει διαγώνιους ίσες και δύο διαδοχικές πλευρές ίσες.
\item Έχει διαγώνιους ίσες και μια απ' αυτές διχοτομεί μια γωνία.
\end{rlist}
Από τα παραπάνω κριτήρια παρατηρούμε ότι συνδυάζονται δύο ιδιότητες του ορθογωνίου με τρεις ιδιότητες του ρόμβου προκειμένου να οριστούν τα κριτήρια αυτά. Οι συνδιασμοί αυτοί φαίνονται στον παρακάτω πίνακα.
\begin{center}
\begin{tabular}{c|c|c|c}
\hline \multicolumn{4}{c}{\textbf{{\boldmath$ AB\varGamma\varDelta $} Παραλληλόγραμμο και}}  \rule[-2ex]{0pt}{5.5ex}\\ 
\hhline{====} \multicolumn{2}{c|}{} & \multicolumn{2}{c}{\textbf{Ιδιότητες Ορθογωνίου}}  \rule[-2ex]{0pt}{5.5ex}\\ 
\hhline{~~|--}  \multicolumn{2}{c|}{}  & Μια ορθή γωνία & Διαγώνιοι ίσες \rule[-2ex]{0pt}{5.5ex}\\ 
\hline \multirow{5}{*}{\textbf{Ιδιότητες ρόμβου}} & Διαδοχικές πλευρές ίσες & 1ο Κριτήριο & 4ο Κριτήριο \rule[-2ex]{0pt}{5.5ex}\\ 
\hhline{~-|--} \rule[-2ex]{0pt}{5.5ex} & Διαγώνιοι κάθετες & 2ο Κριτήριο & 5ο Κριτήριο \\ 
\hhline{~---} \rule[-2ex]{0pt}{5.5ex} & Διαγώνιος διχοτόμεί μια γωνία & 3ο Κριτήριο & 6ο Κριτήριο \\ 
\hline 
\end{tabular} 
\end{center}
\newpage
\noindent
\Thewrhma{Μέσα τετραπλεύρου}
Τα μέσα των πλευρών ενός κυρτού ή μη κυρτού τετραπλεύρου ορίζουν παραλληλόγραμμο.
\begin{center}
\begin{tabular}{p{5cm}cp{5cm}}
\begin{tikzpicture}
\tkzDefPoint[label=left:$A$](1,3){A}
\tkzDefPoint[label=right:$B$](4,2.5){B}
\tkzDefPoint[label=right:$\varGamma$](4.5,1){C}
\tkzDefPoint[label=left:$\varDelta$](0,0.5){D}
\tkzDefMidPoint(A,B) \tkzGetPoint{K}
\tkzDefMidPoint(C,B) \tkzGetPoint{L}
\tkzDefMidPoint(A,D) \tkzGetPoint{N}
\tkzDefMidPoint(C,D) \tkzGetPoint{M}
\tkzLabelPoint[above](K){$K$}
\tkzLabelPoint[right](L){$\varLambda$}
\tkzLabelPoint[below](M){$M$}
\tkzLabelPoint[left](N){$N$}
\draw[pl](A)--(B)--(C)--(D)--cycle;
\draw[pl,\xrwma](K)--(L)--(M)--(N)--cycle;
\tkzDrawPoints(A,B,C,D,K,L,M,N)
\end{tikzpicture} && \begin{tikzpicture}
\tkzDefPoint[label=left:$A$](2,3){A}
\tkzDefPoint[label=right:$B$](4.5,0.5){B}
\tkzDefPoint[label=right:$\varGamma$](2.5,1.2){C}
\tkzDefPoint[label=left:$\varDelta$](.4,0.5){D}
\tkzDefMidPoint(A,B) \tkzGetPoint{K}
\tkzDefMidPoint(C,B) \tkzGetPoint{L}
\tkzDefMidPoint(A,D) \tkzGetPoint{N}
\tkzDefMidPoint(C,D) \tkzGetPoint{M}
\tkzLabelPoint[right](K){$K$}
\tkzLabelPoint[below](L){$\varLambda$}
\tkzLabelPoint[below](M){$M$}
\tkzLabelPoint[left](N){$N$}
\draw[pl](A)--(B)--(C)--(D)--cycle;
\draw[pl,\xrwma](K)--(L)--(M)--(N)--cycle;
\tkzDrawPoints(A,B,C,D,K,L,M,N)
\end{tikzpicture} \\ 
\end{tabular} 
\end{center}
\newpage
\begin{sidewaysfigure}
\begin{tabular}{c|>{\centering\arraybackslash}m{5.3cm}|>{\centering\arraybackslash}m{4.5cm}|>{\centering\arraybackslash}m{5.5cm}|>{\centering\arraybackslash}m{6.5cm}}
\hline\rule[-2ex]{0pt}{5.5ex}& \textbf{Παραλληλόγραμμο} & \textbf{Ορθογώνιο} & \textbf{Ρόμβος} & \textbf{Τετράγωνο} \\
\hhline{=====}\rule[-2ex]{0pt}{5.5ex}\textbf{Σχήμα}  & \begin{tikzpicture}
\tkzDefPoint(-3,-.5){D}
\tkzDefPoint(-2,1){A}
\tkzDefPoint(.5,1){B}
\tkzDefPoint(-.5,-.5){C}
\tkzDefPoint(-2,1){E}
\draw[pl] (-3,-0.5) -- (-2,1) -- (0.5,1) -- (-0.5,-0.5) -- cycle;
\tkzLabelPoint[above](A){$A$}
\tkzLabelPoint[above](B){$B$}
\tkzLabelPoint[below](C){$\varGamma$}
\tkzLabelPoint[below](D){$\varDelta$}
\draw[pl] (A)--(C);
\draw[pl] (B)--(D);
\tkzInterLL(A,C)(B,D)\tkzGetPoint{O}
\tkzLabelPoint[above,xshift=.3mm](O){$O$}
\tkzDrawPoints(A,B,C,D,O)
\end{tikzpicture} & \begin{tikzpicture}[scale=1]
\tkzDefPoint(0,0){D}
\tkzDefPoint(0,1.8){A}
\tkzDefPoint(3,1.8){B}
\tkzDefPoint(3,0){C}
\tkzDefPoint(1.5,.9){O}
\draw[pl] (0,0) -- (0,1.8) -- (3,1.8) -- (3,0) -- cycle;
\draw[pl] (A)--(C);
\draw[pl] (B)--(D);
\tkzMarkRightAngle(C,D,A)
\tkzMarkRightAngle(B,C,D)
\tkzMarkRightAngle(D,A,B)
\tkzMarkRightAngle(A,B,C)
\tkzLabelPoint[above left](A){$A$}
\tkzLabelPoint[above right](B){$B$}
\tkzLabelPoint[right](C){$\varGamma$}
\tkzLabelPoint[left](D){$\varDelta$}
\tkzLabelPoint[above](O){$O$}
\tkzDrawPoints(A,B,C,D,O)
\end{tikzpicture} & \begin{tikzpicture}[scale=.7]
\tkzDefPoint(0,1.5){D}
\tkzDefPoint(3,3){A}
\tkzDefPoint(6,1.5){B}
\tkzDefPoint(3,0){C}
\tkzDefPoint(3,1.5){O}
\tkzMarkRightAngle[size=.4](B,O,A)
\tkzMarkAngle[size=.7](B,D,A)
\tkzMarkAngle[size=.7](C,D,B)
\tkzMarkAngle[size=.7](A,B,D)
\tkzMarkAngle[size=.7](D,B,C)
\tkzMarkAngle[size=.5](D,A,C)
\tkzMarkAngle[size=.5](C,A,B)
\tkzMarkAngle[size=.5](B,C,A)
\tkzMarkAngle[size=.5](A,C,D)
\draw[pl] (A)--(B)--(C)--(D) -- cycle;
\draw[pl] (A)--(C);
\draw[pl] (B)--(D);
\tkzLabelPoint[above](A){$A$}
\tkzLabelPoint[right](B){$B$}
\tkzLabelPoint[below](C){$\varGamma$}
\tkzLabelPoint[left](D){$\varDelta$}
\tkzLabelPoint[above left](O){$O$}
\tkzDrawPoints(A,B,C,D,O)
\end{tikzpicture} & \begin{tikzpicture}[scale=.7]
\tkzDefPoint(0,-1.5){D}
\tkzDefPoint(0,1.5){A}
\tkzDefPoint(3,1.5){B}
\tkzDefPoint(3,-1.5){C}
\tkzDefPoint(1.5,0){O}
\tkzMarkRightAngle[scale=1.5](C,D,A)
\tkzMarkRightAngle[scale=1.5](B,C,D)
\tkzMarkRightAngle[scale=1.5](D,A,B)
\tkzMarkRightAngle[scale=1.5](A,B,C)
\draw[pl] (A) -- (B) -- (C) -- (D) -- cycle;
\draw[pl] (A)--(C);
\draw[pl] (B)--(D);
\tkzLabelPoint[above](A){$A$}
\tkzLabelPoint[above](B){$B$}
\tkzLabelPoint[below](C){$\varGamma$}
\tkzLabelPoint[below](D){$\varDelta$}
\tkzLabelPoint[above](O){$O$}
\tkzDrawPoints(A,B,C,D,O)
\end{tikzpicture} \\
\hline\rule[-7ex]{0pt}{14ex}\textbf{Ορισμός}  & Παραλληλόγραμμο ονομάζεται το τετράπλευρο το οποίο έχει τις απέναντι πλευρές του ανα δύο παράλληλες. & Ορθογώνιο ονομάζεται το παραλληλόγραμμο το οποίο έχει όλες τις γωνίες του ορθές. & Ρόμβος ονομάζεται το παραλληλόγραμμο το οποίο έχει τις διαδοχικές πλευρές του μεταξύ τους ίσες. & Τετράγωνο ονομάζεται το παραλληλόγραμμο το οποίο έιναι και ορθογώνιο και ρόμβος. \\
\hline \textbf{Ιδιότητες} & \begin{rlist}[leftmargin=5mm]
\item Οι απέναντι πλευρές είναι ίσες.
\item Οι απέναντι γωνίες είναι ίσες.
\item Δύο διαδοχικές γωνίες είναι παραπληρωματικές.
\item Οι διαγώνιοι διχοτομούνται.
\end{rlist} & \begin{rlist}[leftmargin=5mm]
\item Οι διαγώνιοι είναι ίσες.
\item Όλες οι γωνίες είναι ίσες.
\item Έχει όλες τις ιδιότητες ενός παραλλαλογράμμου.
\end{rlist} & \begin{rlist}[leftmargin=5mm]
\item Οι διαδοχικές πλευρές είναι ίσες.
\item Οι διαγώνιοί τέμνονται κάθετα.
\item Οι διαγώνιοί διχοτομούν τις γωνίες του.
\vspace{-3mm}
\item Έχει όλες τις ιδιότητες ενός παραλληλογράμμου.
\end{rlist} & \begin{rlist}[leftmargin=5mm]
\item Όλες οι πλευρές είναι ίσες.
\item Όλες οι γωνίες είναι ίσες.
\item Οι απέναντι πλευρές είναι παράλληλες.
\item Οι διαγώνιοί είναι ίσες, διχοτομούν τις γωνίες του και τέμνονται κάθετα.
\end{rlist} \\
\hline\rule[-2ex]{0pt}{5.5ex}\textbf{Κριτήρια}  & \begin{rlist}[leftmargin=5mm]
\item Οι απέναντι πλευρές είναι παράλληλες.
\item Οι απέναντι πλευρές είναι ίσες.
\item Δύο απέναντι πλευρές είναι παράλληλες και ίσες.
\item Οι απέναντι γωνίες είναι ίσες.
\item Οι διαγώνιοί διχοτομούνται.
\end{rlist} & \begin{rlist}[leftmargin=5mm]
\item Είναι παραλληλόγραμμο και έχει μια ορθή γωνία.
\item Είναι παραλληλόγραμμο και οι διαγώνιοί είναι ίσες.
\item Έχει 3 ορθές γωνίες.
\item Έχει όλες τις γωνίες ίσες.
\end{rlist} & \begin{rlist}[leftmargin=5mm]
\item Όλες οι πλευρές του είναι ίσες.
\item Είναι παραλληλόγραμμο και έχει δύο διαδοχικές πλευρές ίσες.
\item Είναι παραλληλόγραμμο και έχει διαγώνιοιυς κάθετες.
\item Είναι παραλληλόγραμμο και μια διγώνιος διχοτομεί μια γωνία.
\end{rlist} & Παραλληλόγραμμο και
\begin{rlist}[leftmargin=5mm]
\item Έχει μια ορθή γωνία και δύο διαδοχικές πλευρές ίσες.
\item Έχει μια ορθή γωνία και διαγώνιους κάθετες.
\item Έχει μια ορθή γωνία και μια διαγώνιος διχοτομεί μια γωνία.
\item Έχει διαγώνιους ίσες και κάθετες.
\item Έχει διαγώνιους ίσες και δύο διαδοχικές πλευρές ίσες.
\item Έχει διαγώνιους ίσες και μια απ' αυτές διχοτομεί μια γωνία.
\end{rlist} \\
\hline
\end{tabular}
\end{sidewaysfigure}
\end{document}
