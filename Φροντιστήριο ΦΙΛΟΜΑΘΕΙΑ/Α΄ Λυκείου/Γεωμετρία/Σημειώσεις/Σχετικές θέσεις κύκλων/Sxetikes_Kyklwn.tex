\PassOptionsToPackage{no-math,cm-default}{fontspec}
\documentclass[twoside,nofonts,internet,shmeiwseis]{thewria}
\usepackage{amsmath}
\usepackage{xgreek}
\let\hbar\relax
\defaultfontfeatures{Mapping=tex-text,Scale=MatchLowercase}
\setmainfont[Mapping=tex-text,Numbers=Lining,Scale=1.0,BoldFont={Minion Pro Bold}]{Minion Pro}
\newfontfamily\scfont{GFS Artemisia}
\font\icon = "Webdings"
\usepackage[amsbb]{mtpro2}
\usepackage{tikz,pgfplots,tkz-euclide,mathtools}
\tkzSetUpPoint[size=7,fill=white]
\xroma{red!70!black}
%------- ΣΥΣΤΗΜΑ -------------------
\usepackage{systeme,regexpatch}
\makeatletter
% change the definition of \sysdelim not to store `\left` and `\right`
\def\sysdelim#1#2{\def\SYS@delim@left{#1}\def\SYS@delim@right{#2}}
\sysdelim\{. % reinitialize

% patch the internal command to use
% \LEFTRIGHT<left delim><right delim>{<system>}
% instead of \left<left delim<system>\right<right delim>
\regexpatchcmd\SYS@systeme@iii
{\cB.\c{SYS@delim@left}(.*)\c{SYS@delim@right}\cE.}
{\c{SYS@MT@LEFTRIGHT}\cB\{\1\cE\}}
{}{}
\def\SYS@MT@LEFTRIGHT{%
\expandafter\expandafter\expandafter\LEFTRIGHT
\expandafter\SYS@delim@left\SYS@delim@right}
\makeatother
\newcommand{\synt}[2]{{\scriptsize \begin{matrix}
\times#1\\\\ \times#2
\end{matrix}}}
%----------------------------------------
%------ ΜΗΚΟΣ ΓΡΑΜΜΗΣ ΚΛΑΣΜΑΤΟΣ ---------
\DeclareRobustCommand{\frac}[3][0pt]{%
{\begingroup\hspace{#1}#2\hspace{#1}\endgroup\over\hspace{#1}#3\hspace{#1}}}
%----------------------------------------

\newlist{rlist}{enumerate}{3}
\setlist[rlist]{itemsep=0mm,label=\roman*.}
\newlist{brlist}{enumerate}{3}
\setlist[brlist]{itemsep=0mm,label=\bf\roman*.}
\newlist{tropos}{enumerate}{3}
\setlist[tropos]{label=\bf\textit{\arabic*\textsuperscript{oς}\;Τρόπος :},leftmargin=0cm,itemindent=2.3cm,ref=\bf{\arabic*\textsuperscript{oς}\;Τρόπος}}
\newcommand{\tss}[1]{\textsuperscript{#1}}
\newcommand{\tssL}[1]{\MakeLowercase{\textsuperscript{#1}}}
\usetkzobj{all}
\usepackage{hhline}
%----------- ΓΡΑΦΙΚΕΣ ΠΑΡΑΣΤΑΣΕΙΣ ---------
\pgfkeys{/pgfplots/aks_on/.style={axis lines=center,
xlabel style={at={(current axis.right of origin)},xshift=1.5ex, anchor=center},
ylabel style={at={(current axis.above origin)},yshift=1.5ex, anchor=center}}}
\pgfkeys{/pgfplots/grafikh parastash/.style={\xrwma,line width=.4mm,samples=200}}
\pgfkeys{/pgfplots/belh ar/.style={tick label style={font=\scriptsize},axis line style={-latex}}}
%-----------------------------------------
\usepackage{multicol}
\usepackage{wrap-rl}
\tkzSetUpPoint[size=7,fill=white]
\tikzstyle{pl}=[line width=0.3mm]
\tikzstyle{plm}=[line width=0.4mm]
\usepackage{gensymb}


\begin{document}
\titlos{Γεωμετρία Α΄ Λυκείου}{Τρίγωνα}{Σχετικές θέσεις κύκλων}
\orismoi
\Orismos{Διάκεντρος κύκλων}
\wrapr{-4mm}{5}{4cm}{-5mm}{\begin{tikzpicture}
\tkzDefPoint[label=left:$K$](0,0){K}
\tkzDefPoint[label=right:$\varLambda$](2.3,0){L}
\draw[pl](K) circle (.9);
\draw[pl](L) circle (.7);
\draw[pl,\xrwma](K)--(L);
\node at(1.3,.2){{\footnotesize $ \delta $}};
\tkzDrawPoints(K,L)
\end{tikzpicture}}{
Διάκεντρος δύο κύκλων ονομάζεται το ευθύγραμμο τμήμα που ενώνει τα κέντρα τους. Συμβολίζεται με $ \delta $.}\mbox{}\\\\\\
\Orismos{Σχετικές θέσεις κύκλων}
Οι τρεις σχετικές θέσεις μεταξύ δύο κύκλων είναι οι ακόλουθες :
\begin{enumerate}[itemsep=0mm,label=\bf\arabic*.]
\item \textbf{Κύκλοι χωρίς κοινά σημεία}\\
Ένας κύκλος λέγεται εξωτερικός ή εσωτερικός ενός άλλου κύκλου όταν όλα τα σημεία του πρώτου βρίσκονται στο εξωτερικό ή εσωτερικό μέρος του δεύτερου αντίστοιχα.. Οι κύκλοι αυτοί δεν έχουν κανένα κοινό σημείο.
\item \textbf{Εφαπτόμενοι κύκλοι}\\
Εφαπτόμενοι ονομάζονται οι κύκλοι οι οποίοι έχουν ένα κοινό σημείο. Το σημείο αυτό λέγεται \textbf{σημείο επαφής}.
\item \textbf{Τεμνόμενοι κύκλοι}\\
Τεμνόμενοι ονομάζονται οι κύκλοι οι οποίοι έχουν δύο κοινά σημεία. Το ευθύγραμμο τμήμα που ενώνει τα σημεία αυτά ονομάζεται \textbf{κοινή χορδή} των δύο κύκλων.
\end{enumerate}
\begin{center}
\begin{tabular}{c|c|c|c|c}
\hline  \multicolumn{2}{c|}{\textbf{Χωρίς κοινά σημεία}} & \multicolumn{2}{c|}{\textbf{Εφαπτόμενοι}}  & \textbf{Τεμνόμενοι} \rule[-2ex]{0pt}{5.5ex}\\ 
\hhline{=====} \rule[-2ex]{0pt}{9.5ex} \begin{tikzpicture}
\tkzDefPoint[label=left:$K$](0,0){K}
\tkzDefPoint[label=left:$\varLambda$](1.5,0){L}
\draw[pl](K) circle (.8);
\draw[pl](L) circle (.5);
\tkzDrawPoints(K,L)
\end{tikzpicture} & \begin{tikzpicture}
\tkzDefPoint[label=left:$K$](0,0){K}
\tkzDefPoint[label=above:$\varLambda$](.3,0){L}
\draw[pl](K) circle (.8);
\draw[pl](L) circle (.4);
\tkzDrawPoints(K,L)
\end{tikzpicture} & \begin{tikzpicture}
\tkzDefPoint(0,0){K}
\tkzDefPoint[label=right:$A$](.8,0){A}
\tkzDefPoint[label=above:$\varLambda$](.3,0){L}
\draw[pl](K) circle (.8);
\draw[pl](L) circle (.5);
\tkzLabelPoint[left,xshift=-.7mm,fill=white,inner sep=.3mm](K){$K$}
\tkzDrawPoints(K,L,A)
\end{tikzpicture} & \begin{tikzpicture}
\tkzDefPoint[label=left:$K$](0,0){K}
\tkzDefPoint[label=left:$A$](.8,0){A}
\tkzDefPoint[label=left:$\varLambda$](1.3,0){L}
\draw[pl](K) circle (.8);
\draw[pl](L) circle (.5);
\tkzDrawPoints(K,L,A)
\end{tikzpicture} & \begin{tikzpicture}
\tkzDefPoint[label=left:$K$](0,0){K}
\tkzDefPoint[label=right:$\varLambda$](1,0){L}
\tkzDefPoint[label=below:$B$](330:.8){B}
\tkzDefPoint[label=above:$A$](30:.8){A}
\draw[pl](K) circle (.8);
\draw[pl](L) circle (.5);
\draw[pl,\xrwma](A)--(B);
\tkzDrawPoints(K,L,B,A)
\end{tikzpicture}\\ 
\hline 
\end{tabular}
\end{center} 
\Orismos{Κοινή εφαπτομένη δύο κύκλων}
Για την κοινή εφαπτομένη δύο κύκλων διακρίνουμε τις εξής δύο περιπτώσεις :
\begin{enumerate}[itemsep=0mm,label=\bf\arabic*.]
\item \textbf{Κοινή εξωτερική εφαπτομένη}\\
Κοινή εξωτερική εφαπτομένη δύο κύκλων ονομάζεται η ευθεία η οποία εφάπτεται και στους δύο κύκλους έτσι ώστε να βρίσκονται και οι δύο κύκλοι στο ίδιο ημιεπίπεδο.
\item \textbf{Κοινή εσωτερική εφαπτομένη}\\
Κοινή εσωτερική εφαπτομένη δύο κύκλων ονομάζεται η ευθεία η οποία εφάπτεται και στους δύο κύκλους έτσι ώστε να βρίσκονται εκατέρωθεν αυτής.
\end{enumerate}
\begin{center}
\begin{tabular}{cc}
\begin{tikzpicture}
\tkzDefPoint[label=left:$K$](0,0){K}
\tkzDefPoint[label=left:$\varLambda$](2.3,0){L}
\tkzDefPoint[label=above:$A$](83.6:1){A}
\tkzDefPoint[label=above:$B$,shift={(2.3,0)}](83.6:.77){B}
\draw[pl](K) circle (1);
\draw[pl](L) circle (.77);
\tkzDrawLine[add=.4 and .4,color=\xrwma](A,B)
\tkzDrawPoints(K,L,A,B)
\end{tikzpicture} & \begin{tikzpicture}
\tkzDefPoint[label=left:$K$](0,0){K}
\tkzDefPoint[label=left:$\varLambda$](2.08,0){L}
\tkzDefPoint[label=right:$A$](31.58:1){A}
\tkzDefPoint[label=left:$B$,shift={(2.08,0)}](211.58:.77){B}
\draw[pl](K) circle (1);
\draw[pl](L) circle (.77);
\tkzDrawLine[add=.5 and .5,color=\xrwma](A,B)
\tkzDrawPoints(K,L,A,B)
\end{tikzpicture} \\ 
\end{tabular} 
\end{center}
\thewrhmata
\Thewrhma{Σχετικές θέσεις κύκλων}
Για τις σχετικές θέσεις μεταξύ δύο κύκλων $ \left( K,R\right) $ και $ (\varLambda,\rho) $, με $ R>\rho $ ισχύουν οι ακόλουθες προτάσεις :
\begin{rlist}
\item Ο κύκλος $ (\varLambda,\rho) $ είναι εξωτερικός του κύκλου $ \left( K,R\right) $ αν και μόνο αν η διάκεντρος είναι μεγαλύτερη από το άθροισμα των ακτίνων τους.
\[ \delta>R+\rho \]
\item Ο κύκλος $ (\varLambda,\rho) $ είναι εσωτερικός του κύκλου $ \left( K,R\right) $ αν και μόνο αν η διάκεντρος είναι μικρότερη από τη διαφορά των ακτίνων τους.
\[ \delta<R-\rho \]
\item Οι δύο κύκλοι $ \left( K,R\right) $ και $ (\varLambda,\rho) $ εφάπτονται εξωτερικά αν και μόνο αν η διάκεντρος είναι ίση με το άθροισμα των ακτίνων τους.
\[ \delta=R+\rho \]
\item Οι δύο κύκλοι $ \left( K,R\right) $ και $ (\varLambda,\rho) $ εφάπτονται εσωτερικά αν και μόνο αν η διάκεντρος είναι ίση με τη διαφορά των ακτίνων τους.
\[ \delta=R-\rho \]
\item Οι δύο κύκλοι $ \left( K,R\right) $ και $ (\varLambda,\rho) $ τέμνονται αν και μόνο αν η διάκεντρος είναι μεταξύ του αθροίσματος και της διαφοράς των ακτίνων τους.
\[ R-\rho<\delta<R+\rho \]
\end{rlist}
Γενικότερα οι προηγούμενες σχέσεις μεταξύ των ακτίνων των δύο κύκλων και της διακέντρου συνοψίζονται για τις τρεις βασικές σχετικές θέσεις των δύο κύκλων και γράφονται ισοδύναμα ως εξής :
\begin{enumerate}[itemsep=0mm,label=\bf\arabic*.]
\item Κύκλοι χωρίς κοινά σημεία : $ \delta>R+\rho\ \textrm{ ή }\ \delta<R-\rho\Leftrightarrow |\delta-\rho|>R $.
\item Εφαπτόμενοι κύκλοι : $ \delta=R+\rho\ \textrm{ ή }\ \delta=R-\rho\Leftrightarrow |\delta-\rho|=R $.
\item Τεμνόμενοι κύκλοι : $ R-\rho<\delta<\delta<R+\rho\Leftrightarrow |\delta-\rho|<R $.
\end{enumerate}
Οι προηγούμενες προτάσεις φαίνονται συνοπτικά στον παρακάτω πίνακα :
\begin{center}
\begin{tabular}{c|c|c|c|c}
\hline \multicolumn{2}{c|}{\textbf{Χωρίς κοινά σημεία}} & \multicolumn{2}{c|}{\textbf{Εφαπτόμενοι}}  & \textbf{Τεμνόμενοι} \rule[-2ex]{0pt}{5.5ex}\\ 
\hhline{=====} \rule[-2ex]{0pt}{12.5ex} \begin{tikzpicture}
\tkzDefPoint[label=left:$K$](0,0){K}
\tkzDefPoint[label=right:$\varLambda$](1.5,0){L}
\tkzDefPoint(30:.8){A}
\tkzDefPoint[shift={(1.5,0)}](150:.5){B}
\draw[pl](K) circle (.8);
\draw[pl](L) circle (.5);
\draw[pl,\xrwma] (K)--(L);
\draw[pl] (K)--(A);
\draw[pl] (L)--(B);
\tkzDrawPoints(K,L)
\node at (0.2,0.4) {\footnotesize$R$};
\node at (1.4,0.25) {\footnotesize$\rho$};
\node at (0.9,-0.2) {\footnotesize$\delta$};
\end{tikzpicture} & \begin{tikzpicture}
\tkzDefPoint[label=left:$K$](0,0){K}
\tkzDefPoint[label=above:$\varLambda$](.3,0){L}
\tkzDefPoint(330:.8){A}
\tkzDefPoint[shift={(.3,0)}](0:.4){B}
\draw[pl](K) circle (.8);
\draw[pl](L) circle (.4);
\draw[pl,\xrwma] (K)--(L);
\draw[pl] (K)--(A);
\draw[pl] (B)--(L);
\tkzDrawPoints(K,L)
\node at (1.1,0.2) {\footnotesize$\rho$};
\node at (1.1,-0.2) {\footnotesize$R$};
\draw[-latex] (1,0.2) -- (0.5,0);
\draw[-latex] (1,-0.2) -- (0.35,-0.2);
\draw[-latex] (-0.2,0.4)node[yshift=1.5mm]{\footnotesize$\delta$} -- (0.2,0);
\end{tikzpicture} & \begin{tikzpicture}
\tkzDefPoint(0,0){K}
\tkzDefPoint[label=above:$\varLambda$](.3,0){L}
\tkzDefPoint[shift={(.3,0)}](0:.5){B}
\draw[pl](K) circle (.8);
\draw[pl](L) circle (.5);
\draw[pl,\xrwma] (K)--(L);
\draw[pl] (B)--(L);
\tkzDrawPoints(K,L)
\tkzLabelPoint[left,fill=white,inner sep=.2mm,xshift=-1mm](K){$K$}
\node (v1) at (0.4,-0.25) {$ \undercbrace{\rule{7mm}{0mm}}_{} $};
\draw[-latex] (0.4,-0.22) -- (.9,-0.4)node[xshift=1mm]{\footnotesize$R$};
\node at (1.1,0.2) {\footnotesize$\rho$};
\draw[-latex] (1,0.2) -- (0.5,0);
\draw[-latex] (-0.2,0.4)node[yshift=1.5mm]{\footnotesize$\delta$} -- (0.2,0);
\end{tikzpicture} & \begin{tikzpicture}
\tkzDefPoint[label=left:$K$](0,0){K}
\tkzDefPoint[label=right:$\varLambda$](1.3,0){L}
\draw[pl](K) circle (.8);
\draw[pl](L) circle (.5);
\draw[pl,\xrwma] (K)--(L);
\tkzDrawPoints(K,L)
\node at (0.4,0.2) {\footnotesize$R$};
\node at (1.1,0.15) {\footnotesize$\rho$};
\draw[|-|](0,-.2)--(1.3,-.2);
\node[fill=white,inner sep=.1mm] at (0.68,-0.4) {\footnotesize$\delta$};
\end{tikzpicture} & \begin{tikzpicture}
\tkzDefPoint[label=left:$K$](0,0){K}
\tkzDefPoint[label=right:$\varLambda$](1,0){L}
\tkzDefPoint[label=below:$B$](330:.8){B}
\tkzDefPoint(30:.8){A}
\tkzLabelPoint[above,xshift=.71mm](A){$A$}
\draw[pl](K) circle (.8);
\draw[pl](L) circle (.5);
\draw[pl,\xrwma](A)--(B);
\draw[pl,\xrwma] (K)--(L);
\draw[pl] (K)--(A);
\draw[pl] (A)--(L);
\tkzDrawPoints(K,L,B,A)
\node at (0.2,0.4) {\footnotesize$R$};
\node at (1,0.25) {\footnotesize$\rho$};
\node at (0.4,-0.15) {\footnotesize$\delta$};
\end{tikzpicture}\\
\hline  \multicolumn{2}{c|}{$\LEFTRIGHT.\}{
\begin{aligned}
& \delta<R-\rho\\
&  \delta>R+\rho\end{aligned}}\Rightarrow\left|\delta-\rho\right|>R$} &  \multicolumn{2}{c|}{$\LEFTRIGHT.\}{
\begin{aligned}
& \delta=R-\rho\\
&  \delta=R+\rho\end{aligned}}\Rightarrow\left|\delta-\rho\right|=R$}  &
\begin{minipage}{4cm}
\begin{center}
$R-\rho<\delta<R+\rho\Rightarrow\left|\delta-\rho\right|<R$
\end{center}
\end{minipage}  \rule[-2ex]{0pt}{7ex}\\
\hline 
\end{tabular}
\end{center} 
\end{document}
