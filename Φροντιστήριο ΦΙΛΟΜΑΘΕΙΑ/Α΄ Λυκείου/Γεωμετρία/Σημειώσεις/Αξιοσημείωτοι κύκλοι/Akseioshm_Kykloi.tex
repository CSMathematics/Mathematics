\PassOptionsToPackage{no-math,cm-default}{fontspec}
\documentclass[twoside,nofonts,internet,shmeiwseis]{thewria}
\usepackage{amsmath}
\usepackage{xgreek}
\let\hbar\relax
\defaultfontfeatures{Mapping=tex-text,Scale=MatchLowercase}
\setmainfont[Mapping=tex-text,Numbers=Lining,Scale=1.0,BoldFont={Minion Pro Bold}]{Minion Pro}
\newfontfamily\scfont{GFS Artemisia}
\font\icon = "Webdings"
\usepackage[amsbb]{mtpro2}
\usepackage{tikz,pgfplots,tkz-euclide,mathtools}
\tkzSetUpPoint[size=7,fill=white]
\xroma{red!70!black}
%------- ΣΥΣΤΗΜΑ -------------------
\usepackage{systeme,regexpatch}
\makeatletter
% change the definition of \sysdelim not to store `\left` and `\right`
\def\sysdelim#1#2{\def\SYS@delim@left{#1}\def\SYS@delim@right{#2}}
\sysdelim\{. % reinitialize

% patch the internal command to use
% \LEFTRIGHT<left delim><right delim>{<system>}
% instead of \left<left delim<system>\right<right delim>
\regexpatchcmd\SYS@systeme@iii
{\cB.\c{SYS@delim@left}(.*)\c{SYS@delim@right}\cE.}
{\c{SYS@MT@LEFTRIGHT}\cB\{\1\cE\}}
{}{}
\def\SYS@MT@LEFTRIGHT{%
\expandafter\expandafter\expandafter\LEFTRIGHT
\expandafter\SYS@delim@left\SYS@delim@right}
\makeatother
\newcommand{\synt}[2]{{\scriptsize \begin{matrix}
\times#1\\\\ \times#2
\end{matrix}}}
%----------------------------------------
%------ ΜΗΚΟΣ ΓΡΑΜΜΗΣ ΚΛΑΣΜΑΤΟΣ ---------
\DeclareRobustCommand{\frac}[3][0pt]{%
{\begingroup\hspace{#1}#2\hspace{#1}\endgroup\over\hspace{#1}#3\hspace{#1}}}
%----------------------------------------

\newlist{rlist}{enumerate}{3}
\setlist[rlist]{itemsep=0mm,label=\roman*.}
\newlist{brlist}{enumerate}{3}
\setlist[brlist]{itemsep=0mm,label=\bf\roman*.}
\newlist{tropos}{enumerate}{3}
\setlist[tropos]{label=\bf\textit{\arabic*\textsuperscript{oς}\;Τρόπος :},leftmargin=0cm,itemindent=2.3cm,ref=\bf{\arabic*\textsuperscript{oς}\;Τρόπος}}
\newcommand{\tss}[1]{\textsuperscript{#1}}
\newcommand{\tssL}[1]{\MakeLowercase{\textsuperscript{#1}}}
\usetkzobj{all}
\usepackage{hhline}
%----------- ΓΡΑΦΙΚΕΣ ΠΑΡΑΣΤΑΣΕΙΣ ---------
\pgfkeys{/pgfplots/aks_on/.style={axis lines=center,
xlabel style={at={(current axis.right of origin)},xshift=1.5ex, anchor=center},
ylabel style={at={(current axis.above origin)},yshift=1.5ex, anchor=center}}}
\pgfkeys{/pgfplots/grafikh parastash/.style={\xrwma,line width=.4mm,samples=200}}
\pgfkeys{/pgfplots/belh ar/.style={tick label style={font=\scriptsize},axis line style={-latex}}}
%-----------------------------------------
\usepackage{multicol}
\usepackage{wrap-rl}
\tkzSetUpPoint[size=7,fill=white]
\tikzstyle{pl}=[line width=0.3mm]
\tikzstyle{plm}=[line width=0.4mm]
\usepackage{gensymb}


\begin{document}
\titlos{Γεωμετρία Α΄ Λυκείου}{Παράλληλες Ευθείες}{Αξειοσημείωτοι κύκλοι}
\orismoi
\Orismos{Περιγεγραμμένος κύκλος}
Περιγεγραμμένος ονομάζεται ο κύκλος που διέρχεται από τις κορυφές ενός τριγώνου $ AB\varGamma $. Το κέντρο του κύκλου ονομάζεται \textbf{περίκεντρο}.\\\\
\Orismos{Εγγεγραμμένος κύκλος}
Εγγεγραμμένος ονομάζεται ο κύκλος που εφάπτεται στις πλευρές ενός τριγώνου $ AB\varGamma $. Το κέντρο του κύκλου ονομάζεται \textbf{εγκεντρο}.\\\\
\Orismos{Παρεγγεγραμμένος κύκλος}
\wrapr{-4mm}{5}{4.2cm}{-5mm}{\begin{tikzpicture}[scale=.7]
\tkzDefPoint(0,0){B}
\tkzDefPoint(2,0){C}
\tkzDefPoint(.7,1.3){A}
\draw[pl](A)--(B)--(C)--cycle;
\tkzDrawLine[add=1.4 and 1.4](A,B)
\tkzDrawLine[add=.8 and 1](A,C)
\tkzDrawLine[add=.9 and .7](C,B)
\tkzDefLine[bisector](B,A,C) \tkzGetPoint{a}
\tkzDefLine[bisector out](B,A,C) \tkzGetPoint{b}
\tkzDefLine[bisector](A,C,B) \tkzGetPoint{c}
\tkzDefLine[bisector out](A,C,B) \tkzGetPoint{d}
\tkzDefLine[bisector](C,B,A) \tkzGetPoint{e}
\tkzDefLine[bisector out](C,B,A) \tkzGetPoint{f}
\tkzInterLL(A,a)(B,f) \tkzGetPoint{k}
\tkzInterLL(C,c)(B,f) \tkzGetPoint{l}
\tkzInterLL(B,e)(A,b) \tkzGetPoint{m}
\tkzDrawPoints(A,B,C,k,l,m)
\tkzDefPointBy[projection=onto C--B](k) \tkzGetPoint{F}
\tkzDrawCircle[color=\xrwma,pl](k,F)
\tkzDefPointBy[projection=onto A--B](l) \tkzGetPoint{G}
\tkzDrawCircle[color=\xrwma,pl](l,G)
\tkzDefPointBy[projection=onto C--A](m) \tkzGetPoint{H}
\tkzDrawCircle[color=\xrwma,pl](m,H)
\tkzLabelPoint[above,yshift=1.2mm](A){$A$}
\tkzLabelPoint[below left,xshift=-1mm](B){$B$}
\tkzLabelPoint[below right,xshift=2.2mm,yshift=.5mm](C){$\varGamma$}
\tkzLabelPoint[below](k){$I_a$}
\tkzLabelPoint[left](l){$I_\gamma$}
\tkzLabelPoint[right](m){$I_\beta$}
\end{tikzpicture}}{
Περιγεγραμμένος ονομάζεται ο κύκλος εφάπτεται σε μια πλευρά και στις προεκτάσεις των άλλων δύο πλευρών ενός τριγώνου $ AB\varGamma $. Το κέντρο του κύκλου ονομάζεται \textbf{παράκεντρο}. Σε κάθε τρίγωνο υπάρχουν τρεις παρεγγεγραμμένοι κύκλοι.}
\begin{center}
\begin{tabular}{ccc}
\begin{tikzpicture}
\tkzDefPoint[label=left:$O$](0,0){O}
\tkzDefPoint(210:1.3){B}
\tkzDefPoint(330:1.3){C}
\tkzDefPoint(120:1.3){A}
\tkzLabelPoint[above](A){$A$}
\tkzLabelPoint[left](B){$B$}
\tkzLabelPoint[right](C){$\varGamma$}
\draw[pl,\xrwma] (0,0) circle (1.3);
\draw[pl](A)--(B)--(C)--cycle;
\tkzDrawPoints(A,B,C,O)
\end{tikzpicture} & \begin{tikzpicture}
\clip (-0.4,-0.2) rectangle (4.4,3);
\tkzDefPoint(0,0){B}
\tkzDefPoint(4,0){C}
\tkzDefPoint(1,2.5){A}
\draw[pl](A)--(B)--(C)--cycle;
\tkzDefCircle[in](A,B,C)
\tkzGetPoint{I} \tkzGetLength{rIN}
\draw[\xrwma,pl](I) circle (\rIN pt);
\tkzLabelPoint[above](A){$A$}
\tkzLabelPoint[left](B){$B$}
\tkzLabelPoint[right](C){$\varGamma$}
\tkzLabelPoint[left](I){$I$}
\tkzDrawPoints(A,B,C,O,I)
\end{tikzpicture} & 
 \\ 
\end{tabular} 
\end{center}
\thewrhmata
\Thewrhma{Περιγεγραμμένος κύκλος - περίκεντρο}
Οι τρεις μεσοκάθετοι των πλευρών ενός τριγώνου διέρχονται από το ίδιο σημείο, το οποίο είναι το κέντρο του περιγεγραμμένου κύκλου, δηλαδή το περίκεντρο.\\\\
\Thewrhma{Εγγεγραμμένος κύκλος - έγκεντρο}
Οι τρεις διχοτόμοι των γωνιών ενός τριγώνου διέρχονται από το ίδιο σημείο, το οποίο είναι το κέντρο του εγγεγραμμένου κύκλου, δηλαδή το έγκεντρο.
\begin{center}
\begin{tabular}{cc}
\begin{tikzpicture}
\tkzDefPoint[label=below left:$O$](0,0){O}
\tkzDefPoint(210:1.3){B}
\tkzDefPoint(330:1.3){C}
\tkzDefPoint(120:1.3){A}
\tkzDefMidPoint(A,B) \tkzGetPoint{M}
\tkzDefMidPoint(B,C) \tkzGetPoint{K}
\tkzDefMidPoint(A,C) \tkzGetPoint{L}
\tkzMarkRightAngle[fill=\xrwma,size=.2](B,M,O)
\tkzMarkRightAngle[fill=\xrwma,size=.2](C,K,O)
\tkzMarkRightAngle[fill=\xrwma,size=.2](C,L,O)
\tkzLabelPoint[above](A){$A$}
\tkzLabelPoint[left](B){$B$}
\tkzLabelPoint[right](C){$\varGamma$}
\draw[pl,\xrwma] (0,0) circle (1.3);
\draw[pl](A)--(B)--(C)--cycle;
\tkzDrawLine[add=.5 and .5](O,M)
\tkzDrawLine[add=.5 and 1](O,L)
\tkzDrawLine[add=.5 and .5](O,K)
\tkzDrawPoints(A,B,C,O)
\end{tikzpicture} & \begin{tikzpicture}
\clip (-0.4,-0.2) rectangle (4.4,3);
\tkzDefPoint(0,0){B}
\tkzDefPoint(4,0){C}
\tkzDefPoint(1,2.5){A}
\draw[pl](A)--(B)--(C)--cycle;
\tkzDefCircle[in](A,B,C)
\tkzGetPoint{I} \tkzGetLength{rIN}
\draw[\xrwma,pl](I) circle (\rIN pt);
\tkzLabelPoint[above](A){$A$}
\tkzLabelPoint[left](B){$B$}
\tkzLabelPoint[right](C){$\varGamma$}
\tkzLabelPoint[below left,xshift=1.5mm](I){$I$}
\tkzDrawBisector(B,A,C)
\tkzDrawBisector(C,B,A)
\tkzDrawBisector(A,C,B)
\tkzDrawPoints(A,B,C,O,I)
\end{tikzpicture} \\ 
\end{tabular} 
\end{center}
\Thewrhma{Παρεγγεγραμμένος κύκλος - παράκεντρο}
Οι διχοτόμοι δύο εξωτερικών γωνιών ενός τριγώνου και η διχοτόμος της τρίτης εσωτερικής γωνίας διέρχονται από το ίδιο σημείο, το οποίο είναι το κέντρο του παρεγγεγραμμένου κύκλου, δηλαδή το παράγκεντρο.
\begin{center}
\begin{tikzpicture}
\tkzDefPoint(0,0){B}
\tkzDefPoint(2,0){C}
\tkzDefPoint(.7,1.3){A}
\draw[pl](A)--(B)--(C)--cycle;
\tkzDrawLine[add=1.4 and 1.4](A,B)
\tkzDrawLine[add=.8 and 1](A,C)
\tkzDrawLine[add=.9 and .7](C,B)
\tkzDefLine[bisector](B,A,C) \tkzGetPoint{a}
\tkzDefLine[bisector out](B,A,C) \tkzGetPoint{b}
\tkzDefLine[bisector](A,C,B) \tkzGetPoint{c}
\tkzDefLine[bisector out](A,C,B) \tkzGetPoint{d}
\tkzDefLine[bisector](C,B,A) \tkzGetPoint{e}
\tkzDefLine[bisector out](C,B,A) \tkzGetPoint{f}
\tkzInterLL(A,a)(B,f) \tkzGetPoint{k}
\tkzInterLL(C,c)(B,f) \tkzGetPoint{l}
\tkzInterLL(B,e)(A,b) \tkzGetPoint{m}
\draw (k)--(l)--(m)--cycle;
\draw (A)--(k);
\draw (B)--(m);
\draw (C)--(l);
\tkzDrawPoints(A,B,C,k,l,m)
\tkzDefPointBy[projection=onto C--B](k) \tkzGetPoint{F}
\tkzDrawCircle[color=\xrwma,pl](k,F)
\tkzDefPointBy[projection=onto A--B](l) \tkzGetPoint{G}
\tkzDrawCircle[color=\xrwma,pl](l,G)
\tkzDefPointBy[projection=onto C--A](m) \tkzGetPoint{H}
\tkzDrawCircle[color=\xrwma,pl](m,H)
\tkzLabelPoint[above,yshift=1.2mm](A){$A$}
\tkzLabelPoint[below left,xshift=-1mm](B){$B$}
\tkzLabelPoint[below right,xshift=2.2mm,yshift=.5mm](C){$\varGamma$}
\tkzLabelPoint[below](k){$I_a$}
\tkzLabelPoint[left](l){$I_\gamma$}
\tkzLabelPoint[right](m){$I_\beta$}
\end{tikzpicture}
\end{center}
\end{document}
