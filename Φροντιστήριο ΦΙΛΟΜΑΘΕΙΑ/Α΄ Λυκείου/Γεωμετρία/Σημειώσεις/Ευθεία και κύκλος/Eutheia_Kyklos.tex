\PassOptionsToPackage{no-math,cm-default}{fontspec}
\documentclass[twoside,nofonts,internet,shmeiwseis]{thewria}
\usepackage{amsmath}
\usepackage{xgreek}
\let\hbar\relax
\defaultfontfeatures{Mapping=tex-text,Scale=MatchLowercase}
\setmainfont[Mapping=tex-text,Numbers=Lining,Scale=1.0,BoldFont={Minion Pro Bold}]{Minion Pro}
\newfontfamily\scfont{GFS Artemisia}
\font\icon = "Webdings"
\usepackage[amsbb]{mtpro2}
\usepackage{tikz,pgfplots,tkz-euclide}
\tkzSetUpPoint[size=7,fill=white]
\xroma{red!70!black}
%------- ΣΥΣΤΗΜΑ -------------------
\usepackage{systeme,regexpatch}
\makeatletter
% change the definition of \sysdelim not to store `\left` and `\right`
\def\sysdelim#1#2{\def\SYS@delim@left{#1}\def\SYS@delim@right{#2}}
\sysdelim\{. % reinitialize

% patch the internal command to use
% \LEFTRIGHT<left delim><right delim>{<system>}
% instead of \left<left delim<system>\right<right delim>
\regexpatchcmd\SYS@systeme@iii
{\cB.\c{SYS@delim@left}(.*)\c{SYS@delim@right}\cE.}
{\c{SYS@MT@LEFTRIGHT}\cB\{\1\cE\}}
{}{}
\def\SYS@MT@LEFTRIGHT{%
\expandafter\expandafter\expandafter\LEFTRIGHT
\expandafter\SYS@delim@left\SYS@delim@right}
\makeatother
\newcommand{\synt}[2]{{\scriptsize \begin{matrix}
\times#1\\\\ \times#2
\end{matrix}}}
%----------------------------------------
%------ ΜΗΚΟΣ ΓΡΑΜΜΗΣ ΚΛΑΣΜΑΤΟΣ ---------
\DeclareRobustCommand{\frac}[3][0pt]{%
{\begingroup\hspace{#1}#2\hspace{#1}\endgroup\over\hspace{#1}#3\hspace{#1}}}
%----------------------------------------

\newlist{rlist}{enumerate}{3}
\setlist[rlist]{itemsep=0mm,label=\roman*.}
\newlist{brlist}{enumerate}{3}
\setlist[brlist]{itemsep=0mm,label=\bf\roman*.}
\newlist{tropos}{enumerate}{3}
\setlist[tropos]{label=\bf\textit{\arabic*\textsuperscript{oς}\;Τρόπος :},leftmargin=0cm,itemindent=2.3cm,ref=\bf{\arabic*\textsuperscript{oς}\;Τρόπος}}
\newcommand{\tss}[1]{\textsuperscript{#1}}
\newcommand{\tssL}[1]{\MakeLowercase{\textsuperscript{#1}}}
\usetkzobj{all}
\usepackage{hhline}
%----------- ΓΡΑΦΙΚΕΣ ΠΑΡΑΣΤΑΣΕΙΣ ---------
\pgfkeys{/pgfplots/aks_on/.style={axis lines=center,
xlabel style={at={(current axis.right of origin)},xshift=1.5ex, anchor=center},
ylabel style={at={(current axis.above origin)},yshift=1.5ex, anchor=center}}}
\pgfkeys{/pgfplots/grafikh parastash/.style={\xrwma,line width=.4mm,samples=200}}
\pgfkeys{/pgfplots/belh ar/.style={tick label style={font=\scriptsize},axis line style={-latex}}}
%-----------------------------------------
\usepackage{multicol}
\usepackage{wrap-rl}
\tkzSetUpPoint[size=7,fill=white]
\tikzstyle{pl}=[line width=0.3mm]
\tikzstyle{plm}=[line width=0.4mm]
\usepackage{gensymb}


\begin{document}
\titlos{Γεωμετρία Α΄ Λυκείου}{Τρίγωνα}{Ευθεία και κύκλος}
\orismoi
\Orismos{Σχετικές θέσεις ευθείας και κύκλου}
Οι τρεις σχετικές θέσεις μεταξύ μιας ευθείας $ \varepsilon $ και ενός κύκλου $ \left( O,\rho\right)  $ είναι οι ακόλουθες :
\begin{enumerate}[itemsep=0mm,label=\bf\arabic*.]
\item \textbf{Εξωτερική ευθεία}\\
Εξωτερική ευθεία ενός κύκλου λέγεται μια ευθεία αν η απόσταση της από το κέντρο του κύκλου είναι μεγαλύτερη από την ακτίνα του.
\[ OA>OB\Leftrightarrow \delta>\rho \]
\item \textbf{Εφαπτόμενη ευθεία}\\
Εφαπτόμενη ευθεία ενός κύκλου λέγεται μια ευθεία αν η απόσταση της από το κέντρο του κύκλου είναι ίση από την ακτίνα του.
Το κοινό σημείο της ευθείας και του κύκλου λέγεται \textbf{σημείο επαφής}.
\[ OA=OB\Leftrightarrow \delta=\rho \]
\item \textbf{Τέμνουσα ευθεία}\\
Τέμνουσα ευθεία ενός κύκλου λέγεται μια ευθεία αν η απόσταση της από το κέντρο του κύκλου είναι μικρότερη από την ακτίνα του.
\[ OA<OB\Leftrightarrow \delta<\rho \]
\end{enumerate}
\begin{center}
\begin{tabular}{ccc}
\begin{tikzpicture}
\draw[pl] (0,0) circle (1);
\tkzDefPoint[label=above:$O$](0,0){O}
\tkzDefPoint[label=below left:$A$](270:1.4){A}
\tkzDefPoint[label=below right:$B$](315:1){B}
\draw(-1.4,-1.4)--(1.4,-1.4);
\draw[pl](O)--(A);
\draw[pl](O)--(B);
\tkzDrawPoints(A,B,O)
\node at (0.5,-0.2) {\footnotesize$\rho$};
\node at (-0.2,-0.6) {\footnotesize$\delta$};
\node at (1.8,-1.4) {\footnotesize$\varepsilon$};
\end{tikzpicture} & \begin{tikzpicture}
\draw[pl] (0,0) circle (1);
\tkzDefPoint[label=above:$O$](0,0){O}
\tkzDefPoint[label=below left:$A$](270:1){A}
\tkzDefPoint(270:1.4){E}
\tkzDefPoint[label=right:$B$](315:1){B}
\draw(-1.4,-1)--(1.4,-1);
\draw[pl](O)--(A);
\draw[pl](O)--(B);
\tkzDrawPoints(A,B,O)
\node at (0.5,-0.2) {\footnotesize$\rho$};
\node at (-0.2,-0.5) {\footnotesize$\delta$};
\node at (1.8,-1) {\footnotesize$\varepsilon$};
\end{tikzpicture} & \begin{tikzpicture}
\draw[pl] (0,0) circle (1);
\tkzDefPoint[label=above:$O$](0,0){O}
\tkzDefPoint(270:.8){A}
\tkzDefPoint(270:1.4){E}
\tkzDefPoint(315:1){B}
\tkzLabelPoint[below left,fill=white,inner sep=.1mm,yshift=-1mm](A){$A$}
\tkzLabelPoint[right,fill=white,inner sep=.1mm,yshift=1mm,xshift=2mm](B){$B$}
\draw(-1.4,-.8)--(1.4,-.8);
\draw[pl](O)--(A);
\draw[pl](O)--(B);
\tkzDrawPoints(A,B,O)
\node at (0.5,-0.2) {\footnotesize$\rho$};
\node at (-0.2,-.4) {\footnotesize$\delta$};
\node at (1.8,-.8) {\footnotesize$\varepsilon$};
\end{tikzpicture} \\ 
\end{tabular} 
\end{center}
\Orismos{Εφαπτόμενα τμήματα}
\wrapr{-4mm}{5}{3.8cm}{-12mm}{\begin{tikzpicture}
\draw[pl] (0,0) circle (1);
\tkzDefPoint[label=right:$O$](0,0){O}
\tkzDefPoint[label=above:$P$](-2.6,0){P}
\tkzDefPoint[label=above:$A$](113:1){A}
\tkzDefPoint[label=below:$B$](247:1){B}
\draw[pl](O)--(A);
\draw[pl](O)--(B);
\draw[pl,\xrwma](P)--(A);
\draw[pl,\xrwma](P)--(B);
\draw[pl,\xrwma](O)--(P);
\tkzDrawPoints(A,B,O,P)
\node at (0,0.6) {\footnotesize$\rho$};
\node at (0,-0.6) {\footnotesize$\rho$};
\node at (-1.2,0.2) {\footnotesize$\delta$};
\end{tikzpicture}}{
Εφαπτόμενα τμήματα ενός κύκλου ονομάζονται τα ευθύγραμμα τμήματα που άγονται από σημείο εκτός του κύκλου και εφάπτονται εκατέρωθεν του. Η ευθεία που διέρχεται από το εξωτερικό σημείο και το κέντρο του κύκλου ονομάζεται \textbf{διακεντρική ευθεία}.}
\thewrhmata
\Thewrhma{Κοινά σημεία κύκλου - ευθείας}
Ένας κύκλος έχει το πολύ δύο κοινά σημεία με μια ευθεία.\\\\
\Thewrhma{Εφαπτόμενη ευθεία}
Η εφαπτόμενη ευθεία σε ένα σημείο του κύκλου είναι μοναδική. Επιπλέον η ακτίνα στο σημείο επαφής είναι κάθετη με την εφαπτομένη.\\\\
\Thewrhma{Εφαπτόμενα τμήματα}
\wrapr{-4mm}{8}{3.7cm}{-4mm}{\begin{tikzpicture}
\draw[pl] (0,0) circle (1);
\tkzDefPoint[label=right:$O$](0,0){O}
\tkzDefPoint[label=above:$P$](-2.6,0){P}
\tkzDefPoint[label=above:$A$](113:1){A}
\tkzDefPoint[label=below:$B$](247:1){B}
\tkzMarkAngle[size=.5,mark=|,fill=\xrwma](O,P,A)
\tkzMarkAngle[size=.5,mark=|,fill=\xrwma](B,P,O)
\tkzMarkAngle[size=.4,mark=||,fill=\xrwma](A,O,P)
\tkzMarkAngle[size=.4,mark=||,fill=\xrwma](P,O,B)
\tkzMarkRightAngle[size=.2,mark=||](P,A,O)
\tkzMarkRightAngle[size=.2,mark=||](O,B,P)
\draw[pl](O)--(A);
\draw[pl](O)--(B);
\draw[pl,\xrwma](P)--(A);
\draw[pl,\xrwma](P)--(B);
\draw[pl](O)--(P);
\tkzDrawPoints(A,B,O,P)
\node at (0,0.6) {\footnotesize$\rho$};
\node at (0,-0.6) {\footnotesize$\rho$};
\node at (-1.2,0.2) {\footnotesize$\delta$};
\end{tikzpicture}}{
Αν $ P $ είναι ένα εξωτερικό σημείο ενός κύκλου $ (O,\rho) $ τότε ισχύουν οι παρακάτω προτάσεις.
\begin{rlist}
\item Τα εφαπτόμενα τμήματα που άγονται από ένα σημείο εκτός ενός κύκλου είναι μεταξύ τους ίσα.
\item Η διακεντρική ευθεία διχοτομεί τη γωνία των εφαπτόμενων τμημάτων και τη γωνία των ακτίνων στα σημεία επαφής.
\end{rlist}}\mbox{}\\
\vspace{-3mm}
\begin{rlist}[start=3]
\item Η διακεντρική ευθεία είναι μεσοκάθετος της χορδής που ενώνει τα σημεία επαφής.
\end{rlist}
\end{document}
