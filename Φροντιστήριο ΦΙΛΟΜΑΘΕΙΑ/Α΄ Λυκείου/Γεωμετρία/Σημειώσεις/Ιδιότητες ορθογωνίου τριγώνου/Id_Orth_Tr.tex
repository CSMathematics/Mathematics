\PassOptionsToPackage{no-math,cm-default}{fontspec}
\documentclass[twoside,nofonts,internet,shmeiwseis]{thewria}
\usepackage{amsmath}
\usepackage{xgreek}
\let\hbar\relax
\defaultfontfeatures{Mapping=tex-text,Scale=MatchLowercase}
\setmainfont[Mapping=tex-text,Numbers=Lining,Scale=1.0,BoldFont={Minion Pro Bold}]{Minion Pro}
\newfontfamily\scfont{GFS Artemisia}
\font\icon = "Webdings"
\usepackage[amsbb]{mtpro2}
\usepackage{tikz,pgfplots,tkz-euclide,enumitem}
\usetkzobj{all}
\tkzSetUpPoint[size=7,fill=white]
\xroma{red!70!black}
\newlist{rlist}{enumerate}{3}
\setlist[rlist]{itemsep=0mm,label=\roman*.}
\newlist{brlist}{enumerate}{3}
\setlist[brlist]{itemsep=0mm,label=\bf\roman*.}
\newlist{tropos}{enumerate}{3}
\setlist[tropos]{label=\bf\textit{\arabic*\textsuperscript{oς}\;Τρόπος :},leftmargin=0cm,itemindent=2.3cm,ref=\bf{\arabic*\textsuperscript{oς}\;Τρόπος}}
\newcommand{\tss}[1]{\textsuperscript{#1}}
\newcommand{\tssL}[1]{\MakeLowercase{\textsuperscript{#1}}}
\usepackage{rotating}
\usepackage{hhline}
\usepackage{multicol,multirow,gensymb,mathimatika}
\usepackage{wrap-rl}


\begin{document}
\titlos{Γεωμετρία Α΄ Λυκείου}{Παραλληλόγραμμα}{Ιδιότητες του ορθογωνίου τριγώνου}
\thewrhmata
\Thewrhma{Διάμεσος από ορθή γωνία}
\wrapr{-4mm}{7}{3cm}{-4mm}{\begin{tikzpicture}
\tkzDefPoint(0,0){A}
\tkzDefPoint(2,0){B}
\tkzDefPoint(0,3){C}
\tkzDefPoint(1,1.5){M}
\draw[pl,\xrwma](A)--(M);
\tkzMarkRightAngle[size=.3](B,A,C)
\draw[pl](A)--(B)--(C)--cycle;
\tkzMarkSegments[mark=|](C,M M,B M,A)
\tkzLabelPoint[left](A){$A$}
\tkzLabelPoint[right](B){$B$}
\tkzLabelPoint[left](C){$\varGamma$}
\tkzLabelPoint[right,yshift=1mm](M){$M$}
\tkzDrawPoints(A,B,C,M)
\end{tikzpicture}}{
Σε κάθε ορθογώνιο τρίγωνο ισχύουν οι εξής προτάσεις που αφορούν τη διάμεσο που αντιστοιχεί στην υποτείνουσα.
\begin{rlist}
\item Η διάμεσος που άγεται από την ορθή γωνία προς την υποτείνουσα σε κάθε ορθογώνιο τρίγωνο, ισούται με το μισό της υποτείνουσας.
\item (Αντίστροφο) Αν σε ένα τρίγωνο, μια διάμεσος ισούται με τη μισή πλευρά στην οποία αντιστοιχεί, τότε το τρίγωνο είναι ορθογώνιο με υποτείνουσα την πελυρά αυτή.
\end{rlist}}\mbox{}\\\\\\
\Thewrhma{Ορθογώνιο με γωνία {$ \mathbold{30\degree} $}}
Σε ένα ορθογώνιο τρίγωνο μια οξεία γωνία ισούται με $ 30\degree $ αν και μόνο αν η απέναντι κάθετη πλευρά είναι ίση με τη μισή υποτείνουσα.
\end{document}
