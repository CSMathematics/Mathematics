\documentclass[twoside,nofonts,ektypwsh,math,spyros]{frontisthrio-diag}
\usepackage[amsbb,subscriptcorrection,zswash,mtpcal,mtphrb,mtpfrak]{mtpro2}
\usepackage[no-math,cm-default]{fontspec}
\usepackage{amsmath}
\usepackage{xunicode}
\usepackage{xgreek}
\let\hbar\relax
\defaultfontfeatures{Mapping=tex-text,Scale=MatchLowercase}
\setmainfont[Mapping=tex-text,Numbers=Lining,Scale=1.0,BoldFont={Minion Pro Bold}]{Minion Pro}
\newfontfamily\scfont{GFS Artemisia}
\font\icon = "Webdings"
\usepackage{fontawesome5}
\newfontfamily{\FA}{fontawesome.otf}
\xroma{red!70!black}
%------TIKZ - ΣΧΗΜΑΤΑ - ΓΡΑΦΙΚΕΣ ΠΑΡΑΣΤΑΣΕΙΣ ----
\usepackage{tikz,pgfplots}
\usepackage{tkz-euclide}
\usetkzobj{all}
\usepackage[framemethod=TikZ]{mdframed}
\usetikzlibrary{decorations.pathreplacing}
\tkzSetUpPoint[size=7,fill=white]
%-----------------------
\usepackage{calc,tcolorbox}
\tcbuselibrary{skins,theorems,breakable}
\usepackage{hhline}
\usepackage[explicit]{titlesec}
\usepackage{graphicx}
\usepackage{multicol}
\usepackage{multirow}
\usepackage{tabularx}
\usetikzlibrary{backgrounds}
\usepackage{sectsty}
\sectionfont{\centering}
\usepackage{enumitem}
\usepackage{adjustbox}
\usepackage{mathimatika,gensymb,eurosym,wrap-rl}
\usepackage{systeme,regexpatch}
%-------- ΜΑΘΗΜΑΤΙΚΑ ΕΡΓΑΛΕΙΑ ---------
\usepackage{mathtools}
%----------------------
%-------- ΠΙΝΑΚΕΣ ---------
\usepackage{booktabs}
%----------------------
%----- ΥΠΟΛΟΓΙΣΤΗΣ ----------
\usepackage{calculator}
%----------------------------
%------ ΔΙΑΓΩΝΙΟ ΣΕ ΠΙΝΑΚΑ -------
\usepackage{array}
\newcommand\diag[5]{%
\multicolumn{1}{|m{#2}|}{\hskip-\tabcolsep
$\vcenter{\begin{tikzpicture}[baseline=0,anchor=south west,outer sep=0]
\path[use as bounding box] (0,0) rectangle (#2+2\tabcolsep,\baselineskip);
\node[minimum width={#2+2\tabcolsep-\pgflinewidth},
minimum  height=\baselineskip+#3-\pgflinewidth] (box) {};
\draw[line cap=round] (box.north west) -- (box.south east);
\node[anchor=south west,align=left,inner sep=#1] at (box.south west) {#4};
\node[anchor=north east,align=right,inner sep=#1] at (box.north east) {#5};
\end{tikzpicture}}\rule{0pt}{.71\baselineskip+#3-\pgflinewidth}$\hskip-\tabcolsep}}
%---------------------------------
%---- ΟΡΙΖΟΝΤΙΟ - ΚΑΤΑΚΟΡΥΦΟ - ΠΛΑΓΙΟ ΑΓΚΙΣΤΡΟ ------
\newcommand{\orag}[3]{\node at (#1)
{$ \overcbrace{\rule{#2mm}{0mm}}^{{\scriptsize #3}} $};}
\newcommand{\kag}[3]{\node at (#1)
{$ \undercbrace{\rule{#2mm}{0mm}}_{{\scriptsize #3}} $};}
\newcommand{\Pag}[4]{\node[rotate=#1] at (#2)
{$ \overcbrace{\rule{#3mm}{0mm}}^{{\rotatebox{-#1}{\scriptsize$#4$}}}$};}
%-----------------------------------------
%------------------------------------------
\newcommand{\tss}[1]{\textsuperscript{#1}}
\newcommand{\tssL}[1]{\MakeLowercase{\textsuperscript{#1}}}
%---------- ΛΙΣΤΕΣ ----------------------
\newlist{bhma}{enumerate}{3}
\setlist[bhma]{label=\bf\textit{\arabic*\textsuperscript{o}\;Βήμα :},leftmargin=0cm,itemindent=1.8cm,ref=\bf{\arabic*\textsuperscript{o}\;Βήμα}}
\newlist{rlist}{enumerate}{3}
\setlist[rlist]{itemsep=0mm,label=\roman*.}
\newlist{brlist}{enumerate}{3}
\setlist[brlist]{itemsep=0mm,label=\bf\roman*.}
\newlist{tropos}{enumerate}{3}
\setlist[tropos]{label=\bf\textit{\arabic*\textsuperscript{oς}\;Τρόπος :},leftmargin=0cm,itemindent=2.3cm,ref=\bf{\arabic*\textsuperscript{oς}\;Τρόπος}}
% Αν μπει το bhma μεσα σε tropo τότε
%\begin{bhma}[leftmargin=.7cm]
\tkzSetUpPoint[size=7,fill=white]
\tikzstyle{pl}=[line width=0.3mm]
\tikzstyle{plm}=[line width=0.4mm]
\usepackage{etoolbox}
\makeatletter
\renewrobustcmd{\anw@true}{\let\ifanw@\iffalse}
\renewrobustcmd{\anw@false}{\let\ifanw@\iffalse}\anw@false
\newrobustcmd{\noanw@true}{\let\ifnoanw@\iffalse}
\newrobustcmd{\noanw@false}{\let\ifnoanw@\iffalse}\noanw@false
\renewrobustcmd{\anw@print}{\ifanw@\ifnoanw@\else\numer@lsign\fi\fi}
\makeatother

\usepackage{path}
\pathal

\begin{document}
\titlos{A΄ Λυκείου}{Γεωμετρία}{Διαγώνισμα - Τρίγωνα}
\begin{thema}
\item \mbox{}\\
\vspace{-7mm}
\begin{erwthma}
\item Να αποδείξετε το παρακάτω θεώρημα: Κάθε σημείο της διχοτόμου μιας γωνίας ισαπέχει από τις πλευρές τις και αντίστροφα, κάθε εσωτερικό σημείο της γωνίας που ισαπέχει από τις πλευρές είναι σημείο της διχοτόμου.\monades{3}
\item Θεωρούμε δύο κύκλους $ (K,R_1) $ και $ (\varLambda,R_2) $.
\begin{alist}
\item Τι ονομάζουμε διάκεντρο $ \delta $ των δύο κύκλων;\monades{1}
\item Να αντιστοιχίσετε κάθε συνθήκη από τη στήλη Α με μια σχετική θέση κύκλων από τη στήλη Β.
\begin{center}
\begin{tabular}{|m{0.3\linewidth} |m{0.4\linewidth}|}
\hline 
\rule[-2ex]{0pt}{5ex} \textbf{Στήλη Α - Συνθήκη} & \textbf{Στήλη Β - Σχετική θέση κύκλων} \\ 
\hline 
\rule[-2ex]{0pt}{5ex} \vspace{-5mm}\begin{itemize}
\item $ \delta=R_1+R_2 $
\item $ \delta<R_1-R_2 $
\item $ R_1-R_2<\delta<R_1+R_2 $
\item $ \delta=R_1-R_2 $
\item $ \delta>R_1+R_2 $
\end{itemize} & \vspace{1mm}\begin{itemize}
\item Τέμνονται
\item Εφάπτονται εσωτερικά
\item Εφάπτονται εξωτερικά
\item Καθένας βρίσκεται εκτός του άλλου
\item Ο ένας βρίσκεται μέσα στον άλλο
\end{itemize} \\ 
\hline 
\end{tabular}
\end{center}
\end{alist}\monades{1}
\end{erwthma}
\item \mbox{}\\
Δίνεται ισοσκελές τρίγωνο $ AB\varGamma $, τα μέσα $ M,N $ των $ AB,A\varGamma $ αντίστοιχα και οι διχοτόμοι του $ B\varDelta $ και $ \varGamma E $, οι οποίες τέμνονται στο $ I $. Να αποδείξετε ότι:
\begin{erwthma}
\item $ BE=\varGamma\varDelta $\monades{1}
\item $ IE=I\varDelta $\monades{2}
\item τα τρίγωνα $ BIM $ και $ \varGamma IN $ είναι ίσα,\monades{1}
\item τα τρίγωνα $ IEM $ και $ I\varDelta N $ είναι ίσα.\monades{1}
\end{erwthma}
\item\mbox{}\\
Δίνεται οξυγώνιο τρίγωνο $ AB\varGamma $, με $ AB<A\varGamma $, και η διχοτόμος $ A\varDelta $. Φέρουμε τη $ BE\perp A\varDelta $ και η προέκτασή της τέμνει την $ A\varGamma $ στο $ Z $.
\begin{erwthma}
\item Να αποδείξετε ότι $ A\varGamma-AB=\varGamma Z $.\monades{2}
\item Αν $ EH\perp AB $ και $ E\varTheta\perp A\varGamma $, να αποδείξετε ότι $ BH=Z\varTheta $.\monades{2}
\item Να αποδείξετε ότι το τρίγωνο $ B\varDelta Z $ είναι ισοσκελές.\monades{1}
\end{erwthma}
\item\mbox{}\\
\wrapr{-5mm}{7}{3.8cm}{-11mm}{\begin{tikzpicture}[line cap=round,line join=round,>=triangle 45,x=1.0cm,y=1.0cm,scale=.8]
\draw [line width=0.8pt] (3.32,1.5) circle (2.cm);
\draw [line width=0.8pt] (3.32,5.26)-- (2.338526684209486,-0.2426158872198927);
\draw [line width=0.8pt] (3.32,5.26)-- (4.301473315790515,-0.2426158872198928);
\draw [line width=0.8pt] (3.32,5.26)-- (1.6264043623717106,2.5638297872340434);
\draw [line width=0.8pt] (3.32,5.26)-- (5.013595637628288,2.563829787234043);
\draw [line width=0.8pt] (3.32,1.5)-- (1.6264043623717106,2.5638297872340434);
\draw [line width=0.8pt] (3.32,1.5)-- (5.013595637628288,2.563829787234043);
\draw [fill=black] (3.32,1.5) circle (1.5pt);
\draw[color=black] (3.32,1.2) node {$O$};
\draw [fill=black] (3.32,5.26) circle (1.5pt);
\draw[color=black] (3.32,5.69) node {$M$};
\draw [fill=black] (2.338526684209486,-0.2426158872198927) circle (1.5pt);
\draw[color=black] (2.3,-0.7) node {$B$};
\draw [fill=black] (3.001526114710125,3.4744807885589495) circle (1.5pt);
\draw[color=black] (2.7,3.73) node {$A$};
\draw [fill=black] (4.301473315790515,-0.2426158872198928) circle (1.5pt);
\draw[color=black] (4.3,-0.7) node {$\varDelta$};
\draw [fill=black] (3.638473885289876,3.4744807885589495) circle (1.5pt);
\draw[color=black] (3.9,3.73) node {$\varGamma$};
\draw [fill=black] (1.6264043623717106,2.5638297872340434) circle (1.5pt);
\draw[color=black] (1.2,2.5) node {$E$};
\draw [fill=black] (5.013595637628288,2.563829787234043) circle (1.5pt);
\draw[color=black] (5.4,2.5) node {$Z$};
\end{tikzpicture}}{
Στον κύκλο κέντρου $ O $ του διπλανού σχήματος, οι χορδές $ AB $ και $ \varGamma\varDelta $ είναι ίσες και τέμνονται στο σημείο $ M $. Από το $ M $ φέρουμε τα εφαπτόμενα τμήματα $ ME $ και $ MZ $. Να αποδείξετε ότι:
\begin{erwthma}
\item $ MA=M\varGamma $\monades{2}
\item $ EB=\varDelta Z $\monades{2}
\item $ \widearc{AE}=\widearc{\varGamma Z} $\monades{1}
\end{erwthma}}
\end{thema}
\end{document}
