\documentclass[11pt,a4paper,modern]{FFExercises}
\usepackage[english,greek]{babel}
\usepackage[utf8]{inputenc}
\usepackage{nimbusserif}
\usepackage[T1]{fontenc}
\usepackage{amsmath}
\let\myBbbk\Bbbk
\let\Bbbk\relax
\usepackage[amsbb,subscriptcorrection,zswash,mtpcal,mtphrb,mtpfrak]{mtpro2}
\usepackage{graphicx,multicol,multirow,enumitem,tabularx,mathimatika,gensymb,venndiagram,hhline,longtable,tkz-euclide,fontawesome5,eurosym,tcolorbox,tabularray,tikzpagenodes,relsize,wrap-rl}
\definecolor{xrwma}{HTML}{aa1212}
\usetikzlibrary{calc}
\usetikzlibrary{positioning}
\tcbuselibrary{skins,theorems,breakable}
\renewcommand{\textstigma}{\textsigma\texttau}
\renewcommand{\textdexiakeraia}{}

\ekthetesdeiktes
\begin{document}

\titlos{Γεωμετρία}{Α' Λυκείου}{Παράλληλες ευθείες}
\paragraph{Παράγραφος}
\askhsh Δίνεται ισοσκελές τρίγωνο $AB\varGamma\ (ΑΒ=Α\varGamma)$ και ημιευθεία $Ax\parallel B\varGamma$.
\begin{alist}
\item Να δείξετε ότι η $Ax$ διχοτομεί τη γωνία $\hat{A}_{\varepsilon\xi}$.
\item Προεκτείνουμε την $\varGamma B$ κατά τμήμα $B\varDelta=AB$ και φέρουμε από το $B$ ευθεία παράλληλη προς την $A\varDelta$ η οποία τέμνει την $A\varGamma$ στο $E$. Να δείξετε ότι η $BE$ είναι διχοτόμος της $\hat{B}$.
\end{alist}  
\end{document}